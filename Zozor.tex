Naci\'o  en  1972 en  Colmar,  Francia.  Obtuvo  el  t\'itulo  de Ingeniero,  de
Licenciada, el grado de Doctor  y la ``Habilitation \`a diriger de Recherches'',
respectivamente en 1995, 1999 y 2012, ambos del Instituto Nacional Polit\'ecnico
de  Grenoble  (Grenoble  INP),  Francia.   En  2001, paso  varios  meses  en  el
Laboratorio de Procesamiento de Se\~nales  de la Escuela Polit\'enica Federal de
Lausanne (EPFL), Suiza como postdoctorante.   Pas\'o un a\~no en el Instituto de
F\'isica  de La  Plata (IFLP)  de la  Universidad Nacional  de La  Plata (UNLP),
Argentina  (2012-2013)  as\'i que  varios  estancias  desde  2010 como  profesor
visitante.   En  2001  ingres\'o   al  Centro  National  de  la  Investigaci\'on
Cientifica  (CNRS),  equivalente  Franc\'es  del  CONICET,  como  ``Charg\'e  de
Recherche''  (cargado  de  investigaci\'on)  y es  ``Directeur  de  Recherches''
(director de investigaci\'on)  desde 2017, ambos en el  Laboratorio de Imagenes,
Palabras, Se\~nales y Autom\'atica de Grenoble (GIPSA-Lab), Francia.  Desde 2015
es editor asociado  de la revista IEEE Signal Processing  Letters.  Sus temas de
investigaci\'on incluyen el procesamiento no lineal de se\~nales, el estudio del
efecto  de resonancia  estoc\'astica, el  estudio de  procesamiento de  datos en
contextos $\alpha$-estables  y/o de distribuciones  de probabilidad el\'ipticas,
la  t\'eoria  de   la  informaci\'on  (medidades  informacionales  generalizadas
cl\'asicas y c\'uanticas) con  aplicaciones en procesamientos de datos, mecanica
c\'uantica  o  ingenier\'ia biom\'edica.   Es  a  cargo  de docencia  en  varias
escuelas  de  Grenoble-INP de  matem\'atica  para  el ingeniero,  probabilidades
aplicadas, procesamiento  estad\'istico de se\~nales,  m\'etodos bayesianos.  Da
regularmente  un   mini-curso  sobre  los   b\'asicos  de  la  teor\'ia   de  la
informaci\'on en la Facultad de Ciencias Exactas de la UNLP.