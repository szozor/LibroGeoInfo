\seccion{Entrop\'ias y divergencias generalizadas}
\label{Sec:SZ:Generalizadas}

\SZ{Partout, voir convexite stricte, en 1,  idem pour h monotone, etc. vis a vis
  des cas d'egalite avec les divergences.}

A pesar de  que la entrop\'ia de Shannon y  sus cantidades asociadas demostraron
sus  potencias  tan  de un  punto  de  vista  descriptivo  que en  t\'ermino  de
aplicaciones en la  transmisi\'on de la informaci\'on y  la compresi\'on, varias
nociones  informacionales, tipo entrop\'ias  o divergencias,  aparecieron luego.
En  esta  secci\'on  no  se  desarollar\'a  todos  los  enfoques  ni  todas  las
aplicaciones  tan  la literatura  es  importante. La  meta  es  dar los  caminos
conduciendo a las generalizaciones de la entrop\'ia de Shannon por un lado, y de
la divergencia de Kullback-Leibler por  el otro lado. No son siempre vinculados,
a pesar  de que sea  deseable que a  cada entrop\'ia sean asociados  nociones de
entrop\'ias condicionales y relativas.

% ================================= Salicru

\subseccion{Entrop\'ias y propiedades}
\label{Ssec:SZ:Salicru}

% ----- Entropias generalizadas particulares
%\paragraph{Unas  primeras  generalizaciones particulares}  
Si  la entrop\'ia  de Shannon  fue el  punto de  salida fundamental  en  todo el
desarollo  de la  teor\'ia de  la  informaci\'on, un  poco m\'as  de una  decada
despu\'es  de  su  papel  clave  y  muy completo,  R\'enyi  propuso  una  medida
generalizada~\cite{Ren61}.   Su  punto  de  vista  fue  m\'as  matem\'atico  que
f\'isico o ingeniero.  Retom\'o los axiomas de Fadeev~\cite{Fad56, Fad58, Khi57}
% Feinstein, cf ref  de R��nyi
%
para  probabilidades  incompletas~\footnote{En  esta  secci\'on, los  $p_i$  son
  componentes  del  vector  $p$   no  necesariamente  asociado  a  una  variable
  aleatoria; Hay  que entender de  que $p_i =  p_X(x_i)$ si son asociados  a una
  variable aleatoria $X$.}  $p = \begin{bmatrix} p_1 & \cdots & p_n
\end{bmatrix}^t,  \quad p_i  \ge  0,  \quad w_p  =  \sum_i p_i  \le  1$: (i)  la
invarianza de  $H(p)$ por permutaci\'on de  os $p_i$, (ii) la  continuidad de la
incerteza elemental  $H(p_i)$ ($p_i$ visto como  probabilidad incompleta), (iii)
$H\left( \frac12 \right) = 1 $,  (iv) la aditividad $H(p \otimes q) = H(p)+H(q)$
donde $p \otimes  q$ es el producto \modif{externo (ver notaciones)},
% ~\footnote{Ver           nota          de          pie~\ref{Foot:MP:Kronecker}
%   pagina~\pageref{Foot:MP:Kronecker}.},
\ie  probabilidad conjunta  de dos  variables independientes,  y  consider\'o en
lugar de la recursividad un axioma  dicho de valor promedio, axioma muy parecido
a la recursividad. Para $p$ \ y \ $q$ probabilidades incompletas tales que $p \,
\cup  \,  q   =  \begin{bmatrix}  p_1  &   \cdots  &  p_n  &  q_1   &  \cdots  &
  q_m \end{bmatrix}^t$ sea incompleta ($w_p + w_q \le 1$), el axioma (v) es $H(p
\, \cup \,  q) = \frac{w_p \, H(p)  + w_q \, H(q)}{w_p +  w_q}$.  Demostr\'o que
con (v) en lugar  de la recursividad, el conjunto de axiomas  conduce de nuevo a
la  entrop\'ia de  Shannon.  La  generalizaci\'on propuesta  por R\'enyi  era de
generalizar  el axioma  (v) reemplazando  la  media aritm\'etica  por una  media
generalizada (v')  $H^\ren(p \, \cup \,  q) = g^{-1} \left(  \frac{w_p \, g\big(
    H^\ren(p) \big) + w_q \,  g\big( H^\ren(q) \big)}{w_p + w_q}\right)$ con $g$
estrictamente mon\'otona  y continua,  llamada media {\it  cuasi-aritm\'etica, o
  cuasi-lineal,  o  de  Kolmogorov-Nagumo}.   De  las propiedades  de  la  media
cuasi-aritm\'etica~\cite{Nag30, Kol30:2, Kol91,  HarLit52}, eso es equivalente a
buscar una entrop\'ia elemental $H^\ren(p_i)$ y reemplazar la media aritm\'etica
\ $\sum_i  p_i H^\ren(p_i)$ por  una media de Kolmogorov-Nagumo,  $g^{-1} \left(
  \sum_i p_i g\big( H^\ren(p_i)  \big) \right)$.  R\'enyi propus\'o la funci\'on
de  Kolmogorov-Nagumo \  $g_\lambda(x) =  2^{(\lambda-1) x},  \quad  \lambda >0,
\quad \lambda  \ne 1$, probando  de que los axiomas  (i)-(ii)-(iii)-(iv)-(v') se
cumplen, conduciendo  a la  entrop\'ia de R\'enyi  de un vector  de probabilidad
$p$,
%
\[
H_\lambda^\ren(p) =  \frac{1}{1-\lambda} \log_2 \left(  \sum_{i=1}^n p_i^\lambda
\right).
\]
%
\noindent  Relaxando   el  axioma  (iii),   se  puede  elegir   $g_\lambda(x)  =
a^{(\lambda-1) x}, \quad a  > 0, \quad a \ne 1$; el  logaritmo ser\'a de la base
$a$ cualquiera. En  lo que sigue, usaremos $\log$ sin  precisar la elecci\'on de
base.   R\'enyi nombr\'o  esta  medida  de incerteza  {\it  entrop\'ia de  orden
  $\lambda$}. Notablemente,
%
\[
H_1^\ren(p)  \equiv   \lim_{\lambda  \to  1}  H_\lambda^\ren(p)   =  H(p)  \quad
\mbox{entrop\'ia de Shannon.}
\]
%
\noindent En otros t\'erminos, la clase de R\'enyi contiene como caso particular
la entrop\'ia de Shannon.  Adem\'as, en su papel, R\'enyi introdujo una ganancia
de informaci\'on,  parecida a  una entrop\'ia relativa,  probando que  las solas
entrop\'ias admisibles son  la de Shannon y la que  introdujo.  Volveremos en la
secci\'on siguiente  sobre esta entrop\'ia  relativa, o divergencia  de R\'enyi.
Por          axiomas,          las         propiedades~\ref{Prop:SZ:continuidad}
(continuidad),~\ref{Prop:SZ:permutacion}    (invarianza    por    permutaci\'on)
y~\ref{Prop:SZ:aditividad}  (additividad)   de  la  entrop\'ia   de  Shannon  se
conservan entonces en el marco de R\'enyi y se pierde~\ref{Prop:SZ:recursividad}
(recursividad), todav\'ia por axiomas. Veremos luego la otras que se conservan o
modifican en un marco m\'as general.

Unos a\~nos  despu\'es de R\'enyi, de  la famosa escuela  matem\'atica checa, J.
Havrda   \&    F.    Charv\'at   en~\cite{HavCha67}    (ver   tambi\'en~\cite[en
checo]{Vaj68}) volvieron a los axiomas de Khintchin, para extender la entopia de
Shannon,  \ie  considerando  (i)   la  invarianza  por  permutaci\'on,  (ii)  la
continuidad, (iii) la expansividad, (iv) $H^\hc(1) = 0$ y $H^\hc\left( \frac12 ,
  \frac12   \right)  =   1$,  pero   generalizando  la   recursividad   por  (v)
$H^\hc(p_1,\ldots,p_n)       =      H^\hc(p_1,\ldots,p_{n-2},p_{n-1}+p_n)      +
(p_{n-1}+p_n)^\lambda         H^\hc\left(        \frac{p_{n-1}}{p_{n-1}        +
    p_p},\frac{p_n}{p_{n-1} + p_p} \right),  \quad \lambda > 0$~\footnote{En sus
  papel, lo  imponen para cualquier par  $(p_i, p_j)$ sin  imponer la invarianza
  por   permutaci\'on,  pero   es  equivalente   a  la   exposici\'on   de  este
  p\'arrafo. Adem\'as,  ponen un factor $\alpha >  0$ a $(p_{n-1}+p_n)^\lambda$,
  pero prueban que  para satisfacer a (iv), necesariamente  $\alpha = 1$.}.  Con
$\lambda = 1$ se recupera la recursividad estandar, pero con $\lambda \ne 1$ eso
permite  dar un peso  diferente a  la incerteza  del estado  interno, \ie  a las
probabilidades  que  se juntan  (la  describen  como clasificaci\'on  refinada).
Estos axiomas conducen necesariamente a la entrop\'ia (teorema~1 del papel)
%
\[
H_\lambda^\hc(p) = \frac{1}{1-2^{1-\lambda}} \left( 1 - \sum_i p_i^\lambda \right)
\]
%
que  nombraron {\it  $\lambda$-entrop\'ia structural}.   De nuevo,  relaxando el
axioma  (iv),   se  puede  reemplazar  en  el   coeficient  $2^{1-\lambda}$  por
$a^{1-\lambda}, \quad a > 0, \quad a \ne 1$. De nuevo, aparece que la entrop\'ia
de Shannon es un caso particular,
%
\[
H_1^\hc(p)  \equiv   \lim_{\lambda  \to   1}  H_\lambda^\hc(p)   =   H(p)  \quad
\mbox{entrop\'ia de Shannon.}
\]
%
Por axioma, se conservan las propiedades~\ref{Prop:SZ:continuidad} (continuidad)
y~\ref{Prop:SZ:expansabilidad}  (expansabilidad) de  Shannon en  este  marco. Se
prob\'o tambi\'en que se conserva la  propiedad de concavidad con respecto a los
$p_i$~\ref{Prop:SZ:concavidad},   la   de   maximalidad~\ref{Prop:SZ:cotamaxima}
alcanzada para una distribuci\'on uniforma (teorema~2). Aun que no aparece as\'i
en        el        papel,        satisface        la        propiedad        de
Schur-concavidad~\ref{Prop:SZ:Schurconcavidad}  (teorema~3).   A  pesar  de  que
mencionan que $H_\lambda^\hc$ sea diferente de $H_\lambda^\ren$, es sencillo ver
que hay un mapa uno-uno entre las dos entrop\'ias.  Se mencionar\'an en un marco
m\'as general otras propiedades de esta entrop\'ia.

Independiente de Havrda  \& Charv\'at, de la escuela h\'ugara  de la teor\'ia de
la informaci\'on,  Z.  Dar\'oczy en~\cite{Dar70} defin\'o la  entrop\'ia $H^f$ a
partir de una {\it funci\'on  informaci\'on} $f$ satifaciendo (i) $f(0) = f(1)$,
(ii) $f\left(\frac12\right) = 1$ \ y  la ecuaci\'on funcional (ii) $f(x) + (1-x)
f\left(  \frac{y}{1-x} \right)  = f(y)  + (1-y)  f\left(  \frac{x}{1-y} \right)$
sobre  $\{  (x,y)  \in [0  \;  1)^2,  \quad  x+y  \le  1 \}$,  siendo  $H^f(p)  =
\sum_{i=2}^n s_i  f\left( \frac{p_i}{s_i} \right), \quad  s_i = \sum_{j=1}^{i-1}
p_j$.   Dar\'oczy mostr\'o  que  si $f$  es medible,  o  continua en  $0$, o  no
negativa  y  acotada, necesariamente  $f(x)  =  h_2(x) =  -x  \log_2  x -  (1-x)
\log_2(1-x)$,   conduciendo  a   la  entrop\'ia   de  Shannon   (teorema~1;  ver
tambi\'en~\cite{Lee64, Tve58,  Ken64}).  En otros  t\'erminos, su axioma  (v) es
alternativa a la recursividad.  Para  extender la entrop\'ia de Shannon, propuso
extender  este   axioma  (v)  por   la  ecuaci\'on  funcional   $f_\lambda(x)  +
(1-x)^\lambda   f_\lambda\left(   \frac{y}{1-x}   \right)   =   f_\lambda(y)   +
(1-y)^\lambda   f_\lambda\left(   \frac{x}{1-y}   \right)$,   lo   que   condujo
necesariamente a la entrop\'ia (teoremas~2 y~3)
%
\[
H_\lambda^\dar(p) = \frac{1}{1-2^{1-\lambda}} \left( 1 - \sum_i p_i^\lambda \right),
\]
%
\noindent  es decir  nada  m\'as que  la  entrop\'ia introducida  por Havdra  \&
Charv\'at. En lo  que sigue, se la denotar\'a $H_\lambda^\hcd$.  Sin embargo, el
estudio  de  Dar\'oczy fue  m\'as  intensivo  que  el  de Havdra  \&  Charv\'at.
Primero, not\'o  el mapa entre su  entrop\'ia y la de  R\'enyi. Adicionalmente a
Havdra-Charv\'at prob\'o que  se conserva la propiedad~\ref{Prop:SZ:permutacion}
(invarianza  por   permutaci\'on,  que  no   era  un  axioma  en   su  enfoque),
$H_\lambda^\hcd\left( \frac12 , \frac12 \right) = 1$ (lo llama normalizaci\'on),
la  expansividad~\ref{Prop:SZ:expansabilidad},  una  aditividad  extendida,  una
recursividad extendida precisamente del  modelo de Havrda-Charv\'at (teorema~4).
Prob\'o  tambi\'en~\ref{Prop:SZ:positividad}, positividad  alcanzado en  el caso
determinista  y  la  maximalidad~\ref{Prop:SZ:cotamaxima} en  el  caso  uniforme
(teorema~6), que  incidentalmente $H_\lambda^\hcd\left( \frac{1}{|\X|}  , \ldots
,  \frac{1}{|\X|}  \right)$  crece  con el  cardenal  $|\X|$.   Muy  interesante
tambi\'en es que se puede definir  una entrop\'ia condicional en el mismo modelo
que en el caso de Shannon, \ $H_\lambda^\hcd(X|Y) = \sum_y \left[ p_{X|Y=y}(x)
\right]^\lambda  H_\lambda^\hcd(   p_{X|Y=y}  )$,   que  existe  una   regla  de
la  cadena~\ref{Prop:SZ:cadena},  $H_\lambda^\hcd(X,Y)   =  H_\lambda^\hcd(Y)  +
H_\lambda^\hcd(X|Y)$ y que condicionar reduce la entrop\'ia $H_\lambda^\hcd(X|Y)
\le    H_\lambda^\hcd(X)$    (teorema~8)~\ref{Prop:SZ:condicionar}.     Mostr\'o
tambi\'en que  si se  pierde la aditividad,  se obiene para  \ $X$  \ e \  $Y$ \
independientes \ $H_\lambda^\hcd(X,Y)  = H_\lambda^\hcd(X) + H_\lambda^\hcd(Y) +
\left(  2^{1-\lambda}  -  1  \right) H_\lambda^\hcd(X)  H_\lambda^\hcd(Y)$.   La
propiedades de regla  de la cadena le permiti\'o  revisitar la caracterisaci\'on
de  un  canal  de  transmisi\'on  y redefinir  una  capacidad  canal  extendidas
(capacidad tipo $\lambda$; basicamente se usa el mismo enfoque que Shannon, pero
usando $H_\lambda^\hcd$ en lugar de $H$, ver secci\'on~6 del papel).

\

Las entrop\'ias  tipo Havdra-Charv\'at-Dar\'oczy fueron  (re)descubiertos varias
otras  veces  y/o estudiadas  m\'as  detenidamente  en  varios campos  y  varios
extensiones fueron  introducidas~\cite[entre otros]{Var66, Oni66,  Kap67, Vaj68,
  LinNie71,  Ari71,  Bur72,   AczDar75,  ShaMit75,  ShaMit75,  ShaTan75,  Mit75,
  BoeLub80, Fer80, Tsa88, Rat91, Kan01, Bec09}.  Un primer enfoque m\'as general
es debido a  S.  Arimoto en los primeros a\~nos  de la decada 1970~\cite{Ari71}.
Fue rediscubierto  y estudiado  con m\'as detalles  una decada despu\'es  por J.
Burbea y C.  R.  Rao~\cite{BurRao82} y luego por M.  Salicr\'u~\cite{Sal87} o M.
Teboulle~\cite{Teb92}   entre  otros.    La  medida   propuesta,   llamada  {\it
  $\phi$-entrop\'ia}, es definida por
%
\[
\hphi[p]  =   -\sum_i  \phi(p_i)  \qquad   \mbox{con}  \qquad  \phi   \:  \mbox{
  estrictamente convexa.}
\]
%
Burbea  y  Rao  asociaron una  medida  de  divegencia  a esta  entrop\'ia.   Las
$\phi$-entrop\'ias  contienen Shannon como  caso particular  ($\phi(x) =  x \log
x$),  as\'i que  la clase  de Havdra-Charv\'at-Dar\'oczy  ($\phi(x) =  \frac{x -
  x^\lambda}{2^{1-\lambda}-1}$) como mencionado, pero no la clase de R\'enyi. De
hecho, las  $\phi$-entrop\'ias se  enmarcan en una  clase un poco  m\'as amplia,
llamada {\it  $(h,\phi)$-entrop\'ias}~\cite{SalMen93, MenMor97}. Cambiamos ac\'a
substancialmente su escritura  de la literatura por razones  de homogeneidad con
la    $\phi$-entrop\'ia   (y    las   divergencias    que    se   introducir\'an
luego)~\footnote{En  la literatura,  no hay  el signo  $-$, y  hay  que invertir
  c\'oncava y convexa.}
%
\begin{definicion}[$(h,\phi)$-entrop\'ia]
\label{Def:SZ:HPhiEntropia}
%
  La $(h,\phi)$-entrop\'ia de una  distribuci\'on de probabilidad $p_X$ definida
  sobre $\X$ de cardenal finito $|\X| < +\infty$ es definida por
  %
  \[
  \hhphi[X] \equiv  \hhphi[p_X] =  h\left( - \sum_{x  \in \X}  \phi\left( p_X(x)
    \right) \right),
  \]
  %
  donde o
  %
  \begin{itemize}
  \item $\phi$ \ es estrictamente convexa y \ $h$ \ creciente, o
  \item $\phi$ \ es estrictamente c\'oncava y \ $h$ \ decreciente
  \end{itemize}
  %
  Frecuentemente, se supone adicionalmente que  $\phi$ y $h$ son de clase $C^2$,
  que $\phi(0) = 0$ (la incerteza elemental asociada a un estado de probabilidad
  nula vale cero) y, sin perdida de generalidad, que $h(-\phi(1)) = 0$.
\end{definicion}
%
\noindent  (ver  tambi\'en~\cite{Est97}  para  una  generalizaci\'on  aun  m\'as
amplia). Cu\'ando $h(x) = x$  se recupera la $\phi$-entrop\'ia, incluyendo la de
Shannon y las de Havdra-Charv\'at-Dar\'oczy. Adem\'as, la familia de R\'enyi cae
tamb\'ien en  esta familia  ($\phi(x) = -  x^\lambda$ \  y \ $h(x)  = \frac{\log
  x}{1-\lambda}$)  as\'i que  todas  las entrop\'ias  evocadas  en el  p\'arrafo
anterior.

Como   en   el   caso   de    Shannon,   para   $X   =   (X_1,\ldots,X_d)$,   la
$(h,\phi)$-entrop\'ia de $X$ es una $(h,\phi)$-entrop\'ia conjunta de los $X_i$.

Obviamente, de  las propiedades  de la entrop\'ia  de Shannon, se  conservan las
propiedades~\ref{Prop:SZ:continuidad}    (continuidad),~\ref{Prop:SZ:permutacion}
(invariaza    por   permutaci\'on),~\ref{Prop:SZ:biyeccion}    (invarianza   por
transformaci\'on       biyectiva      de      $X$),~\ref{Prop:SZ:expansabilidad}
(expansabilidad, debido a $\phi(0) = 0$).

Adem\'as  se conserva  la  Schur-concavidad \ con una rec\'iproca:
%
\begin{propiedadesPhi}\setcounter{enumi}{\value{PropSchurConcavidad}}
%
\item Schur-convavidad:
  %
  \[
  p \prec q \quad \Longleftrightarrow \quad \hhphi[p] \ge \hhphi[q] \quad \forall \:
  (h,\phi).
  \]
  %
  En otros t\'erminos,  se obtiene la relaci\'on de  mayorisaci\'on si se cumple
  la  desigualdad entr\'opica  para todos  los pares  de  funciones entr\'opicas
  $(h,\phi)$.   La  Schur-concavidad (y  su  rec\'iproca)  es  consecuencia de  la
  desigualdad  de Schur~\cite{Sch23}  o Hardy-Littlewood-P\'olya~\cite{HarLit29,
    HarLit52} o Karamata~\cite{Kar32}  (ver tambi\'en~\cite[Cap.~3, Prop.~C.1 \&
  Cap.~4,  Prop.~B.1]{MarOlk11} o~\cite[Teorema~II.3.1]{Bha97}):  $p \prec  q \:
  \Rightarrow  \: \sum_i  \phi(p_i) \le  \sum_i \phi(q_i)$  para  toda funci\'on
  $\phi$ convexa, y reciprocamente.
\end{propiedadesPhi}
% Schur, I.  (1923). Issai Schur Collected  Works (A.  Brauer  and H.  Rohrbach,
% eds.), Vol. II. pp. 416���427. Springer-Verlag, Berlin, 1973]
%
%
% Ver Schur-Ostrowski f  sym, Schur-convexe ssi (xi - xj)  (df/dx_i - df/dxj) \ge
% 0, 1 \le i \ne j \le alpha
%
Como consecuencia, se  conservan la positividad~\ref{Prop:SZ:positividad} gracia
a \ $\phi(0) = 0$ \ y \ $h(-\phi(1)) = 0$ \ (alcanzado en el caso determinista),
la maximalidad~\ref{Prop:SZ:cotamaxima} (caso uniforme),
%
\[
0 \le \hhphi[p_X] \le h\left( - |\X| \, \phi\left( \frac{1}{|\X|} \right) \right),
\]
%
as\'i que
%
\[
\hhphi[\begin{bmatrix}  \frac{1}{|\X|} &  \cdots  & \frac{1}{|\X|}  \end{bmatrix}^t]
\quad \mbox{funci\'on creciente de } |\X|.
\]

Con respecto a la concavidad~\ref{Prop:SZ:concavidad}, no se conserva en general:
%
\begin{propiedadesPhi}\setcounter{enumi}{\value{PropConcavidad}}
\item\label{Prop:SZ:concavidadHPhi} Si $h$ es c\'oncava, entonces $\hhphi[p]$ es
  c\'oncava con  respecto a $p$.   Eso es una  consecuencia de la  concavidad de
  $\phi$ y  decrecencia de $h$ (resp.  convexidad/crecencia)  conjuntamente a la
  concavidad de $h$.  La rec\'iproca no  es verdad.  Por ejemplo, se puede ver que
  si $\lambda  < 1$, la  entrop\'ia de R\'enyi  es c\'oncava, pero se  proba que
  existe  un  $\lambda^c(|\X|)  >  1$  tal que  para  cualquier  $\lambda  \le
  \lambda^c(|\X|)$  se conserva  la  concavidad, a  pesar  de que  $h$ no  sea
  necesariamente c\'oncava~\cite[p.~57]{BenZyc06}.
\end{propiedadesPhi}

Se pierde la propiedad de recursividad~\ref{Prop:SZ:recursividad}, pero se puede
vincular  la  entrop\'ia  total con  la  obtenida  juntando  dos estados  por  una
desigualdad: \SZ{Revoir pour faire comme avec Shannon}
%
\begin{propiedadesPhi}\setcounter{enumi}{\value{PropRecursividad}}
\item  Sean \  $X$  \ definido  sobre  \ $\X$   \ con  $|\X| > 2$,  y \  $\breve{X}$ \  sobre \
  $\breve{\X}$  \  de   cardenal  \  $|\breve{\X}|  =  |\X|-1$   \  formado  por
  agregaci\'on de los dos \'ultimos estados de $\X$, \ie
  %
  \begin{itemize}
  \item    $\breve{\X}    =    \big(    \X    \setminus    \{    x_{|\X|-1}    ,
  x_{|\X|}  \} \big)  \cup \{  \breve{x}_{|\X|-1} \}$ \quad con  el estado
  interno \quad $\breve{x}_{|\X|-1} = \{ x_{|\X|-1} \, , \, x_{|\X|} \}$,
  %
  \item $p_{\breve{X}}(x_i) = p_X(x_i), \quad \forall \, x_i \in \X \setminus \{
      x_{|\X|-1} , x_{|\X|} \}$ \quad y \quad $p_{\breve{X}}(\breve{x}_{|\X|-1})
      = p_X(x_{|\X|-1}) + p_X(x_{|\X|})$, distribuci\'on correspondiente sobre \
      $\breve{\X}$,
  %  
  \item  $\displaystyle   \breve{q}(x_j)  =   \frac{p_X(x_j)}{p_X(x_{|\X|-1})  +
        p_X(x_{|\X|})},  \quad j  = |\X|-1,  |\X|$ \quad  distribuci\'on de  los
        estados  \ $\{  x_{|\X|-1} \,  , \,  x_{|\X|} \}$  \ adentro  del estado
        interno.
  %
  \end{itemize}
  %
  %% \[
  %% \left\{  \begin{array}{l}\breve{\X}  = \{  x_1 \,  ,  \, \ldots \,  , \, x_{|\X|-2} \,  ,
  %%     \, \breve{x}_{|\X|-1} \} \quad \mbox{con el estado interno} \quad
  %%     \breve{x}_{|\X|-1}   =  \{   x_{|\X|-1} \,  , \,   x_{|\X|}  \},\\[2.5mm]
  %%     p_{\breve{X}}(x_i) = p_X(x_i), \quad 1 \le i \le |\X|-1 \quad \mbox{y}
  %%     \quad p_{\breve{X}}(\breve{x}_{|\X|-1}) = p_X(x_{|\X|-1}) +
  %%     p(x_{|\X|})  \quad  \mbox{distribuci\'on  sobre  }  \breve{\X}\\[2.5mm]
  %%     \displaystyle   \breve{q}(x_j)  =   \frac{p_X(x_j)}{p_X(x_{|\X|-1})  +
  %%       p_X(x_{|\X|})},  \quad j =  |\X|-1, |\X|  \quad \mbox{distribuci\'on
  %%       del estado interno}\end{array}\right.,
  %% \]
  %
  entonces,
  %
  \[
  \hhphi[p_X] \ge \hhphi[p_{\breve{X}}].
  \]
  %
  Esta  desigualdad es  consecuencia de  la desigualdad  de Petrovi\'c~\cite[43,
  Teorema~8.7.1]{Kuc09}, $\phi(a + b) \ge \phi(a) + \phi(b)$ para $\phi$ convexa
  y que se cancela en 0 (y  la conversa en el caso c\'oncavo), conjuntamente con
  $h$ creciente  (resp.  decreciente).   Aparte en  el caso de  Shannon y  el de
  Havdra-Charv\'at-Dar\'oczy, no hay  ning\'un v\'inculo explicito general entre
  $\hhphi[p_X]$ \ y \ $\hhphi[p_{\breve{X}}]$.
\end{propiedadesPhi}

Se  conserva  la  super-aditividad~\ref{Prop:SZ:superaditividad}. De  hecho,  si
$\phi$ es convexa (resp.  c\'oncava) con \  $\phi(0) = 0$, \ $\forall \: 0 \le a
\le 1,  \: \phi(a u) =  \phi(a u + (1-a)  0) \le a  \phi(u)$ (resp.  desigualdad
reversa).   Entonces,  \   $\phi\left(  p_{X,Y}(x_i,y_j)  \right)  =  \phi\left(
  p_{X|Y=y_j}(x_i)  p_Y(y_j)  \right)  \le  p_{X|Y=y_j}(x_i)  \phi\left(p_Y(y_j)
\right)$, \ \ie \ $\sum_{i,j} \phi\left( p_{X,Y}(x_i,y_j) \right) \le \sum_{i,j}
p_{X|Y=y_j}(x_i) \phi\left(p_Y(y_j) \right) = \sum_i \phi\left(p_Y(y_j) \right)$
\  (resp.   desigualdad  reversa).   Se   cierra  la  prueba  con  la  crecencia
(resp. decrecencia) de $h$.


Sin  embargo, en  general, se  pierden  las propiedades~\ref{Prop:SZ:aditividad}
(aditividad), y~\ref{Prop:SZ:subaditividad}  (sub-aditividad). En particular, se
conserva solamente en el caso Shannon:
%
\begin{teorema}
\label{Teo:SZ:SoloShannonSubaditiva}
%
  Sea $p_{X,Y}$ distribuci\'on conjunta  de variables aleatorias discretas \ $X$
  \ y \ $Y$ \ y \ $p_X$ \ y \ $p_Y$ \ las de \ $X$ \ y de \ $Y$ (marginales).
  %
  \[
  \hhphi[p_{X,Y}] \:  \le \:  \hhphi[p_X \otimes p_Y]  \quad \forall  \: p_{X,Y}
  \qquad \Longleftrightarrow \qquad \phi(x) = - x \log x,
  \]
  %
  \ie $\hhphi$ es una funci\'on creciente de la entrop\'ia de Shannon.
\end{teorema}
%
\begin{proof}
  La    rec\'iproca     de    este    teorema    es    nada     m\'as    que    la
  propidad~\ref{Prop:SZ:subaditividad} con el hecho que $h$ es creciente en este
  caso.

  A continuaci\'on, la parte directa se demuestra en dos etapas:
  %
  \begin{itemize}
  \item Con  un caso particular  sobre $\X$  e $\Y$ de  cardenal 3 cada  unos se
    proba  de que  la desigualdad  no  se puede  cumplir, salvo  si la  derivada
    $\phi'$ de la funci\'on entr\'opica satisface a una ecuaci\'on funcional.
  %
  \item la sola soluci\'on admisible de esta ecuaci\'on se reduce a $\phi(x) = -
    x \log x$.
  \end{itemize}
  %
  {\bf Etapa 1}: Sea el vector de probabildad
  %
  \[
  p_{X,Y}  =  p_X  \otimes   p_Y  -  c  \begin{bmatrix}  1\\-1\\0  \end{bmatrix}
  \otimes  \begin{bmatrix} 1\\-1\\0 \end{bmatrix}  \qquad \mbox{con}  \qquad p_X
  = \begin{bmatrix} a \\ \tilde{a} \\ 1-a-\tilde{a} \end{bmatrix} \quad \mbox{y}
  \quad p_Y = \begin{bmatrix} b \\ \tilde{b} \\ 1-b-\tilde{b} \end{bmatrix}
  \]
  %
  donde $(a,\tilde{a},b,\tilde{b}) \in D$,%_{a,\tilde{a},b,\tilde{b}}$,
  %
  \[
  %D_{a,\tilde{a},b,\tilde{b}} = \{  u, v, s, t:  \quad 0 < u,s <  1 \quad \&
  %\quad 0 < v \le 1-u \quad \& \quad 0 < t \le 1-s \},
  D =  \{ (u, \tilde{u},  v, \tilde{v}) \in  [0 \; 1]^4: \quad  0 <
    \tilde{u} \le 1-u \quad \& \quad 0 < \tilde{v} \le 1-v \},
  \]
  %
  y \ $c \in C_{a,\tilde{a},b,\tilde{b}}$,
  %
 \[
  C_{a,\tilde{a},b,\tilde{b}} = \big[ - 1 + \max\big\{ a b , \tilde{a} \tilde{b}
  , 1 - a \tilde{b} , 1 - \tilde{a}  b \big\} \, , \, \min\big\{ a b , \tilde{a}
  \tilde{b} , 1 - a \tilde{b} , 1 - \tilde{a} b \big\} \big].
  \]
  %
  Ahora, si $\phi$ es  convexa (resp. c\'oncava)
  %
  \[
  \forall \, u,v \quad \phi(v)  - \phi(u) \:  \ge \:  (v-u) \, \phi'(u),
  \]
  %
  \ie la  variaci\'on (cuerda) es mayor  que la derivada en  $u$, como ilustrado
  figura  Fig.~\ref{Fig:SZ:ConvexidadDerivada} (desigualdad reversa  para $\phi$
  c\'oncava).
  %
  \begin{figure}[h!]
  %
  \begin{center} \begin{tikzpicture}
\shorthandoff{>}
%
% Concavidad de - u ln u
\begin{scope}[xscale=6.5,yscale=3.5]
\pgfmathsetmacro{\u}{.25};
\pgfmathsetmacro{\v}{1};
\pgfmathsetmacro{\l}{.7};
%
\draw[>=stealth,->] (-.1,-.5)--(1.6,-.5) node[right]{\small $t$};
\draw[>=stealth,->] (0,-.6)--(0,.5) node[above]{\small $\phi(t)$};
\draw[thick,domain=.005:1.2,samples=200] (0,0)-- plot (\x,{\x*ln(\x)});
\draw[dotted] (\u,-.55) node[below]{\small $u$} -- (\u,{\u*ln(\u)});
\draw[dotted] (\v,-.55) node[below]{\small $v$} -- (\v,{\v*ln(\v)});
%
%\draw (\u,{-\u*ln(\u)})--(\v,{-\v*ln(\v)});
\draw (.02,{(.02-\u)*(\v*ln(\v)-\u*ln(\u))/(\v-\u)+\u*ln(\u)})
-- (1.25,{(1.25-\u)*(\v*ln(\v)-\u*ln(\u))/(\v-\u)+\u*ln(\u)})
node[right,scale=.7]
{$\displaystyle \quad \frac{\phi(v)-\phi(u)}{v-u} \, (t-u) + \phi(u)$};;
%
%\draw ({.5*(\u+\v)},{.5*(\v-\u)*(\u*ln(\u)-\v*ln(\v))/(\v-\u)-\u*ln(\u)-.05})
%node[below,scale=.7]{$\displaystyle \phi(u) + (t-u) \, \frac{\phi(v)-\phi(u)}{v-u}$};
%
\draw (.02,{(1+ln(\u))*(.02-\u)+\u*ln(\u)})--
(.5,{(1+ln(\u))*(.5-\u)+\u*ln(\u)})
node[right,scale=.7]{$\quad \phi'(u) \, (t-u) + \phi(u)$};
\end{scope}
%
%
% % Concavidad / mezcla
% \begin{scope}[xshift=8.5cm]
% \draw(0,1.25) node{\includegraphics[width=3cm]{TIKZ_SZ/DosDados}};
% \draw(-.5,2.5) node{\small $p_1$};
% \draw(1,2) node{\small $p_2$};
% \draw(2.7,1) node{\small $\lambda p_1 + (1-\lambda) p_2$};
% \draw(-.25,-1) node{\includegraphics[width=1cm]{TIKZ_SZ/Moneda}};
% \draw[>=stealth,->,thick] (-.3,-.35)--(-.75,.45);
% \draw (-.525,0) node[left]{\small $\lambda$};
% \draw[>=stealth,->,thick] (-.2,-.35)--(.3,.45);
% \draw (.05,0) node[right]{\small $1-\lambda$};
% \end{scope}
% %
% \draw (1.25,-2.25) node{(a)};
% \draw (8.25,-2.25) node{(b)};
\end{tikzpicture} \end{center}
  %
  \leyenda{$\phi$ estrictamente convexa:  la variaci\'on (cuerda) $\frac{\phi(v)
      -  \phi(u)}{v-u}$ es  mayor que  la derivada  $\phi'(u)$.  Aplicado  a dos
    distribuciones $p$ y $q$, de componentes $p_i$ y $q_i$, con $u = p_i$ y $v =
    q_i$  y  sumando, se  obtiene  $\hphi[q]  -  \hphi[p] \ge  \sum_i  (p_i-q_i)
    \phi'(p_i)$ con  $\hphi \equiv H_{(\id,\phi)},  \: \id$ siendo  la identidad
    (ver notaciones).}
  %
  \label{Fig:SZ:ConvexidadDerivada}
  \end{figure}
  %
  Aplicamos esta desigualdad a \ $u =  p_{X,Y}(x,y)$ \ y \ $v = p_X(x) p_Y(y)$ \
  y sumamos  en $x,  y$, \  para \  $(a,b) \in (  0 \;  1)^2$ \  (para que
  $C_{a,\tilde{a},b,\tilde{b}}$  \ no  sea reducido  a \  $\{0\}$), y  \  $c \in
  \mathring{C}_{a,\tilde{a},b,\tilde{b}}$ \ donde  \ $\mathring{\cdot}$ \ denota
  el interior de un conjunto, se obtiene para $\phi$ convexa,
  %
  \[
  \hphi[p_X    \otimes   p_Y]    -    \hphi[p_{X,Y}]   \:    \le    \:   c    \:
  g(a,\tilde{a},b,\tilde{b},c)
  \]
  %
  (para $\phi$  c\'oncava se  reemplaza $\hphi$ por  $-H_{-\phi}$ con  la igualdad
  inversa), donde
  %
  \begin{equation}
  g(a , \tilde{a} , b , \tilde{b} , c) = \phi'\big( a b + c \big) + \phi'\big(
  \tilde{a} \tilde{b} + c \big) - \phi'\big( a \tilde{b} - c \big) - \phi'\big(
  \tilde{a} b - c \big).
  \end{equation}
  %
  Supongamos que existe  un \ $(s,\tilde{s},t,\tilde{t}) \in D$,  con $(s,t) \in
  (0 \; 1)^2$,
  % \mathring{D}_{a,\tilde{a},b,\tilde{b}}$
  \  tal que  \  $g(s,\tilde{s},t,\tilde{t},0)  \ne 0$.   De  la continuidad  de
  $\phi'$,  la funci\'on $g$  es continua,  entonces existe  un vecinaje  \ $V_0
  \subset  \mathring{C}_{s,\tilde{s},t,\tilde{t}}$  \ de  \  $0$  \  tal que  la
  funci\'on  \  $c  \mapsto   g(s,\tilde{s},t,\tilde{t},c)$  \  tiene  un  signo
  constante  sobre \  $V_0$.   \ Eso  permite concluir  que  \ $c  \mapsto c  \,
  g(s,\tilde{s},t,\tilde{t},c)$ \ no tiene un signo constante sobre \ $V_0$, \ y
  entonces de concluire que, de la  desigualdad dedibo a la concavidad de $\phi$
  (resp.   convexidad), \ $\hphi[p_{X,Y}]$  puede ser  major (resp.   menor) que
  $\hphi[p_X \otimes p_Y]$, y entonces, con la crecencia (resp.  decrecencia) de
  $h$ que si $g(a,\tilde{a},b,\tilde{b},0)$ no es identicamente cero sobre
  $D$, %$\mathring{D}_{a,\tilde{a},b,\tilde{b}}$,
  $\hhphi$ no puede ser sub-aditiva  (distribuci\'on conjunta vs producto de las
  marginales).

  {\bf Etapa 2}. De la etapa  1, se sabe que la sub-aditividad es potencialmente
  posible  solamente si  $g(u,v,s,t,0) =  0$  sobre $D_{a,\tilde{a},b,\tilde{b}}
  \cap  (0  \,  ; \,  1)^4$.   Eso  significa  que $\phi'$  debe  necesariamente
  satisfacer la ecuaci\'on funcional
  %
  \[
  \phi'\big( a  b \big)  + \phi'\big( \tilde{a}  \tilde{b} \big) -  \phi'\big( a
  \tilde{b} \big) - \phi'\big( \tilde{a} b \big) = 0,
  \]
  %
  as\'i que  no se  puede usar el  argumento de  la etapa~1 para  concluir.  Sin
  embargo,     se     puede     solucionar    esta     ecuaci\'on     funcional,
  siguiendo~\cite[\S~6]{DarJar79} donde una  ecuaci\'on funcional muy similar es
  estudiada.  Por  eso, se fija \  $(a,b) \in (0 \; 1)^2$, \  se deriva la
  identidad  precediente  con  respecto  a  \ $\tilde{a}$  \  se  multiplica  el
  resultado por $\tilde{a}$ \ para obtener
  %
  \[
  \tilde{a}  \tilde{b}   \,  \phi''(\tilde{a}   \tilde{b})  =  \tilde{a}   b  \,
  \phi''(\tilde{a} b) \qquad \mbox{para}  \qquad (\tilde{a},\tilde{b}) \in (0 \;
  1-a) \times (0 \, , \, 1-b).
  \]
  %
  Eso significa de  que $x \, \phi''(x)$  es constante sobre $x \in  (0 \; (1-a)
  \max\{b,1-b\})$, y para cualquier par $(a,b) \in (0 \; 1)^2$.  Entonces, $x \,
  \phi''(x)$  es constante  sobre $x  \in (0  \; 1  )$, es  decir que  $\phi$ es
  necesariamente  de  la  forma $\phi(x)  =  \eta  \,  x  \log  x +  \theta  x  +
  \vartheta$. Debido a la continuidad de $\phi$, queda v\'alido sobre el cerrado
  $[0 \, ; \, 1]$.  De que se aplica a un vector de probabilidad, sumando a uno,
  se puede  reducir el  problema a  $\theta = 0$  (poniendo $\theta$  adentro de
  $\vartheta$  sin cambiar  el  valor de  entrop\'ia  obtenida).  Adem\'as,  del
  requisito $\phi(0) = 0$
  % la  constante $\vartheta$  no  altera la  concavidad  (resp. convexidad)  de
  % $\phi$, as\'i que se la puede  translatar en la funci\'on $h$ sin cambiar la
  % monotonicidad, \ie tomar
  tenemos $\vartheta =  0$.  Para que $\phi$ sea  convexa (resp. c\'oncava) hace
  falta  tener  $\eta >  0$  (resp.   $\eta <  0$)  as\'i  que,  sin perdida  de
  generalidad, $\eta$ puede  ser puesta tambi\'en en $h$. Tomar \  $\phi (x) = x
  \, \log x$ \ con \ $h$ \ creciente o \  $\phi (x) = - x \, \log x$ \ con \ $h$
  \ decreciente es completamente equivalente, as\'i que se puede fijar $\phi (x)
  = - x  \, \log x$ satisfaciendo la ecuaci\'on funcional,  y $h$ creciente.  En
  conclusi\'on,  $g  =  0$ sobre  $D  \cap  (0  \,  ;  \, 1)^2$  es  decir,  por
  continuidad,  sobre  $D$  se reduce  a  necesitar  tener  $\hphi =  H$.   Esta
  entrop\'ia    siendo    sub-aditiva    (propidad~\ref{Prop:SZ:subaditividad}),
  cualquiera funci\'on creciente de $H$ va obviamente quedar sub-aditiva, lo que
  cierra la prueba.
  %
\end{proof}
%
Al rev\'es, a partir de $p_{XY}  = \frac12 \begin{bmatrix} 1 & 0 \end{bmatrix}^t
\otimes\begin{bmatrix}  1 &  0  \end{bmatrix}^t +  \frac12  \begin{bmatrix} 0  &
  1  \end{bmatrix}^t \otimes\begin{bmatrix}  0 &  1 \end{bmatrix}^t$  se obtiene
$p_X = p_Y =  \frac12 \begin{bmatrix} 1 & 1 \end{bmatrix}^t$ \  y entonces \ (i)
$\hhphi[p_{XY}]  =  h\left(  -  2  \, \phi\left(  \frac12  \right)  \right)$,  \
$\hhphi[p_X \otimes p_Y] = h\left( -  4 \, \phi\left( \frac14 \right) \right)$ \
y \ $\hhphi[p_X] + \hhphi[p_Y] = 2  \, h\left( - 2 \, \phi\left( \frac12 \right)
\right)$, \  as\'i que, en este  ejemplo \ $\hhphi[p_{XY}]  > \hhphi[p_X \otimes
p_Y]$ \ (consecuencia de la  Schur-concavidad) y \ $\hhphi[p_{XY}] > \hhphi[p_X]
+ \hhphi[p_Y]$: tampoco las $(h,\phi)$-entrop\'ia son super-aditivas.

\

La definici\'on  de entrop\'ias generalizadas condicionales  aparece mucho m\'as
problem\'atico. Por  ejemplo, si se define  a la Shannon, es  decir definiendo \
$\hhphi[X|Y]$ \ tomando \ $\sum_{y \in \Y} p_Y(y) \hhphi[p_{X|Y=y}]$ \ se pierde
la regla de la cadena~\ref{Prop:SZ:cadena}. Como se  lo ha visto, en el marco de
la entrop\'ia de Havdra-Charv\'at-Dar\'oczy se conserva la regla de la cadena si
se reemplaza $p_Y$ por su potencia $p_Y^\lambda$.  Sin embargo, generalizar este
esquema en el caso general  falla (la gracia en Havdra-Charv\'at-Dar\'oczy viene
de  la  propiedad  de  morfismo  de  la  exponencial  y  del  logaritmo).   Como
consecuencia, generalizar  la noci\'on se vuelve  problem\'atico tambi\'en.  Por
ejemple  se  pierde el  diagrama  de  Venn aparte  si  se  define la  entrop\'ia
condicional  a partir  de la  regla de  la  cadena.  Pero  en este  caso, si  la
super-aditividad  garantiza  la positividad  de  la  entrop\'ia condicional,  se
pierde  la propiedad~\ref{Prop:SZ:independenciacondicional}  por  perdida de  la
aditividad,       y      por       consecuencia       la      propiedad       de
positividad/independencia~\ref{Prop:SZ:Ipositive}  de  una  informaci\'on  mutua
construida sobre un  modelo diagrama de Venn. Veremos en  la secci\'on siguiente
que un tercer camino puede ser usar divergencias.
% \SZ{ver con detalles \ref{Prop:SZ:condicionar} (condiconar)} \SZ{VajVas85 pour
%   la Schur-concavite}

\

Como  en  el caso  de  Shannon,  se puede  extender  la  generalizaci\'on de  la
entrop\'ia al caso de vectores  aleatorios discretos sobre de cardenal infinito,
con las mismas debilidades que en el caso de Shannon. A continuaci\'on, se puede
tambi\'en  extenderla   a  vectores   aleatorios  admitiendo  una   densidad  de
probabilidad, reemplazando la suma por una integraci\'on.

\begin{definicion}[$(h,\phi)$-entrop\'ia diferencial]
\label{Def:SZ:HPhiEntropiaDiferencial}
%
  Sea  $X$ una variable  aleatoria continua  sobre $\Rset^d$  y sea  $p_X(x)$ la
  densidad  (distribuci\'on)  de  probabilidad  de  $X$  de  soporte  $\X$.   La
  $(h,\phi)$-entrop\'ia diferencial de la variable $X$ es definida por
  %
  \[
  \hhphi[p_X] =  \hhphi[X] = h\left( -  \int_\X \phi\left( p_X(x)  \right) \, dx
  \right),
  \]
  %
  con  $h$  \   y  \  $\phi$  cumpliendo  los   requisitos  de  la  definici\'on
  discreta~\ref{Def:SZ:HPhiEntropia}  (de   $\phi(0)$,  se  puede   escribir  la
  integraci\'on sobre $\Rset^d$).
\end{definicion}

De nuevo  para $X =  (X_1,\ldots,X_d)$, la $(h,\phi)$-entrop\'ia  diferencial de
$X$ es una $(h,\phi)$-entrop\'ia diferencial conjunta de los $X_i$.

La  versi\'on diferencial  de la  $(h,\phi)$-entrop\'ia comparte  obviamente las
mismas debilidades  del caso particular de  Shannon: se pierden  la propiedad de
invarianza    por   transformaci\'on    biyectiva~\ref{Prop:SZ:biyeccion},   \ie
independencia       con       respecto       a       los       estados,       la
positividad~\ref{Prop:SZ:positividad},            la           de           cota
superior~\ref{Prop:SZ:cotamaxima}  (salvo  si  se  pone  v\'inculos,  ver  m\'as
adelante), en adici\'on de las que ya la versi\'on discreta perdi\'o.

Sin embargo, se conservan unas propidedades,  y entre otros si $h$ es c\'oncava,
la $(h,\phi)$-entrop\'ia  diferencial es c\'oncava~\ref{Prop:SZ:concavidadHPhi}.
M\'as  sorprendentemente a  primer vista,  se conserva  la $(h,\phi)$-entrop\'ia
diferencial bajo un rearreglo~\ref{Prop:SZ:permutacionC},
%
\[
\hhphi[p_X^\downarrow] = \hhphi[p_X].
\]
%
\noindent De  hecho, como evocado en el  caso de Shannon, eso  fue probado entre
otros  en~\cite{LieLos01} o  \cite[Lema~7.2]{WanMad04}~\footnote{Recuerdense que
  en~\cite[Sec.~3.3]{LieLos01}  lo  muestran   para  $\phi$  diferencia  de  dos
  funciones mon\'otonas, siendo una funci\'on convexa un caso particular.}.

Se  prob\'o  en~~\cite{Cho74} o~\cite[Prop.~7.3]{WanMad04}  que  se conserva  la
Schur-concavidad~\ref{Prop:SZ:Schurconcavidad}              para             las
$\phi$-entrop\'ias.   Entonces,  de   $h$  creciente   (para   $\phi$  c\'oncava
desigualdad reversa para la integral,  pero $h$ es decreciente), se generaliza a
las $(h,\phi)$-entrop\'ias,\ie
%
\[
p  \prec q  \quad \Rightarrow  \quad \hhphi[p]  \ge \hhphi[q]  \quad  \forall \:
(h,\phi).
\]

\SZ{Quide de la rec\'iproca? Quid sub-aditividad ssi fct creciente de Shannon?}

Terminamos esta subsecci\'on notando que, como para la entrop\'ia de Shannon, el
enfoque  discreto  y  diferencial  son  contenido en  la  forma  general  usando
densidades con respecto  a una medida (respectivamente discreta  y de Lebesgue en
estos casos).
%
\begin{definicion}[Escritura \'unica de las $(h,\phi)$-entrop\'ias]
\label{Def:SZ:HPhiEntropiaMu}
%
Sea \ $X$ \ variable aleatoria definida sobre $\X \subset \Rset^d$, admitiendo
una densidad de probabilidad  \ $p_X$ \ con respecto a una  medida \ $\mu$ \ (ej.
$\mu_\X$ \ en  el caso discreto \ $\mu  = \mu_L$ \ en el  caso diferencial).  La
$(h,\phi)$-entrop\'ia de $X$ con respecto a $\mu$ se escribe como
  %
  \[
  \hhphi[X] \equiv \hhphi[p_X] = h\left(  - \int_\X \phi\left( p_X(x) \right) \,
    d\mu(x) \right),
  \]
  %
  con  $h$  \   y  \  $\phi$  cumpliendo  los   requisitos  de  la  definici\'on
  discreta~\ref{Def:SZ:HPhiEntropia}  (de   $\phi(0)$,  se  puede   escribir  la
  integraci\'on sobre $\Rset^d$).\newline Insistamos de nuevo en el hecho que se
  puede entender esta definici\'on para cualquier $\mu$ y densidad con respecto a
  $\mu$, que sea discreta, de Lebesgue, o mixta.
\end{definicion}

% ================================= Csiszar & Bregman

\subseccion{Divergencias y propiedades}
\label{Ssec:SZ:JensenBregmanCzizar}

En esta sub-secci\'on  vamos a ver que la  literatura trat\'o casi conjuntamente
de  tres  enfoques   dando  lugar  a  generalizaciones  de   la  divergencia  de
Kullback-Leibler. Lamentablemente, ninguna  generalizaci\'on contiene las otras,
a  pesar  de   que  divergencias  conocidas  pueden  partenecer   a  varias  clases
distinctas. Practicamente,  cada clase tiene  sus ventajas y  justificaci\'on en
termino de aplicaciones.


% ----- Clase de Jensen

\subsubseccion{Clase de Jensen}
\label{Sssec:SZ:Jensen}

% ----- Jensen-Shannon
%\paragraph{Primer  generalizaciones, saliendo  de Shannon  y  Kullback-Leibler -
%  divergencia de Jensen-Shannon}

Como  se lo  ha visto  tratando  de la  entrop\'ia relativa,  la divergencia  de
Kullback-Leibler no define una distancia entre distribuciones de probabilidades,
siendo no sim\'etrica  entre otros. Un primer paso  para recuperar la simetr\'ia
sin  perder  la  positividad  de  esta medida  informacional  fue  simetrizarla,
definiendo lo que es  conocido como {\it $J$-divergencia}~\cite{KulLei51, Kul68,
  Lin91}~\footnote{Esta   expresi\'on   apareci\'o  en~\cite[Ec.~(1)]{Jef46}   o
  en~\cite{Jef48},   antes   de  la   introducc\'ion   de   la  divergencia   de
  Kullback-Leibler  en el  campo de  la estimaci\'on  bayesiana,  Jeffrey siendo
  citado por Kullback y Leibler.},
%
\[
D_J(Q\|P) = \Dkl[P]{Q} + \Dkl[Q]{P}.
\]
%
\noindent  Esta  versi\'on  simetrizada  de la  divergencia  queda  naturalmente
positiva,  pero sufre  todav\'ia  de  unas debilidades  de  $\Dkl{}$. Esta  bien
definida siempre que  \ $P \ll Q$ \  conjuntamente a \ $Q \ll P$  \ (las medidas
son  dichas {\it  medidas  equivalentes}  en este  caso).   Adem\'as, no  cumple
tampoco  la  desigualdad triangular.   A  pesar  de  sus debilidades,  se  us\'o
bastante en problemas de discriminaci\'on,  debido a su positividad con igualdad
si y solamente si $P = Q$  (propiedad herida del hecho que la suma de t\'erminos
positivos es nula si y solamente si cada uno vale cero).

Unas decadas despu\'es, Lin  introdujo lo que llam\'o $K$-divergencia directada,
$K(P,Q)   =  \Dkl[P]{\frac{P+Q}{2}}$,   su  versi\'on   simetrizada,   antes  de
generalizarla    bajo     la    terminologia    de     {\it    divergencia    de
  Jensen}~\cite{Lin91}~\footnote{De  hecho, apareci\'o implicitamente  en varios
  trabajos anteriores, por  ejemplo en mec\'anica cu\'antica~\cite{Hol73, Hol11}
  o en reconocimiento de patrones~\cite{WonYou85}}.
%
\begin{eqnarray*}
\Djs(P_{(1)},P_{(2)})  &  = &
%
\pi_1 \Dkl[P_{(1)}]{\pi_1 P_{(1)} + \pi_2 P_{(2)}} + \pi_2 \Dkl[P_{(2)}]{\pi_1
P_{(1)} + \pi_2 P_{(2)}}\\[2.5mm]
%
& = & H(\pi_1 p_{(1)} + \pi_2 p_{(2)}) - \pi_1 H(p_{(1)}) - \pi_2 H(p_{(2)})
\qquad \pi = [\pi_1 \quad \pi_2], \quad 0 \le \pi_1 = 1-\pi_2 \le 1
\end{eqnarray*}
%
\noindent con $p_{(i)}$  densidades de $P_{(i)}$ con respecto  a una misma medida
$\mu$ (puede ser discreta, de Lebesgue, o mixta).

$\Djs$ heride obviamente de $\Dkl{}$  su positividad con igualdad si y solamente
si $P_{(1)}  = P_{(2)}$.  La  misma propiedad puede  ser vista a trav\'es  de la
desigualdad de  Jensen, dando este  nombre a la  medida.  Adem\'as, se  quita el
problema  de definici\'on,  siendo de  que $P_{(i)}  \ll \pi_1  P_{(1)}  + \pi_2
P_{(2)}$.  No es  sim\'etrica en general, pero se  obtiene esta propiedad cuando
$\pi =  \pi_{\mathrm{u}} \equiv [\frac12  \quad \frac12]^t$.  Adem\'as,  en este
caso, a pesar de que la divergencia no cumpla la desigualdad triangular, aparece
que $\Big(  J_{\mathrm{js}}^{\pi_{\mathrm{u}}}(P_{(1)},P_{(2)}) \Big)^s, \quad 0
<  s  \le  \frac12$ es  una  metrica~\footnote{Se  necesita  sol\'o de  que  los
  $P_{(i)}$ admiten una  densidad con respecto a una  medida $\sigma$-finita; nos
  referiremos           al           resultado~\ref{Teo:MP:DescompositionMixta}.
%,  pagina~\pageref{Teo:MP:DescompositionMixta}.
}~\cite[Teorema~1  \& Nota~2]{OstVaj03}  o~\cite{EndSch03,  KafOst91, OsaBus18}.
Si   puede   parecer   m\'as    l\'ogico   definir   tal   divergencia   con   a
priori/proporciones $\pi_i$  iguales, de hecho la versi\'on  no sim\'etrica, con
pesos  $\pi_i$  se vuelve  natural  en el  marco  de  la discriminaci\'on  donde
apareci\'o implicitamente esta cantidad.  En particular, cuando estamos frente a
dos  hypotesis $i  = 1,  2$ o  clases,  a las  cuales la  distribuci\'on de  las
observaciones  es $P_{(i)}$,  con probabilidad  a priori  $\pi_i$.  A  partir de
observaciones  $x$ hay que  elegir si  eran sorteando  de $P_{(1)}$  o $P_{(2)}$
(distribuciones de  sampleos, \ie condicionalmente a la  hypotesis).  El enfoque
bayesiano  m\'as  natural  consiste   maximizar  la  probabilidad  a  posteriori
(probabilidad de estar en hypotesis  $i$ condicionalmente a la observaci\'on), y
se  prueba que  la probabilidad  de error  es dada  por \  $\displaystyle  P_e =
\int_\X \min( \pi_1  p_{(1)}(x) \, , \,  \pi_2 p_{(2)}(x) ) \, d\mu(x)$  \ con \
$p_{(i)}$ densidad con respecto a $\mu$~\cite{Kay93}. Prob\'o Lin de que
%
\[
\frac14 \left(  H(\pi) - \Djs(P_{(1)},P_{(2)})  \right)^2 \le P_e  \le \frac12
\left( H(\pi) - \Djs(P_{(1)},P_{(2)}) \right),
\]
%
%con el logaritmo de base 2 en  la definici\'on de $\Djs$, 
lo que da naturalmente un  rol operacional a esta divergencia.  Incidentalmente,
de esta desigualdad es inmediato  ver de que $\Djs(P_{(1)},P_{(2)}) \le H(\pi) -
2 P_e$.  $P_e$ siendo positivo, da
%
\[
0 \le \Djs(P_{(1)},P_{(2)}) \le  H(\pi) \le \log(2)
\]
%
\noindent  Usando  el  logaritmo  de  base  2,  adaptado  a  este  caso  de  dos
distribuciones, la cota vale 1: $\Djs$ es dicha {\it normalizada}.
 

Un otro v\'inculo  natural entre la divergencia de  Jensen-Shannon y las medidas
informacionales a la Shannon viene todav\'ia del campo de la clasificaci\'on. Si
unos datos pueden provenir de una  distribuci\'on $P_{(i)}$, $i = 1, 2$, con una
probabilidad  $\pi_i$, la variable  aleatoria $X$  dada por  los datos  tiene la
distribuci\'on  de mezcla  $P  =  \sum_i \pi_i  P_{(i)}$  como ilustrado  figura
Fig.~\ref{Fig:SZ:Concavidad}-(b).
%,  pagina~\pageref{Fig:SZ:Concavidad}.   
Sea $Z$ la variable  aleatoria binaria sobre $\{ 1 \, , \, 2  \}$ tal que $P(Z =
i) = \pi_i$, variable de selecci\'on entre las distribuciones $P_{(i)}$ (ej.  la
moneda de la figura).  Por  definici\'on de la entrop\'ia condicional, $H(X|Z) =
\sum_i \pi_i H(X|Z = i)  = \sum_i \pi_i H(p_{(i)})$. De $\Djs(P_{(1)},P_{(2)}) =
H(p) - \sum_i  \pi_i H(p_{(i)})$ viene $\Djs(P_{(1)},P_{(2)}) =  H(X) - H(X|Z)$,
es decir
%
\[
\Djs(P_{(1)},P_{(2)}) = I(X;Z).
\]
%
La  divergencia   de  Jensen-Shannon  mide  la  informaci\'on   mutua  entre  la
observaci\'on $X$  y la variable de  selecci\'on $Z$, justificando  aun m\'as su
uso    natural   en    problemas   de    clasificaci\'on   o    selecci\'on   de
modelos. Incidentalmente, de $I(X;Z) = H(Z)  - H(Z|X) \le H(Z) \le \log(2)$ ($Z$
siendo discreta) se recupera las cotas mayor de $\Djs$.

Se  encuentran otras  desigualdades  implicando $\Djs$  y  $D_J$ o  $\Djs$ y  la
distancia  $L^1$  entre  distribuciones   o  divergencia  de  variaci\'on  total
en~\cite{Lin91}.
%~\cite{SchEla03}.

M\'as all\'a,  en el campo de la  clasificaci\'on, se puedre tratar  de m\'as de
dos  clases,   dando  lugar   a  la  generalizaci\'on   de  la   divergencia  de
Jensen-Shannon a  $n$ distribuciones de probabilidad y  $\pi$ un $n$-componentes
vector de probabilidad,
%
\begin{eqnarray*}
\Djs(P_{(1)},\ldots,P_{(n)})
 & = &  H\left( \sum_i  \pi_i  p_{(i)}\right) -  \sum_i \pi_i H(p_{(i)})\\[2.5mm]
%
& = & \sum_i \pi_i \Dkl[P_{(i)}]{\sum_j \pi_j P_{(j)}}.
\end{eqnarray*}
%
De la  desigualdad de  Jensen, esta  cantitad queda positiva  con igualdad  si y
solamente si todos los $P_{(i)}$ son iguales. Se conserva una cota superior
%
\[
\Djs(P_{(1)},\ldots,P_{(n)}) \le H(\pi) \le \log(n),
\]
%
\noindent as\'i  que $\Djs(P_{(1)},P_{(2)}) = I(X;Z)$ con  $X$ de distribuci\'on
la mezcla $\sum_i \pi_i  P_{(i)}$ y $Z$ definida sobre $\{1 \,  , \, \ldots \, ,
\,n \}$ variable de selecci\'on de distribuci\'on $\pi$.

\SZ{convexidad?}

% ----- f-Jensen
%\paragraph{Clase de Burbea-Rao o   divergencias de Jensen}

\

Un  punto  clave  que  dio  lugar   a  la  definici\'on  de  la  divergencia  de
Jensen-Shannon es la  concavidad de la entrop\'ia de  Shannon.  Naturalmente, el
mismo enfoque  se generaliza  a cualquier entrop\'ia  c\'oncava de un  vector de
probabilidad. Tal generalizaci\'on fue propuesta de maner formal por J. Burbea y
C.   R.   Rao  en~\cite{BurRao82},   y  luego  generalizado  y  estudiado  m\'as
detenidamente  por Nielsen  et  al.~\cite{NieBol11, NieNoc17}.  A  pesar de  que
apareci\'o ya en el papel de Burbea \& Rao, Nielsen llam\'o tal generalizaci\'on
``divergencia  de  Burbea-Rao  asimetrizada''.   M\'as  formalemente,  se  puede
definir una divergencia de Jensen de la manera siguiente:
%
\begin{definicion}[Divergencias  de Jensen]
\label{Def:SZ:DivJensen}
%
  Sea \ $f: \U \subset \Rset^m \mapsto \Rset$ \ convexa y de clase $C^1$ \ sobre
  \ $\U$, \ un cerrado convexo de $\Rset^m$ \ y \ $\pi = \begin{bmatrix} \pi_1 &
    \pi_2  \end{bmatrix}^t$  \  con  $0   \le  \pi_1  =  1-\pi_2  \le  1$.   Las
  divergencias  de Jensen  entre dos  puntos \  $u _1  , u_2  \in \U$  \ son
  definidas por
  %
  \[
  J_f^\pi(u_1,u_2) = \pi_1 f(u_1) + \pi_2 f(u_2) - f(\pi_1 u_1 + \pi_2 u_2).
  \]
  %
  Se ilustra  a que corresponde  esta cantidad con  respecto a $f$ en  la figura
  Fig.~\ref{Fig:SZ:BregmanFJensen} m\'as adelante.
\end{definicion}
%
\noindent Esta  definici\'on se generalizada  a densidad de  probabilidad, donde
$f$  es  a   valor  reales,  actuando  sobre  el  convexo   de  las  medidas  de
probabilidades   (o  tomando   densidades   en  un   $x$   e  integrando   sobre
$\X$)~\cite{NieBol11, NieNoc17}.

\begin{definicion}[Divergencia $(h,\phi)$-Jensen entr\'opica]
\label{Def:SZ:DivJensen}
%
Para  $(h,\phi)$-entrop\'ias   c\'oncavas  (ej.   con   $h$  c\'oncava),  siendo
$-\hhphi$ convexa, se puede entonces asociar una divergencia de Jensen
  %
  \[
  \jhphi[p_{(1)}]{p_{(2)}}     \equiv     J_{-\hhphi}^\pi(p_{(1)},p_{(2)})     =
  \hhphi[\pi_1  p_{(1)}  +  \pi_2  p_{(2)}]  -  \pi_1  \hhphi[p_{(1)}]  -  \pi_2
  \hhphi[p_{(2)}].
  \]
  %
  \noindent con \ $p_{(i)}$ \ densidad con respecto a una medida $\mu$. Cuando $h
  \equiv \id$ identidad, se notar\'a $\jphi{}$.

  La definici\'on se generaliza a  cualquier conjunto $\{ p_{(i)} \}_{i=1}^n$ de
  distribuciones   de   probabilidades    y   $\pi$   vector   de   probabilidad
  $n$-dimensional,
  %
  \[
  D_{(h,\phi)}^{\mathrm{j},\pi}( \{ p_{(i)} \}  ) = \hhphi[\sum_i \pi_i p_{(i)}]
  - \sum_i \pi_i \hhphi[p_{(i)}].
  \]
\end{definicion}
%
\noindent Por analog\'ia a la  informaci\'on mutua, Burbea y Rao llamar\'on esta
medida ``informaci\'on mutua  generalizada''. Eso viene de que  si se define una
informaci\'on condicional  en el mismo esquema  que el de Shannon,  \ie para $Y$
discreta, \ $\hhphi[X|Y] =  \sum_y p_Y(y) \hhphi[p_{X|Y=y}]$, entonces, con $\pi
\equiv p_Y$  \ y \  $\{ p_{(i)} \}_i  \equiv \{ p_{X|Y=y}  \}_y $ aparece  que \
$D_{(h,\phi)}^{\mathrm{j},p_Y}( \{ p_{X|Y=y} \}_y  ) = \hhphi[X] - \hhphi[X|Y]$.
\ Esta expresi\'on es parecida a una  de las formas de la informaci\'on mutua de
Shannon,  justificando la  terminolog\'ia de  Burbea-Rao. Sin  embargo, hay  que
tener conciencia  de que no  todo se translata  obviamente del mundo  Shannon al
mundo  generalizado.   Por  ejemplo,  con   tal  definic\'on  de  la  entrop\'ia
condicional, se pierde  la regla de la cadena, y  por consecuencia la simetr\'ia
de tal informaci\'on mutua generalizada o la forma usando la entrop\'ia conjunta
y las marginales.

\SZ{A ver  si se generaliza  bajo la forma  $ \hhphi[\int \pi(y) p(x;y)]  - \int
  \pi(y) \hhphi[p(x,y)]$.}

Se  notar\'a de  que Nielsen  propus\'o generalizaciones  mas  avanzadas, usando
generalizaciones de la noci\'on  de convexidad. Estas generalizaciones van m\'as
all\'a de la meta del cap\'itulo y el lector  puede referirse a~\cite{NieNoc17}.

Las divergencias de Jensen tiene las propiedades siguientes
%
\begin{enumerate}
\item Positividad:
  %
  \[
  J_f^\pi(P,Q) \ge  0 \quad \mbox{con  igualdad si  y solamente si}  \quad P =  Q.
  \]
  %
  Esta propiedad  es la consecuencia directa  de la convexidad  estricta de $f$,
  como ilustrado figura Fig.~\ref{Fig:SZ:BregmanFJensen}.
%
  % \item  \SZ{$J_f(p,q)$  es convexa  con  respecto  al  par $(p,q)$,  pero  no
  %     necesariamente  con  respecto a  $p$  solamente  y/o  $q$. Es  tambi\'en
  %     consecuencia directa de la convexidad de $f$.}
%
\item Pensando a  $J_f^\pi$ con respecto a  $f$, es lineal en el  sentido de que
  $J_{a_1 f_1  + a_2 f_2}^\pi  = a_1 J_{f_1}^\pi  + a_2 J_{f_2}^\pi$  (con $f_i$
  convexas y $a_i \ge 0$).
%
  % Dualidad:  si $\phi$  tiene un  convex conjugado  $\phi^*$  (transformada de
  % Legendre)  D^{\mathrm{b}}_{\phi^*}\left(  \left.  -\nabla \hphi[q]  \right\|
  %   -\nabla \hphi[p] \right) = \bphi[q]{p}$~\cite{NieNoc17}.
%
  % Mean as minimizer: A key result about Bregman divergences is that, given a
  % random vector, the mean vector minimizes the expected Bregman divergence
  % from the random vector~\cite{FriSri08}.  This result is important because it
  % further justifies using a
  % mean as a representative of a random set, particularly in Bayesian
  % estimation.
\end{enumerate}
%
\noindent Desgraciamente,  las divergencias de Jensen no  cumplen la desigualdad
triangular  en general,  y entonces  no son  m\'etricas entre  distribuciones de
probabilidad.  Se  refier\'a a~\cite{BurRao82,  NieBol11,  NieNoc17} para  tener
m\'as propiedades.

Se notar\'a que  la clase de las divergencias de Jensen  contiene el cuadrado de
la distancia de Mahalanobis (por un factor), \ie  con \ $f(u) = u^t K u$ \ con \
$K >  0$ \  sim\'etrica se  obtiene $J_f(u,v) =  \pi_1 \pi_2  (v-u)^t K  (v-u) $
(siendo la distancia $L^2$ un caso particular). Se generaliza al caso continuo y
distancias $L^2$ con un nucleo.
%donde $Q$ es
%una  funci\'on $\Rset^d  \times \Rset^d  \mapsto \Rset_+$  sim\'etrica definida
%positiva llamado nuclo.
%  la distancia  $L^1$  ($\displaystyle f(p)  =  \left( \int  p \right)^2$),  la
%  distancia  de Itakura-Saito  cuando  $\phi(u)  = -  \log  u$  (asociado a  la
% entrop\'ia de Burg), entre otros.


% ----- Clase de Bregman

\subsubseccion{Clase de Bregman}
\label{Sssec:SZ:Bregman}

\SZ{Ver Nielsen and Nock 2010 sobre la divergence entre dos leyes de la famlilia exponencial~\cite{NieNoc10}}

Estas divergencias  fueron intoducidos en  el campo de la  programaci\'on lineal
convexa, para  resolver problemas de  minimizaci\'on convexa~\footnote{A\'un que
  aparece en una revista de matem\'atica y f\'isica matem\'atica, una gracia del
  papel de Bregman es que toma  el ejemplo de maximizaci\'on de la entrop\'ia de
  Shannon sujeto a momentos\ldots}~\cite{Bre67}, pero con aplicaciones en varios
campos~\cite[y ref.]{Bas89, Bas13}:
%
\begin{definicion}[Divergencias de Bregman]
\label{Def:SZ:DivBregman}
%
  Sea \ $f: \U  \subset \Rset^m \mapsto \Rset$ \ convexa y  de clase $C^1$ \
  sobre  \ $\U$, \  un cerrado  convexo de  $\Rset^m$.  Las  divergencias de
  Bregman de  un punto  \ $v \in  \U$ \  relativamente a un  punto \  $u \in
  \U$ \ son definidas por
  %
  \[
  B_f(v\|u) = f(v) - f(u) - (v-u)^t \nabla f(u).
  \]
  %
  Dicho de otra  manera, $B_f$ corresponde al desarollo de Taylor  al orden 1 de
  $f$ en  la referencia  $u$.  Se  ilustra a que  corresponde esta  cantidad con
  respecto a $f$ en la figura Fig.~\ref{Fig:SZ:BregmanFJensen} m\'as adelante.
\end{definicion}
%
Esta  definici\'on fue generalizada  a funciones  actuando sobre  espacios m\'as
generales (ej.  actuando  sobre matrices o operadores en  espacios de Hilbert de
dimensi\'on infinita)~\cite{Pet07}.   En lo que nos concierna  en este capitulo,
tratando posiblemente de densidad de probabilidades, nos interesamos a funciones
de funciones~\cite{FriSri08, NieNoc17}:
%
\begin{definicion}[Divergencias de Bregman funcional]
\label{Def:SZ:DivBregmanFun}
%
  Sea \ $f: \U \mapsto \Rset$ \ convexa y de clase $C^1$ \ sobre \ $\U$,
  \ un cerrado convexo de un  espacio de Banach.  Las divergencias de Bregman de
  un ``punto'' (una funci\'on) \ $v \in \U$ \ relativamente a un ``punto'' \
  $u \in \U$ \ son definidas por
  %
  \[
  B_f(v\|u) = f(v) - f(u) - \lim_{t \to 0}\frac{f( u + t (v-u) ) - f(u)}{t}.
  \]
  %
  El \'ultimo t\'ermino  de esta formula es connocida  como derivada de G\^ateau
  (o derivada direccional) de $f$ en $u$ en la direcci\'on $v-u$ (siendo $u$ una
  funci\'on)~\footnote{De   hecho,   en    la   extensi\'on   de   Frigyik   et
    al.~\cite{FriSri08}, se usa  la derivada de F\'echet, que  es m\'as general.
    Viene  de  un  l\'imite   identica  independientemente  de  la  direcci\'on.
    Entonces,  si   una  funci\'on  tiene  una  derivada   de  Fr\'echet,  tiene
    necesariamente derivadas de G\^ateau,  pero no es rec\'iproca.  Esta subtileza
    va m\'as all\'a de la meta de esta secci\'on.}.
\end{definicion}
%
En  el caso  $\U  \subset  \Rset^m$ se  recupera  sencillamente la  definici\'on
original. Ahora,  se puede aplicar  la definici\'on~\ref{Def:SZ:DivBregmanFun} a
la  funci\'on $-\hhphi[\cdot]$  cuanda  esta  es convexa,  para  asociar a  esta
entrop\'ia una divergencia tipo Bregman:

\begin{definicion}[Divergencia $(h,\phi)$-Bregman entr\'opica]
\label{Def:SZ:DivBregmanEntropicas}
%
  Para  $(h,\phi)$-entrop\'ias c\'oncavas  (ej.   con $h$  c\'oncava), se  puede
  asociar una divergencia de Bregman
  %
  \[
  \bhphi[q]{p} = \hhphi[p] - \hhphi[q]  - h'\big(\hphi[p] \big) \int_\X ( q(x) -
  p(x) ) \phi'(p(x)) \, d\mu(x).
  \]
  %
  Cuando  $h \equiv \id$  identidad, se  notar\'a $\bphi{}$  y es  equivalente a
  salir  de la  definici\'on  inicial  $u =  p(x)$,  \ $v  =  q(x)$  y sumar  la
  divergencia obtenida sobre  \ $\X$ \ con  respecto a la medida \  $\mu$.  En el
  caso particular discreto toma la expresi\'on
  %
  \begin{eqnarray*}
    \bhphi[q]{p} \equiv B_{-\hhphi}(q\|p)
    & = & \hhphi[p] - \hhphi[q] - (p-q)^t \nabla \hhphi[p]\\[2.5mm]
    %
    & = & \hhphi[p] - \hhphi[q] - h'\big(\hphi[p] \big) (q-p)^t \nabla \phi(p)
  \end{eqnarray*}
  %
  % \noindent Cuando  $h \equiv \id$, se  notar\'a $\bphi{}$ y  es equivalente a
  % salir de la definici\'on inicial con \ $\U =  [0 \, ; \, 1]$, $u$ \ y \ $v =
  % q(y_i)$ \ $i$-esima componente de \ $p$ \ y \ $q$ \ respectivamente, y sumar
  % la divergencia obtenida sobre $i$.
\end{definicion}

Aparece de que  las divergencias de Jensen se escriben  a partir de divergencias
de Bregman, y vice-versa:
%
\begin{lema}
\label{Lem:SZ:VinculoJensenBregman}
%
  De   las    definiciones~\ref{Def:SZ:DivJensen},   \ref{Def:SZ:DivBregman}   y
  \ref{Def:SZ:DivBregmanFun}, se  muestra sencillamente de  que las divergencias
  de Jensen se escriben como combinaciones convexas de divergencias de Bregman,
  %
  \[
  J_f^\pi(P_{(1)},P_{(2)}) = \pi_1 B_f(P_{(1)} \| \pi_1 P_{(1)} + \pi_2 P_{(2)})
  + \pi_2 B_f(P_{(1)} \| \pi_1 P_{(1)} + \pi_2 P_{(2)}),
  \]
  %
  Al  rev\'es,  las  divergencias  de   Bregman  se  escriben  como  limites  de
  divergencias de Jensen,
  %
  \[
  B_f(P_{(2)}  \|   P_{(1)})  =  \lim_{\pi_2  \to   0}  \frac{J_f^\pi(P_{(1)}  ,
    P_{(2)})}{\pi_1 \pi_2}
  \]
  %
  (o  similarmente  con densidad  $p_{(i)}$  con  respecto  a una  medida  $\mu$;
  recuerdense tambi\'en que $\pi_1 = 1-\pi_2$).
\end{lema}
%
\noindent (ver~\cite{Zha04, NieBol11, NieNoc17}).

La figura Fig.~\ref{Fig:SZ:BregmanFJensen} ilustra  a que corresponden \ $D_f$ \
y \ $J_f$ con respecto a la funci\'on convexa $f$.
%
\begin{figure}[h!]
%
\begin{center} \begin{tikzpicture}[xscale=8,yscale=6.5]
\shorthandoff{>}
%
\pgfmathsetmacro{\u}{.22};
\pgfmathsetmacro{\v}{1};
\pgfmathsetmacro{\al}{.65};
\pgfmathsetmacro{\mid}{\al*\u+(1-\al)*\v};
%
% Axes et f convexe (t log t ici)
\draw[>=stealth,->] (-.1,-.5)--(1.6,-.5) node[right]{\small $t$};
\draw[>=stealth,->] (0,-.7)--(0,.3) node[above]{\small $f(t)$};
\draw[thick,domain=.005:1.2,samples=199] (0,0)-- plot (\x,{\x*ln(\x)});
%
%\draw[dotted] (\u,-.5)--(\u,{\u*ln(\u)});
\draw (\u,-.5)--(\u,-.52) node[below]{\small $u_1$};
\draw (\v,-.5) node[below right]{\small $u_2$};
%
% ---
% tangente en u_1 y B_f(u_2 || u_1)
\draw (.02,{(1+ln(\u))*(.02-\u)+\u*ln(\u)})--
(1.05,{(1+ln(\u))*(1.05-\u)+\u*ln(\u)})
node[right,scale=.7]{$\quad f'(u_1) \, (t-u_1) + f(u_1)$};
%
\draw[>=stealth,<->] (\v,{(1+ln(\u))*(\v-\u)+\u*ln(\u)}) -- (\v,{\v*ln(\v)});
\draw (\v,{.5*((1+ln(\u))*(\v-\u)+\u*ln(\u)+\v*ln(\v))}) 
node[right,scale=.9]{$B_f(u_2\|u_1)$};
%
% ---
% tangente en pi_1 iu_1 + pi_2 u_2
% y B_f(u_1 || pi_1 u_1 + pi_2 u_2)
%\draw[dashed] (.02,{(1+ln(\mid))*(.02-\mid)+\mid*ln(\mid)})--
%(1.1,{(1+ln(\mid))*(1.1-\mid)+\mid*ln(\mid)})
%node[right,scale=.7]{$\quad \displaystyle f'\big(\sum_i \pi_i u_i \big)
%\, \left( t-\sum_i \pi_i u_i \right) + f\left( \sum_i \pi_i u_i \right)$};
%%
%\draw[>=stealth,dashed,<->] (\v+.01,{(1+ln(\mid))*(\v-\mid)+\mid*ln(\mid)})
% -- (\v+.01,{\v*ln(\v)});
%\draw (\v,{.5*((1+ln(\mid))*(\v-\mid)+\mid*ln(\mid)+\v*ln(\v))}) 
%node[right,scale=.7]{$B_f\big( u_1\|\sum_i \pi_i u_i \big)$};
%
% ---
% corde u_1 - u_2 et f-Jensen
\draw[dashed] (\u,{\u*ln(\u)})--(\v,{\v*ln(\v)});
%
\draw[>=stealth,<->] (\mid,{\mid*ln(\mid)}) --
(\mid,{\al*\u*ln(\u)+(1-\al)*\v*ln(\v)});
\draw (\mid,{.75*\mid*ln(\mid)+.25*(\al*\u*ln(\u)+(1-\al)*\v*ln(\v))})
node[left,scale=.85]{$J_f^\pi(u_1,u_2)$};
\draw (\mid,-.5) -- (\mid,-.52) node[below]{\small $\pi_1 u_1 + \pi_2 u_2$};
%node[right,scale=.9]{$B_f(u_2\|u_1)$};
%
\end{tikzpicture} \end{center}
%
\leyenda{$f$ estrictamente  convexa.  Las cantidades positivas  marcadas por las
  dupla-flechas  representan respectivamente  (a)  la divergencia  de Bregman  \
  $B_f(u_2\|u_1)$ diferencia entre  el valor en $u_2$ (punto  de evaluaci\'on) y
  la  tangente en  $u_1$ (punto  referencia), (b)  la divergencia  de $f$-Jensen
  $J_f^\pi(u_1,u_2)$, diferencia entre la  combinaci\'on convexa de los $f(u_i)$
  y $f$  de la combinaci\'on convexa  $u_\pi = \pi_1 u_1  + \pi_2 u_2$,  \ y las
  divergencias  de  Bregman \  $B_f(u_i\|u_\pi)$;  para  $J_f^\pi$,  se toma  la
  combinaci\'on convexa de los $B_f(u_i\|u_\pi)$.}
  %
\label{Fig:SZ:BregmanFJensen}
\end{figure}
\SZ{Ver si se puede ilustrar el l\'imite dando la Bregman}

La divergencia de Bregman tiene las propiedades siguientes
%
\begin{enumerate}
\item Positividad:
  %
  \[
  B_f(Q\|P) \ge 0 \quad \mbox{con igualdad si y solamente si} \quad P = Q.
  \]
  %
  Esta propiedad  es la consecuencia directa  de la convexidad  estricta de $f$,
  como ilustrado figura Fig.~\ref{Fig:SZ:BregmanFJensen}.
%
\item  $B_f(Q\|P)$ es convexa  con respecto  a $Q$,  pero no  necesariamente con
  respecto a $P$. Es tambi\'en consecuencia directa de la convexidad de $f$.
%
\item  Pensando a  $B_f$ con  respecto a  $f$, es  lineal en  el sentido  de que
  $B_{a_1 f_1  + a_2 f_2}  = a_1 B_{f_1} +  a_2 B_{f_2}$
  (con $f_i$ convexas y $a_i \ge 0$).
%
  % Dualidad:  si $\phi$  tiene un  convex conjugado  $\phi^*$  (transformada de
  % Legendre)  D^{\mathrm{b}}_{\phi^*}\left(  \left.  -\nabla \hphi[q]  \right\|
  %   -\nabla \hphi[p] \right) = \bphi[q]{p}$~\cite{NieNoc17}.
%
  % Mean as minimizer: A key result about Bregman divergences is that, given a
  % random vector, the mean vector minimizes the expected Bregman divergence
  % from the random vector~\cite{FriSri08}.  This result is important because it
  % further justifies using a
  % mean as a representative of a random set, particularly in Bayesian
  % estimation.
\end{enumerate}
%
\noindent Ver~\cite{FriSri08, NieBol11, NieNoc17} para tener m\'as propiedades.

Se notar\'a que la clase de  las divergencias de Bregman contiene el cuadrado de
la distancia de Mahalanobis con \ $f(u) = u^t K u$ \ con \ $K > 0$ \ sim\'etrica
(siendo  la distancia $L^2$  un caso  particular), el  cuadrado de  la distancia
$L^1$ con  $\displaystyle f(u) = \left(  \sum_i u_i \right)^2$,  la distancia de
Itakura-Saito cuando $f(u) = - \log u$ (asociado a la entrop\'ia de Burg), entre
otros.  Unas se extienden sencillamente al caso continuo~\cite{FriSri08}.

Como en  el caso de  divergencias de Jensen, Nielsen  propus\'o generalizaciones
m\'as avanzadas, usando las generalizaciones de la noci\'on de convexidad usadas
para generalizar  las divergencias  de Jensen. Estas  generalizaciones tambi\'en
van  m\'as  all\'a  de la  meta  del  cap\'itulo  y  el lector  puede  referirse
a~\cite{NieNoc17}.

Tambi\'en,   varias  aplicaciones   se  encuentran   en   la  literatura~\cite[y
ref.]{Bas89, Csi95,  CsiMat12, Bas13}  en adici\'on de  las referencias  de esta
secci\'on, para dar unas.


% ----- Clase de Csiszar o Ali-Silvey

\subsubseccion{Clase de Csisz\'ar o Ali-Silvey}
\label{Sssec:SZ:Csiszar}

Un primer paso  generalizando la noci\'on de entrop\'ia  relativa o divergencia,
siguiendo el enfoque de Kullback  y Leibler y sus versiones tipo $J$-divergencia
o divergencia de Jensen-Shannon fue  debido a R\'enyi. En su papel~\cite{Ren61},
A. R\'enyi introdujo una noci\'on de  ganancia o perdida de informaci\'on de una
distribuc\'ion (incompleta) de probabilidad $q$ relativa a una referencia $p$, \
$I^\ren(q\|p)$, teniendo un enfoque axiom\'atico similar al que uso para definir
su entrop\'ia:  (i) la medida  sea invariante a  una misma permutaci\'on  de los
componentes de  $p$ y de $q$,  (ii) si $\forall \,  i, \: p_i  \le q_i$ entonces
$I^\ren(q\|p) \ge 0$ y al rev\'es $I(p\|q)  \le 0$, (iii) $I([1] \| [1/2]) = 1$,
(iv) $I(  q_{(1)} \otimes q_{(2)}  \| p_{(1)} \otimes  p_{(2)}) = I(  q_{(1)} \|
p_{(1)}) +  I( q_{(2)} \|  p_{(2)})$ y (v)  una propiedad de  media generalizada
$I(q_{(1)} \cup q_{(2)}  \| p_{(1)} \cup p_{(2)}) =  g^{-1} \left( \frac{w_{q_1}
    I^\ren(q_{(1)} \|  p_{(1)}) + w_{q_2} I^\ren(q_{(2)}  \| p_{(2)})}{w_{q_1} +
    w_{q_2}} \right)$ conduciendo a
%
\[
I_\lambda^\ren(q\|p)  =  \frac{1}{\lambda-1}   \log_2\left(  \sum_i  p_i  \left(
    \frac{q_i}{p_i} \right)^\lambda \right).
\]

Unos  a\~nos  despu\'es,  se introdujo  una  clase  m\'as  general debido  a  I.
Csisz\'ar~\cite{Csi63, Csi67, CsiShi04}, T.  Morimoto~\cite{Mor63} o S.  M.  Ali
\& S. D.  Silvey~\cite{AliSil66},  clase que llamaremos {\it $\phi$-divergencias
  de Csisz\`ar}. De manera general, con $Q \ll P$, toma la forma
%
\[
\cphi[Q]{P} = \int_\X \phi\left( \frac{dQ}{dP}(x) \right) \, dP(x),
\]
%
donde $\phi$ es  una funci\'on convexa. Estas divergencias  o casos particulares
fueron muy estudiadas las decadas  que siguieron, dando tambi\'en lugar a varias
aplicaciones~\cite{GupSha76,  BurRao82,  CreRea84,  BenCha89,  Teb92,  BenBor92,
  SalMen93, Sal94, Csi95, CrePar00, LieVaj06}.

Como para el caso de las $\phi$-entrop\'ias, esta clase se enmarca dentro de una
clase    un     poco    m\'as    general~\cite[Secs.~4.5~\&~5]{AliSil66}    (ver
tambi\'en~\cite[Sec.~I]{OrsPar95}):
%
\begin{definicion}[$(h,\phi)$-divergencia]
\label{Def:SZ:HPhiDivergencia}
%
  La  $(h,\phi)$-divergencia  de  una  distribuci\'on de  probabilidad  $Q$  con
  respecto a una  distribuci\'on de referencia $P$ tal que $Q  \ll P$ es definida
  por
  %
  \[
  \chphi[Q]{P} =  h\left( \int_\X  \phi\left( \frac{dQ}{dP}(x) \right)  \, dP(x)
  \right)
  \]
  %
  Si $P$  y $Q$ admiten  una densidad  con respecto a  una medida $\mu$,  toma la
  forma
  %
  \[
  \chphi[q]{p}  =  h\left(  \int_\X  p(x) \phi\left(  \frac{q(x)}{p(x)}  \right)
    d\mu(x) \right)
  \]
  %
  %en  su versi\'on  diferencial, 
  donde o
  %
  \begin{itemize}
  \item $\phi$ \ es estrictamente convexa y \ $h$ \ creciente, o
  \item $\phi$ \ es estrictamente c\'oncava y \ $h$ \ decreciente
  \end{itemize}
  %
  y  $\X  = X(\Omega)$  para  $X$  de  distribuci\'on~\footnote{En general,  por
    convenci\'on,  $0 \,  \phi\left(  \frac{0}{0} \right)  \igualc  0$.  Adem\'as,  se
    requiere  de  que  $\displaystyle  0  \, \phi\left(  \frac{a}{0}  \right)  =
    \lim_{\varepsilon \to  0^+} \varepsilon \,  \phi\left( \frac{a}{\varepsilon}
    \right) = a \lim_{u \to +\infty} \frac{\phi(u)}{u}$.} $P$.
  %
  Frecuentemente, se supone adicionalmente que $\phi$ y $h$ son de clase $C^2$ y
  sin perdida de generalidad, que $h(\phi(1)) = 0$.
\end{definicion}
%
\noindent Se notar\'a de que, obviamente, $\cphi{} = D_{(\id,\phi)}^{\mathrm{c}}$.

Notablemente, cuando  \ $\phi(u) =  u \log u$  \ y \ $h  = \id$ \  identidad, se
recupera de  nuevo la divergencia  de Kullback-Leibler: esta  \'ultima partenece
simultaneamente a la clase  de Csisz\`ar y a la de Bregman y  es la sola en este
caso~\cite{Csi91}.   Cuando \ $\phi(u)  = \pi_2  u \log  u -  (\pi_1 +  \pi_2 u)
\log(\pi_1 + \pi_2 u)$ \ y \ $h = \id$ \ identidad se recupera la divergencia de
Jensen-Shannon \SZ{sola de  la clase de Jensen en este  caso?}.  En adici\'on de
$\Dkl{}$, la clase de Csisz\'ar contiene la ganancia de informaci\'on de R\'enyi
para  \  $\phi(u)  = u^\lambda$  \  y  \  $h(u)  = \frac{\log  u}{\lambda-1}$  \
apareciendo tambien  en una forma muy parecida  en~\cite{Hel09, Che52, CreRea84,
  LieVaj06} y  conocida como  divergencia de Chernoff  o de  Hellinger. Contiene
varias otra como la  $J$-divergencia para \ $\phi(u) = u \log u -  \log u$ \ y \
$h = \id$, la distancia de Bhattacharyya~\cite{Bha43, Bha46:07} \ $\displaystyle
-\log \int_\X \sqrt{\frac{dQ}{dP}(x)} \, dP(x)$  \ para \ $\phi(u) = \sqrt{u}$ \
y  \ $h(u)  = -  \log u$,  instancia  particular de  la de  R\'enyi ($\lambda  =
\frac12$), la divergencia de variaci\'on total (o $L^1$ distancia) para $\phi(u)
= |u-1|$ \ y \ $h =  \id$, la divergencia de Pearson o divergencia $\chi^2$ para
\ $\phi(u) = (u-1)^2$ \ o \ $u^2-1$ \ y \ $h = \id$, para mencionar unas.

Las divergencias de Csisz\'ar tienen las propiedades siguientes
%
\begin{enumerate}
\item Positividad:
  %
  \[
  \chphi[Q]{P} \ge 0 \quad \mbox{con igualdad si y solamente si} \quad P = Q.
  \]
  %
  Esta propiedad es la consecuencia  directa de la convexidad estricta de $\phi$
  conjuntamente a  $h(\phi(1)) = 0$. De  hecho, de la desigualdad  de Jensen con
  $X$ de  distribuci\'on $P$ tenemos en  el caso $\phi$ convexa  y $h$ creciente
  $\chphi[Q]{P} = h\left( \Esp\left[ \phi\left( \frac{dQ}{dP}(X) \right) \right]
  \right) \ge  h \left( \phi \left( \Esp\left[  \frac{dQ}{dP}(X) \right] \right)
  \right)  = h(\phi(1))$  (y  similarmente en  el  caso $\phi$  c\'oncava y  $h$
  decreciente).  Fijense de  que la positividad no es  en contradicci\'on con el
  enfoque de R\'enyi  en su caso, porque conider\'o el  caso discreto finito con
  probabilidades  incompletas,  \ie su  axioma  (ii)  se cumpla  potencialemente
  solamente para los vectores de probabilidades incompletos.
%
\item $\chphi{}$ satisface  un teorema de procesamiento de  datos (o secunda ley
  de  la  termodin\'amica)  en el  sentido  de  que  si dos  distribuciones  son
  consecuencias   de  la  misma   probabilidad  de   transici\'on  (condicional)
  $p_{X_{n+1}|X_n=x_n}  =  q_{X_{n+1}|X_n=x_n}$ (densidades  con  respecto a  una
  medida $\mu$), entonces
  %
  \[
  \chphi[p_{X_{n+1}}]{q_{X_{n+1}}} \le \chphi[p_{X_n}]{q_{X_n}}.
  \]
  \SZ{Probar}
  %
\item $\chphi{}$ es convexa con  respecto al par $(P,Q)$, pero no necesariamente
  con  respecto  a  $p$  solamente  y/o  $q$. En  el  caso  $\phi$  convexa,  es
  consecuencia directa de la convexidad~\footnote{Con la hipotesis de que $\phi$
    sea  de clase $C^2$,  es sencillo  ver de  que la  Hessiana de  la funci\'on
    $(u,v) \mapsto  u \, \phi\left(\frac{v}{u}\right)$ con respecto  a $(u,v)$ es
    no    negativa,    implicando    la    convexidad    de    esta    funci\'on
    bi-variada~\cite{CamMar09}.}  (resp.   c\'oncavidad) de $(u,v)  \mapsto u \,
  \phi\left(\frac{v}{u}\right)$ sobre  $\Rset_+^2$ conjuntamente a  la crecencia
  (resp. decrecencia) de $h$.
%
\item Pensando a $\cphi{}$  con respecto a $\phi$, es lineal en  el sentido de que
  $D_{a_1   \phi_1   +    a_2   \phi_2}^{\mathrm{c}}   =   a_1
  D_{\phi_1}^{\mathrm{c}}  +  a_2  D_{\phi_2}^{\mathrm{c}}$ (con  $\phi_i$
  convexas y $a_i \ge 0$).
%
\item Sea $\phi^\star(u) = u \phi\left(  \frac1u \right)$. Es sencillo ver de que si
  $\phi$  es   convexa  (resp.    c\'oncava),  $\phi^\star$  es   tambi\'en  convexa
  (resp.   c\'oncava).   $\phi^\star$  es   llamada   {\it  $^\star$-conjugada   convexa
    (resp. c\'oncava)} de $\phi$. Luego,
  %
  \[
  \chphi[Q]{P} = D^{\mathrm{c}}_{(h,\phi^\star)}\left( \left. P \right\| Q \right).
  \]
%
\item $\chphi{}$ es  sim\'etrica si y solamente si $\phi =  \phi^\star + c (\id-1)$;
  sin perdida  de generalidad, consideramos  $c = 0$.   Sin embargo, en  el caso
  general,  se  puede  definir  una   versi\'on  simetrizada  al  imagen  de  la
  $J$-divergencia,  considerando $\chphi{}  +  D^{\mathrm{c}}_{(h,\phi^\star)}$.  En
  particular, cuando $h = \id$, tenemos
  %
  \[
  \cphi{} + D^{\mathrm{c}}_{\phi^\star} = D^{\mathrm{c}}_{\phi+\phi^\star}
  \]
  %
  que es sim\'etrica ($(\phi^\star)^\star = \phi$).
%
\item Cota superior:
  %
  \[
  \chphi{} \le h\left( \phi(0) + \phi^\star(0) \right)
  \]
  %
  posiblemente   infinita~\footnote{Por   ejemplo,   para  la   divergencia   de
    Kullback-Leibler,  $\phi(u) =  u \log  u$, dando  $\phi^\star(u) =  - \frac{\log
      u}{u}$,  tales que  $\phi(0) =  0$ \  y \  $\phi^\star(0) =  + \infty$:  no es
    acotada por arriba.}.
%
\end{enumerate}
%
\noindent  Estas  propiedades  con   varias  otras  se  encuentran  por  ejemplo
en~\cite{Vaj72,  Csi74, LieVaj87,  KafOst91, Ost96,  OstVaj03,  Vaj09, KumChi05,
  LieVaj06}.   Como en  el caso  de las  divegencias de  Jensen, en  general las
divergencias  sim\'etricas ($\phi  =  \phi^\star$) no  satisfacen  a la  desigualdad
triangular,  y entonces no  dan lugar  a una  distancia entre  distribuciones de
probabilidad, aparte  en casos particulares  (ej. divergencia de  la variaci\'on
total,  divergencia  de Hellinger  o  R\'enyi  con  $\lambda =  \frac12$).   Sin
embargo,  se   prob\'o  en~\cite[Teoremas~1~\&~2,  Remark~6]{KafOst91}   \  y  \
\cite{Ost96, OstVaj03, Vaj09} el lema siguiente, condici\'on suficiente para que
una potencia de la divergencia satisfaga a la desigualdad triangular:
%
\begin{lema}
\label{Lem:SZ:PotenciaCziszarDesigualdadTriangular}
%
Sea \ $\phi$ \  una funci\'on convexa tal que \ $\phi^\star  = \phi$, \ $\phi(0) \ne
0$  \  y  \  $\cphi{}$  su  divergencia  de  Csisz\'ar  asociada  ($h  =  \id$).
Adicionalmente    se   supone    que   $\phi(1)    =   0$    (ver   definici\'on
Def.~\ref{Def:SZ:HPhiDivergencia}) y  de que $\phi$ es  estrictamente convexa en
$1$.  Si existe $\kappa \in \Rset_{0,+}$ tal que
  %
  \[
  h(u) = \frac{\left( 1-u^\kappa  \right)^{\frac{1}{\kappa}}}{\phi(u)}, \quad u \in [ 0
  \; 1) \quad \mbox{es no decreciente,}
  \]
  %
  \noindent entonces
  %
  \[
  \left(  \cphi[\cdot]{\cdot} \right)^s \quad  s \in  (0 \;  \kappa] \quad
  \mbox{satisface la desigualdad triangular}
  \]
  %
  \noindent   y  entonces  es   una  m\'etrica   entre  dos   distribuciones  de
  probabilidades.
\end{lema}
%
\noindent Adem\'as, en~\cite[Sec.~3]{KafOst91} se dan condiciones necesarias que
debe cumplir $\kappa$ cuando $\phi$ tiene un comportamiento particular en $u \to
0$ \  y/o \ $u \to  0$; eso va  m\'as all\'a de la  meta de esta secci\'on  y el
lector se  podr\'a referir a~\cite{KafOst91,  Ost96, OstVaj03} para  tener m\'as
detalles.

Este   lema   se  us\'o   para   probar   el   caracter  m\'etrico   de   $\Big(
J_{\mathrm{js}}^{\pi_{\mathrm{u}}}(p_{(1)},p_{(2)})  \Big)^s, \quad  0  < s  \le
\frac12$~\cite[y   ref.]{OstVaj03,  OsaBus18},   siendo   $J_{\mathrm{js}}$  una
divergencia de Csisz\'ar particular.  Se us\'o tambi\'en para probar que $\left(
  \cphi{} \right)^s$ \ con  \ $\phi(u) = \frac{\lambda}{\lambda-1} \left( \left(
    1+u^\lambda  \right)^{\frac{1}{\lambda}}   -  2^{\frac{1}{\lambda}-1}  (1+u)
\right)$ \ y  \ $\kappa = \min\left( \lambda  \, , \, \frac12 \right)$  \ es una
m\'etrica~\cite{OstVaj03}.

% I.Csisz��r,Informationmeasures:Acriticalsurvey,In:Trans.7thPragueConf.onInformationTheory,Vol.A,1974,73���86.
% [22]
% F.Liese,I.Vajda,ConvexStatisticalDistances,In:Band95,TeubnerTextezurMathematik,Leipzig,1987.

Para  cerrar esta  secci\'on,  se  mencionar\'a de  que  varias aplicaciones  se
encuentran  en   la  literatura  que  sea   en  estimaci\'on,  discriminaci\'on,
reconocimiento de patrones, pruebas  de adecuaci\'on o inferencia estad\'istica,
entre  otros~\cite[y  ref.]{Kai67,  BoeLub79,  Poo88,  Bas89,  Csi95,  OrsPar95,
  MenMor97:5,  Par99, LieVaj06,  Par06,  NieBol11, CsiMat12, Bas13}  en  adici\'on de  las
referencias de esta secci\'on, para dar unas.

\SZ{
cf Bha 43 egalement
}



\SZ{
% ----- Integral Probability Metric

\subsubseccion{IPM}
\label{Sssec:SZ:IPM}
}


% ================================= Identidades

\subseccion{?`Como se generalizan las identidades y desigualdades?}
\label{Ssec:SZ:IdentidadesGeneralizadas}

\paragraph{Principio  de entrop\'ia m\'axima}  Si este  principio naci\'o  en el
marco  de   la  termodynamica   o  f\'isica,  con   la  entrop\'ia   de  Shannon
(Boltzman-Gibbs),  tratando de  las nociones  generalizadas de  incertas, vuelve
natural  preguntarse  sobre  la  extensi\'on   de  este  problema  en  el  marco
general.  Tal estudio  fue  hecho en  varios  trabajos, ej.~\cite{Mor52,  Kap89,
  KesKap89,  KesKap90,  BorLew91:03,  BenBor92,  TebVaj93,  CosHer03,  CsiMat12,
  BerGir15}
%\SZ{, Kagan 63}.
% Kap89

El  problema  se formaliza  como  en el  caso  Shannon,  buscando la  entrop\'ia
m\'axima  sujeto a  v\'inculos: sea  $X$ variable  aleatoria viviendo  sobre $\X
\subset \Rset^d$ con  $K$ momentos \ $\Esp\left[ M_k(X) \right]  = m_k$ \ fijos,
con  $M_x: \X  \to  \Rset$,  el problema  de  $(h,\phi)$-entrop\'ia m\'axima  se
formula de  la manera  siguiente: sean \  $M(x) =  \begin{bmatrix} M_1(x) &
  \cdots & M_K(x) \end{bmatrix}^t$ \ y \ $m = \begin{bmatrix} m_1 & \cdots &
  m_K \end{bmatrix}^t$, \ se busca,
%
\[
p_{\mathrm{em}} = \argmax_p  \hhphi[p] \qquad \mbox{sujeto a} \qquad p  \ge 0, \quad \int_\X
\, p(x) \, d\mu(x) = 1, \quad \int_\X M(x) \, p(x) \, d\mu(x) = m.
\]
%
donde   los  dos   primeros   v\'inculos  aseguran   de  que   $p_{\mathrm{em}}$
(positividad, normalizaci\'on) sea una distribuci\'on de probabilidad. Si $\phi$
es convexa (resp.   c\'oncava), $h$ es creciente (resp.   decreciente) as\'i que
maximizar $\hhphi$ es equivalente a maximizar $\hphi$ (resp.  $H_{-\phi}$).  Sin
perdida de generalidad, se puede considerar la situaci\'on $\phi$ convexa.  Como
en  el caso  de Shannon,  introduciendo factores  de Lagrange  $\eta_0, \:  \eta
= \begin{bmatrix} \eta_1 & \cdots & \eta_K \end{bmatrix}^t$ para tener en cuenta
los  v\'inculos, el  problema (variacional)  consiste  de nuevo  a maximizar  la
integral~\cite{GelFom63, Wei74, Bru04, Cla13, Kom14, Mil00, CamMar09, CovTho06}
%
\[
p_{\mathrm{em}} = \argmax_p \int_\X \left( - \phi\left( p(x) \right) + \eta_0 \,
  p(x) + \eta^t M(x) \, p(x) \right) d\mu(x)
\]
%
donde $\eta_0, \eta$ ser\'an determinados  para satisfacer a los v\'inculos.  Se
puede ver  que, como  en el  caso de la  entrop\'ia de  Shannon, el  problema es
c\'oncava con  respecto a $p$ (\ie  el integrande es una  funci\'on c\'oncava de
$p$), as\'i que existe una soluci\'on \'unica al problema~\cite{GelFom63, Wei74,
Bru04,  Mil00,  CamMar09,   Cla13,  Kom14}.  De  nuevo,  de   la  ecuaci\'on  de
Euler-Lagrange~\footnote{Ver                       nota                       de
pie~\ref{Foot:SZ:EulerLagrange}}~\cite{GelFom63, Wei74, Bru04,  Cla13, Kom14} en
el  caso continuo,  o derivando  con respecto  a las  probabilidades en  el caso
discreto, se obtiene la  ecuaci\'on $- \phi'(p(x)) + \eta_0 +  \eta^t M(x) = 0$.
La funci\'on entr\'opica $\phi$ es c\'oncava y de clase $C^2$, as\'i que $\phi'$
es continua decreciente, y de la monotonicidad es invertible. Entonces,
%
\[
p_{\mathrm{em}}(x) = \phi'^{-1}\left( \eta^t M(x) - \varphi(\eta) \right)
\]
%
con $\eta$  tal que se satisfacen  los v\'inculos de  momentos y $\varphi(\eta)$
tal que  se satisface el  v\'inculo de normalizaci\'on.   Si el resultado  no es
positivo  en $\X$,  de las  condiciones  KKT, $p_{\mathrm{em}}(x)  = \Big(  \phi'^{-1}\left(
  \eta^t M(x) - \varphi(\eta)  \right) \Big)_+$.  Estas distribuci\'ones no caen
en general en la familia exponencial.  De una forma, usando entrop\'ia generales
permite escaparse de esta familia.

Como  en el  caso de  Shannon, queda  obviamente  el hecho  de que  no se  puede
determinar $\eta$ tal que se satisfacen todos los v\'inculos (y en particular
la de normalizaci\'on).

Tal como en el caso Shannon, existe una prueba informacional:
%
\begin{lema}\label{Lem:SZ:MaxPhiEntInformacional}
%
  Sea \  $\displaystyle \P_m =  \left\{ p  \ge 0 \tq  \int_\X  p(x) \,
    d\mu(x) =  1, \quad \int_\X M(x)  \, p(x) \, d\mu(x)  = m \right\}$  \ y \
  $p_{\mathrm{em}}  \in   \P_m$  \   que  satisfaga  \   $\phi'(p_{\mathrm{em}}(x))  =  \eta^t   M(x)  +
  \eta_0$. Entonces
  %
  \[
  \forall  \, p  \in  \P_m,  \quad \hhphi[p]  \le  \hhphi[p_{\mathrm{em}}] \qquad  \mbox{con
    igualdad ssi} \quad p = p_{\mathrm{em}} \quad (\mu\mbox{-c. s.)}.
  \]
  %
\end{lema}
\begin{proof}
  Sin  perdida  de  generalidad,   consideramos  $\phi$  convexa.  Calcuando  la
  divergencia de Bregman asociado a $\phi$ de $p$ relativamente a $p_{\mathrm{em}}$ da
  %
  \begin{eqnarray*}
  \bphi[p]{p_{\mathrm{em}}} & = & \hphi[p_{\mathrm{em}}] - \hphi[p] - \int_\X \big( p(x) - p_{\mathrm{em}}(x) \big)
  \, \phi'\left( p_{\mathrm{em}}(x) \right) \, d\mu(x)\\[2.5mm]
  %
  & = & \hphi[p_{\mathrm{em}}] - \hphi[p] - \eta_0 \int_\X \big( p(x) - p_{\mathrm{em}}(x) \big)
  \, d\mu(x) - \eta^t \int_\X \big( p(x) - p_{\mathrm{em}}(x) \big)
  \, M(x) \, d\mu(x)\\[2.5mm]
  %
  & = & \hphi[p_{\mathrm{em}}] - \hphi[p]
  \end{eqnarray*}
  %
  siendo \  $p$ \ y \  $p_{\mathrm{em}}$ \ en \  $\P_m$. El resulta proviene  entonces de la
  positividad de la divergencia de Bregman,  con igualdad si y solamente si $p =
  p_{\mathrm{em}}$ conjuntamente a la crecencia de $h$.
\end{proof}
%
Este lema prueba que,  dando v\'inculos ``razonables'', la $(h,\phi)$-entrop\'ia
es acotada por arriba, y que se alcanza la cota. Por ejemplo,
%
\begin{itemize}
\item Con  $K =  0$ \  y \  $\X$ \ de  volumen finito  \ $|\X|  < +  \infty$, la
  distribuci\'on de $(h,\phi)$-entrop\'ia m\'axima es la distribuci\'on uniforme
  en el caso discreto tal como en el caso continuo.
%
\item Con $K  = 1$, \ $\X = \Rset^d$  \ y \ $M(x) = x  x^t$ (visto como $\frac{d
    (d+1)}{2}$    v\'inculos),   y    $\phi(u)   =    u^\lambda$    (R\'enyi   o
  Havrda-Charv\'at-Dar\'oczy),  la  distribuci\'on  de  entrop\'ia  m\'axima  es
  Student-$t$  cuando  $\lambda  \in   \left(  \frac{d}{d+2}  \;  1  \right)$  y
  Student-$r$  cuando  $\lambda   >  1$~\cite{Mor52,  CosHer03,  Kap89}.  Cuando
  $\lambda   \to  1$   se  recupera   la   gaussiana  como   caso  limite   (ver
  secciones~\ref{Sssec:MP:StudentT}  y~\ref{Sssec:MP:StudentR}), que corresponde
  precisamente a la maximizante de le entrop\'ia de Shannon, caso l\'imite de la
  entrop\'ia de R\'enyi (ver m\'as arriba y secci\'on~\ref{Ssec:SZ:MaxEnt}).
\end{itemize}

Como  en el  caso de  Shannon, si  $\mu =  Q$ es  una medida  de  referencia, el
problema vuelev  ser un problema de minimizaci\'on  de la $(h,\phi)$-divergencia
de $P  \ll Q$  con respecto a  $Q$.  La  densidad obtenida es  $p_{\mathrm{em}} =
\frac{dP_{\mathrm{em}}}{dQ}$.

\

\SZ{\paragraph{Desigualdad de  la potencia entr\'opica}  EPI variaciones Madiman
  Barron  MadBar07 EPI  generalizada  por Madiman,  etc.  Lutwak, Bercher  etc.,
  Kagan; Boeke 77

\cite{WanMad04}: EPI Renyi y reareglo

\cite[p. 499]{CovTho06}: tipo EPI via Renyi (ver Beckner)

Ver \cite{Sta59, Hir57, Sobolev, DemCov91}
}

\

\SZ{reapparition Fisher comme courbure, cf Varma, Jizba, MenMor97...}

\

\SZ{
   On the theory  of Fisher's
  amount  of  information Sov.  Math.  Dokl., 4  (1963),  pp.  991-993, etc,  la
  codificaci\'on a la Renyi (Cambell,  Hooda 2001, Bercher) 

\

 y la cuantificacion
  fina;
  An extension  of the Fisher information  measure I. Csisz\'ar,  P. Elias (Eds.),
  Topics   in  Information  Theory,   North-Holland,  Berlin/New   York  (1977),
  pp. 113-123  o Hammad o  Vajda 73 o  Ferentinos81 en el marco  Fisher}

\SZ{Revisite capacite a la Daroczy? codage; parler de la quantification fine et HCD}


% Nota para mi
% Schut-Ostrowsky: f sym, Schur convex ssi
%
% $\forall \: 0 \le i \ne j \le |\X|$, \qquad $(x_i-x_j) \left( \frac{\partial
%     f}{\partial x_i} - \frac{\partial f}{\partial x_j} \right) \ge 0
%
% Ver libro de Loss y Ruskai
