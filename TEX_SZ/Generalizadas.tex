\seccion{Entropias y divergencias generalizadas}
\label{sec:SZ:Generalizadas}

A pesar de que la entropia de Shannon y sus cantidades asociadas demostraron sus
potencias tan de un punto de vista descriptivo que en termino de aplicaciones en
la  transmisi\'on  de  la  informaci\'on  y  la  compresi\'on,  varios  nociones
informacionales, de  tipo entropias o  divergencias, aparecieron luego.  En esta
secci\'on no se  desarollar\'a todos los enfoques ni  todas las aplicaciones tan
la  literatura es  importante. La  meta  es dar  los caminos  conduciendo a  las
generalizaciones de la  entropia de Shannon por un lado, y  de la divergencia de
Kullback-Leibler por el otro lado. No son siempre vinculados, a pesar de que sea
desirable que a cada entropia sean asociados nociones de entropias condicionales
y relativas.

% ================================= Salicru

\subseccion{Entropias y propiedades}
\label{sec:SZ:Salicru}

Si  la entropia  de  Shannon  fue el  punto  de salida  fundamental  en todo  el
desarollo de la teoria de la informaci\'on, un poco mas de una decada despues de
su    papel   clave    y   muy    completo,   R\'enyi    propuso    una   medida
generalizada~\cite{Ren61}. Su  punto de  vista fue mas  matematico que  fisico o
ingeniero.  Retom\'o los axiomas de Fadeev~\cite{Fad56, Fad58, Khi57}
% Feinstein, cf ref  de R��nyi
%
a  probabilidades  incompletas   $p  =  \begin{bmatrix}  p_1  &   \cdots  &  p_n
\end{bmatrix}^t,  \quad p_i  \ge  0,  \quad w_p  =  \sum_i p_i  \le  1$: (i)  la
invarianza de  $H(p)$ por permutaci\'on de  os $p_i$, (ii) la  continuidad de la
incerteza elemental  $H(p_i)$ ($p_i$ visto como  probabilidad incompleta), (iii)
$H\left( \frac12 \right) = 1 $,  (iv) la aditividad $H(p \otimes q) = H(p)+H(q)$
donde $p \otimes q$ es el producto de Kronecker~\footnote{$\begin{bmatrix} p_1 &
\cdots & p_n \end{bmatrix}^t \otimes \begin{bmatrix} q_1 & \cdots & q_m
\end{bmatrix}^t = \begin{bmatrix} p_1 q_1 & \cdots  & p_1 q_m & \cdots & p_n q_1
&  \cdots  &  p_n  q_m  \end{bmatrix}^t$.}, \ie  probabilidad  conjunta  de  dos
variables independientes,  y consider\'o en  lugar de la recursividad  un axioma
dicho de valor promedio,  axioma muy parecido a la recursividad. Para  $p$ \ y \
$q$ probabilidades incompletas tales que $p \, \cup \, q = \begin{bmatrix} p_1 &
\cdots & p_n  & q_1 & \cdots  & q_m \end{bmatrix}^t$ sea incompleta  ($w_p + w_q
\le 1$),  el axioma  (v) es  $H(p \, \cup  \, q)  = \frac{w_p \,  H(p) +  w_q \,
H(q)}{w_p  + w_q}$.   Demostr\'o que  con (v)  en lugar  de la  recursividad, el
conjunto  de   axiomas  conduce  de  nuevo   a  la  entropia   de  Shannon.   La
generalizaci\'on  propuesta  por  R\'enyi  era  de  generalizar  el  axioma  (v)
remplazando la media  aritm\'etico por una media generalizada  (v') $H^\ren(p \,
\cup \, q) =  g^{-1} \left( \frac{w_p \, g\big( H^\ren(p) \big)  + w_q \, g\big(
H^\ren(q) \big)}{w_p  + w_q}\right)$ con $g$ estrictamente  monotona y continua,
llamado media {\it cuasi-aritm\'etica,  o quasi-lineal, o de Kolmogorov-Nagumo}.
De  las  propiedades de  la  media  cuasi-aritmetica~\cite{Nag30, Kol30,  Kol91,
HarLit52}, eso  es equivalente a  buscar una entropia elemental  $H^\ren(p_i)$ y
remplazar  la media  aritm\'etica  $\sum_i  p_i H^\ren(p_i)$  por  una media  de
Kolmogorov-Nagumo, $g^{-1} \left( \sum_i  p_i g\big( H^\ren(p_i) \big) \right)$.
R\'enyi propus\'o la funci\'on de Kolmogorov-Nagumo $g_\beta(x) = 2^{(\beta-1)
x},  \quad \beta >0,  \quad \beta  \ne 1$,  probando que  la entropia  que los
axiomas (i)-(ii)-(iii)-(iv)-(v') se  cumplen y conduce a la  entropia de R\'enyi
de un vector de probabilidad $p$,
%
\[
H_\beta^\ren(p) = \frac{1}{1-\beta} \log_2 \left( \sum_{i=1}^n p_i^\beta \right)
\]
%
\noindent   Relaxando  el  axioma   (iii),  se   puede  elegir   $g_\beta(x)  =
a^{(\beta-1) x}, \quad  a > 0, \quad a  \ne 1$; el logaritmo ser\'a  de la base
$a$ cualquiera; En  lo que sigue, usaremog $\log$ sin  precisar la elecci\'on de
base.   R\'enyi  nombr\'o  esta  medida  de incerteza  {\it  entropia  de  orden
$\beta$}. Notablemente,
%
\[
H_1^\ren(p)  \equiv   \lim_{\beta  \to   1}  H_\beta^\ren(p)   =   H(p)  \quad
\mbox{entropia de Shannon}
\]
% \noindent En otros terminos, la clase de R\'enyi contiene como caso particular
la  entropia  de  Shannon.  En  su  papel,  R\'enyi  introdujo una  ganancia  de
informaci\'on,  parecida  a  una  entropia  relativa,  probando  que  las  solas
entropias admisibles  son la  de Shannon  y la que  introdujo. Volveremos  en la
secci\'on siguiente sobre esta entropia  relativa, o divergencia de R\'enyi. Por
axiomas,     las     propiedades    \ref{prop:SZ:continuidad}     (continuidad),
\ref{prop:SZ:permutacion}       (invarianza      por       permutaci\'on)      y
\ref{prop:SZ:aditividad} (additividad)  de la  entropia de Shannon  se conservan
entonces  en  el  marco   de  R\'enyi  y  se  pierde  \ref{prop:SZ:recursividad}
(recursividad), todav\'ia por axiomas. Veremos luego la otras que se conservan o
modifican en un marco mas general.

Unos  a\~nos despu\'es de  R\'enyi, de  la famosa  escuela matematica  checa, J.
Havrda   \&    F.    Charv\'at   en~\cite{HavCha67}    (ver   tambi\'en~\cite[en
checo]{Vaj68}) volvieron a los axiomas de Khintchin, para extender la entopia de
Shannon,  \ie  considerando  (i)   la  invarianza  por  permutaci\'on,  (ii)  la
continuidad, (iii) la expensividad, (iv) $H^\hc(1) = 0$ y $H^\hc\left( \frac12 ,
  \frac12   \right)  =   1$,  pero   generalisando  la   recursividad   por  (v)
$H^\hc(p_1,\ldots,p_n)    =   H^\hc(p_1,\ldots,p_{n-2},p_{n-1}+p_n)    +   \beta
(p_{n-1}+p_n)^\beta H^\hc\left( \frac{p_{n-1}}{p_{n-1} + p_p},\frac{p_n}{p_{n-1}
    + p_p}  \right), \quad  \beta > 0$~\footnote{En  sus papel, lo  imponen para
  cualquier pars $p_i, p_j$ sin imponer la invarianza por permutaci\'on, pero es
  equivalente a la  exposici\'on de este parafo.}.  Con $\beta  = 1$ se recupera
la  recursividad estandar,  pero  con $\beta  \ne  1$ eso  permite  dar un  peso
diferente a la incerteza del estado interno \ie probabilidades que se juntan (la
describen como clasificaci\'on refinada).  Estos axiomas conducen necesariamente
a la entropia (teorema~1)
%
\[
H_\beta^\hc(p) = \frac{1}{1-2^{1-\beta}} \left( 1 - \sum_i p_i^\beta \right)
\]
%
que nombraron {\it $\beta$-entropia structural}.  De nuevo, relaxando el axioma
(iv),  se puede  remplazar en  el coeficient  $2^{1-\beta}$  por $a^{1-\beta},
\quad a >  0, \quad a \ne 1$. De  nuevo, parae que la entropia  de Shannon es un
caso particular,
%
\[
H_1^\hc(p)  \equiv   \lim_{\beta  \to   1}  H_\beta^\hc(p)   =   H(p)  \quad
\mbox{entropia de Shannon}
\]
% Por   axioma,   se   conservan   las   propiedades   \ref{prop:SZ:continuidad}
(continuidad), \ref{prop:SZ:expansabilidad} (expansabilidad)  de Shannon en este
marco. Se  prob\'o tambi\'en  que se conserva  la propiedad de  concavavidad con
respeto    a   los   $p_i$    \ref{prop:SZ:concavidad},   la    de   maximalidad
\ref{prop:SZ:cotamaxima}    alcanzada   para    una    distribuci\'on   uniforma
(teorema~2). Aun  que no aparece  as\'i en el  papel, satisface la  propiedad de
Schur-concavidad  \ref{prop:SZ:Schurconcavidad}  (teorema~3).  A  pesar  de  que
mencionan que $H_\beta^\hc$ sea  diferente que $H_\beta^\ren$, es sencillo ver
que hay  un mapa  uno-uno entre  las dos entropias.  Se notara  en un  marco mas
general otras propiedades.

Independiente  de Havrda  \&  Charv\'at, todav\'ia  en  el este,  en la  escuela
h\'ugara, Z.  Dar\'oczy en~\cite{Dar70} defino la entropia $H^f$ a partir de una
{\it  funci\'on  informaci\'on}  $f$   satifaciendo  (i)  $f(0)  =  f(1)$,  (ii)
$f\left(\frac12\right)  = 1$  \ y  la ecuaci\'on  funcional (ii)  $f(x)  + (1-x)
f\left(  \frac{y}{1-x} \right)  = f(y)  + (1-y)  f\left(  \frac{x}{1-y} \right)$
sobre  $\{  (x,y)  \in [0  ;  1)^2,  \quad  x+y  \le  1 \}$,  siendo  $H^f(p)  =
\sum_{i=2}^n s_i  f\left( \frac{p_i}{s_i} \right), \quad  s_i = \sum_{j=1}^{i-1}
p_j$.   Dar\'oczy mostr\'o  que  si $f$  es medible,  o  continua en  $0$, o  no
negatiba  y  acotada, necesariamente  $f(x)  =  h_2(x) =  -x  \log_2  x -  (1-x)
\log_2(1-x)$,   conduciendo   a  la   entropia   de   Shannon  (teorema~1;   ver
tambi\'en~\cite{Lee64,  Tve58, Ken64}).   En otros  terminos, su  axioma  (v) es
alternativa a  la recursividad.  Para  extender la entropia de  Shannon, propuso
extender este axioma  (v) por la ecuaci\'on funcional  $f_\beta(x) + (1-x)^\beta
f_\beta\left(  \frac{y}{1-x} \right)  = f_\beta(y)  +  (1-y)^\beta f_\beta\left(
  \frac{x}{1-y}  \right)$,   lo  que   condujo  necesariamente  a   la  entropia
(teoremas~2 y~3)
%
\[
H_\beta^\dar(p) = \frac{1}{1-2^{1-\beta}} \left( 1 - \sum_i p_i^\beta \right)
\]
%
\noindent  es  decir  nada  mas  que  la  entropia  introducida  por  Havdra  \&
Charv\'at. En lo que sigue, se la denota $H_\beta^\hcd$. Sin embargo, el estudio
de Dar\'oczy fue  mas intensivo que el de Havdra  \& Charv\'at.  Primero, not\'o
el mapa  entre su  entropia y la  de R\'enyi. Adicionalmente  a Havdra-Charv\'at
probaron que se conserva  la propiedad \ref{prop:SZ:permutacion} (invarianza por
permutaci\'on, que no era un  axioma en su enfoque), $H_\beta^\hcd\left( \frac12
  ,  \frac12   \right)  =  1$   (lo  llama  normalizaci\'on),   la  expansividad
\ref{prop:SZ:expansabilidad},   una  additividad  extendida,   una  recursividad
extendida  precisamente  del modelo  de  Havrda-Charv\'at (teorema~4).   Prob\'o
tambi\'en   \ref{prop:SZ:positividad},   positividad   alcanzado  en   el   caso
deterministico y  la m\'aximalidad \ref{prop:SZ:cotamaxima} en  el caso uniforme
(teorema~6),  que incidentalmente  $H_\beta^\hcd\left( \frac1\alpha  ,  \ldots ,
  \frac1\alpha \right)$ crece con el  cardinal $|\X| = \alpha$.  Muy interesante
tambi\'en es se puede definir una entropia condicional en el mismo modelo que en
el caso de Shannon $H_\beta^\hcd(X|Y) = \sum_y \left[ p_{X|Y}(x,y) \right]^\beta
H_\beta^\hcd(   p_{X|Y}(\cdot,y)   )$,   que   existe  una   regla   de   cadena
\ref{prop:SZ:cadena}, $H_\beta^\hcd(X,Y)  = H_\beta^\hcd(Y) + H_\beta^\hcd(X|Y)$
y    que    \ref{prop:SZ:condicionar}    condicionar    reduce    la    entropia
$H_\beta^\hcd(X|Y) \le H_\beta^\hcd(X)$  (teorema~8).  Mostr\'o tambi\'en que si
se pierde  la additividad, se  obiene para \  $X$ \ e  \ $Y$ \  independientes \
$H_\beta^\hcd(X,Y) = H_\beta^\hcd(X) +  H_\beta^\hcd(Y) + \left( 2^{1-\beta} - 1
\right) H_\beta^\hcd(X) H_\beta^\hcd(Y)$.  La  propiedades de regla de cadena le
permiti\'o  revisitar  la  caracterisaci\'on  de  un canal  de  transmisi\'on  y
redefinir una capacidad canal extendidas (capacidad tipo $\beta$; basicamente se
usa el  mismo enfoque que Shannon,  pero usando $H_\beta^\hcd$ en  lugar de $H$,
ver secci\'on~6 del papel).

Estas  entropia fueron (re)descubiertos  varios otras  veces y/o  estudiados mas
detenidamente    en    varios     campos    y    varios    extensiones    fueron
introducidas~\cite[entre  otros]{Var66, Oni66,  Kap67,  Vaj68, LinNie71,  Ari71,
Bur72, AczDar75,  ShaMit75, ShaMit75,  ShaTan75, Mit75, BoeLub80,  Fer80, Tsa88,
Rat91, Kan01}.   Un primer enfoque  mas general es  debido a S.  Arimoto  en los
primeros a\~nos de  la decada 1970~\cite{Ari71} y rediscubierto  y estudiado con
mas  detaller y  una decada  despues por  Burbea y  Rao~\cite{BurRao82}  y luego
estudiado   por   Salicr\'u~\cite{Sal87}.     La   medida   propuesta,   llamada
$\phi$-entropia, es definida por
%
\[
H_\phi(p) = -\sum_i \phi(p_i) \qquad \mbox{con} \qquad \phi \mbox{ convexa}
\]
%
Burbea  y  Rao  asociaron  una  medida  de  divegencia  a  esta  entropia.   Las
$\phi$-entropias contienen Shannon como caso  particular ($\phi(x) = x \log x$),
as\'i  que  la  clase   de  Havdra-Charv\'at-Dar\'oczy  ($\phi(x)  =  \frac{x  -
x^\beta}{2^{1-\beta}-1}$)  como mencionado, pero  no la  clase de  R\'enyi. De
hecho, las $\phi$-entropias se enmarcan en una clase un poco mas amplia, llamada
$(h,\phi)$-entropias~\cite{SalMen93, MenMor97},
%
\begin{definicion}[$(h,\phi)$-entropia]
La $(h,\phi)$-entropia de una  distribuci\'on de probabilidad $p_X$ definida sobre
$\X$ de cardinal finito $|\X| = \alpha$ es definida por
%
\[
H_{(h,\phi)}(X)  = H_{(h,\phi)}\left(  p_X  \right) =  h\left(  \sum_{x \in  \X}
\phi\left( p_X(x) \right) \right)
\]
%
donde o
%
\begin{itemize}
\item $\phi$ es concava y $h$ creciente, o
\item $\phi$ es convexa y $h$ decreciente
\end{itemize}
%
Frecuentemente, se  supone adicionalmente que $\phi$  y $h$ son  de clase $C^2$,
que $\phi(0) = 0$ (incerteza elemental asociada a un estado de probabilidad nula
vale cero) y, sin perdida de generalidad, que $h(\phi(1)) = 0$.
\end{definicion}
%
\noindent  (ver  tambi\'en~\cite{Est97}  para  una  generalizaci\'on  a\'un  mas
amplia). Cu\'ando  $h(x) = x$ se  recupera la $\phi$-entropia,  incluyendo la de
Shannon y las  de Havdra-Charv\'at-Dar\'oczy. Ademas, la familia  de R\'enyi cae
tamb\'ien  en   esta  familia  ($\phi(x)   =  x^\beta$  y  $h(x)   =  \frac{\log
  x}{1-\beta}$) as\'i que todas las entropias evocadas en el parafo anterior.

Obviamente,  de  las  propiedades  de  la  entropia  de  Shannon,  se  conservan
\ref{prop:SZ:continuidad}  (continuidad),  \ref{prop:SZ:permutacion}  (invariaza
por  permutaci\'on),  \ref{prop:SZ:biyeccion}  (invarianza por  transformaci\'on
biyectiva  de  $X$),   \ref{prop:SZ:expansabilidad}  (expansabilidad,  debido  a
$\phi(0) = 0$).

Ademas  se conserva  la  Schur  convavidad \ con una reciproca
%
\begin{propiedadesPhi}\setcounter{enumi}{\value{PropSchurConcavidad}}
%
\item  \centerline{$\forall \:  (h,\phi),  \quad  p \prec  q  \: \Rightarrow  \:
    H_{(h,\phi)}(p) \ge H_{(h,\phi)}(q)$}
  %
  \centerline{Reciprocamente,   si    $\quad   \forall   \:    (h,\phi),   \quad
    H_{(h,\phi)}(p) \ge  H_{(h,\phi)}(q) \quad \mbox{  entonces } \quad  p \prec
    q$}
  %
  En otros terminos, se obtiene la  relaci\'on de mayorisaci\'on si se cumple la
  relaci\'on  de ordre  entropicas para  cualquier par  de  funciones entropicas
  $(h,\phi)$.   La  Schur-concavidad (y  su  reciproca)  es  consecuencia de  la
  desigualdad  de Schur~\cite{Sch23}  o Hardy-Littlewood-P\'olya~\cite{HarLit29,
    HarLit52} o Karamata~\cite{Kar32}  (ver tambi\'en~\cite[Cap.~3, Prop.~C.1 \&
  Cap.~4,  Prop.~B.1]{MarOlk11} o~\cite[Teorema~II.3.1]{Bha97}):  $p \prec  q \:
  \Rightarrow  \: \sum_i  \phi(p_i) \le  \sum_i \phi(q_i)$  para  toda funci\'on
  $\phi$ convexa.
\end{propiedadesPhi}
% Schur, I.  (1923). Issai Schur Collected  Works (A.  Brauer  and H.  Rohrbach,
% eds.), Vol. II. pp. 416����427. Springer-Verlag, Berlin, 1973]
%
%
% Ver Schur-Ostrowski f  sym, Scur-convexe ssi (xi - xj)  (df/dx_i - df/dxj) \ge
% 0, 1 \le i \ne j \le alpha
%
Como consecuencia, se conserva la positividad \ref{prop:SZ:positividad} gracia a
$\phi(0) =  0$ y  $h(\phi(1)) = 0$  (alcanzado en  el caso deterministico)  y la
maximalidad \ref{prop:SZ:cotamaxima} (caso uniforme),
%
\[
0 \le H_{(h,\phi)}(X) \le h\left( \alpha \, \phi\left( \frac1\alpha \right) \right)
\]

Con respeto a la concavidad  \ref{prop:SZ:concavidad}, no se conserva en general:
%
\begin{propiedadesPhi}\setcounter{enumi}{\value{PropConcavidad}}
\item  Si  $h$ est  concava,  entonces $H_{(h,\phi)}$  es  concava.  Eso es  una
consecuencia   de    la   concavidad   de   $\phi$   y    decrecencia   de   $h$
(resp. convexidad/crecencia) conjuntamente a  la concavidad de $h$. La reciproca
no es  verdad. Por  ejemplo, se puede  ver que  si $\beta <  1$, la  entropia de
R\'enyi es  concava, pero se proba que  existe un $\beta^*(\alpha) >  1$ tal que
para cualquier $\beta \le \beta^*(\alpha)$ se conserva la concavidad, a pesar de
que $h$ no sea necesariamente concava~\cite[p.~57]{BenZyc06}.
\end{propiedadesPhi}

Se pierde la propiedad de recursividad \ref{prop:SZ:recursividad}, pero se puede
vincular  la  entropia  total con  la  obtenida  juntando  dos estados  por  una
desigualdad:
%
\begin{propiedadesPhi}\setcounter{enumi}{\value{PropRecursividad}}
\item  Sean \  $X$ \  definido  sobre \  $\X$ \  y  \ $\overline{X}$  \ sobre  \
  $\overline{X}$,
  %
  \[
  \left\{  \begin{array}{l}\overline{\X} =  \{  x_1 ,  \ldots  , x_{\alpha-2}  ,
      \overline{x}_{\alpha-1}\}  \quad   \mbox{con  el  estado   interno}  \quad
      \overline{x}_{\alpha-1}   =  \{   x_{\alpha-1}  ,   x_\alpha  \},\\[2.5mm]
      p_{\overline{X}}(x_i)  = p_X(x_i),  \quad 1  \le i  \le \alpha-1  \quad \mbox{y}
      \quad p_{\overline{X}}(\overline{x}_{\alpha-1}) = p_X(x_{\alpha-1}) +
      p(x_\alpha)  \quad  \mbox{distribuci\'on  sobre  }  \overline{\X}\\[2.5mm]
      \displaystyle   \overline{q}(x_j)    =   \frac{p_X(x_j)}{p_X(x_{\alpha-1})   +
        p_X(x_\alpha)}, \quad j =  \alpha-1, \alpha \quad \mbox{distribuci\'on del
        estado   interno}\end{array}\right.
  \]
\[
  H(p_X)  \ge H(p_{\overline{X}})
\]
%
Esta   desigualdad   es   una   de  la   desigualdad   de   Petrovi\'c~\cite[43,
Teorema~8.7.1]{Kuc09}, $\phi(a +  b) \le \phi(a) + \phi(b)$  para $\phi$ concava
cancelando  en 0  (y la  conversa  en el  caso convexo),  conjuntamente con  $h$
creciente  (resp.   decreciente).   A parte  en  el  caso  de  Shannon y  el  de
Havdra-Charv\'at-Dar\'oczy, no  hay un  vinculo inmediato entre  $H(p_X)$ \  y \
$H(p_{\overline{X}})$.
\end{propiedadesPhi}

Se  conserva  la  superadditividad~\ref{prop:SZ:superaditividad}. De  hecho,  si
$\phi$ es convexa (resp. concava) con \ $\phi(0) = 0$, \ $\forall \: 0 \le a \le
1,  \: \phi(a  u)  = \phi(a  u  + (1-a)  0) \le  a  \phi(u)$ (resp.  desigualdad
reversa).  Entonces,   \  $\phi\left(  p_{X,Y}(x_i,y_j)   \right)  =  \phi\left(
  p_{X|Y}(x_i,y_j)  p_Y(y_j)  \right)  \le  p_{X|Y}(x_i,y_j)  \phi\left(p_Y(y_j)
\right)$, \ \ie \ $\sum_{i,j} \phi\left( p_{X,Y}(x_i,y_j) \right) \le \sum_{i,j}
p_{X|Y}(x_i,y_j) \phi\left(p_Y(y_j) \right) = \sum_i \phi\left(p_Y(y_j) \right)$
\  (resp.  desigualdad  reversa).  Se   cierra  la  prueba  con  la  decrecencia
(resp. crecencia) de $h$.


Sin  embargo, en  general, se  pierden las  propiedades \ref{prop:SZ:aditividad}
(additividad),     \SZ{ver      con     detalles     \ref{prop:SZ:subaditividad}
  (subadditividad) QINP prop. 11, }

\

La  definici\'on  de entropias  generalizadas  condicionales  aparece mucho  mas
problematico. Por  ejemplo, si  se define  a la Shannon,  es decir  definiendo \
$H_{(h,\phi)}(X|Y)$     \    tomando     \    $\sum_{y     \in     \Y}    p_Y(y)
H_{(h,\phi)}(p_{X|Y}(\cdot,y)$     \      se     pierde     la      regla     de
cadena~\ref{prop:SZ:cadena}. Como se lo ha visto,  en el marco de la entropia de
Havdra-Charv\'at-Dar\'oczy, se conserva la regla  de cadena si se remplaza $p_Y$
por su potencia  $p_Y^\beta$.  Sin embargo, generalizar este  esquema en el caso
general falla (la gracia en  Havdra-Charv\'at-Dar\'oczy viene de la propiedad de
morfismo de la  exponencial y del logaritmo). Como  consecuencia, generalizar la
noci\'on se vuelve problematico tambi\'en.  Por ejemple se pierde el diagrama de
Venn  aparte si  se  define la  entropia condicional  a  partir de  la regla  de
cadena. Pero en este caso, si la superadditividad garantiza la positividad de la
entropia             condicional,            se             pierde            la
propiedad~\ref{prop:SZ:independenciacondicional} por  perdida de la additividad,
y          por           consecuencia          la          propiedad          de
positividad/independencia~\ref{prop:SZ:Ipositive}  de  una  informaci\'on  mutua
construida sobre un  modelo diagrama de Venn. Veremos  en la secci\'on siguiente
que  un  tercero  camino  puede  ser usar  divergencia.   \SZ{ver  con  detalles
  \ref{prop:SZ:condicionar} (condiconar)}
%\SZ{VajVas85 pour la Schur-concavite}


\SZ{versiones diferenciales y ver \ref{prop:SZ:biyeccionC} (biyeccion), \ref{prop:SZ:permutacionC} (invarianza por rearreglo~\cite{toto})}

% ================================= Salicru

\subseccion{Divergencias y propiedades}
\label{sec:SZ:Czizar}

\SZ{ (1) Extension a la Renyi, (2) a  la HC/D/T, Cressie Reads, Cressie Pardo, Vajda;
(3) cf Burbea Rao:  (4) generalization Czizar Vajda, et voir avec h phi  avant meme Salicru. Cf
aussi  Bregmann, autres  de Csizsar  2012  versikon Bregman;  gupta Sharma  1976
BoeLub79, Vajda72, Salicru94 Orsak et  Paris; application a le test d'adequation
Pardo 99; MenMor97:5, Cf Pardo 2006 et ref.}

\SZ{FriSri08 pour Bregman; reapparition Fisher comme courbure, cf Varma, Jizba, MenMor97...}



\SZ{

Aplicaciones poner ac\'a  MaxEnt nous, Kesavan,  Kagan 63 A.M. \

   On the theory  of Fisher's
  amount  of  information Sov.  Math.  Dokl., 4  (1963),  pp.  991-993, etc,  la
  codificaci\'on a la Renyi (Cambell,  Hooda 2001, Bercher) 

\

 y la cuantificacion
  fina; EPI generalizada por Madiman, etc. Lutwak, Bercher etc., Kagan; Boeke 77
  An extension  of the Fisher information  measure I. Csisz\'ar,  P. Elias (Eds.),
  Topics   in  Information  Theory,   North-Holland,  Berlin/New   York  (1977),
  pp. 113-123  o Hammad o  Vajda 73 o  Ferentinos81 en el marco  Fisher; Kesavan
  gene MaxEnt}

\SZ{Revisite capacite a la Daroczy? codage; parler de la quantification fine et HCD}
