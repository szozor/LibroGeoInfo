\seccion{Introducci\'on}
\label{Sec:SZ:Introduccion}

% Photophone - Bell, ver Bru90 sec. 26 

La  noci\'on  de informaci\'on  encuentra  su origen  con  el  desarrollo de  la
 comunicaci\'on  moderna, por ejemplo  a trav\'es  del tel\'egrafo  siguiendo la
 patente de Morse en 1840. La idea  de asignar un c\'odigo (punto o barra, m\'as
 espacio entre letras y entre palabras)  a las letras del alfabeto es la semilla
 de  la  codificaci\'on  entr\'opica,  la  que se  basa  precisamente  sobre  la
 asignaci\'on  de un  c\'odigo a  s\'imbolos  de una  fuente (codificaci\'on  de
 fuente)  seg\'un  las  frecuencias  (o  probabilidad de  aparici\'on)  de  cada
 s\'imbolo en  una cadena.   De hecho,  el principio de  codificar un  mensaje y
 mandar la  versi\'on codificada  por un canal  de transmisi\'on es  mucho m\'as
 antiguo,  a pesar  de que  no hab\'ia  ninguna formalizaci\'on  matem\'atica ni
 siquiera expl\'icitamente una noci\'on de informaci\'on.  Entre otros, se puede
 mencionar el fotofone de A.  G.  Bell en 1880~\cite{Bel1880, Bru90} (sistema de
 comunicaci\'on  con luz),  el  tel\'egrafo \'optico  de  Claude Chappe  (1794),
 experimentos con luces por Guillaume Amontons  (en los a\~nos 1690 en Paris), o
 a\'un m\'as antiguamente la transmisi\'on de mensaje con antorchas en la Grecia
 antigua, con  humo por  los indios o  chiflando en  la prehistoria~\cite{Mon08}
 o~\cite[Cap.~3]{Arn01}.  Cada forma es  una instancia pr\'actica del esquema de
 comunicaci\'on de Shannon~\cite{Sha48, ShaWea64}, es decir la codificaci\'on de
 la informaci\'on, potencialmente  de la manera m\'as econ\'omica  que se puede,
 su   transmisi\'on  a   un  ``receptor''   (por  un   canal  ruidoso)   que  la
 interpreta/lee/decodifica.  Impl\'icitamente,  la noci\'on de  informaci\'on es
 al menos tan antigua como la humanidad.

A  pesar  de  que  la  idea  de codificar  y  transmitir  ``informaci\'on''  sea
tremendamente  antigua,  la  formalizaci\'on  matem\'atica  de  la  noci\'on  de
incerteza o  falta de  informaci\'on, \'intimamente vinculada  a la  noci\'on de
informaci\'on, naci\'o bajo  el impulso de Claude Shannon  y la publicaci\'on de
su   papel   seminal,   ``A    mathematical   theory   of   communication''   en
1948~\cite{Sha48},  o   un  a\~no  despu\'es  en  su   libro  re-titulado  ``The
mathematical  theory  of communication''  reemplazando  el  ``A''  (Una) por  un
``The''  (La). Desde  estos a\~nos,  las herramientas  de dicha  teor\'ia  de la
informaci\'on   dieron   lugar    a   muchas   aplicaciones   especialmente   en
comunicaci\'on~\cite[y  ref.]{CovTho06, Ver98, Gal01},  pero tambi\'en  en otros
campos  muy diversos  tal  como la  estimaci\'on  o la  discriminaci\'on~\cite[y
ref.]{CovTho06,      Kay93,       Bos07,      LehCas98},      la      inferencia
estad\'istica~\cite{Rob07,   Par06},   el   procesamiento   de  se\~nal   o   de
datos~\cite[y   Ref.]{PhiRou92,   EbeMol00,    Bas13},   en   ciencias   de   la
ingener\'ia~\cite{Arn01,    Kap89,    KapKes92,   PhiRou92},    f\'isica~\cite[y
Ref.]{Arn01, OhyPet93,  Mer18} entre  muchas otras (ver  por ejemplo  el esquema
pagina~2 de~\cite{CovTho06}).

La meta de este  cap\'itulo es describir las ideas y los  pasos dando lugar a la
definici\'on  de  la   entrop\'ia,  como  medida  de  incerteza   o  (falta  de)
informaci\'on.  En  este cap\'itulo, se  empieza con la  descripci\'on intuitiva
que  subyace  a  la  noci\'on  de  informaci\'on  contenida  en  una  cadena  de
s\'imbolos,  lo  que   condujo  a  la  definici\'on  de   la  entrop\'ia.   Esta
definici\'on  puede  ser  deducida  tambi\'en  de  un  conjunto  de  propiedades
``razonables''  que   deber\'ia  cumplir   una  medida  de   incerteza  (enfoque
axiom\'atico).   Se  continuar\'a  con  la  descripci\'on  de  tal  noci\'on  de
entrop\'ia, pasando del mundo discreto (s\'imbolos, alfabeto) al mundo continuo,
lo  que no es  trivial ni  siquiera intuitivo.   Se adelantar\'a  presentando el
concepto  de entrop\'ia condicional,  lo que  va a  dar lugar  a la  noci\'on de
informaci\'on  compartida entre  dos sistemas  o variables  aleatorias, concepto
fundamental en  el marco de la  transmisi\'on de informaci\'on o  de mensajes. A
continuaci\'on,  se  presentar\'a  la  noci\'on  de entrop\'ia  relativa  a  una
distribuci\'on de probabilidad de referencia, as\'i que el concepto de distancia
estad\'istica  o   divergencia  de  una   distribuci\'on  con  respecto   a  una
referencia. En  este cap\'itulo veremos  como estas medidas  informacionales son
entrelazadas  a trav\'es varias  identidades y  desigualdades, as\'i  que varias
relaciones  con medidas  del  mundo de  la  estimaci\'on. Al  final, se  dar\'an
ejemplos  y aplicaciones,  as\'i que  varias generalizaciones  de  las medidades
informacionales.

\

\modif{En todo  este cap\'itulo, hablaremos  de ``variable'' aleatoria,  que sea
escalar o multivariata (vector, matriz).}