% Letras caligraficas (alfabetos...)
% ----------------------------------
%
\def\A{\mathcal{A}} % Alfabeto
\def\B{\mathcal{B}} % Ley binomial, Bernoulli (caso especial)
\def\Be{\mathcal{B}e} % ley 
\def\C{\mathcal{C}}
\def\D{\mathcal{D}}
\def\Dir{\mathcal{D}ir} % Dirichlet
\def\E{\mathcal{E}} % funcion energia, 
\def\ED{\mathcal{ED}} % Elipticamente Distribuida
\def\CED{\mathcal{CED}} % Elipticamente Distribuida  Compleja
\def\DED{\mathcal{DED}} % Elipticamente Distribuida a la Derecha
\def\IED{\mathcal{IED}} % Elipticamente Distribuida a la Izquierda
\def\FED{\mathcal{FED}} % Fuertamente Elipticamente Distribuida
\def\F{\mathcal{F}} % Fuente (codificacion), transformada de Fourier
\def\FS{\mathcal{FS}} % transformada de Fourier-Stieltjes
\def\G{\mathcal{G}} % Ley Gamma, Ley Geometrica
\def\H{\mathcal{H}} % Ley hipergeometrica
\def\L{\mathcal{L}} % transformada de Laplace
\def\LS{\mathcal{LS}} % transformada de Laplace-Stieltjes
\def\M{\mathcal{M}} % Ley multinomial, transformada de Mellin
\def\MS{\mathcal{MS}} % transformada de Mellin-Stieltjes
\def\N{\mathcal{N}} % Normal o Gaussiana
\def\CN{\mathcal{CN}} % Normal Compleja
\def\O{\mathcal{O}}
\def\P{\mathcal{P}} % conjunto de k-uplet de matrices de Pd^+ sumando a la unidad
\def\R{\mathcal{R}} % Ley Student-R
\def\CR{\mathcal{CR}} % ley Student-r compleja
\def\Tri{\mathcal{T}} % Triangulo, ley Student-t
\def\T{\mathcal{T}} % Tensor
\def\CT{\mathcal{CT}} % ley Student-t compleja
\def\U{\mathcal{U}} % Uniforma
\def\W{\mathcal{W}} % Wishart
\def\X{\mathcal{X}} % espacio imagen X(Omega)
\def\Y{\mathcal{Y}} % espacio imagen Y(Omega)
\def\Z{\mathcal{Z}} % espacio imagen Z(Omega)
%
% letras frak
\def\PD{\mathfrak{P}} % conjunto de funciones definida positivas
\def\PDSI{\mathfrak{EP}} % conjunto de generadora de funciones eliptica definida positivas {\mathfrak{P}^{\mathrm{e}}}
\def\CM{\mathfrak{CM}} % conjunto de funciones completamente monotona
\def\cod{\mathfrak{c}}
\def\snr{\mathfrak{s}}
\def\perm{\mathfrak{S}}
%
% Conjuntos complejos, reales, enteros...
% --------------------------------------
%
\def\Bset{\mathbb{B}} % Bola
\def\Cset{\mathbb{C}} % Complejos
\def\Hset{\mathbb{H}} % Cuaterniones
\def\Kset{\mathbb{K}} % Cuerpo
\def\Nset{\mathbb{N}} % Enteros naturales
\def\Qset{\mathbb{Q}} % Racionales
\def\Rset{\mathbb{R}} % Reales
\def\Sset{\mathbb{S}} % Esfera
\def\SCset{\mathbb{SC}} % Esfera compleja
\def\Zset{\mathbb{Z}} % Enteros relativos
%
\newcommand{\Simp}[1]{\Delta_{#1}} % simplex
\newcommand{\Part}[1]{\mathrm{P}_{#1}} % partes 
%
%

% Conjuntos de matrices
% ------------------------------------------
%
\def\Mat{\mathrm{M}} % matrices
\def\TriS{\mathrm{T}^\triangle} % matrices triangular superior
\def\TriI{\mathrm{T}^\triangledown} % matrices triangular inferior
\def\Sim{\mathrm{S}} % matrices simetricas
\def\Her{\mathrm{H}} % matrices hermiticas
\def\Pos{\mathrm{P}} % matrices positiva definida
\def\Ort{\mathrm{O}} % matrices ortogonales
\def\Unit{\mathrm{U}} % matrices unitarias
\def\Sti{\mathrm{V}} % variedad de Stiefel
%
% reales e imaginarias
\newcommand{\real}[1]{\Re e \left\{ #1 \right\}}
\newcommand{\imag}[1]{\Im m \left\{ #1 \right\}}
%\def\imag{\Im m}
%
% otras letras
%\def\Rho{P}
%
% Operadores
% ----------
%
\def\optimes{\operatorname*{\bigtimes}}
%\def\otimeso{\otimes_{\mathrm{o}}} % producto externo
\def\Esp{\operatorname{E}} % esperanza matematica
\def\Var{\operatorname{Var}} % Varianza
\def\Cov{\operatorname{Cov}} % Covarianza
\def\pCov{\operatorname{pCov}} % Pseudo-covarianza
\def\Curt{\operatorname{Curt}} % Curtosis
\def\Excurt{\operatorname{ExCurt}} % Curtosis por excesso
\def\Asim{\operatorname{Asim}} % Asimetria {\operatorname{Asim}}
%\def\e{\operatorname{e}} % exponential con un e solo
\def\Jac{\operatorname{J}} % Jacobiana
\def\Hess{\mathcal{H}} % Hessiana
\def\Tr{\operatorname{Tr}} % Traza
\def\diag{\operatorname{diag}} % diagonal operador sobre una matriz
\def\Diag{\operatorname{Diag}} % diagonal operador sobre un vector
\def\det{\operatorname{det}} % determinente operador
%\def\Gauss{\mathcal{N}} % notacion de la gaussiana
\def\argmax{\operatorname*{argm\acute{a}x}} % operador argmax (se usa como el limite con \argmax_x)
\DeclareMathSymbol{\upimath}{\mathalpha}{operators}{"10}% para redefinir el \i en modo math 
\def\argmin{\operatorname*{argm\acute{\upimath}n}} % operador argmin (se usa como el limite con \argmin_x)
%
\def\un{\mathbbm{1}} % funcion indicador
\def\sign{\operatorname{sign}} % funcion signo
%\def\un{\mathbbm{0}} % vector 0
\def\arccos{\operatorname{arccos}} % arcoseno
\def\egald{\stackrel{\mathrm{d}}{=}} % igualdad en distribucion
\def\negald{\stackrel{\mathrm{d}}{\neq}} % diferente en distribucion
%\newcommand{\limitd}[1]{\mathop{\longrightarrow}_{#1}^{\mathrm{d}}}
\newcommand{\limitd}[1]{\xrightarrow[#1]{\mathrm{d}}}
\newcommand{\limitP}[1]{\xrightarrow[#1]{P}}
\newcommand{\limitcs}[1]{\xrightarrow[#1]{\mbox{c. s.}}}
\renewcommand{\vec}[1]{\operatorname{vec}\left(#1\right)} % vectorizacion
%
%
% letras en modo rm (opt, fa para Fano, etc)
% ------------------------------------------
%
\def\opt{\mathrm{opt}} % optimo
\def\sh{\mathrm{sh}} % Shannon
\def\fa{\mathrm{fa}} % Fano
\def\huf{\mathrm{huf}} % Huffman
\def\ren{\mathrm{r}} % Renyi
\def\hc{\mathrm{hc}} % Havrda-Charvat
\def\dar{\mathrm{d}} % Daroczy
\def\hcd{\mathrm{hcd}} % Havrda-Charvat-Daroczy
\def\id{\mathrm{id}} % identidad
%
\def\izq{\mathrm{i}}
\def\der{\mathrm{d}}
\def\col{\mathrm{c}}
\def\lin{\mathrm{l}}
\def\vcol{\mathrm{vc}}
\def\vlin{\mathrm{vl}}
\def\bil{\mathrm{b}}
\def\simed{\mathrm{s}}
%
%
% Pochhammer creciente y decreciente
\newcommand{\PocC}[2]{\left(#1\right)^{\overline{#2}}}
\newcommand{\PocD}[2]{\left(#1\right)^{\underline{#2}}}
% Coefficiente binomial
\newcommand{\bino}[2]{\Big( \hspace{-1mm} \protect\begin{array}{c} #1 \\[-2.5mm] #2 \protect\end{array} \hspace{-1mm} \Big)}
%\newcommand{\bino}[2]{\begin{pmatrix} \! #1 \!\\[-1.5mm] \! #2 \! \end{pmatrix}}
%\newcommand{\bino}[2]{\begin{psmallmatrix} \! #1 \! \\ \! #2 \!\vspace{.25mm} \end{psmallmatrix}}
\newcommand{\smallbino}[2]{\begin{psmallmatrix} \! #1 \! \\ \! #2 \!\vspace{.25mm} \end{psmallmatrix}}
% funcion hipergeometrica
\newcommand{\hypgeom}[2]{\mbox{}_{#1}F_{#2}}
%
\newcommand{\widebar}[1]{\mkern 1.5mu\overline{\mkern-1.5mu#1\mkern-1.5mu}\mkern 1.5mu}
\def\sumint{\mathrel{\ooalign{\hss$\displaystyle\int$\hss\cr$\displaystyle\sum$}}}
%    \setbox0{\int}%
%    \rlap{\hbox to \wd0{\hss?\hss}}\box0}
%{\rlap{#1}{\hss #2 \hss}}
%
% Entropia y divergencias
% -----------------------
%
% Kullback-Leibler
\newcommand{\Dkl}[2][]{
%D_{\mathrm{kl}}\ifthenelse{\isempty{#1}}{}{\left( \left. #1 \right\| #2 \right)}
D_{\mathrm{kl}}\ifthenelse{\isempty{#1}}{}{\left( #1 \left\| #2 \right. \right)}
}
%
% H_\phi
\newcommand{\hphi}[1][]{
H_\phi\ifthenelse{\isempty{#1}}{}{\left( #1 \right)}
}
%
% H_{(h,\phi)}
\newcommand{\hhphi}[1][]{
H_{(h,\phi)}\ifthenelse{\isempty{#1}}{}{\left( #1 \right)}
}
%
% Csizar D_\phi^{c}
\newcommand{\cphi}[2][]{
D^{\mathrm{c}}_{\phi}\ifthenelse{\isempty{#1}}{}{\left( \left. #1 \right\| #2 \right)}
}
%
% Csizar D_{h,\phi}^{c}
\newcommand{\chphi}[2][]{
D^{\mathrm{c}}_{(h,\phi)}\ifthenelse{\isempty{#1}}{}{\left( \left. #1 \right\| #2 \right)}
}
%
% Jensen-Shannon
\def\Djs{D_{\mathrm{js}}^\pi}
%
% Jensen D_\phi^{j,pi}
\newcommand{\jphi}[2][]{
D^{\mathrm{j},\pi}_\phi\ifthenelse{\isempty{#1}}{}{\left( #1 , #2 \right)}
}
% Jensen D_{h,\phi}^{j,pi}
\newcommand{\jhphi}[2][]{
D^{\mathrm{j},\pi}_{(h,\phi)}\ifthenelse{\isempty{#1}}{}{\left( #1 , #2 \right)}
}
%
% Bregman D_\phi^b
\newcommand{\bphi}[2][]{
D^{\mathrm{b}}_\phi\ifthenelse{\isempty{#1}}{}{\left( \left. #1 \right\| #2 \right)}
}
% Bregman D_{h,\phi}^b
\newcommand{\bhphi}[2][]{
D^{\mathrm{b}}_{(h,\phi)}\ifthenelse{\isempty{#1}}{}{\left( \left. #1 \right\| #2 \right)}
}
%
% abreviaciones ej. en lugar de {\it i.e.,} \ie, etc.
% ---------------------------------------------------
%
\def\od{\o} % o danese!
\def\ie{{\it i.~e.,}\xspace}
\def\tq{\:\: \big| \:\:}
\def\ou{\:\: \vee \:\:}
\def\et{\:\: \wedge \:\:}
%\def\tq{\mbox{\: tal que \:}}
\def\;{\, ; \,} % separacion en los intervalos
%
%
% environment para las notaciones
% -------------------------------
%
\newenvironment{notation}[1]
{
\tablefirsthead{%
   %\hline
   \multicolumn{2}{l}{\large\bf #1 \vspace{5mm}}\\
}
%\tablefirsthead{{\small\sl continued from previous page}
%   \hline}
\tabletail{\hline}
%\renewcommand{\arraystretch}{2}
%\begin{longtable}
\begin{supertabular}
{
|>{\vspace{-2mm}}p{.12\textwidth}
|>{\vspace{-2mm}}p{.88\textwidth}
|
}
\hline
}
%
{
\\[2.5mm]
\end{supertabular}\vspace{2cm}
%\end{longtable}
}


% environment para los ejemplos de VA
% ------------------------------------
%
\newenvironment{caracteristicas}
{
\tablefirsthead{%
   %\hline
%   \multicolumn{2}{l}{\large\bf #1 \vspace{5mm}}\\
   \multicolumn{2}{l}{\large\bf \vspace{5mm}}\\
}
%\mbox{ }\newline
%\renewcommand{\arraystretch}{2}
%\begin{longtable}
\tabletail{\hline}
\begin{supertabular}
{
|>{\vspace{-2mm}}p{.275\textwidth}|
>{\vspace{-2mm}}p{.675\textwidth}|
}
\hline
}
%
{
\\[2.5mm]
\end{supertabular}\newline
%\end{longtable}
}
