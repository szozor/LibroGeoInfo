\capitulo{Nociones de theor\'ia de la informaci\'on}{Steeve Zozor}

% Epigrafe de capitulo
\begin{epigrafe}
  \textcolor{red}{Esto es  un ep\'igrafe con  texto simulado.\\ Esto es  un ep\'grafe
    con  texto simulado.
\autortituloepigrafe{Autor del ep\'igrafe, T\'itulo de la obra}
}
\end{epigrafe}


% =============================== Introduccion =============================== %

\seccion{Introducci\'on}
\label{s:SZ:Introduccion}

La noci\'on de informaci\'on encuentra su origen, de una forma, con el desarollo
de la comunicaci\'on  moderna, por ejemplo a trav\'es  del telegrafo (patente de
Moorse, en 1840, aun que los principios  datan de los a\~nos 1690). De hecho, la
idea de  asignar un codig\'o  (punto o  barra) a las  letras del alfabeto  es la
semilla  de la  codificaci\'on  entropica,  que se  basa  precisamente sobre  la
asignaci\'on de  un codigo a simbolos  de una fuente  (codificaci\'on de fuente)
basados a las frequencias (o probabilidad de aparici\'on) de cada simbolo en una
cadena. La formalizaci\'on de la  noci\'on de incerteza, intimamente vinculado a
la de informaci\'on, naci\'o bajo el impulso de C. Shannon y la publicaci\'on de
su   papel   seminal,   ``A    mathematical   theory   of   communication''   en
1948~\cite{Sha48}. Desde esto a\~nos, los  herramientas de la dicha teoria de la
informaci\'on dio lugar  a muchas aplicaciones, en campos  muy diversos tal como
{\bf completar con ref}.

La meta de este capitulo es de describir  las ideas y los pasos dando lugar a la
definicion de la entropia, como medida de incerteza o informaci\'on. {\bf Intuition, axiomas, propiedades. }

% ================================= Entropia ================================= %

\seccion{Entrop\'ia como medida de incerteza}
\label{s:SZ:Entropia}

% ================================= Axiomas

\subseccion{Axiomas, entrop\'ia de Shannon, propiedades}
\label{ss:SZ:Axiomas}

Un de los primeros trabajos  tratando de formalizar la noci\'on de informaci\'on
de una  cadena de simbolos es dado  a Raph Hartley~\cite{Har28}. En  su papel de
1928,  Hartley   defin\'o  la  informaci\'on   de  una  secuencia   como  siendo
proporcional a  su longitud. Mas precisamente,  para simbolos de  un alfabeto de
cardinal  $s$, exiten  $s^n$ cadenas  diferences  de longitud  $n$, defin\'o  la
informaci\'on de  tales cadenas  como siendo $K  n$.  Para ser  consistente, dos
ensembles  de mismo  tamanio $s_1^{n_1}  = s_2^{n_2}$  deben llegar  a  la misma
informaci\'on,  as�  que  la informaci\'on  de  Hatley  es  definida como  $H  =
\log\left( s^n \right)$ donde la base del logaritmo es arbitraria. Dicho de otra
manera, tomando un  logaritmo de base 2, esta informaci\'on es  nada mas que los
numeros de bits  (0-1) necesarios para codificar todas  las cadenas de longitude
$n$ de simbolos de un alpfabeto de cardinal $s$.

Una  debilidad  del  enfoque  de   Hartley  es  que,  de  una  forma,  considera
implicitamente que en un mensage, cada  cadena de longitud puede aparecer con la
misma  frecuencia, o  probabilidad  $1/s^n$, siendo  la  informaci\'on menos  el
logaritmo de estas probabilidades. A contrario, parece mas logico considerar que
secuencias muy frecuentes no llevan  mucha informaci\'on (se sabe que aparecen),
mientras  que las  que aparencen  raramente  llevan mas  informaci\'on (hay  mas
sorpresa, mas  incerteza en observarla). Volviendo los  simbolos elemental, $x$,
vistos  como aleatorios,  la  informaci\'on  o incerteza  va  a ser  intimamente
vinculada a la probabilidad de aparici\'on de estos simbolos $x$.

\begin{definition}[Entropia de Shannon~\cite{Sha48}]
\end{definition}

% ================================= Axiomas

\subseccion{Entrop\'ia diferencial}
\label{ss:SZ:Diferencial}





% ================================= Entropia ================================= %

Este es un  p\'arrafo Normal con texto simulado, (Arial  10, interlineado de 1,5
l\'ineas, sangr\'ia en primera l\'inea de 0,5cm. Este es un p\'arrafo Normal con
texto simulado,  (Arial 10, interlineado  de 1,5 l\'ineas, sangr\'ia  en primera
l\'inea de 0,5cm). Este es un  p\'arrafo Normal con texto simula- do, (Arial 10,
interlineado de 1,5 l\'ineas, sangr\'ia en primera l\'inea de 0,5cm). Este es un
p\'arrafo Normal  con texto simulado,  (Arial 10, interlineado de  1,5 l\'ineas,
sangr\'ia en primera  l\'inea de 0,5cm).  Este es un  p\'arrafo Normal con texto
simulado, (Arial 10, interlineado de  1,5 l\'ineas, sangr\'ia en primera l\'inea
de 0,5cm)\notaalpie{Eso es una footnote sobre varias lineas. Eso es una footnote
  sobre varias  lineas.  Eso  es una  footnote sobre varias  lineas. Eso  es una
  footnote sobre varias lineas. Eso es  una footnote sobre varias lineas. Eso es
  una footnote sobre varias lineas. Eso es una footnote sobre varias lineas.}.

% citas (de mas de 40 palabras)) ==============================================
\begin{citas}
  {\color{red} Esto  es un ejemplo  de cita  de mas de  40 palabras. Esto  es un
    ejemplo de cita de mas de 40 palabras.  Esto es un ejemplo de cita de mas de
    40 palabras. Esto  es un ejemplo de cita  de mas de 40 palabras.  Esto es un
    ejemplo de cita de mas de 40 palabras.}
\end{citas}

Este es un  p\'arrafo Normal con texto simulado, (Arial  10, interlineado de 1,5
l\'ineas, sangr\'ia en primera l\'inea de 0,5cm. Este es un p\'arrafo Normal con
texto simulado,  (Arial 10, interlineado  de 1,5 l\'ineas, sangr\'ia  en primera
l\'inea de 0,5cm). Este es un  p\'arrafo Normal con texto simula- do, (Arial 10,
interlineado de 1,5 l\'ineas, sangr\'ia en primera l\'inea de 0,5cm). Este es un
p\'arrafo Normal  con texto simulado,  (Arial 10, interlineado de  1,5 l\'ineas,
sangr\'ia en  primera l\'inea de 0,5cm).  Este es un p\'arrafo  Normal con texto
simulado, (Arial 10, interlineado de  1,5 l\'ineas, sangr\'ia en primera l\'inea
de 0,5cm).


\begin{figure}[h!]
\centerline{\includegraphics[width=2cm]{logo_large}}
%
\leyenda{Eso es  una figura, con  su leyenda sobre  varias lineas para  ver como
  queda en el texto. Eso es una  figura, con su leyenda sobre varias lineas para
  ver como queda en el texto.}
\end{figure}

Este es un  p\'arrafo Normal con texto simulado, (Arial  10, interlineado de 1,5
l\'ineas, sangr\'ia en primera l\'inea de 0,5cm. Este es un p\'arrafo Normal con
texto simulado,  (Arial 10, interlineado  de 1,5 l\'ineas, sangr\'ia  en primera
l\'inea de 0,5cm). Este es un  p\'arrafo Normal con texto simula- do, (Arial 10,
interlineado de 1,5 l\'ineas, sangr\'ia en primera l\'inea de 0,5cm). Este es un
p\'arrafo Normal  con texto simulado,  (Arial 10, interlineado de  1,5 l\'ineas,
sangr\'ia en  primera l\'inea de 0,5cm).  Este es un p\'arrafo  Normal con texto
simulado, (Arial 10, interlineado de  1,5 l\'ineas, sangr\'ia en primera l\'inea
de 0,5cm).

\begin{table}
\leyenda{Eso es un ejemplo de tabla}
\begin{center}
\begin{tabular}{|c|c|c|}
\hline
{\bf T\'itulo (negrita)} & {\bf T\'itulo (negrita)} & {\bf T\'itulo (negrita)}\\
\hline
A & Texto simulado (normal) & Texto simulado (normal)\\
\hline
B & Texto simulado (normal) & Texto simulado (normal)\\
\hline
\end{tabular}
\fuente{Eso ser\'ia el fuente de la tabla}
\end{center}\end{table}


\

Ejemplo con respeto al capitulo~\ref{cap:teoriaprobabilidades}

Para ver que las referencias de capitulos andan:~\ref{cap:teoriaprobabilidades};
que      las       de      secciones      tambi\'en~\ref{sec:var_aleat},      de
subsecciones~\ref{subsec:var_aleat_sub},                                       de
subsubsecciones~\ref{subsubsec:var_aleat_subsub}, de figuras~\ref{fig:figura}, y
de tablas~\ref{tab:tabla}.

Eso es una cita, para ver como queda~\cite{CovTho06, AmaNag00}.


