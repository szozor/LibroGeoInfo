\seccion{Esperanza, momentos y funciones generadoras}
\label{s:MP:esperanzamomento}


%%%%%%%%%%%%%%%%%%%%%%%%%%%%%%%%%%%%%%%%%%%%%%%%%%%%%%%%%%%%%%%%%%%%%%%%%%

\aver{\emph{introducci\'on...}} %%


% ================================= Media

\subseccion{\modif{Media de un vector aleatorio}}
\label{sec:MP:VectorMedio}


Una  variable aleatoria  $X$  tiene asociado  un  {\it promedio}  o {\it  media}
(tambi\'en  llamado   {\it  valor  esperado  o  de   expectaci\'on  o  esperanza
  matem\'atica})  que  se obtiene  pesando  cada  valor  de \modif{$X$}  con  la
\modif{medida de} probabilidad asociada a ese valor,
%
\modif{
\begin{definicion}[Media o valor/vector medio]
  Formalmente, la media de  una variable aleatoria $X$ \underline{integrable} es
  definida por
  %
  \[
  \Esp[X] = \int_\Omega X(\omega) dP(\omega)
  \]
  %
  Claramente,  por  el  teorema de  la  medida  imagen,  esta media  se  escribe
  tambi\'en a partir de la medida de probabilidad $P_X$ como
  %
  \[
  \Esp[X] = \int_\Rset x \, dP_X(x)
  \]
  %
  En  el caso vectorial  $d$-dimensional, hay  que entender  la media,  o vector
  medio, como  un vector de componentes  $i$-\'esima la media  $\Esp[X_i]$ de la
  componente $i$-\'esima $X_i$ de $X$, dando
  %
  \[
  \Esp[X] = \int_{\Rset^d} x \, dP_X(x)
  \]
  %
  A veces,  se encuentra  tambi\'en la notaci\'on  \ $\langle  x \rangle$ \  o \
  $\langle  x  \rangle_{p_X}$  \  para  el  valor  medio,  especialmente  en  la
  literatura de f\'isica.
\end{definicion}
%
La secunda formulaci\'on  del valor medio se proba  sencillamente, empezando por
$X = \un_A$ para unos $A$.   Entonces $P_X = (1-P(A)) \delta_0 + P(A) \delta_1$.
Luego  $\displaystyle \int_\Omega  \un_A(\omega)  dP(\omega) =  P(A) =  (1-P(A))
\times 0 +  P(A) \times 1 = \int_\Rset  x dP_X(x)$.  Se cierra la  prueba con el
teorema~\ref{Th:MP:MedibleLimite} dando cualquier  funci\'on medible como limite
de funciones escalonadas,  y por la definici\'on~\ref{Def:MP:IntegracionReal} de
la integral de cualquier funci\'on medible.

Luego,   de    la   distribuci\'on   marginal    $\displaystyle   P_{X_i}(B)   =
\int_{\Rset^{i-1}   \times   B   \times   \Rset^{d-i}}  dP_X(x)$,   se   obtiene
$\displaystyle  \Esp[X_i] = \int_{\Rset^d}  x_i \,  dP_X(x)$, dando  la \'ultima
formulaci\'on en el caso vectorial.


Fijense de que }
%$p(x)\,dx$, e integrando sobre el rango permitido de $x$:
%$$
%E[X] = \langle x\rangle = \int_{\Omega} x \ p(x)\,dx \equiv \mu
%$$
%si  la  integral  existe.  La  \emph{esperanza} de  la  variable  aleatoria  $X$
%representa  el valor  medio  que puede  tomar  entre todos  los  eventos de  una
%prueba. U
una   variable   aleatoria   $X$   se   dice   integrable   cuando   $E[|X|]   <
\infty$. \modif{De  la misma manera, un  vector aleatorio admite una  media si y
  solamente si cada componente es integrable. Veremos m\'as adelante que existen
  variables aleatorias que no admiten una media.

  M\'as  all\'a de  la  formulaci\'on matem\'atica  de  la media  \ $\Esp[X]$  \
  representa   la  posici\'on  alrededor   de  la   cual  se   ``distribuye  las
  probabilidades de  occurencia''. Es el equivalente  probabil\'istico de centro
  de gravedad o barycentro en meca\'anica.

  En el  caso de variables  aleatorias discretas, de  soporte $\X = \{  x_i \}$,
  inmediatamente
%
\[
\Esp[X] = \sum_i x_i P_X(\{x_i\}) = \sum_i x_i P(X = x_i)
\]
%
\noindent  Fijense de  que $\Esp[X]$  no partenece  necesariamente a  $\X$.  Por
ejemplo para $X$ uniforme sobre $\X = \{ 1 \, , \, 3 \, , \, 7 \}$, \ie $\forall
\, i \in \X, \quad P(X = i) = \frac13$. Se calcula $\Esp[X] = 1 \times \frac13 +
3 \times  \frac13 + 7 \times \frac13  = \frac{11}{3} \not\in \X$.  Tampoco es el
promedio de los valores extremos.

Un  ejemplo de  variable aleatoria  discreta  no integrable  es dada  por $\X  =
\Nset^*$   con   $P(X   =    n)   =   \frac{6}{\pi^2   \,   n^2}$.   Claramente,
$\sum_n\frac{6}{\pi^2 \, n}$ diverge, as\'i que $X$ no tiene una media.

En el caso de vectores  aleatorios continuos, obtenemos la expresi\'on siguiente
de la media (o vector medio):
%
\[
\Esp[X] = \int_{\Rset^d} x \, p_X(x) \, dx
\]
%
\noindent  Un ejemplo  de  no  teniende de  media  es dado  en  el  caso de  una
distribuci\'on de Cauchy-Lorentz (ver  m\'as adelante) \ $\displaystyle p_X(x) =
\frac{\alpha}{\left( 1 + x^t x  \right)^{\frac{d+1}{2}}}$ \ donde $\alpha$ es un
factor de normalizaci\'on.

En el caso general, para calcular  la media, hay que pasar por la distribuci\'on
$P_X$.   Por ejemplo\label{Ej:MP:EspMixta},  volvemos al  la variable  mixta del
ejemplo pagina~\pageref{Ej:MP:Mixta}, $X  = V \, \un_{U <  \frac12} + \un_{U \ge
  \frac12}$  \ con  \  $U$ \  y  \ $V$  variables  aleatorias independientes  de
distribuci\'on uniformas sobre  $[0 \, ; \,  1)$, \ie $p_U(x) = \un_{[0  \, ; \,
  1)}(x)$.   De \  $X \in  B \:  \Leftrightarrow \:  \left( \left(  U  < \frac12
  \right) \cap  \left( V \in B  \right) \right) \,  \cup \, \left( \left(  U \ge
    \frac12 \right) \cap \left( 1 \in B \right) \right)$, \ del hecho de que los
eventos de la  uni\'on son incompatibles y  de la independencia de $U$  y $V$ (o
saliendo de la funci\'on de  repartici\'on), se obtiene $P_X(B) = \frac12 P_V(B)
+ \frac12  \delta_1(x)$.  A continuaci\'on,  \ $\displaystyle \Esp[X]  = \frac12
\int_\Rset dP_V(x) + \frac12 \int_\Rset d\delta_1(x) = \frac12 \int_\Rset p_V(x)
\, dx + \frac12 \times 1 = \frac12 \int_0^1 dx + \frac12 = \frac34$.

\

Una nota intersante es  de que, en el caso escalar, si \  $X \ge 0$ \ admitiendo
una media, se obtiene
%
\[
\Esp[X]  = \int_{\Rset_+} P(X  > t)  \, dt  = \int_{\Rset_+}  \left( 1  - F_X(t)
\right) \, dt
\]
%
Se proba saliendo  de \ $\displaystyle x = \int_0^x  dt = \int_{\Rset_+} \un_{(t
  \, ;  \, +\infty)}(x)  \, dt$  \ dando \  $\displaystyle \Esp[X]  = \int_\Rset
\left( \int_{\Rset_+}  \un_{(t \, ; \,  +\infty)}(x) \, dt \right)  \, dP_X(x) =
\int_{\Rset_+} \left( \int_\Rset \un_{(t \, ; \, +\infty)}(x) \, dP_X(x) \right)
\,    dt$    \    por    el    teorema    de    Fubini    Th.~\ref{Th:MP:Fubini}
pagina~\pageref{Th:MP:Fubini}. Se  cierra la  prueba observando que  la integral
interior es nada  m\'as que $P(X > t)$.   En el caso discreto con  $\X = \Nset$,
viene  inmediatamente \ $\displaystyle  \sum_{n \in  \Nset} P(X>n)$  que podemos
probar directamente  saliendo de $P(X  = n) =  P(X>n)- P(X>n-1)$. En el  caso de
variable   admitiendo  una   densidad,   se  obtiene   tambi\'en  haciendo   una
integraci\'on  por   partes~\footnote{El  el   casi  discreto,  hay   que  tener
  precauciones separando la series de una diferencia de terminos. En el caso $X$
  continuo admitiendo una densidad, hay  que estudiar bien el coomportamiento de
  $t \mapsto t (1-F_X(t))$ al infinito.}.

\

Terminamos  esta  secci\'on con  la  propiedad  de  linealidad de  la  esperanza
matem\'atica  $\Esp$, como consecuencia  de la  linealidad de  la integraci\'on:
para cualquieras matrices  $C$ \ y \ $D$ \ dadas  de dimensiones compatibles con
las de \  $X$ \ e \  $Y$ \ (incluyendo el caso  escalar), $$\Esp[C X + D  Y] = C
\Esp[X] + D \Esp[Y]$$
%


}


% ================================= Momentos
\modif{
\subseccion{Momentos de un vector aleatorio}
\label{sec:MP:Momentos}
}

Si $X$ es una variable aleatoria, para cualquier funci\'on medible $f$, \ $f(X)$
tambi\'en  lo  es.  \modif{se  puede   entonces  definir  su  valor  medio},  si
existe. \modif{A pesar de necesitar evaluar la distribuci\'on de probabilidad de
  $Y = f(X)$, el valor medio se calcula a partir de la de $X$,
%
\begin{teorema}[Teorema de transferencia]
  Sea  \ $X$  \ un  vector  aleatorio $d$-dimensional  y \  $f: \Rset^d  \mapsto
  \Rset^{d'}$ \ una funci\'on medible tal que $f(X)$ sea integrable. Entonces
  %
  \[
  \Esp\left[   f(X)  \right]   =  \int_\Omega   f(X(\omega))  \,   dP(\omega)  =
  \int_{\Rset^{d'}} f(x) \, dP_X(x)
  \]
  %
  En particular, en el caso $\X = X(\Omega)$ discreto se obtiene
  %
  \[
  \Esp\left[ f(X) \right] = \sum_i f(x_i) P(X = x_i)
  \]
  %
  y para $X$ continuo admitiendo una densidad de probabilidad
  %
  \[
  \Esp\left[ f(X) \right] = \int_{\Rset^d} f(x) \, p_X(x) \, dx
  \]
\end{teorema}
%
\begin{proof}
  Sea $B \in \B(\Rset^d)$ \ y consideramos \ $\phi(x) = \un_B(x)$. Entonces, $\Y
  = \{  0 \, , \,  1 \}$ y inmediatamente  $$P_Y = P_X(B) \, \delta_1 + (1-P_X(B))
  \, \delta_0$$ Entonces
  %
  \[
  \Esp[f(X)]  =  \int_\Rset  P_X(B)  \,  d\delta_1 +  \int_\Rset  (1-P_X(B))  \,
  d\delta_0 = P_X(B) = \int_{\Rset^d} \un_B(x) \, dP_X(x)
  \]
  %
  En el caso  $d' = 1$, para $f  \ge 0$, se cierra entonces la  prueba usando el
  teorema~\ref{Th:MP:MedibleLimite}   pagina~\pageref{Th:MP:MedibleLimite}  para
  ver $f$ como l\'imite creciente  de una sucesi\'on de funciones escalonadas, y
  la definici\'on Def.~\ref{Def:MP:IntegracionReal} de la integraci\'on real. El
  caso $d' > 1$ es nada mas que $d' = 1$, componente a componente.
\end{proof}
%
Hablamos de {\it momentos} de la variable aleatoria $X$. Los momentos relevantes
en general son los siguientes:
%
\begin{itemize}
\item 
\end{itemize}
}

\aver{
En  general,  si $X$  es  una  variable  aleatoria, cualquier  funci\'on  $f(X)$
tambi\'en lo es, y su valor de expectaci\'on, si existe, est\'a dado por
$$
E[f(X)] = \langle f(x)\rangle = \int_{\Omega} f(x) \ p(x)\,dx
$$
En  particular, para  el monomio  $f(x)=x^r$ siendo  $r\in\Nset$, se  obtiene el
$r$\emph{-\'esimo momento (ordinario)} de $X$:
$$
\nu_r \equiv E[X^r] = \langle x^r\rangle = \int_{\Omega} x^r \ p(x)\,dx
$$
que tiene unidades  de $X^r$. Se puede incluir el caso  $r=0$, que corresponde a
la condici\'on de normalizaci\'on:  \ $\nu_0=\int_{\Omega}p(x) \,dx=1$. La media
es  el primer  momento: $\nu_1=\langle  x\rangle=\mu$. Es  f\'acil probar  que \
$\langle x^2 \rangle \geq \langle x \rangle^2$.
%%ejerc
T\'ipicamente, los primeros  momentos son m\'as relevantes que  los de \'ordenes
mayores, para la caracterizaci\'on de una distribuci\'on.
%% Analog\'ia con Mec\'anica ...

Por  ejemplo,  para  la   distribuci\'on  uniforme  $p(x)=\frac{1}{b-a}$  en  el
intervalo  $[a,b]$,  resulta:  \  $\nu_1=\langle  x\rangle=\frac  12  (b+a)$,  \
$\nu_2=\langle       x^2\rangle=\frac       13      (b^2+ab+b^2)$,       \ldots,
$\nu_r=\frac{b^{r+1}-a[{r+1}}{(r+1)(b-a)}$.
%%Simplificar. Ejercicio para otra distrib ?

Cuando una pdf $p(x)$  tiene soporte (semi)infinito, necesariamente la funci\'on
$p$ debe tender a 0  cuando $|x|\rightarrow\infty$.  Si $p(x)$ es \emph{de largo
  alcance}, en  el sentido de  que no cae  a 0 suficientemente r\'apido  con $x$
para  $x$  grandes,  algunos  momentos   pueden  no  existir.  Por  ejemplo,  la
distribuci\'on   de   probabilidad    de   Cauchy--Lorentz   (o   funci\'on   de
Breit--Wigner), dada  por $p(x)=\frac{\gamma}{\pi} \frac{1}{\gamma^2+(x-x_0)^2}$
para  $x\in(-\infty,\infty)$, con $\gamma>0$  y $x_0$  fijos, no  tiene momentos
finitos de orden $r\geq 1$.
%%...

%%Figura: Lorentzianas  centradas en x_0=0, con gamma=1  (forma est\'andar), .5,
%% 2, ...

\hfill

En  el  caso  de  una  variable  aleatoria discreta  $X$  que  toma  valores  en
$\Omega=\{x_1,  \ldots, x_N\}$, la  esperanza de  la variable  viene dada  por \
$E[X] = \sum_{n=1}^N x_n \,  p(x_n)$.  Consideraremos que el espacio muestral es
$\Nset$,
 %%incluyendo n=0
luego resulta
$$
E[X] = \langle n \rangle =  \sum_{n\geq 1} n \, p_n  , 
$$
que se puede obtener tambi\'en como \ $E[X]=\sum_{j=0}^{\infty}
\Pr(X>j)$.
%%ejercicio. Prueba de convergencia: ver O.Fran�ois, Notes de cours, p.55
Para una funci\'on $f$ definida sobre el conjunto $\{0,1,2,\ldots\}$ se tiene
$$
E[f(X)] =  \langle f(n) \rangle = \sum_{n\geq 0} f(n) \, p_n  , 
$$
y se define el $r$-\'esimo momento (ordinario) de $n$ como 
$$
\nu_r \equiv E[X^r] = \langle n^r\rangle = \sum_{n=1}^\infty n^r \, p_n  .
$$
En el  caso de variables discretas  sobre $\Nset$, resulta  \'util introducir el
$r$-\'esimo \emph{momento factorial} de $n$ mediante
$$
\langle n^{(r)} \rangle \equiv \langle n (n-1) \cdots [n-(r-1)] \rangle =
 \sum_{n=r}^\infty n (n-1) \cdots (n-r+1) \, p_n  .
$$

\hfill

Los \emph{momentos centrales} se definen alrededor de $x=\langle x\rangle$, como
el valor de expectaci\'on de potencias de la \emph{desviaci\'on} $\Delta x\equiv
x-\langle x\rangle$:
$$
\mu_r  \equiv \langle (x-\langle  x\rangle)^r\rangle =  \int_{\Omega} (x-\langle
x\rangle)^r \ p(x)\,dx .
$$
Se deduce que si la densidad de probabilidad $p(x)$ es una funci\'on sim\'etrica
respecto  a  la  media,  entonces  todos  los  momentos  centrales  impares  son
nulos.   Los   momentos   centrales   brindan  medidas   que   caracterizan   la
distribuci\'on:
\begin{enumerate}
\item el primer momento central es id\'enticamente nulo para toda pdf:
$$
\mu_1 = \langle x-\langle x\rangle \rangle = 0 ; 
$$
%
\item   el   segundo   momento   central   se   conoce   como   \emph{varianza},
  \emph{dispersi\'on} o tambi\'en \emph{desviaci\'on cuadr\'atica media}:
\begin{equation}
\label{mu2}
\mu_2 = \langle (x-\langle x\rangle)^2 \rangle = \langle x^2\rangle - \langle
x\rangle^2 = \Var(X) \equiv \sigma^2 ,
\end{equation}
y es una medida del cuadrado del ancho  efectivo de una pdf, es no negativo y se
anula s\'olo cuando $p(x)=\delta(x)$, esto  es, cuando no hay incerteza sobre el
resultado.  La varianza est\'a bien definida si $X$ es una variable aleatoria de
cuadrado integrable, esto es, cuando  $E[X^2] < \infty$.  El \emph{ancho} de una
distribuci\'on  est\'a dado por  la \emph{desviaci\'on  est\'andar} \  $\sigma =
\sqrt{\mu_2}$, tiene  las mismas unidades de  $X$, y se usa  para normalizar los
momentos centrales  de orden superior.  El \emph{ancho relativo} es  otra medida
que caracteriza  la distribuci\'on, dado por  $\frac{\sigma}{\langle x\rangle} =
\sqrt{\frac{\langle x^2\rangle}{\langle x\rangle^2}-1}$ cuando $\langle x\rangle
\neq 0$;
%
\item  el  tercer  momento  central  permite  definir  el  \emph{coeficiente  de
    asimetr\'ia}:
  $$
  \alpha_3 \equiv \frac{\mu_3}{\sigma^3} , 
  $$ 
  que resulta adimensional y puede tener signo positivo o negativo, anul\'andose
  para distribuciones que son sim\'etricas respecto del valor medio;
%
\item el cuarto momento central da lugar a la \emph{curtosis}:
  $$
  \alpha_4 \equiv \frac{\mu_4}{\sigma^4} , 
  $$ 
  que  posibilita  diferenciar  entre   distribuciones  altas  y  angostas  (con
  $\alpha_4<3$), de otras bajas y anchas (con $\alpha_4>3$)
\end{enumerate}

\hfill

La relaci\'on entre los momentos  centrales y los momentos ordinarios se obtiene
directamente de las definiciones:
$$
\mu_r = \int (x-\langle x \rangle)^r \, p(x) \, dx = \sum_{s=0}^r
\binom{r}{s}
%% {r \choose s}
(-\langle x  \rangle)^{r-s} \int x^s \,  p(x) \, dx  = \sum_{s=0}^r \binom{r}{s}
\nu_s (-\nu_1)^{r-s}
$$
para    cualquier   $r=1,2,\ldots$,   siendo    $\nu_0=1$.   Por    ejemplo,   \
$\mu_2=\nu_2-\nu_1^2$   como    en   la   Ec.~\eqref{mu2},    mientras   que   \
$\mu_3=\nu_3-3\nu_1\nu_2+2\nu_1^3$.
%% ejerc

\hfill

Dada una variable  aleatoria $X$ con una distribuci\'on  de probabilidad $p(x)$,
teniendo en cuenta que los dos primeros momentos dan las caracter\'isticas m\'as
importantes de la pdf, puede  resultar conveniente hacer una transformaci\'on de
variable   aleatoria   a    la   llamada   \emph{forma   est\'andar}:   $Y\equiv
\frac{X-\langle  X\rangle}{\sigma}$,  que  entonces  tiene  media igual  a  0  y
desviaci\'on est\'andar igual a 1.

\hfill

Mencionamos algunas propiedades de $E[X]$ y de $E[X^2]$.  %%\cite{Fra09}

Proposici\'on: \  Sean $X$  e $Y$ dos  variables aleatorias integrables,  y sean
$a,b\in\Rset$   arbitrarios.  Entonces  la   variable  aleatoria   $Z=aX+bY$  es
integrable, siendo \ $E[Z]=a E[X]+b E[Y]$.
%% Demostracion:   por   linealidad   de   la   integral,   y   la   desigualdad
%% triangular. (ejerc)
%

Proposici\'on: \ Sean  $X$ e $Y$ dos variables aleatorias  integrables. Si $X$ e
$Y$ son independientes, entonces \ $E[X Y]=E[X] E[Y]$.
%% Demostracion: por independencia de los eventos  A y B asociados a las var.X e
%% Y. (ejerc) Indep es cond.sufic para factorizacion
%

Teorema: \  Sean $X$ e $Y$ dos  variables aleatorias reales Las  variables $X$ e
$Y$ son independientes  si y s\'olo si $E[f(X)  g(Y)]=E[f(X)] E[g(Y)]$ para todo
par de funciones $f$ y $g$ en $\Rset$, continuas y acotadas.
%

\modif{Volviendo al  ejemplo de la pagina~\pageref{Ej:MP:Mixta},  tratando de la
  media de \ $X = V \, \un_{U <  \frac12} + \un_{U \ge \frac12}$ \ con \ $U$ \ y
  \ $V$ \ variables independientes de distribuci\'on uniformas sobre $(0 \, ; \,
  1)$, se calcula gracia a la linealidad y a la indepencia, \ $\Esp[X] = \Esp[V]
  \Esp[\un_{U < \frac12}] + \Esp[\un_{U \ge \frac12}] = \frac12 \times \frac12 +
  \frac12   =  \frac34$  \   como  lo   hemos  obtenido   usando  $P_X$   en  la
  pagina~\pageref{Ej:MP:MediaMixta}.}
 
Proposici\'on: \  Sea $X$ una variable  aleatoria de cuadrado  integrable, y sea
$\Var(X)=E[(X-\langle X\rangle)^2]\equiv \sigma^2$ su varianza. Luego:
\begin{enumerate}
\item $\Var(X)=E[X^2]-(E[X])^2$ 
%
\item $\forall \ a\in\Rset : \ \Var(X+a)=\Var(X) , \ \Var(aX) = a^2 \Var(X)$
%
\item Si $Y$ es otra  variable aleatoria de cuadrado integrable, e independiente
  de $X$, entonces: \ $\Var(X+Y)=\Var(X)+\Var(Y)$
  %% Dem:   Var(X+Y)=Var(X)+Var(Y)+2(E[XY]-E[X]E[Y]),    donde   la   covarianza
  %% Cov(X,Y)=E[XY]-E[X]E[Y] se anula si X e Y son indep.
\end{enumerate}

\hfill

\emph{Desigualdades de Chebyshev y de Bienaym\'e--Chebyshev}
%% referencias

\hfill

Estas desigualdades dan una cota superior  a la probabilidad de que una cantidad
que  fluct\'ua aleatoriamente  exceda  cierto valor  umbral,  a\'un sin  conocer
detalladamente la forma de la distribuci\'on de probabilidad.

\underline{Desigualdad de Chebyshev}: \\
Sea  $X$ una  variable aleatoria  real  con funci\'on  densidad de  probabilidad
$p(x)$. Sea  $g(x)\geq 0 \ \forall \,  x\in\Rset$, con $g(x)\geq K  \ \forall \,
x\in D\subset\Rset$, para alg\'un $K>0$. Entonces por un lado
$$
\Pr[g(X)\geq K] = \Pr[X\in D] = \int_D p(x)\,dx
$$
y por otro lado
$$
\langle g(X)\rangle=  \int_{\Rset} g(x)p(x)\,dx \geq \int_D  g(x)p(x)\,dx \geq K
\int_D p(x)\,dx ,
$$
luego se tiene la desigualdad: 
\begin{equation}
\label{eq:desigChebyshev}
\Pr[g(X)\geq K] \leq \frac{\langle g(X)\rangle}{K} .
\end{equation}

\underline{Desigualdad de Bienaym\'e--Chebyshev}: \\
Sea $X$  una variable  aleatoria real de  esperanza $\mu$ y  varianza $\sigma^2$
finita. Entonces, $\forall \, \epsilon >0$ se tiene la desigualdad:
$$
\Pr[|X-\mu| > \epsilon] \leq \frac{\sigma^2}{\epsilon^2} .
$$
En forma equivalente, se puede plantear  la probabilidad de que $X$ se aparte de
su valor medio en m\'as de  cierto n\'umero $\eta$ de desviaciones est\'andar: \
tomando  $g(x)=(\Delta   x)^2=(x-\mu)^2$  en  la  Ec.~\eqref{eq:desigChebyshev},
resulta la desigualdad :
\begin{equation}
\label{eq:desigChebyshevBienayme}
\Pr[|\Delta X| \geq \eta \sigma ] = \Pr[(\Delta X)^2 \geq \eta^2 \sigma^2 ] \leq
\frac{1}{\eta^2}
\end{equation}

Estas  relaciones  afirman que  cuanto  m\'as chica  es  la  varianza, m\'as  se
concentra la variable en torno a su media. Ambas cotas son en general d\'ebiles;
por  ejemplo,  la  desigualdad~\eqref{eq:desigChebyshevBienayme} indica  que  la
probabilidad  de encontrar  una fluctuaci\'on  superior a  $\eta=3$ desviaciones
est\'andar alrededor de la media, est\'a  por debajo de $1/9$; el c\'alculo para
una  distribuci\'on t\'ipica  como la  Gaussiana ajusta  dicha  probabilidad por
debajo de 0.003.

\hfill

\emph{Momentos para varias variables aleatorias}

\hfill

En  el caso  de  varias  variables aleatorias  $X,Y,Z,\ldots$  con pdf  conjunta
$p(x,y,z,\ldots)$ se  define el  \emph{momento central de  orden} $r,s,t,\ldots$
como~\cite{ManWol95,CovTho06}
$$
\mu_{r,s,t,\ldots} \equiv \langle (\Delta  x)^r (\Delta y)^s (\Delta z)^t \ldots
\rangle  =   \int  (x-\langle  x\rangle)^r   (y-\langle  y\rangle)^s  (z-\langle
z\rangle)^t \cdots \ p(x,y,z,\ldots)\,dx\,dy\,dz\cdots .
$$
Por  ejemplo,  para $\begin{pmatrix}  X  \\  Y  \end{pmatrix} \sim  p(x,y)$  los
momentos centrales  de orden lineal  resultan: \ $\mu_{1,0}=\mu_{0,1}=0$,  y los
momentos centrales de orden cuadr\'atico est\'an dados por las varianzas de cada
variable   y   por    la   llamada   covarianza:   \   $\mu_{2,0}={\sigma_X}^2$,
$\mu_{0,2}={\sigma_Y}^2$, y $\mu_{1,1}=\langle  \Delta X \Delta Y\rangle$. Estos
\'ultimos se  pueden acomodar en  una matriz, con propiedades  interesantes como
veremos a continuaci\'on.

Sea   $X^1,\ldots,X^d$   un   conjunto   de   $d$   variables   aleatorias.   La
\emph{covarianza} entre $X^i$ y $X^j$ se define como
$$
\mu^{ij} \equiv \langle \Delta x^i \Delta x^j \rangle = \mu^{ji}
$$
para $i,j=1,\ldots,d$.  Las $d(d+1)/2$ cantidades de este tipo se disponen en un
arreglo (sim\'etrico)  de $d\times d$, la  \emph{matriz de covarianza}~$\Sigma$,
cuya diagonal son las varianzas $(\sigma^i)^2$. Por ejemplo, si $d=2$ se tiene
$$
\begin{pmatrix}  X^1 \\  X^2 \end{pmatrix}  \sim  p(x^1,x^2) \  : \qquad  \Sigma
=     \begin{pmatrix}    (\sigma^1)^2     &    \mu^{12}     \\     \mu^{21}    &
  (\sigma^2)^2 \end{pmatrix} .
$$

Proposici\'on: 
$$
|\mu^{ij}|^2 \leq \mu^{ii} \mu^{jj}
$$
La   demostraci\'on  de   esta  proposici\'on   involucra  la   desigualadad  de
Cauchy--Schwarz ........... %%

Se  define el  \emph{coeficiente de  correlaci\'on} que  es adimensional  y toma
valores entre $-1$ (variables  completamente anticorrelacionadas) y 1 (variables
completamente correlacionadas) como: \ $\rho^{ij}=\rho^{ji}\equiv
\frac{\mu^{ij}}{\sigma^i \sigma^j}$.\\
Como ejemplo,  dadas $X^1$ y  $X^2=aX^1+b$ que fluct\'uan  en fase ($a>0$)  o al
rev\'es    ($a<0$),    se   tiene    $\Delta    x^2=a    \Delta   x^1$,    luego
$\rho^{12}=\frac{a}{|a|}=\pm 1$.

....


\vspace{1.5pt}
%%%%%%%%%%%%%%%%%%%%%%%%%%%%%%%%%%%%%%%%%%%%%%%%%%%%%%%%%%%%%%%%%%%%%%%%%%
\subseccion{Funciones generatrices}
%%%%%%%%%%%%%%%%%%%%%%%%%%%%%%%%%%%%%%%%%%%%%%%%%%%%%%%%%%%%%%%%%%%%%%%%%%

Se  definen  un conjunto  de  funciones  que  permiten hallar  f\'acilmente  los
distintos   momentos  de   una   distribuci\'on  de   probabilidad.  Se   llaman
\emph{funciones  generadoras} o \emph{funciones  generatrices}, y  est\'an dadas
como valores de expectaci\'on de  funciones de la variable aleatoria (discreta o
continua), con un par\'ametro real o complejo.

La  \emph{funci\'on  generadora   de  momentos}  (MGF,  \emph{moment  generating
  function}) se define como
$$
M(\xi) \equiv \langle e^{\xi X} \rangle =  \int e^{\xi x} p(x) \, dx , \quad \xi
\in \Rset
$$
en el  caso de una variable  aleatoria continua $X$  con pdf $p(x)$. Se  tiene \
$M(0)=\int p(x)\,dx=1$ (que corresponde a la condici\'on de normalizaci\'on). Si
la  variable $X$ es  positiva y  se toma  $\xi=-s$ con  $s>0$, se  interpreta en
t\'erminos de la transformada de Laplace de la funci\'on $p$.  %%
\\
Si existe, la  MGF posibilita obtener f\'acilmente los  momentos (ordinarios) de
$X$ a  distintos \'ordenes, mediante los  coeficientes del desarrollo  de $M$ en
serie de potencias de $\xi$:
$$
M(\xi)  =  \sum_{r=0}^{\infty}  \frac{\xi^r}{r!}  \int  x^r  p(x)  \,  dx  =  1+
\sum_{r=1}^{\infty} \frac{\nu_r}{r!} \xi^r
$$
o, alternativamente, mediante  las sucesivas derivadas de $M$  respecto de $\xi$
en 0:
$$
\nu_r=\left.  \frac{d^r  M(\xi)}{d\xi^r}\right|_{\xi=0}  ,  \quad  r=1,2,\ldots;
\quad \nu_0\equiv 1 .
$$ 

En  el  caso de  una  variable aleatoria  discreta,  suponiendo  que el  espacio
muestral es $\Nset$, se definen  dos funciones: la \emph{funci\'on generadora de
  momentos (ordinarios)} (MGF) dada por
$$
M(\xi) \equiv \langle e^{\xi N} \rangle = \sum_{n\geq 0} e^{\xi n} p_n ,
%% = \sum_{r\equiv 0} \frac{\langle n^r\rangle}{r!} \xi^r , 
$$
y la \emph{funci\'on generadora  de momentos factoriales} (FMGF, \emph{factorial
  moment generating function}) como
$$
F(\xi) \equiv \langle (1+\xi)^N \rangle = \sum_{n\geq 0} (1+\xi)^n p_n
$$
para    $\xi     \in    \Rset$    en     ambos    casos.    Se     verifica    \
$M(0)=F(0)=\sum_{n=0}^{\infty} p_n=1$. Se muestra simplemente que
$$
M(\xi) = \sum_{r=0}^{\infty} \frac{\langle n^r\rangle}{r!} \xi^r , 
$$
lo que  permite obtener los momentos  de la distribuci\'on  para cualquier orden
$r\geq 1$. Por otro lado, el desarrollo de la FMGF da
$$
F(\xi)  =   \sum_{n=  0}^{\infty}  \sum_{r=   0}^n  \binom{n}{r}  \xi^r   p_n  =
\sum_{r=0}^\infty \sum_{n=r}^{\infty} \frac{n(n-1)\cdots (n-r+1)}{r!}  \xi^r p_n
= \sum_{r=0}^{\infty} \frac{\langle n^{(r)}\rangle}{r!} \xi^r
$$
teniendo  en cuenta  en las  dobles sumas  que $0\leq  r\leq n$,  con  $n$ hasta
$n_{\max}$ \'o  $\infty$. Se  ve entonces que  $F$ permite obtener  los momentos
factoriales de orden $r$ arbitrario.

Dada   una   variable   aleatoria   a   valores  naturales,   la   funci\'on   \
$G(\xi)=\sum_{n=0}^{\infty} p_n \xi^n$, con $-1\leq \xi\leq 1$, %% \leq o < ?
es  tambi\'en una  funci\'on generatriz.  Por ejemplo,  si $G$  admite derivadas
primera  y segunda en  $\xi=1$ se  obtienen: $\langle  N\rangle=G'(1)$, $\langle
N(N-1)\rangle=G''(1)$, $\Var(N)=G''(1)+G'(1)-[G'(1)]^2$; adem\'as, se obtiene la
ley   de   distribuci\'on   evaluando   derivadas   de   $G$   en   $\xi=0$:   \
$p_n=\frac{G^{(n)}(0)}{n!}$.  %% Ej: probar
\cite{Fra09}%%p.73

\hfill

La \emph{funci\'on caracter\'istica}  (CF, \emph{characteristic function}) tiene
argumento complejo: \cite{Luk61}
$$
C_X(\xi) \equiv \langle e^{i \xi X} \rangle = \int e^{i \xi x} p(x) \, dx .
$$
La importancia  de esta  funci\'on reside  en que siempre  existe y  est\'a bien
definida, dado que es la  transformada de Fourier de una funci\'on absolutamente
integrable (i.e. $\int |f(x)| \, dx < \infty$) \cite{Gol61}

Si la pdf \ $p(x)$ es de cuadrado integrable, entonces 
$$
p(x) = \frac{1}{2	pi} \int e^{-i \xi x} C_X(\xi) \, d\xi .
$$
El requisito  para esta importante relaci\'on es  que \ $\int_{-\infty}^{\infty}
|p(x)|^2 \, dx<\infty$;  sin embargo, a\'un es v\'alida  para distribuciones con
una contribuci\'on  tipo $\delta$.  Por otro  lado los momentos,  si existen, se
obtienen derivando la funci\'on $C$ tal como expresa la siguiente proposici\'on:

\textbf{Proposici\'on:} \ %%
La  variable aleatoria  $X$  admite  momento de  orden  $r$ si  y  s\'olo si  la
funci\'on caracter\'istica $C$ es $r$ veces derivable en $\xi=0$, siendo
$$
\langle X^r\rangle = (-i)^r C_X^{(r)}(0) . 
$$

Por ejemplo, en el caso de la distribuci\'on de Cauchy--Lorentz resulta
$$
C(\xi)     =     \frac{\gamma}{\pi}    \int_{-\infty}^{\infty}     \frac{e^{i\xi
    x}}{\gamma^2+(x-x_0)^2} dx = e^{-\gamma |\xi| e^{i x_0\xi}}
$$
tomando $\gamma >0$. Esta funci\'on est\'a  definida para todo $\xi$, pero no es
derivable en $\xi=0$,  lo que coincide con el hecho de  que no est\'an definidos
los momentos para esta pdf.

Para  una   variable  aleatoria  compleja   $Z=X+iY$,  usando  la   noci\'on  de
transformada de Fourier bidimensional, se define:
$$
C_Z(\mu) \equiv \int e^{\mu^* z-\mu z^*} p(z) \, d^2z .
$$

Resumimos algunas propiedades importantes de la funci\'on caracter\'istica:
\begin{enumerate}
\item $C(0) =1$
%
\item $|C(\xi)|\leq C(0)$ %%dem.
%
\item $C(\xi)$  es una  funci\'on continua  en $\Rset$ (a\'un  si la  pdf $p(x)$
  tiene discontinuidades) %dem.
%
\item $C(-\xi) = C(\xi)*$
%
\item  $C(\xi)$ es  definida no  negativa,  de tal  forma que  para un  conjunto
  arbitrario  de  $N$  n\'umeros  reales $\xi_1,\ldots,\xi_N$  y  $N$  n\'umeros
  complejos $a_1,\ldots,a_N$, se cumple
  $$
  \sum_{i,j=1}^N a_i^* a_j C(\xi_j-\xi_i) \geq 0 .
  $$
%
\item  $C(\xi) =  M(i\xi) =  F(e^{i\xi}-1)$, si  $M$ y  $F$ existen;  \ $F(\xi)=
  M(\ln(1+\xi))$
\end{enumerate}

{\teorema (Bochner, Goldberg).... } %%

\textbf{Proposici\'on:} \ %%
Sean $X$ e  $Y$ dos variables aleatorias reales  independientes, cuyas funciones
caracter\'isticas son $C_X$ y $C_Y$. Entonces \ $C_{X+Y}=C_X C_Y$.

\hfill

Cumulant generating function .... %%

\hfill

Extendemos  la  definici\'on  de   funci\'on  caracter\'istica  para  un  vector
aleatorio. ... %%


....
}



