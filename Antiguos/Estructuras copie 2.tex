\seccion{Estructuras}
\label{s:3.1}

Una  de  las  nociones m\'as  elementales  de  la  matem\'atica  es la  de  {\it
  Conjunto}.  Un   conjunto  es  una  colecci\'on   de  elementos  perfectamente
caracterizados.  Los   elementos  pueden  ser  de   cualquier  tipo:  n\'umeros,
funciones, personas, autos, etc. El  enfoque matem\'atico moderno es ir montando
estructuras de  distinta naturaleza sobre  un dado conjunto. En  este cap\'itulo
comenzaremos con la noci\'on de Espacio Topol\'ogico y llegaremos al concepto de
Variedad Riemanniana. Este procedimiento ha mostrado ser de utilidad en el marco
de la f\'isica, que es nuestro principal \'ambito de inter\'es.  El mapa de ruta
de  las de  las  distintas estructuras  que  veremos en  este  cap\'itulo es  el
siguiente:
%
\begin{itemize}
\item Espacio Topol\'ogico
\item Espacio M\'etrico
\item Variedad Topol\'ogica
\item Estructura Diferenciable (Variedad Diferenciable)
\item Estructura Afin (Noci\'on de paralelismo)
\item Estructura m\'etrica (Finsler y Riemann)
\end{itemize}

Si  bien   existe  una  estructura   intermedia  entre  la  topol\'ogica   y  la
diferenciable,  que se  conoce como  {\it  estructura lineal  a trozos},  aqu\'a
prescindiremos de su estudio. A  su vez, hay otras estructuras matem\'aticas que
son usadas en el marco de  las teor\'ias f\'isicas. Se destacan la estructura de
producto interno  sobre un conjunto complejo,  la cual conduce a  la noci\'on de
espacio  de Hilbert,  de fundamental  importancia en  Mec\'anica  Cu\'antica; la
estructura  simpl\'ectica, \'util  en Mec\'anica  Cl\'asica y  la  estructura de
Khaler, de relevancia en teor\'ia de cuerdas.

Comenzaremos con la noci\'on de espacio topol\'ogico.


%%%%%%%%%%%%%%%%%%%%%%%%%%%%%%%%%%%%%%

\subseccion{Espacio Topol\'ogico}

Un conjunto  arbitrario $X$  est\'a desprovisto de  toda estructura  que permita
definir nociones tales como la {\it convergencia} de una sucesi\'on de elementos
de  $X$, la  {\it proximidad}  de dos  elementos de  $X$, etc.  En  principio se
dispone s\'olo de las operaciones  elementales de {\it uni\'on} $\bigcup$ e {\it
  intersecci\'on} $\bigcap$ de  subconjuntos. Estas operaciones tambi\'en pueden
realizarse entre  distintos conjuntos.  Denotaremos con $\emptyset$  al conjunto
vac\'io.  Surge entonces el  desaf\'io entonces  de construir  alguna estructura
matem\'atica  definida sobre  $X$ que  permita  definir, de  manera precisa  las
nociones  de  proximidad,  continuidad,  convergencia,  etc.  Esto  se  logra  a
trave\'es de la idea de una {\bf topolog\'ia} sobre $X$.

\begin{definicion}
  Una  {\bf  Topolog\'ia}  $\tau$  sobre  el  conjunto $X$  es  una  familia  de
  subconjuntos de $X$ que cumple con las siguientes condiciones:
  %
  \begin{enumerate}
  \item $X$ y $\emptyset$ est\'an en $\tau$: $X, \: \emptyset \in \tau$
  %
  \item La intersecci\'on de cualquier colecci\'on finita de elementos de $\tau$
    est\'a en $\tau$:
    %
    \[
    A_i  \in  \tau,  \  \forall \,  i  =  1  ,  \ldots  , n  \:  \Rightarrow  \:
    \bigcap_{i=1}^n A_i \in \tau
    \]
  %
  \item La uni\'on de una colecci\'on  arbitraria - finita o no- de elementos de
    $\tau$, pertenece a $\tau$:
    %
    \[
    A_\alpha \in \tau \: \Rightarrow \: \bigcup_\alpha A_\alpha \in \tau
    \]
\end{enumerate}
\end{definicion}

\begin{definicion}
  Al par $(X,\tau)$ lo llamaremos  {\bf Espacio Topol\'ogico}. Los conjuntos que
  est\'an en $\tau$ se llaman {\it abiertos}.
\end{definicion}

{\it Ejemplos}:

\begin{itemize}
\item {\bf  Topolog\'ia trivial}. Es la  que consta de s\'olo  dos elementos, el
  conjunto vac\'io y el conjunto total $X: \tau = \{ \emptyset , X \}$.
% 
\item {\bf Topolog\'ia discreta}. Es la que en todo subconjunto de $X$ est\'a en
  $\tau$, es decir $\tau = \P(X)$ donde $\P(X)$ representa a las partes de $X$
    
\item En  los cursos elementales  de an\'alisis matem\'atico hemos  estudiado en
  $\Rset^n$, es decir el conjunto de $n-tuplas$ de n\'umeros reales, la noci\'on
  de bolas abiertas. M\'as precisamente,  una bola abierta en $\Rset^n$ centrada
  en el  punto $p =  (p_1 , \ldots ,  p_n) \in \Rset^n$  y de radio $r>0$  es el
  conjunto
  %
  \[
  \B_{r,p}  =  \{ (x_1  ,  \ldots  , x_n)  \quad  \mbox{tal  que}  \quad 0  \leq
  \sqrt{\sum_i (x_i-p_i)^2} < r \}
  \]
  %
  La  colecci\'on de  todas  las  bolas abiertas  en  $\Rset^n$ constituyen  una
  topolog\'ia  para $\Rset^n$.  Se conoce  como  la {\bf  topolog\'ia usual}  de
  $\Rset^n$
\end{itemize}

\begin{definicion}
  Un {\bf entorno} de un punto $x \in X$ es un conjunto $U$ que contiene a $x$ y
  tal que existe un abierto $V$ contenido en  $U: \ x \in V \subseteq U$ \ con \
  $U \in \tau$.
\end{definicion}

\begin{definicion}
  Sea $f: X  \rightarrow Y$ una funci\'on entre  dos espacios topol\'ogicos $(X,
  \tau)$ e  $(Y,\omega)$. $f$ es una {\bf  funci\'on continua} en $x  \in X$ sii
  dado cualquier entorno  abierto $U \subset Y$ de $f(x)$,  existe un entorno de
  $x$, $V \subset X$ tal que $f(V) \subset U$.
\end{definicion}

\begin{definicion}
  Un  {\bf homomorfismo} $\Psi$  entre dos  espacios topol\'ogicos  $(X,\tau)$ e
  $(Y,\omega)$ es una funci\'on
  %
  \[
  \Psi: X \rightarrow V \subseteq Y
  \]
  %
  biyectiva, continua y con inversa continua.
\end{definicion}

\begin{definicion}
  Una  {\bf  sucesi\'on}  en un  conjunto  $X$  es  una aplicaci\'on  $s:  \Nset
  \rightarrow   X$   donde   $\Nset$   es   el   conjunto   de   los   n\'umeros
  naturales. Denotaremos  a la  sucesi\'on por  $\{ x_n \}$  \ donde  \ $  n \in
  \Nset$.
\end{definicion}

En un espacio topol\'ogico podemos introducir la noci\'on de convergencia de una
sucesi\'on. Obs\'ervese  que \'esto  es posible gracias  a que disponemos  de la
noci\'on de conjunto abierto.

\begin{definicion}
  Sea $(X,  \tau)$ un espacio  topol\'ogico y  $\{ x_n \},  \: n \in  \Nset$ una
  sucesi\'on en $X$. Diremos que $x$ es el {\bf l\'imite} de $x_n $ si para todo
  entorno $V$ de $x$, existe un $n_0  \in \Nset$ tal que $\forall \, n \geq n_0$
  se tiene que $x_n \in V$.
\end{definicion}

Los l\'imites de  las sucesiones no tienen porque  ser \'unicos. Una condici\'on
que  debe cumplir  el espacio  topol\'ogico $(X,\tau)$  para que  las sucesiones
tengan un \'unico l\'imite es que dados  dos puntos distintos $x \neq y$,con $ x
, y \in X$ existen entornos disjuntos de $x$ e $y$.

A los  espacios topol\'ogicos que satisfacen  con esta condici\'on  se los llama
espacios de Hausdorff o espacio $T_2$.


%%%%%%%%%%%%%%%%%%%%%%%%%%%%%%%%%%%%

\subseccion{Espacios m\'etricos}

En el  tercer ejemplo de espacio  topol\'ogico, usamos la  noci\'on de m\'etrica
eucl\'idea  para definir  las  bolas abiertas  en  $\Rset$. El  disponer de  una
m\'etrica  no es  algo que  ocurre  en todo  conjunto. Eso  motiva la  siguiente
definici\'on:
%
\begin{definicion}
  Un {\bf  Espacio M\'etrico} en un conjunto  $X$ munido de una  funci\'on $d: X
  \times X \rightarrow \Rset_+$ tal que se cumplen las condiciones:
  %
  \begin{enumerate}
  \item\label{positiva} $d(x,y)  \geq 0 \forall  x , y  \in X$ y la  igualdad se
    cumple sii $x=y$
  %
  \item\label{simetria} $d(x,y) = d(y,x)$
  %
  \item\label{triangular} $d(x,y) \leq d(x,z) + d(z,y)  \quad \forall \, x , y ,
    z \in X$
  \end{enumerate}
\end{definicion}
%
La  \'ultima  condici\'on se  conoce  como  {\it  desigualdad triangular}.   Mas
adelante en este libro veremos funciones $d: X \times X \rightarrow \Rset^0$ que
no      satisfacen     ni      la      condici\'on~\ref{simetria}     ni      la
condici\'on~\ref{triangular},  pero que  sin  embargo sirven  para medir  cu\'an
separados  est\'an dos  puntos  de $X$.   En ese  caso  diremos que  $d$ es  una
distancia en $X$.


%%%%%%%%%%%%%%%%%%%%

\subseccion{Variedad Topol\'ogica}

Nuestra experiencia cotidiana de percibir  que estamos inmersos en un espacio de
3 dimensiones, en el cual  podemos medir \'angulos y determinar distancias entre
dos puntos, ha  hecho que usemos estas caracteristicas  de nuestro habitat, como
motivaci\'on de  la defici\'on de ciertas estructuras  matem\'aticas en espacios
abstractos.

En primer lugar, con la noci\'on de una variedad topol\'ogica buscaremos simular
en  un conjunto  cualquiera, la  noci\'on  de cercan\'ia  y dimensionalidad  que
tenemos en $\Rset^n$.

\begin{definicion}
  Una  {\bf Variedad  Topol\'ogica $n$-dimensional}  es un  espacio topol\'ogico
  $\M$ tal que es {\it localmente eucl\'ideo}, es decir que para cada $x \in \M$
  existe  un  entorno  abierto $U$  de  $x$,  homeomorfo  a  un abierto  $V$  de
  $\Rset^n$:
  %
  \[
  \phi: U \subseteq \M \rightarrow \Rset^n
  \]
  %
  tal que
  %
  \[
  \phi:U \rightarrow V
  \]
  %
  y $\phi$ es un homeomorfismo.  Tambi\'en pediremos que $\M$, como espacio
  topol\'ogico, sea un espacio Hausdorff.
\end{definicion}
%
A los pares $(U,\phi)$ se llaman cartas sobre $\M$. Se supone que la colecci\'on
de todas  las cartas cumbren completamente  a $\M$. Las  cartas permiten asignar
{\it coordenadas} a $\M$:
%
\[
\mbox{Si}  \quad  p \in  U  \subseteq \M  \quad  \mbox{entonces}  \quad \phi:  p
\rightarrow (p_1 , \ldots , p_n) \in \Rset^n
\]
%
la  colecci\'on  de n\'umeros  reales  $(p_1  , \ldots  ,  p_n)$  se llaman  las
coordenadas de  $p$ de acuerdo  a la carta  $(U,\phi)$. Podr\'ia suceder  que un
mismo punto $p$ pertenezca a m\'as de una carta, digamos $(U_1,\phi_1)$ y $(U_2,
\phi_2)$. En ese caso hablaremos de un cambio de coordenadas:
%
\begin{equation}\label{cc}
\psi_2 \circ \phi_1^{-1}: \phi_1(U_1 \cap U_2) \rightarrow \psi_2(U_1 \cap U_2)
\end{equation}
%
Si denotamos por $(p_1 , \ldots  , p_n)$ a las coordenadas correspondientes a la
carta  $(U_1,\phi_1)$  y  por  $(\tilde{p}_1  , \ldots  ,  \tilde{p}_n)$  a  las
correspondientes a la carta  $(U_2,\psi_2)$, entonces las funciones $\tilde{p}_i
= \tilde{p}_i(p_1 ,  \ldots , p_n)$ son funciones continuas, y  dan el cambio de
coordenadas. Estas funciones son invertibles con inversa continua.

Ejemplos de variedades topol\'ogicas son:

\begin{itemize}
\item $\Rset^n$. En este caso hay  una carta coordenada global que cubre toda la
  variedad y donde el homeomorfismo es la identidad.
%
\item $\Sset^n$, la esfera de dimensi\'on $n$. Est\'a definida como el conjunto:
  %
  \[
  \Sset^n = \left\{  (x_1 , \ldots , x_{n+1})  \in \Rset^{n+1} \quad \mbox{tales
      que} x_1^2 + \cdots + x_{n+1}^2 = 1 \right\}
  \]
  %
  Se debe observar  que al definir $\Sset^n$ no estamos  pensando que est\'a inmersa
  en $\Rset^n$.
\end{itemize}

