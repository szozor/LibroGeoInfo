\begin{tikzpicture}%[scale=.8]
\shorthandoff{>}
%
\tikzset{declare function={
xplus(\x) = max(\x,0);
%ifthenelse(\x > 0 , \x , NaN);
}}
%
% Approximation de la cdf gaussienne
\tikzset{declare function={
normcdf(\x)=1/(1 + exp(-0.07056*(\x)^3 - 1.5976*(\x)));
}}
%
% gamma incompleta para k entero
\tikzmath{function gammainc(\x,\k) {
    if \k == 1 then {return (1-exp(-\x));}
    else {return gammainc(\x,\k-1)*(\k-1)-((\x)^(\k-1))*exp(-\x);};
};};
%
%
% beta incompleta (k,l) int_0^r u^(k-1) (1-u)^() para l entero, k > 0
\tikzmath{function betainc(\x,\k,\l) {
    if \x <= 0 then {return 0;}
    else {if \x > 1 then {return betainc(1,\k,\l);}
          else {if \l == 1 then {return ((\x)^(\k))/\k;}
                else {return ((\l-1)/\k)*betainc(\x,\k+1,\l-1)+(((\x)^(\k))*((1-\x)^(\l-1))/\k);};
               };
         };
};};
%
\pgfmathsetmacro{\d}{2};% dimension
\pgfmathsetmacro{\nu}{6};% degres de liberte
\pgfmathsetmacro{\mux}{1/sqrt(2)};% mu_1
\pgfmathsetmacro{\muy}{1/sqrt(2)};% mu_2
\pgfmathsetmacro{\a}{factorial(\nu/2)/(2*(pi^(\d/2))*factorial((\nu-\d)/2))};

\tdplotsetmaincoords{45}{65}
\begin{scope}[tdplot_main_coords,scale=.65]
%
% densidad d=2, nu = 4, mu = [1 1]/sqrt(2)
% Mezcla de Student-r
\begin{axis}[
    colormap = {whiteblack}{color(0cm)  = (white);color(1cm) = (black)},
    width=.5\textwidth,
    view={45}{65},
    enlargelimits=false,
    %grid=major,
    domain=-2.5:2.5,
    y domain=-2.5:2.5,
    zmax={.85*\a},
    color=black,
    samples=70,
    xlabel=$x_1$,
    ylabel=$x_2$,
    zlabel=$p_X$,
]
%
% Mezcla de Student
\addplot3 [surf] {
\a*((xplus(1-(x+\mux)^2-(y+\muy)^2))^((\nu-\d)/2)+(xplus(1-(x-\mux)^2-(y-\muy)^2))^((\nu-\d)/2))};
%%
\end{axis}
\end{scope}
%
%
% densidad rearreglada
\begin{scope}[tdplot_main_coords,xshift=5.5cm,scale=.65]
%
\begin{axis}[
    colormap = {whiteblack}{color(0cm)  = (white);color(1cm) = (black)},
    width=.5\textwidth,
    view={45}{65},
    enlargelimits=false,
    %grid=major,
    domain=-2:2,
    y domain=-2:2,
    zmax={.8*\a},
    color=black,
    samples=70,
    xlabel=$x_1$,
    ylabel=$x_2$,
    zlabel=$p^{\downarrow}_X$,
]
%
% Mezcla rearreglada
\addplot3 [surf,opacity=.8] {\a*((xplus(1-(x^2+y^2)/(4^(1/\d))))^((\nu-\d)/2))};
%
\end{axis}
\end{scope}
%
% curva de Lorentz
\begin{scope}[xshift=11.25cm,yshift=.5cm,scale=.65]
\pgfmathsetmacro{\sx}{1.5};% x-scaling
\pgfmathsetmacro{\sy}{2.75};% y-scaling 
\pgfmathsetmacro{\mx}{3.25};% maximo en x
\pgfmathsetmacro{\B}{betainc(1,\d/2,(\nu-\d)/2+1)};% B(d/2,(nu-d)/2+1)
\pgfmathsetmacro{\ss}{(1/(\nu+2))+(((\mux)^2)+((\muy)^2))/\d};
\pgfmathsetmacro{\nut}{2.25};% nu Student-t
\pgfmathsetmacro{\Bt}{betainc(1,\nut/2,\d/2)};% B(d/2,nust/2)
\pgfmathsetmacro{\st}{(\nut-2)*((1/(\nu+2))+(((\mux)^2)+((\muy)^2))/\d)};

\draw[>=stealth,->] (-.25,0)--({\sx*\mx+.5},0) node[right]{\small $r$};
\draw[>=stealth,->] (0,-.15)--(0,{\sy+.5}) node[above]{\small $\L_p(r)$};
%

% Lorentz de la mezcla
\draw[thick,domain=0:\mx,samples=50,smooth] plot({\x*\sx},{\sy*betainc(((\x)^2)/(4^(1/\d)),\d/2,(\nu-\d)/2+1)/\B});
%
% Lorentz de la gausiana
\draw[thick,dashed,domain=0:\mx,samples=50,smooth] plot({\x*\sx},{\sy*gammainc(((\x)^2)/2/\ss,\d/2)/factorial(\d/2-1)});
%
% Lorentz de la student-t nu' = 3;
\draw[thick,dotted,domain=0:\mx,samples=50,smooth] plot({\x*\sx},{\sy*betainc(((\x)^2)/(\st+(\x)^2),\nut/2,\d/2)/\Bt});
%
\foreach \m in {0,...,\mx} {
\draw ({\m*\sx},0)--({\m*\sx},-.1) node[below,scale=.7]{$\m$};
}
%
\draw (0,\sy)--(-.1,\sy) node[left,scale=.7]{$1$};
\end{scope}
%
\node at (2.05,-.75){(a)};
\node at (7.55,-.75){(b)};
\node at (13,-.75){(c)};
\end{tikzpicture}