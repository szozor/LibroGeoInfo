\begin{tikzpicture}
\shorthandoff{>}
%
% Para el caso univariado
\pgfmathsetmacro{\sx}{.65};% x-scaling
\pgfmathsetmacro{\mx}{3.5};% x maximo del plot
\pgfmathsetmacro{\lm}{floor(\mx)};% entero max 
%
% para el caso bi-variado
\pgfmathsetmacro{\a}{.25};% x-scaling
\pgfmathsetmacro{\t}{30};% angulo de rotacion
\pgfmathsetmacro{\c}{cos(\t)};% coseno
\pgfmathsetmacro{\s}{sin(\t)};% seno
\pgfmathsetmacro{\su}{sqrt(\c^2+(\a*\s)^2)};% ecart-type 1
\pgfmathsetmacro{\sd}{sqrt(\s^2+(\a*\c)^2)};% ecart-type 2
\pgfmathsetmacro{\dx}{2.45};% dominio x del plot -dx:dx
\pgfmathsetmacro{\dy}{2.05};% dominio y del plot -dy:dy
\pgfmathsetmacro{\ma}{max(1/min(\su,\sd),1/\a/sqrt(2*pi))/sqrt(2*pi)};% z max
%
% Approximation de la cdf gaussienne
\tikzset{declare function={
normcdf(\x)=1/(1 + exp(-0.07056*(\x)^3 - 1.5976*(\x)));
}}
%
%
% densidad
\begin{scope}[scale=.9]
%
\pgfmathsetmacro{\sy}{2.5*sqrt(2*pi)};% y-scaling 
\draw[>=stealth,->] ({-\sx*\mx-.25},0)--({\sx*\mx+.25},0) node[right]{\small $x$};
\draw[>=stealth,->] (0,-.15)--(0,2.75) node[above]{\small $p_X$};
%
\draw[thick,domain=-\mx:\mx,samples=50,smooth] plot ({\x*\sx},{\sy*exp(-.5*\x*\x)/sqrt(2*pi)});
%
\draw (0,{\sy/sqrt(2*pi)})--(-.2,{\sy/sqrt(2*pi)}) node[left,scale=.7]{$\frac1{\sqrt{2 \pi}}$};
\draw (0,0)--(0,-.1) node[below,scale=.7]{$0$};
\foreach \m in {1,...,\lm} {
\draw ({-\m*\sx},0)--({-\m*\sx},-.1) node[below,scale=.7]{$-\m$};
\draw ({\m*\sx},0)--({\m*\sx},-.1) node[below,scale=.7]{$\m$};
}
%
\node at (0,-1) [scale=.9]{(a)};
\end{scope}
%
%
% reparticion
\begin{scope}[xshift=5.5cm,scale=.9]
%
\pgfmathsetmacro{\sy}{2.5};% y-scaling 
%
\draw[>=stealth,->] ({-\sx*\mx-.25},0)--({\sx*\mx+.25},0) node[right]{\small $x$};
\draw[>=stealth,->] (0,-.15)--(0,{\sy+.25}) node[above]{\small $F_X$};
%
% cumulativa
\draw[thick,domain=-\mx:\mx,samples=50,smooth] plot({\x*\sx},{\sy*normcdf(\x)});
%
\draw (0,0)--(0,-.1) node[below,scale=.7]{$0$};
\foreach \m in {1,...,\lm} {
\draw ({-\m*\sx},0)--({-\m*\sx},-.1) node[below,scale=.7]{$-\m$};
\draw ({\m*\sx},0)--({\m*\sx},-.1) node[below,scale=.7]{$\m$};
}
%
\node at (0,-1) [scale=.9]{(b)};
\end{scope}
%
%
% densidad 2D
\begin{scope}[xshift=9.5cm,yshift=-2.5mm,scale=.7]
%
\begin{axis}[
    colormap = {whiteblack}{color(0cm)  = (white);color(1cm) = (black)},
    width=.5\textwidth,
    view={45}{65},
    enlargelimits=false,
    %grid=major,
    domain=-\dx:\dx,
    y domain=-\dy:\dy,
    color=black,
    samples=80,
    xlabel=$x_1$,
    ylabel=$x_2$,
    zlabel=$p_X$,
    zmax={1.05*\ma},
]
%
% Gaussiana 2D
\addplot3 [surf] {exp(-.5*((\c*x+\s*y)^2+((-\s*x+\c*y)/\a)^2))/(2*pi*\a)};
%
% Marginales
\addplot3 [domain=-\dx:\dx,samples=50, samples y=0, thick, smooth, color=black] (x,\dy,{exp(-.5*(x/\su)^2)/sqrt(2*pi)/\su});
\addplot3 [domain=-\dy:\dy,samples=50, samples y=0, thick, smooth, color=black] (-\dx,x,{exp(-.5*(x/\sd)^2)/sqrt(2*pi)/\sd});
%
\node at (axis cs:{\dx/5},\dy,{exp(-.5*(\dx/5/\su)^2)/sqrt(2*pi)/\su})[above right]{$p_{X_1}$};
\node at (axis cs:-\dx,{\dy/5},{exp(-.5*(\dy/5/\sd)^2)/sqrt(2*pi)/\sd})[above right]{$p_{X_2}$};
%
%\node at (axis cs:{\dx},{-\dy},-1) [scale=.9]{(c)};
\end{axis}
%
\node at ({1.5*\dx},-1) [scale=.9]{(c)};
\end{scope}
\end{tikzpicture}