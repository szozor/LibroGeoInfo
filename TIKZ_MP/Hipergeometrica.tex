\begin{tikzpicture}[fixed point arithmetic]%[scale=.9]
\shorthandoff{>}
%
\pgfmathsetmacro{\sx}{.375};% x-scaling
\pgfmathsetmacro{\r}{.05};% radius arc non continuity F_X
%\pgfmathsetmacro{\p}{1/3};% probabilidad p
\pgfmathsetmacro{\n}{100};% numero n de la poblacion
\pgfmathsetmacro{\k}{12};% numero k de estados exitosos
\pgfmathsetmacro{\m}{40};% numero m de tiros
%
% Nota : con el fixed point, no anda min & max
% pero max(a,b) = (a+b+abs(a-b))/2  & min(a,b) = (a+b-abs(a-b))/2;
\pgfmathsetmacro{\f}{(\k+\m-abs(\k-\m))/2}; % ultimo indice de proba non nula
\pgfmathsetmacro{\F}{(\k+\m+abs(\k-\m))/2}; %
\pgfmathsetmacro{\d}{(abs(\m-\n+\k)+\m-\n+\k)/2}; % primer indice de proba non nula
%
% ultima proba non nula F (F-1) ... (F-f+1) / n (n-1) ... (n-f+1)
% finhiper(\F,\n,\f)
\tikzmath{function finhiper(\a,\b,\c) {
    if \c == 1 then {return (\a/\b;}
    else {return (\a/\b)*finhiper(\a-1,\b-1,\c-1);};
};};
%

\pgfmathsetmacro{\dn}{\d-1}; % proba nula hasta d-1
\pgfmathsetmacro{\fn}{\f+1}; % proba nula de nuevo a partid de f-1
\pgfmathsetmacro{\ui}{\f+3}; % ultimo indice dibujado
% f, f-1... hasta d => y de 0 hasta f-d & x = f-y
\pgfmathsetmacro{\finy}{\f-\d}
%
% masa
\begin{scope}
%
\pgfmathsetmacro{\sy}{10};% y-scaling
%
% proba nulas del principio 0 -> d-1
\foreach \y in {-2,...,\dn} {
\pgfmathsetmacro{\xl}{int(\y)};\global\let\x\xl;
\draw ({\sx*\x},0)--({\sx*\x},-.1) node[below,scale=.7]{$\x$};
\draw ({\sx*\x},0) node[scale=.6]{$\bullet$};
}
%
% proba nulas del fin f+1 -> ui
\foreach \y in {\fn,...,\ui} {
\pgfmathsetmacro{\xl}{int(\y)};\global\let\x\xl;
\draw ({\sx*\x},0)--({\sx*\x},-.1) node[below,scale=.7]{$\x$};
\draw ({\sx*\x},0) node[scale=.6]{$\bullet$};
}
%
\pgfmathsetmacro{\pr}{finhiper(\F,\n,\f)};% valor del ultima proba no nula
\pgfmathsetmacro{\maxp}{\pr};% proba maximal (inicializacion)
%
\foreach \y in {0,...,\finy} {
\pgfmathsetmacro{\xl}{int(\f-\y)};\global\let\x\xl;
\draw ({\sx*\x},0)--({\sx*\x},-.1) node[below,scale=.7]{$\x$};
\draw[dotted] ({\sx*\x},0)--({\sx*\x},{\sy*\pr}) node[scale=.6]{$\bullet$};
%
\pgfmathsetmacro{\prl}{\pr*\x*(\n-\k-\m+\x)/((\m-\x+1)*(\k-\x+1))};\global\let\pr\prl;% proba actualizado
\pgfmathsetmacro{\maxpl}{(abs(\pr-\maxp)+\pr+\maxp)/2};\global\let\maxp\maxpl;% proba max actualizado
}
%
\draw[>=stealth,->] ({-2*\sx-.25},0)--({\sx*\ui+.35},0) node[right]{\small $x$};
\draw[>=stealth,->] (0,-.15)--(0,{\sy*\maxp+.25}) node[above]{\small $p_X$};
%\draw (0,{((1-\p)^\n)*\sy})--(-.1,{((1-\p)^\n)*\sy}) node[left,scale=.7]{$(1-p)^n$};
%\draw (0,{\n*\p*((1-\p)^(\n-1))*\sy})--(-.1,{\n*\p*((1-\p)^(\n-1))*\sy}) node[left,scale=.7]{$n \, p \, (1-p)^{n-1}$};
%
\node at ({(\sx*(2+\f)+.25)/2},-1) [scale=.9]{(a)};
\end{scope}
%
%
% reparticion
\begin{scope}[xshift=8.25cm]
%
\pgfmathsetmacro{\sy}{2.5};% y-scaling 
%
\draw[>=stealth,->] ({-2*\sx-.25},0)--({\sx*\ui+.5},0) node[right]{\small $x$};
\draw[>=stealth,->] (0,-.15)--(0,{\sy+.25}) node[above]{\small $F_X$};
%
% proba nulas del principio 0 -> d-1
\foreach \y in {-2,...,\dn} {
\pgfmathsetmacro{\xl}{int(\y)};\global\let\x\xl;
\draw ({\sx*\x},0)--({\sx*\x},-.1) node[below,scale=.7]{$\x$};
}
\draw ({-2*\sx},0)--({\sx*\d},0);
%
% proba nulas del fin f+1 -> ui
\foreach \y in {\fn,...,\ui} {
\pgfmathsetmacro{\xl}{int(\y)};\global\let\x\xl;
\draw ({\sx*\x},0)--({\sx*\x},-.1) node[below,scale=.7]{$\x$};
}
\draw ({\sx*\ui},\sy)--({\sx*(\f+1)},\sy);
%
\pgfmathsetmacro{\pr}{finhiper(\F,\n,\f)};% valor del ultima proba no nula
\pgfmathsetmacro{\cum}{1};% valor final de la cumulativa
% f, f-1... hasta d => y de 0 hasta f-d & x = f-y
\foreach \y in {0,...,\finy} {
\pgfmathsetmacro{\xl}{int(\f-\y)};\global\let\x\xl;
\draw ({\sx*\x},0)--({\sx*\x},-.1) node[below,scale=.7]{$\x$};
\draw ({\sx*(\x+1)},{\sy*\cum})--({\sx*\x},{\sy*\cum}) node[scale=.6]{$\bullet$};
\draw ({\sx*\x+\r},{\sy*(\cum-\pr)+\r}) arc (90:270:\r);
\draw[dotted] ({\sx*\x},{\sy*\cum})--({\sx*\x},{\sy*(\cum-\pr)});
%
\pgfmathsetmacro{\cuml}{\cum-\pr}\global\let\cum\cuml;% cumulativa actualizada
\pgfmathsetmacro{\prl}{\pr*\x*(\n-\k-\m+\x)/((\m-\x+1)*(\k-\x+1))};\global\let\pr\prl;% proba actualizado
}
%\draw (0,{((1-\p)^\n)*\sy})--(-.1,{((1-\p)^\n)*\sy}) node[left,scale=.7]{$(1-p)^n$};
%\draw (0,{\n*\p*((1-\p)^(\n-1))*\sy})--(-.1,{\n*\p*((1-\p)^(\n-1))*\sy}) node[left,scale=.7]{$n \, p \, (1-p)^{n-1}$};
%
\node at ({(\sx*(\ui+2)+.5)/2},-1) [scale=.9]{(b)};
\end{scope}
%
\end{tikzpicture}