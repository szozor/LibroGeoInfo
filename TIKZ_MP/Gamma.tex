\begin{tikzpicture}%[scale=.9]
\shorthandoff{>}
%
\pgfmathsetmacro{\sx}{.75};% x-scaling
\pgfmathsetmacro{\mx}{7};% x maximo del plot
%
% Approximation de la cdf gaussienne
%\tikzset{declare function={
%normcdf(\x)=1/(1 + exp(-0.07056*(\x)^3 - 1.5976*(\x)));
%}}
% densidad
\begin{scope}
%
\pgfmathsetmacro{\sy}{2.5};% y-scaling 
\draw[>=stealth,->] (-.75,0)--({\sx*\mx+.25},0) node[right]{\small $x$};
\draw[>=stealth,->] (0,-.1)--(0,2.75) node[above]{\small $p_X$};
%
%\foreach \a in {1,...,3} {
\draw[thick] (-.5,0)--(0,0);
\draw[thick,dotted,domain=.17:\mx,samples=100] plot ({\x*\sx},{\sy*(\x^(-.5))*exp(-\x)/sqrt(pi)});
\draw[thick,dashed,domain=0:\mx,samples=100] plot ({\x*\sx},{\sy*exp(-\x)});
\draw[thick,dash dot,domain=0:\mx,samples=100] plot ({\x*\sx},{\sy*\x*exp(-\x)});
%\draw[thick,domain=0:\mx,samples=100] plot ({\x*\sx},{\sy*4*\x*sqrt(\x)*exp(-\x)/3/sqrt(pi)});
\draw[thick,domain=0:\mx,samples=100] plot ({\x*\sx},{\sy*\x*\x*exp(-\x)/2});
%}
%
\draw (0,\sy)--(-.1,\sy) node[left,scale=.7]{$1$};
\draw (0,{\sy*exp(-1)})--(-.1,{\sy*exp(-1)}) node[left,scale=.7]{$e^{-1}$};
\draw (0,{\sy*2*exp(-2)})--(-.1,{\sy*2*exp(-2)}) node[left,scale=.7]{$2 \, e^{-2}$};
\draw (\sx,0)--(\sx,-.1) node[below,scale=.7]{$1$};
\draw ({2*\sx},0)--({2*\sx},-.1) node[below,scale=.7]{$2$};
%
\end{scope}
%
%
% reparticion
\begin{scope}[xshift=8.5cm]
%
\pgfmathsetmacro{\sy}{2.5};% y-scaling 
%
\draw[>=stealth,->] (-.75,0)--({\sx*\mx+.25},0) node[right]{\small $x$};
\draw[>=stealth,->] (0,-.1)--(0,{\sy+.25}) node[above]{\small $F_X$};
%
% cumulativa
\draw[thick,domain=-\mx:\mx,samples=100] (-.5,0)--(0,0);
% plot({\x*\sx},{\sy*normcdf(\x)});
%
\draw (0,\sy)--(-.1,\sy) node[left,scale=.7]{\small $1$};
\end{scope}
%
\end{tikzpicture}