\seccion{Funciones generadoras}
\label{Sec:MP:generatrices}

Como lo hemos visto, un vector aleatorio es completamente definida por su medida
de probabilidad $P$, o equivalemente por la medida imagen $P_X$, o a trav\'es de
la funci\'on de repartici\'on $F_X$. Sin  embargo, bajo el impulso de Laplace en
el  siglo XVII  (entre  otros), se  introdujo  caracterizaciones alternativas  a
trav\'es de transformaciones  de la medida de probabilidad,  conocidas como {\em
  funciones generadoras}  o {\em funciones  generatrices}~\footnote{De hecho, de
  manera general,  se introdujeron  tales funciones en  un marco  m\'as general,
  asociado   a  sucesiones   de   n\'umeros,   bajo  el   impulso   de  A.    de
  Moivre~\cite{Moi30};   ver   tambi\'en~\cite{Sti30,   Eul41,   Eul50,   Moi56}
  o~\cite[Sec.~1.2.9]{Knu97_v1}.         \label{Foot:MP:Generadora}}~\cite{Lap12,
  Lap14,   Lap20}.    Existen  varias   funciones,   cuyas  tienes   propiedades
particulares que vamos a ver  en las subsecciones siguientes. Entre otros, estas
funciones  dadas como  valores  de  expectaci\'on de  funciones  de la  variable
aleatoria (discreta  o continua), con  un par\'ametro real o  complejo, permiten
hallar   f\'acilmente  los   distintos   momentos  de   una  distribuci\'on   de
probabilidad.


% ================================= Funcion generadora de probabilidad

\subseccion{Funci\'on generadora de probabilidad}
\label{Ssec:MP:GeneradoraProbabilidad}

De  manera  general,  siguiendo  el  enfoque  de A.   de  Moivre  (ver  nota  de
pie~\ref{Foot:MP:Generadora}) dada una sucesi\'on \ $a_n, \quad n \in \Nset$, se
define la  funci\'on generadora  dicha {\em ordinaria}  de la sucesi\'on  como \
$G(\{ a_n  \}_{n \in \Nset} \,  , \, z) =  \sum_{n \in \Nset} \,  z^n$. A veces,
esta serie es conocida como transformada en $z$ de la sucesi\'on \ $\{ a_n \}_{n
  \in \Nset}$. Tratando de variables aleatorias discretas sobre \ $\Nset$, con \
$p_n  = P_X(n)  = P(X  = n)$,  se puede  definir as\'i  la  funci\'on generadora
asociada a  la sucesi\'on \  $p_n$ y se  puede ver que no  es nada m\'as  que el
momento $\Esp\left[ z^X \right]$.  De manera general, la funci\'on generadora de
probabilidad  se define  de  la manera  siguiente~\cite{Fel68, JohKot97,  Muk00,
  AthLah06}:
%
\begin{definicion}[funci\'on   generadora   de   probabilidad  o   de   momentos
  factoriales]
\label{Def:MP:GeneradoraProbabilidadFactorial}
%
  Sea $X = \begin{bmatrix} X_1  & \cdots & X_d \end{bmatrix}^t$ vector aleatorio
  $d$ dimensional  definido sobre $\X  \subset \Rset^d$.  La  funci\'on definida
  por
  %
  \[
  G_X(z) =  \Esp\left[ \prod_{i=1}^d  z_i^{X_i} \right]\quad \mbox{con}  \quad z
  = \begin{bmatrix} z_1 & \cdots & z_d \end{bmatrix}^t \in \Cset^d
  \]
  %
  es conocida como  {\em funci\'on generadora de probabilidad}  o {\em funci\'on
    generadora de momentos factoriales} de $X$.
\end{definicion}
%
Esta funci\'on est definida sobre un producto cartesiano de anillos~\footnote{Si
  para \ $\left| z_i \right| \,  = \, R_i$ \ la integral converge uniformamente,
  para cualquier  \ $z \in  \Cset^d, \:  r_i \le \left|  z_i \right| \le  R_i$ \
  tenemos  por ejemplo  \ $\displaystyle  \left|  \int_{\Rset_+^d} \prod_{i=1}^d
    z_i^{x_i}  \,  dP_X(x)  \right|  \le \int_{\Rset_+^d}  \left|  \prod_{i=1}^d
    z_i^{x_i}  \right|  \,  dP_X(x)  \le \int_{\Rset_+^d}  \left|  \prod_{i=1}^d
    R_i^{x_i} \right| \, dP_X(x) < +\infty$.  El mismo enfoque se usa para todos
  los hipercuadrantes  \ $\optimes_i  \Rset_{\pm}$. Adem\'as, claramente,  para \
  $\left| z_i \right| \, =  \, 1$ \ tenemos $\displaystyle \left| \int_{\Rset^d}
    \prod_{i=1}^d  z_i^{x_i}  \,   dP_X(x)  \right|  \le  \int_{\Rset^d}  \left|
    \prod_{i=1}^d z_i^{x_i} \right| \,  dP_X(x) \le \int_{\Rset^d} dP_X(x) = 1$,
  lo       que      prueba       que      \       $r_i      \le       1      \le
  R_i$.\label{Foot:MP:GeneradoraProbabilidadExistencia}} en  el plano complejos,
$r_i \, \le \, \left|  z_i \right| \, \le \, R_i$ \ con \ $r_i  \le 1$ \ y \ $R_i
\ge 1$.

\

La denominaci\'on  {\em generadora de  probabilidad} (pgf para  {\em probability
  generating function} en ingles) se entiende sencillamente del hecho siguiente:
%
\begin{lema}
\label{Lem:MP:GeneracionProbabilidades}
%
  Cuando \ $\X = \Nset^d$ \ para cualquier \ \modif{$x = \begin{bmatrix} x_1 & \ldots &
    x_d \end{bmatrix}^t \in 
  \Part{k,d}$ (ver notaciones)}
  %
  \[
  \frac1{\prod_{i=1}^d  \modif{x_i!}} \, \left.\frac{\partial^k  G_X}{\partial z_1^{\modif{x_1}}
      \ldots \partial z_d^{\modif{x_d}}}\right|_{z=0} = P_X(\modif{x}) = P(X = \modif{x})
  \]
\end{lema}
%
\begin{proof}
  Se puede escribir la funci\'on \ $G_X$ \ bajo su forma de generadora ordinaria
  \ $\displaystyle  G_X(z) =  \sum_{\modif{x} \in \Nset^d}  \left( \prod_{i=1}^d
    z_i^{\modif{x_i}}   \right)   P(X  =   \modif{x})$   \   con  \   $\modif{x}
  =  \begin{bmatrix} \modif{x_1}  &  \cdots &  \modif{x_d} \end{bmatrix}^t$.   A
  continuaci\'on, se nota que la serie converge uniformamente por lo menos en la
  bola  $\Bset_d \equiv  \Bset_d(1)$,  probando que  $G_X$  es diferenciable  en
  $\Bset_d$, as\'i que  se puede ver esta series como el  desarollo de Taylor de
  $G_X$ (o, equivalentemente, diferenciar bajo la suma y tomar la derivada en $z
  = 0$), lo que cierra la prueba.
\end{proof}

De este resultado,  se puede notar que, en el caso  discreto, hay una relaci\'on
uno-a-uno entre  la medida  de probabilidad $P_X$  y la funci\'on  generadora de
probabilidad   $G_X$.     \modif{En   el   caso   general},    veremos   en   la
subsecci\'on~\ref{Ssec:MP:FuncionCaracterisica} que para  $z_j$ de la forma $z_j
= e^{\imath u_j}$ \  con \ $u_j \in \Rset$ \ la  transformaci\'on se inversa, de
manera que se  puede recuperar la medida  de probabilidad \ $P_X$ \  a partir de
$G_X$. Dicho de  otra manera, como la medida  \ $P_X$, \ la funci\'on  \ $G_X$ \
caracteriza completamente el vector aleatorio \ $X$.

\

Aparece  que la  funci\'on generadora  \ $G_X$  \ se  vincula tambi\'en  con los
momentos factoriales, justificando su secunda denominaci\'on, {\em generadora de
  momentos factoriales}  (fmgf para {\em factorial  moments generating function}
en ingles):
%
\begin{lema}
\label{Lem:MP:GeneracionMomentosFactoriales}
%
  \modif{Para cualquier \  $k = \begin{bmatrix} k_1 & \cdots  & k_d \end{bmatrix}^t \in \Part{K,d}$,}
  %\Nset^d$ \ con \ $K =  \sum_{i=1}^d k_i$,
 derivando $G_X$ se proba que, cuando
  existen~\footnote{En el  caso extremo,  el rayo de  convergencia de  la serie
    dando $G_X$ es igual a 1, as\'i  que no hay garantia que las derivadas en $z
    = 1$ existen.}
  %
  \[
  \left.\frac{\partial^K     G_X}{\partial     z_1^{k_1}     \cdots     \partial
      z_d^{k_d}}\right|_{z=1} = \Esp\left[ \prod_{i=1}^d \PocD{X_i}{k_i} \right]
  \]
  %
  momento factorial~\footnote{Recuerdense que  $\PocD{x}{n} = \prod_{i=0}^{n-1} (x-i),
    \quad n > 0$ \ s\'imbolo de Pochhammer, con la convenci\'on $(x)_0 = 1$; ver
    pagina~\pageref{Foot:MP:Pochhammer}} de $X$.
\end{lema}

De  este resultado,  se ve  por ejemplo  que, cuando  existen, se  recuperan los
momentos de $X$ a trav\'es de las derivadas de $G_X$:
%
\begin{itemize}
\item $G_X(1) = 1$, condici\'on de normalizaci\'on.
%
\item $\nabla_z G_X(1) = \Esp[X]$.
%
\item $\Hess_z G_X(1)  + \diag\left( \nabla_z G_X(1) \right)  = \Esp\left[ X X^t
  \right]$  \  donde  \ $\Hess_z$  \  es  la  matrice  Hessiana y  \  $\diag(a)$
  \modif{matriz  diagonal  de  componentes   \  $(i,i)$-esima  \  $a_i$  \  (ver
    notaciones).}
  % vector $a$ sobre la diagonal).
  Entonces la  matriz de covarianza  es dada por  \ $\Cov[X] = \Hess_z  G_X(1) +
  \diag\left( \nabla_z G_X(1) \right) - \nabla_z G_X(1) \nabla_z^t G_X(1)$.
\end{itemize}

\

La funci\'on \ $G_X$ \ tiene unas propiedades permitiendo por ejemplo de manejar
sencillamente  distribuciones  de probabilidades  de  combinaciones lineales  de
vectores aleatorios independientes,  como lo vamos a ver  a trav\'es del teorema
siguiente.

\begin{teorema}%[Funci\'on generadora de probabilidad de una ]
\label{Teo:MP:PropiedadesGeneradoraProbabilidad}
%
  Sean  \   $X$  \  e  \   $Y$  \  dos   vectores  aleatorios  $d$-dimensionales
  independientes, $a  = \begin{bmatrix} a_1  & \cdots & a_d  \end{bmatrix}^t \in
  \Rset^d$ \  y \ $b  = \begin{bmatrix} b_1  & \cdots & b_d  \end{bmatrix}^t \in
  \Rset^d$.  Entonces  para  cualquier  $z  = \begin{bmatrix}  z_1  &  \cdots  &
    z_d \end{bmatrix}\in \Cset^d$ \ (donde existen las funciones):
  %
  \[
  G_{\diag(a) X + b}(z) =  \prod_{i=1}^d z_i^{b_i} G_X\left( z_1^{a_1} , \ldots ,
    z_d^{a_d} \right),
  \]
  %
  \[
  G_{X+Y}(z) = G_X(z) \, G_Y(z)
  \]
  %
  y para $z \in \Cset$
  %
  \[
  G_X\left( z^{a_1} , \ldots , z^{a_d} \right) = G_{a^t X}(z)
  \]
\end{teorema}
%
\begin{proof}
  El  primer  resultado es  inmediato,  escribiendo \  $z_i^{a_i  X_i  + b_i}  =
  z_i^{b_i}   \left(  z_i^{a_i}   \right)^{X_i}$.    El  secundo   viene  de   \
  $z_i^{X_i+Y_i}    =    z_i^{X_i}    z^{Y_i}$    \   conjuntamente    con    el
  teorema~\ref{Teo:MP:IndependenciaMomentos}   con   \   $f(X)  =   \prod_{i=1}^d
  z_i^{X_i}$  \ y \  $g(Y) =  \prod_{i=1}^d z_i^{Y_i}$.  El tercer  resultado es
  consecuencia de $\prod_{i=1}^d  \left( z^{a_i} \right)^{X_i} = z^{\sum_{i=1}^d
    a_i X_i}$.
\end{proof}
%
Estos  resultados permiten manejar  sencillamente la  medida de  probabilidad de
combinaciones lineales  de vectores aleatorios independientes y  de marginales a
trav\'es esta funci\'on generadora.

\

De  la  tercera identidad,  se  puede  hacer un  paso  m\'as  tratando de  sumas
aleatorias de vectores aleatorios:
%
\begin{teorema}
\label{Teo:MP:SumaAleatoriaGeneradoraProbabilidad}
%
  Sea  \ $X_n,  \quad n  \in  \Nset$ \,  una sucesi\'on  de vectores  aleatorios
  indepedientes de  misma distribuci\'on  (resp.  generadora de  probabilidad) \
  $P_X$ \  (resp. $G_X$)  \ y  \ $N$ \  una variable  definida sobre  \ $\Nset$,
  independiente de los  \ $X_n$. Sea el vector aleatorio \  $ S_N = \sum_{n=0}^N
  X_n$. Entonces
  %
  \[
  G_{S_N}(z) =  G_N \big( G_X(z) \big),
  \]
  %
\end{teorema}
%
\begin{proof}
  Usando la formula de esperanza total del teorema.~\ref{Teo:MP:EsperanzaTotal}, se escribe
  %
  \begin{eqnarray*}
  G_{S_N}(z) & = & \Esp\left[ z^{\sum_{n=0}^N X_n} \right]\\[2.5mm]
  %
  & = & \Esp\left[ \Esp\left[ \left. z^{\sum_{n=0}^N X_n} \right| N  \right] \right]\\[2.5mm]
  %
  & = & \Esp\left[ G_X(z)^N \right]
  \end{eqnarray*}
\end{proof}


% ================================= Funcion generadora de momentos

\subseccion{Funci\'on generadora de momentos}
\label{Ssec:MP:GeneradoraMomentos}


Como lo hemos visto, la  funci\'on generadora de probabilidad permite recucperar
los  momentos  de  un  vector  aleatorio  a trav\'es  de  combinaciones  de  sus
derivadas.   Con una pequa\~na  modificaci\'on, se  puede definir  una funci\'on
generada  permitiendo  recuperar  m\'as  directamente los  momentos,  de  manera
siguiente~\cite{Fel68, JohKot97, Muk00, AthLah06}:
%
\begin{definicion}[funci\'on generadora de momentos]
\label{Def:MP:GeneradoraMomentos}
%
  La {\em  funci\'on generadora  de momentos} (mgf  para {\em  moment generating
    function} en ingles) de un vector aleatorio $d$-dimensional se define como
  %
  \[
  M_X(u) = \Esp\left[ e^{u^t X} \right]
  \]
  %
  para $u \in \Cset^d$.
\end{definicion}
%
De esta definici\'on se nota inmediatamente que
%
\[
M_X(u) = G_X\left( e^u \right)  \qquad \mbox{donde} \qquad e^u = \begin{bmatrix}
  e^{u_1} & \cdots & e^{u_d} \end{bmatrix}^t
\]
%
Entonces, como \ $G_X$, la  generadora de los momentos caracteriza completamente
el vector  aleatorio \ $X$.   Adem\'as, de  este v\'inculo entre  \ $G_X$ \  y \
$M_X$, y del  dominio de definici\'on de \  $G_X$, queda claro que \  $M_X$ \ es
definida sobre un  producto cartesiano de franjas del plano  complejo, $ v_i \le
\real{u_i} \le  V_i$ \ donde \  $v_i \le 0 \le  V_i$ \ llamamos  {\em indices de
  convergencia}.  En el  caso de variables escalares admitiendo  una densidad de
probabilidad \ $p_X$,  denotando $s = -u$, esta funci\'on  se interpreta como la
transformada (bilateral) de Laplace de \ $p_X$.


Se muestra  que esta  funci\'on es  continua en su  dominio de  definici\'on, en
particular en  un entorno de  \ $u =  0$ donde queda  positiva. Eso viene  de la
continuidad  de $x \mapsto  e^{u^t x}$  \ y  de la  convergencia uniforme  de la
integral en el dominio de definici\'on (ver secciones anteriores y el teorema de
convergencia  dominada). Adem\'as,  si admite  un  desarollo de  Taylor en  este
punto, la generadora de los momentos permite recuperar directamente los momentos
a trav\'es de derivadas, sin hacer combinaciones:
%
\begin{lema}
\label{Lem:MP:GeneracionMomentos}
%
  Para cualquier \  \modif{$k > 0, \quad (i_1,\ldots,i_k) \in \{ 1 , \ldots , d \}^k $}
%  \Nset^d$ \ con \ $K =  \sum_{i=1}^d k_i$, 
  derivando $M_X$ se proba que, cuando existen
  %
  \[
  \modif{\left.\frac{\partial^k     M_X}{\partial     u_{i_1}     \cdots     \partial
      u_{i_k}}\right|_{u=0}  = \Esp\left[  \prod_{j=1}^k  X_{i_j} \right]  =
  m_{i_1,\ldots,i_k}[X]}
  \]
  %
  momento de orden \ $k$ \ de \ $X$.
\end{lema}
%
En particular, se recuperan
%
\begin{itemize}
\item $M_X(0) = 1$, condici\'on de normalizaci\'on.
%
\item $\nabla_u M_X(0) = \Esp[X]$ \ promedio,
%
\item $\Hess_u M_X(0) = \Esp\left[ X X^t \right]$, \ie $\Cov[X] = \Hess_u M_X(0)
  - \nabla_u M_X(0) \nabla_u^t M_X(0)$ \ matriz de covarianza.
\end{itemize}

Como la funci\'on \ $G_X$, la  generadora de los momentos tiene unas propiedades
similares  a las  de  los teoremas~\ref{Teo:MP:PropiedadesGeneradoraProbabilidad}
y~\ref{Teo:MP:SumaAleatoriaGeneradoraProbabilidad}:
%
\begin{teorema}%[Funci\'on generadora de probabilidad de una ]
\label{Teo:MP:PropiedadesGeneradoraMomentos}
%
  Sean  \   $X$  \  e  \   $Y$  \  dos   vectores  aleatorios  $d$-dimensionales
  independientes,  \ $A$  \ una  matriz de  \  $\M_{d',d}(\Rset)$ \  y \  $b
  =   \begin{bmatrix}    b_1   &    \cdots   &   b_{d'}    \end{bmatrix}^t   \in
  \Rset^{d'}$.  Entonces para  cualquier $u  =  \begin{bmatrix} u_1  & \cdots  &
    u_{d'} \end{bmatrix}^t \in \Cset^{d'}$ \ (donde la funci\'on existe):
  %
  \[
  M_{A X + b}(u) =  e^{u^t b} M_X\left( A^t u \right),
  \]
  %
  y para cualquier  $u = \begin{bmatrix} u_1 & \cdots  & u_d \end{bmatrix}^t \in
  \Cset^d$ \ (donde la funci\'on existe):
  %
  \[
  M_{X+Y}(u) = M_X(u) \, M_Y(u)
  \]
  %
  Adem\'as,  para \  $X_n, \quad  n  \in \Nset$  \, una  sucesi\'on de  vectores
  aleatorios  indepedientes  de   misma  distribuci\'on  (resp.   generadora  de
  momentos) \ $P_X$ \ (resp. $M_X$) \  y \ $N$ \ una variable aleatoria definida
  sobre \ $\Nset$, independiente de los \ $X_n$, y \ $ S_N = \sum_{n=0}^N X_n$,
  %
  \[
  M_{S_N}(u) =  G_N \big( M_X(u) \big),
  \]
\end{teorema}
%
\begin{proof}
  Las  pruebas  siguen   punto  a  punto  los  mismos  pasos   que  las  de  los
  teoremas~\ref{Teo:MP:PropiedadesGeneradoraProbabilidad}
  y~\ref{Teo:MP:SumaAleatoriaGeneradoraProbabilidad}.
\end{proof}

\

De nuevo, se puede hacer un  paso m\'as tratando de sumas aleatorias de vectores
aleatorios como en el teorema~\ref{Teo:MP:SumaAleatoriaGeneradoraProbabilidad}:
%
\begin{teorema}
\label{Teo:MP:SumaAleatoriaGeneradoraMomentos}
%
  Sea  \ $X_n,  \quad n  \in  \Nset$ \,  una sucesi\'on  de vectores  aleatorios
  indepedientes de  misma distribuci\'on  (resp.  generadora de  probabilidad) \
  $P_X$ \  (resp. $M_X$) \  e \  $N$ \ una  variable aleatoria definida  sobre \
  $\Nset$,  independiente de los  \ $X_n$.  Sea el  vector aleatorio  \ $  S_N =
  \sum_{n=0}^N X_n$. Entonces
  %
  \[
  M_{S_N}(u) =  G_N \big( M_X(u) \big),
  \]
  %
\end{teorema}
%
\begin{proof}
  El         resultado        es         consecuencia         directa        del
  teorema~\ref{Teo:MP:SumaAleatoriaGeneradoraProbabilidad}.
\end{proof}
%\cite{Fra09}%%p.73


% ================================= Funcion caracteristica

\subseccion{Funci\'on caracter\'istica}
\label{Ssec:MP:FuncionCaracteristica}

Si  la funci\'on generadora  de momentos  permite recuperar  los momentos  de un
vector  aleatorio, no  es definida  sobre todo  $\Cset^d$.  Sin  embargo, cuando
$\real{u_i} = 0$,  esta funci\'on es siempre definida.   Entonces, una funci\'on
generadora muy \'util que se usa frecuentemente es la de momentos para este tipo
de argumentos,  lo que es conocida  como funci\'on caracter\'istica y  que es al
final definida  sobre $\Rset^d$  de manera siguiente~\cite{Luk61,  Gol61, Fel68,
  SteWei71, JohKot97, Muk00, AshDol99, AthLah06, Sas13}:


\begin{definicion}[funci\'on   caracter\'istica]
\label{Def:MP:FuncionCaracteristica}
% 
  La {\em funci\'on caracter\'istica}  (cf para {\em characteristic function} en
  ingles) de un vector aleatorio $d$-dimensional se define como
 %
  \[
  \Phi_X(\omega) = \Esp\left[ e^{\imath \omega^t X} \right]
  \]
  %
  para $\omega \in \Rset^d$.
\end{definicion}
%
De esta definici\'on se nota inmediatamente que
%
\[
\Phi_X(\omega) = M_X(\imath \omega) = G_X\left( e^{\imath \omega} \right) \qquad
\mbox{donde} \qquad e^{\imath \omega}  = \begin{bmatrix} e^{\imath u_1} & \cdots
  & e^{\imath u_d} \end{bmatrix}^t
\]
%
De hecho,  se puede definir esta  funci\'on para un argumento  complejo, pero es
equivalente a volver a la definici\'on de la generadora de momentos.

En su forma general, la funci\'on caracter\'istica se escribe
%
\[
\Phi_X(\omega) = \int_{\Rset^d} e^{\imath \omega^t x} \, dP_X(x)
\]
%
y  es  relacionada   a  la  transformada  de  Fourier-Stieltjes   de  la  medida
$P_X$~\cite[Chap.~5]{Pin09}. Cuando  \ $P_X$ \  admite una densidad \  $p_X$, la
funci\'on  es  una  transformada  de  Fourier  usual de  la  densidad  \  $p_X$,
introducida bajo el  impulso de Fourier en 1822 para  estudiar la difusi\'on del
calor~\cite{Fou22}.

Insistamos sobre  el hecho que  la importancia de  esta funci\'on reside  en que
siempre existe y est\'a bien  definida, dado que \ $\displaystyle \int_{\Rset^d}
\left| e^{\imath \omega^t x} \right| \, dP_X(x) = \int_{\Rset^d} dP_X(x) = 1$.
% \cite{Gol61}

Como para las generadoras ya introducidas, la funci\'on caracter\'istica permite
recuperar directamente los momentos a trav\'es de derivadas:
%
\begin{lema}
\label{Lem:MP:GeneracionMomentoViaCaracteristica}
%
  Para cualquier \  \modif{$k > 0, \quad (i_1,\ldots,i_k) \in \{ 1 , \ldots , d \}^k$},
%  \Nset^d$ \  con \  $K =  \sum_{i=1}^d k_i$, 
  derivando $\Phi_X$ se proba que, cuando existen
  %
  \[
  (-   \imath)^k  \,   \left.\frac{\partial^k   \Phi_X}{\partial  \omega_{i_1}
      \cdots    \partial    \omega_{i_k}}\right|_{\omega=0}    =    \Esp\left[
    \prod_{j=1}^k X_{i_j} \right] = m_{i_1,\ldots,i_k}[X]
  \]
  %
  momento de orden \ $k$ \ de \ $X$.
\end{lema}
%
En particular, se recuperan
%
\begin{itemize}
\item $\Phi_X(0) = 1$, condici\'on de normalizaci\'on.
%
\item $- \imath \nabla_\omega M_X(0) = \Esp[X]$ \ promedio,
%
\item  $- \Hess_\omega  M_X(0) =  \Esp\left[ X  X^t \right]$,  \ie $\Cov[X]  = -
  \Hess_\omega M_X(0) + \nabla_\omega M_X(0) \nabla_\omega^t M_X(0)$ \ matriz de
  covarianza.
\end{itemize}

Fijense de  que \  $\Phi_X$ \  no es siempre  diferencial en  $\omega =  0$; Por
ejemplo,    en   el    caso    de   la    distribuci\'on   de    Cauchy--Lorentz
univariada~\footnote{Lo      mismo       occure      en      la      extensi\'on
  multivariada~\cite{SamTaq94}.}\ $p_X(x) =  \frac{\gamma}{\pi \left( \gamma^2 +
    (x-x_0)^2  \right)}$ \  con  \ $\gamma  >  0$, resulta  \ $\Phi_X(\omega)  =
e^{-\imath x_0  \omega -\gamma |\omega}  $. Esta funci\'on est\'a  definida para
todo $\omega$,  pero no es  derivable en  $\omega = 0$,  lo que coincide  con el
hecho  de  que  no  est\'an   definidos  los  momentos  para  esta  densidad  de
probabilidad.

Resumimos algunas otras propiedades importantes de la funci\'on caracter\'istica:
%
\modif{
\begin{teorema}[Propiedades principales de la funci\'on caracter\'istica]
\label{Teo:MP:PropCarac}
\mbox{ }

\begin{enumerate}
\item\label{Prop:MP:ContinuidadPhiX}  $\Phi_X$   es  una  funci\'on   medible  y
  continua en $\Rset^d$~\cite[Prop.~5.2.1]{Pin09}.   Eso es una consecuencia del
  teorema           de            convergencia           dominada           (ver
  teorema~\ref{Teo:MP:ConvergenciaDominada}
  pagina~\pageref{Teo:MP:ConvergenciaDominada}).
%
\item\label{Prop:MP:PhiXZero} $\Phi_X(0) = 1  $: Eso es inmediato escribiendo la
  integral, siendo $P_X$ una medida de probabilidad.
%
\item\label{Prop:MP:MaximoPhiX}   $\left|  \Phi_X(\omega)   \right|   \le  1   =
  \Phi_X(0)$: \  $\left| \Phi_X(\omega)  \right|$ es m\'axima  en $\omega  = 0$.
  Eso viene directamente de $\left| e^{\imath \omega^t x} \right| = 1$.
%
\item\label{Prop:MP:HermiticaPhiX}    $\Phi_X(-\omega)    =   \Phi_X^*(\omega)$:
  $\Phi_X$ tiene una s\'imetria hermitica.
%
\item  $\Phi_X$ es  una no negativa  definida, \ie  para  un conjunto
  arbitrario de \ $n  \ge 1$ \ n\'umeros complejos \ $a_1 , \ldots  , a_n$ \ y \
  $n$ \ vectores \ $w_1 , \ldots , w_n$ \ de \ $\Rset^d$, se cumple
  \[
  \sum_{k,l=1}^n a_k^* a_l \Phi_X(w_l-w_k) \ge 0
  \]
  %
  Dicho   de   otra   manera,   la   matriz  de   componente   \   $(k,l)$-esima
  $\Phi_X(w_l-w_k)$ es a hermitica (s\'imetria herm\'itica dada por la propiedad
  anterior, y  no negativa  definida). Esta positividad  viene de  \ $\sum_{k,l}
  a_k^*  a_l e^{\imath (w_l-w_k)^t  x} =  \left| \sum_l  a_l e^{\imath  w_l^t x}
  \right|^2 \ge 0$.
%
\end{enumerate}
\end{teorema}

De hecho, existe una reciproca de este teorema, debido a S. Bochner~\footnote{De
  hecho, lo prob\'o Bochner en el caso escalar $d = 1$, pero se extiende al caso
  multivariado.}~\cite{Boc32, Boc59, Gol61, Pin09, Sas13}
%
\begin{teorema}[Bochner]\label{Teo:MP:Bochner}
%
  Una funci\'on \ $\Phi: \Rset^d \mapsto \Cset$ \ es continua, definida no negativa
  (con  \ $\Phi(0)  =  1$)  si y  solamente  existe una  medida  \  $\mu$ \  (de
  probabilidad) sobre \ $\B(\Rset^d)$ \ tal que
  %
  \[
  \forall \:  \omega \in \Rset^d, \quad \Phi(\omega)  = \int_{\Rset^d} e^{\imath
    \, \omega^t x} d\mu(x)
  \]
  %
  Dicho de  otra manera, cualquier  funci\'on continua, definida positiva  con \
  $\Phi(0) = 1$ \ es la funci\'on caracter\'istica de un vector aleatorio.
\end{teorema}
%
%\begin{proof}
En el  teorema, vimos que la  transformada de Fourier-Stieljes de  una medida de
probabilidad $P_X$  es medible, continua,  definida no negativa, con  $\Phi(0) =
1$. La  reciproca es  m\'as dificil  a probar y  necesita lemas  adicionales. Se
puede  encontrar una  linda prueba  en~\cite[Sec.~1.7]{Sas13} adonde  dejamor el
lector.
%\end{proof}

Como   lo  hemos   notado   en  las   subsecciones   anteriores,  la   funci\'on
caracter\'istica define  completamente el  vector aleatorio. En  particular, hay
una relaci\'on uno-uno (casi siempre) entre \ $\Phi_X$ \ y la medida \ $P_X$. En
particular, existe una f\'ormula de inversi\'on permitiendo volver a la medida \
$P_X$ \ a partir de \ $\Phi_X$~\cite{AshDol99, Sas13}:
%
\begin{teorema}[F\'ormula de inversi\'on]\label{Teo:MP:InversionFourierStieljes}
%
  Sea \ $X$  \ vector aleatorio $d$-dimensional de  funci\'on caracter\'istica \
  $\Phi_X$.   Sea  \  $\displaystyle  A  = \optimes_{i=1}^d  (a_i  \;  b_i)  \in
  \B(\Rset^d)$ \ y \ $\partial A  = \optimes_{i=1}^d [a_i \; b_i] \setminus A$ \
  su borde. Entonces~\footnote{Se prolonga  la funci\'on \ $\frac{e^{- \imath \,
        a_j \omega_j} - e^{- \imath  \, b_j \omega_j}}{\imath \, \omega_j}$ \ en
    \ $\omega_j  = 0$ \ por  su l\'imite \ $\displaystyle  \lim_{\omega_j \to 0}
    \frac{e^{- \imath \, a_j \omega_j} - e^{- \imath \, b_j \omega_j}}{\imath \,
      \omega_j} = b_j - a_j$.},
  %
  \[
  \begin{array}{c}
  %
  \displaystyle \lim_{w_1 \to +\infty,\ldots, w_d \to \infty} \: \frac{1}{(2
  \pi)^d} \int_{\optimes_{j=1}^d [-w_j \; w_j]} \Phi_X(\omega) \prod_{j = 1}^d
  \frac{e^{- \imath \, a_j \omega_j} - e^{- \imath \, b_j \omega_j}}{\imath \,
  \omega_j} \: d\omega\\[2.5mm]
  %
  =\\[2.5mm]
  %
   \displaystyle \int_{\Rset^d} \prod_{j=1}^d \left(
     \un_{\left(  a_j \;  b_j \right)}(x_j)  +  \frac12 \un_{\left\{  a_j \;  b_j
       \right\}}(x_j) \right) dP_X(x)
  \end{array}
  \]
  %
  En particular, cuando $P_X$ vale 0  sobre el borde de $A$, es decir $P_X\left(
  \partial A \right) = 0$, se obtiene
  %

  \[
  \lim_{w_1  \to   +\infty,\ldots,  w_d  \to  \infty}   \:  \frac{1}{(2  \pi)^d}
  \int_{\optimes_{j=1}^d   [-w_j  \;  w_j]}   \Phi_X(\omega)  \prod_{j   =  1}^d
  \frac{e^{- \imath \,  a_j \omega_j} - e^{- \imath  \, b_j \omega_j}}{\imath \,
    \omega_j} \, d\omega \: = \: P_X(A).
  \]
\end{teorema}
%
Nota: el l\'imite \ $\displaystyle\lim_{T  \to +\infty} \int_{-T}^T$ \ se nota a
veces \ $\displaystyle \operatorname{vp} \!  \int_\Rset$, integral {\it en valor
  principal}.
%
\begin{proof}
  Por definici\'on de la funci\'on caracter\'istica, tenemos
  %
  \[
  \int_{\optimes_{j=1}^d   [-w_j  \;  w_j]} \Phi_X(\omega)  \prod_{j = 1}^d  \frac{e^{- \imath  \, a_j
      \omega_j} - e^{- \imath \, b_j \omega_j}}{\imath \, \omega_j} \, d\omega =
  \int_{\optimes_{j=1}^d   [-w_j  \;  w_j]} \int_{\Rset^d} e^{\imath  \omega^t x} \, dP_X(x) \prod_{j =
    1}^d  \frac{e^{- \imath a_j  \omega_j} -  e^{- \imath  b_j \omega_j}}{\imath
    \omega_j} \, d\omega
  \]
  %
  Ahora, notando que $\left| \frac{e^{- \imath \, a_j \omega_j} - e^{- \imath \,
        b_j \omega_j}}{\imath  \, \omega_j} \, e^{\imath \,  \omega^t x} \right|
  \le  b_j - a_j$  \ es  uniformamente acotado,  se puede  evocar el  teorema de
  Fubini~\ref{Teo:MP:Fubini} para intercambiar las  integrales, as\'i que, con \
  $e^{\imath \, \omega^t x} = \prod_{j=1}^d e^{\imath \, \omega_j x_j}$, tenemos
  %
  \[
  \int_{\optimes_{j=1}^d   [-w_j  \;  w_j]}   \Phi_X(\omega)  \prod_{j   =  1}^d
  \frac{e^{- \imath \,  a_j \omega_j} - e^{- \imath  \, b_j \omega_j}}{\imath \,
    \omega_j} \, d\omega = \int_{\Rset^d} \left( \prod_{j=1}^d \int_{-w_j}^{w_j}
    \frac{e^{- \imath \, \omega_j (a_j -  x_j)} - e^{- \imath \, \omega_j (b_j -
        x_j)}}{\imath \, \omega_j} \, d\omega_j \right) \, dP_X(x)
  \]
  %
  Se nota que
  %
  \begin{eqnarray*}
  \int_{-w_j}^{w_j} \frac{e^{- \imath \, \omega_j (a_j - x_j)} - e^{- \imath \, \omega_j
  (b_j - x_j)}}{\imath \, \omega_j} \, d\omega_j & = & - \int_{-w_j}^{w_j} \frac{e^{+
  \imath \, \omega_j (a_j - x_j)} - e^{+ \imath \, \omega_j (b_j - x_j)}}{\imath \,
  \omega_j} \, d\omega_j\\[2mm]
  %
  & = & \int_{-w_j}^{w_j} \frac{\sin(\omega_j (b_j - x_j)) - \sin(\omega_j (a_j -
  x_j))}{\omega_j} \, d\omega_j
  \end{eqnarray*}
  %
  por cambio de variables $\omega_j \to - \omega_j$ en la primera linea, tomando
  entonce  la media  suma de  los  terminos derecho/izquierdo  dando la  secunda
  linea. Seguimos notando que
  %
  \[
  \int_{-w}^w  \frac{\sin(\omega (c  - x))}{\omega}  \, d\omega  = \sign(c  - x)
  \int_{-w |c - x|}{w |c - x|} \frac{\sin(\omega)}{\omega} \, d\omega
  \]
  %
  es  decir, de  \ $\displaystyle  \lim_{T \to  +\infty}  \int_{-T}^T \frac{\sin
    \omega}{\omega} d\omega = \pi$~\cite[Ec.~3.721]{GraRyz15}, se obtiene
  %
  \[
  \lim_{w \to  +\infty} \int_{-w}^w \frac{\sin(\omega  (c - x))}{\omega}
  \, d\omega = \pi \sign(c-x)
  \]
  %
  Se  acaba  la   prueba  de  \  $\left|  \frac{\sin(\omega_j   (b_j  -  x_j)  -
      \sin(\omega_j  (a_j -  x_j))}{\omega_j} \right|  < 2$  \  conjuntamente al
  teorema de convergencia dominada~\ref{Teo:MP:ConvergenciaDominada} permitiendo
  permutar integral  y l\'imite,  y de $\sign(b_j-x_i)  - \sign(a_j-x_i) =  2 \,
  \un_{\left(   a_j   \;  b_j   \right)}(x_j)   +   \un_{\left\{   a_j  \;   b_j
    \right\}}(x_j)$.
\end{proof}

Dos teoremas de inversi\'on en los casos particular continuo y discreto permiten
respectivamente  volver  a   la  densidad  de  probabilidad  o   a  la  masa  de
probabilidad.
%
\begin{teorema}[Inversi\'on, caso continuo]\label{Teo:MP:InversionDensidad}
%
  Si \ $\Phi_X$ \ es integrable, entonces \ $P_X$ \ admite una densidad tal que
  %
  \[
  p_X(x)  =  \frac{1}{(2  \pi)^d}  \int_{\Rset^d} \Phi(\omega) \, e^{-  \imath  \,
    \omega^t x} \, d\omega
\]
\end{teorema}
%
\begin{proof}
  Varias pruebas  existen (ej.~\cite[p.~21]{Sas13}). Una  bastante directa puede
  ser       de      salir      de       la      f\'ormula       general      del
  teorema~\ref{Teo:MP:InversionFourierStieljes}, de  fijar \ $a$ \ y  \ poner $b
  \equiv x \in  \Rset^d$. Se toma la derivada  \ $\frac{\partial^d}{\partial x_1
    \cdots \partial x_d}$ \ de  la integral \ $\displaystyle \frac{1}{(2 \pi)^d}
  \int_{\optimes_{j=1}^d   [-w_j  \;  w_j]}   \Phi_X(\omega)  \prod_{j   =  1}^d
  \frac{e^{- \imath \,  a_j \omega_j} - e^{- \imath  \, b_j \omega_j}}{\imath \,
    \omega_j} \: d\omega$ \ y se  evoca el teorema de convergencia dominada para
  intercambiar  derivaci\'on e integraci\'on,  y luego  tomar el  l\'imite, para
  obtener el resultado.
\end{proof}

\begin{teorema}[Inversi\'on, caso discreto]\label{Teo:MP:InversionMasa}
%
  Para cualquier $x \in \Rset^d$,
  %
  \[
  \lim_{w_1  \to  \infty,\ldots,w_d \to  \infty} \,  \frac{1}{2^d  w_1 \ldots  w_d}
  \int_{\optimes_{j=1}^d [-w_j \; w_j]}  \Phi(\omega) \, e^{- \imath \, \omega^t x}
  \, d\omega = P_X(x)
  \]
\end{teorema}
%
\begin{proof}
  Por definici\'on de  la funci\'on caracter\'istica, y aplicando  el teorema de
  Fubini como en el caso general (mismo enfoque),
  %
  \begin{eqnarray*}
  \frac{1}{2^d w_1 \ldots w_d} \int_{\optimes_{j=1}^d [-w_j \; w_j]}
  \Phi_X(\omega) e^{- \imath \omega^t x} \, d\omega & = & \frac{1}{2^d w_1 \ldots
  w_d} \int_{\optimes_{j=1}^d [-w_j \; w_j]} \int_{\Rset^d} e^{\imath \omega^t y}
  \, dP_X(y) \, e^{- \imath \omega^t x} d\omega\\[2mm]
  %
  & = & \int_{\Rset^d} \left( \prod_{j=1}^d \frac{1}{2 w_j} \int_{-w_j}^{w_j}
  e^{\imath \, (y_j - x_j) w_j} dw_j \right) dP_X(y)\\[2mm]
  %
  & = & \int_{\Rset^d} \prod_{j=1}^d \frac{\sin(w_j (y_j-x_j))}{(y_j-x_j) w_j}  dP_X(y)
  \end{eqnarray*}
  %
  con   el  l\'imite   \  $\displaystyle   \lim_{y_j  \to   x_j}  \frac{\sin(w_j
    (y_j-x_j))}{w_j( y_j-x_j)}  = 1$. Ad\'emas, con  el mismo enfoque  que en el
  cas general acotando  el integrande, por el teorema  de convergencia dominada,
  se  puede intercambiar  l\'imite  e integral,  as\'i  que, por  $\displaystyle
  \lim_{x_j   \to   \infty}    \frac{\sin(w_j   (y_j-x_j))}{w_j(   y_j-x_j)}   =
  \un_{x_j}(y_j)$, lo cierra la prueba
\end{proof}

Como las  funci\'ones \ $G_X$ \  y \ $M_X$, la  funci\'on caracter\'istica tiene
entre    otros    propiedades    similares     a    las    a    las    de    los
teoremas~\ref{Teo:MP:PropiedadesGeneradoraMomentos}
y~\ref{Teo:MP:SumaAleatoriaGeneradoraMomentos}:
%
\begin{teorema}%[Funci\'on generadora de probabilidad de una ]
\label{Teo:MP:PropiedadesFuncionCaracteristica}
%
  Sean  \   $X$  \  e  \   $Y$  \  dos   vectores  aleatorios  $d$-dimensionales
  independientes,  \ $A$  \ una  matriz de  \  $\M_{d',d}(\Rset)$ \  y \  $b
  =  \begin{bmatrix} b_1  &  \cdots &  b_{d'}  \end{bmatrix}^t \in  \Rset^{d'}$.
  Entonces  para  cualquier  $\omega  =  \begin{bmatrix}  \omega_1  &  \cdots  &
    \omega_{d'} \end{bmatrix}^t \in \Rset^{d'}$:
  %
  \[
  \Phi_{A X + b}(\omega) =  e^{\imath \omega^t b} \Phi_X\left( A^t \omega \right),
  \]
  %
  y   para   cualquier  $\omega   =   \begin{bmatrix}   \omega_1   &  \cdots   &
    \omega_d \end{bmatrix}^t \in \Rset^d$:
  %
  \[
  \Phi_{X+Y}(\omega) = \Phi_X(\omega) \, \Phi_Y(\omega)
  \]
  %
  Adem\'as,  para \  $X_n, \quad  n  \in \Nset$  \, una  sucesi\'on de  vectores
  aleatorios  indepedientes  de   misma  distribuci\'on  (resp.   generadora  de
  momentos) \ $P_X$ \ (resp. $M_X$) \  e \ $N$ \ una variable aleatoria definida
  sobre \ $\Nset$, independiente de los \ $X_n$, y \ $ S_N = \sum_{n=0}^N X_n$,
  %
  \[
  \Phi_{S_N}(\omega) =  G_N \big( \Phi_X(\omega) \big),
  \]
\end{teorema}

Gracia a la  funci\'on caractr\'istica, se muestra tambi\'en  que para un vector
aleatorio $d$-dimensional \ $X$, conocer  la distribuci\'on de cualquier $a^t X$
para $a  \in \Rset^d$  permite conocer la  distribuci\'on de  \ $X$~\cite{Mui82,
  Sas13}:
%
\begin{teorema}
  Sea \ $X$ \ vector aleatorio $d$-dimensional. La distribuci\'on \ $p_X$ \ de \
  $X$  \  es completamente  determinada  por  el  conjunto de  distribuciones  \
  $\left\{ p_{a^t X},  \: \forall \: a \in  \Rset^d, a \ne 0 \right\}$  \ de las
  variables \ $a^t X$.
\end{teorema}
%
\begin{proof}
  Por definici\'on  de la funci\'on caracter\'istica, $\forall \: \omega \in \Rset$
  %
  \begin{eqnarray*}
  \Phi_{a^t X}(\omega) & = & \Esp\left[ e^{\imath \, \omega a^t X} \right]\\[2mm]
  %
  & = & \Phi_X(\omega a)
  \end{eqnarray*}
  %
  Se concluye  notando que \ $\left\{  a \omega \tq a  \in \Rset^d, a  \ne 0, \:
    \omega \in \Rset \right\} = \Rset^d$.
\end{proof}

Un otro resultado  interesante se v\'incula a la noci\'on de  mezcla de escala y
toma la forma siguiente:
%
\begin{teorema}
  Sea \  $X$ \ vector aleatorio de  funci\'on caracter\'istica \ $\Phi_X$  \ y \
  $R$ \ variable aleatoria independiente de  \ $X$ \ y de medida de probabilidad
  \ $P_R$. Entonces, la funci\'on caracter\'istica de \ $R X$ \ es dada por
  %
  \[
  \Phi_{R X}(\omega) = \int_\Rset \Phi_X(r \omega) \, dP_R(r)
  \]
\end{teorema}
%
\begin{proof}
  Por definici\'on  de la funci\'on caracter\'istica, la  f\'ormula de esperanza
  total~\ref{Teo:MP:EsperanzaTotal}                     y                     el
  teorema~\ref{Teo:MP:EsperanzaCondicionalFXY}, se tiene
  %
  \begin{eqnarray*}
  \Phi_{R X}(\omega) & = & \Esp\left[ e^{\imath \, \omega^t R X} \right]\\[2mm]
  %
  & = & \Esp\left[ \Esp\left[ \left. e^{\imath \, \omega^t R X} \right| R \right] \right]\\[2mm]
  %
  & = & \int_\Rset \Esp\left[ \left. e^{\imath \, \omega^t r X} \right| R=r \right] dP_R(r)\\[2mm]
  %
  & = & \int_\Rset \Esp\left[ e^{\imath \, \omega^t r X} \right] dP_R(r)\\[2mm]
  %
  & = & \int_\Rset \Phi_X(r \omega) \, dP_R(r)
  \end{eqnarray*}
\end{proof}
 
De la funci\'on  caracter\'istica (o tambi\'en de la  generadoda de momentos) se
puede probar resultados sobre la escritura estocastica de vectores aleaorios que
se va a revelar frecuentement muy \'util:
%
\begin{teorema}\label{Teo:MP:IgualdadDistribucionFuncionVA}
  Sean  dos vectores  aleatorios $d$-dimensional  \ $X$  \ e  \ $Y$  \  de misma
  distribuci\'on  de  probabilidad. Entonces,  para  cualquier  funci\'on \  $f:
  \Rset^d  \to \Rset^{d'}$ \  medible, $f(X)$  \ y  \ $f(Y)$  \ tienen  la misma
  distribuci\'on de probabilidad ($d'$ puede  ser menor, igual o mayor que $d$),
  \ie
  %
  \[
  X  \egald  Y \quad  \Longrightarrow  \quad  f(X) \, \egald \, f(Y)
  \]
\end{teorema}
%
\begin{proof}
  Este resultado se encuentra entre otros en~\cite[aserci\'on~2~p.~13]{FanKot90}
  o \cite{Zol86} y se proba sencillamente por la funci\'on caracter\'istica:
  %
  \[
  \Phi_{f(X)}(\omega)  =  \Esp\left[  e^{\imath  \,  \omega^t  f(X)}  \right]  =
  \int_{\Rset^d}  e^{\imath  \,  \omega^t  f(x)}  \,  dP_X(x)  =  \int_{\Rset^d}
  e^{\imath \, \omega^t f(x)} \, dP_Y(x) = \Phi_{f(Y)}(\omega)
  \]
  %
  Se concluye  del caract\'er uno-uno  entre la funci\'on caracter\'istica  y la
  medida de probabilidad.
\end{proof} 
}

%\SZ{Hablar de la forma en el caso escamar con le funcion quantile?}

%{\teorema (Bochner, Goldberg).... } %%


% ================================= Funcion generadora de momentos

\modif{
\subseccion{Funci\'on generadora de cumulantes y secunda funci\'on caracter\'istica}
\label{Ssec:MP:GeneradoraCumulantes}

A  veces aparece  m\'as comodo  trabajar con  momentes especiales  llamados {\em
  cumulantes}.   Esos   fueron  introducidos  T.   N.   Thiele   en  los  a\~nos
1889~\cite{Thi89,   Thi03,  Cram46,   Hal00,   Hal98}  y   aparecen  bajo   esta
denominaci\'on   en  un  papel   de  R.   Fisher  \&   J.   Wishart   4  decadas
despu\'es~\cite{FisWis32}.  Estos momentos aparecen  a trav\'es del logaritmo de
la   funci\'on  generadora   de  momentos.    Como  lo   vamos  a   ver   en  la
seci\'on~\ref{Ssec:MP:FamiliaExponencial}  tratando de  la  familia exponencial,
aparece que el logarotmo de  la generadora de momentos tiene una significaci\'on
f\'isica,  permitiendo  de  calcular   energias  libre  o  interna  en  f\'isica
estadistica  a  trav\'es  de  lo  que conocido  como  funci\'on  de  partici\'on
(ver~\cite{Gib01,  Gib02,  Fis30}). Se  refiera  tambi\'en a~\cite{Luc70,  titi,
  toto, Amblard} para lo que sigue.
% Gibbs 1901, Fisher 1929

\begin{definicion}[Generadora de los cumulantes]
  Sea \ $X$ \ vector aleatorio y  \ $M_X(u) = \Esp\left[ e^{u^t X} \right]$ \ su
  generadora de momentos. Etonces, se define la funci\'on {\em generadora de los
  cumulantes} como
  %
  \[
  C_X(u) = \log M_X(u) = \log\left( \Esp\left[ e^{u^t X} \right] \right)
  \]
\end{definicion}
%
Del hecho que que $M_X$ es compleja, hace falta entender el logaritmo como de un
n\'umero  complejo~\cite{Abl03,  CarKro05}.  De  la continuidad  de  $M_X$,  con
$M_X(0) = 1$, la generadora de los  momentos es real positiva por lo menos en un
enterno de $u = 0$, as\'i que  la generadora de los cumulantes va a ser definida
por lo menos en un enterno de $u = 0$.

A partir de esta definici\'on, se define los dichos {\em cumulantes} de la misma
manera que para los momentos:
%
\begin{definicion}[Cumulantes]
  Sea  \ $X$  \ vector  aleatorio $d$-dimensional  y $C_X(u)  = \log  M_X(u)$ la
  generadora de  los momentos.  Esta funci\'on  es definida por  lo menos  en un
  entorno  de $u  = 0$,  $C_X(0)  = 0$  \ y  si  admite en  desarollo de  Taylor
  alrededor de $u = 0$, se escribe
  %
  \[
  C_X(u)     =    \sum_{k=1}^{+\infty}     \,     \sum_{(i_1,\ldots,i_k)    \in     \{1 , \ldots , d \}^k}
  \kappa_{i_1,\ldots,i_k}[X] \, \frac{u_{i_1} \ldots u_{i_k}}{k!}
  \]
  %
  donde  el  tensor  \ $\kappa_k[X]$  \  de  order  \  $k$  \ y  de  componentes
  $\kappa_{i_1,\ldots,i_k}[X]$ llamados {\em cumulentes},
  %
  \[
  \kappa_{i_1,\ldots,i_k}[X] \, = \, \left.\frac{\partial^k C_X}{\partial u_{i_1}
      \ldots \partial u_{i_k}} \right|_{u = 0}
  \]
  %
  es llamado {\em cumulente de orden $k$ de $X$.}
\end{definicion}

Se podra notar que
%
\begin{itemize}
\item $\kappa_1[X] = \nabla C_X(0) = m_X$ \ media de $X$;
%
\item $\kappa_2[X] = \Hess C_X(0) = \Sigma_X = \zeta_2[X]$ \ covarianza de $X$;
%
\item $\kappa_3[X] = \zeta_3[X]$ \ momento centrado de orden 3;
%
\item Pero, $\forall \: k > 3, \quad \kappa_k[X] \ne \zeta_k[X]$.
\end{itemize}

Sin embargo, en  el caso escalar \  $d=1$ \ de \ $m_k[X]  = \left. \frac{\partial^k
    M_X}{\partial     u^k}    \right|_{u=0}    =     \left.     \frac{\partial^k
    \exp(C_X)}{\partial   u^k}   \right|_{u=0}$   \   o   $\kappa_k[X]   =   \left.
  \frac{\partial^k C_X}{\partial  u^k} \right|_{u=0} =  \left.  \frac{\partial^k
    \log(C_X)}{\partial u^k} \right|_{u=0}$. Se relacionan cumulentes y momentos
de  la  formula  de  Fa\`a   di  Bruno~\footnote{Esta  formula  es  asociada  al
  matem\'atico italiano Francesco  Fa\`a di Bruno que publico  su formula en los
  a\~nos 1855, 1857.  Pero fue  probada desde 1800 por el matem\'atico franc\'es
  Louis Fran\c{c}ois Antoine Arbogast~\cite{Abo00}}  que da las derivadas de una
funci\'on compuesta~\cite{Faa55,  Faa57} mediante los  polinimios incompletos de
Bell~\cite{Bel27}, dados para $n \ge k$
%
\[
B_{n,k}(x_1,\ldots,x_{n-k+1})  = n!   \sum_{j  \in D_{n,k}}  \prod_{i=1}^{n-k+1}
\frac{1}{j_j!} \left( \frac{x_i}{i!} \right)^{j_i}
\]
%
con
%
\[
D_{n,k}  =  \left\{ j  \in  \Nset^{n-k+1} \tq  \sum_{i=1}^{n-k+1}  j_i  = k  \et
  \sum_{i=1}^{n-k+1} i j_i = n \right\}
\]

\begin{lema}[Relaci\'on momentos-cumulantes -- caso escalar]
  Sea  \  $X$ \  variable  aleatoria escalar.  Cuando  existen,  los momentos y
  cumulantes son relacionados por las formulas
%
\[\left\{\begin{array}{lll}
m_k[X] & = & \displaystyle \sum_{n=1}^k B_{k,n}( \kappa_1[X] , \ldots , \kappa_{n-k+1}[X]) \\[2mm]
%
\kappa_k[X] & = & \displaystyle \sum_{n=1}^k (-1)^n (n-1)! B_{k,n}( m_1[X] , \ldots , m_{n-k+1}[X])
\end{array}\right.\]
%
Notando que \ $C_{X-m_X}(u) = -u m_X + C_X(u)$ \ se obtiene inmediatamente
%
\[\left\{\begin{array}{lll}
\zeta_k[X] & = & \displaystyle \sum_{n=1}^k B_{k,n}( 0 , \kappa_2[X] , \ldots , \kappa_{n-k+1}[X]) \\[2mm]
%
\kappa_k[X] & = & \displaystyle \sum_{n=1}^k (-1)^n (n-1)! B_{k,n}( 0 , \zeta_2[X] , \ldots , \zeta_{n-k+1}[X])
\end{array}\right.\]
\end{lema}
%
De hecho,  se podri\'a definir as\'i  de manera {\em ad-hoc},  los cumulantes que
las generadoras admiten o no desarollos de Taylor en $u = 0$.

La  generalizaci\'on  en el  caso  multivariado  es  posible, mediante  calculos
bastante   pesados,  usando   el  mismo   enfoque  de   derivada   de  funciones
compuestas~\cite[Teo.~5.1.4]{Sta99} o~\cite{Har06,  LacAmb97, Bri01, Shi84}. Eso
conduce a las relaciones siguientes (formula de Leonov y Shiryayev):
%
\begin{lema}[Relaci\'on momentos-cumulantes -- caso general]
  Sea \  $X$ \ vector  aleatoro. Cuando existen,  los momentos y  cumulantes son
  relacionados por las formulas
  %
  \[\left\{\begin{array}{lll}
  m_{i_1,\ldots,i_k}[X] & = & \displaystyle \sum_{\pi \in \Pi_k} \prod_{B \in
  \pi} \kappa_{i_B}[X] \\[5mm]
  %
  \kappa_{i_1,\ldots,i_k}[X] & = & \displaystyle \sum_{\pi \in \Pi_k}
  (-1)^{|\pi|-1} \Gamma(|\pi|) \prod_{B \in \pi} m_{i_B}[X]
  \end{array}\right.\]
  %
  donde \  $\Pi_k$ \  es el conjunto  de las  particiones~\footnote{Por ejemplo,
    $\Pi_3  = \Big\{  \big\{ \{1,2,3\}  \big\} \:  , \:  \big\{ \{1,2\}  , \{3\}
    \big\} \: , \: \big\{ \{1,3\} ,  \{2\} \big\} \: , \: \big\{ \{2,3\} , \{1\}
    \big\}$; cuando \ $\pi = \big\{ \{1,2\} , \{3\} \Big\}$, tenemos el producot
    para \ $B = \{1,2\}$ con los indices \  $i_1,i_2$ \ y \ $B = \{3\}$ \ con el
    indice $i_3$.}  de \ $\{  1 ,  \ldots , k  \}$ \ y  \ $m_{j_B}$ \  denota el
  momento de indices $\{ j_l \tq \: l \in B \}$.

  Notando  que \ $C_X(u)  = u^t  m_X +  \log \left(  M_{X-m_X}(u) \right)$  \ se
  obtiene inmediatamente
  %
  \[\left\{\begin{array}{lll} \zeta_{i_1,\ldots,i_k}[X] & = & \displaystyle
  \sum_{\pi \in \Pi_k} \prod_{B \in \pi} \Big( \kappa_{i_B}[X] + \big( m_{X_{i_B}}
  - \kappa_{i_B}[X] \big) \un_{\{1\}}(|B|) \Big) \\[2mm]
  %
  \kappa_{i_1,\ldots,i_k}[X]  &  = &  \displaystyle  \sum_{\pi \in  \Pi_k}
  (-1)^{|\pi|-1}  \Gamma(|\pi|) \prod_{B \in  \pi} \Big(  \zeta_{i_B}[X] +
  \big( m_{_{i_B}} - \zeta_{i_B}[X] \big) \un_{\{1\}}(|B|) \Big)
  \end{array}\right.\]
  %
\end{lema}
%
\begin{proof}
  Recordamosnos que para \ $k \in \Nset^*, \quad (i_1 , \ldots , i_k) \in \{ 1 ,
  \ldots , d \}^k$,
  %
  \[
  m_{i_1,\ldots,i_k}[X]    =   \left.\frac{\partial^k    M_X}{\partial   u_{i_1}
      \cdots    \partial    u_{i_k}}\right|_{u=0}    \qquad   \mbox{y}    \qquad
  \kappa_{i_1,\ldots,i_k}[X]  =   \left.\frac{\partial^k  C_X}{\partial  u_{i_1}
      \cdots \partial u_{i_k}}\right|_{u=0}
  \]
  %
  Salimos ahora  de la formula de  Hardy~\cite[Prop.~1]{Har06}, generalizando la
  formula de  Fa\`a di  Bruno~\cite{Faa55, Faa57}: Para  \ $h(x)  = f\left(
    g(u) \right), \quad  \forall \: n \in \Nset^*, \quad  \forall \: (i_1 ,
  \ldots , i_n ) \in \{ 1 , \ldots , d \}^n$,
  %
  \[
  \frac{\partial^n h}{\partial u_{i_1} \cdots  \partial u_{i_n}} = \sum_{\pi \in
    \Pi_n} f^{(|\pi|)}\left( g(u) \right) \prod_{B \in \pi} \frac{\partial^{|B|}
    g}{\displaystyle \prod_{j \in B} \partial u_{i_j}}
  \]
  %
  donde \ $f^{(l)}$ \ es la $l$-esima derivada de \ $f$. Cuando se lo aplica a \
  $g = C_X$ \  y \ $f(u) = \exp(u)$, dando \  $f^{(n)}(u) = \exp(u)$, se obtiene
  inmediatamente  la   relaci\'on  dando  los  momentos  en   funci\'on  de  los
  cumulantes. Al rev\'es, cuando  se lo aplica a \ $g = M_X$ \  y \ $f(u) = \log
  u$,  dando  \  $f^{(n)}(u)  = \frac{(-1)^{n-1}  \Gamma(n)}{u^n}$,  se  obtiene
  inmediatamente  la  relaci\'on  dando  los  cumulantes  en  funci\'on  de  los
  momentos.

  Para lo de los momentos centrales, se nota simplemente que para \ $k = 1$ \ el
  cumulante y el momento coinciden, y que para $k > 1$ lo cumulantes de $X$ y de
  $X-m_X$ coinciden.
  % que, de $\frac{\partial^n e^{u^t
  %     m_X}$. Reciprocamente,  \ $\frac{\partial \Psi}{\partial  \omega_i}(0) =
  %   \imath m_{X_i}$ \  y \ para \ $n  > 1$, $\frac{\partial^n \Psi_X}{\partial
  %     \omega_{i_1}   \cdots  \partial   \omega_{i_n}}(0)   =  \frac{\partial^n
  %     \log(\Phi_{X-m_X})}{\partial      \omega_{i_1}      \cdots      \partial
  %     \omega_{i_n}}(0)$
\end{proof}
%
% Se notar\'a  que para \ $k = 1$  \ el cumulante y el  momento coinciden, y que
%  para $k  > 1$  lo cumulantes  de $X$  y de  $X-m_X$ coinciden,  as\'i  que se
% vincular\'a  sencillamente cumulantes y momentos centrales  con estas formulas
% Para lo de los momentos centrales, se nota simplemente que para \ $k = 1$ \ el
% cumulante y el momento coinciden, y que para $k > 1$ lo cumulantes de $X$ y de
% $X-m_X$ coinciden.

\

Para cerra esta secci\'on, se puede evocar  el hecho de que la generadora de los
momentos no  esta siempre  bien definida sobre  todo $\Cset^d$, mientras  que la
funci\'on caracter\'istica es  siempre bien definida. A veces, y  a pesar de que
no sea necesario (el comportamiento en $u=0$ es importante), se usa el logaritmo
de la funci\'on caracter\'istica para definir los cumulantes.
%
\begin{definicion}[Secunda funci\'on caracter\'istica]
  Sea  \ $X$  \  vector aleatorio  y  \ $\Phi_X(\omega)  = \Esp\left[  e^{\imath
      \omega^t X}  \right]$ \ su funci\'on caracter\'istica.  Etonces, se define
  la {\em secunda funci\'on caracter\'istica} como
  %
  \[
  \Psi_X(\omega)  =  \log   \Phi_X(\omega)  =  \log\left(  \Esp\left[  e^{\imath
        \omega^t X} \right] \right)
  \]
\end{definicion}
%
De  nuevo  del hecho  que  que  $\Phi_X$ es  compleja,  hace  falta entender  el
logaritmo  como   de  un  n\'umero  complejo~\cite{Abl03,  CarKro05}   y  de  la
continuidad de $\Phi_X$,  con $\Phi_X(0) = 1$, la  funci\'on caracter\'istica es
real positiva por lo menos en un enterno de $u = 0$, as\'i que la $Psi$ va a ser
definida  por lo menos  en un  enterno de  $u =  0$. Si  admite un  desarollo de
Taylor,  se muestra inmediatamente  que los  cumulantes satisfacen  tambi\'en la
relaci\'on siguiente:
%
\begin{lema}[Cumulantes a partir de la secunda funci\'on caracter\'istica]\label{Lem:MP:CumSecFctCarac}
%
  Sea \ $X$ \ vector aleatorio $d$-dimensional y $\Psi_X(u) = \log \Phi_X(u)$ su
  secunda  funci\'on caracter\'istica.  Si  admite en  desarollo de  Taylor, los
  cumulantes de orden $k$ satisfacen
  %
  \[
  \kappa_{i_1,\ldots,i_k}[X]    \,   =   \,    (-\imath)^k   \left.\frac{\partial^k
      \Psi_X}{\partial    \omega_{i_1}    \ldots   \partial    \omega_{i_k}}
  \right|_{\omega   =    0},   \qquad    (i_1 , \ldots , i_k) \in \{ 1 , \ldots , d \}^k
  \]
\end{lema}


Varias propiedades de la  funci\'on \ $C_X$ \ se deducen de  las de \ $M_X$. Nos
enfocamos  en la  propiedad  relacionana a  combinaciones  lineales de  vectores
independientes, con consecuencias sobre los cumulantes:
%
\begin{teorema}%[Funci\'on generadora de probabilidad de una ]
\label{Teo:MP:PropiedadesGeneradoraCumulantes}
%
  Sean  \   $X$  \  e  \   $Y$  \  dos   vectores  aleatorios  $d$-dimensionales
  independientes,  \ $A$  \ una  matriz de  \ $\M_{d',d}(\Rset)$  \ y  \  $b \in
  \Rset^{d'}$.  Entonces para cualquier $u \in \Cset^{d'}$ \ (donde la funci\'on
  existe):
  %
  \[
  C_{A X + b}(u) =  u^t b +  C_X\left( A^t u \right),
  \]
  %
  y para cualquier $u \in \Cset^d$ \ (donde la funci\'on existe):
  %
  \[
  M_{X+Y}(u) = C_X(u) + C_Y(u)
  \]
  %
  Las  consecuencias de este  sobre los  cumulantes es  que~\footnote{La primera
    relaci\'on  se generaliza  sencillamente con  $A$ matricial,  pero  para dar
    expresiones compactas,  se necesita introducir  m\'as profundamente calculos
    tensiorales, lo  que va  m\'as all\'a del  enfoque de este  libro.},
  %
  \[
  \forall \: a \in \Rset, \quad \kappa_K[a X] = a^K \kappa_K[X] \qquad \mbox{y}
  \qquad \kappa_K[X+Y] = \kappa_K[X] + \kappa_K[Y]
  \]
\end{teorema}
%
\begin{proof}
  Las relaciones tratanto de $C$  son consecuencias directas del teorema pruebas
  siguen    punto   a    punto   los    mismos    pasos   que    las   de    los
  teoremas~\ref{Teo:MP:PropiedadesGeneradoraMomentos}, tomando el logaritmo de \
  $M_X$. Las relaciones sobre los cumulantes es entonces consecuencias de las de
  \  $C_X$ \ y  de la  definici\'on de  los cumulantes  a trav\'es  derivadas de
  $C_X$.
\end{proof}
%
Se notar\'a que, remarcablemente y contrariamente a los momentos, los cumulantes
de cualquier orden son funciones lineales de variables aleatorias independientes
(para los momentos, vale sol\'o hasta el orden $3$).  }




\vspace{2cm}

\centerline{\underline{\hspace{10cm}}}

\SZ{teorema de Polya?}

\SZ{hablar de la cota de Chernoff con la mgf o pgf?}


\SZ{Hablar del problema de Hamburger?}