\seccion{Esperanza condicional}
\label{Sec:MP:EsperanzaCondicional}

Vimos en  la secci\'on~\ref{Sec:MP:LeyesCondicionales} que  una pregunta natural
era de, dados  dos vectores aleatorios \  $X$ \ e \ $Y$,  caracterizar el vector
$Y$ si ``observamos $X$''.  M\'as adelante,  nos podemos interesar a la media de
$Y$ cuando observamos  $X$. Una manera intuitiva es de  definir tal media cuando
``sabemos'' que  $X=x$ a partir  de la ley  condicional $P_{Y|X=x}$~\cite{Fel68,
  Fel71, AthLah06, Spi76, Kol56, JacPro03}:

\begin{definicion}[Esperanza condicional]
  Sean   $X$   e  $Y$   dos   vectores   aleatorios   respectivamente  $d_X$   y
  $d_Y$-dimensionales, y sea la funci\'on
  %
  \[
  f(x) \equiv \Esp[Y | X=x ] = \int_{\Rset^{d_Y}} y \, dP_{Y|X=x}(y)
  \]
  %
  Se define la esperanza condicional de $Y$ condicionalment a $X$ como siendo la
  variable aleatoria
  %
  \[
  \Esp[Y|X] = f(X)
  \]
\end{definicion}

Como  en  el   caso  de  medida  de  probabilidad,   cuando  dos  variables  son
independientes, condicionar no cambia la esperanza:
%
\begin{lema}
Cuando \ $X$ \ e \ $Y$ \ son independientes, la esperanza condicional coincide con la de \ $Y$,
%
\[
X \:\: \mbox{e} \:\: Y \:\: \mbox{independientes} \quad \Rightarrow \quad \Esp[Y|X] = \Esp[Y]
\]
\end{lema}
\begin{proof}
\SZ{Hacerla}
%Considera el caso $Y = \un_B$ \ con \ $B \in \Y$
\end{proof}

La media condicional se revela muy  \'util y poderoso para evaluar esperanzas de
variables  aleatorias por ejemplo  gracia a  la formula  de la  esperanza total,
equivalente      de      las       formulas      de      probabilidad      total
lema~\ref{Lem:MP:ProbaTotalDiscreto},    lema~\ref{Lem:MP:ProbaTotalGeneral}   y
lema~\ref{Lem:MP:ProbaTotalContinuo}.
%
\begin{teorema}[Media total]\label{Teo:MP:EsperanzaTotal}
%
  La media  (total) del  vector aleatorio  \ $Y$ \  concide con  la media  de la
  esperanza condicional, \ie
%
\[
\Esp[Y] = \Esp\left[ \Esp\left[ Y | X\right] \right]
\]
\end{teorema}
%
\begin{proof}
\SZ{Hacerla}
%Considera el caso $Y = \un_B$ \ con \ $B \in \Y$
\end{proof}

Un  otro   resultado  importante,  permitiendo   frecuentemente  simplificar  la
evaluac\'ion de momentos a partir de esperanza condicional es el siguiente:
%
\begin{teorema}\label{Teo:MP:EsperanzaF(X)Y}
%
  Para cualquier funci\'on medible $f$, tenemos
  %
  \[
  \Esp\left[ \left. f(X) Y \, \right| \, X \right] = f(X) \Esp\left[ Y | X\right]
  \]
\end{teorema}
%
\begin{proof}
\SZ{Hacerla}
%Considera el caso $Y = \un_B$ \ con \ $B \in \Y$
\end{proof}

\SZ{
\begin{teorema}
La esperanza condicional \ $E[Y|X]$ \ es la \'unica variable $Z$ tal que para cualquier variable $U$ acotada $\Esp[Y U] = \Esp[Z U]$
\end{teorema}
\begin{proof}
%Considera el caso $Y = \un_B$ \ con \ $B \in \Y$
\end{proof}
%
A veces, este resultado sirve como defnici\'on de la esperanza condicional.}

Un  otro resultado  que  sirve a  veces  como definici\'on,  en  el contexto  de
variable  de  cuadrado integrable,  se  voncula con  la  idea  de aproximar  una
variable por una funcci\'on de una otra:
%
\begin{teorema}
  Sea \ $Y$  \ de cuadrado integrable, la esperanza condicional  \ $E[Y|X]$ \ es
  la \'unica  variable \ $Z  = f(X)$, funci\'on  de \ $X$, minimizando  el error
  promedio cuadratico \ $\Esp[  \| Y - Z \|^2 ]$.  Dicho  de otra manera, con el
  criterio de error  cuadratico promedio m\'inimo, \ $E[Y|X]$  \ es la ``mejor''
  funci\'on de $X$ aproximando $Y$.
\end{teorema}
\begin{proof}
Usando la f\'ormula de esperanza total, y el teorema~\ref{}, se escribe
%
\begin{eqnarray*}
\Esp\left[ \| Y - f(X) \|^2 \right] & = & \Esp\left[ \Esp\left[ \left. \| Y -
f(X) \|^2 \, \right| \, X \right] \right]\\
%
& = & \Esp\left[ f(X)^2 - 2 f(X) \Esp[Y|X] +  \Esp\left[ Y^2 | X \right] \right]
\end{eqnarray*}
%
Ahora, buscando $\lambda \equiv f(x)$  minimizando $\| \lambda\|^2 - 2 \lambda^t
\Esp[Y|X=x] + \Esp\left[ \| Y \|^2 |  X=x \right]$ para cualquier $x \in \X$, se
minimizar\'a  el  promedio en  $X$.   Inmediatamente,  notando  que buscamos  el
m\'inimo de  un paraboliodo de concavivad  por arriba, anulando  el gradiente en
$\lambda$ so obtiene  $\lambda \equiv f(x) = \Esp[Y|X=x]$,  el \'unico m\'inimo,
lo que cierra la prueba.
\end{proof}
%
Este resultado  es muy conocido en el  mundo de la estimaci\'on  donde se quiere
aproximar una variable minimizar el error cuadratico promedio~\cite{Kay93, Rob07}.
%, AthLah06, JacPro03}.  


\SZ{Cerrar esta secci\'on}