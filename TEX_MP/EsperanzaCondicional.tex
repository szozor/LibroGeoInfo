\seccion{Esperanza condicional}
\label{Sec:MP:EsperanzaCondicional}

Vimos en  la secci\'on~\ref{Sec:MP:LeyesCondicionales} que  una pregunta natural
era de, dados  dos vectores aleatorios \  $X$ \ e \ $Y$,  caracterizar el vector
$Y$ si ``observamos $X$''.  M\'as adelante,  nos podemos interesar a la media de
$Y$ cuando observamos  $X$. Una manera intuitiva es de  definir tal media cuando
``sabemos'' que  $X=x$ a partir  de la ley  condicional $P_{Y|X=x}$~\cite{Fel68,
  Fel71, AthLah06, Spi76, Kol56, JacPro03}:

\begin{definicion}[Esperanza condicional]
  Sean   $X$   e  $Y$   dos   vectores   aleatorios   respectivamente  $d_X$   y
  $d_Y$-dimensionales, y sea la funci\'on
  %
  \[
  f(x) \equiv \Esp[Y | X=x ] = \int_{\Rset^{d_Y}} y \, dP_{Y|X=x}(y)
  \]
  %
  Se define la esperanza condicional de $Y$ condicionalment a $X$ como siendo la
  variable aleatoria
  %
  \[
  \Esp[Y|X] = f(X)
  \]
\end{definicion}

La media condicional se revela muy  \'util y poderoso para evaluar esperanzas de
variables  aleatorias por ejemplo  gracia a  la formula  de la  esperanza total,
equivalente      de      las       formulas      de      probabilidad      total
lema~\ref{Lem:MP:ProbaTotalDiscreto},    lema~\ref{Lem:MP:ProbaTotalGeneral}   y
lema~\ref{Lem:MP:ProbaTotalContinuo}.

%
\begin{teorema}[Media total]\label{Teo:MP:EsperanzaTotal}
%
La media (total) del vector aleatorio \ $Y$ \ concide con la media de la esperanza condicional, \ie
%
\[
\Esp[Y] = \Esp\left[ \Esp\left[ Y | X\right] \right]
\]
\end{teorema}
%
\begin{proof}
%Considera el caso $Y = \un_B$ \ con \ $B \in \Y$
\end{proof}

\SZ{Hablar esperanza total


$\Esp[Y] = \Esp\left[ \Esp[Y|X=x] \right]$ esperanza total. Similar a la formula de probabilidades totales pagina~\pageref{:MP}.

}
