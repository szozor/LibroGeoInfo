\seccion{Probabilidades}
\label{s:MP:Probabilidad}

El  concepto  de  {\it  probabilidad}  es importante  en  situaciones  donde  el
resultado (o {\it outcome}) de un  dado proceso o medici\'on es incierto, cuando
la salida de una experiencia no  es totalmente previsible. La probabilidad de un
evento es una medida que se asocia con cu\'an probable es el evento o resultado.

Una  definici\'on de  probabilidad puede  obtenerse en  base a  la enumeraci\'on
exhaustiva de los resultados posibles de un experimento o proceso,
%lo que no siempre es factible
suponiendo que el conjunto de posibilidades es completo en el sentido de que una
de  ellas debe ser  verdad. Si  el proceso  tiene $K$  resultados distinguibles,
mutuamente  excluyentes e  igualmente probables  (esto  es, no  se prefiere  una
posibilidad frente a  otras), y si $k$  de esos $K$ tienen un  dado atributo, la
probabilidad asociada a dicho atributo en un dado procesos es $\frac{k}{K}$. Por
ejemplo, sorteando un n\'umero entre los  naturales del 1 al 10, la probabilidad
de ``obtener un n\'umero par'' es $\frac5{10} = \frac12$.

Otra  definici\'on  de  probabilidad  se  basa  en  la  frecuencia  relativa  de
ocurrencia  de  un evento.   Si  en  una cantidad  $K$  muy  grande de  procesos
independientes  cierto   atributo  aparece  $k$   veces,  se  identifica   a  la
probabilidad  asociada a  un  proceso o  ensayo  con la  frecuencia relativa  de
ocurrencia $\frac{k}{K}$ del atributo.
%lim N->infty n/N indef. 

Los  axiomas  de  Kolmogorov  proveen  requisitos  suficientes  para  determinar
completamente las propiedades de la medida de probabilidad \modif{$P(A)$} que se
puede asociar a  un evento $A$ entre  un conjunto de resultados o  eventos de un
proceso.

\modif{Llamemos $\Omega$ al {\it  espacio muestral} o {\it espacio fundamental},
  que  es  el espacio  de  {\it muestras  (outcomes  en  ingl\'es)} $\omega  \in
  \Omega$.  Se asocia  $\A$ una colecci\'on de conjuntos  de $\Omega$, donde los
  elementos de $\A$ son llamados {\it eventos}.
  % total de eventos.
  Por ejemplo, $\Omega$ puede ser las caras de un dado de 6 caras (los n\'umeros
  naturales del 1 al 6, o las letras {\it a, b, c, d, e, f}, u otro etiquetado),
  $\A$ teniendo los eventos
  % si $A$ es el evento
  $A$ ``es  un n\'umero natural par''  y $B$ indicando ``es  un n\'umero natural
  impar''.}
% ,  el espacio  muestral  $\Omega  = \{  A  , B  \}$  indica  ``es un  n\'umero
% natural'';
En el caso de analizar el tiempo de vida de un aparato, $\Omega \sim \Rset_+$.
% en el  lanzamiento de un  dado de 6  caras es $\Omega$  es el conjunto  de las
% etiquetas que se asigne a cada una de las caras (los n\'umeros naturales del 1
% al 6, o las letras {\it a, b, c, d, e, f}, u otro etiquetado).
El  conjunto  de  resultados  posibles  se  supone  conocido,  a\'un  cuando  se
desconozca de antemano el resultado de una prueba.

Entre  los eventos  se  pueden considerar  operaciones  an\'alogas a  las de  la
teor\'ia de conjuntos (ej.~\cite{Spi76}):
%% <-- Ej: representar mediante conjuntos las operaciones entre eventos
\begin{itemize}
\item Combinaci\'on o uni\'on de eventos: \
  % $A+B$ se corresponde con
  $A\cup  B$, implicando  que  se da  $A$,  \'o $B$,  o  ambos; \modif{Seg\'un  la
    literatura, se denota a veces $A+B$ o $A \wedge B$.}
%
\item  Intersecci\'on  de  eventos:  \ 
  % $A,B$ se corresponde con
  $A\cap B$,  implicando que  se dan ambos  $A$~y~$B$; \modif{Se denota  a veces
    $(A,B)$ o $A \vee B$.}
%
\item Complemento  de un evento: \ 
  % $-A$ se corresponde con
  $\bar{A}$ e indica que no se da $A$; \modif{Se denota a veces $-A$ o $A^c$.}
%
\item  Eventos {\it  disjuntos} o  {\it mutuamente  excluyentes} o  \modif{\it o
    incompatibles}: \  son aquellos que no se  superponen, se anota $A  \cap B =
  \emptyset$ donde $\emptyset = \bar{\Omega}$
  % $A,B = \emptyset$ donde $\emptyset=-\Omega$
  denota  el evento  nulo (evento  que no  puede ocurrir,  es el  complemento de
  $\Omega$).
\end{itemize}
%
\cite{Jef48, Hal50, Fel71, Bre88, ManWol95, IbaPar97, LehCas98, AthLah06, Coh13}:

\modif{Formalmente,   se  define   de  manera   abstracta  un   espacio  medible
  $(\Omega,\A)$ de  la manera siguiente~\cite{Hal50,  Bre88, IbaPar97, AthLah06,
    Coh13}:
%
\begin{definicion}[Espacio medible]
  $(\Omega, \A)$ \ formado de un espacio muestral \ $\Omega$ \ y una colecci\'on
  \ $\A$ \ de conjuntos de \ $\Omega$ \ es llamado {\it espacio medible} si
%
\begin{enumerate}%[label={(\Roman*)}]
\item  $\emptyset \in \A$,
%
\item si $A \in \A$, entonces \ $\bar{A} \in \A$,
%
\item la uni\'on  numerable de conjuntos de $\A$ queda en  $\A$ ($\A$ es cerrado
  por la un\'ion numerable).
%
\end{enumerate}
%
Con esta propiedades, $\A$ es llamado {\it $\sigma$-\'algebra}.
\end{definicion}
%
Es  sencillo mostrar  de que  $\Omega$ tambien  es en  $\A$, y  de que  $\A$ est
cerrado por la intersecci\'on numerable.}

Las propiedades de la probabilidad $P$ de un dado evento quedan determinadas por
los siguientes (ej.~\cite{Spi76}):
%
\modif{
\noindent {\it Axiomas de Kolmogorov}
%
\begin{enumerate}
\item $P(A_i) \geq 0 \ \ \forall \ A_i \A$
%
\item  Si $\{ A_i  \}_i$ son  eventos mutuamente  excluyentes de  $\A$, entonces
  $\displaystyle P( \cup_i A_i) = \sum_i P(A_i)$
%
\item $P(\Omega) = 1$
\end{enumerate}
%
Formalmente,   se   define   un   espacio   de   probabilidad   de   la   manera
siguiente~\cite{Hal50, Bre88, IbaPar97, AthLah06, Coh13}:
%
\begin{definicion}[Espacio de probabilidad]
  Sea $(\Omega,\A)$ un espacio medible.  Una funci\'on $\mu: \A \mapsto \Rset_+$
  tal que
  %
  \begin{enumerate}
  \item $\mu(\emptyset) = 0$, y
  \item para cualquier  conjunto numerable $\{ A_i \}$  de elementos mutualmente
    excluyentes de $\A$ se tiene $\mu(\cup_i A_i) = \sum_i \mu(A_i)$
  \end{enumerate}
  %
  es  llamada {\it  funci\'on  medida}  o {\it  medida  $\sigma$-aditiva} y  el
  espacio $(\Omega,\A,\mu)$ es  llamado {\it espacio de medida}  , donde $\omega
  \in \Omega$ son  llamados {\it muestras (outcomes)} y  $\A$ es una colecci\'on
  de subconjuntos de $\Omega$ llamados  {\it eventos} con las propiedades de que
  (i) $\emptyset \in \A$, (ii) si $A \in \A$, $\bar{A} \in \A$, (iii) la uni\'on
  numerable de conjuntos  de $\A$ queda en $\A$ ($\A$ es  cerrado por la un\'ion
  numerable). $\sigma$  es una medida, $\A \mapsto  \Rset_+$ dicha sigma-aditiva
  tal que  $\sigma(\emptyset) = 0$  y si $A$  y $B$ son  mutualmente exluyentes,
  $\sigma( A  \cup B) = \sigma(A)  + \sigma(B)$. Cuando $\sigma$  es acotada por
  arriba,  la medida es  dicha {\it  finita} y  si $\sigma(\Omega)  = 1$  es una
  medida  de {\it  probabilidad},  $\sigma \sim  P$.  En este  caso, el  espacio
  $(\Omega,\A,P)$ es llamado {\it espacio de probabilidad}. Ver~\cite{Bre, toto}
  para mas detalles
\end{definicion}
}

A partir de \modif{los axiomas de Kolmogorov} se pueden probar varios corolarios y propiedades:
%
\begin{itemize}
\item la probabilidad de un evento seguro o cierto es 1;
%
\item  la   probabilidad  de   un  evento   que  no  puede   ocurrir  es   0:  \
  $P(\emptyset) = 0$;
%
\item el rango  de las probabilidades est\'a  acotado: \ $0 \leq P(A)  \leq 1\ \
  \forall \ A$;
%
\item  condici\'on de  normalizaci\'on: \  si $\Omega  = \cup_{i=1}^n  A_i$, con
  $A_i$ mutuamente excluyentes, entonces  \ $\sum_{i=1}^n P(A_i) = 1$; \modif{el
    conjunto $\{  A_i \}$  es dicho {\it  conjunto completo de  eventos posibles
      excluyentes entre s\'i};}
%
\item si $A$ es subconjunto de $B$, entonces \ $P(A) \leq P(B)$.
\end{itemize}

La  {\it probabilidad conjunta}  \ \modif{$P(A  \cap B)  = P(B  \cap A)$}  es la
probabilidad del evento conjunto dado por  la composici\'on de los eventos $A$ y
$B$. Se demuestra que
%
\begin{itemize}
\item $P(A  \cap B)$ est\'a acotada:  \ $0 \leq P(A  \cap B) = P(B  \cap A) \leq
  \min\{ P(A)  , P(B)\}$  \ \SZ{(viene  de $A \cap  B$ subconjunto  de $A$  y de
    $B$);}
%
\item si $A$ y $B$ son mutuamente excluyentes, entonces \ $p(A \cap B) = 0$;
%
\item  si  $\{ B_j  \}_{j=1}^m$  es un  conjunto  completo  de eventos  posibles
  excluyentes entre s\'i, entonces \ $\sum_{j=1}^m P(A \cap B_j) = P(A)$.
\end{itemize}

En el caso de eventos no necesariamente mutuamente excluyentes, se prueba que la
{\it ley de composici\'on} es
%
\[
P(A \cup B) = P(A) + P(B) - P(A \cap B)\leq P(A) + P(B), 
\]
%
y que para $n$ eventos resulta 
%
\[
P( \cup_{i=1}^n) \leq \sum_{i=1}^n P(A_i). 
\]
%
La  igualdad  vale  en  el  caso  especial  de  eventos  mutuamente  excluyentes
(recuperando el tercer axioma de Kolmogorov).

La  {\it probabilidad condicional}  de  $A$ dado  $B$  es la  raz\'on entre  la
probabilidad del  evento conjunto y la  probabilidad de que se  d\'e $B$ (cuando
\'este es un evento no nulo):
%
$$
p(A|B)=\frac{p(A,B)}{p(B)} .
$$ 
%
Es f\'acil demostrar %% <-- ejercicio
que  esta cantidad  toma valores  entre 0  y 1,  con $p(\Omega|B)=1$,  y  que es
aditiva  para  una  uni\'on  de  eventos  mutuamente  excluyentes  referidos  al
cumplimiento de  $B$. Luego, $p(A|B)$  es una probabilidad.  Algunas propiedades
interesantes son las siguientes:
%
\begin{itemize}
\item  condici\'on  de  normalizaci\'on:  \  $\sum_{i=1}^N  p(A_i|B)=1$,  siendo
  $A_1,\ldots,A_N$  un  conjunto  completo  de  resultados  posibles  mutuamente
  excluyentes;
%
\item    relaci\'on    entre    probabilidades   condicionales    inversas:    \
  $p(B|A)=\frac{p(B)}{p(A)}  p(A|B)$,  de donde  $p(A|B)$  y $p(B|A)$  coinciden
  s\'olo cuando $A$ y $B$ tienen la misma probabilidad;
%
\item {\it f\'ormula de Bayes}: \ si $B_1, B_2, \ldots$ es un conjunto completo
  de eventos no nulos mutuamente excluyentes, entonces
  %
  $$
  p(B_i|A)=\frac{p(A,B_i)}{p(A)}   =   \frac{p(A|B_i)  p(B_i)}{\sum_j   p(A|B_j)
    p(B_j)} .
  $$ 
\end{itemize}

Dos eventos  $A$ y $B$  se dicen {\it estad\'isticamente independientes}  si la
probabilidad  condicional   de  $A$  dado   $B$  es  igual  a   la  probabilidad
incondicional  de  $A$:  \   $p(A,B)=p(A)  p(B)$.  La  condici\'on  necesaria  y
suficiente  para  que   $N$  eventos  $A_1,\ldots,A_N$  sean  estad\'isticamente
independientes es que la probabilidad conjunta se factorice como
%
$$
p(A_1,\ldots,A_N)=p(A_1) \cdots p(A_N) .
$$
%
Se  deduce que  los  eventos mutuamente  excluyentes  no son  estad\'isticamente
independientes.

\cite{ManWol95}