\seccion{Introducci\'on}
\label{Sec:MP:Introduccion}

\modif{

A pesar  de que las nociones  de azar (que  proviene del \'arabe {\it  zahr} que
significa dado, flor)  o de aleatoriedad (del lat\'in {\it  alea} que es suerte,
dado) son muy antiguas~\cite{Ser00}, el matem\'atico italiano y jugador de dados
y cartas  Gerolamo Cardano  es ``probablemente'' uno  de los primeros  en tratar
matem\'aticamente el  concepto de  probabilidad en el  siglo~{XVI}, en  su libro
sobre los  juegos de  azar escrito en  1564 pero publicado  en 1663~\cite{Car63}
(ver~\cite{Bel05}  o~\cite[Cap.~4]{Hal90}).  Entre  los  numerosos matem\'aticos
que desarrollaron la teor\'ia de las probabilidades, en particular los franceses
Pierre  de  Fermat  y  Blaise  Pascal~\cite[Cap.~5]{Hal90},  hay  que  mencionar
tambi\'en   al   suizo  Jacob   Bernoulli   (miembro   de   una  dinast\'ia   de
matem\'aticos)~\cite[en     lat\'in]{Ber1713}     o~\cite{Ber1713:2}    y     al
franco-ingl\'es Abraham  de Moivre~\cite{Dem56}. Hasta la  \'epoca de Bernoulli,
el enfoque era puramente discreto, es  decir que el conjunto de estados posibles
era discreto de tama\~no finiot (6 caras de un dado, 32 tarjetad, 2 caras de una
moneda,\ldots).

El franc\'es Pierre Simon Laplace~\cite{Lap20}  fue quiz\'as uno de los primeros
en  proveer  un   aporte  importante  al  desarrollo  de   la  teor\'ia  de  las
probabilidades  en  los  siglos  XVIII-XIX,   a  trav\'es  del  punto  de  vista
``frecuentista''    y    combinatorial    (ver   tambi\'en~\cite[Caps.~13,    15
\&~22]{Hal90}). En la misma \'epoca, hay que mencionar \SZ{C. Gauss} quien trabaj\'o,
entre  muchas cosas,  en la  predicci\'ion de  la trayector\'ia  del planetisimo
C\'eres.  Proponiendo  un error  cuadratico,  apareci\'o  implicitamente la  ley
Normal,  o Gausiana, que  tiene su  nombre, a  pesar de  que la  desarollo m\'as
Laplace (a\'un  que, sobre el  mismo problema, propuso  el un error  tipo $L^1$,
v\'inculado a la ley doble-exponencial o de Laplace).

Un  paso  muy  importante,  especialmente tratanto  de  aleatoriadidad  continua
(ej. medida de una velocidad, que puede tomar cualquier valor real si tomamos en
cuenta  la direcci\'on),  fue debido  entre otros  a Kolmogorov  en 1933  que se
apoy\'o sobre  trabajos de Richard  von Mises~\cite{Mis32} y tambi\'en  sobre la
teor\'ia de la medida y de la integraci\'on, debidas entre otros a \'Emile Borel
y  Henri  Lebesgue~\cite{Bor98, Bor09,  Leb04,  Leb18,  Hal50}, para  formalizar
anal\'iticamente   la  teoria   de  las   probabilidades~\cite{Kol56,  BarNov78,
JacPro03}. Este punto  de vista permite tratar formalmente  el caso de variables
discretas, continuas, o mezcal de ambas,  que sean escalar o multivariada, en un
marco \'unico y  muy poderoso, sin perder las intuici\'on que  lleva el punto de
vista frecuencista.

}