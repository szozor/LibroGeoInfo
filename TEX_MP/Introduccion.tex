\seccion{Introducci\'on}
\label{Sec:MP:Introduccion}

\modif{

A pesar  de que las nociones  de azar (que  proviene del \'arabe {\it  zahr} que
significa dado, flor)  o de aleatoriedad (del lat\'in {\it  alea} que es suerte,
dado) son muy antiguas~\cite{Ser00}, el matem\'atico italiano y jugador de dados
y cartas  Gerolamo Cardano  es ``probablemente'' uno  de los primeros  en tratar
matem\'aticamente el  concepto de  probabilidad en el  siglo~{XVI}, en  su libro
sobre los  juegos de  azar escrito en  1564 pero publicado  en 1663~\cite{Car63}
(ver~\cite{Bel05}  o~\cite[Cap.~4]{Hal90}). La  denominaci\'on  de probabilidad,
ella, viene  de Aristote y  designaba una percepci\'on  de una idea.   Tom\'o su
sentido  m\'as  actual solamente  durante  la edad  media  en  Europa, por  mala
traducci\'on de  la escritura de  Aristote.  Despu\'es de la  primeras semillas,
debido a  Cardano, hay  que mencionar  los franceses Pierre  de Fermat  y Blaise
Pascal  en el  medio del  siglo XVII~\cite{Pas79}  o~\cite[Cap.~5]{Hal90},  y el
neerlandese   C.   Huygens~\cite{Huy57}   o~\cite[Cap.~6]{Hal90}),   que  fueron
claramente unos de los primeros a desarrollar la teor\'ia de las probabilidades.
M\'as tarde, pasos  importante fureon debidos al suizo  Jacob Bernoulli (miembro
de una dinast\'ia  de matem\'aticos)~\cite[en lat\'in]{Ber1713} o~\cite{Ber13:2,
Hal90,  DavEdw01, Hal06}  y  al franco-ingl\'es  Abraham de  Moivre~\cite{Moi56,
Hal90,  DavEdw01,  Hal06}.   Hasta  la  \'epoca de  Bernoulli,  el  enfoque  era
puramente discreto, es decir que el conjunto de estados posibles era discreto de
tama\~no  finito   (6  caras  de   un  dado,  32   tarjetas,  2  caras   de  una
moneda,\ldots). La  meta de la  mayor\'ia de los  estudios eran dedicados  a los
juegos  (dados,  cartas), problema  de  seguro/riesgo,  o  estudios sociales  en
problaciones.

El franc\'es Pierre Simon Laplace~\cite{Lap12, Lap14, Lap20, Lap36} fue quiz\'as
uno de los primeros en proveer un aporte importante al desarrollo de la teor\'ia
de las  probabilidades en los  siglos XVIII-XIX, a  trav\'es del punto  de vista
``frecuentista'' y combinatorial (ver tambi\'en~\cite[Caps.~13, 15 \&~22]{Hal90}
o~\cite{DavEdw01,  Hal06}).  En  la  misma  \'epoca, hay  que  mencionar C.   F.
Gauss, matem\'atico muy prol\'ifico, quien  trabaj\'o, entre muchas cosas, en la
predicci\'ion  de la trayector\'ia  del planetisimo  C\'eres~\cite{Gau09, Gau10}
o~\cite[Cap.~7]{Hal06}.     Proponiendo   un   error    cuadratico,   apareci\'o
implicitamente la ley Normal, o Gausiana, que tiene su nombre, a pesar de que la
desarollo m\'as Laplace  (aun que, sobre el mismo problema,  propuso el un error
tipo $L^1$,  v\'inculado a la  ley doble-exponencial o  de Laplace)~\cite{Lap09,
Lap09:Supp, Lap12,  Lap14, Lap20}.   A veces  la ley de  Gauss, quizas  la m\'as
importante en  la teor\'ia de  las probabilidades, llamada tambi\'en  gausiana o
normal, es conocida como ley de Laplace-Gauss.

Un  paso  muy  importante,  especialmente tratanto  de  aleatoriadidad  continua
(ej. medida de una velocidad, que puede tomar cualquier valor real si tomamos en
cuenta  la direcci\'on),  fue debido  entre otros  a Kolmogorov  en 1933  que se
apoy\'o sobre  trabajos de Richard  von Mises~\cite{Mis32} y tambi\'en  sobre la
teor\'ia de la medida y de la integraci\'on, debidas entre otros a \'Emile Borel
y  Henri  Lebesgue~\cite{Bor98, Bor09,  Leb04,  Leb18,  Hal50}, para  formalizar
anal\'iticamente   la  teoria   de  las   probabilidades~\cite{Kol56,  BarNov78,
JacPro03}. Este punto  de vista permite tratar formalmente  el caso de variables
discretas, continuas, o mezcla de  ambas, que sean escalares o multivariadas, en
un marco \'unico  y muy poderoso, sin perder las intuici\'on  que lleva el punto
de vista frecuencista.

}