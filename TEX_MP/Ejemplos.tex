\SZ{Hablar de convergencia?}

\seccion{Algunos ejemplos de distribuciones de probabilidad}
\label{Sec:MP:EjemplosDistribucionesProb}

En esta secci\'on, vamos a ver unos ejemplos de distribuciones que se encuentran
frecuentemente   en   problema   pr\'acticos   de   varias   areas   cientificas
(estad\'istica,  f\'isica, ingener\'ia,\ldots).  Daremos las  caracteristicas de
cada ley presentada, as\'i que sus propiedades remarcables. El num\'ero de leyes
de probababilidad es tan importante que es dificil, para no decir imposible, ser
exahustivo.  Para  tener  m\'as  detalles,  se  puede  referirse  a  los  libros
especializados  en  este   marco,  como  por  ejemplo~\cite{JohKot92,  JohKot97,
  JohKot95:v1, JohKot95:v2, KotBal00, GupNag99, FanKot90, SamTaq94}.


% ================================= Variables discretas
\subseccion{Distribuciones de variable discreta}
\label{Ssec:MP:EjemplosDistribucionesDiscretas}

% --------------------------------- Certeza
\subsubseccion{Variable real con certeza}
\label{Sssec:MP:Certeza}

El caso \ $X = a \in \Rset^d$ \ deterministico ($\forall \, \omega, \: X(\omega)
= a$)  puede ser ver  visto como un  caso degenerado de vector  aleatorio. Visto
as\'i, sus caracter\'isticas principales  vistas en las secciones anteriores son
resimidas en la tabla siguiente:

\begin{caracteristicas}
%
Dominio de definici\'on & $\X = \{ a \}, \quad a \in \Rset^d$\\[2mm]
\hline
%
Distribuci\'on de probabilidad & $p_X(x) = \un_{\{a\}}(x)$\\[2mm]
\hline
%
Promedio & $\displaystyle m_X = a$\\[2mm]
\hline
%
Covarianza~\footnote{Siendo cero la covarianza, no se define ni la asimetr\'ia,
ni la curtosis. Sin embargo, de una manera se puede decir que la ley no es
asim\'etrica, y con cola livianas (no hay colas).} & $\displaystyle \Sigma_X =
0$\\[2mm]
\hline
%
%\modif{Asimetr\'ia} & $\gamma_X = 0$\\[2mm]
%\hline
%%
%Curtosis por exceso & $\displaystyle \widebar{\kappa}_X = - \sum_{i,j=1}^d \Big( \! \left(
%    \un_i \un_i^t \right) \otimes \left(  \un_j \un_j^t \right) +  \left( \un_i
%    \un_j^t \right) \otimes \left( \un_i  \un_j^t \right) + \left( \un_i \un_j^t
%  \right) \otimes \left( \un_j \un_i^t \right) \! \Big)$\\[2mm]
%\hline
%
Generadora de probabilidad & $\displaystyle G_X(z) = \prod_{i=1}^d z_i^{a_i}$ \ para \ $z_i \in \Cset$
\ si $a_i \ge 0$ \ y \ $\Cset_0$ \ si no\\[2mm]
\hline
%
Generadora de momentos & $\displaystyle M_X(u) = e^{a^t u}$ \ para \ $u \in
\Cset^d$\\[2mm]
\hline
%
Funci\'on caracter\'istica & $\displaystyle \Phi_X(\omega) = e^{\imath \, a^t
\omega}$
\end{caracteristicas}

% Momentos & $ \Esp\left[ X^k \right] = p^k$\\[2mm]
% Momento factorial & $\Esp\left[ (X)_k \right] = ?$\\[2mm]

La funci\'on de masa y funci\'on de repartici\'on son representadas en la figura
Fig.~\ref{Fig:MP:Certeza} en el caso escalar.
%
\begin{figure}[h!]
\begin{center} \begin{tikzpicture}%[scale=.9]
\shorthandoff{>}
%
\pgfmathsetmacro{\a}{2};% a
\pgfmathsetmacro{\sy}{2.5};% y-scaling
\pgfmathsetmacro{\r}{.05};% radius arc non continuity F_X
%
% masa
\begin{scope}
%
%
\draw[>=stealth,->] (-.25,0)--({\a+1.75},0) node[right]{\small $x$};
\draw[>=stealth,->] (0,-.1)--(0,{\sy+.25}) node[above]{\small $p_X$};
%
\draw[dotted] (\a,0)--(\a,\sy) node[scale=.4]{$\bullet$};
\draw (0,\sy)--(-.1,\sy) node[left,scale=.7]{$1$};
\draw (\a,0)--(\a,-.1) node[below,scale=.7]{$a$};
%%
\end{scope}
%
%
% reparticion
\begin{scope}[xshift=8.5cm]
%
\draw[>=stealth,->] (-.75,0)--({\a+1.75},0) node[right]{\small $x$};
\draw[>=stealth,->] (0,-.1)--(0,{\sy+.25}) node[above]{\small $F_X$};
%
% cumulativa
\draw[thick] (-.5,0)--(\a,0);
\draw ({\a+\r},\r) arc (90:270:\r);
\draw[dotted] (\a,0)--(\a,\sy);
\draw[thick] (\a,\sy) node[scale=.4]{$\bullet$}--({\a+1.5},\sy);
%
\draw (0,\sy)--(-.1,\sy) node[left,scale=.7]{$1$};
\draw (\a,0)--(\a,-.1) node[below,scale=.7]{$a$};
\end{scope}
%
\end{tikzpicture} \end{center}
% 
\leyenda{Ilustraci\'on  de una  distribuci\'on  cierta (a),  y  la funci\'on  de
  repartici\'on asociada (b).}
\label{Fig:MP:Certeza}
\end{figure}

\

Notar que todo se extiende al caso complejo sin costo adicional.

\

\index{Ley de gran n\'umeros}
El caso  de variables  deterministicas puede ser  visto como caso  degenerado de
variables aleatorias, pero aparecen de vez a cuando tambi\'en como caso l\'imite
de  sucesiones o  series  de  variables aleatorias.   En  particular, aparece  a
trav\'es de  la ley  de gran n\'umeros,  un de  los primeros casos  de l\'imites
estudiado tratando  de variables aleatorias. Historicamente, un  de los primeros
que  estudio  la convergencia  (sin  prueba  e  implicitamente) de  un  promedio
empirico a esta ``ley'' es el  matem\'atico italiano y jugador de dados y cartas
Gerolamo Cardano en el siglo~{XVI}, en su libro sobre los juegos de azar escrito
en  1564   (ver  introducci\'on  y~\cite{Car63,   Bel05}  o~\cite[Cap.~4]{Hal90}
o~\cite[Cpa.~3]{Mlo08}).  En  otras palabras,  explic\'o que la  precisi\'on las
estadisticas empiricos se mejora con el  n\'umero de datos, lo que es nada m\'as
que, en palabras,  el resultado de la ley dicha de  gran n\'umeros. En palabras,
saliendo  de  variables  aleatoria  independientes  de misma  ley,  el  promedio
empirico tiende a  la media donde enfatisaremos en que  sentido hay que entender
``tiende a''. Tal  convergencia fue estudiada y probada  mucho m\'as tarde, bajo
el  impulso  del suizo  Jacob  Bernoulli~\cite[Pars  4]{Ber1713} (ver  tambi\'en
Montmort~\cite{Mon13, Pea25})  en el  contexto de variables  binarias, conocidos
hoy  como  variables  de Bernoulli  (ver  subecci\'on~\ref{Sssec:MP:Bernoulli}).
Luego,  el  teorema  fue   mejorado  por  ejemplo  por  de  Moivre~\cite{Moi56},
Laplace~\cite{Lap20} o Poisson~\cite{Poi37}, yendo  m\'as all\'a de solamente la
convergencia del  promedio empirico a la  media.  El teorema  fue ampliado m\'as
all\'a  de la  ley  binomial como  suma  de variables  de  Bernoulli (ver  m\'as
adelante),    por    varios    autores   tales    que    Chebyshev~\cite{Tch46},
Markov~\cite{Mar13},    Borel~\cite{Bor09:12}     ,    Kinchin~\cite{Kin29}    o
Kolmogoroff~\cite{Kol30} entre otros (ver~\cite{Sen13} y referencias).

Formalmente,  las dos  versiones usuales  del  teoremas de  formaliza de  manera
siguiente  (ver  tambi\'en~\cite{Fel71,  Shi84,  AshDol99,  JacPro03,  AthLah06,
  Bil12, Coh13}).

\begin{teorema}[Ley debil de los gran n\'umero]
  Sea  \ $\left\{ X_k  \right\}_{k \in  \Nset_0}$ \  una sucesi\'on  de vectores
  aleatorios  independientes e identicamente  distribuidas (iid),  admitiendo una
  media  $m =  \Esp[X_k]$  \ y  sea  \ $\displaystyle  \widebar{X}_n =  \frac1n
  \sum_{k=1}^n X_k$ \ el promedio empirico. Entonces
  %
  \[
  \widebar{X}_n \limitP{n \to +\infty} m
  \]
  %
  donde $\limitP{}$ significa que el l\'imite es en probabilidad, \ie
  %
  \[
  \forall  \:  \varepsilon  >0,  \quad  \lim_{n  \to  +\infty}  P\left(  \left\|
      \widebar{X}_n - m \right\| > \varepsilon \right) = 0
  \]
\end{teorema}
\begin{proof}
  Una    prueba    sencilla   se    apoya    en    el    teorema   de    Markov,
  Cor.~\ref{Cor:MP:Markov},  cuando  los $X_k$  admiten  una  covarianza. De  la
  independencia, es  sencillo ver que  \ $\Cov\left[ \widebar{X}_n \,  \right] =
  \frac1n \Cov\left[ X_1 \right]$. Entonces,
  %
  \[
  P\left(  \left\|  \widebar{X}_n  -  m  \right\|  >  \varepsilon  \right)  \le
  \frac{\Esp\left[ \left\|  \widebar{X}_n - m\right\|^2 \right]}{\varepsilon^2}
  =  \frac{\Tr\left(   \Cov\left[  X_1  \right]   \right)}{n  \,  \varepsilon^2}
  \xrightarrow[n \to \infty]{} 0
  \]
  %
  lo que cierra la prueba.

  De hecho,  no es necesario  que los $X_k$  admitan una covarianza.  Una prueba
  alternativa    se   apoya   sobre    la   funci\'on    caracter\'istica.   Del
  teorema~\ref{Teo:MP:PropiedadesFuncionCaracteristica},   se   obtiene  de   la
  independencia
  %
  \[
  \Phi_{\widebar{X}_n}(\omega)   =  \left(   \Phi_{X_1}\left(  \frac{\omega}{n}
    \right) \right)^n = \left( 1 + \frac{\imath}{n} m^t \omega + o\left( \left\|
        \frac{\omega}{n} \right\| \right) \right)^n \xrightarrow[n \to \infty]{}
  e^{\imath m^t \omega}
  \]
  %
  En otros t\'erminos, la funci\'on caracter\'istica de $\widebar{X}_n$ tiende a
  la   de  $m$   punto  a   punto.  Se   usa  el   teorema  de   continuidad  de
  L\'evy~\cite{AshDol99, AthLah06, Bil12, Coh13}, no probado en este libro, para
  concluir  que  \  $\widebar{X}_n$  \  tiende  en distribuci\'on  a  \  $m$,  y
  equivalentemente tiende en probabilidad.
\end{proof}
%
Pasando,  de  la primera  prueba,  se  puede notar  que  se  puede debilitar  la
hypotesis  de  independencia,  y  a\'un  la  de misma  ley  para  los  $X_k$,  a
condici\'on de que $\Cov\left[ \widebar{X}_n \right]$ \ tiende a cero cuando $n
\to +\infty$ (por ejemplo, queda valide con la independencia y varianza acotada).

En palabras, el teorema traduce el  pensamiento de Cardano, que es que cualquier
sea el rayo de la bola centrada en $m$, cuando crece el n\'umero de variables en
el promedio  empirico, la probabilidad de  que este promedio sea  afuera de esta
bola tiende a cero.

De hecho, como  para series de funciones (lo que  son las variables aleatorias),
hay varias manera  de converger. Una m\'as fuerte  es conocido como convergencia
casi  siempre,  dando  lugar a  la  ley  dicha  fuerte  de los  gran  n\'umeros.
Historicamente, este teorema es dada en  el caso escalar, pero se extiende en el
caso  vectorial.    No  daremos  la   prueba,  que  se  encuentra   por  ejemplo
en~\cite[Teo.~6.4.2]{Gre63}    en   el   caso    vectorial,   o    entre   otros
en~\cite[Teo.~22.1]{Bil12} en el caso escalar.
%
\begin{teorema}[Ley fuert de los gran n\'umero o teorema de Kolmogorov-Khintchine]
%
  Sea  \ $\left\{ X_k  \right\}_{k \in  \Nset_0}$ \  una sucesi\'on  de vectores
  aleatorios independientes  e identicamente distribuidas  (iid), admitiendo una
  media  $m =  \Esp[X_k]$ \  y  tales que  tambi\'en \  $\Esp\left[ \left\|  X_k
    \right\| \right] <  \infty$, y sea \ $\displaystyle  \widebar{X}_n = \frac1n
  \sum_{k=1}^n X_k$ \ el promedio empirico. Entonces
  %
  \[
  \widebar{X}_n \limitcs{n \to +\infty} m
  \]
  %
  donde $\limitcs{}$ significa que el l\'imite  es casi siempre (o a veces dicho
  ``con probabiludad uno''), \ie
  %
  \[
  P\left(    \lim_{n \to +\infty}   \widebar{X}_n =  m \right) = 1
  \]
  %
  o, dicho  de otra  manera, la medida  del conjuto  $\{ \omega \tq  \lim_{n \to
    +\infty} \widebar{X}_n \ne m \}$ es cero.
\end{teorema}
%\begin{proof}
%Ver~\cite[Teo.~6.4.2]{Gre63}
%\end{proof}
%

Esta versi\'on es dicha fuerte porque la convergencia casi siempre implica la en
probabilidad~\cite{Fel71, Shi84, AshDol99, JacPro03, AthLah06, Bil12, Coh13}. Se
puede debilitar un paso m\'as  las condiciones (ej. indemendencia, etc.) pero va
m\'as all\'a  de la  meta de esta  secci\'on. El  lector se podr\'ea  referir en
libros  especializados,  por  ejemplo~\cite{Fel71,  Shi84,  AshDol99,  JacPro03,
  AthLah06, Bil12, Coh13}.

Una consecuencia de la ley fuerte de gran n\'umeros es conocido como theorema de
Borel. Dice  que, en el  contexto de variables  discretas, si una  experienca se
repite de  manera independiente  un gran n\'umero  de veces, la  proporci\'on de
ocurencia de un  estado tiendo a su probabilidad  de ocurencia (con probabilidad
uno).  Se podr\'a  referir  por  ejemplo a~\cite{Wen91}  para  tener una  prueba
``moderna''.

%\SZ{Poner ac\'a la ley de los gran n\'umeros? M\'as notas historicas.}

% --------------------------------- Uniforme discreta
\subsubseccion{Ley Uniforme sobre un ``intervalo'' de $\Zset$}
\label{Sssec:MP:UniformeDiscreta}

Se denota $X \, \sim \, \U\{ a \; b \}$ \ con $(a,b) \in \Zset^2, \: b \ge
a$.  Las caracter\'isticas de \ $X$ \ son las siguientes:

\begin{caracteristicas}
%
Par\'ametros & $(a,b) \in \Zset^2, \: b \ge a$\\[2mm]
\hline
%
Dominio de definici\'on & $\X = \{ a  \; a+1 \; \ldots \; b \}$\\[2mm]
\hline
%
Distribuci\'on de probabilidad & $p_X(x) = \frac1{b-a+1}$\\[2mm]
\hline
%
Promedio & $\displaystyle m_X = \frac{a+b}{2}$\\[2mm]
\hline
%
Varianza & $\displaystyle \sigma_X^2 = \frac{(b-a) (b-a+2)}{12}$\\[2mm]
\hline
%%
\modif{Asimetr\'ia} & $\gamma_X = 0$\\[2mm]
\hline
%
Curtosis por exceso & $\displaystyle \widebar{\kappa}_X = -\frac65 \frac{(b-a)
(b-a+2)+2}{(b-a) (b-a+2)}$\\[2mm]
\hline
%
Generadora de probabilidad & $\displaystyle G_X(z) = \frac{z^a-z^{b+1}}{1-z}$ \
para~\footnote{En el caso l\'imite \ $z \to 1$, \ $\lim_{z \to 1} \frac{ z^a -
z^{b+1}}{1-z} = b+1-a$} \ $z \in \Cset$ \ si $a \ge 0$ \ y \ $\Cset^*$ \ sino\\[2mm]
\hline
%
Generadora de momentos & $\displaystyle M_X(u) = \frac{ e^{a u} - e^{(b+1)
u}}{1-e^u}$ \ para~\footnote{En el caso l\'imite \ $u \to 0$, \ $\lim_{u \to 0}
\frac{ e^{a u} - e^{(b+1) u}}{1-e^u} = b+1-a$, y similarmente para la funci\'on
caracter\'istica.}  \ $u \in \Cset$\\[2mm]
\hline
%
Funci\'on caracter\'istica & $\displaystyle  \Phi_X(\omega) = \frac{ e^{\imath a
\omega} - e^{\imath (b+1) \omega}}{1-e^{\imath \omega}}$
\end{caracteristicas}

% Momentos & $ \Esp\left[ X^k \right] = p^k$\\[2mm]
% Momento factorial & $\Esp\left[ (X)_k \right] = ?$\\[2mm]
% modo 0
% Mediana \ln(2)/\lambda
% CDF 1-e^{-\lambda x}

La distribuci\'on  de masa de probabilidad  y funci\'on de  repartici\'on de una
variable uniforme  \ $\U\{  a \; b  \}$ \ son  representadas en  la figura
Fig.~\ref{Fig:MP:UniformeDiscreta}.
%
\begin{figure}[h!]
\begin{center} \begin{tikzpicture}%[scale=.9]
\shorthandoff{>}
%
\pgfmathsetmacro{\sx}{.75};% x-scaling
\pgfmathsetmacro{\r}{.05};% radius arc non continuity F_X
\pgfmathsetmacro{\n}{6};% n de la uniforme
\pgfmathsetmacro{\m}{\n-1};
%
% masa
\begin{scope}
%
%
\pgfmathsetmacro{\sy}{2.5};% y-scaling 
\draw[>=stealth,->] (-.25,0)--({\sx*(\n+.5)+.25},0) node[right]{\small $x$};
\draw[>=stealth,->] (0,-.1)--(0,{\sy+.25}) node[above]{\small $p_X$};
%
\foreach \k in {1,...,\n} {
\draw[dotted] ({\k*\sx},0)--({\k*\sx},\sy) node[scale=.4]{$\bullet$};
\draw ({\k*\sx},0)--({\k*\sx},-.1) node[below,scale=.7]{$\k$};
}
\draw (0,\sy)--(-.1,\sy) node[left,scale=.7]{$\frac1\n$};
%%
\end{scope}
%
%
% reparticion
\begin{scope}[xshift=8.5cm]
%
\pgfmathsetmacro{\sy}{2.5};% y-scaling 
%
\draw[>=stealth,->] ({-\sx/2-.25},0)--({\sx*(\n+1.5)+.25},0) node[right]{\small $x$};
\draw[>=stealth,->] (0,-.1)--(0,{\sy+.25}) node[above]{\small $F_X$};
%
% cumulativa
\draw[thick] ({-\sx/2},0)--(\sx,0);
\draw ({\sx+\r},\r) arc (90:270:\r);
\draw (0,0)--(0,-.1) node[below,scale=.7]{$0$};
%
\foreach \k in {1,...,\m} {
\draw ({\k*\sx},0)--({\k*\sx},-.1) node[below,scale=.7]{$\k$};
\draw[thick] ({\k*\sx},{\k*\sy/\n}) node[scale=.4]{$\bullet$}--({(\k+1)*\sx},{\k*\sy/\n});
\draw ({(\k+1)*\sx+\r},{\k*\sy/\n+\r}) arc (90:270:\r);
\draw[dotted] ({\k*\sx},{(\k-1)*\sy/\n})--({\k*\sx},{\k*\sy/\n});
}
\draw ({\n*\sx},0)--({\n*\sx},-.1) node[below,scale=.7]{$\n$};
\draw[thick] ({\n*\sx},\sy) node[scale=.4]{$\bullet$}--({(\n+1.5)*\sx},\sy);
\draw[dotted] ({\n*\sx},{(\n-1)*\sy/\n})--({\n*\sx},\sy);
%%
\draw (0,\sy)--(-.1,\sy) node[left,scale=.7]{$1$};
\end{scope}
%
\end{tikzpicture} \end{center}
% 
\leyenda{Ilustraci\'on  de  una densidad  de  probabilidad  uniforme  (a), y  la
  funci\'on  de repartici\'on  asociada (b).  $a =  1,  \: b  = 6$  \ (ej.  dado
  equilibriado).}
\label{Fig:MP:UniformeDiscreta}
\end{figure}

Cuando \ $b = a$, la variable tiende a una variable cierta \ $X = a$.
%
%La ley tiene propiedades de reflexividad obvia:
%%
%\begin{lema}[Reflexividad]\label{Lem:MP:ReflexividadUniformeDiscreta}
%%
%  Si \ $X \sim \U\{ a ; b \}$\ entonces
%  %
%  \[
%  a+b-X \sim \U\{ a ; b \}
%  \]
%\end{lema}
%\begin{proof}
%  Sea  $Y  =  a+b-X$. Entonces,  $P_Y(y)  =  P(a+b-X  =  y)  = P(X  =  a+b-y)  =
%  \frac{1}{b-a+1} \, \un_{\{ a \; \ldots \; b \}}(y)$.
%\end{proof}

La  distribuci\'on  uniforme  aparece  por   ejemplo  en  el  tiro  de  un  dado
equilibriado con \ $a = 1, \: b = 6$.


% --------------------------------- Bernoulli
\subsubseccion{Ley de Bernoulli}
\label{Sssec:MP:Bernoulli}

Esta ley aparece  cuando se hace una experiencia con dos  estados posibles, tipo un
tiro  de  moneda.   Apareci\'o en  trabajos  muy  antiguos,  entre otros  el  de
J.  Bernoulli  tratando  de  la  ley  de  gran  n\'umeros~\cite{Ber1713,  Hal90,
  DavEdw01}.

Se  denota \  $X \,  \sim \,  \B(p)$ \  con \  $p \in  [0 \;  1]$ \  y sus
caracter\'isticas son las siguientes:

\begin{caracteristicas}
%
Dominio de definici\'on & $\X = \{ 0 \; 1 \}$\\[2mm]
\hline
%
Par\'ametro & $p \in [ 0 \; 1 ]$\\[2mm]
\hline
%
Distribuci\'on de probabilidad & $p_X(1) = 1 - p_X (0) = p$\\[2mm]
\hline
%
Promedio & $ m_X = p$\\[2mm]
\hline
%
Varianza & $\sigma_X^2 = p \, (1-p)$\\[2mm]
\hline
%
\modif{Asimetr\'ia} & $\displaystyle \gamma_X =  \frac{1 - 2 \, p}{\sqrt{p \, (1-p)}}$ \quad para \ $p \notin \{ 0 \; 1 \}$ (ver m\'as adelante)\\[2mm]
\hline
%
Curtosis por exceso & $\displaystyle \widebar{\kappa}_X = \frac{1 - 6 \, p + 6
\, p^2}{p \, (1-p)}$ \quad para \ $p \notin \{ 0 \; 1 \}$ (ver m\'as adelante)\\[2mm]
\hline
%
Generadora de probabilidad & $G_X(z) = 1 - p + p z$ \quad sobre \ $\Cset$\\[2mm]
\hline
%
Generadora de momentos & $M_X(u) = 1 - p + p \, e^u$ \quad sobre \ $\Cset$\\[2mm]
\hline
%
Funci\'on caracter\'istica & $\Phi_X(\omega) = 1 - p + p \, e^{\imath \omega}$
\end{caracteristicas}


% Momentos & $ \Esp\left[ X^k \right] = p^k\\[2mm]
% Momento factorial & $\Esp\left[ (X)_k \right] = p^k \un_{\{0 \, , \, 1 \}}(k)$\\[2mm]

Su masa  de probabilidad  y funci\'on de  repartici\'on son representadas  en la
figura Fig.~\ref{Fig:MP:Bernoulli}.
%
\begin{figure}[h!]
\begin{center} \begin{tikzpicture}%[scale=.9]
\shorthandoff{>}
%
\pgfmathsetmacro{\sx}{2};% x-scaling
\pgfmathsetmacro{\r}{.05};% radius arc non continuity F_X
\pgfmathsetmacro{\p}{1/3};% probabilidad p
% masa
\begin{scope}
%
\pgfmathsetmacro{\sy}{2/max(\p,1-\p)};% y-scaling
%
\pgfmathsetmacro{\ss}{\sy*(1-\p)};
\draw[>=stealth,->] (-.5,0)--({\sx+.75},0) node[right]{\small $x$};
\draw[>=stealth,->] (0,-.15)--(0,2.5) node[above]{\small $p_X$};
%
\draw (0,-.1) node[below,scale=.7]{$0$} --(0,0);
\draw[dotted] (0,0)--(0,{\sy*(1-\p)}) node[scale=.7]{$\bullet$};
\draw (0,{\sy*(1-\p)})--(-.1,{\sy*(1-\p)}) node[left,scale=.7]{$1-p$};
%
\draw (\sx,-.1) node[below,scale=.8]{\small $1$} --(\sx,0);
\draw[dotted] (\sx,0)--(\sx,{\sy*\p}) node[scale=.7]{$\bullet$};
\draw (0,{\sy*\p})--(-.1,{\sy*\p}) node[left,scale=.7]{\small $p$};
%
\node at ({(\sx+.75)/2},-1) [scale=.9]{(a)};
\end{scope}
%
%
% reparticion
\begin{scope}[xshift=7cm]
%
\pgfmathsetmacro{\sy}{2};% y-scaling 
%
\draw[>=stealth,->] (-.5,0)--({\sx+1.5},0) node[right]{\small $x$};
\draw[>=stealth,->] (0,-.15)--(0,{\sy+.5}) node[above]{\small $F_X$};
%
\draw (0,0)--(0,-.1) node[below,scale=.7]{$0$};
\draw (\sx,0)--(\sx,-.1) node[below,scale=.7]{$1$};
\draw (0,{\sy*(1-\p)})--(-.1,{\sy*(1-\p)}) node[left,scale=.7]{$1-p$};
\draw (0,\sy)--(-.1,\sy) node[left,scale=.7]{$1$};
%
\draw[thick](-.25,0)--(0,0);
\draw ({0+\r},\r) arc (90:270:\r);
%
\draw[dotted] (0,0)--(0,{\sy*(1-\p)});
\draw[thick](0,{\sy*(1-\p)}) node[scale=.7]{$\bullet$}--(\sx,{\sy*(1-\p)});
\draw ({\sx+\r},{\r+\sy*(1-\p)}) arc (90:270:\r);
%
\draw[dotted] (\sx,{\sy*(1-\p)})--(\sx,\sy);
\draw[thick](\sx,\sy) node[scale=.7]{$\bullet$}--({\sx+1},\sy);
%\draw ({\sx+\r},{\r+\sy*(1-\p)}) arc (90:270:\r);
%
\node at ({(\sx+1.5)/2},-1) [scale=.9]{(b)};
\end{scope}
%
\end{tikzpicture} \end{center}
%
\leyenda{Ilustraci\'on de una distribuci\'on de probabilidad de Bernoulli (a), y
  la funci\'on de repartici\'on asociada (b), con $p = \frac13$.}
\label{Fig:MP:Bernoulli}
\end{figure}

Notar que cuando $p = 0$ (resp. $p = 1$) la variable es cierta $X = 0$ (resp. $X
= 1$). En estos casos, nuevamente,  siendo la varianza cero, no se puede definir
ni asimetr\'ia  (pero no  hay asimetr\'ia),  ni curtosis (pero  la ley  no tiene
colas, \ie colas livianas), como ya lo hemos visto anterioramente.

Se  notar\'a  tambi\'en  que  la   ley  de  Bernoulli  tiene  una  propiedad  de
reflexividad trivial:
%
\begin{lema}[Reflexividad]
\label{Lem:MP:ReflexividadBernoulli}
%
  Sea \ $X \, \sim \, \B(p)$. Entonces
  %
  \[
  1-X \, \sim \, \B(1-p)
  \]
  %
\end{lema}
\begin{proof}
El resultado es inmediato de $P(1-X = 1) = P(X = 0) = 1-p$.
\end{proof}

% --------------------------------- Binomial
\subsubseccion{Ley binomial}
\label{Sssec:MP:Binomial}

Esta ley  apareci\'o en  trabajos muy antiguos,  de nuevo y  naturalmente, entre
otros, en de J. Bernoulli  en 1713~\cite{Ber1713, Hal90, DavEdw01}.  Se la puede
ver como  una extension de  la ley  de Bernoulli a  \ $n \ge  1$ \ tiros  de una
moneda, contando por ejemplo cuantas veces aparecen una cara.

Se denota \ $X \,  \sim \, \B(n,p)$ \ con \ $n \in \Nset^*$,  \quad $p \in [0 \;
1]$ \ y sus caracter\'isticas son las siguientes:

\begin{caracteristicas}
%
Dominio de definici\'on & $\X = \{ 0 \; \ldots \; n \}$\\[2mm]
\hline
%
Par\'ametros & $n  \in \Nset^*,  \quad p \in [0  \;
1]$\\[2mm]
\hline
%
Distribuci\'on de probabilidad & \protect$\displaystyle p_X(x) = \bino{n}{x} \, p^x
(1-p)^{n-x}$\protect\\[2mm]
\hline
%
Promedio & $ m_X = n p$\\[2mm]
\hline
%
Varianza & $\sigma_X^2 = n p (1-p)$\\[2mm]
\hline
%
\modif{Asimetr\'ia} & $\displaystyle \gamma_X = \frac{1 - 2 p}{\sqrt{n p \, (1-p)}}$ \quad para \ $p \not\in \{ 0 \; 1 \}$ \ (ver m\'as adelante)\\[2mm]
\hline
%
Curtosis por exceso & $\displaystyle \widebar{\kappa}_X = \frac{1 - 6 \, p + 6
\, p^2}{n \, p \, (1-p)} $ \quad para \ $p \not\in \{ 0 \; 1 \}$ \ (ver m\'as adelante)\\[2mm]
\hline
%
Generadora  de probabilidad  &  $\displaystyle  G_X(z) =  \left(  1 -  p  + p  z
\right)^n$ \quad sobre \ $\Cset$\\[2mm]
\hline
%
Generadora  de momentos  &  $\displaystyle  M_X(u) =  \left(1  - p  +  p \,  e^u
\right)^n$ \quad sobre \ $\Cset$\\[2mm]
\hline
%
Funci\'on caracter\'istica  & $\displaystyle \Phi_X(\omega) =  \left( 1 -  p + p
\, e^{\imath \omega} \right)^n$
\end{caracteristicas}

% Momentos & $ \Esp\left[ X^k \right] = ??\\[2mm]
% Momento factorial & $\Esp\left[ (X)_k \right] = 
% \frac{n!}{(n-k)!} p^k \un_{\{ 0 \, , \, \ldots \, , \, n \}}(k)$\\[2mm]
% Modo $\left\lfloor (n+1) p \right\rfloor$
% Mediana $\left\lfloor n p \right\rfloor$ o $\left\lceil n p \right\rceil
% CDF	$I_{1-p}(n-k,k+1)$ regularized incomplete beta function

Su masa  de probabilidad  y funci\'on de  repartici\'on son representadas  en la
figura Fig.~\ref{Fig:MP:Binomial}.
%
\begin{figure}[h!]
\begin{center} \begin{tikzpicture}%[scale=.9]
\shorthandoff{>}
%
\pgfmathsetmacro{\sx}{.75};% x-scaling
\pgfmathsetmacro{\r}{.05};% radius arc non continuity F_X
\pgfmathsetmacro{\p}{1/3};% probabilidad p
\pgfmathsetmacro{\n}{6};% numero n de la binomial
\pgfmathsetmacro{\q}{floor((\n+1)*\p)};% modo de la binomial
\pgfmathsetmacro{\m}{factorial(\n)/factorial(\q)/factorial(\n-\q)*(\p^\q)*((1-\p)^(\n-\q))};% maximo de la binomial
% masa
\begin{scope}
%
\pgfmathsetmacro{\sy}{2.5/\m};% y-scaling 
\draw[>=stealth,->] (-.25,0)--({\sx*\n+.25},0) node[right]{\small $x$};
\draw[>=stealth,->] (0,-.1)--(0,{\sy*\m+.25}) node[above]{\small $p_X$};
%
\pgfmathsetmacro{\b}{(1-\p)^\n};% coeficiente binomial por la probabilidad
%
\foreach \k in {0,...,\n} {
\draw ({\k*\sx},0)--({\k*\sx},-.1) node[below,scale=.7]{\k};
\draw[dotted] ({\k*\sx},0)--({\k*\sx},{\sy*\b}) node[scale=.7]{$\bullet$};
%
\pgfmathsetmacro{\bl}{\b*\p*(\n-\k)/((\k+1)*(1-\p))};\global\let\b\bl;% proba actualizado
}
\draw (0,{((1-\p)^\n)*\sy})--(-.1,{((1-\p)^\n)*\sy}) node[left,scale=.7]{$(1-p)^n$};
\draw (0,{\n*\p*((1-\p)^(\n-1))*\sy})--(-.1,{\n*\p*((1-\p)^(\n-1))*\sy}) node[left,scale=.7]{$n p (1-p)^{n-1}$};
%
\end{scope}
%
%
% reparticion
\begin{scope}[xshift=8.5cm]
%
\pgfmathsetmacro{\sy}{2.5};% y-scaling 
%
\draw[>=stealth,->] (-.6,0)--({\sx*(\n+.5)+.5},0) node[right]{\small $x$};
\draw[>=stealth,->] (0,-.1)--(0,{\sy+.25}) node[above]{\small $F_X$};
%
\pgfmathsetmacro{\b}{(1-\p)^\n};% coeficiente binomial por la probabilidad
\pgfmathsetmacro{\c}{(1-\p)^\n};% cumulativa binomial por la probabilidad
%
% cumulativa x < 0
\draw (0,0)--(0,-.1) node[below,scale=.7]{0};
\draw[thick] (-.5,0)--(0,0);
\draw (\r,\r) arc (90:270:\r);
%
% cumulativa x de 0 a n-1
\foreach \k in {1,...,\n} {
\draw ({\k*\sx},0)--({\k*\sx},-.1) node[below,scale=.7]{\k};
\draw[thick]({(\k-1)*\sx},{\sy*\c}) node[scale=.7]{$\bullet$}--({\k*\sx},{\sy*\c});
\draw ({\k*\sx+\r},{\sy*\c+\r}) arc (90:270:\r);
\draw[dotted] ({(\k-1)*\sx},{(\c-\b)*\sy})--({(\k-1)*\sx},{\c*\sy});
%
\pgfmathsetmacro{\bl}{\b*\p*(\n-\k+1)/(\k*(1-\p))};\global\let\b\bl;% proba actualizado
\pgfmathsetmacro{\cl}{\c+\b};\global\let\c\cl;% cumulativa actualizada
}
%
% cumulativa x > n
\draw[dotted] ({\n*\sx},{(1-\b)*\sy})--({\n*\sx},\sy);
\draw[thick]({\n*\sx},\sy) node[scale=.7]{$\bullet$}--({(\n+.5)*\sx},\sy);
%
\draw (0,{((1-\p)^\n)*\sy})--(-.1,{((1-\p)^\n)*\sy}) node[left,scale=.7]{$(1-p)^n$};
\draw (0,{(\n*\p+1-\p)*((1-\p)^(\n-1))*\sy})--(-.1,{(\n*\p+1-\p)*((1-\p)^(\n-1))*\sy}) node[left,scale=.7]{$(1-p+np) (1-p)^{n-1}$};
\draw (-.2,{((\n*\p+1-\p)*((1-\p)^(\n-1))+1)/2*\sy}) node[scale=.7]{$\vdots$};
\draw (0,\sy)--(-.1,\sy) node[left,scale=.7]{$1$};
\end{scope}
%
\end{tikzpicture} \end{center}
%
\leyenda{Ilustraci\'on de una distribuci\'on  de probabilidad binomial (a), y la
  funci\'on de repartici\'on asociada (b), con $n = 6$, \quad $p = \frac13$.}
\label{Fig:MP:Binomial}
\end{figure}

\SZ{Otros ilustraciones para otros $p$?}

Cuando  $n  = 1$,  se  recupera  la lei  de  Bernoulli  $\B(p) \equiv  \B(1,p)$.
Ad\'emas, se muestra  sencillamente usando la generadora de  probabilidad que
%
\begin{lema}
\label{Lem:BinomialSumaBernoulli}
%
  Sean \  $X_i \,  \sim \, \B(p),  \quad i  = 1, \ldots  , n$  \ independientes,
  entonces
  %
  \[
  \sum_{i=1}^n X_i \, \sim \, \B(n,p)
  \]
\end{lema}
%
De este resultado,  se puede notar que, por  ejemplo, le distribuci\'on binomial
aparece en el conteo de eventos independientes de misma probabilidad entre $n$.

Tambi\'en,  la ley binomial  tiene una  propiedad de  reflexividad, consecuencia
directa de la de Bernoulli:
%
\begin{lema}[Reflexividad]
\label{Lem:MP:ReflexividadBinomial}
%
  Sea \ $X \, \sim \, \B(n,p)$. Entonces
  %
  \[
  n-X \, \sim \, \B(n,1-p)
  \]
  %
\end{lema}
%
\begin{proof}
  El  resultado es  inmediato  de la  propiedad  de reflexividad  de  la ley  de
  Bernoulli,                           conjuntamente                          al
  lema~\ref{Lem:BinomialSumaBernoulli}. Alternativamente,  se nota que  $P(n-X =
  x) = P(X = n-x) = \bino{n}{n-x} p^{n-x} (1-p)^x = \bino{n}{x} (1-p)^x p^{n-x}$
  \ notando que $\bino{n}{n-x} = \bino{n}{x}$.
\end{proof}
%
Si  tomamos  el  ejemplo  de  una  moneda  que se  tira  $n$  veces  de  maneras
independientes, con  probabilidad $p$ que  aparezca una cara, $X$  representa el
n\'umero de caras tiradas. Entonces, $n-X$  es el n\'umero de secas: en $n-X$ se
intercambian los roles de la cara y de la seca.

Nota que cuando $p = 0$ (resp. $p = 1$) la variable es cierta $X = 0$ (resp.  $X
= n$).  En estos casos, nuevamente,  siendo la varianza cero, no se puede definir
ni asimetr\'ia  (pero no  hay asimetr\'ia),  ni curtosis (pero  la ley  no tiene
colas, \ie colas livianas), como ya lo hemos visto anterioramente.


% --------------------------------- Binomial negativa
\subsubseccion{Ley Binomial negativa}
\label{Sssec:MP:BinomialNegativa}

Se denota \  $X \, \sim \, N\B(r,p)$ \  con \ $r \in \Nset^*, \quad  p \in [0 \;
1)$ \ y sus caracter\'isticas son las siguientes:

\begin{caracteristicas}
%
Dominio de definici\'on & $\X = \Nset$\\[2mm]
\hline
%
Parametros & $r  \in \Nset^*,  \quad p \in [0  \;
1)$\\[2mm]
\hline
%
Distribuci\'on  de  probabilidad  &  \protect$\displaystyle  p_X(k)  =  \bino{k+r-1}{k}  p^k
(1-p)^r$\protect\\[2mm]
\hline
%
Promedio & $\displaystyle m_X = \frac{r \, p}{1-p}$\\[2mm]
\hline
%
Varianza & $\displaystyle \sigma_X^2 = \frac{r \, p}{(1-p)^2}$\\[2mm]
\hline
%
\modif{Sesgo} & $\displaystyle \gamma_X = \frac{1 + p}{\sqrt{r \, p}}$\\[2mm]
\hline
%
Curtosis por exceso & $\displaystyle \widebar{\kappa}_X = \frac{1 + 4 \, p +
p^2}{r \, p} $\\[2mm]
\hline
%
Generadora  de probabilidad  &  $\displaystyle  G_X(z) =  \left(  \frac{1 -  p}{1  - p \, z}
\right)^r$ \ para \ $|z| < p^{-1} $\\[2mm]
\hline
%
Generadora de momentos & $\displaystyle M_X(u) = \left( \frac{1 - p}{1 - p \,
e^u } \right)^r$ \ para \ $\real{u} < - \ln p$\\[2mm]
\hline
%
Funci\'on caracter\'istica & $\displaystyle \Phi_X(\omega) = \left( \frac{1 -
p}{1 - p \, e^{i \omega} } \right)^r$
\end{caracteristicas}

% Momentos & $ \Esp\left[ X^k \right] = ??\\[2mm]
% Momento factorial & $\Esp\left[ (X)_k \right] = 
% \frac{(r+k-1)!}{(r-1)!} \left( \frac{p}{1-p} \right)^k$\\[2mm]
% Modo $\left\lfloor (n+1) p \right\rfloor$
% Mediana $\left\lfloor n p \right\rfloor$ o $\left\lceil n p \right\rceil
% CDF	$I_{1-p}(n-k,k+1)$ regularized incomplete beta function

Su masa  de probabilidad  y funci\'on de  repartici\'on son representadas  en la
figura Fig.~\ref{Fig:MP:BinomialNegativa}.
%
\begin{figure}[h!]
\begin{center} \begin{tikzpicture}%[scale=.9]
\shorthandoff{>}
%
\pgfmathsetmacro{\sx}{.45};% x-scaling
\pgfmathsetmacro{\r}{.05};% radius arc non continuity F_X
\pgfmathsetmacro{\p}{3/5};% probabilidad p de suceso
\pgfmathsetmacro{\rp}{3};% numero r de fracascos
\pgfmathsetmacro{\n}{12};% numero maximo a dibujar
\pgfmathsetmacro{\q}{max(floor((\rp-1)*\p/(1-\p)),0)};% modo de la binomial negativa
\pgfmathsetmacro{\m}{factorial(\q+\rp-1)*((1-\p)^\rp)*(\p^\q)/factorial(\rp-1)/factorial(\q)};
%{factorial(\n)/factorial(\q)/factorial(\n-\q)*(\p^\q)*((1-\p)^(\n-\q))};% maximo de la binomial
% masa
\begin{scope}
%
\pgfmathsetmacro{\sy}{2.5/\m};% y-scaling 
\draw[>=stealth,->] (-.25,0)--({\sx*(\n+.75)+.25},0) node[right]{\small $x$};
\draw[>=stealth,->] (0,-.1)--(0,{\sy*\m+.25}) node[above]{\small $p_X$};
%
\pgfmathsetmacro{\b}{(1-\p)^\rp};% coeficiente binomial por la probabilidad
%
\foreach \k in {0,...,\n} {
\draw ({\k*\sx},0)--({\k*\sx},-.1) node[below,scale=.7]{$\k$};
\draw[dotted] ({\k*\sx},0)--({\k*\sx},{\sy*\b}) node[scale=.7]{$\bullet$};
%
\pgfmathsetmacro{\bl}{\b*\p*(\k+\rp)/(\k+1)};\global\let\b\bl;% proba actualizada
}
\draw ({(\n+.25)*\sx},{\sy*\b*(\n+1)/\p/(\n+\rp)/2}) node[right,scale=.7]{\ldots};
\draw (0,{((1-\p)^\rp)*\sy})--(-.1,{((1-\p)^\rp)*\sy}) node[left,scale=.7]{$(1-p)^r$};
\draw (0,{\rp*\p*((1-\p)^\rp)*\sy})--(-.1,{\rp*\p*((1-\p)^\rp)*\sy}) node[left,scale=.7]{$r \, p \, (1-p)^r$};
%\draw (0,{(\rp*\p*((1-\p)^\rp)+\m)/2*\sy}) node[scale=.7]{$r \, p \, (1-p)^r$};
%
\end{scope}
%
%
% reparticion
\begin{scope}[xshift=8.5cm]
%
\pgfmathsetmacro{\sy}{2.5};% y-scaling 
%
\draw[>=stealth,->] (-.6,0)--({\sx*(\n+.75)+.5},0) node[right]{\small $x$};
\draw[>=stealth,->] (0,-.1)--(0,{\sy+.25}) node[above]{\small $F_X$};
%
\pgfmathsetmacro{\b}{(1-\p)^\rp};% coeficiente binomial por la probabilidad
\pgfmathsetmacro{\c}{(1-\p)^\rp};% cumulativa binomial por la probabilidad
%
% cumulativa x < 0
\draw (0,0)--(0,-.1) node[below,scale=.7]{$0$};
\draw[thick] (-.5,0)--(0,0);
\draw (\r,\r) arc (90:270:\r);
%
% cumulativa x de 0 a n-1
\foreach \k in {1,...,\n} {
\draw ({\k*\sx},0)--({\k*\sx},-.1) node[below,scale=.7]{$\k$};
\draw[thick]({(\k-1)*\sx},{\sy*\c}) node[scale=.7]{$\bullet$}--({\k*\sx},{\sy*\c});
\draw ({\k*\sx+\r},{\sy*\c+\r}) arc (90:270:\r);
\draw[dotted] ({(\k-1)*\sx},{(\c-\b)*\sy})--({(\k-1)*\sx},{\c*\sy});
%
\pgfmathsetmacro{\bl}{\b*\p*(\k+\rp-1)/\k};\global\let\b\bl;% proba actualizada
\pgfmathsetmacro{\cl}{\c+\b};\global\let\c\cl;% cumulativa actualizada
}
%
\draw ({\n*\sx},{\sy*(\c+1)/2}) node[left,scale=.7]{\ldots};
\draw (0,{((1-\p)^\rp)*\sy})--(-.1,{((1-\p)^\rp)*\sy}) node[left,scale=.7]{$(1-p)^r$};
\draw (0,{(1+\rp*\p)*((1-\p)^\rp)*\sy})--(-.1,{(1+\rp*\p)*((1-\p)^\rp)*\sy}) node[left,scale=.7]{$(1+r \, p) (1-p)^r$};
\draw (-.75,{((1+\rp*\p)*((1-\p)^\rp)+1)/2*\sy}) node[right,scale=.7]{$\vdots$};
\draw (0,\sy)--(-.1,\sy) node[left,scale=.7]{$1$};
\end{scope}
%
\end{tikzpicture} \end{center}
%
\leyenda{Ilustraci\'on de  una distribuci\'on de  probabilidad binomial negativa
  (a), y  la funci\'on  de repartici\'on  asociada (b), con  $r =  3, \quad  p =
  \frac35$.}
\label{Fig:MP:BinomialNegativa}
\end{figure}
\SZ{Otros ilustraciones para otros $r, p$?}

Esta ley aparece cuando se repite una experencia binaria \ $X_i \in \{ 0 \, , \,
1 \},  i = 1, \ldots$  \ con \ $P(X_i=1)  = p$ \ de  manera independiente ($X_i$
independientes) hasta  que \  $r$ \  variables valen 0,  con \  $r$ \  fijo. Los
n\'umeros de  excitos \ $X_i =  1$ \ sigue una  ley \ $N\B(r,p)$  (el calculo es
directo).  Dicho de otra  manera, $X = \sum_{i=1}^N X_i$ \ con  \ $N$ \ variable
aleatoria tal que $X_N = 0$ \ y \ $r = \sum_{i=1}^N (1-X_i)$: condicionalmente a
\ $N$, la variable es binomial de parametro $p$.  Se puede ver que \ $P(N = n) =
\bino{n}{r-1} (1-p)^r  p^{n-r}$ \ y la  ley de la binomial  negativa se recupera
tambi\'en,          por         ejemplo,         a          trav\'es         del
teorema~\ref{Teo:MP:SumaAleatoriaGeneradoraProbabilidad}. En el caso \ $r = 1$ \
aparece que \ $N \sim \G(1-p)$ \ y\ldots tambi\'en \ $X \sim \G(1-p)$.
%
% Blaise PAscal - Polya caso r real

Esta distribuci\'on se  generaliza para \ $r \in \Rset_+^*$ \  pero se pierde la
interpretaci\'on que v\'imos en el p\'arafo anterior.

Nota: cuando \ $p = 0$ \ la variable es cierta \ $X = r$.

% --------------------------------- Multinomial
\subsubseccion{Ley multinomial}
\label{Sssec:MP:Multinomial}

Esta ley es una generalizaci\'on de la ley binomial y aparece por ejemplo cuando
se  repite  una  experiencia  a  \  $k$  \ estados  \  $n$  \  veces  de  manera
independiente y nos  interesamos a la probabilidad que  el primer evento aparece
$n_1$ veces,  el secundo  $n_2$ veces, \ldots  (ej. para  $k = 6$,  contamos los
n\'umeros de $1$, de $2$, \ldots cuando tiramos $n$ veces este dado). Apareci\'o
tambi\'en   esta   ley   por   la    primera   vez   en   el   trabajo   de   J.
Bernoulli~\cite{Ber1713, Hal90,  DavEdw01} (ver tambi\'en el  ensayo de Montmort
de 1708 con otras notaciones~\cite{Mon13}).

Se  denota  \  $X  \ \sim  \  \M(n,p)$  \  con  \  $n  \in  \Nset^*$ \  y  \  $p
= \begin{bmatrix}  p_1 & \cdots  & p_k \end{bmatrix}^t  \in \Simp{k-1}$ \  the \
$(k-1)$-simplex estandar  (ver figura~\ref{Fig:MP:Dirichlet}-(a) m\'as adelante,
y  notaciones).   Entonces,  a pesar  de  que  se  escribe  \  $X$ \  de  manera
$k$-dimensional,  el vector  partenece a  un espacio  claramente \  $d =  k-1$ \
dimensional y en el caso \ $k = 2$ \ se recupera la ley binomial.  El dominio de
definici\'on es claramente $\Part{n}{k}$ (ver notaciones). Las caracter\'isticas
de \ $X \ \sim \ \M(n,p)$ \ son las siguientes:

\begin{caracteristicas}
%
Dominio de definici\'on
%~\footnote{De hecho, se puede considerar que el vector
%aleatorio es \ $(k-1)$-dimensional \ $\widetilde{X} = \begin{bmatrix}
%\widetilde{X}_1 & \cdots & \widetilde{X}_{k-1} \end{bmatrix}^t$ \ definido sobre
%el dominio \ $\widetilde{\X} = \left\{ x \in \{ 0 \; \ldots \; n\}^{k-1}, \:
%\sum_{i=1}^{k-1} x_i \le n \right\}$.\label{Foot:MP:MultinomialDominio}}
 & $\X = \Part{n}{k}$
%\left\{ x \in \{ 0 \; \ldots \; n\}^k \tq \sum_{i=1}^k x_i = n \right\}$
\\[2mm]
\hline
%
Par\'ametros
%~\footnote{El par\'ametro de \ $\widetilde{X}$ \ es \ $\widetilde{p} =
%\protect\begin{bmatrix} p_1 & \cdots & p_{k-1} \end{bmatrix}^t\protect \in
%\left\{ q \in [0 \; 1]^{k-1} \tq \sum_{i=1}^{k-1} q_i \le 1
%\right\}$.\label{Foot:MP:MultinomialParametro}}
 & $n \in \Nset^*$, \quad $p \in
\Simp{k-1}$\\[2mm]
\hline
%
Distribuci\'on de probabilidad
%~\footnote{La masa de probabilidad de \
%$\widetilde{X}$ \ es \ $p_{\widetilde{X}}(x) = \frac{n!}{\prod_{i=1}^{k-1} x_i!
%(n-\sum_{i=1}^{k-1} x_i)!}  \prod_{i=1}^{k-1} p_i^{x_i} \, \left( 1 -
%\sum_{i=1}^{k-1} p_i \right)^{n-\sum_{i=1}^{k-1}
%x_i}$.\label{Foot:MP:MultinomialMasa}}
 & $\displaystyle p_X(x) =
\frac{n!}{\prod_{i=1}^k x_i!}  \prod_{i=1}^k p_i^{x_i}$\\[2mm]
\hline
%
Promedio & $\displaystyle m_X = n p$\\[2mm]
\hline
%
Covarianza
%~\footnote{$\Sigma_X \in \Pos_k(\Rset)$, pero de \ $\un^t \Sigma_X \un =
%0$ \ viene \ $\Sigma_X \not\in \Pos_k^+(\Rset)$. Eso es la consecuencia directa del
%hecho de que \ $X$ \ $d$-dimensional, vive sobre \ $\Simp{k-1}$,
%$(d-1)$-dimensional.\label{Foot:MP::MultinomialCovarianza}}
 & $\displaystyle
\Sigma_X = n \left( \diag(p) - p p^t \right)$\\[2mm]
\hline
%
Generadora de probabilidad
%~\footnote{Notar: $G_{\widetilde{X}}\left(
%\widetilde{z} \right) = G_X\left( \begin{bmatrix} \widetilde{z} &
%1 \end{bmatrix}^t \right)$ \ y al rev\'es \ $G_X(z) = z_k^n \,
%G_{\widetilde{X}}\left( \begin{bmatrix} \frac{z_1}{z_k} & \cdots &
%\frac{z_{k-1}}{z_k} \end{bmatrix}^t
%\right)$.\label{Foot:MP:MultinomialGeneProba}}
 & $\displaystyle G_X(z) = \left(
p^t z \right)^n$ \ para \ $z \in \Cset^k$\\[2mm]
\hline
%
Generadora de momentos
%~\footnote{Notar: $M_{\widetilde{X}}\left( \widetilde{u}
%\right) = M_X\left( \begin{bmatrix} \widetilde{u} & 0 \end{bmatrix}^t \right)$ \
%y \ $M_X(u) = e^{n \, u_k} M_{\widetilde{X}}\left( \begin{bmatrix} u_1 - u_k &
%\cdots & u_{k-1} - u_k \end{bmatrix}^t
%\right)$.\label{Foot:MP:MultinomialGeneMomentos}}
 & \protect$\displaystyle
M_X(u) = \left( p^t e^u \right)^n, \: e^u = \begin{bmatrix} e^{u_1} & \cdots &
e^{u_k} \end{bmatrix}^t$\protect \ para \ $u \in \Cset^k$\\[2mm]
\hline
%
Funci\'on caracter\'istica
%~\footnote{Notar: $\Phi_{\widetilde{X}}\left(
%\widetilde{\omega} \right) = \Phi_X\left( \begin{bmatrix} \widetilde{\omega} &
%0 \end{bmatrix}^t \right)$ \ o \ $\Phi_X(\omega) = e^{\imath \, n \, \omega_k}
%\Phi_{\widetilde{X}}\left( \begin{bmatrix} \omega_1 - \omega_k & \cdots &
%%\omega_{k-1} - \omega_k \end{bmatrix}^t
%\right)$.\label{Foot:MP:MultinomialCaracteristica}} 
& $\displaystyle
\Phi_X(\omega) = \left( p^t e^{\imath \omega} \right)^n$
\end{caracteristicas}

% Momentos & $ \Esp\left[ X^k \right] = ??\\[2mm]
% Momento factorial & $\Esp\left[ (X)_k \right] = 
% \frac{(r+k-1)!}{(r-1)!} \left( \frac{p}{1-p} \right)^k$\\[2mm]
% Modo $\left\lfloor (n+1) p \right\rfloor$
% Mediana $\left\lfloor n p \right\rfloor$ o $\left\lceil n p \right\rceil
% CDF	$I_{1-p}(n-k,k+1)$ regularized incomplete beta function

De hecho, se puede considerar que el vector aleatorio es \ $(k-1)$-dimensional \
$\widetilde{X}     =    \begin{bmatrix}     \widetilde{X}_1    &     \cdots    &
  \widetilde{X}_{k-1}   \end{bmatrix}^t$   \  definido   sobre   el  dominio   \
$\widetilde{\X} = \left\{ x \in \{ 0 \; \ldots \; n\}^{k-1}, \: \sum_{i=1}^{k-1}
  x_i  \le n  \right\}$. El  par\'ametro de  \ $\widetilde{X}$  \ es  entonces \
$\widetilde{p}     =     \protect\begin{bmatrix}      p_1     &     \cdots     &
  p_{k-1}  \end{bmatrix}^t\protect  \in  \left\{   q  \in  [0  \;  1]^{k-1}  \tq
  \sum_{i=1}^{k-1}  q_i  \le  1  \right\}$.    La  masa  de  probabilidad  de  \
$\widetilde{X}$   \   se    escribe   obviamente   \   $p_{\widetilde{X}}(x)   =
\frac{n!}{\prod_{i=1}^{k-1} x_i!   (n-\sum_{i=1}^{k-1} x_i)!}  \prod_{i=1}^{k-1}
p_i^{x_i} \, \left( 1  - \sum_{i=1}^{k-1} p_i \right)^{n-\sum_{i=1}^{k-1} x_i}$.
Se  notar\'a al  final que  \ $G_{\widetilde{X}}\left(  \widetilde{z}  \right) =
G_X\left(  \begin{bmatrix} \widetilde{z}  & 1  \end{bmatrix}^t \right)$  \  y al
rev\'es   \   $G_X(z)  =   z_k^n   \,  G_{\widetilde{X}}\left(   \begin{bmatrix}
    \frac{z_1}{z_k}  & \cdots  &  \frac{z_{k-1}}{z_k} \end{bmatrix}^t  \right)$.
Similarmente,    \     $M_{\widetilde{X}}\left(    \widetilde{u}    \right)    =
M_X\left(  \begin{bmatrix} \widetilde{u}  &  0 \end{bmatrix}^t  \right)$  \ y  \
$M_X(u)  = e^{n  \, u_k}  M_{\widetilde{X}}\left(  \begin{bmatrix} u_1  - u_k  &
    \cdots  &  u_{k-1}  -  u_k  \end{bmatrix}^t \right)$  (y  similarmente  para
$\Phi_X$ y $\Phi_{\widetilde{X}}$).

Se puede ver  tambi\'en que $\Sigma_X \un  = 0$ \ as\'i que  \ $\Sigma_X \not\in
\Pos_k^+(\Rset)$.   Eso  es la  consecuencia  directa  del hecho  de  que  \ $X$  \
$k$-dimensional, vive sobre  \ $\Simp{k-1}$, $(k-1)$-dimensional. Aparentemente,
siendo  $\Sigma_X$  no  invertible,  no  se  puede  definir  ni  asimetr\'ia,  ni
curtosis. Sin embargo, habr\'ia que considerar \ $\widetilde{X}$, de promedio $n
\widetilde{p}$ y de covarianza el bloque $(k-1) \times (k-1)$ de $\Sigma_X$, que es
ahora invertible. $\gamma_{\widetilde{X}}$ \  y \ $\kappa_{\widetilde{X}}$ \ son
bien definidos. Las expresiones, demasiado pesadas, no son dadas ac\'a.

Deos ejemplos de masa de probabilidad de esta ley son representadas en la figura
Fig.~\ref{Fig:MP:Multinomial}.
%
\begin{figure}[h!]
\begin{center} \begin{tikzpicture}[scale=.8]
\shorthandoff{>}
%
%
\pgfmathsetmacro{\n}{5};% numeros para la multinomial
\pgfmathsetmacro{\dec}{.5};% shitf para dibujar las marginales
%
% Ejemplo [6 5 4]/15
\begin{scope}
%
\pgfmathsetmacro{\pu}{2/5};% p_1
\pgfmathsetmacro{\pd}{1/3};% p_2
\pgfmathsetmacro{\qu}{floor((\n+1)*\pu)};% modo de la binomial 1
\pgfmathsetmacro{\qd}{floor((\n+1)*\pd)};% modo de la binomial 2
\pgfmathsetmacro{\mau}{factorial(\n)/factorial(\qu)/factorial(\n-\qu)*(\pu^\qu)*((1-\pu)^(\n-\qu))};% maximo de la binomial 1
\pgfmathsetmacro{\mad}{factorial(\n)/factorial(\qd)/factorial(\n-\qd)*(\pd^\qd)*((1-\pd)^(\n-\qd))};% maximo de la binomial 2
\pgfmathsetmacro{\ma}{max(\mau,\mad)};% maximo de ambas binomiales
%
\begin{axis}[
    colormap = {whiteblack}{color(0cm)  = (white);color(1cm) = (black)},
    width=.55\textwidth,
    view={35}{70},
    enlargelimits=false,
    xmin={-\dec},
    xmax={\n+\dec},
    ymin={-\dec},
    ymax={\n+\dec},
    zmax={1.1*\ma},
    color=black,
    xtick={0,...,\n},
    ytick={0,...,\n},
    xlabel=$x_1$,
    ylabel=$x_2$,
    zlabel=$p_{\widetilde{X}}$,
]
%
\pgfmathsetmacro{\bu}{(1-\pu-\pd)^\n};% coeficiente binomial por la probabilidad p1
\pgfmathsetmacro{\bd}{\bu};% coeficiente binomial por la probabilidad p2
%
\pgfmathsetmacro{\bmu}{(1-\pu)^\n};% lo mismo para la marginale 1
\pgfmathsetmacro{\bmd}{(1-\pd)^\n};% lo mismo para la marginale 2
%
\foreach \mu in {0,...,\n} {
  \foreach \md in {0,...,\n} {
    \ifnum \numexpr\mu+\md < \numexpr\n+1
      \addplot3 [dotted,domain=0:\bd,samples=2, samples y=0,color=black] (\mu,\md,\x)  node[scale=.85]{$\bullet$};
      %
      \pgfmathsetmacro{\bld}{\bd*\pd*(\n-\md)/((\md+1)*(1-\pu-\pd))};
      \global\let\bd\bld;% proba en m2 (m1 fijo) actualizado
    \fi
  }
  %
  % Marginales
  \addplot3 [dotted,domain=0:\bmu,samples=2, samples y=0,color=black] (\mu,{\n+\dec},\x)  node[scale=.55]{$\bullet$};
  \addplot3 [dotted,domain=0:\bmd,samples=2, samples y=0,color=black] ({-\dec},\mu,\x)  node[scale=.55]{$\bullet$};
  %
  % lineas (m1,m2) abajo
  \addplot3 [domain={-\dec}:{\n+\dec},samples=2, samples y=0,color=black!10] (\mu,\x,0);
  \addplot3 [domain={-\dec}:{\n+\dec},samples=2, samples y=0,color=black!10] (\x,\mu,0);
  %
  \pgfmathsetmacro{\blu}{\bu*\pu*(\n-\mu)/((\mu+1)*(1-\pu-\pd))};
  \global\let\bu\blu;\global\let\bd\blu;% proba inicial en m1 actualizada
  %
  % lo mismo para cada marginal
  \pgfmathsetmacro{\blmu}{\bmu*\pu*(\n-\mu)/((\mu+1)*(1-\pu))};
  \global\let\bmu\blmu;% proba 1 actualizada
  \pgfmathsetmacro{\blmd}{\bmd*\pd*(\n-\mu)/((\mu+1)*(1-\pd))};
  \global\let\bmd\blmd;% proba 2 actualizada
}
%
\node at (axis cs:{3*\n/4},{\n+\dec},{\mau/2})[right]{$p_{X_1}$};
\node at (axis cs:{-\dec},{3*\n/4},{\mad/2})[above]{$p_{X_2}$};
\end{axis}
\node at ({3*\n/4},-1)[scale=.9]{(a)};
\end{scope}
%
%
%
%
% Ejemplo [1 1 1]/3
\begin{scope}[xshift = 11cm]
%
\pgfmathsetmacro{\pu}{1/3};% p_1
\pgfmathsetmacro{\pd}{1/2};% p_2
\pgfmathsetmacro{\qu}{floor((\n+1)*\pu)};% modo de la binomial 1
\pgfmathsetmacro{\qd}{floor((\n+1)*\pd)};% modo de la binomial 2
\pgfmathsetmacro{\mau}{factorial(\n)/factorial(\qu)/factorial(\n-\qu)*(\pu^\qu)*((1-\pu)^(\n-\qu))};% maximo de la binomial 1
\pgfmathsetmacro{\mad}{factorial(\n)/factorial(\qd)/factorial(\n-\qd)*(\pd^\qd)*((1-\pd)^(\n-\qd))};% maximo de la binomial 2
\pgfmathsetmacro{\ma}{max(\mau,\mad)};% maximo de ambas binomiales
%
\begin{axis}[
    colormap = {whiteblack}{color(0cm)  = (white);color(1cm) = (black)},
    width=.55\textwidth,
    view={35}{70},
    enlargelimits=false,
    xmin={-\dec},
    xmax={\n+\dec},
    ymin={-\dec},
    ymax={\n+\dec},
    zmax={1.1*\ma},
    color=black,
    xtick={0,...,\n},
    ytick={0,...,\n},
    xlabel=$x_1$,
    ylabel=$x_2$,
    zlabel=$p_{\widetilde{X}}$,
]
%
\pgfmathsetmacro{\bu}{(1-\pu-\pd)^\n};% coeficiente binomial por la probabilidad p1
\pgfmathsetmacro{\bd}{\bu};% coeficiente binomial por la probabilidad p2
%
\pgfmathsetmacro{\bmu}{(1-\pu)^\n};% lo mismo para la marginale 1
\pgfmathsetmacro{\bmd}{(1-\pd)^\n};% lo mismo para la marginale 2
%
\foreach \mu in {0,...,\n} {
  \foreach \md in {0,...,\n} {
    \ifnum \numexpr\mu+\md < \numexpr\n+1
      \addplot3 [dotted,domain=0:\bd,samples=2, samples y=0,color=black] (\mu,\md,\x)  node[scale=.85]{$\bullet$};
      %
      \pgfmathsetmacro{\bld}{\bd*\pd*(\n-\md)/((\md+1)*(1-\pu-\pd))};
      \global\let\bd\bld;% proba en m2 (m1 fijo) actualizado
    \fi
  }
  %
  % Marginales
  \addplot3 [dotted,domain=0:\bmu,samples=2, samples y=0,color=black] (\mu,{\n+\dec},\x)  node[scale=.55]{$\bullet$};
  \addplot3 [dotted,domain=0:\bmd,samples=2, samples y=0,color=black] ({-\dec},\mu,\x)  node[scale=.55]{$\bullet$};
  %
  % lineas (m1,m2) abajo
  \addplot3 [domain={-\dec}:{\n+\dec},samples=2, samples y=0,color=black!10] (\mu,\x,0);
  \addplot3 [domain={-\dec}:{\n+\dec},samples=2, samples y=0,color=black!10] (\x,\mu,0);
  %
  \pgfmathsetmacro{\blu}{\bu*\pu*(\n-\mu)/((\mu+1)*(1-\pu-\pd))};
  \global\let\bu\blu;\global\let\bd\blu;% proba inicial en m1 actualizada
  %
  % lo mismo para cada marginal
  \pgfmathsetmacro{\blmu}{\bmu*\pu*(\n-\mu)/((\mu+1)*(1-\pu))};
  \global\let\bmu\blmu;% proba 1 actualizada
  \pgfmathsetmacro{\blmd}{\bmd*\pd*(\n-\mu)/((\mu+1)*(1-\pd))};
  \global\let\bmd\blmd;% proba 2 actualizada
}
%
\node at (axis cs:{3*\n/4},{\n+\dec},{\mau/2})[right]{$p_{X_1}$};
\node at (axis cs:{-\dec},{3*\n/4},{\mad/2})[above]{$p_{X_2}$};
\end{axis}
\node at ({3*\n/4},-1)[scale=.9]{(b)};
\end{scope}
%
\end{tikzpicture} \end{center}
%
\leyenda{Ilustraci\'on de una distribuci\'on  de probabilidad multinomial para \
  $k   =  3$   \   del   vector  \   $(k-1)$-dimensional   \  $\widetilde{X}   =
  \protect\begin{bmatrix}   X_1  &  X_2   \protect\end{bmatrix}^t$  \   ($X_3  =
  1-X_1-X_2$)  \ con  las marginales  \ $p_{X_1},  \: p_{X_2}$.
  %      \      (ver      notas     de      pie~\ref{Foot:MP:MultinomialDominio}
  % y~\ref{Foot:MP:MultinomialMasa}).
  Es dibujada  solamente la  distribuci\'on sobre $\widetilde{\X}$,  siendo esta
  nula afuera de  $\widetilde{\X}$.  Los par\'ametros son \  $n = 5$ \ y  \ $p =
  \protect\begin{bmatrix}      \frac25      &      \frac13     &      \frac4{15}
    \protect\end{bmatrix}^t$ (a), $p = \protect\begin{bmatrix} \frac13 & \frac12
    & \frac16 \protect\end{bmatrix}^t$ (b).}
\label{Fig:MP:Multinomial}
\end{figure}


Notar: cuando $p = \un_i$, la variable es cierta $X = n \un_i$.

\SZ{Otros ilustraciones para otros $n, p$?}


Vectores  de  distribuci\'on  multinomial  tienen  una  propiedade  notable  con
respecto a una permutaci\'on de variable, parecidas a la de la binomial:
%
\begin{lema}[Efecto de una permutaci\'on]\label{Lem:MP:PermutacionMultinomial}
%
  Sea \ $X \, \sim \, \M(n,p), \: p \in \Simp{k-1}$ \ y \ $\Pi \in \perm_k(\Rset)$ \
  matriz \ de permutaci\'on (ver notaciones). Entonces
  %
  \[
  \Pi X \, \sim \, \M\left( n ,  \Pi p \right)
  \]
  %
\end{lema}
%
\begin{proof}
  El  resultado  es  inmediato  saliendo  de  la  funci\'on  caracter\'istica  y
  aplicando  el  teorema~\ref{Teo:MP:PropiedadesFuncionCaracteristica} (recordar
  que $\Pi^{-1} = \Pi^t$). M\'as directamente, notando la permutation \ $\sigma$
  \ tal que  \ $\Pi = \sum_{i=1}^k \un_i \un_{\sigma(i)}^t$, se  puede ver que \
  $\displaystyle  P(\Pi X =  x) =  P(X =  \Pi^{-1} x)  = \frac{n!}{\prod_{i=1}^k
    x_{\sigma^{-1}(i)}!}       \prod_{i=1}^k      p_i^{x_{\sigma^{-1}(i)}}     =
  \frac{n!}{\prod_{i=1}^k x_i!}  \prod_{i=1}^k p_{\sigma(i)}^{x_i}$ \ por cambio
  de indices.
\end{proof}
%
Adem\'as, la ley multinomial  exhibe una stabilidad reemplazando dos componentes
por su suma:
%
\begin{lema}[Stabilidad por agregaci\'on]\label{Lem:MP:StabAgregacionMultinomial}
%
  Sea  \ $X =  \begin{bmatrix} X_1  & \cdots  & X_k  \end{bmatrix}^t \,  \sim \,
  \M(n,p), \:  p \in \Simp{k-1}$ \ y  \ $G^{(i,j)}$ \ matriz  de agrupaci\'on de
  las $(i,j)$-\'esima componentes (ver notaciones). Entonces,
  %
  \[
  G^{(i,j)} X \, \sim \, \M\left( n , G^{(i,j)} p \right)  
  \]
  %
\end{lema}
%
Este resultado es intuitivo del hecho que vuelve a agrupar los estados \ $i$ \ e
\ $j$  \ en un  estado, que tiene  entonces la probabilidad \  $p_i + p_j$  \ de
aparecer.
%
\begin{proof}
  Suponemos $i <  j$ (el otro caso  se recupera por simetr\'ia). A  partir de la
  funci\'on                 caracter\'istica                 y                el
  teorema~\ref{Teo:MP:PropiedadesFuncionCaracteristica} se tiene,
  %
  \begin{eqnarray*}
  \forall \: \omega \in \Rset^{k-1}, \quad \Phi_{G^{(i,j)} X}(\omega) & = &
  \Phi_X\left( G^{(i,j) \, t} \omega \right)\\[2mm]
  %
  & = & \left( \sum_{l=1}^k p_l \, e^{\imath \, \left( G^{(i,j) \, t} \omega \right)_l } \right)^n
  \end{eqnarray*}
  %
  Ahora,  se nota  que \  $G^{(i,j) \,  t} \omega  = \begin{bmatrix}  \omega_1 &
    \cdots    &   \omega_{j-1}   &    \omega_i   &    \omega_{j+1}   &    \cdots   &
    \omega_{k-1} \end{bmatrix}^t$, entonces
  %
  \begin{eqnarray*}
  \forall \: \omega \in \Rset^{k-1}, \quad \Phi_{G^{(i,j)} X}(\omega) & = &
  \left( \sum_{l=1, l \ne j}^k p_l \, e^{\imath \, \omega_l} + p_j \, e^{\imath \,
  \omega_i } \right)^n\\[2mm]
  %
  & = & \left( \sum_{l=1, l \ne i, l \ne j}^k p_l \, e^{\imath \, \omega_l} +
  (p_i+p_j) \, e^{\imath \, \omega_i } \right)^n
  %
  \end{eqnarray*}
  %
  lo que  cierra la  prueba. Se puede  tener un  enfoque m\'as directo,  con los
  mismos         pasos        que         en        la         prueba        del
  lema~\ref{Lem:MP:StabAgregacionHipergeomMulti}    tratando     de    la    ley
  hipergeometrica multivaluada.
\end{proof}

De este lema, aplicado de manera recursiva, se obtiene los corolarios siguientes:
%
\begin{corolario}\label{Cor:MP:MarginalMultinomial}
%
  Sea  \ $X  \,  \sim \,  \M(n,p)$, entonces  \  $\displaystyle X_i  \, \sim  \,
  \B(n,p_i)$.
\end{corolario}


Al  final,  por  una  an\'alisis  combinatorial,  se  muestra  sencillamente  un
resultado similar al de la binomial como suma de Bernoulli independientes:
%
\begin{lema}\label{Lem:MultinomialSumaMultiBernoulli}
%
  Sean \ $U_i, \quad i  = 1, \ldots , n$ \ discretas sobre $\U  = \{ 1 \; \ldots
  \;  k   \}$  de  masa  de   probabilidad  $p_{U_i}  =   p  \in  \Delta_{k-1}$,
  independientes, y $X_i  = \un_{U_i}$ vectores aleatorios $k$-dimensionales
  (son, por construcci\'on, independientes). Entonces
  %
  \[
  \sum_{i=1}^n X_i \, \sim \, \M(n,p)
  \]
\end{lema}

\

Nota: esta ley  se generaliza de la misma manera que  para la binomial negativa,
dando una  ley multinomial negativa  o, de manera equivalente,  generalizando la
binomial  negativa  a  m\'as  de  dos  clases  se  obtiene  la  ley  multinomial
negativa. \SZ{Anadirlo en una seccion?}


% --------------------------------- Geometrica
\subsubseccion{Ley Geom\'etrica}
\label{Sssec:MP:Geometrica}

Se  denota  \  $X \,  \sim  \,  \G(p)$  \  con \  $p  \in  (0  \;  1]$ \  y  sus
caracter\'isticas son las siguientes:

\begin{caracteristicas}
%
Dominio de definici\'on & $\X = \Nset^*$\\[2mm]
\hline
%
Parametro & $p \in (0 \; 1]$\\[2mm]
\hline
%
Distribuci\'on  de  probabilidad &  $\displaystyle  p_X(x)  =  (1-p)^{x-1} p$  \
(convenci\'on $0^0 = 1$)\\[2mm]
\hline
%
Promedio & $m_X = \frac1p$\\[2mm]
\hline
%
Varianza & $\displaystyle \sigma_X^2 = \frac{1-p}{p^2}$\\[2mm]
\hline
%
\modif{Sesgo} & $\displaystyle \gamma_X = \frac{2-p}{\sqrt{1-p}}$\\[2mm]
\hline
%
Curtosis por exceso & $\displaystyle \widebar{\kappa}_X = \frac{6 - 6 \, p + p^2}{1-p}$\\[2mm]
\hline
%
Generadora de  probabilidad & $\displaystyle  G_X(z) = \frac{p z}{1-(1-p)  z}$ \
para \ $|z| < \frac1{1-p}$\\[2mm]
\hline
%
Generadora de  momentos & $\displaystyle M_X(u)  = \frac{p \, e^u}{1  - (1-p) \,
e^u}$ \ para \ $\real{u} < - \ln(1-p)$\\[2mm]
\hline
%
Funci\'on caracter\'istica  & $\displaystyle \Phi_X(\omega)  = \frac{p \, e^{\imath
\omega}}{1 - (1-p) \, e^{\imath \omega}}$
\end{caracteristicas}

% Momentos & $ \Esp\left[ X^k \right] = ?$\\[2mm]
% Momento factorial & $\Esp\left[ (X)_k \right] = \frac{p^{k-1} k!}{(1-p)^k}$\\[2mm]
% Modo 1
% Mediana $\left\lceil \frac{-1}{\log_2(1-p)} \right\rceil$ 
% CDF	$1-(1-p)^k$

Su masa  de probabilidad  y funci\'on de  repartici\'on son representadas  en la
figura Fig.~\ref{Fig:MP:Geometrica}.
%
\begin{figure}[h!]
\begin{center} \begin{tikzpicture}%[scale=.9]
\shorthandoff{>}
%
\pgfmathsetmacro{\sx}{.75};% x-scaling
\pgfmathsetmacro{\r}{.05};% radius arc non continuity F_X
\pgfmathsetmacro{\p}{1/3};% probabilidad p
\pgfmathsetmacro{\n}{7};% k mas grande del plot (k in Nset^*)
%
% masa
\begin{scope}
%
\pgfmathsetmacro{\sy}{2.5/\p};% y-scaling 
\draw[>=stealth,->] (-.25,0)--({\sx*\n+.75},0) node[right]{\small $x$};
\draw[>=stealth,->] (0,-.1)--(0,{\sy*\p+.25}) node[above]{\small $p_X$};
%
\pgfmathsetmacro{\pr}{\p};% probabilidad
%
\foreach \k in {1,...,\n} {
\draw ({\k*\sx},0)--({\k*\sx},-.1) node[below,scale=.7]{\k};
\draw[dotted] ({\k*\sx},0)--({\k*\sx},{\sy*\pr}) node[scale=.7]{$\bullet$};
%
\pgfmathsetmacro{\prl}{\pr*(1-\p)};\global\let\pr\prl;% proba actualizado
}
\draw ({(\n+.5)*\sx},-.2) node[below,scale=.7]{$\ldots$};
\draw ({(\n+.5)*\sx},{(\pr/(1-\p)/2*\sy}) node[scale=.7]{$\cdots$};
\draw (0,{\p*\sy})--(-.1,{\p*\sy}) node[left,scale=.7]{$p$};
\draw (0,{\p*(1-\p)*\sy})--(-.1,{\p*(1-\p)*\sy}) node[left,scale=.7]{$p \, (1-p)$};
\draw (-.5,{\p*(1-\p)/2*\sy}) node[left,scale=.7]{$\vdots$};
%
\end{scope}
%
%
% reparticion
\begin{scope}[xshift=8.5cm]
%
\pgfmathsetmacro{\sy}{2.5};% y-scaling 
%
\draw[>=stealth,->] (-.6,0)--({\sx*\n+.75},0) node[right]{\small $x$};
\draw[>=stealth,->] (0,-.1)--(0,{\sy+.25}) node[above]{\small $F_X$};
%
\pgfmathsetmacro{\pr}{\p};% probabilidad
\pgfmathsetmacro{\c}{\p};% cumulativa
%
% cumulativa x < 1
\draw (1,0)--(1,-.1) node[below,scale=.7]{0};
\draw[thick] (-.5,0)--(\sx,0);
\draw ({\sx+\r},\r) arc (90:270:\r);
%
% cumulativa x de 1 a n
\foreach \k in {2,...,\n} {
\draw ({\k*\sx},0)--({\k*\sx},-.1) node[below,scale=.7]{\k};
\draw[thick]({(\k-1)*\sx},{\sy*\c}) node[scale=.7]{$\bullet$}--({\k*\sx},{\sy*\c});
\draw ({\k*\sx+\r},{\sy*\c+\r}) arc (90:270:\r);
\draw[dotted] ({(\k-1)*\sx},{(\c-\pr)*\sy})--({(\k-1)*\sx},{\c*\sy});
%
\pgfmathsetmacro{\prl}{\pr*(1-\p)};\global\let\pr\prl;% proba actualizado
\pgfmathsetmacro{\cl}{\c+\pr};\global\let\c\cl;% cumulativa actualizada
}
%
% cumulativa x > n
\draw ({(\n+.5)*\sx},-.2) node[below,scale=.7]{$\ldots$};
\draw ({(\n+.5)*\sx},{((\c+1)/2*\sy}) node[scale=.7]{$\cdots$};
\draw (0,{\p*\sy})--(-.1,{\p*\sy}) node[left,scale=.7]{$p$};
\draw (0,{\p*(2-\p)*\sy})--(-.1,{\p*(2-\p)*\sy}) node[left,scale=.7]{$p \, (2-p)$};
\draw (-.3,{(1+\p*(2-\p))/2*\sy}) node[left,scale=.7]{$\vdots$};
\draw (0,\sy)--(-.1,\sy) node[left,scale=.7]{$1$};
\end{scope}
%
\end{tikzpicture} \end{center}
%
\leyenda{Ilustraci\'on de una distribuci\'on de probabilidad Geom\'etrica (a), y
  la funci\'on de repartici\'on asociada (b), con $p = \frac13$.}
\label{Fig:MP:Geometrica}
\end{figure}
\SZ{Otros ilustraciones para otros $p$?}

Esta distribuci\'on  aparece en el conteo  de conteo de une  repetici\'on de una
experiencia de maneja  independiente hasta que occure un  evento de probabilidad
$p$; por ejemplo  el n\'umero de tiro de un dado  equilibriado hasta que occurre
un ``6'' sigue una ley geom\'etrica de parametro $p = \frac16$.

Nota que cuando \  $p =  1$ \ la variable es cierta \  $X = 1$.   


\SZ{?`Que propiedad mas?}

% --------------------------------- Poisson
\subsubseccion{Ley de Poisson}
\label{Sssec:MP:Poisson}

Esta  ley fue  introducida por  Poisson en  1837 como  caso l\'imite  de  la ley
binomial     para     $n$     grande,      con     el     producto     $n     p$
fijo~\cite[Cap.~3]{Poi37},~\cite{Hal90, DavEdw01}.  Se  interes\'o Poisson en su
estudio  al  comportamentio  probabil\'istico   del  conteo  de  experiencia  de
Bernoulli bajo la hipotesis de independencia  (dando lugar a la ley binomial) en
ciencia humana, para una poblaci\'on  importante ($n$ grande), pero con un valor
promedio dado.  De hecho, se conoc\'ia esta ley, tambi\'en como caso l\'imite de
la  binomial,  por  lo  menos  desde  un  trabajo  de  de  Moivre  unas  decadas
antes~\cite{Moi10}.   Apareci\'o  tambi\'en   m\'as  tarde  en  muchos  procesos
f\'isicos, como el conteo de desintegraci\'on atomica por secundo en un material
radioactivo, o, (aproximadamente) a trav\'es del conteo de part\'iculas que caen
en una peque\~na  superficia, cuanto se tiran part\'iculas  uniformamente en una
grande superficia  en trabajos de  W. S. Gosset~\footnote{Fue connocido  bajo en
  nombre ``Student''; ver nota de pie~\ref{Foot:MP:Student}.}~\cite{Stu07}.

Se denota $X \,  \sim \, \P(\lambda)$ \ con \ $\lambda  \in \Rset_{0,+}$ \ llamada
{\em taza}, y sus caracter\'isticas son las siguientes:

\begin{caracteristicas}
%
Dominio de definici\'on & $\X = \Nset$\\[2mm]
\hline
%
Par\'ametro & $\lambda \in \Rset_{0,+}$\\[2mm]
\hline
%
%Distribuci\'on  de  probabilidad
Funci\'on de masa &  $\displaystyle  p_X(x)  =  \frac{\lambda^x
e^{-\lambda}}{x!}$\\[2mm]
\hline
%
Promedio & $ m_X = \lambda$\\[2mm]
\hline
%
Varianza & $\sigma_X^2 = \lambda$\\[2mm]
\hline
%
\modif{Asimetr\'ia} & $\displaystyle \gamma_X = \frac1{\sqrt\lambda}$\\[2mm]
\hline
%
Curtosis por exceso & $\displaystyle \widebar{\kappa}_X = \frac1\lambda$\\[2mm]
\hline
%
Generadora de probabilidad & $\displaystyle G_X(z) = e^{\lambda (z-1)}$ \quad para \
$z \in \Cset$\\[2mm]
\hline
%
Generadora  de momentos  & $\displaystyle  M_X(u) =  e^{\lambda \left(  e^u  - 1
\right)}$ \quad para \ $u \in \Cset$\\[2mm]
\hline
%
Funci\'on  caracter\'istica  &  $\displaystyle  \Phi_X(\omega) =  e^{\lambda  \,
\left( e^{\imath \omega} - 1 \right)}$
\end{caracteristicas}

% Momentos & $ \Esp\left[ X^k \right] = ?$\\[2mm]
% Momento factorial & $\Esp\left[ (X)_k \right] = \lambda^k$\\[2mm]
% modo \lfloor \lambda \rfloor 
% Mediana \approx \lfloor \lambda +1/3-0.02/\lambda \rfloor 
% CDF {\frac {\Gamma
% (\lfloor k+1\rfloor  ,\lambda )}{\lfloor k\rfloor !}} where  $\Gamma (x,y)$ is
% the upper incomplete gamma function,

Su masa  de probabilidad  y funci\'on de  repartici\'on son representadas  en la
figura Fig.~\ref{Fig:MP:Poisson}.
%
\begin{figure}[h!]
\begin{center} \begin{tikzpicture}%[scale=.9]
\shorthandoff{>}
%
\pgfmathsetmacro{\sx}{.75};% x-scaling
\pgfmathsetmacro{\r}{.05};% radius arc non continuity F_X
\pgfmathsetmacro{\l}{3};% lambda
\pgfmathsetmacro{\n}{7};% k mas grande del plot (k in Nset)
\pgfmathsetmacro{\q}{floor(\l)};% modo
\pgfmathsetmacro{\m}{(\l^\q)*exp(-\l)/factorial(\q)};% maximo
%
% masa
\begin{scope}
%
\pgfmathsetmacro{\sy}{2.75/\m};% y-scaling 
\draw[>=stealth,->] (-.25,0)--({\sx*\n+.75},0) node[right]{\small $x$};
\draw[>=stealth,->] (0,-.1)--(0,{\sy*\m+.25}) node[above]{\small $p_X$};
%
\pgfmathsetmacro{\pr}{exp(-\l)};% probabilidad
%
\foreach \k in {0,...,\n} {
\draw ({\k*\sx},0)--({\k*\sx},-.1) node[below,scale=.7]{\k};
\draw[dotted] ({\k*\sx},0)--({\k*\sx},{\sy*\pr}) node[scale=.7]{$\bullet$};
%
\pgfmathsetmacro{\prl}{\pr*\l/(\k+1)};\global\let\pr\prl;% proba actualizado
}
\draw ({(\n+.5)*\sx},-.2) node[below,scale=.7]{$\ldots$};
\draw ({(\n+.5)*\sx},{(\pr/\l*\n/2*\sy}) node[scale=.7]{$\cdots$};
\draw (0,{exp(-\l)*\sy})--(-.1,{exp(-\l)*\sy}) node[left,scale=.7]{$e^{-\lambda}$};
\draw (0,{\l*exp(-\l)*\sy})--(-.1,{\l*exp(-\l)*\sy}) node[left,scale=.7]{$\lambda e^{-\lambda}$};
\draw (0,{\l*\l*exp(-\l)/2*\sy})--(-.1,{\l*\l*exp(-\l)/2*\sy}) node[left,scale=.7]{$\frac{\lambda^2 e^{-\lambda}}{2}$};
%\draw (-.5,{\l*exp(-\l)/2*\sy}) node[left,scale=.7]{$\vdots$};
%
\end{scope}
%
%
% reparticion
\begin{scope}[xshift=8.5cm]
%
\pgfmathsetmacro{\sy}{2.75};% y-scaling 
%
\draw[>=stealth,->] (-.6,0)--({\sx*\n+.75},0) node[right]{\small $x$};
\draw[>=stealth,->] (0,-.1)--(0,{\sy+.25}) node[above]{\small $F_X$};
%
\pgfmathsetmacro{\pr}{exp(-\l)};% probabilidad
\pgfmathsetmacro{\c}{exp(-\l)};% cumulativa
%
% cumulativa x < 0
\draw (0,0)--(0,-.1) node[below,scale=.7]{0};
\draw[thick] (-.5,0)--(0,0);
\draw (\r,\r) arc (90:270:\r);
%
% cumulativa x de 0 a n
\foreach \k in {1,...,\n} {
\draw ({\k*\sx},0)--({\k*\sx},-.1) node[below,scale=.7]{\k};
\draw[thick]({(\k-1)*\sx},{\sy*\c}) node[scale=.7]{$\bullet$}--({\k*\sx},{\sy*\c});
\draw ({\k*\sx+\r},{\sy*\c+\r}) arc (90:270:\r);
\draw[dotted] ({(\k-1)*\sx},{(\c-\pr)*\sy})--({(\k-1)*\sx},{\c*\sy});
%
\pgfmathsetmacro{\prl}{\pr*\l/\k};\global\let\pr\prl;% proba actualizado
\pgfmathsetmacro{\cl}{\c+\pr};\global\let\c\cl;% cumulativa actualizada
}
%
% cumulativa x > n
\draw ({(\n+.5)*\sx},-.2) node[below,scale=.7]{$\ldots$};
\draw ({(\n+.5)*\sx},{((\c+1)/2*\sy}) node[scale=.7]{$\cdots$};
\draw (0,{exp(-\l)*\sy})--(-.1,{exp(-\l)*\sy}) node[left,scale=.7]{$e^{-\lambda}$};
\draw (0,{(1+\l)*exp(-\l)*\sy})--(-.1,{(1+\l)*exp(-\l)*\sy}) node[left,scale=.7]{$(1+\lambda) e^{-\lambda}$};
\draw (-.3,{(1+(1+\l+\l*\l/2)*exp(-\l))/2*\sy}) node[left,scale=.7]{$\vdots$};
\draw (0,\sy)--(-.1,\sy) node[left,scale=.7]{\small $1$};
\end{scope}
%
\end{tikzpicture} \end{center}
%
\leyenda{Ilustraci\'on de  una distribuci\'on de probabilidad de  Poisson (a), y
  la funci\'on de repartici\'on asociada (b), con $\lambda = 3$.}
\label{Fig:MP:Poisson}
\end{figure}

\SZ{Otras ilustraciones para otros $\lambda$?}

Ad\'emas, se muestra  sencillamente usando la generadora de  probabilidad que
%
\begin{lema}[Stabilidad]
\label{Lem:MP:StabilidadPoisson}
%
  Sean  \  $X_i  \,  \sim  \,  \P(\lambda_i),  \quad  i  =  1,  \ldots  ,  n$  \
  independientes, entonces
  %
  \[
  \sum_{i=1}^n X_i \, \sim \, \P\left( \sum_{i=1}^n \lambda_i \right)
  \]
\end{lema}


Como lo hemos introducido, la ley de Poisson esta v\'inculada a la ley binomial, como caso l\'imite:
%
\begin{lema}[V\'inculo con la ley binomial]
\label{Lem:MP:VinvuloPoissonBinomial}
%
  Sean  \  $X_n  \,  \sim  \,  \B\left( n \, , \, \frac{\lambda}{n} \right)$  \
  con $\lambda > 0$ fijo, entonces
  %
  \[
  X_n \, \limitd{n \to \infty} \, X \, \sim \, \P(\lambda)
  \]
  %
  donde  \ $\limitd{}$ \  significa que  el l\'imite  es en  distribuci\'on (ver
  notaciones).
\end{lema}
\begin{proof}
  Se  sale  de la  forma  de  la distribuci\'on  binomial  y  de  la formula  de
  Stirling~\footnote{De hecho, esta  formula es probablemente debida previamente
    a  A.  De  Moivre~\cite{Moi33, Moi56,  Pea24,  Cam86, Dut91,  Dem33}, y  fue
    mejorada por  Stirling m\'as tarde. Fue  mejorada a\'un m\'as  por el famoso
    matem\'atico                          S.                           Ramanujan
    recientemente~\cite[\S~4.1]{AndBer13}.\label{Foot:MP:Stirling}}:            \
  $\log\Gamma(z) = \left( z - \frac12 \right) \log z - z + \frac12 \log(2 \pi) +
  o(1)$ \ en \ $z \to +\infty$~\cite{Sti30, AbrSte70, GraRyz15}.
\end{proof}

Aparece  que la  ley de  Poisson esta  v\'inculada tambi\'en  a la  ley binomial
negativa, tambi\'en como caso l\'imite:
%
\begin{lema}[V\'inculo con la binomial negativa]
\label{Lem:MP:VinvuloPoissonBinomialNegativa}
%
Sean \ $X_r \, \sim \, \B_-\left( \frac{\lambda}{r+\lambda} \, , \, r \right)$ \
con $\lambda > 0$ fijo, entonces
  %
  \[
  X_r \, \limitd{r \to \infty} \, X \, \sim \, \P(\lambda)
  \]
\end{lema}
\begin{proof}
  Se  sale de nuevo  la forma  de la  distribuci\'on binomial  negativa y  de la
  formula de Stirling para probarlo.
\end{proof}

M\'as all\'a  del contexto discreto, esta  ley esta tambi\'en  v\'inculada a ley
exponencial, por  el processo dicho de  Poisson.  Si eventos  pueden aparecer de
manera aleatoria  en el tiempo tal que,  entre dos eventos, el  tiempo sigue una
ley   exponencial  de   par\'ametro   $\lambda$,  y   que   estos  tiempos   son
independientes, entonces dado  un intervalo $T$ de tiempo,  el n\'umero de estos
eventos sigue una ley de Poisson de  par\'ametro $\lambda T$.  Lo vamos a ver en
el ejemplo de la ley exponencial m\'as adelante.

Al  final, notar  que  cuando  $\lambda =  0$  la variable  es  cierta  $X =  0$
(con la convenci\'on $0^0 \igualc 1$, de $\displaystyle \lim_{x \to 0^+} x^x = 1$).


%%%%%%%%%%%%%%%%%%%%%%%%%%%%%%%%%%%%%%%%%%%%%%%%%%%%%%%%%%%%%%%%%%%%%%%%%%%%%%%%
\aver{ Estad\'istica  de los n\'umeros de ocupaci\'on  de niveles energ\'eticos:
  distribuciones    de    Maxwell--Boltzmann,     de    Fermi--Dirac,    y    de
  Bose--Einstein\newline Leyes de los grandes n\'umeros}
%%%%%%%%%%%%%%%%%%%%%%%%%%%%%%%%%%%%%%%%%%%%%%%%%%%%%%%%%%%%%%%%%%%%%%%%%%%%%%%%


% ================================= Variables continuas

\subseccion{Distribuciones de variable continua}
\label{Ssec:MP:EjemplosDistribucionescontinuas}

\SZ{$\sigma \to 0$ caso cierto}


% --------------------------------- uniforme escalar
\subsubseccion{Distribuci\'on uniforme sobre un intervalo}
\label{Sssec:MP:UniformeContinua}

Esta distribuci\'on es  la m\'as natural que se usa  cuando queremos modelar una
falta de  informaci\'on sobre una variable,  sabiendo que vive en  un espacio de
volumen  finito: sin  a  priori  m\'as, una  tendencia  natural/intuitiva es  de
asignar la  ``misma probabilidad''  a cada punto  del conjunto.   En particular,
aparece as\'i  naturalmente en  la inferencia bayesiana  que consiste  a modelar
como aleatorio un par\'ametro  que se quierre inferir~\footnote{En la inferencia
  bayesiana, nos interesamos al paremetro (posiblemente multivariado) \ $\theta$
  \ subyancente  a una distribuci\'on. Por ejemplo,  sabemos tener observaciones
  sorteados  de  una  distribuci\'on  de  Poisson, pero  con  el  par\'ametro  \
  $\lambda$ \  desconocido y  nos interesamos a  \ $\theta \equiv  \lambda$.  El
  enfoque bayesiano consiste a considerar el par\'ametro \ $\Theta$ \ aleatorio,
  tal que la  distribuci\'on de las observaciones sea  vista como distribuci\'on
  condicional \  $p_{X|\Theta = \theta}(x)$, llamada  distribuci\'on de sampleo.
  Dadas las  observaciones $X = x$,  la meta es de  determinar la distribuci\'on
  dicha a posteriori \  $p_{\Theta|X = x}$ \ a partir de  la cual se puede hacer
  estimaci\'on  de \ $\theta$  dadas las  observaciones, calcular  intervalos de
  confianza, etc. Se interpreta  como distribuci\'on explicando el par\'ametro a
  partir de  las observaciones.  Por eso, el  metodo se apoya sobre  la regla de
  Bayes $p_{\Theta|X=x}(\theta) \propto p_{X|\Theta=\theta}(x) p_\Theta(\theta)$
  \  as\'i que  se necesita  elegir una  distribuci\'on \  $p_\Theta$ \  dicha a
  priori.\label{Foot:MP:BayesPrior} }
%~\footnote{A  partir de una
%  distribuci\'on  parametrizada   por  un  par\'ametro   $\theta$.   El  enfoque
%  bayesiano consiste  a modelizar $\theta$ aleatorio, digamos  $\Theta$, tal que
%  la     distribuci\'on     de     observaciones     se     escribe     entonces
%  $p_{X|\Theta=\theta}(x)$.   Inferir  $\theta$ a  partir  de observaciones  $x$
%  consiste   a   determinar  la   distribuci\'on   dicha   {\it  a   posteriori}
%  $p_{\Theta|X=x}(\theta)$.  Por eso, hace  falta darse una distribuci\'on dicha
%  {\it a priori} $p_\Theta(\theta)$.\label{Foot:SZ:BayesPrior}}
~\cite{Rob07} (la
ley  es dicha  ley  {\em a  priori};  ver tambi\'en~\cite{Bay63}  o~\cite{Lap12,
  Lap14, Lap20}; tal a priori es conocido como a priori de Laplace).

Se denota $X \, \sim \, \U([a \; b])$. Las caracter\'isticas de \ $X$ \ son las
siguientes:

\begin{caracteristicas}
%
Dominio de definici\'on & $\X = [a \; b]$\\[2mm]
\hline
%
Par\'ametros & $(a,b) \in \Rset, \: b > a$\\[2mm]
\hline
%
Densidad de probabilidad & $p_X(x) = \frac{1}{b-a}$\\[2mm]
\hline
%
Promedio & $\displaystyle m_X = \frac{a+b}{2}$\\[2mm]
\hline
%
Varianza & $\displaystyle \sigma_X^2 = \frac{(b-a)^2}{12}$\\[2mm]
\hline
%
\modif{Asimetr\'ia} & $\gamma_X = 0$ \quad para \ $b \ne a$\\[2mm]
\hline
%
Curtosis por exceso & $\displaystyle \widebar{\kappa}_X = -\frac65$ \quad para \ $b \ne a$\\[2mm]
\hline
%
Generadora de momentos & $\displaystyle M_X(u) = \frac{ e^{b u} - e^{a u}}{u}$ \quad
para~\footnote{En el caso l\'imite \ $u \to  0$, \ $\lim_{u \to 0} \frac{ e^{b u}
- e^{a u}}{u} = b-a$, y similarmente para la funci\'on caracter\'istica}  \ $u \in \Cset$\\[2mm]
\hline
%
Funci\'on caracter\'istica & $\displaystyle  \Phi_X(\omega) = \frac{ e^{\imath a
\omega} - e^{\imath b \omega}}{\imath \, \omega}$
\end{caracteristicas}

% Momentos & $ \Esp\left[ X^k \right] = p^k$\\[2mm]
% Momento factorial & $\Esp\left[ (X)_k \right] = ?$\\[2mm]
% Generadora de probabilidad & $G_X(z) = e^{\lambda (z-1)}$ \ para \ $z \in \Cset$\\[2mm]
% modo 0
% Mediana \ln(2)/\lambda
% CDF 1-e^{-\lambda x}

Obviamente, se puede escribir \ $X \, \egald  \, a + (b-a) U$ \ donde \ $\egald$
\ significa que la equalidad es en distribuci\'on (las variables tienen la misma
distribuci\'on de probabilidad), con \ \ $U \, \sim \, \U \left( [ 0 \; 1 ]
\right)$ \ llamada {\em uniforme estandar}.

La densidad de probabilidad y funci\'on de repartici\'on de la variable estandar
son representadas en la figura Fig.~\ref{Fig:MP:Uniformecontinua}.
%
\begin{figure}[h!]
\begin{center} \begin{tikzpicture}%[scale=.9]
\shorthandoff{>}
%
\pgfmathsetmacro{\sx}{2.5};% x-scaling
\pgfmathsetmacro{\r}{.05};% radius arc non continuity F_X
%
% densidad
\begin{scope}
%
\pgfmathsetmacro{\sy}{2.5};% y-scaling 
\draw[>=stealth,->] ({-\sx/2-.25},0)--({\sx*1.5+.25},0) node[right]{\small $x$};
\draw[>=stealth,->] (0,-.15)--(0,{\sy+.25}) node[above]{\small $p_X$};
%
\draw[thick] ({-\sx/2},0)--(0,0);
\draw (\r,\r) arc (90:270:\r);
\draw[dotted] (0,0)--(0,\sy) node[scale=.4]{$\bullet$};
\draw[thick] plot (0,\sy)--(\sx,\sy) node[scale=.4]{$\bullet$};
\draw[dotted] (\sx,\sy)--(\sx,0);
\draw ({\sx-\r},{-\r}) arc (-90:90:\r);
\draw[thick] (\sx,0)--({1.5*\sx},0);
%
\draw (0,\sy)--(-.1,\sy) node[left,scale=.7]{$1$};
\draw (0,0)--(0,-.1) node[below,scale=.7]{$0$};
\draw (\sx,0)--(\sx,-.1) node[below,scale=.7]{$1$};
%
\node at ({(\sx*1.5+.5)/2},-1) [scale=.9]{(a)};

\end{scope}
%
%
% reparticion
\begin{scope}[xshift=8.5cm]
%
\pgfmathsetmacro{\sy}{2.5};% y-scaling 
%
\draw[>=stealth,->] ({-\sx/2-.25},0)--({\sx*1.5+.25},0) node[right]{\small $x$};
\draw[>=stealth,->] (0,-.1)--(0,{\sy+.25}) node[above]{\small $F_X$};
%
% cumulativa
\draw[thick] ({-\sx/2},0)--(0,0)--(\sx,\sy)--({1.5*\sx},\sy);
%
\draw (0,\sy)--(-.1,\sy) node[left,scale=.7]{$1$};
\draw (0,0)--(0,-.1) node[below,scale=.7]{$0$};
\draw (\sx,0)--(\sx,-.1) node[below,scale=.7]{$1$};
%
\node at ({(\sx*1.5+.5)/2},-1) [scale=.9]{(b)};
\end{scope}
%
\end{tikzpicture} \end{center}
% 
\leyenda{Ilustraci\'on  de  una densidad  de  probabilidad  uniforme  (a), y  la
funci\'on de repartici\'on asociada (b).}
\label{Fig:MP:Uniformecontinua}
\end{figure}

Una nota  importante es que cada ley  continua es v\'inculada a  la ley uniforme
sobre $(0 \; 1)$ de la manera siguiente:
%
\begin{lema}[Inversi\'on]\label{Lem:MP:InversionUniforme}
Sea $X$, continua sobre $\X \subset \Rset$, de funci\'on de repartici\'on $F_X$. Entonces
%
\[
U \equiv F_X(X) \sim \U(0 \; 1)
\]
%
Reciprocamente, definiendo la funci\'on de repartici\'on inversa (o quantile)
%
\[
F_X^{-1}(u) = \inf \{ x \tq F(x) \ge u \}
\]
%
si \ $V \sim \U( 0 \; 1 )$,
%
\[
Y = F_X^{-1}(V) \quad \Rightarrow \quad F_Y(y) = F_X(y)
\]
\end{lema}
%
Cuando  $F_X$ se  inversa  sencillamente eso  da  una manera  sencilla de  tirar
sampleos de funci\'on de repartici\'on $F_X$ a partir de sampleos tirados seg\'un
una ley uniforme.
%
\begin{proof}
Inmediatamente, $F_X$ siendo creciente,
%
\begin{eqnarray*}
P(U \le u) & = &  P( F_X(X) \le u)\\[2mm]
%
& = & P(X \le F_X^{-1}(u))\\[2mm]
%
& = & F_X\left( F_X^{-1} (u) \right)
\end{eqnarray*}
%
Similarmente
%
\begin{eqnarray*}
P(Y \le y) & = &  P( F_X^{-1}(V) \le y)\\[2mm]
%
& = & P(V \le F_X(y))\\[2mm]
%
& = & F_X(y)
\end{eqnarray*}
%
\end{proof}

De manera  general, para  cualquier ensemble $\D  \varsubsetneq \Rset^d$ de  volumen \
finito $|\D|$ \,  la variable uniforma sobre $\D$  tiene la densidad con  respecto a la
medida  ``natural'' sobre  $\D$  (Lebesque, discreta,\ldots)  constante sobre  \
$\D$,
%
\[
p_X(x) = \frac{1}{|\D|} \un_{\D}(x)
\]
%
La media va a ser el centro de gravedad de $\D$.

Vamos a ver en el cap\'itulo~\ref{Cap:SZ:Informacion} que esta distribuci\'on es
la distribuci\'on definida  sobre un conjunto de volumen  finito que maximiza la
entrop\'ia, \ie  que es la  ``menos informativa''. Por  ejemplo, si se  busca un
par\'ametro  modelizado como  aleatorio (enfoque  bayesiano), definido  sobre un
conjunto de volumen finito, sin  a priori m\'as, una tendencia natural/intuitiva
es de  asignar la ``misma probabilidad'' a  cada punto del conjunto.  Es por eso
que aparece as\'i naturalmente en la inferencia bayesiana~\cite{Rob07}.
 
Notar que  cuando $b \to a$,  la variable tiende a  una variable cierta  $X = a$
(ver principio de esta secci\'on).


% --------------------------------- Exponencial
\subsubseccion{Distribuci\'on exponencial}
\label{Sssec:MP:Exponencial}

A pesar de que sea un caso particular de la distribuci\'on Gamma que vamos a ver
m\'as  adelante,  estudiada por  Pearson  desde  el  a\~no 1895~\cite{Pea95},  o
apareci\'o  quizas   un  poco   antes  en  trabajos   de  L.   Boltzmann   o  de
Whitworth~\cite{BalBas95} (como caso lim\'ite  de la ley de Poisson), apareci\'o
esta  ley  de  manera  ``propia''   mucho  m\'as  tarde,  entre  otros  en  1930
en~\cite[Ec.~(46)]{Kon30}.

Se denota $X \,  \sim \, \E(\lambda)$ \ con \ $\lambda  \in \Rset_+^*$ \ llamada
{\em  taza}  (inversa  de  {\em   escala}),  y  sus  caracter\'isticas  son  las
siguientes:

\begin{caracteristicas}
%
Dominio de definici\'on & $\X = \Rset_+$\\[2mm]
\hline
%
Par\'ametro & $\lambda \in \Rset_+^*$\\[2mm]
\hline
%
Densidad  de probabilidad &  $\displaystyle p_X(x)  = \lambda  e^{-\lambda x}$\\[2mm]
\hline
%
Promedio & $\displaystyle m_X = \frac1\lambda$\\[2mm]
\hline
%
Varianza & $\displaystyle \sigma_X^2 = \frac1{\lambda^2}$\\[2mm]
\hline
%
\modif{Asimetr\'ia} & $\gamma_X = 2$\\[2mm]
\hline
%
Curtosis por exceso & $\widebar{\kappa}_X = 6$\\[2mm]
\hline
%
Generadora de  momentos &  $\displaystyle M_X(u) =  \frac{\lambda}{\lambda-u}$ \
para \ $\real{u} < \lambda$\\[2mm]
\hline
%
Funci\'on     caracter\'istica     &     $\displaystyle     \Phi_X(\omega)     =
\frac{\lambda}{\lambda - \imath \omega}$
\end{caracteristicas}

% Momentos & $ \Esp\left[ X^k \right] = p^k$\\[2mm]
% Momento factorial & $\Esp\left[ (X)_k \right] = ?$\\[2mm]
% Generadora de probabilidad & $G_X(z) = e^{\lambda (z-1)}$ \ para \ $z \in \Cset$\\[2mm]
% modo 0
% Mediana \ln(2)/\lambda
% CDF 1-e^{-\lambda x}

Su densidad  de probabilidad  y funci\'on de  repartici\'on son representadas  en la
figura Fig.~\ref{Fig:MP:Exponencial}.
%
\begin{figure}[h!]
\begin{center} \begin{tikzpicture}%[scale=.9]
\shorthandoff{>}
%
\pgfmathsetmacro{\sx}{.75};% x-scaling
\pgfmathsetmacro{\r}{.05};% radius arc non continuity F_X
\pgfmathsetmacro{\l}{1.5};% lambda
\pgfmathsetmacro{\mx}{6};% x maximo del plot
%
% densidad
\begin{scope}
%
\pgfmathsetmacro{\sy}{2.5/\l};% y-scaling 
\draw[>=stealth,->] ({-\sx-.25},0)--({\sx*\mx+.25},0) node[right]{\small $x$};
\draw[>=stealth,->] (0,-.1)--(0,{\sy*\l+.25}) node[above]{\small $p_X$};
%
\draw[thick] ({-\sx},0)--(0,0);
\draw (\r,\r) arc (90:270:\r);
\draw[dotted] (0,0)--(0,{\sy*\l}) node[scale=.4]{$\bullet$};
\draw[thick,domain=0:\mx,samples=100] plot ({\x*\sx},{\sy*\l*exp(-\l*\x)});
%
\draw (0,{\l*\sy})--(-.1,{\l*\sy}) node[left,scale=.7]{$\lambda$};
%
\end{scope}
%
%
% reparticion
\begin{scope}[xshift=8.5cm]
%
\pgfmathsetmacro{\sy}{2.5};% y-scaling 
%
\draw[>=stealth,->] (-.6,0)--({\sx*\mx+.25},0) node[right]{\small $x$};
\draw[>=stealth,->] (0,-.1)--(0,{\sy+.25}) node[above]{\small $F_X$};
%
% cumulativa
\draw[thick,domain=0:\mx,samples=100] (-.5,0)--(0,0) plot({\x*\sx},{(1-exp(-\l*\x))*\sy});
%
\draw (0,\sy)--(-.1,\sy) node[left,scale=.7]{$1$};
\end{scope}
%
\end{tikzpicture} \end{center}
% 
\leyenda{Ilustraci\'on  de una densidad  de probabilidad  exponencial (a),  y la
funci\'on de repartici\'on asociada (b), con $\lambda = 1.5$.}
\label{Fig:MP:Exponencial}
\end{figure}
\SZ{Poner escalas; Otros ilustraciones para otros $\lambda$?}

La  ley exponencial  es conocida  como siendo  {\em sin  memoria}, es  decir, si
buscamos \  $X$ \  visto como un  tiempo (ej.  tiempo de desintegraci\'on  de un
atomo radioactivo) tal que
%
\[
\forall \: x_0 \ge 0, \, x \ge 0, \quad P( X > x+x_0 | X > x_0) = P(X > x)
\]
%
\ie  la  probabilidad  que  $X  >  x+x_0$  (extra  tiempo  despu\'es  de  $x_0$)
condicionalmente  a $X >  x_0$ es  exactamente la  de $X  > x+x_0$  (se olvid\'o
$x_0$), tenemos, por la definici\'on de la probabilidad condicional
%
\[
\forall \: x_0 \ge 0, \, x \ge 0, \quad \frac{1-F_X(x+x_0)}{1-F_X(x_0)} = 1-F_X(x)
\]
%
Por diferenciaci\'on con respeto a $x$ eso da, en $x \to 0$,
%
\[
\forall  \: x_0  \ge 0,  \quad  F_X'(x_0) +  \lambda F_X(x_0)  = \lambda  \qquad
\mbox{con} \qquad \lambda = F_X'(0)
\]
%
Teneiendo en cuenta de que $F_X$  es una funci\'on de repartici\'on, aparece que
$F_X(x) = \left( 1 - e^{-\lambda x} \right) \un_{\Rset_+}(x)$, ley exponencial.

Como  lo hemos  evocado  tratando de  la  ley de  Poisson,  esta es  v\'inculada
intimamente a la ley exponencial a trav\'es del processo dicho de poisson:
%
\begin{lema}[V\'inculo con la ley de Poisson]
\label{Lem:MP:VinculoExponencialPoisson}
%
  Sea  $T_0 =  0$ \  y \  $\forall \:  n \in  \Nset^*$ las  variables aleatorias
  positivas \ $T_n$ \ tales que $T_{n+1}  - T_n \ge 0$ \ son independientes y de
  distribuci\'on $\E(\lambda)$. Fijamos \ $T > 0$ \ y sea \ $X$ \ el n\'umero de
  variables  \  $T_n$  \  que  partenecen  a   $(0  \;  T)$,  \ie  $T_X  <  T  <
  T_{X+1}$. Entonces
  %
  \[
  X \sim \P(\lambda T)
  \]
\end{lema}
%
Dicho  de otra manera,  si tenemos  eventos que  aparecen en  tiempos aleatorios
tales  que los  incrementos de  tiempos entre  eventos son  independientes  y de
distribuci\'on  exponencial de  taza $\lambda$,  el  n\'umero de  eventos en  un
intervalo  de tiempo $T$  dado sigue  una ley  de Poisson,  de taza  $\lambda T$
proporcional al  intervalo, y proporcional a  la taza de la  ley exponencial. El
par\'ametro \ $\lambda$ \ representa la taza de evento por unidad de tiempo.
%
\begin{proof}
Por definici\'on,
%
\begin{eqnarray*}
P(X = n) & = & P(X\le n) - P(X \le n-1) \\[2mm]
%
& = & P(T_{n+1} > T) - P(T_n > T)\\[2mm]
%
& = & F_{T_n}(T) - F_{T_{n+1}}(T)
\end{eqnarray*}
%
Ahora, notando que
%
\[
T_n = \sum_{i=0}^{n-1} \left( T_{i+1} - T_i \right)
\]
%
de la  independencia de los  incrementos de tiempo,  y de las propiedades  de la
funci\'on caracter\'istica, tenemos
%
\[
\Phi_{T_n}(\omega) = \frac{\lambda^n}{(1-\imath \, \omega)^n}
\]
%
De  la  f\'ormula  de  inversion del  teorema~\ref{Teo:MP:InversionDensidad}  se
prueba  que~\footnote{Una manera  es  de  hacer una  integraci\'on  en el  plano
  complejo  y usar  los lemas  de Jordan  y teorema  de residuos~\cite{CarKro05}
  o~\cite[Cap.~4]{AblFok03}.  Nota: de hecho se  reconoce en \ $\Phi_{T_n}$ \ la
  funci\'on caracter\'istica de una ley gamma \ $\G(n,\lambda)$, ley que vamos a
  ver en la secci\'on~\ref{Sssec:MP:Gamma}.}
%
\[
p_{T_n}(x) = \frac{\lambda^n x^{n-1} e^{-\lambda x}}{(n-1)!} \un_{\Rset_+}(x)
\]
%
Con integraciones por partes, se obtiene sencillamente
%
\[
F_{T_n}(T) = 1 - \sum_{i=0}^{n-1} \frac{\lambda^i T^i e^{-\lambda T}}{i!}
\]
%
lo que cierra la prueba.
\end{proof}
%
En  f\'isica,  se  modela la  ley  de  tiempo  de desintegraci\'on  como  siendo
exponencial,  y   se  supone  que  los   desintegraciuones  son  independientes,
explicando el modelo de Poisson  para el n\'umero de desintegraci\'on durante un
tiempo dado.

Una otra caracter\'istica de esta ley  es su stabilidad con respecto al operador
no lineal m\'inimo:
%
\begin{lema}[Stabilidad por el m\'in]
\label{Lem:MP:StabilidadExponencialMinimo}
%
  Sean  $X_i \sim \E(\lambda), \: i = 1, \ldots , n$ \  independientes. Entonces,
  \[
  \min_{i=1,\ldots,n} X_i \equiv X \sim \E(n \lambda)
  \]
\end{lema}
%
\begin{proof}
Inmediatamente, para cualquief $x \ge 0$
%
\begin{eqnarray*}
1-F_X(x) & = & P(X > x) \\[2mm]
%
& = & P\left( \bigcap_{i=1}^n \big( X_i > x \big) \right)\\[2mm]
%
& = & \prod_{i=1}^n P(X_i > x)\\[2mm]
%
& = & e^{- n \lambda x}
\end{eqnarray*}
%
La   secunda   linea   viene   de   la  equivalencia   entre   los   eventos   $
\min_{i=1,\ldots,n}  X_i >  x$ y  $\bigcap_{i=1}^n  \big( X_i  > x  \big)$ y  la
tercera de la independencia de los $X_i$.
\end{proof}

% --------------------------------- Gaussiana
\subsubseccion{Distribuci\'on normal o gaussiana multivariada real}
\label{Sssec:MP:Gaussiana}

En el caso escalar,  esta ley parece aparecer por unas de  las primeras veces en
trabajos de de  Moivre como approximaci\'on de la ley  binomial para $n$ grande,
usando la formula de Stirling~\cite{Moi30, Moi33, Moi56, Pea24, PeaMoi26, Dem33,
  Hal84,  Hal90,  JohKot95:v1, DavEdw01,  Hal06}.   Se  puede  ver tambi\'en  el
trabajo  de F.   Galton,  quien construy\'o  un  experimento, la  caja dicha  de
Galton, que  ilustra por una  parte como se  puede obtener la ley  binomial como
suma   de  Bernoulli,   y   la  convergencia   a  la   Gausiana~\cite[Figs.~7-9,
p.~63]{Gal89}  o~\cite[p.~38]{Pea20}.  Aparte  de  Moivre, la  ley gausiana  fue
desarollado mucho por los matem\'aticos  como Gauss en el estudio del movimiento
de   planetas   con   perturbaciones   (predicci\'on  de   la   trayectoria   de
C\'eres)~\cite{Gau09,  Pea24, DavEdw01,  Hal06}, basado  en trabajos  de  A.  M.
Legendre~\cite{Leg05,   DavEdw01,  Hal06},   o  Laplace   en  mismos   tipos  de
problemas~\cite{Lap09, Lap09:Supp, Lap12, Lap14, Lap20, Pea24, DavEdw01, Hal06}.
De hecho,  apoyandose en  trabajos de  de Moivre, la  formaliz\'o antes  y m\'as
claramente  Laplace,   quien  revandic\'o   entonces  su  partenidad   (ver  por
ejemplo~\cite{Pea20}).   Por eso,  esta ley  es tambi\'en  conocida como  ley de
Laplace-Gauss.

En el contexto multivariado, la extensi\'on natural de la ley binomial siendo la
ley multinomial, es sin sorpresa  que se introdujo la gausiana multivaluada como
approximaci\'on de la multinomial.  Este trabajo  es debido entre otros a J.  L.
Lagrange en  los a\~nos 1770, con  correcciones debido unas  decadas despu\'es a
A. de  Morgan~\cite{Mor38}. Pero apareci\'o  antes en el caso  bidimensional, en
particular  a  trav\'es  del  estudio  del coeficiente  de  correlaci\'on  entre
variables   aleatorias  (ver   por  ejemplo   trabajos   de  Galton~\cite{Gal77,
  Gal77:Nature, Pea20}).

A pesar de que parece menos natural en la modelisaci\'on de fenomenos aleatorios
que leyes uniformes, la ley gausiana es seguramente unas de las m\'as importante
en  probabilidad, sino  que  la m\'as  importante  y la  m\'as  expendida en  la
naturaleza.   Eso  viene sin  duda  del teorema  del  l\'imite  central. En  dos
palabras,  cuando  se  suman  un  n\'umero importante  de  variables  aleatorias
(independientes,   de  misma  ley,   admitiendo  una   varianza,  o   con  menos
restricciones~\cite[Cap.~11]{AthLah06}),  correctamente  normalizado, esta  suma
tiende a una gausiana~\footnote{De hecho,  la approximaci\'on de la ley binomial
  por una  gausiana cuando  $n$ es  grande es una  caso particular  del teorema,
  siendo la binomial una suma  de Bernoulli independientes.}.  En la naturaleza,
se puede ver el ruido (se\~nales) como suma de un n\'umero importante de fuentes
de  ruido independientes,  justificando el  modelo  gausiano~\cite{Fel71, Cam86,
  AshDol99, JacPro03, AthLah06, Ren07, Bil12}.  Ad\'emas, como lo vamos a ver en
el  cap\'itulo~\ref{Cap:SZ:Informacion}, esta  ley es  la de  incerteza m\'axima
(maximizando la entrop\'ia) teniendo  una dada varianza. Aparece naturalmente en
termod\'inamica    (gaz    perfecto,   con    un    n\'umero    muy   alto    de
particulas)~\cite{Max67, Bol96,  Bol98, Gib02, Jay65}. En  estimaci\'on, bajo la
hipotesis  gausiana,  los  estimadores  de  par\'ametros  minimizando  el  error
cuadr\'atico promedio son  generalmente lineal~\cite{Kay93, Rob07}.  Todas estas
consideraciones  dan a la  ley gausiana  un rol  central en  la teor\'ia  de las
probabilidades.

Se denota \  $X \, \sim \, \N(m,\Sigma)$ \  con \ $m \in \Rset^d$  \ y \ $\Sigma
\in  P_d^+(\Rset)$  \  conjunto  de   las  matrices  de  \  $\M_{d,d}(\Rset)$  \
s\'imetricas definidas positivas. Las  caracter\'isticas de la gaussiana son las
siguientes:

\begin{caracteristicas}
%
Dominio de definici\'on & $\X = \Rset^d$\\[2mm]
\hline
%
Par\'ametros & $m \in \Rset^d, \:\: \Sigma \in P_d^+(\Rset)$\\[2mm]
\hline
%
Densidad de probabilidad & $\displaystyle p_X(x) = \frac{1}{(2
\pi)^{\frac{d}{2}} \left| \Sigma \right|^{\frac12}} \, e^{-\frac12 (x-m)^t
\Sigma^{-1} (x-m)}$\\[2.5mm]
\hline
%
Promedio & $ m_X = m$\\[2mm]
\hline
%
Covarianza & $\Sigma_X = \Sigma$\\[2mm]
\hline
%
\modif{Asimetr\'ia} & $\gamma_X = 0$\\[2mm]
\hline
%
Curtosis por exceso & $\widebar{\kappa}_X = 0$\\[2mm]
\hline
%
Generadora de momentos & $\displaystyle M_X(u) = e^{u^t \Sigma u + u^t m}$ \
para \ $u \in \Cset^d$\\[2mm]
\hline
%
Funci\'on  caracter\'istica   &  $\displaystyle  \Phi_X(\omega)   =  e^{-\frac12
\omega^t \Sigma \omega + \imath \omega^t m}$
\end{caracteristicas}

Nota: trivialmente, se puede escribir $X  \, \egald \, \Sigma^{\frac12} N + m$ \
con \ $N \, \sim \, \N(0,I)$ \  donde \ $N$ \ es dicha {\em Gausiana estandar} o
{\em centrada-normalizada}. Las caracter\'isticas de  \ $X$ \ son v\'inculadas a
las  de  \  $N$ \  (y  vice-versa)  por  transformaci\'on afine  (ver  secciones
anteriores).


La densidad de probabilidad gausiana y  la funci\'on de repartici\'on en el caso
escalar son  representadas en la figura Fig.~\ref{Fig:MP:Gaussiana}-(a)  y (b) y
una      densidad      en       un      contexto      bi-dimensional      figura
Fig.~\ref{Fig:MP:Gaussiana}(c).
%
\begin{figure}[h!]
\begin{center} \begin{tikzpicture}%[scale=.9]
\shorthandoff{>}
%
\pgfmathsetmacro{\sx}{.75};% x-scaling
\pgfmathsetmacro{\mx}{3.5};% x maximo del plot
%
% Approximation de la cdf gaussienne
\tikzset{declare function={
normcdf(\x)=1/(1 + exp(-0.07056*(\x)^3 - 1.5976*(\x)));
}}
% densidad
\begin{scope}
%
\pgfmathsetmacro{\sy}{2.5*sqrt(2*pi)};% y-scaling 
\draw[>=stealth,->] ({-\sx*\mx-.25},0)--({\sx*\mx+.25},0) node[right]{\small $x$};
\draw[>=stealth,->] (0,-.1)--(0,2.75) node[above]{\small $p_X$};
%
\draw[thick,domain=-\mx:\mx,samples=100] plot ({\x*\sx},{\sy*exp(-.5*\x*\x)/sqrt(2*pi)});
%
\draw (0,{\sy/sqrt(2*pi)})--(-.1,{\sy/sqrt(2*pi)}) node[left,scale=.7]{$\frac1{\sqrt{2 \pi}}$};
%
\end{scope}
%
%
% reparticion
\begin{scope}[xshift=8.5cm]
%
\pgfmathsetmacro{\sy}{2.5};% y-scaling 
%
\draw[>=stealth,->] ({-\sx*\mx-.25},0)--({\sx*\mx+.25},0) node[right]{\small $x$};
\draw[>=stealth,->] (0,-.1)--(0,{\sy+.25}) node[above]{\small $F_X$};
%
% cumulativa
\draw[thick,domain=-\mx:\mx,samples=100] plot({\x*\sx},{\sy*normcdf(\x)});
%
\draw (0,\sy)--(-.1,\sy) node[left,scale=.7]{$1$};
\end{scope}
%
\end{tikzpicture} \end{center}
% 
\leyenda{Ilustraci\'on  de  una   densidad  de  probabilidad  gaussiana  escalar
  estandar  (a), y la  funci\'on de  repartici\'on asociada  (b), as\'i  que una
  densidad  de probabilidad  gaussiana bi-dimensional  centrada y  de  matriz de
  covarianza \ $\Sigma_X = R(\theta)  \Delta^2 R(\theta)^t$ \ con \ $R(\theta) =
  \protect\begin{bmatrix}   \cos\theta  &   -  \sin\theta\\[2mm]   \sin\theta  &
    \cos\theta  \protect\end{bmatrix}$ \  matriz  de rotaci\'on  y  \ $\Delta  =
  \diag\left(\protect\begin{bmatrix}  1   &  a\protect\end{bmatrix}  \right)$  \
  matriz  de   cambio  de  escala,   y  sus  marginales   \  $X_1  \,   \sim  \,
  \N\left(0,\cos^2\theta  + a^2  \sin^2\theta \right)$  \ y  \ $X_2  \,  \sim \,
  \N\left(0,\sin^2\theta + a^2 \cos^2\theta  \right)$ \ (ver m\'as adelante). En
  la figura, $a = \frac14$ \ y \ $\theta = \frac{\pi}{6}$.}
\label{Fig:MP:Gaussiana}
\end{figure}

La gaussiana tiene un par de propiedades particulares:
%
\begin{lema}[Gausiana y cumulantes]
%
  Sea \  $X$ \ vector aleatorio de  media $m$, covarianza $\Sigma$  y de secunda
  funci\'on caracter\'istica admtiendo un desarollo de Taylor. Entonces
  %
  \[
  \kappa_k[X] =  0 \quad \forall \: k  \ge 4 \quad \Longleftrightarrow  \quad X \sim
  \N(m,\Sigma)
  \]
\end{lema}
%
\begin{proof}
  La pueba es inmediata del lema~\ref{Lem:MP:CumSecFctCarac},
  %
  \[
  \kappa_k[X]  =  0  \quad  \forall  \:  k \ge  4  \quad  \Longleftrightarrow  \quad
  \Psi_X(\omega) = \imath \, \omega^t m - \frac12 \omega^t \Sigma \omega
  \]
  %
  lo que es nada m\'as que la secunda funci\'on caracter\'istica de la gausiana,
  esa determiniendo completamente la ley.
\end{proof}
%
\begin{teorema}[Stabilidad]
\label{Teo:MP:StabilidadGaussiana}
%
  Sean \ $A_i , i = 1,\ldots,n$  \ matrices de \ $\M_{d',d}(\Rset), \: d' \le d$
  \ de rango lleno, $b_i \in \Rset^{d'}$ \ y \ $X_i \, \sim \, \N(m_i,\Sigma_i)$
  \ independientes, entonces
  %
  \[
  \sum_{i=1}^n \left(  A_i X_i  + b_i \right)  \, \sim \,  \N\left( \sum_{i=1}^n
    \left( m_i + b_i \right) \, , \, \sum_{i=1}^n A_i \Sigma_i A_i^t \right)
  \]
  % 
  En particular, cualquier combinaci\'on lineal  de los componentes de un vector
  gaussiano da una gaussiana.  Reciprocamente, si cualquier combinaci\'on lineal
  de los componentes de un vector aleatorio sigue una ley gaussiana, entonces el
  vector es gaussiano.
\end{teorema}
%
\begin{proof}
  Este  resultato se proba  usando funci\'on  caracter\'istica de  la gaussiana,
  conjuntamente al teorema~\ref{Teo:MP:PropiedadesFuncionCaracteristica}.
\end{proof}
%
\begin{corolario}[Media empirica]\label{Cor:MP:MediaEmpiricaGauss}
%
  Sean \ $X_i \, \sim \, \N(m,\Sigma), \: i = 1, \ldots , n$ \ independientes. Entonces,
  %
  \[
  \overline{X} =  \frac{1}{n} \sum_{i=1}^n  X_i \,  \sim \, \N\left(  m \,  , \,
    \frac{1}{n} \Sigma \right)
  \]
   %
  $\overline{X}$  es llamada {\em  media empirica}~\footnote{Es  la estimaci\'on
    \'optima de  la media  $m$ a  partir de los  $X_i$ en  el sentido  del error
    cuadratico  promedio   m\'inimo,  o  en   el  sentido  de   la  verosimlitud
    m\'axima~\cite{Kay93, Rob07}.}, y es un estimador ``natural'' de la media de
  un vector aleatorio a partir de copias independientes de misma ley.
  %
\end{corolario}
%
\begin{teorema}[Independencia]
\label{Teo:MP:IndependenciaGaussiana}
%
  Sea   \   $X  \,   \sim   \,   \N(m,\Delta)$  \   con   \   $\Delta  =   \diag
  \left(  \begin{bmatrix}  \sigma_1^2  &  \cdots  &  \sigma_d^2  \end{bmatrix}^t
  \right)$   \  diagonal.   Entonces  los   componentes  \   $X_i  \,   \sim  \,
  \N(m_i,\sigma_i^2)$ \ son independientes.
\end{teorema}
%
\begin{proof}
  Este resultato se proba  trivialmente escribiendo la densidad de probabilidad,
  notando que se factorisa.
\end{proof}
%
Hemos visto que cuando un  vector tiene componentes independientes, la matriz de
covarianza  es   diagonal  (lema~\ref{Lem:MP:IndependenciaCov}),  pero   que  la
rec\'iproca es falsa en general.  El \'ultimo teorema muestra que la rec\'iproca
vale en el caso gausiano.

Volvemos  ahora al rol  central de  la gausiana  como modelo  probabilistico muy
frecuente  de fenomeos  aleatorios. Este  rol particular  viene del  teorema del
l\'imite  central que  ya introdujimos.  A veces,  es conocido  como  teorema de
Lindenberg-Feller (por lo menos la forma con condiciones m\'as debiles que en la
formulaci\'on original).   Para unas de  las formulaciones originales,  se puede
referirse  al trabajo  de Laplace  de  1809 o  de 1912~\cite{Lap09,  Lap09:Supp,
  Lap12,  Lap14, Lap20}.   El nombre  ``central'' viene  de un  documento  de G.
P\'olya   de   1920,  titulado   ``\"Uber   den   zentralen  Grenzwertsatz   der
Wahrscheinlichkeitsrechnung  und das Momentenproblem''  (``Sobre el  teorema del
l\'imite central del  c\'alculo probabil\'istico y el problema  de los momentos;
el  teorema  es  central\ldots~\cite{Pol20,   Cam86}).   Se  enuncia  de  manera
siguiente~\cite{Spi76, BroDav87, LehCas98, AshDol99, JacPro03, AthLah06, Bil12}:
% ~\footnote{Aparte    en~\cite{AshDol99,   JacPro03},   el    teorema   aparece
%   frecuentemente en los libros en el contexto escalar, seguramente por razones
%   historicas.    Pero,  con   el  mismo   enfoque,  se   prueba  en   el  caso
%   multivariado.}:

\begin{teorema}[Teorema del l\'imite central]\label{Teo:MP:CLT}
%
  Sea  \  $\{  X_i \}_{i  \in  \Nset^*}$  \  una  sucesi\'on de  vectores  aleatorios
  independientes, de misma ley,  y que admiten un promedio \ $m$  \ y una matriz
  de covarianza \ $\Sigma$. Entonces
  %
  \[
  \frac{1}{\sqrt{n}}  \sum_{i=1}^n  \left( X_i  -  m  \right)  \: \limitd{n  \to
    +\infty} \: Y \sim \N\left( 0 , \Sigma \right)
  \]
  %
  donde  \ $\limitd{}$ \  significa que  el l\'imite  es en  distribuci\'on (ver
  notaciones).
\end{teorema}
\begin{proof}
  Hay varias pruebas de este resultado.  Quiz\'as la m\'as simple se apoya sobre
  la funci\'on  c\'aracteristica.  Sin perdida de generalidad,  supongamos que \
  $m = 0$. Sea \ $\displaystyle Y_n = \frac{1}{\sqrt{n}} \sum_{i=1}^n X_i$.  Sea
  \    $\omega$    \    fijo.     Por    independencia    y    relaciones    del
  teorema~\ref{Teo:MP:PropiedadesFuncionCaracteristica}~\footnote{$o\left(
      n^{-1} \right)$ significa que  el termino que queda, digamos $\varepsilon$
    es tal que $n \varepsilon$ tiende a cero cuando $n$ tiende al infinito.}:
  %
  \begin{eqnarray*}
  \Phi_{Y_n}(\omega) & = & \left( \Phi_{X_i}\left( \frac{\omega}{\sqrt{n}}
  \right) \right)^n\\[2mm]
  %
& = & \left( \Phi_{X_i}(0) + \frac{1}{\sqrt{n}} \, \omega^t \, \nabla_\omega
  \Phi_{X_i}(0) + \frac{1}{2 n} \, \omega^t \, \Hess_\omega\Phi_{X_i}(0) \, \omega +
  o\left( n^{-1} \right) \right)^n\\[2mm]
  %
  & = & \left( 1 - \frac{1}{2 n} \, \omega^t \, \Sigma \, \omega +
  o\left( n^{-1} \right) \right)^n\\[2mm]
  %
  & \xrightarrow[n \to +\infty]{} & \exp\left( -\frac12 \omega^t \Sigma \omega \right)
  \end{eqnarray*}
  %
  porque \ $\Phi_{X_i}(0) = 1$, $X_i$ \  siendo de media nula el gradiente de la
  funci\'on   caracter\'istica   se  cancela   en   \   $\omega   =  0$,   y   \
  $\Hess_\omega\Phi_{X_i}(0)  =  -  \Sigma$.   Se reconoce  ahora  la  funci\'on
  caracter\'istica   de  la   gausiana,   lo  que   prueba   que  la   funci\'on
  caracter\'istica  de  \  $Y_n$  \  converge  simplemente  hacia  la  funci\'on
  caracter\'istica  de la gausiana.   Se cierra  la pruba  usando el  teorema de
  convergencia de  L\'evy, diciendo que  la convergencia simple de  la funci\'on
  caracter\'istica  implica  la  convergencia en  distribuci\'on~\cite{AshDol99,
    Bil12, AthLah06}.
\end{proof}
%
En particular, la  media empirica hechas a partir  de vectores independientes de
media  $m$,  admitiendo  una  covarianza  \  $\Sigma$  \  y  de  misma  ley  (no
necesariamente gausiana), tiende a ser gausiana  de media \ $m$ \ y covarianza \
$\frac{1}{n} \Sigma$.

Existen  varias   variantes  de  este   teorema  que  enunciamos,  sin   dar  la
prueba.  Dejamos el  lector a  libros m\'as  especializados como~\cite{AshDol99,
  Bil12, AthLah06, Lin22}.
% Lindeberg 1920

\begin{teorema}[Teorema de Lindenberg-Feller]\label{Teo:MP:LindenbergFeller}
%
  Sean  $\{  X_i \}_{i  \in  \Nset^*}$  vectores  aleatorios independientes,  no
  necesariamente de misma distribuci\'on de probabilidad, con \ $X_i$ \ de media
  \ $m_i = \Esp[X_i]$ \ y de matriz de covarianza \ $\Sigma_i \in P_d^+(\Rset)$.
  Sean \ $C_n = \sum_{i=1}^n \Sigma_i$, \ $c_n^2$ \ al autovalor m\'as peque\~na
  de $C_n$,  \ y  \ $Y_n  = C_n^{-\frac12} \sum_{i=1}^n  \left( X_i  - \Esp[X_i]
  \right)$.
  %
  \[
  \mbox{Si} \quad \lambda_n > 0 \quad \mbox{y} \quad \forall \: \varepsilon > 0,
  \quad  \lim_{n  \to  +\infty}  \sum_{i=1}^n  \Esp\left[  \left\|  \frac{X_i  -
        m_i}{c_n} \right\|^2  \un_{\left[ \varepsilon \;  +\infty \right)}\left(
      \left\| \frac{X_i - m_i}{c_n} \right\| \right) \right] = 0
  \]
  %
  entonces
  %
  \[
  Y_n \: \limitd{n \to +\infty} \: Y \sim \N\left( 0 , I \right)
  \]
\end{teorema}
%
En  numerosos libros,  este teorema  es  dado en  el caso  escalar. Se  extiende
sencillamente al caso multivariado gracia a lo que es conocido como {\it teorema
  de Cram\'er-Wold},  diciendo que una  secuencia de vectores aleatorios  \ $Y_n
\limitd{}  Y$ \  si y  solamente para  cualquier \  $u \in  \Rset^d$ \  $u^t Y_n
\limitd{} u^t Y$~\cite{AshDol99, AthLah06, Bil12}.

Sin dar  la prueba,  la condici\'on  de Lindenberg dice  que si  la suma  de las
``dispersi\'ones'' de  los vectores normalizados  por los que es  basicamente la
varianza  la   m\'as  peque\~na  de  los   componentes  de  la   suma  (una  vez
diagonalizada)  se  concentra  asintoticamente,  la  suma  renormalizada  de  lo
vectores centrados tiende a la gausiana (en distribuci\'on).

Se  puede ver  que  se  satisface la  condici\'on  de Lindeberg  en  el caso  de
variables independientes de misma ley, del hecho  que \ $C_n = n \Sigma$, lo que
da \ $c_n^2 = n c^2$ \ con  \ $c^2$ \ autovalor m\'as peque\~na de \ $\Sigma$. A
continuaci\'on da la condici\'on \ $\displaystyle \lim_{n \to \infty} \Esp\left[
  \left\| X_i - m_i \right\|^2 \un_{\left[ \varepsilon \; +\infty \right)}\left(
    \left\|  \frac{X_i  -  m_i}{\sqrt{n}  c}  \right\|  \right)  \right]  =  0$,
satisfecha  porque el  argumento de  la funci\'on  indicadora tiende  a  0 (casi
siempre).

Un otro caso  ``trivial'' aparece cuando la secuencia  es uniformamente acotada,
\ie \  $\forall \: i, \quad  \| X_i \| \le  M$. Se puede  retomar los argumentos
anteriores, remplazando las variables por la cota.

Nota: si  se satisface  la condici\'on  dicha {\it de  Lyapunov}, \ie  si existe
$\delta > 0$ tal que
%
\[
\lim_{n   \to  \infty}   \sum_{i=1}^n   \Esp\left[  \frac{\left\|   X_i  -   m_i
    \right\|^2}{c_n^{2+\delta}} \right] = 0,
\]
%
entonces         se          satisface         la         condici\'on         de
Lindeberg~\cite{AshDol99}.  Frecuentemente,  es   m\'as  sencillo  verificar  la
condici\'on m\'as fuerte de Lyapunov para  probar la convergencia de una suma de
vectores aleatorios a la gausiana.

\

Aparece que se puede a\'un  debilitar la condici\'on de independencia sin perder
la   convergencia    a   la   gausiana.    Para   m\'as   detalles,    ver   por
ejemplo~\cite[Sec.~6.4]{BroDav87}.


\SZ{
% --------------------------------- Gaussiana complejas
\subsubseccion{Distribuci\'on normal o gaussiana multivariada complejas}
\label{Sssec:MP:GaussianaComplejas}

Por definici\'on, un vector aleatorio complejo $d$-dimensional \ $Z = X + \imath
Y$ \  es gaussiano significa que  el vector $2 d$-dimensional  \ $\widetilde{Z} =
\begin{bmatrix}  X^t &  Y^t \end{bmatrix}^t$  \ es  gaussiano. Se  puede entonces
referirse  en el  caso de  vectores gaussianos,  pero como  lo hemos presentado  en la
secci\'on~\ref{Ssec:MP:VAComplejos}, es frecuentemente m\'as comodo trabajar con
\  $Z$ \  en lugar  de \  $\widetilde{Z}$.
%  En  particular, en  el marco  de las
%comunicaciones en  ingeneria, se  trabaja con modulaciones  dichas en fase  y en
%cuadratura (se\~nal multiplicado  respectivamente por un seno y  un coseno) y en
%lugar  de trabajar  con dos  componente se  considera una  modulaci\'on  con una
%exponencial  compleja y  la  se\~nal/variable compleja.   Se  puede por  ejemplo
%referirse  a~\cite{Lap17} (ver en  particular el  capitulo~24) o~\cite{SchSch03,
%  EriKoi06}.

En el caso general, la gaussiana  real siendo completamente descrita por su media
y su matriz de covarianza, la  gaussiana compleja va a ser completamente definida
por   la  media,   la  matriz   de  covarianza   y  la   pseudo-covarianza  (ver
Sec.~\ref{Sec:MP:VectoresComplejosMatricesAleatorias}  por las  relaciones entre
la  covarianza  de  \ $\widetilde{Z}$  \  y  estas  matrices).

Se denota \ $Z \, \sim \, \CN(m,\Sigma,\check{\Sigma})$ \ con \ $m \in \Cset^d$,
\ $\Sigma \in \Pos_d^+(\Cset)$ \ conjunto  de las matrices de \ $\Mat_{d,d}(\Cset)$ \
herm\'iticas definidas positivas, y  \ $\check{\Sigma} \in \Sim_d(\Cset)$ \ conjunto
de las matrices de \  $\Mat_{d,d}(\Cset)$ \ symmetricas (ver notaciones).  Un caso
particular  aparece cuando  \ $Z$  \  es propio  en torno  a  \ $m$,  lo que  es
equivalente en el caso gaussiano a tener \ $Z$ \ circular (ver m\'as adelante) en
torno a  \ $m$, dado  cuando \  $\check{\Sigma} = 0$:  en este caso  usaremos la
misma notaci\'on,  $Z \,  \sim \, \CN(m,\Sigma)$.   Las caracter\'isticas  de la
gaussiana  compleja   son  las  siguientes~\cite{Lap17,   Pic96,  Goo63,  Bos95,
  SchSch03, EriKoi06}:

\begin{caracteristicas}
%
Dominio de definici\'on & $\Z = \Cset^d$\\[2mm]
\hline
%
Par\'ametros & $m \in \Cset^d, \:\: \Sigma \in \Pos_d^+(\Cset), \:\: \check{\Sigma}
\in \Sim_d(\Cset)$\\[2mm]
\hline
%
Densidad de probabilidad & \\[1mm]
%
Caso general: & $\displaystyle p_Z(z) = \frac{1}{\pi^d \left| \Sigma
 \right|^{\frac12} \left| P \right|^{\frac12}} \: e^{- (z-m)^\dag P^{-1} (z-m) +
 \real{(z-m)^t R^t P^{-1} (z-m)}}$\vspace{2.5mm}\newline
 con~\footnote{En~\cite{Pic96} la expresi\'on es ligieramente diferente, pero se
 recupera usando la simetr\'ia \ $\check{\Sigma}^* =
 \check{\Sigma}^\dag$. Recordar que \ $\cdot^{-*} = \left( \cdot^* \right)^{-1}$
 \ (ver notaciones).} \ $P = \Sigma - \check{\Sigma} \, \Sigma^{-*}
 \, \check{\Sigma}^\dag, \quad R = \check{\Sigma}^\dag \, \Sigma^{-1}$.\\[2.5mm]
%
Caso circular: & $\displaystyle p_Z(z) = \frac{1}{\pi^d \left| \Sigma \right|}
 \: e^{- (z-m)^\dag \Sigma^{-1} (z-m)}$\\[2.5mm]
\hline
%
Promedio & $ m_Z = m$\\[2mm]
\hline
%
Covarianza & $\Sigma_Z = \Sigma$\\[2mm]
\hline
%
Pseudo-covarianza & $\check{\Sigma}_Z = \check{\Sigma}$\\[2mm]
\hline
%%
%Generadora de  momentos &  $\displaystyle M_X(u) =  e^{u^t \Sigma u + u^t m}$  \ para \  $u \in
%\Cset^d$\\[2mm]
%\hline
%%
Funci\'on caracter\'istica & \\[1mm]
%
Caso general: & $\displaystyle \Phi_Z(\omega) = e^{-\frac14
\omega^\dag \Sigma \omega - \frac14 \real{\omega^\dag \check{\Sigma} \omega^*} +
\imath \real{\omega^\dag m}}, \quad \omega \in \Cset^d$\\[1mm]
%
Caso circular: & $\displaystyle \Phi_Z(\omega) = e^{-\frac14
\omega^\dag \Sigma \omega +
\imath \real{\omega^\dag m}}, \quad \omega \in \Cset^d$
\end{caracteristicas}

Notar que  en el caso  escalar propio (circular),  la varianza de  \ $Z$ \  es \
$\sigma_Z^2  = 2  \sigma^2$.  El coefficiente  2 viene  del  hecho que  \ $Z$  \
contiene dos componentes independientes de varianza $\sigma^2$.

Los vectores aleatorios complejos van a compartir las propiedades del caso real,
siendo equivalente  a un vector  $2d$-dimensional gaussiano real.

% De  manera  general,  las caracter\'isticas  de  \  $X$  \ gaussiano  real  son
% v\'inculadas a las  de \ $N$ \ (y vice-versa)  por transformaci\'on afine (ver
% secciones anteriores).

Primero, los cumulantes de orden superior o igual a $4$ valen cero:
%
\begin{lema}[Gaussiana compleja y cumulantes]
%
  Sea \ $Z$ \ vector aleatorio complejo de media $m$, de covarianza $\Sigma$, de
  pseudo-covarianze  $\check{\Sigma}$ y  de  secunda funci\'on  caracter\'istica
  admtiendo un desarollo de Taylor. Entonces para cualquier
  %
  \[
  \kappa_{i_1,\ldots,i_l,i'_1,\ldots,i'_m}[Z] = 0 \quad \forall \: ( i_1, \ldots
  , i_l  , i'_1  , i'_m  \in \{  1 , \ldots  , d  \}^{l+m}, \,  l+m \ge  4 \quad
  \Longleftrightarrow \quad X \sim \CN(m,\Sigma,\check{\sigma})
  \]
  %
\end{lema}
%
%\begin{proof}
%  La pueba sigue paso a paso la del lema~\ref{Lem:MP:CumSecFctCarac}.
%\end{proof}

Secundo, como en  el caso real, la gaussiana es  estable por combinaci\'on lineal
de vectores independientes:
%
\begin{teorema}[Stabilidad]
\label{Teo:MP:StabilidadGaussianaCompleja}
%
  Sean \ $A_i , i = 1,\ldots,n$  \ matrices de \ $\Mat_{d',d}(\Cset), \: d' \le d$
  \  de   rango  lleno,   $b_i  \in  \Cset^{d'}$   \  y   \  $Z_i  \,   \sim  \,
  \CN(m_i,\Sigma_i,\check{\Sigma}_i)$   \   $d$-dimensionales,  independientes,
  entonces
  %
  \[
  \sum_{i=1}^n \left( A_i  Z_i + b_i \right) \,  \sim \, \CN\left( \sum_{i=1}^n
    \left( m_i + b_i \right) \, ,  \, \sum_{i=1}^n A_i \Sigma_i A_i^\dag \, , \,
    \sum_{i=1}^n A_i \check{\Sigma}_i A_i^t \right)
  \]
  %
  En particular, cualquier combinaci\'on lineal  de los componentes de un vector
  gaussiano complejo  da una  gaussiana compleja.  Reciprocamente,  si cualquier
  combinaci\'on lineal de  los componentes de un vector  aleatorio sigue una ley
  gaussiana compleja, entonces el vector es gaussiano complejo.
\end{teorema}
%
El corolario~\ref{Cor:MP:MediaEmpiricaGauss} se extiende naturalmente al caso complejo:
%
\begin{corolario}[Media emp\'irica]\label{Cor:MP:MediaEmpiricaGaussCompleja}
%
  Sean \ $Z_i \, \sim \, \CN\left(  m , \Sigma , \check{\Sigma} \right), \: i =
  1, \ldots , n$ \ independientes. Entonces,
  %
  \[
  \overline{Z} =  \frac{1}{n} \sum_{i=1}^n Z_i \,  \sim \, \CN\left( m  \, , \,
    \frac{1}{n} \Sigma \, , \, \frac{1}{n} \check{\Sigma} \right)
  \]
  %
\end{corolario}

Adem\'as, en el  caso complejo se tiene  una estabilidad combinando \ $Z$  \ y \
$Z^*$:
%
\begin{teorema}%[Stabilidad]
\label{Teo:MP:StabilidadGaussianaComplejaZZestrella}
%
Sean \ $A \in \Mat_{d',d}(\Cset)$, \  $B \in \Mat_{d',d}(\Cset)$ \ tales que \ ambas
\ $A+B$ \ y \  $A-B$ \ \SZ{sean de rango lleno}, $c \in \Cset^{d'}$  \ y \ $Z \,
\sim \, \CN(m,\Sigma,\check{\Sigma})$ \ $d$-dimensonal, entonces
  %
  \[
  A Z + B Z^* + c \: \sim \: \CN\left( \mu , C , \check{C}  \right)
  \]
  %
  con
  %
  \[
  \begin{array}{lll}
  \mu & = & A m + B m^* + c\\[2.5mm]
  %
  C & = & A \Sigma A^\dag + B \Sigma^* B^\dag + A \check{\Sigma} B^\dag + B
  \check{\Sigma}^* A^\dag\\[2.5mm]
  %
  \check{C} & = & A \check{\Sigma} A^t + B \check{\Sigma} B^t + A \Sigma B^t + B
  \Sigma^* A^t
  \end{array}
  \]
\end{teorema}
\begin{proof}
  Tomando la forma real $2d$-dimensional \ $Z = X + \imath Y$ \ con $X, Y$
  reales,   es  en   biyecci\'on  con   $\widetilde{Z}  =   \begin{bmatrix}  X\\
    Y   \end{bmatrix}$  \  y   entonces  \   $Z^*$  \   en  biyecci\'on   con  \
  $\widetilde{Z^*} = \begin{bmatrix} X\\-Y \end{bmatrix}$. Eso da \ $A Z + B Z^*
  +   c$   \   en   biyecci\'on   con   $\begin{bmatrix}   A+B   &   0\\   0   &
    A-B \end{bmatrix} \begin{bmatrix} X\\ Y \end{bmatrix} + \begin{bmatrix}
    \real{c}\\ \imag{c} \end{bmatrix}$. Notando que \ $\begin{bmatrix} A+B & 0\\
    0    &   A-B    \end{bmatrix}$   \    es    de   rango    lleno,   por    el
  teorema~\ref{Teo:MP:StabilidadGaussiana} este vector es gaussiano, lo que proba
  que \  $A Z  + B  Z^* + c$  \ es  gaussiano complejo. Las  formas de  la media,
  covarianza y  pseudo-covarianza siguen de calculos directos  de la expresi\'on
  $A Z + B Z^* + c$.
\end{proof}
%
Evidentemente, se puede combinar los dos teoremas anteriores.

El teorema del l\'imite central y sus variantes se recuperan del caso real.
%
\begin{teorema}[Teorema del l\'imite central (caso complejo)]
\label{Teo:MP:CLTComplejo}
%
Sea  \ $\{  Z_i \}_{i  \in  \Nset^*}$ \  una sucesi\'on  de vectores  aleatorios
independientes, de  misma ley, y  que admiten un  promedio \ $m$, una  matriz de
covarianza   \    $\Sigma$   \   y    una   matriz   de    pseudo-covarianza   \
$\check{\Sigma}$. Entonces
  %
  \[
  \frac{1}{\sqrt{n}}  \sum_{i=1}^m  \left( Z_i  -  m  \right)  \: \limitd{n  \to
    +\infty} \: Z \sim \CN\left( 0 , \Sigma , \check{\Sigma} \right)
  \]
  %
  donde  \ $\limitd{}$ \  significa que  el l\'imite  es en  distribuci\'on (ver
  notaciones).
\end{teorema}
%
%
Como en el caso real, aparece que  la media emp\'irica hechas a partir de vectores
complejos independientes de media \  $m$, admitiendo una covarianza \ $\Sigma$ \
una  pseudo-covarianza \  $\check{\Sigma}$, y  de misma  ley  (no necesariamente
gaussiana),  tiende a  ser gaussiana  compleja  de media  \ $m$,  de covarianza  \
$\frac{1}{n} \Sigma$, y de pseudo-covarianza \ $\frac{1}{n} \check{\Sigma}$.

No   lo   presentamos,  pero   se   transpone   sencillamente   el  teorema   de
Lindenberg-Feller~\ref{Teo:MP:LindenbergFeller} al caso complejo.

Notamos tambi\'en que, en el caso  circular, se puede escribir naturalmente \ $Z
\, \egald \, \Sigma^{\frac12} N + m$ \  con \ $N \, \sim \, \CN(0,I)$ \ donde \
$N$ \  es dicha  {\em Gaussiana estandar}  o {\em centrada-normalizada}.   Eso se
generaliza en dos direcciones.  La  primera pone tambi\'en en juega una gaussiana
estandar~\cite{Lap17}:
%
\begin{teorema}
\label{Teo:MP:GaussianaComplejaWWestrella}
%
Sea \  $Z \sim  \CN(m,\Sigma,\check{\Sigma})$.  Entonces, existen  matrices (no
\'unicas) \ $A \in \Mat_{d,d}(\Cset)$, \ $B \in \Mat_{d,d}(\Cset)$ \ tales que
  %
  \[
  Z \egald A W + B W^* + m
  \]
  %
  con \ $W \sim \CN(0,I)$ \ gaussiana estandar.
  %
\end{teorema}
\begin{proof}
  Inmediatamente
  %
  \[
  Z   =  \begin{bmatrix}   I   &  \imath   I\end{bmatrix}  \begin{bmatrix}   X\\
    Y\end{bmatrix}
  \egald \begin{bmatrix} I & \imath I\end{bmatrix} M \begin{bmatrix} U\\
    V\end{bmatrix}
  \]
  %
  con \ $U \sim \N(0,I)$ \ y \  $V \sim \N(0,I)$ \ independientes, y \ $M$ \ tal
  que  \ $M  M^t =  \begin{bmatrix} \Sigma_X  & \Sigma_{X,Y}  \\  \Sigma_{X,Y}^t &
    \Sigma_Y  \end{bmatrix}$  \   (ej.  raiz  cuadrade  de  esta   matriz  de  \
  $\Pos_{2d}^+(\Rset)$,    o     descomposici\'on    de    Cholesky~\cite{HorJoh13,
    Bha07}). Ahora, volviendo a la forma compleja tenemos
  %
  \[
  Z     \egald   \begin{bmatrix}    I   &   \imath   I\end{bmatrix}
  M \begin{bmatrix} I  & I\\ -\imath I &  \imath I \end{bmatrix} \begin{bmatrix}
    \frac12 (U + \imath V)\\ \frac12 (U - \imath V)\end{bmatrix}
  \]
  %
  Se cierra la prueba denotando
  %
  \[
  \begin{bmatrix}   A   &   B\end{bmatrix}   =  \begin{bmatrix} I &
    \imath  I\end{bmatrix}  M  \begin{bmatrix}  I  &  I\\  -\imath  I  &  \imath
    I \end{bmatrix}
  \]
  %
  y notando que \ $W \equiv \frac12 (U + \imath V) \sim \CN(0,I)$.
\end{proof}
%
Notar  que, usando  la descomposici\'on  de Cholesky,  tenemos \  $M$ triangular
inferior~\footnote{Se puede hacer el  mismo razonamiento con la forma triangular
  superior;  se cambia  los roles  de \  $X$  \ e  \ $Y$  \ en  las matrices  de
  covarianza.},  y  entonces bloc-triangular  inferior  \  $M =  \begin{bmatrix}
  \alpha  &  0  \\  \beta  &  \gamma \end{bmatrix}$.   Eso  conduce,  a  $M  M^t
= \begin{bmatrix} \Sigma_X & \Sigma_{X,Y} \\ \Sigma_{X,Y}^t &
  \Sigma_Y \end{bmatrix}  = \begin{bmatrix} \alpha \alpha^t &  \alpha \beta^t \\
  \beta \alpha^t &  \beta \beta^t + \gamma \gamma^t  \end{bmatrix}$.  Eso da por
ejemplo   \   $\alpha  =   \Sigma_X^{\frac12}$,   \   $\beta  =   \Sigma_{X,Y}^t
\Sigma_X^{-\frac12}$  \   y  \  $\gamma  =  \left(   \Sigma_Y  -  \Sigma_{X,Y}^t
  \Sigma_X^{-1} \Sigma_{X,Y} \right)^{\frac12}$.   A continuaci\'on, $A = \alpha
+  \gamma +  \imath \beta$  \ y  \ $B  = \alpha  - \gamma  + \imath  \beta$. Una
soluci\'on posible es entonces
%
\[
\left\{\begin{array}{lll}
A & = & \Sigma_X^{\frac12} + \left( \Sigma_Y - \Sigma_{X,Y}^t \Sigma_X^{-1} \Sigma_{X,Y}
\right)^{\frac12}  + \imath \Sigma_{X,Y}^t
\Sigma_X^{-\frac12}\\[2.5mm]
%
B & = &  \Sigma_X^{\frac12} - \left( \Sigma_Y - \Sigma_{X,Y}^t \Sigma_X^{-1} \Sigma_{X,Y}
\right)^{\frac12} + \imath \Sigma_{X,Y}^t
\Sigma_X^{-\frac12}
\end{array}\right.
\]
%
Se  puede   re-escribir  estas  matrices   a  partir  de   \  $\Sigma$  \   y  \
$\check{\Sigma}$       \      usando       las       relaciones      de       la
secci\'on~\ref{Ssec:MP:VAComplejos}.

La  secunda  extensi\'on  pone  en  juega  una sola  gaussiana  compleja  sin  su
conjugada~\cite{EriKoi06, SchSch03}:
%
\begin{teorema}
\label{Teo:MP:GaussianaComplejaWIDiago}
%
  Sea \ $Z \sim \CN(m,\Sigma,\check{\Sigma})$. Entonces, existe una matriz \ $C
  \in \Mat_{d,d}(\Cset)$ \ tal que
  %
  \[
  Z \egald C W + m
  \]
  %
  con \ $W \sim \CN(0,I,\Delta)$ \ con \ $\Delta \in \Pos_d(\Rset)$ \ (real) diagonal.
  %
\end{teorema}
\begin{proof}
  Eso  viene  de teoremas  de  diagonalizaci\'on  conjunta. M\'as  precisamente,
  siendo \ $\Sigma  \in \Pos_d^+(\Cset)$ \ y \  $\check{\Sigma} \in \Sim_d(\Cset)$, se
  aplica el  teorema~\cite[Teo.~7.6.5]{HorJoh13} diciendo que  existe una matriz
  no  singular (invertible)  \  $C$ \  tal  que \  $\Sigma  = C  C^\dag$  \ y  \
  $\check{\Sigma} = C  \Delta C^t$ \ con $\Delta$ \  real diagonal con elementos
  positivos  ($\Delta  \in  \Pos_d(\Rset)$  \  diagonal).  Inmediatamente,  por  el
  teorema~\ref{Teo:MP:StabilidadGaussianaCompleja}, tenemos
  %
  \[
  C^{-1} (Z - m) \egald W \sim \CN(0,I,\Delta)
  \]
  lo que cierra la prueba.
\end{proof}


Al final, v\'imos en la  secci\'on~\ref{Ssec:MP:VAComplejos} que si un vector es
circular, entonces su pseudo-covarianza es  nula, pero la rec\'iproca no vale en
general. Aparece que en el contexto gaussiano tenemos la rec\'iproca:
%

\begin{teorema}[Circularidad]\label{Teo:MP:CircularidadGaussiana}
%
Sea \ $Z \, \sim \, \CN(m,\Sigma,\check{\Sigma})$.  Entonces,
  %
  \[
  Z \: \mbox{ circular  en torno a } \: m \qquad  \Longleftrightarrow \qquad Z \:
\mbox{ propio en torno a } \: m
  \]
\end{teorema}
%
\begin{proof}
  V\'imos     la    directa    en     la    secci\'on~\ref{Ssec:MP:VAComplejos},
  teorema~\ref{Teo:MP:Circularidad}.  Reciprocamente,  si \  $Z$ \ es  propio en
  torno a \ $m$, por definici\'on \ $\check{\Sigma} = 0$ \ y el resultado viene
  de la forma de  la funci\'on caracter\'istica por ejemplo: $\Phi_{Z-m}(\omega)
  = e^{-\frac14 \omega^\dag \Sigma \omega } = \Phi_{Z-m}\left( e^{\imath \theta}
    \omega \right) = \Phi_{e^{\imath \theta} (Z-m)}(\omega)$.
\end{proof}


%\SZ{Caso $X \sim \N(m,\Sigma)$; modulaci\'on $Z = e^{\imath \theta} X$}
}

% --------------------------------- Gamma
\subsubseccion{Distribuci\'on gamma}
\label{Sssec:MP:Gamma}

Como lo introdujimos en el ejemplo  de la ley exponencial, esta familia de leyes
fue  estudiada  por primera  vez  al  fin del  siglo  XIV,  bajo  el impulso  de
Pearson~\cite{Pea95}.   De hecho,  seg\'un Lancaster~\cite{Lan66}  se encuentran
trazas  de esta  ley en  trabajos de  Laplace como  posterior  distribuci\'on en
inferencia Bayesiana (elementos conduciendo a la ley gamma) para la estimaci\'on
de  la dispersi\'on  $\frac1{\sigma^2}$  de  una ley  gaussiana.   De hecho,  la
distribuci\'on gamma aparece frecuentemente en problemas de inferencia Bayesiana
como    distribuci\'on     a    priori    conjugado~\footnote{Ver     nota    de
  pie~\ref{Foot:MP:BayesPrior}  por la explicaci\'on  del enfoque  bayesiano que
  consiste  a calcular  la distribuci\'on  a  posteriori $p_{\Theta|X=x}(\theta)
  \propto p_{X|\Theta=\theta}(x) p_\Theta(\theta)$ \  usando la ley de los datos
  parametrizado  por $\theta$  que queremos  inferir, modelizado  aleatorio. Por
  eso,  como  se  lo ve  en  la  f\'ormula  de  Bayes,  se necesita  elegir  una
  distribuci\'on a priori  \ $p_\Theta$.  V\'imos que una  elecci\'on posible es
  tomarla uniforme. Puede  ser problematico por ejemplo cuando  $\theta$ vive en
  un  espacio de  volumen inifinito  (a  priori impropio),  a\'un si  se lo  usa
  frecuentemente (en estimaci\'on es equivalente a considerar la verosimulitud).
  Una otra elecci\'on posible es tomar  el a priori en una familia parametrizada
  tal que la  distribuci\'on a posterior partenece tambi\'en  a esta familia: es
  lo que se llama {\em a  priori conjugado} para la ley de sampleos $p_{X|\Theta
    = \theta}$. La idea es que  si vienen observaciones, en lugar de re-calcular
  la ley a posteriori, se  puede actualizar solamente los par\'ametros (llamados
  hiperpar\'ametros).\label{Foot:MP:BayesPriorConjugado}}     del    par\'ametro
$\lambda$ de la  ley de Poisson~\cite{Rob07}. Se encuentren  tambi\'en trazas de
esta ley en trabajos de J.  Bienaym\'e como distribuci\'on l\'imite del promedio
centrado   y  renormalizado   de   los   componentes  de   un   vector  de   ley
multinomial~\cite{Bie38, Lan66}.

Se  denota $X \,  \sim \,  \G(a,b)$ \  con \  $a \in  \Rset_+^*$ \  llamado {\em
par\'ametro de  forma} \ y \  $b \in \Rset_+^*$  \ llamada {\em taza}  (inversa de
{\em escala}). Las caracter\'isticas son:

\begin{caracteristicas}
%
Dominio de definici\'on & $\X = \Rset_+$\\[2mm]
\hline
%
Par\'ametros & $a \in \Rset_+^*$ \ (forma), \: $b \in \Rset_+^*$ \ (taza)\\[2mm]
\hline
%
Densidad  de probabilidad  &  $\displaystyle p_X(x)  =  \frac{b^a \, x^{a-1} \,  e^{-b
x}}{\Gamma(a)}$\\[2mm]
\hline
%
Promedio & $\displaystyle m_X = \frac{a}{b}$\\[2mm]
\hline
%
Varianza & $\displaystyle \sigma_X^2 = \frac{a}{b^2}$\\[2mm]
\hline
%
\modif{Asimetr\'ia} & $\displaystyle \gamma_X = \frac2{\sqrt{a}}$\\[2mm]
\hline
%
Curtosis por exceso & $\displaystyle \widebar{\kappa}_X = \frac6{a}$\\[2mm]
\hline
%
Generadora  de momentos  & $\displaystyle  M_X(u) =  \left( 1  - \frac{u}{b}
\right)^{-a}$ \ para \ $\real{u} < b$\\[2mm]
\hline
%
Funci\'on  caracter\'istica  &  $\displaystyle   \Phi_X(\omega)  =  \left(  1  -
\frac{ \imath \omega}{b} \right)^{-a}$
\end{caracteristicas}

% Momentos & $ \Esp\left[ X^k \right] = p^k$\\[2mm]
% Momento factorial & $\Esp\left[ (X)_k \right] = ?$\\[2mm]
% Generadora de probabilidad & $G_X(z) = e^{\lambda (z-1)}$ \ para \ $z \in \Cset$\\[2mm]
% modo max(a-1,0)
% Mediana no close ver inverse gamma

Nota: trivialmente, se puede escribir $X \,  \egald \, \frac{1}{b} G$ \ con \ $G
\, \sim \, \G(a,1)$  \ donde \ $G$ \ es estandardizada  o normalizada. De nuevo,
las  caracter\'isticas  de \  $X$  \  son  v\'inculadas a  las  de  \ $G$  \  (y
vice-versa) por transformaci\'on lineal (ver secciones anteriores).

Unas densidades de probabilidad gamma y las funciones de repartici\'on asociadas
son representadas en la figura Fig.~\ref{Fig:MP:Gamma}  para varios $a$ \ y \ $b
= 1$.
%
\begin{figure}[h!]
\begin{center} \begin{tikzpicture}%[scale=.9]
\shorthandoff{>}
%
\pgfmathsetmacro{\sx}{.75};% x-scaling
\pgfmathsetmacro{\mx}{8};% x maximo del plot
%
% Approximation de la cdf gaussienne
\tikzset{declare function={
normcdf(\x)=1/(1 + exp(-0.07056*(\x)^3 - 1.5976*(\x)));
}}
%
% densidad
\begin{scope}
%
\pgfmathsetmacro{\sy}{2.5};% y-scaling 
\draw[>=stealth,->] (-.75,0)--({\sx*\mx+.25},0) node[right]{\small $x$};
\draw[>=stealth,->] (0,-.1)--(0,2.75) node[above]{\small $p_X$};
%
%\foreach \a in {1,...,3} {
\draw[thick] (-.5,0)--(0,0);
\draw[thick,dotted,domain=.175:\mx,samples=100] plot ({\x*\sx},{\sy*(\x^(-.5))*exp(-\x)/sqrt(pi)});
\draw[thick,dashed,domain=0:\mx,samples=100] plot ({\x*\sx},{\sy*exp(-\x)});
\draw[thick,dash dot,domain=0:\mx,samples=100] plot ({\x*\sx},{\sy*\x*exp(-\x)});
%\draw[thick,domain=0:\mx,samples=100] plot ({\x*\sx},{\sy*4*\x*sqrt(\x)*exp(-\x)/3/sqrt(pi)});
\draw[thick,domain=0:\mx,samples=100] plot ({\x*\sx},{\sy*\x*\x*exp(-\x)/2});
%}
%
\draw (0,\sy)--(-.1,\sy) node[left,scale=.7]{$1$};
\draw (0,{\sy*exp(-1)})--(-.1,{\sy*exp(-1)}) node[left,scale=.7]{$e^{-1}$};
\draw (0,{\sy*2*exp(-2)})--(-.1,{\sy*2*exp(-2)}) node[left,scale=.7]{$2 \, e^{-2}$};
\draw (\sx,0)--(\sx,-.1) node[below,scale=.7]{$1$};
\draw ({2*\sx},0)--({2*\sx},-.1) node[below,scale=.7]{$2$};
%
\end{scope}
%
%
% reparticion
\begin{scope}[xshift=8.5cm]
%
\pgfmathsetmacro{\sy}{2.5};% y-scaling 
%
\draw[>=stealth,->] (-.75,0)--({\sx*\mx+.25},0) node[right]{\small $x$};
\draw[>=stealth,->] (0,-.1)--(0,{\sy+.25}) node[above]{\small $F_X$};
%
% cumulativa
\draw[thick] (-.5,0)--(0,0);
\draw[thick,dotted,domain=0:\mx,samples=100] plot ({\x*\sx},{(2*normcdf(sqrt(2*\x))-1)*\sy});
\draw[thick,dashed,domain=0:\mx,samples=100] plot ({\x*\sx},{\sy*(1-exp(-\x))});
\draw[thick,dash dot,domain=0:\mx,samples=100] plot ({\x*\sx},{\sy*(1-(1+\x)*exp(-\x))});
\draw[thick,domain=0:\mx,samples=100] plot ({\x*\sx},{\sy*(1-(1+\x+\x*\x/2)*exp(-\x))});
% plot({\x*\sx},{\sy*normcdf(\x)});
%
\draw (0,\sy)--(-.1,\sy) node[left,scale=.7]{$1$};
\end{scope}
%
\end{tikzpicture} \end{center}
%
\leyenda{Ilustraci\'on de una densidad de probabilidad gamma (a), y la funci\'on
de  repartici\'on asociada  (b).   $b  = 1$  \  y \  $a  = 0.5$  (linea
punteada), $1$ (linea mixta), $2$ (linea guionada) y $3$ (linea llena).}
\label{Fig:MP:Gamma}
\end{figure}

Cuando $a \in \Nset^*$ es entero, la ley es a veces conocida como ley de Erlang,
del nombre  de un ingeniero dan\'es  trabajando en (fundador de  la) teor\'ia de
colas~\cite{Cox62, Erl09, Erl25, BroHal48}.  Si \  $a = \frac{n}{2}$ \ con \ $n$
\ entero y  \ $\beta = \frac12$, se conoce tambi\'e  como ley {\em chi-cuadrado}
con \ $n$ \ grados de libertad (ver ej.~\cite{JohKot95:v1}).
% cf archivo queueing theory "Hillier"en mi carpeta

Notar que \ $X \, \sim \,  \G(1,b)$ \ es una variable exponencial de par\'ametro
\ $b$,  \ie \ $X  \, \sim \,  \E(b)$. Cuando \  $a < 1$,  la densidad \  $p_X$ \
diverge  para \  $x  \to 0$  \  (divergencia integrable).  Adem\'as, se  muestra
tambi\'en sencillamente con las funciones caracter\'isticas que:
%
\begin{lema}[Stabilidad]
\label{Lem:MP:StabilidadGamma}
%
  Sean $X_i  \, \sim  \, \G\left( a_i  , b  \right), \: i  = 1 ,  \ldots ,  n$ \
  independientes. Entonces
  % 
  \[
  \sum_{i=1}^n X_i \, \sim \, \G\left( \sum_{i=1}^n a_i \, , \, b \right)
  \]
\end{lema}
%
En particular, la suma de  variables independientes de ley exponencial de mismos
par\'ametro sigue una distribuci\'on de Erlang de par\'ametro de forma $n$.

Adem\'as, se  muestra sencillamente por cambio  de variables y  con la funci\'on
caracter\'istica un v\'inculo con variables gaussianas:
%
\begin{lema}[V\'inculo con la gaussiana]
\label{Lem:MP:VinculoGammaGaussiana}
%
  Sean $X_i \, \sim \,  \N\left( 0 , \sigma^2 \right), \: i = 1  , \ldots , n$ \
  independientes. Entonces
  %
  \[
  \sum_{i=1}^n  X_i^2 \,  \sim \,  \G\left( \frac{n}{2}  \, ,  \,  \frac{1}{2 \,
      \sigma^2} \right)
  \]
  %
  En  esta situaci\'on,  con  $n$ entero,  la  ley es  precisamente  la ley  del
  chi-cuadrado, con \ $n$ \ grados de libertad.
\end{lema}

\begin{ejemplo}[Distibuci\'on de Maxwell-Boltzmann]
  Esta distribuci\'on apareci\'o en el  estudio de las velocidades de particulas
  en el gas perfecto, bajo  el impulso de Maxwell~\cite{Max60A, Max60B, Max67} y
  m\'as tarde de  Boltzmann~\cite{Bol77, Bol96, Bol98}.  En un  gas perfecto, se
  supone que  las particulas se  mueven libremente, sin interacciones  entre si,
  aparte  colisiones  breves con  intercambio  de  energ\'ia  entre si  (choques
  elasticos).   La energia  de cada  particula es  su energ\'ia  cinetica  $\E =
  \frac12   m  \|   v  \|^2$   donde   $v  =   \begin{bmatrix}  v_x   &  v_y   &
    v_z\end{bmatrix}^t$ es  el vector velocidad  3-dimensional.  En un  gas, hay
  tantas  particulas~\footnote{`!En   condiciones  normales  de   temperatura  y
    presi\'on, un  litro contiene  \ $2.7 \times  10^{22}$ particulas!}   que es
  imposible  describir tal  gas con  las leyes  de la  m\'ecanica.   Se modeliza
  entonces  las  velocidades  como   aleatorias.   Adem\'as,  en  este  contexto
  ``perfecto'', las particulas son  supuestas independientes entre si.  Se puede
  focalizarse sobre una  particula, que representa de una  manera el conjunto de
  particulas.          Como          lo         veremos         en         ambas
  secci\'on~\ref{Ssec:MP:FamiliaExponencial}                                    y
  cap\'itulo~\ref{Cap:SZ:Informacion},   sin   v\'inculos   adicional,   en   el
  equilibrio termodin\'amico, la ley del  vector velocidad de la particula es la
  que  maximiza la  entrop\'ia. Es  precisamente una  gausiana  3-dimensional de
  covarianza  \  $\Sigma =  \frac{m}{2  \,  k_B  T} I$,  donde  \  $T$ \  es  la
  temperatura del gas en Kelvin, y  \ $k_B \approx 1.38 \times 10^{-23}$ \ julio
  por Kelvin es la constante  de Boltzmann. En otros t\'erminos, les velocidades
  en cada direcci\'on \ $v_x, v_y,  v_z$ \ son gausianas de varianza \ $\sigma^2
  = \frac{m}{2 \,  k_B T}$ independientes.  Resuelte que la ley  de la energia \
  $\E = \frac12 m \| v \|^2$ \ es precisamente una ley chi-cuadrado con 3 grados
  de  libertad  y  la  de  \  $\|  v \|$,  la  raiz  cuadrada  de  variable  del
  chi-cuadrado, es conocida como ley  chi. En el vcaso presente, es precisamente
  conocida como ley de Maxwell-Boltzmann.
\end{ejemplo}

%\SZ{Esta distribuci\'on aparece...}
% en  el conteo  de conteo  de  une repetici\'on  de una  experiencia de  maneja
% independiente hasta que  occure un evento de probabilidad  $p$; por ejemplo el
% n\'umero de tiro de un dado  equilibriado hasta que occurre un ``6'' sigue una
% ley geometrica de par\'ametro $p = \frac16$.

% --------------------------------- Wishart
\subsubseccion{Distribuci\'on matriz-variada de Wishart}
\label{Sssec:MP:Wishart}

Este ejemplo es una  generalizaci\'on matriz-variada de la distribuci\'on gamma.
Se puede ver  una matriz como un vector, guardando por  ejemplo sus columnas una
bajo la  precediente.  Sin embargo, tal  distribuci\'on apareciendo naturalmente
en un contexto de estimaci\'on de  matriz de covarianza (ver m\'as adelante), es
m\'as  natural  verla  matriz-variate.    Tal  distrubuci\'on  es  debido  a  J.
Wishart~\cite{Wis28, GupNag99, And03}, y se denota \ $X \, \sim \, W_d(V,\nu)$ \
donde  el  dominio  de  definici\'on  es  \  $P_d^+(\Rset)$,  conjunto  matrices
simetricas definida positivas, $V \in P_d^+(\Rset)$ par\'ametro de escala y \ $\nu
> d-1$ \ grados de libertad.  Las caracter\'isticas de la distribuci\'on son las
siguientes:

\begin{caracteristicas}
%
Dominio de definici\'on~\footnote{De hecho, se puede considerar que la matriz
aleatoria es equivalent a tener un vector \ $\frac{d (d+1)}{2}$-dimensional; por
la simetria, claramente \ $X$ \ tiene solamente \ $\frac{d (d+1)}{2}$ \
componentes diferentes; adem\'as, se puede probar que cualquier matriz \ $A \in
P_d^+(\Rset)$ \ se descompone bajo la forma \ $A = L L^t$ \ con \ $L$ \
triangular inferior con elementos no nulos sobre su diagonal, llamado
descomposici\'on de Cholesky~\cite{GupNag99, Bha07, Har08, HorJoh13} y
reciprocamente. Eso muestra que \ $A$ \ se define a partir de \ $\frac{d
(d+1)}{2}$ \ ``grados de libertad''.\label{Foot:MP:WishartXtilde}} & $\X =
P_d^+(\Rset), \: d \in \Nset^*$\\[2mm]
\hline
%
Par\'ametros & $V \in P_d^+(\Rset)$ (escala) y \ $\nu > d-1$ \ (grados de
libertad)\\[2mm]
\hline
%
Densidad de probabilidad~\footnote{La densidad de probabilidad corresponde a la
densidad conjunta de los \ $\frac{d (d+1)}{2}$ \ elementos \ $X_{i,j}, \: 1 \le
i \le j \le d$~\cite{Wis28, PedRic91, SulTra96, GupNag99,
And03}.\label{Foot:MP:WishartDensidad}} & $\displaystyle p_X(x) =
\frac{|x|^{\frac{\nu-d-1}{2}} \, e^{-\frac12 \Tr\left( V^{-1}
x\right)}}{2^{\frac{d \nu}{2}} \, |V|^{\frac{\nu}{2}} \, \Gamma_d\left(
\frac{\nu}{2} \right)}$\\[2mm]
\hline
%
Promedio & $\displaystyle m_X = \nu \, V$\\[2mm]
\hline
%
Covarianza & $\displaystyle \Sigma_X = \nu \big( J (V \otimes V) + (V \otimes I)
K (V \otimes I) \big)$\\[2mm]
% $\Cov[X_{i,j},X_{k,l}] = nu \left( V_{i,k} V_{j,l} +V_{i,l} V_{j,k} \right)$}
\hline
%
Funci\'on caracter\'istica~\footnote{\SZ{Se proba que la funci\'on generadora de
momentos no existe en
general}.\label{Foot:MP:CaracteristicaWishart}} &
$\displaystyle \Phi_X(\omega) = \left| I - 2 \imath \omega V
\right|^{-\frac{\nu}{2}}, \quad \omega \in S_d(\Rset)$
\end{caracteristicas}

(ver~\cite{PedRic91, SulTra96, And03}).

Fijense que $p_X$ no es la distribuci\'on conjuntos de los componentes de \ $X$:
el hecho de  que \ $X$ \ sea  uan matriz aleatoria de \  $P_d^+(\Rset)$ \ impone
v\'inculos sobre sus compnentes; entre otros, $X_{i,j} = X_{j,i}$.

Inmediatamente, si  $d = 1$, la  distribuci\'on de Whishart \  $W_1(V,\nu)$ \ se
reduce  a la  distribuci\'on Gamma  $\G\left(\frac{\nu}{2} \,  , \,  \frac1{2 V}
\right)$. De este  hecho, se la podr\'ia ver  como extensi\'on matriz-variada de
la  distribuci\'on  gamma.  La  distribuci\'on  de  Wishart  tiene varias  otras
propiedades como las siguientes.
%
\begin{lema}[Stabilidad por transformaci\'on lineal]
\label{Lem:MP:StabilidadWishartLineal}
%
  Sea $X \,  \sim \, W_d(V,\nu)$ \ y \  $A \in \Rset^{d \times d'}$  \ con \ $d'
  \le d$ \ y de rango lleno. Entonces
  \[
  A^t X A \, \sim \, W_{d'}\left( A^t V A , \nu \right)
  \]
  %
  En particular, si $d'  = 1$, \ $A^t X A \, \sim  \, G\left( \frac{\nu}{2} \, ,
    \, \frac1{2 \, A^t V A} \right)$. M\'as all\'a, tomando $A = \un_j$, aparece
  de que  las componentes diagonales de \  $X$ \ son de  distribuci\'on gamma, \
  $X_{j,j}  \, \sim  \,  \G\left( \frac{\nu}{2}  \,  , \,  \frac1{2 \,  V_{j,j}}
  \right)$.
\end{lema}
%
\begin{proof}
  El     resultado     es     inmediato     saliendo     de     la     funci\'on
  caract\'eristica~\footref{Foot:MP:CaracteristicaWishart} y notando de que
%
\begin{eqnarray*}
\Phi_{A^t X A}(\omega) & = & \Esp\left[ e^{\imath \Tr\left( \omega^t A^t X A
\right)}\right]\\[2mm]
%
& = & \Phi_X\left( A \omega^t A^t\right)\\[2mm]
%
& = &  \left| I - 2 \imath A \omega A^t V \right|^{-\frac{\nu}{2}}\\[2mm]
%
& = &  \left| I - 2 \imath \omega A^t V A \right|^{-\frac{\nu}{2}}
%
\end{eqnarray*}
%
de   \   $\Tr(AB)   =   \Tr(BA)$~\cite{Har08}   \   y   de   la   identidad   de
Sylvester~\cite{Syl51,  AkrAkr96}  o~\cite[\S~18.1]{Har08} \  $\left|  I  + A  B
\right| = \left| I + B A \right|$.  .
\end{proof}
%
De hecho, si los elementos diagonales son de distribuci\'on gamma, no es el caso
de         los        elementos         no-diagonales~\cite{Seb04,        And03}
o~\cite[Teo.~3.3.4]{GupNag99}.    De    eso   resuelte   delicado    llamar   la
distribuci\'on como gamma matriz-variada.

\begin{lema}[Stabilidad por suma]
\label{Lem:MP:StabilidadWishartSuma}
%
  Sea $X_i \,  \sim \, W_d(V,\nu_i), \: i = 1,\ldots,n$ independientes. Entonces
  \[
  \sum_{i=1}^n X_i \, \sim \, W_d\left( V \, , \, \sum_{i=1}^n \nu_i \right)
  \]
\end{lema}
%
\begin{proof}
  El     resultado     es     inmediato     saliendo     de     la     funci\'on
  caract\'eristica~\footref{Foot:MP:CaracteristicaWishart} y notando que como el
  el context vectorial $\Phi_{\sum_i X_i} = \prod_i \Phi_{X_i}$.
\end{proof}

La distribuci\'on  de Wishart aparece naturalmente en  problemas de estimaci\'on
de matriz de covarianza en el contexto gausiano:
%
\begin{lema}[V\'inculo con vectores gausianos~\cite{Seb04}]
  Sean \ $X_i \, \sim \, \N(0,V), \: i = 1, \ldots , n > d-1$ \ independientes y
  la  matriz  \  $S  =  \sum_{i=1}^n   X_i  X_i^t$  \  llamada  {\em  matriz  de
    dispersi\'on} (scatter matrix en ingl\'es). Entonces, \ $S \in P_d^+(\Rset)$
  (c.   s.)   \  ($S$ es  sim\'etrica  definida  positiva  casi siempre,  o  con
  probabilidad uno) y \ $S \,\sim \, W_d(V,n)$.
\end{lema}
%
Este resultado  permite tambi\'en probar  el lema~\ref{Lem:MP:StabilidadWishart}
para \ $\nu = n$ \ entero  escribiendo \ $X \egald \sum_{i=1}^n X_i X_i^t$ \ tal
que \ $A^t X A \egald \sum_{i=1}^n A^t X_i X_i^t A = \frac1n \sum_{i=1}^n \left(
  A^t X_i \right)  \left( A^t X_i \right)^t$ \  y notando que los \  $A^t X_i \,
\sim \,  \N(0, A^t V  A)$ \ son independientes~\cite{Seb04}.   Adem\'as, permite
re-obtener las  expreciones del promedio y de  las covarianzas~\footnote{Para la
  covarianza, su usa la formula \ $\Esp[Y_1 Y_2 Y_3 Y_4] = \Esp[Y_1 Y_2]\Esp[Y_3
  Y_4]  + \Esp[Y_1 Y_3]\Esp[Y_2  Y_4] +  \Esp[Y_1 Y_4]\Esp[Y_2  Y_3]$ \  para $Y
  = \begin{bmatrix}  Y_1 & Y_2 &  Y_3 & Y_4 \end{bmatrix}^t$  \ vector gausiano,
  formula que se  obtiene por ejemplo a partir  de la funci\'on caracter\'istica
  de un vecor  gausiano.}. Notar que cuando los \ $X_i$  \ tienen un promemedio,
el  lema conduce  a  lo que  es  conocido como  Wishart no  central~\cite{And03,
  Seb04}.

\SZ{Que propedad mas? Ver Gupta Nagar 1999}

La distribuci\'on  Wishart aparece as\'i naturalmente en  problema de inferencia
Bayesiana  como distribuci\'on  a  priori conjugado~\footref{Foot:MP:BayesPrior}
del par\'ametro $p$ de la ley gaussiana multivariada~\cite{Rob07}.

\SZ{Y donde aparece mas?}

% --------------------------------- Beta
\subsubseccion{Distribuci\'on beta}
\label{Sssec:MP:Beta}

Estas   distribuciones  fueron   popularizadas   por  Pearson   en  los   a\~nos
1895~\cite{Pea95, Pea16, DavEdw01}  bajo la denominaci\'on Pearson tipo  I en su
estudio  de  la teoria  de  la evoluci\'on  y  la  modelizaci\'on con  variables
asim\'etricas.  De  hecho, apareci\'o mucho  tiempo antes, en trabajos  de Bayes
publicado en un papel postumo  por R. Price en 1763~\cite{Bay63}. Aparentemente,
la denominaci\'on  estandar ``beta'' es  debido al estad\'istico,  dem\'ografo y
soci\'ologo  italiano   C.   Gini   en  1911  su   estudio  del   ``sex  ratio''
(desequilibrio  entre los  nacimientos  de muchachos/muchachas)  con un  enfoque
bayesiano~\cite{Gin11,  For17,   DavEdw01}.   La  distribuci\'on   beta  aparece
precisamente,   entre  otros,   en   problema  de   inferencia  bayesiana   como
distribuci\'on   a   priori   conjugado   del   par\'ametro  $p$   de   la   ley
binomial~\cite{Rob07}     (ver     notas     de     pie~\ref{Foot:MP:BayesPrior}
y~\ref{Foot:MP:BayesPriorConjugado}).

Se denota $X  \sim \beta(a,b)$ \ con  \ $(a,b) \in \Rset_+^{* \,  2}$ \ llamados
{\em par\'ametros de forma}.  Las caracter\'isticas son:

\begin{caracteristicas}
%
Dominio de definici\'on & $\X = [0 \; 1]$\\[2mm]
\hline
%
Par\'ametros & $(a,b) \in \Rset_+^{* \, 2}$ (forma)\\[2mm]
\hline
%
Densidad   de    probabilidad   &   $\displaystyle    p_X(x)   =   \frac{x^{a-1}
(1-x)^{b-1}}{B(a,b)}$\\[2mm]
\hline
%
Promedio & $\displaystyle m_X = \frac{a}{a+b}$\\[2mm]
\hline
%
Varianza &  $\displaystyle \sigma_X^2  = \frac{a b}{(a  + b)^2  (a + b  + 1)}$\\[2mm]
\hline
%
\modif{Asimetr\'ia} & $\displaystyle \gamma_X = \frac{2 \, (b - a) \sqrt{a + b + 1}}{( a
+ b + 2) \sqrt{a b}}$\\[2mm]
\hline
%
Curtosis por exceso & $\displaystyle \widebar{\kappa}_X = \frac{6 \, \left( (a - b)^2 (a + b + 1) - a
b (a  + b  + 2)  \right)}{a \, b  \left( a  + b  + 2 \right)  \left( a  + b  + 3
\right)}$\\[2mm]
\hline
%
Generadora de momentos & $\displaystyle M_X(u)  = \hypgeom{1}{1}\left( a , a + b
\, ; \, u \right)$ \ para \ $u \in \Cset$\\[2mm]
\hline
%
Funci\'on     caracter\'istica     &     $\displaystyle     \Phi_X(\omega)     =
\hypgeom{1}{1}\left( a , a + b \, ; \, \imath \omega \right)$
\end{caracteristicas}

Unas  densidades de  probabilidad  y funciones  de  repartici\'on asociadas  son
representadas en la figura Fig.~\ref{Fig:MP:Beta} para varios $a$ \ y \ $b$.
%
\begin{figure}[h!]
\begin{center} \begin{tikzpicture}%[scale=.9]
\shorthandoff{>}
%
\pgfmathsetmacro{\sx}{5};% x-scaling
\pgfmathsetmacro{\b}{7};% tercera eleccion de (3,beta)
\pgfmathsetmacro{\r}{.05};% radius arc non continuity F_X
%\pgfmathsetmacro{\mb}{max(2,\b*(\b+1)*((\b-1)^(\b-1))/((\b-2)^\b))};% maximo pdf para alpha = 2
\pgfmathsetmacro{\mb}{max(3,2*\b*(\b+2)/(\b+1)*(((\b-1)/(\b+1))^(\b-1))};% maximo pdf para alpha = 3
%
%
% densidad
\begin{scope}
%
%\pgfmathsetmacro{\sy}{2.5*((\b/(\b-1))^(\b-1))/(\b+1)};% y-scaling 
\pgfmathsetmacro{\sy}{2.5/\mb};% y-scaling
\draw[>=stealth,->] (-.5,0)--({\sx+.5},0) node[right]{\small $x$};
\draw[>=stealth,->] (0,-.1)--(0,3) node[above]{\small $p_X$};
%
%\draw[thick] (-.25,0)--(0,0);\draw (\r,\r) arc (90:270:\r);
%\draw[thick] (\sx,0)--({\sx+.25},0);\draw ({\sx-\r},{-\r}) arc (-90:90:\r);
% (a,b) = ( .5 , .5 )
\draw[thick,dotted,domain=.01:.99,samples=100] plot ({\x*\sx},{\sy*1/(pi*sqrt(\x*(1-\x)))});
% (a,b) = ( 2 , 1 )
%\draw[thick,dashed,domain=0:1,samples=100] plot ({\x*\sx},{\sy*2*\x}) node[scale=.4]{$\bullet$};
%\draw[dotted] (\sx,{2*\sy})--(\sx,0);
% (a,b) = ( 3 , 1 )
\draw[thick,dash dot dot,domain=0:1,samples=100] plot ({\x*\sx},{\sy*3*(\x^2)}) node[scale=.4]{$\bullet$};
\draw[dotted] (\sx,{3*\sy})--(\sx,0);
% (a,b) = ( 2 , 2 )
%\draw[thick,dash dot,domain=0:1,samples=100] plot ({\x*\sx},{\sy*6*\x*(1-\x)});
% (a,b) = ( 3 , 2 )
\draw[thick,dash dot,domain=0:1,samples=100] plot ({\x*\sx},{\sy*12*(\x^2)*(1-\x)});
% (a,b) = ( 3 , 3 )
\draw[thick,dashed,domain=0:1,samples=100] plot ({\x*\sx},{\sy*30*(\x^2)*((1-\x)^2)});
% (a,b) = ( 2 , b_3 )
%\draw[thick,domain=0:1,samples=100] plot ({\x*\sx},{\sy*\b*(\b+1)*\x*((1-\x)^(\b-1))});
% (a,b) = ( 3 , b_4 )
\draw[thick,domain=0:1,samples=100] plot ({\x*\sx},{\sy*.5*\b*(\b+1)*(\b+2)*(\x^2)*((1-\x)^(\b-1))});
%}
%
\draw (0,{\sy*2/pi})--(-.1,{\sy*2/pi}) node[left,scale=.7]{$\frac2\pi$};
\draw (0,{\sy*3})--(-.1,{\sy*3}) node[left,scale=.7]{$3$};
\draw (0,{\sy*15/8})--(-.1,{\sy*15/8}) node[left,scale=.7]{$\frac{15}{8}$};
%\draw (0,{\sy*1.5})--(-.1,{\sy*1.5}) node[left,scale=.7]{$\frac32$};
%\draw (0,{\sy*(\b+1)*((1-1/\b)^(\b-1))})--(-.1,{\sy*(\b+1)*((1-1/\b)^(\b-1))}) node[left,scale=.7]{$6 \left(\frac{4}{5} \right)^4$};
%\draw (\sx,0)--(\sx,-.1) node[below,scale=.7]{$1$};
%
\end{scope}
%
%
% reparticion
\begin{scope}[xshift=8.5cm]
%
\pgfmathsetmacro{\sy}{2.5};% y-scaling 
%
\draw[>=stealth,->] (-.75,0)--({\sx*1.25+.25},0) node[right]{\small $x$};
\draw[>=stealth,->] (0,-.1)--(0,{\sy+.25}) node[above]{\small $F_X$};
%
% cumulativa
\draw[thick] (-.5,0)--(0,0); \draw[thick] (\sx,\sy)--({\sx*1.25},\sy);
% (a,b) = ( .5 , .5 )
\draw[thick,dotted,domain=0:1,samples=100] plot ({\x*\sx},{\sy*(.5+asin(2*\x-1)/180)});
% (a,b) = ( 2 , 1 )
%\draw[thick,dashed,domain=0:1,samples=100] plot ({\x*\sx},{\sy*(1-(1+\x)*(1-\x))});
% (a,b) = ( 3 , 1 )
\draw[thick,dash dot dot,domain=0:1,samples=100] plot ({\x*\sx},{\sy*(1-(1+\x+\x^2)*((1-\x)^2))});
% (a,b) = ( 2 , 2 )
%\draw[thick,dash dot,domain=0:1,samples=100] plot ({\x*\sx},{\sy*(1-(1+2*\x)*((1-\x)^2))});
% (a,b) = ( 3 , 2 )
\draw[thick,dash dot,domain=0:1,samples=100] plot ({\x*\sx},{\sy*(1-(1+2*\x+3*\x^2)*((1-\x)^2))});
% (a,b) = ( 2 , b_3 )
%\draw[thick,domain=0:1,samples=100] plot ({\x*\sx},{\sy*(1-(1+\b*\x)*((1-\x)^\b))});
% (a,b) = ( 3 , b_4 )
\draw[thick,domain=0:1,samples=100] plot ({\x*\sx},{\sy*(1-(1+\b*\x+.5*\b*(\b+1)*\x^2)*((1-\x)^\b))});
%
\draw (0,\sy)--(-.1,\sy) node[left,scale=.7]{$1$};
\draw (\sx,0)--(\sx,-.1) node[below,scale=.7]{$1$};
\end{scope}
%
\end{tikzpicture} \end{center}
%
\leyenda{Ilustraci\'on de una densidad de  probabilidad beta (a), y la funci\'on
de  repartici\'on asociada (b).   $(a,b) =  (0.5 \,  , \,  0.5)$ (linea
punteada), $(3 \, , \, 1)$ (linea  mixta doble punteada), $(3 \, , \, 2)$ (linea
mixta), $(3 \, , \, 3)$ (linea
guionada), $(3 \, , \, 7)$ (linea llena).}
\label{Fig:MP:Beta}
\end{figure}

Notar que  se recupera la  ley uniforme sobre  \ $[0 \;  1]$ \ para  \ $a =  b =
1$. Se  conoce la ley de $  Y = 2 \,  B - 1$ \  con \ $B \,  \sim \, \beta\left(
  \frac12 , \frac12 \right)$ \ como {\em ley arcseno}.

Variables beta  tienen tambi\'en unas propiedades notables.  Primero, por cambio
de variables, se demuestra el lema siguiente:
%
\begin{lema}[Reflexividad]
\label{Lem:MP:ReflexividadBeta}
%
  Sea \ $X \, \sim \, \beta(a,b)$. Entonces
  %
  \[
  1-X \, \sim \, \beta(b,a)
  \]
  %
\end{lema}


\begin{lema}[Un v\'inculo con la ley exponencial]
\label{Lem:MP:VinculoBetaExponencial}
%
  Sea  \   $X  \,  \sim  \,   \beta(a,1)$. Entonces
  %
  \[
  - \log X \, \sim \, \E(a)
  \]
  %
\end{lema}
%
\begin{proof}
  El   resultado  es   inmediato  de   la  f\'ormula   de   transformaci\'on  del
  corolario~\ref{Cor:MP:TransformacionInyectivaDensidadEscalar}.
\end{proof}


\begin{lema}[Un v\'inculo con la ley uniforme]
\label{Lem:MP:VinculoBetaUniforme}
%
  Sea  \   $X  \,  \sim  \,   \U([0 \; 1])$ \ y \ $a > 0$. Entonces
  %
  \[
  U^{\frac{1}{a}} \, \sim \, \beta(a,1)
  \]
  %
\end{lema}
%
\begin{proof}
  El   resultado  es   inmediato  de   la  f\'ormula   de   transformaci\'on  del
  corolario~\ref{Cor:MP:TransformacionInyectivaDensidadEscalar}.
\end{proof}

\begin{lema}[Un v\'inculo con la ley gamma]
\label{Lem:MP:VinculoBetaGamma}
%
  Sea  \   $X  \,  \sim  \,   \G(a,c)$  \  e  \   $Y  \,  \sim   \,  \G(b,c)$  \
  independientes. Entonces
  %
  \[
  \frac{X}{X+Y} \, \sim \, \beta(a,b)
  \]
  %
  (independientemente  de $c$).   Adem\'as, $\frac{X}{X+Y}$  \ y  \ $X+Y$  \ son
  independientes.
\end{lema}
%
\begin{proof}
  La independencia de \ $c$ \ es obvia del hecho de que para cualquier $\theta >
  0, \: \theta^{-1} X \, \sim \, \G(a,\theta c)$ \ e \ $\theta^{-1} Y \, \sim \,
  \G(b,\theta  c)$,  la independencia  con  respeto  a \  $c$  \  viniendo de  \
  $\frac{\theta^{-1}     X}{\theta^{-1}     X     +     \theta^{-1}     Y}     =
  \frac{X}{X+Y}$.  Entonces, se  puede considerar  \ $c  = 1$  \ sin  perdida de
  generalidad. Ahora, sea la transformaci\'on
  %
  \[
  \begin{array}{lccl}
    g\ : & \Rset_+^2 & \mapsto & [0 \; 1] \times \Rset_+\\[1.5mm]
    %
    & (x,y) & \to & (u,v) = \left( \frac{x}{x+y} \, , \, x+y \right)
  \end{array}
  \]
  %
  Entonces, la transformaci\'on inversa se escribe
  %
  \[
  g^{-1}(u,v) = \left( u v \, , \, (1-u) v \right)
  \]
  %
  de matriz Jacobiana
  %
  \[
  \Jac_{g^{-1}} = \begin{bmatrix} v & u \\[2mm] -v & 1-u \end{bmatrix}
  \]
  %
  Del          teorema          de          cambio         de          variables
  teorema~\ref{Teo:MP:TransformacionBiyectiva},      notando     que     $\left|
    \Jac_{g^{-1}} \right| =  v$ \ y de la  independencia de \ $X$ \ e  \ $Y$, se
  obtiene para el vector aleatorio \ $W = \begin{bmatrix} U & V \end{bmatrix}^t$
  \ la densidad de probabilidad, definida sobre $[0 \; 1] \times \Rset_+$, como
  %
  \begin{eqnarray*}
    p_W(u,v) & = & p_X( u v ) \, p_Y( (1-u) v ) \, v\\[2mm]
    %
    & = & \frac{\left( u v \right)^{a-1} \, e^{- u v}}{\Gamma(a)} \times
    \frac{\left( (1-u) v \right)^{b-1} \, e^{- (1-u) v}}{\Gamma(b)} \times v\\[2mm]
    %
    & = & \frac{u^{a-1} (1-u)^{b-1}}{B(a,b)} \times \frac{v^{a+b-1} e^{-v}}{\Gamma(a+b)}
  \end{eqnarray*}
  %
  Inmediatamente, factorizandose, aparece  claramente que \ $U$ \ y  \ $V$ \ son
  independientes. Adem\'as, se reconoce en  el primer factor la densidad beta de
  par\'ametros $(a,b)$.   Pasando, se recupera  el hecho que  \ $X+Y \,  \sim \,
  \G(a+b,1)$.
\end{proof}

\begin{lema}[Stabilidad por producto]
\label{Lem:StabilidadBeta}
%
  Sea  \  $X \,  \sim  \, \beta(a,b)$  \  e  \ $Y  \,  \sim  \, \beta(a+b,c)$  \
  independientes. Entonces
  %
  \[
  X Y \, \sim \, \beta(a,b+c)
  \]
  %
\end{lema}
%
\begin{proof}
  Sean \ $U  \, \sim \, \G(a,1)$, \ $V \,  \sim \, \G(b,1)$ \ y \  $W \, \sim \,
  \G(c,1)$   \   independientes  y   sean   \   $X   =  \frac{U}{U+V}$,   $Y   =
  \frac{U+V}{U+V+W}$  \ y  \ $Z  = U+V+W$.  Del lema  anterior \  $X \,  \sim \,
  \beta(a,b)$ \ y \ $Y \, \sim \, \beta(a+b,c)$. Sea la transformaci\'on
  %
  \[
  \begin{array}{lccl}
    g\ : & \Rset_+^3 & \mapsto & [0 \; 1]^2 \times \Rset_+\\[1.5mm]
    %
    & (u,v,w) & \to & (x,y,z) = \left( \frac{u}{u+v} \, , \, \frac{u+v}{u+v+w} \, , \, u+v+w \right)
  \end{array}
  \]
  %
  Entonces, la transformaci\'on inversa se escribe
  %
  \[
  g^{-1}(x,y,z) = \left( x y z \, , \, (1-x) y z \, , \, z (1-y) \right)
  \]
  %
  de matriz Jacobiana
  %
  \[
  \Jac_{g^{-1}} = \begin{bmatrix}
  %
    y z  &   x z   &   x y   \\[2mm]
  %
  - y z  & (1-x) z & (1-x) y \\[2mm]
  %
    0    &  - z    &  1-y
  %
  \end{bmatrix}
  \]
  %
  De      nuevo,      del      teorema      de     cambio      de      variables
  teorema~\ref{Teo:MP:TransformacionBiyectiva}, notando que $\left| \Jac_{g^{-1}}
  \right| = y  z^2$ \ y de la independencia  de \ $U, V, W$,  se obtiene para el
  vector aleatorio  \ $  T =  \begin{bmatrix} X &  Y &  Z \end{bmatrix}^t$  \ la
  densidad de probabilidad probabilidad
  %
  \begin{eqnarray*}
    p_T(x,y,z) & = & p_u( x y z ) \, p_V( (1-x) y z ) \, p_W( y (1-z) ) \, y z^2\\[2mm]
    %
    & = & \frac{\left( x y z \right)^{a-1} \, e^{- x y z}}{\Gamma(a)} \times
    \frac{\left( (1-x) y z \right)^{b-1} \, e^{- (1-x) y z}}{\Gamma(b)} \times
    \frac{\left( z (1-y) \right)^{c-1} \, e^{- z (1-y)}}{\Gamma(c)} \times y
    z^2\\[2mm]
    %
    & = & \frac{x^{a-1} (1-x)^{b-1}}{B(a,b)} \times \frac{y^{a+b-1}
      (1-y)^{c-1}}{B(a+b,c)} \times \frac{z^{a+b+c-1} e^{-z}}{\Gamma(a+b+c)}
  \end{eqnarray*}
  %
  Eso proba  que \  $X, \ Y$  \ y  $Z$ \ son  independientes (las  densidades se
  factorizan). Adem\'as,
  %
  \[
  X  Y =  \frac{U}{U+V} \times  \frac{U+V}{U+V+W} =  \frac{U}{U+V+W} \,  \sim \,
  \beta(a,b+c)
  \]
  %
  el      \'ultimo       resultado      como      consecuencia       de      los
  lemas~\ref{Lem:MP:VinculoBetaGamma}    y~\ref{Lem:MP:StabilidadGamma}.     Eso
  cierra la prueba.
\end{proof}


\begin{lema}[Ley gamma como caso l\'imite de beta]
\label{Lem:GamaLimiteBeta}
%
  Sea \ $X_n \, \sim \, \beta(a,n)$. Entonces
  %
  \[
  n X_n \limitd{n \to +\infty} X \, \sim \, \G(a,1)
  \]
  %
  con \ $\displaystyle \limitd{}$ \ l\'imite es en distribuci\'on
\end{lema}
%
\begin{proof}
  De la f\'ormula de transformaci\'on tenemos la distribuci\'on de $n X_n$
  %
  \begin{eqnarray*}
  p_{n X_n}(x) & = & \frac{1}{n} , \frac{\left( \frac{x}{n} \right)^{a-1} \left( 1
  - \frac{x}{n} \right)^{n-1}}{B(a,n)} \, \un_{(0 \; 1)}\left( \frac{x}{n} \right)\\[2mm]
  %
  & = & \frac{x^{a-1}}{\Gamma(a)} \: \frac{\Gamma(n+a)}{n^a \Gamma(n)} \: \left( 1 -
  \frac{x}{n} \right)^{n-1} \: \un_{(0 \; n)(x)}
  \end{eqnarray*}
  %
  El resultado sigue  notando que \ $\un_{(0 \;  n)} \to \un_{\Rset_{0,+}}$, \quad
  $\left(  1 -  \frac{x}{n} \right)^{n-1}  \to e^{-x}$  \ y  de la  f\'ormula de
  Stirling (ver secci\'on~\ref{Sssec:MP:Poisson}).
\end{proof}

\

La distribuci\'on beta  se generaliza al caso matriz-variada  $X$ definido sobre
$\X$ tal que $X$ y $I-X$ partenecen  a $\Pos_d^+(\Rset)$; se denota \ $X \, \sim
\, \beta_d(a,b)$ \ donde  \ $(a,b) \in \Rset_+^{* \, 2}$ y  la densidad est dada
por   $\displaystyle  p_X(x)   =  \frac{|x|^{a   -  \frac{d+1}{2}}   |I-x|^{b  -
    \frac{d+1}{2}}}{B_p\left([a  \quad  b]^t\right)},  \quad  (a,b)  \in  \left(
  \frac{d-1}{2} \; +\infty \right)^2$. Se refiera a~\cite[Cap.~5]{GupNag99} para
tener m\'as detalles.   Notar que esta distribuci\'on cae en  en una clase dicha
el\'iptica,      que      vamos       a      ver      brevemente      en      la
secci\'on~\ref{Ssec:MP:FamiliaElipticaMatriz},  as\'i  que  propiedades en  este
marco general.


% --------------------------------- Dirichlet
\subsubseccion{Distribuci\'on de Dirichlet}
\label{Sssec:MP:Dirichlet}

Esta distribuci\'on  tiene su nombre de  integrales on a  simplex estudiados por
M. Lejeune-Dirichlet  y J. Liouville  en 1839~\cite{GupRic01, Dir39,  Lio39}. Es
una extensi\'on  multivariada de las variables  beta a veces  conocida como {\em
  beta multivariada}~\cite{OlkRub64}. Escribiendo  la forma de la distribuci\'on
solamente  con  la variables  $x_i$,  la  integral  permitiendo normalizarla  es
precisamente la estudiada por Lejeune-Dirichlet y Liouville.

Se nota \ $X  \, \sim \, \Dir(a)$ \ con  \ $a \in \Rset_+^{* \, k}$ \  y \ $X$ \
vive  sobre  el  $(k-1)$-simplex   estandar  \  $\Simp{k-1}$.   $a$  es  llamado
par\'ametro de forma. Como en el caso de vectores de distribuci\'on multinomial,
a pesar de que se escribe \ $X$ \ de manera $k$-dimensional, el vector partenece
a una variedad \ $d = k-1$ \ dimensional y en el caso \ $k = 2$ \ se recupera la
ley beta. A veces  se parametriza la ley con un par\'ametro  escalar \ $\alpha >
0$ \ y un vector del simplex estandar \ $\bar{a} \in \Simp{k-1}$ \ tal que
%
\[
a  = \alpha \bar{a},  \quad \mbox{\ie}  \quad \alpha  = \sum_{i=1}^k  a_i, \quad
\bar{a} = \frac{a}{\alpha}
\]
%
$\alpha$ \  es conocido como  par\'ametro de {\em  concentraci\'on} y el  vector \
$\bar{a}$ \ como {\em medida de base}.

Las caracter\'isticas de un vector de Dirichlet son:

\begin{caracteristicas}
%
Dominio de definici\'on &
$\X = \Simp{k-1}, \: k \in \Nset \setminus \{ 0 \; 1 \}$\\[2mm]
\hline
%
Par\'ametros & $a = \alpha \, \bar{a} \, \in \, \Rset_+^{* \, k}$ \ (forma) \ con
\ $\alpha \in \Rset_{0,+}$ \ (concentraci\'on) y \ $\bar{a} \in \Simp{k-1}$
(medida de base)\\[2mm]
\hline
%
Densidad de probabilidad~\footnote{La densidad de probabilidad es dada con
respeto a la medida de Lebesgue restricta al simplex $\Simp{k-1}$.\label{Foot:MP:DirichletDensidad}} & $\displaystyle
p_X(x) = \frac{\prod_{i=1}^k x_i^{a_i-1}}{B(a)}$\\[2mm]
\hline
%
Promedio & $\displaystyle m_X = \bar{a}$\\[2.5mm]
%\frac{a}{\sum_{i=1}^k a_i} \equiv \overline{a}$\\[2.5mm]
\hline
%
Covarianza~\footnote{Ver nota de pie~\ref{Foot:MP:MultinomialCov}.} &
$\displaystyle \Sigma_X = \frac{\diag\left( \bar{a} \right) - \bar{a}
\bar{a}^t}{1 + \alpha}$\\[2.5mm]
\hline
%
Generadora de momentos  &
$\displaystyle M_X(u) = \Phi_2^{(k)}( a , \alpha \, ; \, u )$ \ para \ $u \in
\Cset$\\[2mm]
\hline
%
Funci\'on caracter\'istica & $\displaystyle
\Phi_X(\omega) = \Phi_2^{(k)}( a , \alpha \, ; \, \imath \omega )$
\end{caracteristicas}

%$k$-variada~\footnote{$\Phi_2^{(k)}(a;b;z)     =     \sum_{m    \in     \Nset^k}
%  \frac{(a_1)_{(m_1)}     \ldots    (a_k)_{(m_k)}     \,     z_1^{m_1}    \ldots
%    z_k^{m_k}}{(b)_{(m_1+\cdots+m_k)} m_1!  \ldots m_k!}$  \ con \ $(x)_{(n)}$ \
%  s\'imbolo  de Pochhammer  usual o  factorial creciente,  $(x)_{(n)} =  x (x+1)
%  \ldots (x+n)$ \,  con la convenci\'ion \ $(x)_{(0)} = 1$.   De hecho, la forma

De nuevo, se puede considerar que el vector aleatorio es \ $(k-1)$-dimensional \
$\widetilde{X}     =    \begin{bmatrix}     \widetilde{X}_1    &     \cdots    &
  \widetilde{X}_{k-1}  \end{bmatrix}^t$  \ definido  sobre  el hipertriangulo  \
$\widetilde{\X}  = \Tri_{k-1}  = \left\{  \widetilde{x} \in  [0 \;  1]^{k-1} \tq
  \sum_{i=1}^{k-1}  \widetilde{x}_i \le  1 \right\}$,  proyecci\'on  del simplex
sobre el  hiperplano \ $x_k =  0$. As\'i, \ $\widetilde{X}$,  tiene una densidad
con   respeto   a   la  medida   de   Lebesgue   usual,   y   es  dada   por   \
$p_{\widetilde{X}}\left(   \widetilde{x}   \right)   =   \frac{\prod_{i=1}^{k-1}
  \widetilde{x}_i^{\,  a_i-1}  \, \left(  1  - \sum_{i=1}^{k-1}  \widetilde{x}_i
  \right)^{a_k-1}}{B(a)}$.      A      final,     se     notar\'a      que     \
$\Phi_{\widetilde{X}}\left(         \widetilde{\omega}         \right)         =
\Phi_X\left( \begin{bmatrix} \widetilde{\omega} & 0 \end{bmatrix}^t \right)$ \ y
\ $\Phi_X(u) =  e^{\omega_k} \Phi_{\widetilde{X}}\left( \begin{bmatrix} \omega_1
    - \omega_k  & \cdots &  \omega_{k-1} - \omega_k \end{bmatrix}^t  \right)$ (y
similarmente  para $G_X$  con respeto  a $G_{\widetilde{X}}$).   Notar tambi\'en
que, la  forma de la  funci\'on generadora de  momento viene directamente  de la
escritura de las series  de Taylor de $e^{u_i x_i}$ \ o  de la forma integral de
la funci\'on confluente hipergeom\'etrica~\cite{Phi88}.

Naturalmente,  $\Sigma_X \un  = 0$  \  as\'i que  de nuevo  \ $\Sigma_X  \notin
\Pos_k^+(\Rset)$, como consecuencia directa del  hecho que \ $X$ \ $k$-dimensional,
vive  sobre   \  $\Simp{k-1}$,  $(k-1)$-dimensional.  De   nuevo,  para  definir
asimetr\'ia y  curtosis habr\'ia que  considerar \ $\widetilde{X}$,  de promedio
$\begin{bmatrix}  \bar{a}_1  &  \cdots  & \bar{a}_{k-1}  \end{bmatrix}^t$  y  de
covarianza  el  bloque  $(k-1)  \times   (k-1)$  de  $\Sigma_X$,  que  es  ahora
invertible. $\gamma_{\widetilde{X}}$  \ y \ $\kappa_{\widetilde{X}}$  \ son bien
definidos. Las expresiones, demasiado pesadas, no son dadas ac\'a.

La figura Fig.~\ref{Fig:MP:Dirichlet} representa  el dominio de definici\'on del
vector (a) y la densidad de  probabildad con las marginales (ver m\'as adelante)
para $k = 3$ y dos ejemplos de par\'ametro $a$.
%
\begin{figure}[h!]
\begin{center} \begin{tikzpicture}%[scale=.8]
\shorthandoff{>}
%
\tikzset{declare function={
xplus(\x) = max(\x,0);
%ifthenelse(\x > 0 , \x , NaN);
}}
%}

% Simplex
\tdplotsetmaincoords{45}{65}
\begin{scope}[tdplot_main_coords,scale=.75]
%
% Dirichlet: \X = S_{k-1} y \widetilde{X}
\pgfmathsetmacro{\dx}{3};% scaling
%
\draw[->,>=stealth] (-.25,0,0)--({\dx+.5},0,0) node[below right,scale=.9]{$x_1$};
%\node at (\dx,0,0)[left,scale=.8]{$1$};
\draw (\dx,0,0)--(\dx,-.15,0) node[left,scale=.8]{$1$};
%
\draw[->,>=stealth] (0,-.25,0)--(0,{\dx+.5},0) node[right,scale=.9]{$x_2$};
%\node at (0,\dx,0)[below,scale=.8]{$1$};
\draw (0,\dx,0)--(.15,\dx,0) node[below,scale=.8]{$1$};
%
\draw[->,>=stealth] (0,0,-.25)--(0,0,{\dx+.5}) node[above,scale=.9]{$x_3$};
%\node at (0,0,\dx)[left,scale=.8]{$1$};
\draw (0,0,\dx)--(0,-.15,\dx) node[left,scale=.8]{$1$};
%
\node at (0,0,0)[below left,scale=.8]{$0$};
%
% tilde X
\filldraw[fill=black!50,opacity=.5] (0,0,0)--(\dx,0,0)--(0,\dx,0);
\draw[thick,color=black,dashed] (0,0,0)--(\dx,0,0)--(0,\dx,0)--(0,0,0);
\node at ({\dx/15},{\dx/20},0)[right,scale=.7]{$\Tri_2$};
%
% Simplex Delta_2
\filldraw[fill=black!75,opacity=.5] (\dx,0,0)--(0,\dx,0)--(0,0,\dx);
\draw[thick,color=black] (\dx,0,0)--(0,\dx,0)--(0,0,\dx)--(\dx,0,0);
\node at ({.05*\dx},{.05*\dx},{.8*\dx})[right,scale=.7]{$\Simp{2}$};
%
\end{scope}
%
%
% densidad (3,2,2)
\begin{scope}[xshift=4cm,yshift=-2cm,scale=.75]
%
\pgfmathsetmacro{\au}{3};% a1
\pgfmathsetmacro{\ad}{2};% a2
\pgfmathsetmacro{\at}{2};% a3
\pgfmathsetmacro{\B}{factorial(\au-1)*factorial(\ad-1)*factorial(\at-1)/factorial(\au+\ad+\at-1)};% normalizacion
\pgfmathsetmacro{\Bu}{factorial(\au-1)*factorial(\ad+\at-1)/factorial(\au+\ad+\at-1)};% normalizacion 1
\pgfmathsetmacro{\Bd}{factorial(\ad-1)*factorial(\au+\at-1)/factorial(\au+\ad+\at-1)};% normalizacion 2
\pgfmathsetmacro{\ma}{((\au-1)^(\au-1))*((\ad-1)^(\ad-1))*((\at-1)^(\at-1))/((\au+\ad+\at-3)^(\au+\ad+\at-3))/\B};
%
% Dirichlet & marginales
\begin{axis}[
    colormap = {whiteblack}{color(0cm)  = (white);color(1cm) = (black)},
    width=.5\textwidth,
    view={45}{65},
    enlargelimits=false,
    %grid=major,
    domain=0:1,
    y domain=0:1,
    %unbounded coords=jump, % para tener un dominio no cuadrado
    %filter point/.code={%
    %\pgfmathparse
    %{\pgfkeysvalueof{/data point/x} + \pgfkeysvalueof{/data point/y} > 1.0}%
    %  \ifpgfmathfloatcomparison
    %     \pgfkeyssetvalue{/data point/x}{nan}%
    %  \fi
    %},
    zmax={.8*\ma},
    color=black,
    samples=70,
    xlabel=$x_1$,
    ylabel=$x_2$,
    zlabel=$p_{\widetilde{X}}$,
]
%
% Dirichlet
\addplot3 [surf] {(x^(\au-1))*(y^(\ad-1))*(xplus(1-x-y)^(\at-1))/\B};
%
% Marginales
\addplot3 [domain=0:1,samples=50, samples y=0, thick, smooth, color=black] (x,1,{(x^(\au-1))*((1-x)^(\ad+\at-1))/\Bu});
\addplot3 [domain=0:1,samples=50, samples y=0, thick, smooth, color=black] (0,x,{(x^(\ad-1))*((1-x)^(\au+\at-1))/\Bd});
%
\node at (axis cs:.5,1,{1/(2^(\au+\ad+\at-2))/\Bu})[right]{$p_{X_1}$};
\node at (axis cs:0,.5,{1/(2^(\au+\ad+\at-2))/\Bd})[above]{$p_{X_2}$};
\end{axis}
\end{scope}
%
%
% densidad (3,2,2)
\begin{scope}[xshift=11cm,yshift=-2cm,scale=.75]
%
\pgfmathsetmacro{\au}{3};% a1
\pgfmathsetmacro{\ad}{1};% a2
\pgfmathsetmacro{\at}{2};% a3
\pgfmathsetmacro{\B}{factorial(\au-1)*factorial(\ad-1)*factorial(\at-1)/factorial(\au+\ad+\at-1)};% normalizacion
\pgfmathsetmacro{\Bu}{factorial(\au-1)*factorial(\ad+\at-1)/factorial(\au+\ad+\at-1)};% normalizacion 1
\pgfmathsetmacro{\Bd}{factorial(\ad-1)*factorial(\au+\at-1)/factorial(\au+\ad+\at-1)};% normalizacion 2
\pgfmathsetmacro{\ma}{((\au-1)^(\au-1))*((\ad-1)^(\ad-1))*((\at-1)^(\at-1))/((\au+\ad+\at-3)^(\au+\ad+\at-3))/\B};
%
\begin{axis}[
    colormap = {whiteblack}{color(0cm)  = (white);color(1cm) = (black)},
    width=.5\textwidth,
    view={45}{65},
    enlargelimits=false,
    %grid=major,
    domain=0:1,
    y domain=0:1,
    zmax={.65*\ma},
    color=black,
    samples=70,
    xlabel=$x_1$,
    ylabel=$x_2$,
    zlabel=$p_{\widetilde{X}}$,
]
%
% Dirichlet
\addplot3 [surf,opacity=.8] {(x^(\au-1))*(y^(\ad-1))*(xplus(1-x-y)^(\at-1))/\B};
%
% Marginales
\addplot3 [domain=0:1,samples=50, samples y=0, thick, smooth, color=black] (x,1,{(x^(\au-1))*((1-x)^(\ad+\at-1))/\Bu});
\addplot3 [domain=0:1,samples=50, samples y=0, thick, smooth, color=black] (0,x,{(x^(\ad-1))*((1-x)^(\au+\at-1))/\Bd});%
%
\node at (axis cs:.5,1,{1/(2^(\au+\ad+\at-2))/\Bu})[right]{$p_{X_1}$};
\node at (axis cs:0,.5,{1/(2^(\au+\ad+\at-2))/\Bd})[above]{$p_{X_2}$};
\end{axis}
\end{scope}
%
\node at (1.2,-3){(a)};
\node at (6.6,-3){(b)};
\node at (13.6,-3){(c)};
\end{tikzpicture} \end{center}
%
\leyenda{Ilustraci\'on  del dominio $\Simp{k-1}$  de definici\'on  de la  ley de
  Dirichlet   para  \   $k  =   3$   \  (grise   oscuro),  con   el  dominio   \
  $(k-1)$-dimensional   \  $\Tri_{k-1}$   \  del   vector  \   $\widetilde{X}  =
  \protect\begin{bmatrix}   X_1  &  X_2   \protect\end{bmatrix}^t$  \   ($X_3  =
  1-X_1-X_2$) \ (grise claro) (a), y densidad de probabilidad de $\widetilde{X}$
  \ con  las marginales \  $p_{X_1}, \: p_{X_2}$.   Los par\'ametros son \  $a =
  \protect\begin{bmatrix}  3 &  2  & 2  \protect\end{bmatrix}^t$  (b) y  \ $a  =
  \protect\begin{bmatrix} 3 & 1 & 2 \protect\end{bmatrix}^t$ (c).}
\label{Fig:MP:Dirichlet}
\end{figure}


Vectores  de  distribuci\'on  de  Dirichlet tienen  tambi\'en  unas  propiedades
notables, parecidas a las de la beta:
%
\begin{lema}[Reflexividad]\label{Lem:MP:ReflexividadDir}
%
  Sea  \ $X \,  \sim \,  \Dir(a), \:  a \in  \Rset_+^{* \,  k}$ \  y \  $\Pi \in
  \perm_k(\Rset)$ \ matriz \ de permutaci\'on. Entonces
  %
  \[
  \Pi X \, \sim \, \Dir\left( \Pi a \right)
  \]
  %
\end{lema}

%
\begin{proof}
  El resultado es inmediato por cambio  de variables $x \to \Pi x$, la Jacobiana
  siendo   $\Pi$,   de   valor   absoluto   determinente  igual   a   $1$   (ver
  secci\'on~\ref{Sec:MP:Transformacion}).
\end{proof}
%
Adem\'as, se muestra una stabilidad reemplazando dos componentes por su suma:
%
\begin{lema}[Stabilidad por agregaci\'on]\label{Lem:MP:StabSumaDir}
%
  Sea  \ $X =  \begin{bmatrix} X_1  & \cdots  & X_k  \end{bmatrix}^t \,  \sim \,
  \Dir(a),  \:  a =  \begin{bmatrix}  a_1 &  \cdots  &  a_k \end{bmatrix}^t  \in
  \Rset_+^{*  \,  k}$  \  y  \  $G^{(i,j)}$ \  matriz  de  agrupaci\'on  de  las
  $(i,j)$-\'esima componentes (ver notaciones). Entonces,
  %
  \[
  G^{(i,j)} X \, \sim \, \Dir\left( G^{(i,j)} a \right)  
  \]
  %
\end{lema}
%
\begin{proof}
  Se  puede probar  este resultado  a partir  de la  funci\'on caracter\'istica,
  usando      las      propiedades      de      la      funci\'on      confluent
  hipergeom\'etrica~\cite{SriKar85, Hum22, App25,  AppKam26, Erd37, Erd40}. Pero
  se  puede tambi\'en  tener un  enfoque m\'as  directo.  Del  lema precediente,
  notando  que  existen  matrices  de permutaci\'on~\footnote{$\Pi_k$  pone  las
    componentes $i$ \ e \ $j$ el  las posiciones $1$ y $2$, sin cambiar el orden
    de  las  siguientes;  $\Pi_{k-1}$  trazlada  la  primera  componente  en  la
    posici\'on $\min(i,j)$.}  \ $\Pi_k \in  \perm_k(\Rset)$ \ y \ $\Pi_{k-1} \in
  \perm_{k-1}(\Rset)$ \ tal que \ $G^{(i,j)} = \Pi_{k-1} \, G^{(1,2)} \, \Pi_k$,
  se puede concentrarse en el caso \ $(i,j) = (1,2)$. Sea el cambio de variables
  $g:    x    =    (x_1,\ldots,x_k)    \mapsto   u    =    (u_1,\ldots,u_k)    =
  (x_1,x_1+x_2,x_3,\ldots,x_k)$.        Entonces       \      $g^{-1}(u)       =
  (u_1,u_2-u_1,u_3,\ldots,u_k)$ \ es de determinente de matriz Jacobiana igual a
  \ $1$ \ dando para $U = g(X)$ \ la densidad
  %
  \[
  p_U(u)  = \frac{u_1^{a_1-1}  \left(  u_2 -  u_1 \right)^{a_2-1}  \prod_{i=3}^k
    u_i^{a_i-1}}{B(a)}
  \]
  %
  sobre $g\left(  \Simp{k-1} \right)$. Para $u_2  \in [0 \; 1]$  \ tenemos $u_1
  \in [  0 \;  u_2]$ \ as\'i  que, por  marginalizaci\'on en $u_1$  obtenemos la
  densidad
  %
  \begin{eqnarray*}
  p_{G^{(1,2)} X}(u_2,\ldots,u_k) & = & \frac{\prod_{i=3}^k u_i^{a_i-1}}{B(a)}
  \int_0^{u_2} u_1^{a_1-1} \left( u_2 - u_1 \right)^{a_2-1} \, du_1\\[2mm]
  %
  & = & \frac{\prod_{i=3}^k u_i^{a_i-1}}{B(a)} \, u_2^{a_1+a_2-1} \int_0^1
  v_1^{a_1-1} \left( 1 - v_1 \right)^{a_2-1} \, dv_1
  \end{eqnarray*}
  %
  con el cambio de variables $u_1 = u_2 v_1$. Se cierra la prueba notando que la
  integral   vale  \  $B(a_1,a_2)$   \  y   que  \   $\frac{B(a_1,a_2)}{B(a)}  =
  \frac{1}{B\left( G^{(1,2)} a \right)}$.
\end{proof}

De este lema, aplicado de manera recursiva, se obtiene en corolario siguiente:
%
\begin{corolario}
\label{Cor:MP:MarginalDirichletBeta}
%
  Sea  \ $X  \,  \sim \,  \Dir(a)$, entonces  \  $\displaystyle X_i  \, \sim  \,
  \beta\left( a_i \, , \, \alpha-a_i \right)$.
\end{corolario}

Naturalmente, la ley de Dirichelt siendo  una extensi\'on de la ley beta, existe
tambi\'en un v\'inculo entre esta ley y variables de distribuci\'on gamma:
%
\begin{lema}[V\'inculo con la ley gamma]
\label{Lem:MP:VinculoDirichletGamma}
%
Sea \ $X$  \ vector $k$-dimensional de componentes \ $X_i  \, \sim \, \G(a_i,c),
\:  i  = 1,  \ldots  , k$  \  independientes  \ y  \  $a$  vector de  componente
$i$-\'esima \ $a_i$. Entonces
  %
  \[
  \frac{X}{\sum_{i=1}^k X_i} \, \sim \, \Dir(a)
  \]
  %
  (independientemente  de $c$).   Adem\'as, $\frac{X}{\sum_{i=1}^k  X_i}$ \  y \
  $\sum_{i=1}^k X_i$ \ son independientes.
\end{lema}
%
\begin{proof}
  La    prueba   sigue    exactamente   los    mismos   pasos    que    la   del
  lema~\ref{Lem:MP:VinculoBetaGamma} \ trabajando con \ $\widetilde{X}$.
\end{proof}

Naturalmente,  la  distribuci\'on de  Dirichlet,  extensi\'on  de  la ley  beta,
aparece entre  otros en problema  de inferencia bayesiana como  distribuci\'on a
priori  conjugado  del  par\'ametro  $p$  de  la  ley  multinomial~\cite{Rob07},
extensi\'on de la ley binomial.

%\SZ{
%Polya urn schemes (ver Ash entre otros) , Chinese restaurant
%}

\

La distribuci\'on de Dirichlet se generaliza al caso matriz-variada $X$ definido
sobre $\P_{d,k}(\Rset)$,  conjunto de las  $k$-uplas de matrices  de $\Pos_d^+(\Rset)$
cumpliando la relaci\'on  de completud (ver notaciones); se denota  \ $X \, \sim
\, \Dir_d(a)$  \ donde \  $a \in \left(  \frac{d-1}{2} \; +\infty  \right)^k$ la
densidad  est dada por  $\displaystyle p_X(x)  = \frac{\prod_{i=1}^k  \left| x_i
  \right|^{a_i-\frac{d+1}{2}}}{B_d(a)}$.   Se  refiera a~\cite[Cap.~6]{GupNag99}
para tener m\'as detalles.

% --------------------------------- Student-t
\subsubseccion{Distribuci\'on Student-$t$ multivariada}
\label{Sssec:MP:Student}

En   el  caso   escalar,  esta   ley   fue  introducida   inicialmente  por   F.
R. Helmert~\cite{Hel75, Hel76, She95}  y J.  L\"uroth~\cite{Lur76, Pfa96}.  Pero
es m\'as  conocida por su  introducci\'on por William  Sealy Gosset~\footnote{De
  hecho,  Gosset  fue  un  estudiante  trabajando en  la  f\'abrica  de  cerveza
  irlandesa  Guiness  sobre estad\'istica  relacionada  a  la  qui\'imica de  la
  cerveza.   A pesar  que hay  varias  explicaciones sobre  el hecho  de que  se
  public\'o este trabajo bajo el nombre ``Student''. Unas es que fue para que no
  se  sabe que  la f\'abrica  estaba  trabajando sobre  estas estadisticas  para
  estudiar  la calidad  de  la cerveza~\cite{Wen16}.\label{Foot:MP:Student}}  en
1908,  trabajando  sobre variables  centradas  normalizadas  por  el promedio  y
varianza empiricos~\cite{Stu08}.   Fue estudiada entre  otros intensivamente por
el  famoso matematico  R. Fisher~\cite{Fis25}.   En la  literatura, esta  ley es
conocida bajo  los nombres {\em  Student}, {\em Student-$t$} o  simplemente {\em
  $t$-distribuci\'on} o  a\'un bajo el nombre  {\em Pearson tipo IV}  en el caso
escalar  y {\em  Pearson tipo  VII}  (para $\frac{\nu+d}{2}$  entero; ver  m\'as
abajo), debido  a la  familia de Pearson~\cite{Pea95,  JohKot95:v1, JohKot95:v1,
  KotBal00, FanKot90}.  Esta distribuci\'on aparece como a  priori conjugado del
promedio de una gausiana en inferencia bayesiana~\cite{Rob07, KotNad04}.

Se denota con \ $X \sim t_\nu(m,\Sigma)$  \ con \ $m \in \Rset^d$, \ $\Sigma \in
P_d^+(\Rset)$ \ conjunto  de las matrices de \  $\M_{d,d}(\Rset)$ \ s\'imetricas
definidas positivas. $m$ \ es llamado  {\em par\'ametro de posici\'on} (no es la
media   que  puede  no   existir),  \   $\Sigma$  \   es  llamada   {\em  matriz
  caracter\'istica} (no es [proporcional a]  la covarianza que puede no existir)
y \ $\nu > 0$ \ llamado  {\em grados de libertad}.  Las caracter\'isticas de una
Student-$t$ son las siguientes:
%
\begin{caracteristicas}
%
Dominio de definici\'on & $\X = \Rset^d$\\[2mm]
\hline
%
Par\'ametro & $\nu \in \Rset_+^*$ \ (grados de libertad), \ $m \in \Rset^d$ \
(posici\'on), \ $\Sigma \in P_d^+(\Rset)$ \ (matriz caracer\'istica)\\[2mm]
\hline
%
Densidad de probabilidad & $\displaystyle p_X(x) = \frac{\Gamma\left(
\frac{\nu+d}{2} \right)}{\pi^{\frac{d}{2}} \nu^{\frac{d}{2}} \Gamma\left(
\frac{\nu}{2} \right) \, \left| \Sigma \right|^{\frac12}} \, \left( 1 +
\frac{(x-m)^t \Sigma^{-1} (x-m)}{\nu} \right)^{- \, \frac{\nu+d}{2}}$\\[2mm]
\hline
%
Promedio & $\displaystyle m_X = m$ \ si \ $\nu > 1$; \ no
existe si no~\footnote{De manera general, esta ley admite momentos de orden \ $k$ \
si y solamente si \ $\nu > k$.\label{Foot:MP:ExistenciaMomentosStudent}}.\\[2.5mm]
\hline
%
Covarianza~\footnote{Fijense de que $\Sigma$ no es la covarianza, pero es
proporcional a la covarianza\ldots cuando existe. Se podr\'ia imaginar
renormalizar la ley tal que \ $\Sigma_X$ \ y \ $\Sigma$ \ coinciden, pero no
ser\'ia posible en el caso \ $\nu \le 2$.} & $\displaystyle \Sigma_X =
\frac{\nu}{\nu-2} \, \Sigma$ \ si \ $\nu > 2$; \ no existe si
no~\footref{Foot:MP:ExistenciaMomentosStudent}.\\[2.5mm]
\hline
%
\modif{Asimetr\'ia} & $\displaystyle \gamma_X = 0$ \ si \ $\nu > 3$; \ no existe
si no~\footref{Foot:MP:ExistenciaMomentosStudent}.\\[2mm]
\hline
%
Curtosis por exceso & \modif{$\displaystyle \widebar{\kappa}_X = \frac{2}{\nu-4}
\sum_{i,j=1}^d \Big( \! \left(
    \un_i \un_i^t \right) \otimes \left(  \un_j \un_j^t \right) +  \left( \un_i
    \un_j^t \right) \otimes \left( \un_i  \un_j^t \right) + \left( \un_i \un_j^t
  \right) \otimes \left( \un_j \un_i^t \right) \! \Big)$}\newline si \ $\nu >
4$; \ no existe si no~\footref{Foot:MP:ExistenciaMomentosStudent}.\\[2mm]
\hline
%
Funci\'on caracter\'istica~\footnote{Se muestra sencillamente que la funci\'on
generatriz de momentos puede existir si y solamente si \ $\real{u} = 0$. La
funci\'on genetratriz de momentos restricta al producto cartesiano de bandas \
$\real{u} = 0$ \ es nada m\'as que la funci\'on caracter\'istica. Adem\'as, esta
funci\'on fue calculdada, especialmente en el caso multivariado, relativamente
recientemente~\cite{Sut86, Hur95, KibJoa06, SonPar14}.} & $\displaystyle
\Phi_X(\omega) = \frac{\nu^{\frac{\nu}{4}}}{2^{\frac{\nu}{2}-1} \Gamma\left(
\frac{\nu}{2} \right)} \, e^{\imath \omega^t m} \, \left( \omega^t \Sigma \omega
\right)^{\frac{\nu}{4}} K_{\frac{\nu}{2}}\left( \sqrt{\nu \, \omega^t \Sigma
\omega} \right)$
\end{caracteristicas}

Nota: nuevamente se puede escribir $X \, \egald \, \Sigma^{\frac12} S + m$ \ con
\ $S \, \sim \, t_\nu(0,I)$ \  donde \ $S$ \ es dicha {\em Student-$t$ estandar}
y  las caracter\'isticas  de \  $X$  \ son  v\'inculadas a  las  de \  $S$ \  (y
vice-versa) por transformaci\'on lineal (ver secciones anteriores).

La densidad de probabilidad Student-$t$ estandar y la funci\'on de repartici\'on
en el caso escalar  son representadas en la figura Fig.~\ref{Fig:MP:Student}-(a)
y    (b)   y    una   densidad    en   un    contexto    bi-dimensional   figura
Fig.~\ref{Fig:MP:Student}(c).
%
\begin{figure}[h!]
\begin{center} \begin{tikzpicture}
\shorthandoff{>}
%
% Para el caso univariado
\pgfmathsetmacro{\sx}{.43};% x-scaling
\pgfmathsetmacro{\mu}{0};% para tomar los grados de libertad impar; 0 => Cauchy
\pgfmathsetmacro{\md}{1};
\pgfmathsetmacro{\mt}{3};
%\pgfmathsetmacro{\mq}{3};
%
%
% para el caso bi-variado
\pgfmathsetmacro{\mdd}{0};%
\pgfmathsetmacro{\nu}{2*\mdd+1};% grados de libertad
\pgfmathsetmacro{\a}{1/3};% x-scaling
\pgfmathsetmacro{\t}{30};% angulo de rotacion
\pgfmathsetmacro{\c}{cos(\t)};% coseno
\pgfmathsetmacro{\s}{sin(\t)};% seno
\pgfmathsetmacro{\su}{sqrt(\c^2+(\a*\s)^2)};% ecart-type 1
\pgfmathsetmacro{\sd}{sqrt(\s^2+(\a*\c)^2)};% ecart-type 2
\pgfmathsetmacro{\dx}{3};% dominio x del plot -dx:dx
\pgfmathsetmacro{\dy}{2.5};% dominio y del plot -dy:dy
%
%
% Approximacion de la funcion Gamma
%\tikzset{declare function={gamma(\z)=
%(2.506628274631*sqrt(1/\z) + 0.20888568*(1/\z)^(1.5) + 0.00870357*(1/\z)^(2.5) -
%(174.2106599*(1/\z)^(3.5))/25920 - (715.6423511*(1/\z)^(4.5))/1244160)*exp((-ln(1/\z)-1)*\z);}}
%
% Approximation de la cdf gaussienne
\tikzmath{function normcdf(\x) {return 1/(1 + exp(-0.07056*(\x)^3 - 1.5976*(\x)));};};
%
% coefficiente binomial, para no tener factoriales muy grandes
\tikzmath{function binocoef(\m,\k) {if \k == 0 then {return 1;} else {return ((\m-\k+1)/\k)*binocoef(\m,\k-1);};};};
%
% coefficient que aparece en la pdf y cdf (ver doubling formula GraRyz 8.335-5 con x = m+1/2)
% y coefficiente de normalizacion
%\tikzset{declare function={
\tikzmath{function coefstud(\m) {return (4^\m)/(pi*sqrt(2*\m+1)*binocoef(2*\m,\m));};}
%
%
% cdf Student que se calcula recursivamente para nu = 2 m + 1, m entero
\tikzmath{function studcdfS(\x,\k) {
    if \k == 0 then {return .5+(atan(\x))/180;}
    else {return studcdfS(\x,\k-1)+((4^\k)*(\x)/(2*pi*\k*binocoef(2*\k,\k)))/((1+((\x)^2))^\k);};
};};
% Calculo de
%  - x maximo del plot para tener pdf a 7% del max
%  - la pdf Student para nu = 2 m + 1, m entero
%  - la cdf Student para nu = 2 m + 1, m entero
%\tikzset{declare function={
\tikzmath{function maxplotpdf(\m) {return sqrt((2*\m+1)*((.03^(-1/(\m+1)))-1));};};% x maximo del plot para tener pdf a 3% del max
\tikzmath{function studpdf(\x,\m) {return coefstud(\m)*((1/(1+((\x)^2)/(2*\m+1)))^(\m+1));};};% pdf Student
\tikzmath{function studcdf(\x,\m) {return studcdfS(\x/(sqrt(2*\m+1)),\m);};};% pdf Student
%}}
%
%
%
% mismas escalas x-max para cada ejemplo
\pgfmathsetmacro{\mx}{max(maxplotpdf(\mu),maxplotpdf(\md),maxplotpdf(\mt))};

% maximo de las marginales del caso 2D
\pgfmathsetmacro{\ma}{coefstud(\mdd)/min(\su,\sd)};
%
% densidad
\begin{scope}[scale=.9]
%
\pgfmathsetmacro{\sy}{2.75*sqrt(2*pi)};% y-scaling 
\draw[>=stealth,->] ({-\sx*\mx-.1},0)--({\sx*\mx+.25},0) node[right]{\small $x$};
\draw[>=stealth,->] (0,-.15)--(0,3) node[above]{\small $p_X$};
%
\draw[thick,domain=-\mx:\mx,samples=50,smooth] plot ({\x*\sx},{\sy*studpdf(\x,\mu)});
\draw[thick,dashed,domain=-\mx:\mx,samples=50,smooth] plot ({\x*\sx},{\sy*studpdf(\x,\md)});
\draw[thick,dotted,domain=-\mx:\mx,samples=50,smooth] plot ({\x*\sx},{\sy*studpdf(\x,\mt)});
\draw[thin,domain=-\mx:\mx,samples=50,smooth] plot ({\x*\sx},{\sy*exp(-.5*((\x)^2))/sqrt(2*pi)});
%
\draw (0,{\sy/sqrt(2*pi)})--(-.2,{\sy/sqrt(2*pi)}) node[left,scale=.7]{$\displaystyle \frac1{\sqrt{2 \pi}}$};
\draw (0,0)--(0,-.1) node[below,scale=.7]{$0$};
\pgfmathsetmacro{\lm}{2*floor(\mx/2)};
\foreach \m in {2,4,...,\lm} {
\draw ({-\m*\sx},0)--({-\m*\sx},-.1) node[below,scale=.7]{$-\m$};
\draw ({\m*\sx},0)--({\m*\sx},-.1) node[below,scale=.7]{$\m$};
}
%
\node at (0,-1) [scale=.9]{(a)};
\end{scope}
%
%
% reparticion
\begin{scope}[xshift=5.75cm,scale=.9]
%
\pgfmathsetmacro{\extx}{1.1};% extnsion del dominio para la cdf (que se vea mejor) 
\pgfmathsetmacro{\sy}{2.75};% y-scaling 
%
\draw[>=stealth,->] ({-\sx*\mx*\extx-.1},0)--({\sx*\mx*\extx+.25},0) node[right]{\small $x$};
\draw[>=stealth,->] (0,-.15)--(0,{\sy+.25}) node[above]{\small $F_X$};
%
% cumulativa
%
\draw[thick,domain={-\mx*\extx}:{\mx*\extx},samples=50,smooth] plot ({\x*\sx},{\sy*studcdf(\x,\mu)});
\draw[thick,dashed,domain={-\mx*\extx}:{\mx*\extx},samples=50,smooth] plot ({\x*\sx},{\sy*studcdf(\x,\md)});
\draw[thick,dotted,domain={-\mx*\extx}:{\mx*\extx},samples=50,smooth] plot ({\x*\sx},{\sy*studcdf(\x,\mt)});
\draw[thin,domain={max(-\mx*\extx,-3.5)}:{\mx*\extx},samples=50,smooth]  plot ({\x*\sx},{\sy*normcdf(\x)});
%
\draw (0,0)--(0,-.1) node[below,scale=.7]{$0$};
\draw (0,\sy)--(-.1,\sy) node[left,scale=.7]{$1$};
\pgfmathsetmacro{\lm}{2*floor(\mx*\extx/2)};
\foreach \m in {2,4,...,\lm} {
\draw ({-\m*\sx},0)--({-\m*\sx},-.1) node[below,scale=.7]{$-\m$};
\draw ({\m*\sx},0)--({\m*\sx},-.1) node[below,scale=.7]{$\m$};
}
%
\node at (0,-1) [scale=.9]{(b)};
\end{scope}
%
%
% densidad 2D
\begin{scope}[xshift=9.5cm,yshift=-2.5mm,scale=.7]
%
\begin{axis}[
    colormap = {whiteblack}{color(0cm)  = (white);color(1cm) = (black)},
    width=.45\textwidth,
    view={45}{65},
    enlargelimits=false,
    %grid=major,
    domain=-\dx:\dx,
    y domain=-\dy:\dy,
    color=black,
    samples=80,
    xlabel=$x_1$,
    ylabel=$x_2$,
    zlabel=$p_X$,
    zmax={1.05*\ma},
]
%
% Student-t 2D
\addplot3 [surf] {1/(2*pi*\a*((1+((\c*x+\s*y)^2+((-\s*x+\c*y)/\a)^2)/(2*\mdd+1))^(1.5+\mdd)))};
%
% Marginales
\pgfmathsetmacro{\cproj}{coefstud(\mdd)};
\addplot3 [domain=-\dx:\dx,samples=50, samples y=0, thick, smooth, color=black]
(x,\dy,{\cproj*((1+((x/\su)^2)/(2*\mdd+1))^(-\mdd-1))/\su});
\addplot3 [domain=-\dy:\dy,samples=50, samples y=0, thick, smooth, color=black]
(-\dx,x,{\cproj*((1+((x/\sd)^2)/(2*\mdd+1))^(-\mdd-1))/\sd});
%\addplot3 [domain=-\dx:\dx,samples=51, thick, smooth, color=black] (x,\dy,{studpdf(x/\su,\mdd)});
%\addplot3 [domain=-\dy:\dy,samples=51, samples y=0, thick, smooth, color=black] (-\dx,x,{studpdf(x/\sd,\mdd)/\sd});
%
\node at (axis cs:{\dx/5},\dy,{\cproj*((1+((\dx/5/\su)^2)/(2*\mdd+1))^(-\mdd-1))/\su})[above right]{$p_{X_1}$};
\node at (axis cs:-\dx,{\dy/5},{\cproj*((1+((\dy/5/\sd)^2)/(2*\mdd+1))^(-\mdd-1))/\sd})[above right]{$p_{X_2}$};
%
\end{axis}
%
\node at ({\dx},-1) [scale=.9]{(c)};
\end{scope}
\end{tikzpicture} \end{center}
% 
\leyenda{Ilustraci\'on  de  una  densidad  de probabilidad  Student-$t$  escalar
  estandar (a),  y la funci\'on  de repartici\'on asociada  (b) con \ $\nu  = 1$
  (linea llena), \ $\nu = 3$ (linea  guionada), \ $\nu = 7$ (linea punteada) \ y
  \ $\nu \to +\infty$ (linea llena fina; ver m\'as adelante) grados de libertad,
  as\'i que una densidad de probabilidad Student-$t$ bi-dimensional con \ $\nu =
  1$ \  grado de libertad,  centrada, y de  matriz caracter\'istica \  $\Sigma =
  R(\theta) \Delta^2  R(\theta)^t$ \ con \  $R(\theta) = \protect\begin{bmatrix}
    \cos\theta    &     -    \sin\theta\\[2mm]    \sin\theta     &    \cos\theta
    \protect\end{bmatrix}$   \   matriz   de    rotaci\'on   y   \   $\Delta   =
  \diag\left(\protect\begin{bmatrix}  1   &  a\protect\end{bmatrix}  \right)$  \
  matriz  de   cambio  de  escala,   y  sus  marginales   \  $X_1  \,   \sim  \,
  t_\nu\left(0,\cos^2\theta +  a^2 \sin^2\theta \right)$ \  y \ $X_2  \, \sim \,
  t_\nu\left(0,\sin^2\theta   +   a^2  \cos^2\theta   \right)$   \  (ver   m\'as
  adelante). En la figura, $a = \frac13$ \ y \ $\theta = \frac{\pi}{6}$.}
\label{Fig:MP:Student}
\end{figure}

Nota: el caso  \ $\nu = 1$ \  es conocido como distribuci\'on de  {\em Cauchy} o
{\em Cauchy  Breit-Wigner}~\cite{SamTaq94, toto,  titi}.  Es un  caso particular
tambi\'en  de distrubuci\'on  $\alpha$-estables~\cite{SamTaq94}.  En particular,
una  combinaci\'on  lineal  de  variables  de  Cauchy  independientes  queda  de
Cauchy. Pero,  no viola  el teorema del  l\'imite central  del hecho de  que una
variable de Cauchy no admite covarianza.

Contrariamente al caso gaussiano, de la forma de la densidad de probabilidad, es
claro que si la matriz \ $\Sigma$ \ es diagonal, la densidad no factoriza, as\'i
que  las componentes  del vector  no son  independientes.  Este  ejemplo muestra
claramente que  la reciproca del lema~\ref{Lem:MP:IndependenciaCov}  es falsa en
general.

Sin embargo, las distribuciones Student-$t$ tienen varias propiedades notables.

\begin{lema}[Stabilidad por transformaci\'on lineal]
\label{Lem:MP:StabilidadLineal}
%
  Sea \ $X \, \sim \,  t_\nu(m,\Sigma)$, \ $A$ \ matriz de \ $\M_{d',d}(\Rset))$
  \ con \ $d' \le d$, y de rango lleno y \ $b \in \Rset^{d'}$. Entonces
  %
  \[
  A X + b\, \sim \, t_\nu( A m + b , A \Sigma A^t)
  \]
  %
  En particular los componentes de \ $X$ \ son student-$t$,
  %
  \[
  X_i \, \sim \, t_\nu(m_i , \Sigma_{i,i} )
  \]
\end{lema}
\begin{proof}
  La prueba es inmediata usando  la funci\'on caracter\'istica y sus propiedades
  por  transformaci\'on lineal.  La condici\'on  sobre \  $A$ \  es  necesaria y
  suficiente para que \ $A \Sigma A^t \in P_{d'}^+(\Rset)$.
\end{proof}

\begin{lema}[V\'inculo con las distribuciones Gamma y Gausiana (mezcla Gaussiana de escala)]
\label{Lem:MP:MezclaGaussianaEscalaStudent}
%
  Sea \ $V \sim \G\left( \frac{\nu}{2} \, ,  \, \frac{\nu}{2} \right)$ \ y \ $G \, \sim \,
  \N(0,I)$ \ independientes. Entonces
  %
  \[
  \frac{G}{\sqrt{V}} \, \sim \, t_\nu( 0 , I )
  \]
  %
  Dicho de  otra manera, se  puede escribir \  $X \, \sim \,  t_\nu(m,\Sigma)$ \
  esticasticamente   bajo   la    forme   \   $X   \egald   \sqrt{\frac{\nu}{V}}
  \Sigma^{\frac12} G + m $ \ donde  \ $\egald$ \ significa que la igualdad es en
  distribuci\'on.
\end{lema}
\begin{proof}
  Lo  m\'as  simple es  de  salir  de la  formula  de  probabilidad total  vista
  pagina~\pageref{:MP:}, notando que condicionalmente a \ $V=v$ \ la variable es
  gausiana de covarianza $\frac{1}{\sqrt{v}} I$,
%
\begin{eqnarray*}
p_X(x) & = & \int_\Rset p_{X|V=v}(x) \, p_V(v) \, dv\\[2mm]
%
& \propto & \int_0^{+\infty} v^{\frac{d}{2}} e^{-\frac{v}{2} x^t x}
v^{\frac{\nu}{2}-1} e^{-\frac{\nu}{2} v} \, dv\\[2mm]
%
& \propto & \left( 1 + \frac{x^t x}{\nu} \right)^{- \frac{d+\nu}{2}}
\int_0^{+\infty} u^{\frac{d+\nu}{2}-1} e^{-u} \, du\\[2mm]
%
& \propto & \left( 1 + \frac{x^t x}{\nu} \right)^{- \frac{d+\nu}{2}}
\end{eqnarray*}
%
con  \ $\propto$  \ significando  ``proporcional a''  (el coeficiente  es  lo de
normalizaci\'on) y  el cambio de  variables $v  = \frac{2 \,  u}{\nu + x^t  x} =
\frac{\frac{2}{\nu}}{1 + \frac{x^t x}{\nu}} \, u$.
\end{proof}
%
Nota: este  lema permite tambi\'en  probar el lema~\ref{Lem:MP:StabilidadLineal}
escribiendo \ $A X + b \egald  \sqrt{\frac{\nu}{V}} A \Sigma^{\frac12} G + A m +
b$.

\begin{lema}[L\'imite Gausiana]
\label{Lem:MP:LimiteGaussiana}
%
  Sea \ $X_\nu \, \sim \, t_\nu(m,\Sigma)$ \ vector Student-$t$ parametizado por
  \ $\nu$ \ sus grados de libertad. Entonces
  %
  \[
  X_\nu \, \limitd{\nu \to \infty} \, = \, X \, \sim \, \N(m,\Sigma)
  \]
  %
  con \ $\displaystyle \limitd{}$ \ l\'imite es en distribuci\'on.
\end{lema}
\begin{proof}
  La prueba  es inmediata tomando el  logaritmo de la  densidad de probabilidad,
  usando      la     formula      de     Stirling~\footnote{Ver      nota     de
    pie~\footref{Foot:MP:Stirling}} para  \ $\log\Gamma(z) = \left(  z - \frac12
  \right)  \log  z  -  z  +  \frac12   \log(2  \pi)  +  o(1)$  \  en  \  $z  \to
  +\infty$~\cite{Sti30, AbrSte70, GraRyz15} \ y \ $-\frac{d+\nu}{2} \log\left( 1
    +  \frac{(x-m)^t \Sigma^{-1} (x-m)}{\nu}  \right) =  -\frac{d+\nu}{2} \left(
    \frac{(x-m)^t \Sigma^{-1} (x-m)}{\nu} + o\left( \nu^{-1} \right) \right) = -
  \frac{(x-m)^t \Sigma^{-1} (x-m)}{2} + o(1)$.
\end{proof}

Las   variables   Student-t   tienen   varias   representaciones   estocasticas,
relacionadas a la gausiana~\cite{FanKot90, And03, KotNad04, AndKau65}:
% ej. KotNad p. 7 para la secunda
%
\begin{lema}[Relaci\'on con la distribuci\'on Gamma]\label{Lem:MP:StudentGamma}
%
  Sea \ $G \,  \sim \, \G\left( \frac{\nu}{2} , \frac{\nu}{2} \right)$  \ y \ $Y
  \,  \sim  \, \N(0,I)$  \  con  $\nu >  0$  \  e \  $Y$  \  independiente de  \
  $G$. Entonces, para $\Sigma \in P_d^+(\Rset)$ y $m \in \Rset^d$,
  %
  \[
  \frac{\Sigma^{\frac12} Y}{\sqrt{G}} + m  \, \sim \, t_\nu(m,\Sigma)
  \]
  %
\end{lema}
\begin{proof}
  Sea \ $X = \frac{Y}{\sqrt{G}}$. De la nota siguiendo la tabla de caracter\'isticas
  es  necesario  y suficiente  probar  que $X  \sim  t_\nu(0,I)$.  Ahora, de  la
  independencia tenemos
  %
  \[
  p_{X|G=g}(x)  = (2  \pi)^{-\frac{d}{2}}  g^{\frac{d}{2}} e^{-  \frac{x^t x g}{2}}
  \]
  %
  Entonces, multiplicando \ $p_{X|G=g}$ \ por \ $p_G$ \ y por marginalizaci\'on,
  obtenemos
  %
  \begin{eqnarray*}
  p_X(x) & = & \frac{\nu^{\frac{\nu}{2}}}{2^{\frac{\nu+d}{2}} \pi^{\frac{d}{2}}
  \Gamma\left( \frac{\nu}{2} \right)} \, \int_{\Rset_+} g^{\frac{\nu+d}{2}-1} \,
  e^{- \frac{x^t x + \nu}{2} \, g} \, dg\\[2mm]
  %
  & = & \frac{\nu^{\frac{\nu}{2}} \left( \nu + x^t x \right)^{-
  \frac{\nu+d}{2}}}{\pi^{\frac{d}{2}} \Gamma\left( \frac{\nu}{2} \right)} \,
  \int_{\Rset_+} u^{\frac{\nu+d}{2}-1} \, e^{- u} \, du\\[2mm]
  %
  & = & \frac{\Gamma\left( \frac{\nu+d}{2} \right)}{(\pi \nu)^{\frac{d}{2}}
  \Gamma\left( \frac{\nu}{2} \right)} \, \left( 1 + \frac{x^t x}{\nu} \right)^{-
  \frac{\nu+d}{2}}
  \end{eqnarray*}
 %
  La secunda linea viene del cambio de variables \ $u = \frac{x^t x + \nu}{2} \,
  g$  \  y la  tercera  reconociendo  en la  integral  la  funci\'on Gamma  (ver
  notaciones).
\end{proof}
%
\begin{lema}[Relaci\'on con la distribuci\'on de Wishart]\label{Lem:MP:StudentWishart}
%
  Sea \ $W \, \sim \, \W( \Sigma^{-1} \, , \, \nu+d-1)$ \ $d \times d$ \ Wishart
  con \ $\Sigma \in P_d^+(\Rset)$, \ $Y \,  \sim \, \N(0,\nu I)$ \ con $\nu > 0$
  \ e \ $Y$ \ independiente de \  $W$. Entonces, para \ 
  %$R \in P_d^+(\Rset)$ \ y
  \ $m \in \Rset^d$,
  %
  \[
  W^{-\frac12} Y + m \, \sim \, t_\nu\left( m , \Sigma \right)
  \]
  %
\end{lema}
\begin{proof}
  Sea \ $X = W^{-\frac12} Y$. De la nota siguiendo la tabla de caracter\'isticas
  es  necesario  y suficiente  probar  que $X  \sim  t_\nu(0,\Sigma)$.  Ahora, de  la
  independencia tenemos
  %
  \[
  p_{X|W=w}(x)  = (2  \pi \nu)^{-\frac{d}{2}}  |w|^{\frac12} e^{-  \frac{x^t w  x}{2
      \nu}}
  \]
  %
  Denotamos  por \  $D =  \left\{ w_{ij},  \: 1  \le j  \le i  \le d  \tq  w \in
    P_d^+(\Rset) \right\}$ \ y, por abuso de  escritura, \ $dv = \prod_{ 1 \le j
    \le i \le d} dw_{ij}$.  Entonces,  multiplicando \ $p_{X|W=w}$ \ por \ $p_W$
  \ y por marginalizaci\'on, obtenemos
  % ($\propto$ significa ``proporcional  a'', i.e., olvidando el coefficiente de
  % normalizaci\'on)
  %
  \begin{eqnarray*}
  p_X(x) & = & \int_D \frac{|w|^{\frac{\nu-1}{2}} e^{- \frac{x^t w x}{2 \nu} -
  \frac12 \Tr\left( \Sigma w \right)}}{2^{\frac{d (\nu+d)}{2}} (\pi
  \nu)^{\frac{d}{2}} \left| \Sigma^{-1} \right|^{\frac{\nu+d-1}{2}} \Gamma_d \left(
  \frac{\nu+d-1}{2} \right)} \, dw\\[2mm]
  %
  & = & \frac{\Gamma\left( \frac{\nu+d}{2} \right)}{(\pi \nu)^{\frac{d}{2}}
  \Gamma\left( \frac{\nu}{2} \right)} \left| \Sigma + \frac{x x^t}{\nu}
  \right|^{-\frac{\nu+d}{2}} \left| \Sigma
  \right|^{\frac{\nu+d-1}{2}} \: \int_D \frac{|w|^{\frac{\nu+d-d-1}{2}} e^{-
  \frac12 \Tr\left( \left[ \Sigma + \frac{x x^t}{\nu} \right] w \right)}}{2^{\frac{d
  (\nu+d)}{2}} \left| \left( \Sigma + \frac{x x^t}{\nu} \right)^{-1}
  \right|^{\frac{\nu+d}{2}} \Gamma_d \left( \frac{\nu+d}{2} \right)} \, dw\\[2mm]
  %
  & = & \frac{\Gamma\left( \frac{\nu+d}{2} \right)}{(\pi \nu)^{\frac{d}{2}}
  \Gamma\left( \frac{\nu}{2} \right) \left| \Sigma \right|^{\frac12}} \: \left( 1
  + \frac{x^t \Sigma^{-1} x}{\nu} \right)^{-\frac{\nu+d}{2}} \, \int_D
  \frac{|w|^{\frac{\nu+d-d-1}{2}} e^{- \frac12 \Tr\left( \left[ I + \frac{x
  x^t}{\nu} \right] w \right)}}{2^{\frac{d (\nu+d)}{2}} \left| \left( I + \frac{x
  x^t}{\nu} \right)^{-1} \right|^{\frac{\nu+d}{2}} \Gamma_d \left( \frac{\nu+d}{2}
  \right)} \, dw
  \end{eqnarray*}
  %
  Para  \ $a, b  \in \Rset^d,  \: M  \in \M_{d,d}(\Rset)$,  en la  secunda linea
  usamos la  identidad \  $a^t M b  = \Tr(b a^t  M)$ \  y \ $\Gamma_d\left(  x -
    \frac12 \right) = \frac{\Gamma\left(  x - \frac{d}{2} \right)}{\Gamma(x)} \,
  \Gamma_d(x)$ \ (ver notaciones) y en  la tercera linea usamos $\left| \Sigma +
    \frac{x   x^t}{\nu}   \right|  =   \left|   \Sigma   \right|   \left|  I   +
    \frac{\Sigma^{-1}   x    x^t}{\nu}   \right|$   \   y    la   identidad   de
  Sylvester~\cite{Syl51} o~\cite[\S~18.1]{Har08} \ $\left| I + a b^t \right| = 1
  + b^t  a$. Se  concluye que \  $X \sim  t_\nu(0,\Sigma)$ \ reconociendo  en el
  factor de  la integral como  la distribuci\'on \  $t_\nu(0,\Sigma)$ \ y  en el
  integrande la  distribuci\'on de Wishart \  $\W( \left( I  + \frac{x x^t}{\nu}
  \right),\nu+d)$ \ que suma entonces a la unidad.
  %  la  formula de  Sherman-Morrison-Woodbury  $\left(  I  + \frac{x  x^t}{\nu}
  % \right)^{-1} = I$~\cite{HorJoh13, Har08}
\end{proof}

Como lo hemos introducido, la  distribuci\'on de Student aparece naturalmente en
el  marco  de la  estimaci\'on,  especialmente  a  trav\'es de  la  estimaci\'on
empirica de la media y covarianza~\cite{Mui82, GupNag99, BilBre99, And03, Seb04}:
% resp. p 80 teo 3.2.1 -- p. 92 teo 3.3.6 -- p. 87 prop. 7.1 -- p. 77 teo. 3.3.2 -- p. 63 teo. 3.1
% Nota : ver corolarios 2 y 3, p. 25 de Seber
% VER GupNag Th. 4.2.1
%
\begin{teorema}%[]
%
  Sean  \  $X_i \,  \sim  \,  \N(m,\Sigma), \:  i  =  1, \ldots  ,  n  > d-1$  \
  independientes,       y      sea       la       media      empirica       (ver
  corolario~\ref{Cor:MP:MediaEmpiricaGauss})
  %
  \[
  \overline{X} = \frac{1}{n} \sum_{i=1}^n X_i
  \]
  %
  y  la  covarianza empirica  construida  a partir  de  la  media empirica  (ver
  corolario~\ref{Cor:MP:WishartEstimacion})
  %
  \[
  \overline{\Sigma}  =  \frac{1}{n-1}  \sum_{i=1}^n  \left( X_i  -  \overline{X}
  \right) \left( X_i - \overline{X} \right)^t
  \]
  %
  Entonces:
  %
  \begin{itemize}
  \item $\overline{X} -  m \, \sim \,  \N\left( 0 \, , \,  \frac{1}{n} \, \Sigma
    \right)$ \ y  \ $\overline{\Sigma} \, \sim \, \W(  \frac{1}{n-1} \Sigma \, ,
    \, n-1 ) $ \ son independientes;
  %
  \item $\sqrt{\frac{n (n-d)}{n-1}} \: \overline{\Sigma}^{\, -\frac12} \, \left(
      \overline{X} - m \right) \, \sim \, t_{n-d}\left( 0 \, , \, I \right)$
  \end{itemize}
\end{teorema}
%
\begin{proof}
  Se      refiera      a     los      corolarios~\ref{Cor:MP:MediaEmpiricaGauss}
  y~\ref{Cor:MP:WishartEstimacion}   por   lo  de   las   distribuciones  de   \
  $\overline{X}-m$ \ y de \ $\overline{\Sigma}$ \ respectivamente.

  A continuaci\'on,  sean \  $\widetilde{X}_i =  X_i - m$  \ y  \ $\widetilde{X}
  =        \begin{bmatrix}        \widetilde{X}_1        &       \cdots        &
    \widetilde{X}_n \end{bmatrix}$. Obviamente
  %
  \[
  \overline{\widetilde{X}} \equiv \overline{X} - m = \frac{1}{n} \, \widetilde{X} \un
  \]
  %
  con \ $\un \in  \Rset^n$ \ vector de componentes iguales a \  $1$ \ y vimos en
  la prueba del corolario~\ref{Cor:MP:WishartEstimacion} que
  %
  \[
  \overline{\Sigma} = \frac{1}{n-1} \widetilde{X} \left( I - \frac{\un \un^t}{n}
  \right) \widetilde{X}^t
  \]
  %
  $A = I - \frac{\un \un^t}{n} \in  P_n(\Rset)$ \ es idemponenta de rango 1, con
  $A  \un = 0$,  as\'i que  por diagonalizaci\'on~\cite{HorJoh13,  Bat97, Bat07}
  tenemos
  %
  \[
  A = P \begin{bmatrix} I_{n-1} & 0\\ 0 & 0 \end{bmatrix} P^t \qquad \mbox{con} \qquad
  P = \begin{bmatrix} B & \frac{1}{\sqrt{n}} \un \end{bmatrix}
  \]
  %
  $P P^t = P P^t = I$ \ y
  %
  \[
  B \in \M_{n,n-1}(\Rset) \quad \mbox{tal que} \quad  B^t B = I \: \mbox{ y } \:
  \un^t B = 0
  \]
  %
  Ahora,   poniendo   la  descomposici\'on   diagonal   de   \   $A$  \   en   \
  $\overline{\Sigma}$ \ obtenemos (ver~corolario~\ref{Cor:MP:WishartEstimacion})
  %
  \[
  \overline{\Sigma} = \frac{1}{n-1} \, Y Y^t \qquad \mbox{con} \qquad Y = \widetilde{X} B
  \]
  %
  Luego, de la gausianidad y independencia de los \ $\widetilde{X}_i$ \ tenemos,
  para \   $\widetilde{x}   =    \begin{bmatrix}   \widetilde{x}_1   &   \cdots   &
    \widetilde{x}_n \end{bmatrix} \in \M_{d,n}(\Rset)$
  %
  \begin{eqnarray*}
  p_{\widetilde{X}}(\widetilde{x}) & = & (2 \pi)^{-\frac{n d}{2}}
  |\Sigma|^{-\frac{n}{2}} \exp\left(- \frac12 \sum_{i=1}^n \widetilde{x}_i^t
  \Sigma^{-1} \widetilde{x}_i \right)\\[2mm]
  %
  & = & (2 \pi)^{-\frac{n d}{2}} |\Sigma|^{-\frac{n}{2}} \exp\left(- \frac12
  \sum_{i=1}^n \Tr\left( \Sigma^{-1} \widetilde{x}_i \widetilde{x}_i^t
  \right) \right)\\[2mm]
  %
  & = & (2 \pi)^{-\frac{n d}{2}} |\Sigma|^{-\frac{n}{2}} \exp\left(- \frac12
  \Tr\left( \Sigma^{-1} \widetilde{x} \widetilde{x}^t \right) \right)
  \end{eqnarray*}
  %
  Sea    la     transformaci\'on    \    $\begin{bmatrix}     Y    &    \sqrt{n}
    \overline{\widetilde{X}}   \end{bmatrix}   =   \widetilde{X}   P$,   \ie   \
  $\widetilde{X}        =       \begin{bmatrix}        Y        &       \sqrt{n}
    \overline{\widetilde{X}} \end{bmatrix}  P^t$.  Se nota que  \ $|P| =  1$ \ y
  por                            transformaci\'on                           (ver
  teorema~\ref{Teo:MP:TransformacionInyectivaDensidad}),    para   \    $y   \in
  \M_{d,n-1}(\Rset)$ \ y \ $x \in \Rset^d$
  %
  \begin{eqnarray*}
  p_{Y,\sqrt{n} \overline{\widetilde{X}}}(y,x) & = & (2 \pi)^{-\frac{n d}{2}}
  |\Sigma|^{-\frac{n}{2}} \exp\left(- \frac12 \Tr\left(
  \Sigma^{-1} \begin{bmatrix} y & x \end{bmatrix} P^t P \begin{bmatrix} y^t\\
  x \end{bmatrix}\right) \right)\\[2mm]
  %
  & = & (2 \pi)^{-\frac{n d}{2}} |\Sigma|^{-\frac{n}{2}} \exp\left(- \frac12
  \Tr\left( \Sigma^{-1} \left( y y^t + x x^t \right) \right) \right)\\[2mm]
  %
  & = & (2 \pi)^{-\frac{(n-1) d}{2}} |\Sigma|^{-\frac{n-1}{2}} \exp\left(-
  \frac12 \Tr\left( \Sigma^{-1} y y^t \right) \right) \times (2
  \pi)^{-\frac{d}{2}} |\Sigma|^{-\frac12} \exp\left(- \frac12 x^t \Sigma^{-1} x
  \right)
  \end{eqnarray*}
  %
  Claramente,  de la factorizaci\'on  de las  distribuciones, $Y  = X  B$ \  y \
  $\sqrt{n}  \overline{\widetilde{X}}$  \ son  independientes,  es  decir que  \
  $\frac{1}{n-1} \, Y Y^t = \overline{\Sigma}$ \ y \ $\overline{\widetilde{X}} =
  \overline{X} -  m$ \ son  independientes, lo que  cierra la prueba  del primer
  item.  Pasando, la forma  de $p_{Y,\sqrt{n}  \overline{\widetilde{X}}}(y,x)$ \
  confirma  que  \ $\overline{X}-m$  \  es  gausiana  centrada de  covarianza  \
  $\frac{1}{n} \,  \Sigma$, y  que los \  $Y_i$ \ son  independientes gausianos,
  dando la distribuci\'on  de Wishart del lema~\ref{Lem:MP:WishartGausiana} para
  la covarianza empirica.

  A continuaci\'on,
  %
  \[
  \sqrt{\frac{n   (n-d)}{n-1}}   \,   \overline{\Sigma}^{\,   -\frac12}   \left(
    \overline{X}-m \right)  = \frac{1}{\sqrt{n-1}} \:  \Sigma^{- \frac12} \left(
    \Sigma^{-1}  \, \overline{\Sigma}  \, \Sigma^{-1}  \right)^{-\frac12} \left(
    \sqrt{n (n-d)} \: \Sigma^{-\frac12} \left( \overline{X}-m \right) \right)
  \]
  %
  Del           teorema~\ref{Teo:MP:StabilidadGaussiana}          y          del
  lema~\ref{Lem:MP:StabilidadWishartLineal} tenemos
  %
  \[
  \sqrt{n (n-d)}  \: \Sigma^{-\frac12} \left( \overline{X}-m \right)  \, \sim \,
  \N(0 ,  (n-d) I)  \qquad \mbox{y} \qquad  \Sigma^{-1} \,  \overline{\Sigma} \,
  \Sigma^{-1}  \, \sim  \, \W\left(  \left( (n-1)  \Sigma \right)^{-1}  \,  , \,
    n-d+d-1 \right)
  \]
  %
  Se   cierra   la  prueba   usando   los  lemas~\ref{Lem:MP:StudentWishart}   \
  y~\ref{Lem:MP:StabilidadLineal}.
\end{proof}

%\SZ{
% VER LO QUE PASA SI overline{Sigma} si X_i y X_i-overline{X} para estandardizar los datos.

%Sampling distribution, Gosset, Fisher 25. Applications
%}

M\'as propiedades de esta distribuci\'on se encuentran en libros especializados,
por ejemplo~\cite{KotNad04} completamente dedicado a esta distribuci\'on.

\

La distribuci\'on  Student-t se generaliza  al caso matriz-variada  $X$ definido
sobre $M_{d,d'}(\Rset)$;  se denota  \ $X \,  \sim \,  t_\nu(M,\Sigma,\Omega)$ \
donde  \ $M  \in  M_{d,d'}(\Rset), \:  \Sigma  \in P_d^+(\Rset),  \: \Omega  \in
P_{d'}^+(\Rset)$   y  la  densidad   est  dada   por  $\displaystyle   p_X(x)  =
\frac{\Gamma_d\left(      \frac{\nu+d+d'-1}{2}\right)}{\pi^{\frac{\nu     d}{2}}
  \Gamma_d\left(       \frac{\nu+d-1}{2}\right)      \,       \left|      \Sigma
  \right|^{\frac{d'}{2}}  \left|  \Omega \right|^{\frac{d}{2}}}  \:  \left| I  +
  \Sigma^{-1}     (x-M)     \,     \Omega^{-1}     \,     (x-M)^t     \right|^{-
  \frac{\nu+d+d'-1}{2}}$. Se refiera  a~\cite[Cap.~4]{GupNag99} para tener m\'as
detalles.



\SZ{
% --------------------------------- Familia exponencial
\subsubseccion{Familia  exponencial}
\cite{Dar35, Koo36,  And70,  Kay93, LehCas98,  Rob07}.;
Muchas de estas leyes entran en una familia que juega un rol particular en problema de maximizacion de entropie (ver cap 2): es la familia exponencial...
}

\SZ{
% --------------------------------- Familia eliptica
\subsubseccion{Familia eliptica}
Invariante por rotacion... GSM
}


\SZ{Teorema del l\'imite central, y relajando la independencia, y versiones con leyes diferentes pero uniformamente acotadas.}

\SZ{hablar de  simulaci\'on? Metoto inverso,  mezcla, rejeccion, a traves  de la
  condicional para el caso vectorial?}

