\seccion{Algunos ejemplos de distribuciones de probabilidad}
\label{Sec:MP:EjemplosDistribucionesProb}


\emph{introducci\'on...}

% ================================= Variables discretas

\subseccion{Distribuciones de variable discreta}
\label{Ssec:MP:EjemplosDistribucionesDiscretas}


% --------------------------------- Certeza

\subsubseccion{Variable con certeza}

...

% --------------------------------- Bernoulli

\subsubseccion{Ley de Bernoulli}

Se denota $X \sim \B(p)$ con $p \in [0 \, , \, 1]$ y sus caracter\'isticas son las
siguientes:

\begin{center}
\begin{tabular}
{
|>{\vspace{-2mm}}p{.3\textwidth}|
>{\vspace{-2mm}\hspace{2mm}}p{.5\textwidth}|
}
%
\hline
%
Dominio de definici\'on & $\X = \{ 0 \, , \, 1 \}$\\
\hline
%
Distribuci\'on de probabilidad & $p_X(1) = 1 - p_X (0) = p$\\
\hline
%
%Momentos & $ \Esp\left[ X^k \right] = p^k$\\
%\hline
%
Promedio & $ \Esp[X] = p$\\
\hline
%
Varianza & $\sigma_X^2 = p (1-p)$\\
\hline
%
Asimetr\'ia & $\gamma_X = 0$\\
\hline
%
Curtosis & $\kappa_X = 0$\\
\hline
%
Generadora de probabilidad & $G_X(z) = 1 -p + p z$ \ sobre \ $\Cset$\\
\hline
%
Generadora de momentos & $M_X(u) = 1 -p + p e^u$ \ sobre \ $\Cset$\\
\hline
%
Funci\'on caracter\'istica & $\Phi_X(\omega) = 1 - p + p e^{\imath \omega}$\\
\hline
\end{tabular}
\end{center}

Su masa de probabilidad y funci\'on de repartici\'on son representadas en la figura Fig.~\ref{Fig:MP:Bernoulli}.
%
\begin{figure}[h!]
\begin{center} \begin{tikzpicture}%[scale=.9]
\shorthandoff{>}
%
\pgfmathsetmacro{\sx}{2};% x-scaling
\pgfmathsetmacro{\r}{.05};% radius arc non continuity F_X
\pgfmathsetmacro{\p}{1/3};% probabilidad p
% masa
\begin{scope}
%
\pgfmathsetmacro{\sy}{2/max(\p,1-\p)};% y-scaling
%
\pgfmathsetmacro{\ss}{\sy*(1-\p)};
\draw[>=stealth,->] (-.5,0)--({\sx+.75},0) node[right]{\small $x$};
\draw[>=stealth,->] (0,-.15)--(0,2.5) node[above]{\small $p_X$};
%
\draw (0,-.1) node[below,scale=.7]{$0$} --(0,0);
\draw[dotted] (0,0)--(0,{\sy*(1-\p)}) node[scale=.7]{$\bullet$};
\draw (0,{\sy*(1-\p)})--(-.1,{\sy*(1-\p)}) node[left,scale=.7]{$1-p$};
%
\draw (\sx,-.1) node[below,scale=.8]{\small $1$} --(\sx,0);
\draw[dotted] (\sx,0)--(\sx,{\sy*\p}) node[scale=.7]{$\bullet$};
\draw (0,{\sy*\p})--(-.1,{\sy*\p}) node[left,scale=.7]{\small $p$};
%
\node at ({(\sx+.75)/2},-1) [scale=.9]{(a)};
\end{scope}
%
%
% reparticion
\begin{scope}[xshift=7cm]
%
\pgfmathsetmacro{\sy}{2};% y-scaling 
%
\draw[>=stealth,->] (-.5,0)--({\sx+1.5},0) node[right]{\small $x$};
\draw[>=stealth,->] (0,-.15)--(0,{\sy+.5}) node[above]{\small $F_X$};
%
\draw (0,0)--(0,-.1) node[below,scale=.7]{$0$};
\draw (\sx,0)--(\sx,-.1) node[below,scale=.7]{$1$};
\draw (0,{\sy*(1-\p)})--(-.1,{\sy*(1-\p)}) node[left,scale=.7]{$1-p$};
\draw (0,\sy)--(-.1,\sy) node[left,scale=.7]{$1$};
%
\draw[thick](-.25,0)--(0,0);
\draw ({0+\r},\r) arc (90:270:\r);
%
\draw[dotted] (0,0)--(0,{\sy*(1-\p)});
\draw[thick](0,{\sy*(1-\p)}) node[scale=.7]{$\bullet$}--(\sx,{\sy*(1-\p)});
\draw ({\sx+\r},{\r+\sy*(1-\p)}) arc (90:270:\r);
%
\draw[dotted] (\sx,{\sy*(1-\p)})--(\sx,\sy);
\draw[thick](\sx,\sy) node[scale=.7]{$\bullet$}--({\sx+1},\sy);
%\draw ({\sx+\r},{\r+\sy*(1-\p)}) arc (90:270:\r);
%
\node at ({(\sx+1.5)/2},-1) [scale=.9]{(b)};
\end{scope}
%
\end{tikzpicture} \end{center}
%
\leyenda{Ilustraci\'on de una distribuci\'on  de probabilidad de Bernoulli (a), y la
  funci\'on de repartici\'on asociada (b), con $p = \frac13$.}
\label{Fig:MP:Bernoulli}
\end{figure}

Cuando $p = 0$ (resp. $p = 1$) la variable es cierta $X = 0$ (resp. $X = 1$).



% --------------------------------- Binomial

\subsubseccion{Ley Binomial}

Se denota $X \sim  \B(n,p)$ con $n \in \Nset \backslash \{0 ,  1 \}, \quad p \in
[0 \, , \, 1]$ y sus caracter\'isticas son las siguientes:

\begin{center}
\begin{tabular}
{
|>{\vspace{-2mm}}p{.3\textwidth}|
>{\vspace{-2mm}\hspace{2mm}}p{.5\textwidth}|
}
%
\hline
%
Dominio de definici\'on & $\X = \{ 0 \, , \, \ldots \, , \, n \}$\\
\hline
%
Distribuci\'on de probabilidad & $p_X(k) = \bino{n}{k} p^k (1-p)^{n-k}$\\
\hline
%
%Momentos & $ \Esp\left[ X^k \right] = p^k$\\
%\hline
%
Promedio & $ \Esp[X] = n p$\\
\hline
%
Varianza & $\sigma_X^2 = n p (1-p)$\\
\hline
%
Asimetr\'ia & $\gamma_X = \frac{1-2 p}{\sqrt{n p (1-p)}}$\\
\hline
%
Curtosis & $\kappa_X = \frac{1-6 p}{n p (1-p)}$\\
\hline
%
Generadora de probabilidad & $G_X(z) = \left( 1 - p + p z \right)^n$ \ sobre \ $\Cset$\\
\hline
%
Generadora de momentos & $M_X(u) = \left(1 - p + p e^u \right)^n$ \ sobre \ $\Cset$\\
\hline
%
Funci\'on caracter\'istica & $\Phi_X(\omega) = \left( 1 - p + p e^{\imath \omega} \right)^n$\\
\hline
\end{tabular}
\end{center}
%
\noindent con $\bino{n}{k} = \frac{n!}{k! (n-k)!}$ coefficiente binomial.
% Modo $\left\lfloor (n+1) p \right\rfloor$
% Mediana $\left\lfloor n p \right\rfloor$ o $\left\lceil n p \right\rceil
% CDF	$I_{1-p}(n-k,k+1)$ regularized incomplete beta function

Su masa  de probabilidad  y funci\'on de  repartici\'on son representadas  en la
figura Fig.~\ref{Fig:MP:Binomial}.
%
\begin{figure}[h!]
\begin{center} \begin{tikzpicture}%[scale=.9]
\shorthandoff{>}
%
\pgfmathsetmacro{\sx}{.75};% x-scaling
\pgfmathsetmacro{\r}{.05};% radius arc non continuity F_X
\pgfmathsetmacro{\p}{1/3};% probabilidad p
\pgfmathsetmacro{\n}{6};% numero n de la binomial
\pgfmathsetmacro{\q}{floor((\n+1)*\p)};% modo de la binomial
\pgfmathsetmacro{\m}{factorial(\n)/factorial(\q)/factorial(\n-\q)*(\p^\q)*((1-\p)^(\n-\q))};% maximo de la binomial
% masa
\begin{scope}
%
\pgfmathsetmacro{\sy}{2.5/\m};% y-scaling 
\draw[>=stealth,->] (-.25,0)--({\sx*\n+.25},0) node[right]{\small $x$};
\draw[>=stealth,->] (0,-.1)--(0,{\sy*\m+.25}) node[above]{\small $p_X$};
%
\pgfmathsetmacro{\b}{(1-\p)^\n};% coeficiente binomial por la probabilidad
%
\foreach \k in {0,...,\n} {
\draw ({\k*\sx},0)--({\k*\sx},-.1) node[below,scale=.7]{\k};
\draw[dotted] ({\k*\sx},0)--({\k*\sx},{\sy*\b}) node[scale=.7]{$\bullet$};
%
\pgfmathsetmacro{\bl}{\b*\p*(\n-\k)/((\k+1)*(1-\p))};\global\let\b\bl;% proba actualizado
}
\draw (0,{((1-\p)^\n)*\sy})--(-.1,{((1-\p)^\n)*\sy}) node[left,scale=.7]{$(1-p)^n$};
\draw (0,{\n*\p*((1-\p)^(\n-1))*\sy})--(-.1,{\n*\p*((1-\p)^(\n-1))*\sy}) node[left,scale=.7]{$n p (1-p)^{n-1}$};
%
\end{scope}
%
%
% reparticion
\begin{scope}[xshift=8.5cm]
%
\pgfmathsetmacro{\sy}{2.5};% y-scaling 
%
\draw[>=stealth,->] (-.6,0)--({\sx*(\n+.5)+.5},0) node[right]{\small $x$};
\draw[>=stealth,->] (0,-.1)--(0,{\sy+.25}) node[above]{\small $F_X$};
%
\pgfmathsetmacro{\b}{(1-\p)^\n};% coeficiente binomial por la probabilidad
\pgfmathsetmacro{\c}{(1-\p)^\n};% cumulativa binomial por la probabilidad
%
% cumulativa x < 0
\draw (0,0)--(0,-.1) node[below,scale=.7]{0};
\draw[thick] (-.5,0)--(0,0);
\draw (\r,\r) arc (90:270:\r);
%
% cumulativa x de 0 a n-1
\foreach \k in {1,...,\n} {
\draw ({\k*\sx},0)--({\k*\sx},-.1) node[below,scale=.7]{\k};
\draw[thick]({(\k-1)*\sx},{\sy*\c}) node[scale=.7]{$\bullet$}--({\k*\sx},{\sy*\c});
\draw ({\k*\sx+\r},{\sy*\c+\r}) arc (90:270:\r);
\draw[dotted] ({(\k-1)*\sx},{(\c-\b)*\sy})--({(\k-1)*\sx},{\c*\sy});
%
\pgfmathsetmacro{\bl}{\b*\p*(\n-\k+1)/(\k*(1-\p))};\global\let\b\bl;% proba actualizado
\pgfmathsetmacro{\cl}{\c+\b};\global\let\c\cl;% cumulativa actualizada
}
%
% cumulativa x > n
\draw[dotted] ({\n*\sx},{(1-\b)*\sy})--({\n*\sx},\sy);
\draw[thick]({\n*\sx},\sy) node[scale=.7]{$\bullet$}--({(\n+.5)*\sx},\sy);
%
\draw (0,{((1-\p)^\n)*\sy})--(-.1,{((1-\p)^\n)*\sy}) node[left,scale=.7]{$(1-p)^n$};
\draw (0,{(\n*\p+1-\p)*((1-\p)^(\n-1))*\sy})--(-.1,{(\n*\p+1-\p)*((1-\p)^(\n-1))*\sy}) node[left,scale=.7]{$(1-p+np) (1-p)^{n-1}$};
\draw (-.2,{((\n*\p+1-\p)*((1-\p)^(\n-1))+1)/2*\sy}) node[scale=.7]{$\vdots$};
\draw (0,\sy)--(-.1,\sy) node[left,scale=.7]{$1$};
\end{scope}
%
\end{tikzpicture} \end{center}
%
\leyenda{Ilustraci\'on de una distribuci\'on  de probabilidad Binomial (a), y la
  funci\'on de repartici\'on asociada (b), con $n = 6, \quad p = \frac13$.}
\label{Fig:MP:Binomial}
\end{figure}

Cuando $n = 2$, se recupera la  lei de Bernoulli $\B(p) \equiv \B(2,p)$ y cuando
$p  = 0$  (resp. $p  = 1$)  la variable  es cierta  $X =  0$ (resp.   $X  = n$).
Ad\'emas, se muestra  sencillamente usando la generadora de  probabilidad que si
$X_i \sim \B(p),  \quad i = 1,  \ldots , n$ \ independientes,  $X = \sum_{i=1}^n
X_i  \sim  \B(n,p)$:  esta  distribuci\'on  aparece  en  el  conteo  de  eventos
independientes de misma probabilidad entre $n$.


\aver{
%%%%%%%

...

Ley uniforme

...
}


% --------------------------------- Geometrica

\subsubseccion{Ley Geometrica}

Se denota $X  \sim \G(p)$ con $ p \in  (0 \, , \, 1)$  y sus caracter\'isticas son
las siguientes:

\begin{center}
\begin{tabular}
{
|>{\vspace{-2mm}}p{.3\textwidth}|
>{\vspace{-2mm}\hspace{2mm}}p{.5\textwidth}|
}
%
\hline
%
Dominio de definici\'on & $\X = \Nset^*$\\
\hline
%
Distribuci\'on de probabilidad & $p_X(k) = (1-p)^{k-1} p$\\
\hline
%
%Momentos & $ \Esp\left[ X^k \right] = p^k$\\
%\hline
%
Promedio & $ \Esp[X] = \frac1p$\\
\hline
%
Varianza & $\sigma_X^2 = \frac{1-p}{p^2}$\\
\hline
%
Asimetr\'ia & $\gamma_X = \frac{2-p}{\sqrt{1-p}}$\\
\hline
%
Curtosis & $\kappa_X = 6 + \frac{p^2}{1-p}$\\
\hline
%
Generadora de probabilidad & $G_X(z) = \frac{p z}{1-(1-p) z}$ \ para \ $|z| < \frac1{1-p}$\\
\hline
%
Generadora de momentos & $M_X(u) = \frac{p e^u}{1-(1-p) e^u}$ \ para \ $\real{u} < - \ln(1-p)$\\
\hline
%
Funci\'on caracter\'istica & $\Phi_X(\omega) = \frac{p e^{\imath \omega}}{1-(1-p) e^{\imath u}}$\\
\hline
\end{tabular}
\end{center}
%
%\noindent con $\bino{n}{k} = \frac{n!}{k! (n-k)!}$ coefficiente binomial.
% Modo 1
% Mediana $\left\lceil \frac{-1}{\log_2(1-p)} \right\rceil$ 
% CDF	$1-(1-p)^k$

Su masa  de probabilidad  y funci\'on de  repartici\'on son representadas  en la
figura Fig.~\ref{Fig:MP:Binomial}.
%
\begin{figure}[h!]
\begin{center} \begin{tikzpicture}%[scale=.9]
\shorthandoff{>}
%
\pgfmathsetmacro{\sx}{.75};% x-scaling
\pgfmathsetmacro{\r}{.05};% radius arc non continuity F_X
\pgfmathsetmacro{\p}{1/3};% probabilidad p
\pgfmathsetmacro{\n}{7};% k mas grande del plot (k in Nset^*)
%
% masa
\begin{scope}
%
\pgfmathsetmacro{\sy}{2.5/\p};% y-scaling 
\draw[>=stealth,->] (-.25,0)--({\sx*\n+.75},0) node[right]{\small $x$};
\draw[>=stealth,->] (0,-.1)--(0,{\sy*\p+.25}) node[above]{\small $p_X$};
%
\pgfmathsetmacro{\pr}{\p};% probabilidad
%
\foreach \k in {1,...,\n} {
\draw ({\k*\sx},0)--({\k*\sx},-.1) node[below,scale=.7]{\k};
\draw[dotted] ({\k*\sx},0)--({\k*\sx},{\sy*\pr}) node[scale=.7]{$\bullet$};
%
\pgfmathsetmacro{\prl}{\pr*(1-\p)};\global\let\pr\prl;% proba actualizado
}
\draw ({(\n+.5)*\sx},-.2) node[below,scale=.7]{$\ldots$};
\draw ({(\n+.5)*\sx},{(\pr/(1-\p)/2*\sy}) node[scale=.7]{$\cdots$};
\draw (0,{\p*\sy})--(-.1,{\p*\sy}) node[left,scale=.7]{$p$};
\draw (0,{\p*(1-\p)*\sy})--(-.1,{\p*(1-\p)*\sy}) node[left,scale=.7]{$p \, (1-p)$};
\draw (-.5,{\p*(1-\p)/2*\sy}) node[left,scale=.7]{$\vdots$};
%
\end{scope}
%
%
% reparticion
\begin{scope}[xshift=8.5cm]
%
\pgfmathsetmacro{\sy}{2.5};% y-scaling 
%
\draw[>=stealth,->] (-.6,0)--({\sx*\n+.75},0) node[right]{\small $x$};
\draw[>=stealth,->] (0,-.1)--(0,{\sy+.25}) node[above]{\small $F_X$};
%
\pgfmathsetmacro{\pr}{\p};% probabilidad
\pgfmathsetmacro{\c}{\p};% cumulativa
%
% cumulativa x < 1
\draw (1,0)--(1,-.1) node[below,scale=.7]{0};
\draw[thick] (-.5,0)--(\sx,0);
\draw ({\sx+\r},\r) arc (90:270:\r);
%
% cumulativa x de 1 a n
\foreach \k in {2,...,\n} {
\draw ({\k*\sx},0)--({\k*\sx},-.1) node[below,scale=.7]{\k};
\draw[thick]({(\k-1)*\sx},{\sy*\c}) node[scale=.7]{$\bullet$}--({\k*\sx},{\sy*\c});
\draw ({\k*\sx+\r},{\sy*\c+\r}) arc (90:270:\r);
\draw[dotted] ({(\k-1)*\sx},{(\c-\pr)*\sy})--({(\k-1)*\sx},{\c*\sy});
%
\pgfmathsetmacro{\prl}{\pr*(1-\p)};\global\let\pr\prl;% proba actualizado
\pgfmathsetmacro{\cl}{\c+\pr};\global\let\c\cl;% cumulativa actualizada
}
%
% cumulativa x > n
\draw ({(\n+.5)*\sx},-.2) node[below,scale=.7]{$\ldots$};
\draw ({(\n+.5)*\sx},{((\c+1)/2*\sy}) node[scale=.7]{$\cdots$};
\draw (0,{\p*\sy})--(-.1,{\p*\sy}) node[left,scale=.7]{$p$};
\draw (0,{\p*(2-\p)*\sy})--(-.1,{\p*(2-\p)*\sy}) node[left,scale=.7]{$p \, (2-p)$};
\draw (-.3,{(1+\p*(2-\p))/2*\sy}) node[left,scale=.7]{$\vdots$};
\draw (0,\sy)--(-.1,\sy) node[left,scale=.7]{$1$};
\end{scope}
%
\end{tikzpicture} \end{center}
%
\leyenda{Ilustraci\'on de una distribuci\'on  de probabilidad Geometrica (a), y la
  funci\'on de repartici\'on asociada (b), con $p = \frac13$.}
\label{Fig:MP:Geometrica}
\end{figure}

Cuando $p =  0$ (resp. $p =  1$) la variable es cierta  $X = 0$ (resp.   $X = 1$
usando la convenci\'on $0^0 = 1$).   Esta distribuci\'on aparece en el conteo de
conteo de une repetici\'on de  una experiencia de maneja independiente hasta que
occure un evento de probabilidad $p$; por ejemplo el n\'umero de tiro de un dado
equilibriado hasta que occurre un ``6'' sigue una ley geometrica de parametro $p
= \frac16$.


% --------------------------------- Poisson

\subsubseccion{Ley de Poisson}

Se  denota  $X  \sim  \P(\lambda)$  con   $  \lambda  \in  \Rset_+^*$  \  y  sus
caracter\'isticas son las siguientes:

\begin{center}
\begin{tabular}
{
|>{\vspace{-2mm}}p{.3\textwidth}|
>{\vspace{-2mm}\hspace{2mm}}p{.5\textwidth}|
}
%
\hline
%
Dominio de definici\'on & $\X = \Nset$\\
\hline
%
Distribuci\'on de probabilidad & $p_X(k) = \frac{\lambda^k e^{-\lambda}}{k!}$\\
\hline
%
%Momentos & $ \Esp\left[ X^k \right] = p^k$\\
%\hline
%
Promedio & $ \Esp[X] = \lambda$\\
\hline
%
Varianza & $\sigma_X^2 = \lambda$\\
\hline
%
Asimetr\'ia & $\gamma_X = \lambda^{-\frac12}$\\
\hline
%
Curtosis & $\kappa_X = \lambda^{-1}$\\
\hline
%
Generadora de probabilidad & $G_X(z) = e^{\lambda (z-1)}$ \ para \ $z \in \Cset$\\
\hline
%
Generadora de momentos & $M_X(u) = e^{\lambda \left( e^u - 1 \right)}$ \ para \ $u \in \Cset$\\
\hline
%
Funci\'on caracter\'istica & $\Phi_X(\omega) = e^{\lambda \left( e^{\imath \omega} - 1 \right)}$\\
\hline
\end{tabular}
\end{center}
%
% modo \lfloor \lambda \rfloor 
% Mediana \approx \lfloor \lambda +1/3-0.02/\lambda \rfloor 
% CDF {\frac {\Gamma
% (\lfloor k+1\rfloor  ,\lambda )}{\lfloor k\rfloor !}} where  $\Gamma (x,y)$ is
% the upper incomplete gamma function,

Su masa  de probabilidad  y funci\'on de  repartici\'on son representadas  en la
figura Fig.~\ref{Fig:MP:Poisson}.
%
\begin{figure}[h!]
\begin{center} \begin{tikzpicture}%[scale=.9]
\shorthandoff{>}
%
\pgfmathsetmacro{\sx}{.75};% x-scaling
\pgfmathsetmacro{\r}{.05};% radius arc non continuity F_X
\pgfmathsetmacro{\l}{3};% lambda
\pgfmathsetmacro{\n}{7};% k mas grande del plot (k in Nset)
\pgfmathsetmacro{\q}{floor(\l)};% modo
\pgfmathsetmacro{\m}{(\l^\q)*exp(-\l)/factorial(\q)};% maximo
%
% masa
\begin{scope}
%
\pgfmathsetmacro{\sy}{2.75/\m};% y-scaling 
\draw[>=stealth,->] (-.25,0)--({\sx*\n+.75},0) node[right]{\small $x$};
\draw[>=stealth,->] (0,-.1)--(0,{\sy*\m+.25}) node[above]{\small $p_X$};
%
\pgfmathsetmacro{\pr}{exp(-\l)};% probabilidad
%
\foreach \k in {0,...,\n} {
\draw ({\k*\sx},0)--({\k*\sx},-.1) node[below,scale=.7]{\k};
\draw[dotted] ({\k*\sx},0)--({\k*\sx},{\sy*\pr}) node[scale=.7]{$\bullet$};
%
\pgfmathsetmacro{\prl}{\pr*\l/(\k+1)};\global\let\pr\prl;% proba actualizado
}
\draw ({(\n+.5)*\sx},-.2) node[below,scale=.7]{$\ldots$};
\draw ({(\n+.5)*\sx},{(\pr/\l*\n/2*\sy}) node[scale=.7]{$\cdots$};
\draw (0,{exp(-\l)*\sy})--(-.1,{exp(-\l)*\sy}) node[left,scale=.7]{$e^{-\lambda}$};
\draw (0,{\l*exp(-\l)*\sy})--(-.1,{\l*exp(-\l)*\sy}) node[left,scale=.7]{$\lambda e^{-\lambda}$};
\draw (0,{\l*\l*exp(-\l)/2*\sy})--(-.1,{\l*\l*exp(-\l)/2*\sy}) node[left,scale=.7]{$\frac{\lambda^2 e^{-\lambda}}{2}$};
%\draw (-.5,{\l*exp(-\l)/2*\sy}) node[left,scale=.7]{$\vdots$};
%
\end{scope}
%
%
% reparticion
\begin{scope}[xshift=8.5cm]
%
\pgfmathsetmacro{\sy}{2.75};% y-scaling 
%
\draw[>=stealth,->] (-.6,0)--({\sx*\n+.75},0) node[right]{\small $x$};
\draw[>=stealth,->] (0,-.1)--(0,{\sy+.25}) node[above]{\small $F_X$};
%
\pgfmathsetmacro{\pr}{exp(-\l)};% probabilidad
\pgfmathsetmacro{\c}{exp(-\l)};% cumulativa
%
% cumulativa x < 0
\draw (0,0)--(0,-.1) node[below,scale=.7]{0};
\draw[thick] (-.5,0)--(0,0);
\draw (\r,\r) arc (90:270:\r);
%
% cumulativa x de 0 a n
\foreach \k in {1,...,\n} {
\draw ({\k*\sx},0)--({\k*\sx},-.1) node[below,scale=.7]{\k};
\draw[thick]({(\k-1)*\sx},{\sy*\c}) node[scale=.7]{$\bullet$}--({\k*\sx},{\sy*\c});
\draw ({\k*\sx+\r},{\sy*\c+\r}) arc (90:270:\r);
\draw[dotted] ({(\k-1)*\sx},{(\c-\pr)*\sy})--({(\k-1)*\sx},{\c*\sy});
%
\pgfmathsetmacro{\prl}{\pr*\l/\k};\global\let\pr\prl;% proba actualizado
\pgfmathsetmacro{\cl}{\c+\pr};\global\let\c\cl;% cumulativa actualizada
}
%
% cumulativa x > n
\draw ({(\n+.5)*\sx},-.2) node[below,scale=.7]{$\ldots$};
\draw ({(\n+.5)*\sx},{((\c+1)/2*\sy}) node[scale=.7]{$\cdots$};
\draw (0,{exp(-\l)*\sy})--(-.1,{exp(-\l)*\sy}) node[left,scale=.7]{$e^{-\lambda}$};
\draw (0,{(1+\l)*exp(-\l)*\sy})--(-.1,{(1+\l)*exp(-\l)*\sy}) node[left,scale=.7]{$(1+\lambda) e^{-\lambda}$};
\draw (-.3,{(1+(1+\l+\l*\l/2)*exp(-\l))/2*\sy}) node[left,scale=.7]{$\vdots$};
\draw (0,\sy)--(-.1,\sy) node[left,scale=.7]{\small $1$};
\end{scope}
%
\end{tikzpicture} \end{center}
%
\leyenda{Ilustraci\'on de una distribuci\'on  de probabilidad de Poisson (a), y la
  funci\'on de repartici\'on asociada (b), con $\lambda = 3$.}
\label{Fig:MP:Poisson}
\end{figure}

Cuando $\lambda =  0$ la variable es cierta  $X = 0$ (usando la convenci\'on $0^0 = 1$).
\SZ{   Esta distribuci\'on aparece...}
% en el conteo de
%conteo de une repetici\'on de  una experiencia de maneja independiente hasta que
%occure un evento de probabilidad $p$; por ejemplo el n\'umero de tiro de un dado
%equilibriado hasta que occurre un ``6'' sigue una ley geometrica de parametro $p
%= \frac16$.


\aver{
%%%%%%%

...

Estad\'istica  de  los  n\'umeros   de  ocupaci\'on  de  niveles  energ\'eticos:
distribuciones de Maxwell--Boltzmann, de Fermi--Dirac, y de Bose--Einstein

...


Leyes de los grandes n\'umeros


...

Ley multinomial
}


% ================================= Variables continuas

\subseccion{Distribuciones de variable continua}
\label{Ssec:MP:EjemplosDistribucionescontinuas}

\aver{
Distribuci\'on uniforme sobre un intervalo

...
}



% --------------------------------- Exponencial

\subsubseccion{Distribuci\'on exponencial}

Se  denota  $X  \sim  \E(\lambda)$  con   $  \lambda  \in  \Rset_+^*$  \  y  sus
caracter\'isticas son las siguientes:

\begin{center}
\begin{tabular}
{
|>{\vspace{-2mm}}p{.3\textwidth}|
>{\vspace{-2mm}\hspace{2mm}}p{.5\textwidth}|
}
%
\hline
%
Dominio de definici\'on & $\X = \Rset_+$\\
\hline
%
Densidad de probabilidad & $p_X(x) = \lambda e^{-\lambda x}$\\
\hline
%
%Momentos & $ \Esp\left[ X^k \right] = p^k$\\
%\hline
%
Promedio & $ \Esp[X] = \frac1\lambda$\\
\hline
%
Varianza & $\sigma_X^2 = \frac1{\lambda^2}$\\
\hline
%
Asimetr\'ia & $\gamma_X = 2$\\
\hline
%
Curtosis & $\kappa_X = 6$\\
\hline
%
%Generadora de probabilidad & $G_X(z) = e^{\lambda (z-1)}$ \ para \ $z \in \Cset$\\
%\hline
%
Generadora de momentos & $M_X(u) = \frac{\lambda}{\lambda-u}$ \ para \ $\real{u} < \lambda$\\
\hline
%
Funci\'on caracter\'istica & $\Phi_X(\omega) = \frac{\lambda}{\lambda-\imath \omega}$\\
\hline
\end{tabular}
\end{center}
%
% modo 0
% Mediana \ln(2)/\lambda
% CDF 1-e^{-\lambda x}

Su densidad  de probabilidad  y funci\'on de  repartici\'on son representadas  en la
figura Fig.~\ref{Fig:MP:Exponencial}.
%
\begin{figure}[h!]
\begin{center} \begin{tikzpicture}%[scale=.9]
\shorthandoff{>}
%
\pgfmathsetmacro{\sx}{.75};% x-scaling
\pgfmathsetmacro{\r}{.05};% radius arc non continuity F_X
\pgfmathsetmacro{\l}{1.5};% lambda
\pgfmathsetmacro{\mx}{6};% x maximo del plot
%
% densidad
\begin{scope}
%
\pgfmathsetmacro{\sy}{2.5/\l};% y-scaling 
\draw[>=stealth,->] ({-\sx-.25},0)--({\sx*\mx+.25},0) node[right]{\small $x$};
\draw[>=stealth,->] (0,-.1)--(0,{\sy*\l+.25}) node[above]{\small $p_X$};
%
\draw[thick] ({-\sx},0)--(0,0);
\draw (\r,\r) arc (90:270:\r);
\draw[dotted] (0,0)--(0,{\sy*\l}) node[scale=.4]{$\bullet$};
\draw[thick,domain=0:\mx,samples=100] plot ({\x*\sx},{\sy*\l*exp(-\l*\x)});
%
\draw (0,{\l*\sy})--(-.1,{\l*\sy}) node[left,scale=.7]{$\lambda$};
%
\end{scope}
%
%
% reparticion
\begin{scope}[xshift=8.5cm]
%
\pgfmathsetmacro{\sy}{2.5};% y-scaling 
%
\draw[>=stealth,->] (-.6,0)--({\sx*\mx+.25},0) node[right]{\small $x$};
\draw[>=stealth,->] (0,-.1)--(0,{\sy+.25}) node[above]{\small $F_X$};
%
% cumulativa
\draw[thick,domain=0:\mx,samples=100] (-.5,0)--(0,0) plot({\x*\sx},{(1-exp(-\l*\x))*\sy});
%
\draw (0,\sy)--(-.1,\sy) node[left,scale=.7]{$1$};
\end{scope}
%
\end{tikzpicture} \end{center}
% 
\leyenda{Ilustraci\'on  de una densidad  de probabilidad  exponencial (a),  y la
funci\'on de repartici\'on asociada (b), con $\lambda = 1.5$.}
\label{Fig:MP:Exponencial}
\end{figure}

Cuando $\lambda \to +\infty$ la variable tiende a una variable cierta $X = 0$.
\SZ{Esta distribuci\'on aparece...}
% en el conteo de
%conteo de une repetici\'on de  una experiencia de maneja independiente hasta que
%occure un evento de probabilidad $p$; por ejemplo el n\'umero de tiro de un dado
%equilibriado hasta que occurre un ``6'' sigue una ley geometrica de parametro $p
%= \frac16$.



% --------------------------------- Gausiana

\subsubseccion{Distribuci\'on normal o Gaussiana multivariada}

Se denota $X \sim \N(m,\Sigma)$ \ con \  $m \in \Rset^d$ \ y \ $\sigma$ \ matriz
$d  \times  d$  s\'imetrica  definida  positiva. Se  puede  escribir  $X  \egald
\Sigma^{\frac12} N + m$ \ con \ $N \sim \N(0,I)$ \ donde \ $I$ \ es la identidad
y \ $N$ \ es dicha Gausiana estandar o centrada-normalizada. Las caracteristicas
de \ $N \sim \N(0,I)$ \  son las siguientes (se deducen para cualquier Gaussiana
por transformaci\'on lineal; ver secciones anteriores):

\begin{center}
\begin{tabular}
{
|>{\vspace{-2mm}}p{.3\textwidth}|
>{\vspace{-2mm}\hspace{2mm}}p{.5\textwidth}|
}
%
\hline
%
Dominio de definici\'on & $\X = \Rset^d$\\
\hline
%
Densidad de probabilidad & $p_X(x) = \frac{1}{(2 \pi)^{\frac{d}{2}}} e^{-\frac12 x^t x}$\\
\hline
%
%Momentos & $ \Esp\left[ X^k \right] = p^k$\\
%\hline
%
Promedio & $ \Esp[X] = 0$\\
\hline
%
Covarianza & $\Sigma = I$\\
\hline
%
Asimetr\'ia (caso escalar) & $\gamma_X = 0$\\
\hline
%
Curtosis (caso escalar) & $\kappa_X = 0$\\
\hline
%
%Generadora de probabilidad & $G_X(z) = e^{\lambda (z-1)}$ \ para \ $z \in \Cset$\\
%\hline
%
Generadora de momentos & $M_X(u) = e^{u^t u}$ \ para \ $u \in \Cset^d$\\
\hline
%
Funci\'on caracter\'istica & $\Phi_X(\omega) = e^{-\frac12 \omega^t \omega}$\\
\hline
\end{tabular}
\end{center}
%
% modo 0
% Mediana 0

Su densidad de probabilidad y funci\'on  de repartici\'on en el caso escalar son
representadas en la figura Fig.~\ref{Fig:MP:Gaussiana}.
%
\begin{figure}[h!]
\begin{center} \begin{tikzpicture}%[scale=.9]
\shorthandoff{>}
%
\pgfmathsetmacro{\sx}{.75};% x-scaling
\pgfmathsetmacro{\mx}{3.5};% x maximo del plot
%
% Approximation de la cdf gaussienne
\tikzset{declare function={
normcdf(\x)=1/(1 + exp(-0.07056*(\x)^3 - 1.5976*(\x)));
}}
% densidad
\begin{scope}
%
\pgfmathsetmacro{\sy}{2.5*sqrt(2*pi)};% y-scaling 
\draw[>=stealth,->] ({-\sx*\mx-.25},0)--({\sx*\mx+.25},0) node[right]{\small $x$};
\draw[>=stealth,->] (0,-.1)--(0,2.75) node[above]{\small $p_X$};
%
\draw[thick,domain=-\mx:\mx,samples=100] plot ({\x*\sx},{\sy*exp(-.5*\x*\x)/sqrt(2*pi)});
%
\draw (0,{\sy/sqrt(2*pi)})--(-.1,{\sy/sqrt(2*pi)}) node[left,scale=.7]{$\frac1{\sqrt{2 \pi}}$};
%
\end{scope}
%
%
% reparticion
\begin{scope}[xshift=8.5cm]
%
\pgfmathsetmacro{\sy}{2.5};% y-scaling 
%
\draw[>=stealth,->] ({-\sx*\mx-.25},0)--({\sx*\mx+.25},0) node[right]{\small $x$};
\draw[>=stealth,->] (0,-.1)--(0,{\sy+.25}) node[above]{\small $F_X$};
%
% cumulativa
\draw[thick,domain=-\mx:\mx,samples=100] plot({\x*\sx},{\sy*normcdf(\x)});
%
\draw (0,\sy)--(-.1,\sy) node[left,scale=.7]{$1$};
\end{scope}
%
\end{tikzpicture} \end{center}
% 
\leyenda{Ilustraci\'on  de una densidad  de probabilidad  gaussiana escalar estandar (a),  y la
funci\'on de repartici\'on asociada (b).}
\label{Fig:MP:Gaussiana}
\end{figure}

%Cuando $\lambda \to +\infty$ la variable tiende a una variable cierta $X = 0$.
\SZ{Esta distribuci\'on aparece...}
% en el conteo de
%conteo de une repetici\'on de  una experiencia de maneja independiente hasta que
%occure un evento de probabilidad $p$; por ejemplo el n\'umero de tiro de un dado
%equilibriado hasta que occurre un ``6'' sigue una ley geometrica de parametro $p
%= \frac16$.


% --------------------------------- Gamma

\subsubseccion{Distribuci\'on Gamma}

Se denota $X \sim \G(\alpha,\beta)$ \ con \ $\alpha \in \Rset_+^*$, parametro de
forma \ y \ $\beta \in \Rset_+^*$ taza o inversa de escala. Se puede escribir $X
\egald \frac{1}{\beta} G$ \ con  \ $G \sim \G(\alpha,1)$. Las caracteristicas de
\ $X  \sim \G(\alpha,1)$ \ son  las siguientes (se deducen  para cualquier Gamma
$\G(\alpha,\beta)$ por transformaci\'on lineal; ver secciones anteriores):

\begin{center}
\begin{tabular}
{
|>{\vspace{-2mm}}p{.3\textwidth}|
>{\vspace{-2mm}\hspace{2mm}}p{.5\textwidth}|
}
%
\hline
%
Dominio de definici\'on & $\X = \Rset_+$\\
\hline
%
Densidad de probabilidad & $p_X(x) = \frac{x^{\alpha-1} e^{-x}}{\Gamma(\alpha)}$\\
\hline
%
%Momentos & $ \Esp\left[ X^k \right] = p^k$\\
%\hline
%
Promedio & $ \Esp[X] = \alpha$\\
\hline
%
Varianza & $\sigma_X^2 = \alpha$\\
\hline
%
Asimetr\'ia & $\gamma_X = \frac2{\sqrt{\alpha}}$\\
\hline
%
Curtosis & $\kappa_X = \frac6\alpha$\\
\hline
%
%Generadora de probabilidad & $G_X(z) = e^{\lambda (z-1)}$ \ para \ $z \in \Cset$\\
%\hline
%
Generadora de momentos & $M_X(u) = \frac1{(1-u)^\alpha}$ \ para \ $\real{u} < 1$\\
\hline
%
Funci\'on caracter\'istica & $\Phi_X(\omega) = \frac1{(1-\imath\omega)^\alpha}$\\
\hline
\end{tabular}
\end{center}
%
% modo max(\alpha-1,0)
% Mediana no close ver inverse gamma

Su densidad de probabilidad y funci\'on  de repartici\'on son
representadas en la figura Fig.~\ref{Fig:MP:Gamma} para varios $\alpha$.
%
\begin{figure}[h!]
\begin{center} \begin{tikzpicture}%[scale=.9]
\shorthandoff{>}
%
\pgfmathsetmacro{\sx}{.75};% x-scaling
\pgfmathsetmacro{\mx}{8};% x maximo del plot
%
% Approximation de la cdf gaussienne
\tikzset{declare function={
normcdf(\x)=1/(1 + exp(-0.07056*(\x)^3 - 1.5976*(\x)));
}}
%
% densidad
\begin{scope}
%
\pgfmathsetmacro{\sy}{2.5};% y-scaling 
\draw[>=stealth,->] (-.75,0)--({\sx*\mx+.25},0) node[right]{\small $x$};
\draw[>=stealth,->] (0,-.1)--(0,2.75) node[above]{\small $p_X$};
%
%\foreach \a in {1,...,3} {
\draw[thick] (-.5,0)--(0,0);
\draw[thick,dotted,domain=.175:\mx,samples=100] plot ({\x*\sx},{\sy*(\x^(-.5))*exp(-\x)/sqrt(pi)});
\draw[thick,dashed,domain=0:\mx,samples=100] plot ({\x*\sx},{\sy*exp(-\x)});
\draw[thick,dash dot,domain=0:\mx,samples=100] plot ({\x*\sx},{\sy*\x*exp(-\x)});
%\draw[thick,domain=0:\mx,samples=100] plot ({\x*\sx},{\sy*4*\x*sqrt(\x)*exp(-\x)/3/sqrt(pi)});
\draw[thick,domain=0:\mx,samples=100] plot ({\x*\sx},{\sy*\x*\x*exp(-\x)/2});
%}
%
\draw (0,\sy)--(-.1,\sy) node[left,scale=.7]{$1$};
\draw (0,{\sy*exp(-1)})--(-.1,{\sy*exp(-1)}) node[left,scale=.7]{$e^{-1}$};
\draw (0,{\sy*2*exp(-2)})--(-.1,{\sy*2*exp(-2)}) node[left,scale=.7]{$2 \, e^{-2}$};
\draw (\sx,0)--(\sx,-.1) node[below,scale=.7]{$1$};
\draw ({2*\sx},0)--({2*\sx},-.1) node[below,scale=.7]{$2$};
%
\end{scope}
%
%
% reparticion
\begin{scope}[xshift=8.5cm]
%
\pgfmathsetmacro{\sy}{2.5};% y-scaling 
%
\draw[>=stealth,->] (-.75,0)--({\sx*\mx+.25},0) node[right]{\small $x$};
\draw[>=stealth,->] (0,-.1)--(0,{\sy+.25}) node[above]{\small $F_X$};
%
% cumulativa
\draw[thick] (-.5,0)--(0,0);
\draw[thick,dotted,domain=0:\mx,samples=100] plot ({\x*\sx},{(2*normcdf(sqrt(2*\x))-1)*\sy});
\draw[thick,dashed,domain=0:\mx,samples=100] plot ({\x*\sx},{\sy*(1-exp(-\x))});
\draw[thick,dash dot,domain=0:\mx,samples=100] plot ({\x*\sx},{\sy*(1-(1+\x)*exp(-\x))});
\draw[thick,domain=0:\mx,samples=100] plot ({\x*\sx},{\sy*(1-(1+\x+\x*\x/2)*exp(-\x))});
% plot({\x*\sx},{\sy*normcdf(\x)});
%
\draw (0,\sy)--(-.1,\sy) node[left,scale=.7]{$1$};
\end{scope}
%
\end{tikzpicture} \end{center}
\leyenda{Ilustraci\'on de una densidad de probabilidad gamma (a), y la funci\'on
de repartici\'on  asociada (b). $\alpha  = .5$ (linea punteada),  $1$ (guiones),
$2$ (linea mixta) y $3$ (linea llena).}
\label{Fig:MP:Gamma}
\end{figure}

%Cuando $\lambda \to +\infty$ la variable tiende a una variable cierta $X = 0$.
\SZ{Esta distribuci\'on aparece...}
% en el conteo de
%conteo de une repetici\'on de  una experiencia de maneja independiente hasta que
%occure un evento de probabilidad $p$; por ejemplo el n\'umero de tiro de un dado
%equilibriado hasta que occurre un ``6'' sigue una ley geometrica de parametro $p
%= \frac16$.


\aver{


Teorema del l\'imite central \SZ{Ref relaxando la independencia, y versiones con leyes diferentes pero uniformamente acotadas.}

...

\SZ{Familia  exponencial~\cite{Dar35, Koo36,  And70,  Kay93, LehCas98,  Rob07}.;
Wishart como ejemplo de 'matrix variate'}

\SZ{Familia  invariante por rotacion}

%%%%%%%%%%%%%

}

\

\SZ{hablar de  simulaci\'on? Metoto inverso,  mezcla, rejeccion, a traves  de la
  condicional para el caso vectorial?}


\

\SZ{hablar del CLT y prueba}
