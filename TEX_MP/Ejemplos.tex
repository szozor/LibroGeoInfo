%\SZ{Hablar de  convergencia? DIBUJAR en  el plano complejo,  escalar, $\Phi_X$?
%Unas curvas son ``bellas'' ver Sasvari 2013.}

%de   Moivre  (1733):Normal   approximation  to   Binomial  distribution.Laplace
%(1800s):Normal  distribution  as  model  for experimental  error.Gauss  (1800s)
%:Justified   least   squares   estimation   by   assuming   normallydistributed
%errors.Gosset/Student      (1908):Derived      thet-distribution.Gosset/Student
%(1925):“testing  the significance”Fisher  (1925):“level  of significance”Fisher
%(1920s?):Fisher’s   exact  test.Fisher   (1935):“It  seems   to   have  escaped
%recognition that the physical act ofrandomization, which, as has been shown, is
%necessary for  the validity ofany test  of significance, affords  the means, in
%respect of  any particular  bodyof data, of  examining the wider  hypothesis in
%which    no   normality   ofdistribution    is   implied.”Box    and   Andersen
%(1955):Approximation   of   randomization   null   distributions(and   a   long
%discussion).Rodgers (1999):“If Fisher’s ANOVA  had been invented 30 years later
%orcomputers had beens available 30 years sooner, our statistical procedureswould
%probably be less  tied to theoretical distributions as  what they aretoday”36 /
%42 Image of page 193

\seccion{Algunos ejemplos de distribuciones de probabilidad}
\label{Sec:MP:EjemplosDistribucionesProb}

En esta secci\'on, vamos a ver unos ejemplos de distribuciones que se encuentran
frecuentemente   en   problema   pr\'acticos   de   varias   areas   cientificas
(estad\'istica, f\'isica,  ingener\'ia,\ldots).  Daremos las  caracteristicas de
cada ley presentada, as\'i que sus propiedades remarcables. El num\'ero de leyes
de probababilidad es tan importante que es dificil, para no decir imposible, ser
exahustivo. Adem\'as, existen muchas  relaciones entre leyes. En esta secci\'on,
vamos  a  ver algunas  leyes  que  aparecen  frecuentemente, y  nos  enfocaremos
solamente sobre  algunos v\'inculos entre  leyes (los principales).   Para tener
m\'as detalles,  se puede referirse a  los libros especializados  en este marco,
como  por  ejemplo~\cite{Spi76,  JohKot92, JohKot97,  JohKot95:v1,  JohKot95:v2,
KotBal00, GupNag99, FanKot90, SamTaq94}.


% ================================= Variables discretas
\subseccion{Distribuciones de variable discreta}
\label{Ssec:MP:EjemplosDistribucionesDiscretas}


% --------------------------------- Certeza
%\subsubseccion{Variable real con certeza}
\label{Sssec:MP:Certeza}

El caso \ $X = a \in \Rset^d$ \ deterministico ($\forall \, \omega, \: X(\omega)
= a$)  puede ser ver  visto como un  caso degenerado de vector  aleatorio. Visto
as\'i, sus caracter\'isticas principales  vistas en las secciones anteriores son
resimidas en la tabla siguiente:

\begin{caracteristicas}
%
Dominio de definici\'on & $\X = \{ a \}, \quad a \in \Rset^d$\\[2mm]
\hline
%
Distribuci\'on de probabilidad & $p_X(x) = \un_{\{a\}}(x)$\\[2mm]
\hline
%
Promedio & $\displaystyle m_X = a$\\[2mm]
\hline
%
Covarianza~\footnote{Siendo cero la covarianza, no se define ni la asimetr\'ia,
ni la curtosis. Sin embargo, de una manera se puede decir que la ley no es
asim\'etrica, y con cola livianas (no hay colas).} & $\displaystyle \Sigma_X =
0$\\[2mm]
\hline
%
%\modif{Asimetr\'ia} & $\gamma_X = 0$\\[2mm]
%\hline
%%
%Curtosis por exceso & $\displaystyle \widebar{\kappa}_X = - \sum_{i,j=1}^d \Big( \! \left(
%    \un_i \un_i^t \right) \otimes \left(  \un_j \un_j^t \right) +  \left( \un_i
%    \un_j^t \right) \otimes \left( \un_i  \un_j^t \right) + \left( \un_i \un_j^t
%  \right) \otimes \left( \un_j \un_i^t \right) \! \Big)$\\[2mm]
%\hline
%
Generadora de probabilidad & $\displaystyle G_X(z) = \prod_{i=1}^d z_i^{a_i}$ \ para \ $z_i \in \Cset$
\ si $a_i \ge 0$ \ y \ $\Cset_0$ \ si no\\[2mm]
\hline
%
Generadora de momentos & $\displaystyle M_X(u) = e^{a^t u}$ \ para \ $u \in
\Cset^d$\\[2mm]
\hline
%
Funci\'on caracter\'istica & $\displaystyle \Phi_X(\omega) = e^{\imath \, a^t
\omega}$
\end{caracteristicas}

% Momentos & $ \Esp\left[ X^k \right] = p^k$\\[2mm]
% Momento factorial & $\Esp\left[ (X)_k \right] = ?$\\[2mm]

La funci\'on de masa y funci\'on de repartici\'on son representadas en la figura
Fig.~\ref{Fig:MP:Certeza} en el caso escalar.
%
\begin{figure}[h!]
\begin{center} \begin{tikzpicture}%[scale=.9]
\shorthandoff{>}
%
\pgfmathsetmacro{\a}{2};% a
\pgfmathsetmacro{\sy}{2.5};% y-scaling
\pgfmathsetmacro{\r}{.05};% radius arc non continuity F_X
%
% masa
\begin{scope}
%
%
\draw[>=stealth,->] (-.25,0)--({\a+1.75},0) node[right]{\small $x$};
\draw[>=stealth,->] (0,-.1)--(0,{\sy+.25}) node[above]{\small $p_X$};
%
\draw[dotted] (\a,0)--(\a,\sy) node[scale=.4]{$\bullet$};
\draw (0,\sy)--(-.1,\sy) node[left,scale=.7]{$1$};
\draw (\a,0)--(\a,-.1) node[below,scale=.7]{$a$};
%%
\end{scope}
%
%
% reparticion
\begin{scope}[xshift=8.5cm]
%
\draw[>=stealth,->] (-.75,0)--({\a+1.75},0) node[right]{\small $x$};
\draw[>=stealth,->] (0,-.1)--(0,{\sy+.25}) node[above]{\small $F_X$};
%
% cumulativa
\draw[thick] (-.5,0)--(\a,0);
\draw ({\a+\r},\r) arc (90:270:\r);
\draw[dotted] (\a,0)--(\a,\sy);
\draw[thick] (\a,\sy) node[scale=.4]{$\bullet$}--({\a+1.5},\sy);
%
\draw (0,\sy)--(-.1,\sy) node[left,scale=.7]{$1$};
\draw (\a,0)--(\a,-.1) node[below,scale=.7]{$a$};
\end{scope}
%
\end{tikzpicture} \end{center}
% 
\leyenda{Ilustraci\'on  de una  distribuci\'on  cierta (a),  y  la funci\'on  de
  repartici\'on asociada (b).}
\label{Fig:MP:Certeza}
\end{figure}

\

Notar que todo se extiende al caso complejo sin costo adicional.

\

\index{Ley de gran n\'umeros}
El caso  de variables  deterministicas puede ser  visto como caso  degenerado de
variables aleatorias, pero aparecen de vez a cuando tambi\'en como caso l\'imite
de  sucesiones o  series  de  variables aleatorias.   En  particular, aparece  a
trav\'es de  la ley  de gran n\'umeros,  un de  los primeros casos  de l\'imites
estudiado tratando  de variables aleatorias. Historicamente, un  de los primeros
que  estudio  la convergencia  (sin  prueba  e  implicitamente) de  un  promedio
empirico a esta ``ley'' es el  matem\'atico italiano y jugador de dados y cartas
Gerolamo Cardano en el siglo~{XVI}, en su libro sobre los juegos de azar escrito
en  1564   (ver  introducci\'on  y~\cite{Car63,   Bel05}  o~\cite[Cap.~4]{Hal90}
o~\cite[Cpa.~3]{Mlo08}).  En  otras palabras,  explic\'o que la  precisi\'on las
estadisticas empiricos se mejora con el  n\'umero de datos, lo que es nada m\'as
que, en palabras,  el resultado de la ley dicha de  gran n\'umeros. En palabras,
saliendo  de  variables  aleatoria  independientes  de misma  ley,  el  promedio
empirico tiende a  la media donde enfatisaremos en que  sentido hay que entender
``tiende a''. Tal  convergencia fue estudiada y probada  mucho m\'as tarde, bajo
el  impulso  del suizo  Jacob  Bernoulli~\cite[Pars  4]{Ber1713} (ver  tambi\'en
Montmort~\cite{Mon13, Pea25})  en el  contexto de variables  binarias, conocidos
hoy  como  variables  de Bernoulli  (ver  subecci\'on~\ref{Sssec:MP:Bernoulli}).
Luego,  el  teorema  fue   mejorado  por  ejemplo  por  de  Moivre~\cite{Moi56},
Laplace~\cite{Lap20} o Poisson~\cite{Poi37}, yendo  m\'as all\'a de solamente la
convergencia del  promedio empirico a la  media.  El teorema  fue ampliado m\'as
all\'a  de la  ley  binomial como  suma  de variables  de  Bernoulli (ver  m\'as
adelante),    por    varios    autores   tales    que    Chebyshev~\cite{Tch46},
Markov~\cite{Mar13},    Borel~\cite{Bor09:12}     ,    Kinchin~\cite{Kin29}    o
Kolmogoroff~\cite{Kol30} entre otros (ver~\cite{Sen13} y referencias).

Formalmente,  las dos  versiones usuales  del  teoremas de  formaliza de  manera
siguiente  (ver  tambi\'en~\cite{Fel71,  Shi84,  AshDol99,  JacPro03,  AthLah06,
  Bil12, Coh13}).

\begin{teorema}[Ley debil de los gran n\'umero]
  Sea  \ $\left\{ X_k  \right\}_{k \in  \Nset_0}$ \  una sucesi\'on  de vectores
  aleatorios  independientes e identicamente  distribuidas (iid),  admitiendo una
  media  $m =  \Esp[X_k]$  \ y  sea  \ $\displaystyle  \widebar{X}_n =  \frac1n
  \sum_{k=1}^n X_k$ \ el promedio empirico. Entonces
  %
  \[
  \widebar{X}_n \limitP{n \to +\infty} m
  \]
  %
  donde $\limitP{}$ significa que el l\'imite es en probabilidad, \ie
  %
  \[
  \forall  \:  \varepsilon  >0,  \quad  \lim_{n  \to  +\infty}  P\left(  \left\|
      \widebar{X}_n - m \right\| > \varepsilon \right) = 0
  \]
\end{teorema}
\begin{proof}
  Una    prueba    sencilla   se    apoya    en    el    teorema   de    Markov,
  Cor.~\ref{Cor:MP:Markov},  cuando  los $X_k$  admiten  una  covarianza. De  la
  independencia, es  sencillo ver que  \ $\Cov\left[ \widebar{X}_n \,  \right] =
  \frac1n \Cov\left[ X_1 \right]$. Entonces,
  %
  \[
  P\left(  \left\|  \widebar{X}_n  -  m  \right\|  >  \varepsilon  \right)  \le
  \frac{\Esp\left[ \left\|  \widebar{X}_n - m\right\|^2 \right]}{\varepsilon^2}
  =  \frac{\Tr\left(   \Cov\left[  X_1  \right]   \right)}{n  \,  \varepsilon^2}
  \xrightarrow[n \to \infty]{} 0
  \]
  %
  lo que cierra la prueba.

  De hecho,  no es necesario  que los $X_k$  admitan una covarianza.  Una prueba
  alternativa    se   apoya   sobre    la   funci\'on    caracter\'istica.   Del
  teorema~\ref{Teo:MP:PropiedadesFuncionCaracteristica},   se   obtiene  de   la
  independencia
  %
  \[
  \Phi_{\widebar{X}_n}(\omega)   =  \left(   \Phi_{X_1}\left(  \frac{\omega}{n}
    \right) \right)^n = \left( 1 + \frac{\imath}{n} m^t \omega + o\left( \left\|
        \frac{\omega}{n} \right\| \right) \right)^n \xrightarrow[n \to \infty]{}
  e^{\imath m^t \omega}
  \]
  %
  En otros t\'erminos, la funci\'on caracter\'istica de $\widebar{X}_n$ tiende a
  la   de  $m$   punto  a   punto.  Se   usa  el   teorema  de   continuidad  de
  L\'evy~\cite{AshDol99, AthLah06, Bil12, Coh13}, no probado en este libro, para
  concluir  que  \  $\widebar{X}_n$  \  tiende  en distribuci\'on  a  \  $m$,  y
  equivalentemente tiende en probabilidad.
\end{proof}
%
Pasando,  de  la primera  prueba,  se  puede notar  que  se  puede debilitar  la
hypotesis  de  independencia,  y  a\'un  la  de misma  ley  para  los  $X_k$,  a
condici\'on de que $\Cov\left[ \widebar{X}_n \right]$ \ tiende a cero cuando $n
\to +\infty$ (por ejemplo, queda valide con la independencia y varianza acotada).

En palabras, el teorema traduce el  pensamiento de Cardano, que es que cualquier
sea el rayo de la bola centrada en $m$, cuando crece el n\'umero de variables en
el promedio  empirico, la probabilidad de  que este promedio sea  afuera de esta
bola tiende a cero.

De hecho, como  para series de funciones (lo que  son las variables aleatorias),
hay varias manera  de converger. Una m\'as fuerte  es conocido como convergencia
casi  siempre,  dando  lugar a  la  ley  dicha  fuerte  de los  gran  n\'umeros.
Historicamente, este teorema es dada en  el caso escalar, pero se extiende en el
caso  vectorial.    No  daremos  la   prueba,  que  se  encuentra   por  ejemplo
en~\cite[Teo.~6.4.2]{Gre63}    en   el   caso    vectorial,   o    entre   otros
en~\cite[Teo.~22.1]{Bil12} en el caso escalar.
%
\begin{teorema}[Ley fuert de los gran n\'umero o teorema de Kolmogorov-Khintchine]
%
  Sea  \ $\left\{ X_k  \right\}_{k \in  \Nset_0}$ \  una sucesi\'on  de vectores
  aleatorios independientes  e identicamente distribuidas  (iid), admitiendo una
  media  $m =  \Esp[X_k]$ \  y  tales que  tambi\'en \  $\Esp\left[ \left\|  X_k
    \right\| \right] <  \infty$, y sea \ $\displaystyle  \widebar{X}_n = \frac1n
  \sum_{k=1}^n X_k$ \ el promedio empirico. Entonces
  %
  \[
  \widebar{X}_n \limitcs{n \to +\infty} m
  \]
  %
  donde $\limitcs{}$ significa que el l\'imite  es casi siempre (o a veces dicho
  ``con probabiludad uno''), \ie
  %
  \[
  P\left(    \lim_{n \to +\infty}   \widebar{X}_n =  m \right) = 1
  \]
  %
  o, dicho  de otra  manera, la medida  del conjuto  $\{ \omega \tq  \lim_{n \to
    +\infty} \widebar{X}_n \ne m \}$ es cero.
\end{teorema}
%\begin{proof}
%Ver~\cite[Teo.~6.4.2]{Gre63}
%\end{proof}
%

Esta versi\'on es dicha fuerte porque la convergencia casi siempre implica la en
probabilidad~\cite{Fel71, Shi84, AshDol99, JacPro03, AthLah06, Bil12, Coh13}. Se
puede debilitar un paso m\'as  las condiciones (ej. indemendencia, etc.) pero va
m\'as all\'a  de la  meta de esta  secci\'on. El  lector se podr\'ea  referir en
libros  especializados,  por  ejemplo~\cite{Fel71,  Shi84,  AshDol99,  JacPro03,
  AthLah06, Bil12, Coh13}.

Una consecuencia de la ley fuerte de gran n\'umeros es conocido como theorema de
Borel. Dice  que, en el  contexto de variables  discretas, si una  experienca se
repite de  manera independiente  un gran n\'umero  de veces, la  proporci\'on de
ocurencia de un  estado tiendo a su probabilidad  de ocurencia (con probabilidad
uno).  Se podr\'a  referir  por  ejemplo a~\cite{Wen91}  para  tener una  prueba
``moderna''.

%\SZ{Poner ac\'a la ley de los gran n\'umeros? M\'as notas historicas.}


% --------------------------------- Uniforme discreta
%\subsubseccion{Ley Uniforme sobre un ``intervalo'' de $\Zset$}
\label{Sssec:MP:UniformeDiscreta}

Se denota $X \, \sim \, \U\{ a \; b \}$ \ con $(a,b) \in \Zset^2, \: b \ge
a$.  Las caracter\'isticas de \ $X$ \ son las siguientes:

\begin{caracteristicas}
%
Par\'ametros & $(a,b) \in \Zset^2, \: b \ge a$\\[2mm]
\hline
%
Dominio de definici\'on & $\X = \{ a  \; a+1 \; \ldots \; b \}$\\[2mm]
\hline
%
Distribuci\'on de probabilidad & $p_X(x) = \frac1{b-a+1}$\\[2mm]
\hline
%
Promedio & $\displaystyle m_X = \frac{a+b}{2}$\\[2mm]
\hline
%
Varianza & $\displaystyle \sigma_X^2 = \frac{(b-a) (b-a+2)}{12}$\\[2mm]
\hline
%%
\modif{Asimetr\'ia} & $\gamma_X = 0$\\[2mm]
\hline
%
Curtosis por exceso & $\displaystyle \widebar{\kappa}_X = -\frac65 \frac{(b-a)
(b-a+2)+2}{(b-a) (b-a+2)}$\\[2mm]
\hline
%
Generadora de probabilidad & $\displaystyle G_X(z) = \frac{z^a-z^{b+1}}{1-z}$ \
para~\footnote{En el caso l\'imite \ $z \to 1$, \ $\lim_{z \to 1} \frac{ z^a -
z^{b+1}}{1-z} = b+1-a$} \ $z \in \Cset$ \ si $a \ge 0$ \ y \ $\Cset^*$ \ sino\\[2mm]
\hline
%
Generadora de momentos & $\displaystyle M_X(u) = \frac{ e^{a u} - e^{(b+1)
u}}{1-e^u}$ \ para~\footnote{En el caso l\'imite \ $u \to 0$, \ $\lim_{u \to 0}
\frac{ e^{a u} - e^{(b+1) u}}{1-e^u} = b+1-a$, y similarmente para la funci\'on
caracter\'istica.}  \ $u \in \Cset$\\[2mm]
\hline
%
Funci\'on caracter\'istica & $\displaystyle  \Phi_X(\omega) = \frac{ e^{\imath a
\omega} - e^{\imath (b+1) \omega}}{1-e^{\imath \omega}}$
\end{caracteristicas}

% Momentos & $ \Esp\left[ X^k \right] = p^k$\\[2mm]
% Momento factorial & $\Esp\left[ (X)_k \right] = ?$\\[2mm]
% modo 0
% Mediana \ln(2)/\lambda
% CDF 1-e^{-\lambda x}

La distribuci\'on  de masa de probabilidad  y funci\'on de  repartici\'on de una
variable uniforme  \ $\U\{  a \; b  \}$ \ son  representadas en  la figura
Fig.~\ref{Fig:MP:UniformeDiscreta}.
%
\begin{figure}[h!]
\begin{center} \begin{tikzpicture}%[scale=.9]
\shorthandoff{>}
%
\pgfmathsetmacro{\sx}{.75};% x-scaling
\pgfmathsetmacro{\r}{.05};% radius arc non continuity F_X
\pgfmathsetmacro{\n}{6};% n de la uniforme
\pgfmathsetmacro{\m}{\n-1};
%
% masa
\begin{scope}
%
%
\pgfmathsetmacro{\sy}{2.5};% y-scaling 
\draw[>=stealth,->] (-.25,0)--({\sx*(\n+.5)+.25},0) node[right]{\small $x$};
\draw[>=stealth,->] (0,-.1)--(0,{\sy+.25}) node[above]{\small $p_X$};
%
\foreach \k in {1,...,\n} {
\draw[dotted] ({\k*\sx},0)--({\k*\sx},\sy) node[scale=.4]{$\bullet$};
\draw ({\k*\sx},0)--({\k*\sx},-.1) node[below,scale=.7]{$\k$};
}
\draw (0,\sy)--(-.1,\sy) node[left,scale=.7]{$\frac1\n$};
%%
\end{scope}
%
%
% reparticion
\begin{scope}[xshift=8.5cm]
%
\pgfmathsetmacro{\sy}{2.5};% y-scaling 
%
\draw[>=stealth,->] ({-\sx/2-.25},0)--({\sx*(\n+1.5)+.25},0) node[right]{\small $x$};
\draw[>=stealth,->] (0,-.1)--(0,{\sy+.25}) node[above]{\small $F_X$};
%
% cumulativa
\draw[thick] ({-\sx/2},0)--(\sx,0);
\draw ({\sx+\r},\r) arc (90:270:\r);
\draw (0,0)--(0,-.1) node[below,scale=.7]{$0$};
%
\foreach \k in {1,...,\m} {
\draw ({\k*\sx},0)--({\k*\sx},-.1) node[below,scale=.7]{$\k$};
\draw[thick] ({\k*\sx},{\k*\sy/\n}) node[scale=.4]{$\bullet$}--({(\k+1)*\sx},{\k*\sy/\n});
\draw ({(\k+1)*\sx+\r},{\k*\sy/\n+\r}) arc (90:270:\r);
\draw[dotted] ({\k*\sx},{(\k-1)*\sy/\n})--({\k*\sx},{\k*\sy/\n});
}
\draw ({\n*\sx},0)--({\n*\sx},-.1) node[below,scale=.7]{$\n$};
\draw[thick] ({\n*\sx},\sy) node[scale=.4]{$\bullet$}--({(\n+1.5)*\sx},\sy);
\draw[dotted] ({\n*\sx},{(\n-1)*\sy/\n})--({\n*\sx},\sy);
%%
\draw (0,\sy)--(-.1,\sy) node[left,scale=.7]{$1$};
\end{scope}
%
\end{tikzpicture} \end{center}
% 
\leyenda{Ilustraci\'on  de  una densidad  de  probabilidad  uniforme  (a), y  la
  funci\'on  de repartici\'on  asociada (b).  $a =  1,  \: b  = 6$  \ (ej.  dado
  equilibriado).}
\label{Fig:MP:UniformeDiscreta}
\end{figure}

Cuando \ $b = a$, la variable tiende a una variable cierta \ $X = a$.
%
%La ley tiene propiedades de reflexividad obvia:
%%
%\begin{lema}[Reflexividad]\label{Lem:MP:ReflexividadUniformeDiscreta}
%%
%  Si \ $X \sim \U\{ a ; b \}$\ entonces
%  %
%  \[
%  a+b-X \sim \U\{ a ; b \}
%  \]
%\end{lema}
%\begin{proof}
%  Sea  $Y  =  a+b-X$. Entonces,  $P_Y(y)  =  P(a+b-X  =  y)  = P(X  =  a+b-y)  =
%  \frac{1}{b-a+1} \, \un_{\{ a \; \ldots \; b \}}(y)$.
%\end{proof}

La  distribuci\'on  uniforme  aparece  por   ejemplo  en  el  tiro  de  un  dado
equilibriado con \ $a = 1, \: b = 6$.



% --------------------------------- Bernoulli
%\subsubseccion{Ley de Bernoulli}
\label{Sssec:MP:Bernoulli}

Esta ley aparece  cuando se hace una experiencia con dos  estados posibles, tipo un
tiro  de  moneda.   Apareci\'o en  trabajos  muy  antiguos,  entre otros  el  de
J.  Bernoulli  tratando  de  la  ley  de  gran  n\'umeros~\cite{Ber1713,  Hal90,
  DavEdw01}.

Se  denota \  $X \,  \sim \,  \B(p)$ \  con \  $p \in  [0 \;  1]$ \  y sus
caracter\'isticas son las siguientes:

\begin{caracteristicas}
%
Dominio de definici\'on & $\X = \{ 0 \; 1 \}$\\[2mm]
\hline
%
Par\'ametro & $p \in [ 0 \; 1 ]$\\[2mm]
\hline
%
Distribuci\'on de probabilidad & $p_X(1) = 1 - p_X (0) = p$\\[2mm]
\hline
%
Promedio & $ m_X = p$\\[2mm]
\hline
%
Varianza & $\sigma_X^2 = p \, (1-p)$\\[2mm]
\hline
%
\modif{Asimetr\'ia} & $\displaystyle \gamma_X =  \frac{1 - 2 \, p}{\sqrt{p \, (1-p)}}$ \quad para \ $p \notin \{ 0 \; 1 \}$ (ver m\'as adelante)\\[2mm]
\hline
%
Curtosis por exceso & $\displaystyle \widebar{\kappa}_X = \frac{1 - 6 \, p + 6
\, p^2}{p \, (1-p)}$ \quad para \ $p \notin \{ 0 \; 1 \}$ (ver m\'as adelante)\\[2mm]
\hline
%
Generadora de probabilidad & $G_X(z) = 1 - p + p z$ \quad sobre \ $\Cset$\\[2mm]
\hline
%
Generadora de momentos & $M_X(u) = 1 - p + p \, e^u$ \quad sobre \ $\Cset$\\[2mm]
\hline
%
Funci\'on caracter\'istica & $\Phi_X(\omega) = 1 - p + p \, e^{\imath \omega}$
\end{caracteristicas}


% Momentos & $ \Esp\left[ X^k \right] = p^k\\[2mm]
% Momento factorial & $\Esp\left[ (X)_k \right] = p^k \un_{\{0 \, , \, 1 \}}(k)$\\[2mm]

Su masa  de probabilidad  y funci\'on de  repartici\'on son representadas  en la
figura Fig.~\ref{Fig:MP:Bernoulli}.
%
\begin{figure}[h!]
\begin{center} \begin{tikzpicture}%[scale=.9]
\shorthandoff{>}
%
\pgfmathsetmacro{\sx}{2};% x-scaling
\pgfmathsetmacro{\r}{.05};% radius arc non continuity F_X
\pgfmathsetmacro{\p}{1/3};% probabilidad p
% masa
\begin{scope}
%
\pgfmathsetmacro{\sy}{2/max(\p,1-\p)};% y-scaling
%
\pgfmathsetmacro{\ss}{\sy*(1-\p)};
\draw[>=stealth,->] (-.5,0)--({\sx+.75},0) node[right]{\small $x$};
\draw[>=stealth,->] (0,-.15)--(0,2.5) node[above]{\small $p_X$};
%
\draw (0,-.1) node[below,scale=.7]{$0$} --(0,0);
\draw[dotted] (0,0)--(0,{\sy*(1-\p)}) node[scale=.7]{$\bullet$};
\draw (0,{\sy*(1-\p)})--(-.1,{\sy*(1-\p)}) node[left,scale=.7]{$1-p$};
%
\draw (\sx,-.1) node[below,scale=.8]{\small $1$} --(\sx,0);
\draw[dotted] (\sx,0)--(\sx,{\sy*\p}) node[scale=.7]{$\bullet$};
\draw (0,{\sy*\p})--(-.1,{\sy*\p}) node[left,scale=.7]{\small $p$};
%
\node at ({(\sx+.75)/2},-1) [scale=.9]{(a)};
\end{scope}
%
%
% reparticion
\begin{scope}[xshift=7cm]
%
\pgfmathsetmacro{\sy}{2};% y-scaling 
%
\draw[>=stealth,->] (-.5,0)--({\sx+1.5},0) node[right]{\small $x$};
\draw[>=stealth,->] (0,-.15)--(0,{\sy+.5}) node[above]{\small $F_X$};
%
\draw (0,0)--(0,-.1) node[below,scale=.7]{$0$};
\draw (\sx,0)--(\sx,-.1) node[below,scale=.7]{$1$};
\draw (0,{\sy*(1-\p)})--(-.1,{\sy*(1-\p)}) node[left,scale=.7]{$1-p$};
\draw (0,\sy)--(-.1,\sy) node[left,scale=.7]{$1$};
%
\draw[thick](-.25,0)--(0,0);
\draw ({0+\r},\r) arc (90:270:\r);
%
\draw[dotted] (0,0)--(0,{\sy*(1-\p)});
\draw[thick](0,{\sy*(1-\p)}) node[scale=.7]{$\bullet$}--(\sx,{\sy*(1-\p)});
\draw ({\sx+\r},{\r+\sy*(1-\p)}) arc (90:270:\r);
%
\draw[dotted] (\sx,{\sy*(1-\p)})--(\sx,\sy);
\draw[thick](\sx,\sy) node[scale=.7]{$\bullet$}--({\sx+1},\sy);
%\draw ({\sx+\r},{\r+\sy*(1-\p)}) arc (90:270:\r);
%
\node at ({(\sx+1.5)/2},-1) [scale=.9]{(b)};
\end{scope}
%
\end{tikzpicture} \end{center}
%
\leyenda{Ilustraci\'on de una distribuci\'on de probabilidad de Bernoulli (a), y
  la funci\'on de repartici\'on asociada (b), con $p = \frac13$.}
\label{Fig:MP:Bernoulli}
\end{figure}

Notar que cuando $p = 0$ (resp. $p = 1$) la variable es cierta $X = 0$ (resp. $X
= 1$). En estos casos, nuevamente,  siendo la varianza cero, no se puede definir
ni asimetr\'ia  (pero no  hay asimetr\'ia),  ni curtosis (pero  la ley  no tiene
colas, \ie colas livianas), como ya lo hemos visto anterioramente.

Se  notar\'a  tambi\'en  que  la   ley  de  Bernoulli  tiene  una  propiedad  de
reflexividad trivial:
%
\begin{lema}[Reflexividad]
\label{Lem:MP:ReflexividadBernoulli}
%
  Sea \ $X \, \sim \, \B(p)$. Entonces
  %
  \[
  1-X \, \sim \, \B(1-p)
  \]
  %
\end{lema}
\begin{proof}
El resultado es inmediato de $P(1-X = 1) = P(X = 0) = 1-p$.
\end{proof}


% --------------------------------- Binomial
%\subsubseccion{Ley binomial}
\label{Sssec:MP:Binomial}

Esta ley  apareci\'o en  trabajos muy antiguos,  de nuevo y  naturalmente, entre
otros, en de J. Bernoulli  en 1713~\cite{Ber1713, Hal90, DavEdw01}.  Se la puede
ver como  una extension de  la ley  de Bernoulli a  \ $n \ge  1$ \ tiros  de una
moneda, contando por ejemplo cuantas veces aparecen una cara.

Se denota \ $X \,  \sim \, \B(n,p)$ \ con \ $n \in \Nset^*$,  \quad $p \in [0 \;
1]$ \ y sus caracter\'isticas son las siguientes:

\begin{caracteristicas}
%
Dominio de definici\'on & $\X = \{ 0 \; \ldots \; n \}$\\[2mm]
\hline
%
Par\'ametros & $n  \in \Nset^*,  \quad p \in [0  \;
1]$\\[2mm]
\hline
%
Distribuci\'on de probabilidad & \protect$\displaystyle p_X(x) = \bino{n}{x} \, p^x
(1-p)^{n-x}$\protect\\[2mm]
\hline
%
Promedio & $ m_X = n p$\\[2mm]
\hline
%
Varianza & $\sigma_X^2 = n p (1-p)$\\[2mm]
\hline
%
\modif{Asimetr\'ia} & $\displaystyle \gamma_X = \frac{1 - 2 p}{\sqrt{n p \, (1-p)}}$ \quad para \ $p \not\in \{ 0 \; 1 \}$ \ (ver m\'as adelante)\\[2mm]
\hline
%
Curtosis por exceso & $\displaystyle \widebar{\kappa}_X = \frac{1 - 6 \, p + 6
\, p^2}{n \, p \, (1-p)} $ \quad para \ $p \not\in \{ 0 \; 1 \}$ \ (ver m\'as adelante)\\[2mm]
\hline
%
Generadora  de probabilidad  &  $\displaystyle  G_X(z) =  \left(  1 -  p  + p  z
\right)^n$ \quad sobre \ $\Cset$\\[2mm]
\hline
%
Generadora  de momentos  &  $\displaystyle  M_X(u) =  \left(1  - p  +  p \,  e^u
\right)^n$ \quad sobre \ $\Cset$\\[2mm]
\hline
%
Funci\'on caracter\'istica  & $\displaystyle \Phi_X(\omega) =  \left( 1 -  p + p
\, e^{\imath \omega} \right)^n$
\end{caracteristicas}

% Momentos & $ \Esp\left[ X^k \right] = ??\\[2mm]
% Momento factorial & $\Esp\left[ (X)_k \right] = 
% \frac{n!}{(n-k)!} p^k \un_{\{ 0 \, , \, \ldots \, , \, n \}}(k)$\\[2mm]
% Modo $\left\lfloor (n+1) p \right\rfloor$
% Mediana $\left\lfloor n p \right\rfloor$ o $\left\lceil n p \right\rceil
% CDF	$I_{1-p}(n-k,k+1)$ regularized incomplete beta function

Su masa  de probabilidad  y funci\'on de  repartici\'on son representadas  en la
figura Fig.~\ref{Fig:MP:Binomial}.
%
\begin{figure}[h!]
\begin{center} \begin{tikzpicture}%[scale=.9]
\shorthandoff{>}
%
\pgfmathsetmacro{\sx}{.75};% x-scaling
\pgfmathsetmacro{\r}{.05};% radius arc non continuity F_X
\pgfmathsetmacro{\p}{1/3};% probabilidad p
\pgfmathsetmacro{\n}{6};% numero n de la binomial
\pgfmathsetmacro{\q}{floor((\n+1)*\p)};% modo de la binomial
\pgfmathsetmacro{\m}{factorial(\n)/factorial(\q)/factorial(\n-\q)*(\p^\q)*((1-\p)^(\n-\q))};% maximo de la binomial
% masa
\begin{scope}
%
\pgfmathsetmacro{\sy}{2.5/\m};% y-scaling 
\draw[>=stealth,->] (-.25,0)--({\sx*\n+.25},0) node[right]{\small $x$};
\draw[>=stealth,->] (0,-.1)--(0,{\sy*\m+.25}) node[above]{\small $p_X$};
%
\pgfmathsetmacro{\b}{(1-\p)^\n};% coeficiente binomial por la probabilidad
%
\foreach \k in {0,...,\n} {
\draw ({\k*\sx},0)--({\k*\sx},-.1) node[below,scale=.7]{\k};
\draw[dotted] ({\k*\sx},0)--({\k*\sx},{\sy*\b}) node[scale=.7]{$\bullet$};
%
\pgfmathsetmacro{\bl}{\b*\p*(\n-\k)/((\k+1)*(1-\p))};\global\let\b\bl;% proba actualizado
}
\draw (0,{((1-\p)^\n)*\sy})--(-.1,{((1-\p)^\n)*\sy}) node[left,scale=.7]{$(1-p)^n$};
\draw (0,{\n*\p*((1-\p)^(\n-1))*\sy})--(-.1,{\n*\p*((1-\p)^(\n-1))*\sy}) node[left,scale=.7]{$n p (1-p)^{n-1}$};
%
\end{scope}
%
%
% reparticion
\begin{scope}[xshift=8.5cm]
%
\pgfmathsetmacro{\sy}{2.5};% y-scaling 
%
\draw[>=stealth,->] (-.6,0)--({\sx*(\n+.5)+.5},0) node[right]{\small $x$};
\draw[>=stealth,->] (0,-.1)--(0,{\sy+.25}) node[above]{\small $F_X$};
%
\pgfmathsetmacro{\b}{(1-\p)^\n};% coeficiente binomial por la probabilidad
\pgfmathsetmacro{\c}{(1-\p)^\n};% cumulativa binomial por la probabilidad
%
% cumulativa x < 0
\draw (0,0)--(0,-.1) node[below,scale=.7]{0};
\draw[thick] (-.5,0)--(0,0);
\draw (\r,\r) arc (90:270:\r);
%
% cumulativa x de 0 a n-1
\foreach \k in {1,...,\n} {
\draw ({\k*\sx},0)--({\k*\sx},-.1) node[below,scale=.7]{\k};
\draw[thick]({(\k-1)*\sx},{\sy*\c}) node[scale=.7]{$\bullet$}--({\k*\sx},{\sy*\c});
\draw ({\k*\sx+\r},{\sy*\c+\r}) arc (90:270:\r);
\draw[dotted] ({(\k-1)*\sx},{(\c-\b)*\sy})--({(\k-1)*\sx},{\c*\sy});
%
\pgfmathsetmacro{\bl}{\b*\p*(\n-\k+1)/(\k*(1-\p))};\global\let\b\bl;% proba actualizado
\pgfmathsetmacro{\cl}{\c+\b};\global\let\c\cl;% cumulativa actualizada
}
%
% cumulativa x > n
\draw[dotted] ({\n*\sx},{(1-\b)*\sy})--({\n*\sx},\sy);
\draw[thick]({\n*\sx},\sy) node[scale=.7]{$\bullet$}--({(\n+.5)*\sx},\sy);
%
\draw (0,{((1-\p)^\n)*\sy})--(-.1,{((1-\p)^\n)*\sy}) node[left,scale=.7]{$(1-p)^n$};
\draw (0,{(\n*\p+1-\p)*((1-\p)^(\n-1))*\sy})--(-.1,{(\n*\p+1-\p)*((1-\p)^(\n-1))*\sy}) node[left,scale=.7]{$(1-p+np) (1-p)^{n-1}$};
\draw (-.2,{((\n*\p+1-\p)*((1-\p)^(\n-1))+1)/2*\sy}) node[scale=.7]{$\vdots$};
\draw (0,\sy)--(-.1,\sy) node[left,scale=.7]{$1$};
\end{scope}
%
\end{tikzpicture} \end{center}
%
\leyenda{Ilustraci\'on de una distribuci\'on  de probabilidad binomial (a), y la
  funci\'on de repartici\'on asociada (b), con $n = 6$, \quad $p = \frac13$.}
\label{Fig:MP:Binomial}
\end{figure}

\SZ{Otros ilustraciones para otros $p$?}

Cuando  $n  = 1$,  se  recupera  la lei  de  Bernoulli  $\B(p) \equiv  \B(1,p)$.
Ad\'emas, se muestra  sencillamente usando la generadora de  probabilidad que
%
\begin{lema}
\label{Lem:BinomialSumaBernoulli}
%
  Sean \  $X_i \,  \sim \, \B(p),  \quad i  = 1, \ldots  , n$  \ independientes,
  entonces
  %
  \[
  \sum_{i=1}^n X_i \, \sim \, \B(n,p)
  \]
\end{lema}
%
De este resultado,  se puede notar que, por  ejemplo, le distribuci\'on binomial
aparece en el conteo de eventos independientes de misma probabilidad entre $n$.

Tambi\'en,  la ley binomial  tiene una  propiedad de  reflexividad, consecuencia
directa de la de Bernoulli:
%
\begin{lema}[Reflexividad]
\label{Lem:MP:ReflexividadBinomial}
%
  Sea \ $X \, \sim \, \B(n,p)$. Entonces
  %
  \[
  n-X \, \sim \, \B(n,1-p)
  \]
  %
\end{lema}
%
\begin{proof}
  El  resultado es  inmediato  de la  propiedad  de reflexividad  de  la ley  de
  Bernoulli,                           conjuntamente                          al
  lema~\ref{Lem:BinomialSumaBernoulli}. Alternativamente,  se nota que  $P(n-X =
  x) = P(X = n-x) = \bino{n}{n-x} p^{n-x} (1-p)^x = \bino{n}{x} (1-p)^x p^{n-x}$
  \ notando que $\bino{n}{n-x} = \bino{n}{x}$.
\end{proof}
%
Si  tomamos  el  ejemplo  de  una  moneda  que se  tira  $n$  veces  de  maneras
independientes, con  probabilidad $p$ que  aparezca una cara, $X$  representa el
n\'umero de caras tiradas. Entonces, $n-X$  es el n\'umero de secas: en $n-X$ se
intercambian los roles de la cara y de la seca.

Nota que cuando $p = 0$ (resp. $p = 1$) la variable es cierta $X = 0$ (resp.  $X
= n$).  En estos casos, nuevamente,  siendo la varianza cero, no se puede definir
ni asimetr\'ia  (pero no  hay asimetr\'ia),  ni curtosis (pero  la ley  no tiene
colas, \ie colas livianas), como ya lo hemos visto anterioramente.



% --------------------------------- Binomial negativa
%\subsubseccion{Ley Binomial negativa}
\label{Sssec:MP:BinomialNegativa}

Se denota \  $X \, \sim \, N\B(r,p)$ \  con \ $r \in \Nset^*, \quad  p \in [0 \;
1)$ \ y sus caracter\'isticas son las siguientes:

\begin{caracteristicas}
%
Dominio de definici\'on & $\X = \Nset$\\[2mm]
\hline
%
Parametros & $r  \in \Nset^*,  \quad p \in [0  \;
1)$\\[2mm]
\hline
%
Distribuci\'on  de  probabilidad  &  \protect$\displaystyle  p_X(k)  =  \bino{k+r-1}{k}  p^k
(1-p)^r$\protect\\[2mm]
\hline
%
Promedio & $\displaystyle m_X = \frac{r \, p}{1-p}$\\[2mm]
\hline
%
Varianza & $\displaystyle \sigma_X^2 = \frac{r \, p}{(1-p)^2}$\\[2mm]
\hline
%
\modif{Sesgo} & $\displaystyle \gamma_X = \frac{1 + p}{\sqrt{r \, p}}$\\[2mm]
\hline
%
Curtosis por exceso & $\displaystyle \widebar{\kappa}_X = \frac{1 + 4 \, p +
p^2}{r \, p} $\\[2mm]
\hline
%
Generadora  de probabilidad  &  $\displaystyle  G_X(z) =  \left(  \frac{1 -  p}{1  - p \, z}
\right)^r$ \ para \ $|z| < p^{-1} $\\[2mm]
\hline
%
Generadora de momentos & $\displaystyle M_X(u) = \left( \frac{1 - p}{1 - p \,
e^u } \right)^r$ \ para \ $\real{u} < - \ln p$\\[2mm]
\hline
%
Funci\'on caracter\'istica & $\displaystyle \Phi_X(\omega) = \left( \frac{1 -
p}{1 - p \, e^{i \omega} } \right)^r$
\end{caracteristicas}

% Momentos & $ \Esp\left[ X^k \right] = ??\\[2mm]
% Momento factorial & $\Esp\left[ (X)_k \right] = 
% \frac{(r+k-1)!}{(r-1)!} \left( \frac{p}{1-p} \right)^k$\\[2mm]
% Modo $\left\lfloor (n+1) p \right\rfloor$
% Mediana $\left\lfloor n p \right\rfloor$ o $\left\lceil n p \right\rceil
% CDF	$I_{1-p}(n-k,k+1)$ regularized incomplete beta function

Su masa  de probabilidad  y funci\'on de  repartici\'on son representadas  en la
figura Fig.~\ref{Fig:MP:BinomialNegativa}.
%
\begin{figure}[h!]
\begin{center} \begin{tikzpicture}%[scale=.9]
\shorthandoff{>}
%
\pgfmathsetmacro{\sx}{.45};% x-scaling
\pgfmathsetmacro{\r}{.05};% radius arc non continuity F_X
\pgfmathsetmacro{\p}{3/5};% probabilidad p de suceso
\pgfmathsetmacro{\rp}{3};% numero r de fracascos
\pgfmathsetmacro{\n}{12};% numero maximo a dibujar
\pgfmathsetmacro{\q}{max(floor((\rp-1)*\p/(1-\p)),0)};% modo de la binomial negativa
\pgfmathsetmacro{\m}{factorial(\q+\rp-1)*((1-\p)^\rp)*(\p^\q)/factorial(\rp-1)/factorial(\q)};
%{factorial(\n)/factorial(\q)/factorial(\n-\q)*(\p^\q)*((1-\p)^(\n-\q))};% maximo de la binomial
% masa
\begin{scope}
%
\pgfmathsetmacro{\sy}{2.5/\m};% y-scaling 
\draw[>=stealth,->] (-.25,0)--({\sx*(\n+.75)+.25},0) node[right]{\small $x$};
\draw[>=stealth,->] (0,-.1)--(0,{\sy*\m+.25}) node[above]{\small $p_X$};
%
\pgfmathsetmacro{\b}{(1-\p)^\rp};% coeficiente binomial por la probabilidad
%
\foreach \k in {0,...,\n} {
\draw ({\k*\sx},0)--({\k*\sx},-.1) node[below,scale=.7]{$\k$};
\draw[dotted] ({\k*\sx},0)--({\k*\sx},{\sy*\b}) node[scale=.7]{$\bullet$};
%
\pgfmathsetmacro{\bl}{\b*\p*(\k+\rp)/(\k+1)};\global\let\b\bl;% proba actualizada
}
\draw ({(\n+.25)*\sx},{\sy*\b*(\n+1)/\p/(\n+\rp)/2}) node[right,scale=.7]{\ldots};
\draw (0,{((1-\p)^\rp)*\sy})--(-.1,{((1-\p)^\rp)*\sy}) node[left,scale=.7]{$(1-p)^r$};
\draw (0,{\rp*\p*((1-\p)^\rp)*\sy})--(-.1,{\rp*\p*((1-\p)^\rp)*\sy}) node[left,scale=.7]{$r \, p \, (1-p)^r$};
%\draw (0,{(\rp*\p*((1-\p)^\rp)+\m)/2*\sy}) node[scale=.7]{$r \, p \, (1-p)^r$};
%
\end{scope}
%
%
% reparticion
\begin{scope}[xshift=8.5cm]
%
\pgfmathsetmacro{\sy}{2.5};% y-scaling 
%
\draw[>=stealth,->] (-.6,0)--({\sx*(\n+.75)+.5},0) node[right]{\small $x$};
\draw[>=stealth,->] (0,-.1)--(0,{\sy+.25}) node[above]{\small $F_X$};
%
\pgfmathsetmacro{\b}{(1-\p)^\rp};% coeficiente binomial por la probabilidad
\pgfmathsetmacro{\c}{(1-\p)^\rp};% cumulativa binomial por la probabilidad
%
% cumulativa x < 0
\draw (0,0)--(0,-.1) node[below,scale=.7]{$0$};
\draw[thick] (-.5,0)--(0,0);
\draw (\r,\r) arc (90:270:\r);
%
% cumulativa x de 0 a n-1
\foreach \k in {1,...,\n} {
\draw ({\k*\sx},0)--({\k*\sx},-.1) node[below,scale=.7]{$\k$};
\draw[thick]({(\k-1)*\sx},{\sy*\c}) node[scale=.7]{$\bullet$}--({\k*\sx},{\sy*\c});
\draw ({\k*\sx+\r},{\sy*\c+\r}) arc (90:270:\r);
\draw[dotted] ({(\k-1)*\sx},{(\c-\b)*\sy})--({(\k-1)*\sx},{\c*\sy});
%
\pgfmathsetmacro{\bl}{\b*\p*(\k+\rp-1)/\k};\global\let\b\bl;% proba actualizada
\pgfmathsetmacro{\cl}{\c+\b};\global\let\c\cl;% cumulativa actualizada
}
%
\draw ({\n*\sx},{\sy*(\c+1)/2}) node[left,scale=.7]{\ldots};
\draw (0,{((1-\p)^\rp)*\sy})--(-.1,{((1-\p)^\rp)*\sy}) node[left,scale=.7]{$(1-p)^r$};
\draw (0,{(1+\rp*\p)*((1-\p)^\rp)*\sy})--(-.1,{(1+\rp*\p)*((1-\p)^\rp)*\sy}) node[left,scale=.7]{$(1+r \, p) (1-p)^r$};
\draw (-.75,{((1+\rp*\p)*((1-\p)^\rp)+1)/2*\sy}) node[right,scale=.7]{$\vdots$};
\draw (0,\sy)--(-.1,\sy) node[left,scale=.7]{$1$};
\end{scope}
%
\end{tikzpicture} \end{center}
%
\leyenda{Ilustraci\'on de  una distribuci\'on de  probabilidad binomial negativa
  (a), y  la funci\'on  de repartici\'on  asociada (b), con  $r =  3, \quad  p =
  \frac35$.}
\label{Fig:MP:BinomialNegativa}
\end{figure}
\SZ{Otros ilustraciones para otros $r, p$?}

Esta ley aparece cuando se repite una experencia binaria \ $X_i \in \{ 0 \, , \,
1 \},  i = 1, \ldots$  \ con \ $P(X_i=1)  = p$ \ de  manera independiente ($X_i$
independientes) hasta  que \  $r$ \  variables valen 0,  con \  $r$ \  fijo. Los
n\'umeros de  excitos \ $X_i =  1$ \ sigue una  ley \ $N\B(r,p)$  (el calculo es
directo).  Dicho de otra  manera, $X = \sum_{i=1}^N X_i$ \ con  \ $N$ \ variable
aleatoria tal que $X_N = 0$ \ y \ $r = \sum_{i=1}^N (1-X_i)$: condicionalmente a
\ $N$, la variable es binomial de parametro $p$.  Se puede ver que \ $P(N = n) =
\bino{n}{r-1} (1-p)^r  p^{n-r}$ \ y la  ley de la binomial  negativa se recupera
tambi\'en,          por         ejemplo,         a          trav\'es         del
teorema~\ref{Teo:MP:SumaAleatoriaGeneradoraProbabilidad}. En el caso \ $r = 1$ \
aparece que \ $N \sim \G(1-p)$ \ y\ldots tambi\'en \ $X \sim \G(1-p)$.
%
% Blaise PAscal - Polya caso r real

Esta distribuci\'on se  generaliza para \ $r \in \Rset_+^*$ \  pero se pierde la
interpretaci\'on que v\'imos en el p\'arafo anterior.

Nota: cuando \ $p = 0$ \ la variable es cierta \ $X = r$.


% --------------------------------- Multinomial
%\subsubseccion{Ley multinomial}
\label{Sssec:MP:Multinomial}

Esta ley es una generalizaci\'on de la ley binomial y aparece por ejemplo cuando
se  repite  una  experiencia  a  \  $k$  \ estados  \  $n$  \  veces  de  manera
independiente y nos  interesamos a la probabilidad que  el primer evento aparece
$n_1$ veces,  el secundo  $n_2$ veces, \ldots  (ej. para  $k = 6$,  contamos los
n\'umeros de $1$, de $2$, \ldots cuando tiramos $n$ veces este dado). Apareci\'o
tambi\'en   esta   ley   por   la    primera   vez   en   el   trabajo   de   J.
Bernoulli~\cite{Ber1713, Hal90,  DavEdw01} (ver tambi\'en el  ensayo de Montmort
de 1708 con otras notaciones~\cite{Mon13}).

Se  denota  \  $X  \ \sim  \  \M(n,p)$  \  con  \  $n  \in  \Nset^*$ \  y  \  $p
= \begin{bmatrix}  p_1 & \cdots  & p_k \end{bmatrix}^t  \in \Simp{k-1}$ \  the \
$(k-1)$-simplex estandar  (ver figura~\ref{Fig:MP:Dirichlet}-(a) m\'as adelante,
y  notaciones).   Entonces,  a pesar  de  que  se  escribe  \  $X$ \  de  manera
$k$-dimensional,  el vector  partenece a  un espacio  claramente \  $d =  k-1$ \
dimensional y en el caso \ $k = 2$ \ se recupera la ley binomial.  El dominio de
definici\'on es claramente $\Part{n}{k}$ (ver notaciones). Las caracter\'isticas
de \ $X \ \sim \ \M(n,p)$ \ son las siguientes:

\begin{caracteristicas}
%
Dominio de definici\'on
%~\footnote{De hecho, se puede considerar que el vector
%aleatorio es \ $(k-1)$-dimensional \ $\widetilde{X} = \begin{bmatrix}
%\widetilde{X}_1 & \cdots & \widetilde{X}_{k-1} \end{bmatrix}^t$ \ definido sobre
%el dominio \ $\widetilde{\X} = \left\{ x \in \{ 0 \; \ldots \; n\}^{k-1}, \:
%\sum_{i=1}^{k-1} x_i \le n \right\}$.\label{Foot:MP:MultinomialDominio}}
 & $\X = \Part{n}{k}$
%\left\{ x \in \{ 0 \; \ldots \; n\}^k \tq \sum_{i=1}^k x_i = n \right\}$
\\[2mm]
\hline
%
Par\'ametros
%~\footnote{El par\'ametro de \ $\widetilde{X}$ \ es \ $\widetilde{p} =
%\protect\begin{bmatrix} p_1 & \cdots & p_{k-1} \end{bmatrix}^t\protect \in
%\left\{ q \in [0 \; 1]^{k-1} \tq \sum_{i=1}^{k-1} q_i \le 1
%\right\}$.\label{Foot:MP:MultinomialParametro}}
 & $n \in \Nset^*$, \quad $p \in
\Simp{k-1}$\\[2mm]
\hline
%
Distribuci\'on de probabilidad
%~\footnote{La masa de probabilidad de \
%$\widetilde{X}$ \ es \ $p_{\widetilde{X}}(x) = \frac{n!}{\prod_{i=1}^{k-1} x_i!
%(n-\sum_{i=1}^{k-1} x_i)!}  \prod_{i=1}^{k-1} p_i^{x_i} \, \left( 1 -
%\sum_{i=1}^{k-1} p_i \right)^{n-\sum_{i=1}^{k-1}
%x_i}$.\label{Foot:MP:MultinomialMasa}}
 & $\displaystyle p_X(x) =
\frac{n!}{\prod_{i=1}^k x_i!}  \prod_{i=1}^k p_i^{x_i}$\\[2mm]
\hline
%
Promedio & $\displaystyle m_X = n p$\\[2mm]
\hline
%
Covarianza
%~\footnote{$\Sigma_X \in \Pos_k(\Rset)$, pero de \ $\un^t \Sigma_X \un =
%0$ \ viene \ $\Sigma_X \not\in \Pos_k^+(\Rset)$. Eso es la consecuencia directa del
%hecho de que \ $X$ \ $d$-dimensional, vive sobre \ $\Simp{k-1}$,
%$(d-1)$-dimensional.\label{Foot:MP::MultinomialCovarianza}}
 & $\displaystyle
\Sigma_X = n \left( \diag(p) - p p^t \right)$\\[2mm]
\hline
%
Generadora de probabilidad
%~\footnote{Notar: $G_{\widetilde{X}}\left(
%\widetilde{z} \right) = G_X\left( \begin{bmatrix} \widetilde{z} &
%1 \end{bmatrix}^t \right)$ \ y al rev\'es \ $G_X(z) = z_k^n \,
%G_{\widetilde{X}}\left( \begin{bmatrix} \frac{z_1}{z_k} & \cdots &
%\frac{z_{k-1}}{z_k} \end{bmatrix}^t
%\right)$.\label{Foot:MP:MultinomialGeneProba}}
 & $\displaystyle G_X(z) = \left(
p^t z \right)^n$ \ para \ $z \in \Cset^k$\\[2mm]
\hline
%
Generadora de momentos
%~\footnote{Notar: $M_{\widetilde{X}}\left( \widetilde{u}
%\right) = M_X\left( \begin{bmatrix} \widetilde{u} & 0 \end{bmatrix}^t \right)$ \
%y \ $M_X(u) = e^{n \, u_k} M_{\widetilde{X}}\left( \begin{bmatrix} u_1 - u_k &
%\cdots & u_{k-1} - u_k \end{bmatrix}^t
%\right)$.\label{Foot:MP:MultinomialGeneMomentos}}
 & \protect$\displaystyle
M_X(u) = \left( p^t e^u \right)^n, \: e^u = \begin{bmatrix} e^{u_1} & \cdots &
e^{u_k} \end{bmatrix}^t$\protect \ para \ $u \in \Cset^k$\\[2mm]
\hline
%
Funci\'on caracter\'istica
%~\footnote{Notar: $\Phi_{\widetilde{X}}\left(
%\widetilde{\omega} \right) = \Phi_X\left( \begin{bmatrix} \widetilde{\omega} &
%0 \end{bmatrix}^t \right)$ \ o \ $\Phi_X(\omega) = e^{\imath \, n \, \omega_k}
%\Phi_{\widetilde{X}}\left( \begin{bmatrix} \omega_1 - \omega_k & \cdots &
%%\omega_{k-1} - \omega_k \end{bmatrix}^t
%\right)$.\label{Foot:MP:MultinomialCaracteristica}} 
& $\displaystyle
\Phi_X(\omega) = \left( p^t e^{\imath \omega} \right)^n$
\end{caracteristicas}

% Momentos & $ \Esp\left[ X^k \right] = ??\\[2mm]
% Momento factorial & $\Esp\left[ (X)_k \right] = 
% \frac{(r+k-1)!}{(r-1)!} \left( \frac{p}{1-p} \right)^k$\\[2mm]
% Modo $\left\lfloor (n+1) p \right\rfloor$
% Mediana $\left\lfloor n p \right\rfloor$ o $\left\lceil n p \right\rceil
% CDF	$I_{1-p}(n-k,k+1)$ regularized incomplete beta function

De hecho, se puede considerar que el vector aleatorio es \ $(k-1)$-dimensional \
$\widetilde{X}     =    \begin{bmatrix}     \widetilde{X}_1    &     \cdots    &
  \widetilde{X}_{k-1}   \end{bmatrix}^t$   \  definido   sobre   el  dominio   \
$\widetilde{\X} = \left\{ x \in \{ 0 \; \ldots \; n\}^{k-1}, \: \sum_{i=1}^{k-1}
  x_i  \le n  \right\}$. El  par\'ametro de  \ $\widetilde{X}$  \ es  entonces \
$\widetilde{p}     =     \protect\begin{bmatrix}      p_1     &     \cdots     &
  p_{k-1}  \end{bmatrix}^t\protect  \in  \left\{   q  \in  [0  \;  1]^{k-1}  \tq
  \sum_{i=1}^{k-1}  q_i  \le  1  \right\}$.    La  masa  de  probabilidad  de  \
$\widetilde{X}$   \   se    escribe   obviamente   \   $p_{\widetilde{X}}(x)   =
\frac{n!}{\prod_{i=1}^{k-1} x_i!   (n-\sum_{i=1}^{k-1} x_i)!}  \prod_{i=1}^{k-1}
p_i^{x_i} \, \left( 1  - \sum_{i=1}^{k-1} p_i \right)^{n-\sum_{i=1}^{k-1} x_i}$.
Se  notar\'a al  final que  \ $G_{\widetilde{X}}\left(  \widetilde{z}  \right) =
G_X\left(  \begin{bmatrix} \widetilde{z}  & 1  \end{bmatrix}^t \right)$  \  y al
rev\'es   \   $G_X(z)  =   z_k^n   \,  G_{\widetilde{X}}\left(   \begin{bmatrix}
    \frac{z_1}{z_k}  & \cdots  &  \frac{z_{k-1}}{z_k} \end{bmatrix}^t  \right)$.
Similarmente,    \     $M_{\widetilde{X}}\left(    \widetilde{u}    \right)    =
M_X\left(  \begin{bmatrix} \widetilde{u}  &  0 \end{bmatrix}^t  \right)$  \ y  \
$M_X(u)  = e^{n  \, u_k}  M_{\widetilde{X}}\left(  \begin{bmatrix} u_1  - u_k  &
    \cdots  &  u_{k-1}  -  u_k  \end{bmatrix}^t \right)$  (y  similarmente  para
$\Phi_X$ y $\Phi_{\widetilde{X}}$).

Se puede ver  tambi\'en que $\Sigma_X \un  = 0$ \ as\'i que  \ $\Sigma_X \not\in
\Pos_k^+(\Rset)$.   Eso  es la  consecuencia  directa  del hecho  de  que  \ $X$  \
$k$-dimensional, vive sobre  \ $\Simp{k-1}$, $(k-1)$-dimensional. Aparentemente,
siendo  $\Sigma_X$  no  invertible,  no  se  puede  definir  ni  asimetr\'ia,  ni
curtosis. Sin embargo, habr\'ia que considerar \ $\widetilde{X}$, de promedio $n
\widetilde{p}$ y de covarianza el bloque $(k-1) \times (k-1)$ de $\Sigma_X$, que es
ahora invertible. $\gamma_{\widetilde{X}}$ \  y \ $\kappa_{\widetilde{X}}$ \ son
bien definidos. Las expresiones, demasiado pesadas, no son dadas ac\'a.

Deos ejemplos de masa de probabilidad de esta ley son representadas en la figura
Fig.~\ref{Fig:MP:Multinomial}.
%
\begin{figure}[h!]
\begin{center} \begin{tikzpicture}[scale=.8]
\shorthandoff{>}
%
%
\pgfmathsetmacro{\n}{5};% numeros para la multinomial
\pgfmathsetmacro{\dec}{.5};% shitf para dibujar las marginales
%
% Ejemplo [6 5 4]/15
\begin{scope}
%
\pgfmathsetmacro{\pu}{2/5};% p_1
\pgfmathsetmacro{\pd}{1/3};% p_2
\pgfmathsetmacro{\qu}{floor((\n+1)*\pu)};% modo de la binomial 1
\pgfmathsetmacro{\qd}{floor((\n+1)*\pd)};% modo de la binomial 2
\pgfmathsetmacro{\mau}{factorial(\n)/factorial(\qu)/factorial(\n-\qu)*(\pu^\qu)*((1-\pu)^(\n-\qu))};% maximo de la binomial 1
\pgfmathsetmacro{\mad}{factorial(\n)/factorial(\qd)/factorial(\n-\qd)*(\pd^\qd)*((1-\pd)^(\n-\qd))};% maximo de la binomial 2
\pgfmathsetmacro{\ma}{max(\mau,\mad)};% maximo de ambas binomiales
%
\begin{axis}[
    colormap = {whiteblack}{color(0cm)  = (white);color(1cm) = (black)},
    width=.55\textwidth,
    view={35}{70},
    enlargelimits=false,
    xmin={-\dec},
    xmax={\n+\dec},
    ymin={-\dec},
    ymax={\n+\dec},
    zmax={1.1*\ma},
    color=black,
    xtick={0,...,\n},
    ytick={0,...,\n},
    xlabel=$x_1$,
    ylabel=$x_2$,
    zlabel=$p_{\widetilde{X}}$,
]
%
\pgfmathsetmacro{\bu}{(1-\pu-\pd)^\n};% coeficiente binomial por la probabilidad p1
\pgfmathsetmacro{\bd}{\bu};% coeficiente binomial por la probabilidad p2
%
\pgfmathsetmacro{\bmu}{(1-\pu)^\n};% lo mismo para la marginale 1
\pgfmathsetmacro{\bmd}{(1-\pd)^\n};% lo mismo para la marginale 2
%
\foreach \mu in {0,...,\n} {
  \foreach \md in {0,...,\n} {
    \ifnum \numexpr\mu+\md < \numexpr\n+1
      \addplot3 [dotted,domain=0:\bd,samples=2, samples y=0,color=black] (\mu,\md,\x)  node[scale=.85]{$\bullet$};
      %
      \pgfmathsetmacro{\bld}{\bd*\pd*(\n-\md)/((\md+1)*(1-\pu-\pd))};
      \global\let\bd\bld;% proba en m2 (m1 fijo) actualizado
    \fi
  }
  %
  % Marginales
  \addplot3 [dotted,domain=0:\bmu,samples=2, samples y=0,color=black] (\mu,{\n+\dec},\x)  node[scale=.55]{$\bullet$};
  \addplot3 [dotted,domain=0:\bmd,samples=2, samples y=0,color=black] ({-\dec},\mu,\x)  node[scale=.55]{$\bullet$};
  %
  % lineas (m1,m2) abajo
  \addplot3 [domain={-\dec}:{\n+\dec},samples=2, samples y=0,color=black!10] (\mu,\x,0);
  \addplot3 [domain={-\dec}:{\n+\dec},samples=2, samples y=0,color=black!10] (\x,\mu,0);
  %
  \pgfmathsetmacro{\blu}{\bu*\pu*(\n-\mu)/((\mu+1)*(1-\pu-\pd))};
  \global\let\bu\blu;\global\let\bd\blu;% proba inicial en m1 actualizada
  %
  % lo mismo para cada marginal
  \pgfmathsetmacro{\blmu}{\bmu*\pu*(\n-\mu)/((\mu+1)*(1-\pu))};
  \global\let\bmu\blmu;% proba 1 actualizada
  \pgfmathsetmacro{\blmd}{\bmd*\pd*(\n-\mu)/((\mu+1)*(1-\pd))};
  \global\let\bmd\blmd;% proba 2 actualizada
}
%
\node at (axis cs:{3*\n/4},{\n+\dec},{\mau/2})[right]{$p_{X_1}$};
\node at (axis cs:{-\dec},{3*\n/4},{\mad/2})[above]{$p_{X_2}$};
\end{axis}
\node at ({3*\n/4},-1)[scale=.9]{(a)};
\end{scope}
%
%
%
%
% Ejemplo [1 1 1]/3
\begin{scope}[xshift = 11cm]
%
\pgfmathsetmacro{\pu}{1/3};% p_1
\pgfmathsetmacro{\pd}{1/2};% p_2
\pgfmathsetmacro{\qu}{floor((\n+1)*\pu)};% modo de la binomial 1
\pgfmathsetmacro{\qd}{floor((\n+1)*\pd)};% modo de la binomial 2
\pgfmathsetmacro{\mau}{factorial(\n)/factorial(\qu)/factorial(\n-\qu)*(\pu^\qu)*((1-\pu)^(\n-\qu))};% maximo de la binomial 1
\pgfmathsetmacro{\mad}{factorial(\n)/factorial(\qd)/factorial(\n-\qd)*(\pd^\qd)*((1-\pd)^(\n-\qd))};% maximo de la binomial 2
\pgfmathsetmacro{\ma}{max(\mau,\mad)};% maximo de ambas binomiales
%
\begin{axis}[
    colormap = {whiteblack}{color(0cm)  = (white);color(1cm) = (black)},
    width=.55\textwidth,
    view={35}{70},
    enlargelimits=false,
    xmin={-\dec},
    xmax={\n+\dec},
    ymin={-\dec},
    ymax={\n+\dec},
    zmax={1.1*\ma},
    color=black,
    xtick={0,...,\n},
    ytick={0,...,\n},
    xlabel=$x_1$,
    ylabel=$x_2$,
    zlabel=$p_{\widetilde{X}}$,
]
%
\pgfmathsetmacro{\bu}{(1-\pu-\pd)^\n};% coeficiente binomial por la probabilidad p1
\pgfmathsetmacro{\bd}{\bu};% coeficiente binomial por la probabilidad p2
%
\pgfmathsetmacro{\bmu}{(1-\pu)^\n};% lo mismo para la marginale 1
\pgfmathsetmacro{\bmd}{(1-\pd)^\n};% lo mismo para la marginale 2
%
\foreach \mu in {0,...,\n} {
  \foreach \md in {0,...,\n} {
    \ifnum \numexpr\mu+\md < \numexpr\n+1
      \addplot3 [dotted,domain=0:\bd,samples=2, samples y=0,color=black] (\mu,\md,\x)  node[scale=.85]{$\bullet$};
      %
      \pgfmathsetmacro{\bld}{\bd*\pd*(\n-\md)/((\md+1)*(1-\pu-\pd))};
      \global\let\bd\bld;% proba en m2 (m1 fijo) actualizado
    \fi
  }
  %
  % Marginales
  \addplot3 [dotted,domain=0:\bmu,samples=2, samples y=0,color=black] (\mu,{\n+\dec},\x)  node[scale=.55]{$\bullet$};
  \addplot3 [dotted,domain=0:\bmd,samples=2, samples y=0,color=black] ({-\dec},\mu,\x)  node[scale=.55]{$\bullet$};
  %
  % lineas (m1,m2) abajo
  \addplot3 [domain={-\dec}:{\n+\dec},samples=2, samples y=0,color=black!10] (\mu,\x,0);
  \addplot3 [domain={-\dec}:{\n+\dec},samples=2, samples y=0,color=black!10] (\x,\mu,0);
  %
  \pgfmathsetmacro{\blu}{\bu*\pu*(\n-\mu)/((\mu+1)*(1-\pu-\pd))};
  \global\let\bu\blu;\global\let\bd\blu;% proba inicial en m1 actualizada
  %
  % lo mismo para cada marginal
  \pgfmathsetmacro{\blmu}{\bmu*\pu*(\n-\mu)/((\mu+1)*(1-\pu))};
  \global\let\bmu\blmu;% proba 1 actualizada
  \pgfmathsetmacro{\blmd}{\bmd*\pd*(\n-\mu)/((\mu+1)*(1-\pd))};
  \global\let\bmd\blmd;% proba 2 actualizada
}
%
\node at (axis cs:{3*\n/4},{\n+\dec},{\mau/2})[right]{$p_{X_1}$};
\node at (axis cs:{-\dec},{3*\n/4},{\mad/2})[above]{$p_{X_2}$};
\end{axis}
\node at ({3*\n/4},-1)[scale=.9]{(b)};
\end{scope}
%
\end{tikzpicture} \end{center}
%
\leyenda{Ilustraci\'on de una distribuci\'on  de probabilidad multinomial para \
  $k   =  3$   \   del   vector  \   $(k-1)$-dimensional   \  $\widetilde{X}   =
  \protect\begin{bmatrix}   X_1  &  X_2   \protect\end{bmatrix}^t$  \   ($X_3  =
  1-X_1-X_2$)  \ con  las marginales  \ $p_{X_1},  \: p_{X_2}$.
  %      \      (ver      notas     de      pie~\ref{Foot:MP:MultinomialDominio}
  % y~\ref{Foot:MP:MultinomialMasa}).
  Es dibujada  solamente la  distribuci\'on sobre $\widetilde{\X}$,  siendo esta
  nula afuera de  $\widetilde{\X}$.  Los par\'ametros son \  $n = 5$ \ y  \ $p =
  \protect\begin{bmatrix}      \frac25      &      \frac13     &      \frac4{15}
    \protect\end{bmatrix}^t$ (a), $p = \protect\begin{bmatrix} \frac13 & \frac12
    & \frac16 \protect\end{bmatrix}^t$ (b).}
\label{Fig:MP:Multinomial}
\end{figure}


Notar: cuando $p = \un_i$, la variable es cierta $X = n \un_i$.

\SZ{Otros ilustraciones para otros $n, p$?}


Vectores  de  distribuci\'on  multinomial  tienen  una  propiedade  notable  con
respecto a una permutaci\'on de variable, parecidas a la de la binomial:
%
\begin{lema}[Efecto de una permutaci\'on]\label{Lem:MP:PermutacionMultinomial}
%
  Sea \ $X \, \sim \, \M(n,p), \: p \in \Simp{k-1}$ \ y \ $\Pi \in \perm_k(\Rset)$ \
  matriz \ de permutaci\'on (ver notaciones). Entonces
  %
  \[
  \Pi X \, \sim \, \M\left( n ,  \Pi p \right)
  \]
  %
\end{lema}
%
\begin{proof}
  El  resultado  es  inmediato  saliendo  de  la  funci\'on  caracter\'istica  y
  aplicando  el  teorema~\ref{Teo:MP:PropiedadesFuncionCaracteristica} (recordar
  que $\Pi^{-1} = \Pi^t$). M\'as directamente, notando la permutation \ $\sigma$
  \ tal que  \ $\Pi = \sum_{i=1}^k \un_i \un_{\sigma(i)}^t$, se  puede ver que \
  $\displaystyle  P(\Pi X =  x) =  P(X =  \Pi^{-1} x)  = \frac{n!}{\prod_{i=1}^k
    x_{\sigma^{-1}(i)}!}       \prod_{i=1}^k      p_i^{x_{\sigma^{-1}(i)}}     =
  \frac{n!}{\prod_{i=1}^k x_i!}  \prod_{i=1}^k p_{\sigma(i)}^{x_i}$ \ por cambio
  de indices.
\end{proof}
%
Adem\'as, la ley multinomial  exhibe una stabilidad reemplazando dos componentes
por su suma:
%
\begin{lema}[Stabilidad por agregaci\'on]\label{Lem:MP:StabAgregacionMultinomial}
%
  Sea  \ $X =  \begin{bmatrix} X_1  & \cdots  & X_k  \end{bmatrix}^t \,  \sim \,
  \M(n,p), \:  p \in \Simp{k-1}$ \ y  \ $G^{(i,j)}$ \ matriz  de agrupaci\'on de
  las $(i,j)$-\'esima componentes (ver notaciones). Entonces,
  %
  \[
  G^{(i,j)} X \, \sim \, \M\left( n , G^{(i,j)} p \right)  
  \]
  %
\end{lema}
%
Este resultado es intuitivo del hecho que vuelve a agrupar los estados \ $i$ \ e
\ $j$  \ en un  estado, que tiene  entonces la probabilidad \  $p_i + p_j$  \ de
aparecer.
%
\begin{proof}
  Suponemos $i <  j$ (el otro caso  se recupera por simetr\'ia). A  partir de la
  funci\'on                 caracter\'istica                 y                el
  teorema~\ref{Teo:MP:PropiedadesFuncionCaracteristica} se tiene,
  %
  \begin{eqnarray*}
  \forall \: \omega \in \Rset^{k-1}, \quad \Phi_{G^{(i,j)} X}(\omega) & = &
  \Phi_X\left( G^{(i,j) \, t} \omega \right)\\[2mm]
  %
  & = & \left( \sum_{l=1}^k p_l \, e^{\imath \, \left( G^{(i,j) \, t} \omega \right)_l } \right)^n
  \end{eqnarray*}
  %
  Ahora,  se nota  que \  $G^{(i,j) \,  t} \omega  = \begin{bmatrix}  \omega_1 &
    \cdots    &   \omega_{j-1}   &    \omega_i   &    \omega_{j+1}   &    \cdots   &
    \omega_{k-1} \end{bmatrix}^t$, entonces
  %
  \begin{eqnarray*}
  \forall \: \omega \in \Rset^{k-1}, \quad \Phi_{G^{(i,j)} X}(\omega) & = &
  \left( \sum_{l=1, l \ne j}^k p_l \, e^{\imath \, \omega_l} + p_j \, e^{\imath \,
  \omega_i } \right)^n\\[2mm]
  %
  & = & \left( \sum_{l=1, l \ne i, l \ne j}^k p_l \, e^{\imath \, \omega_l} +
  (p_i+p_j) \, e^{\imath \, \omega_i } \right)^n
  %
  \end{eqnarray*}
  %
  lo que  cierra la  prueba. Se puede  tener un  enfoque m\'as directo,  con los
  mismos         pasos        que         en        la         prueba        del
  lema~\ref{Lem:MP:StabAgregacionHipergeomMulti}    tratando     de    la    ley
  hipergeometrica multivaluada.
\end{proof}

De este lema, aplicado de manera recursiva, se obtiene los corolarios siguientes:
%
\begin{corolario}\label{Cor:MP:MarginalMultinomial}
%
  Sea  \ $X  \,  \sim \,  \M(n,p)$, entonces  \  $\displaystyle X_i  \, \sim  \,
  \B(n,p_i)$.
\end{corolario}


Al  final,  por  una  an\'alisis  combinatorial,  se  muestra  sencillamente  un
resultado similar al de la binomial como suma de Bernoulli independientes:
%
\begin{lema}\label{Lem:MultinomialSumaMultiBernoulli}
%
  Sean \ $U_i, \quad i  = 1, \ldots , n$ \ discretas sobre $\U  = \{ 1 \; \ldots
  \;  k   \}$  de  masa  de   probabilidad  $p_{U_i}  =   p  \in  \Delta_{k-1}$,
  independientes, y $X_i  = \un_{U_i}$ vectores aleatorios $k$-dimensionales
  (son, por construcci\'on, independientes). Entonces
  %
  \[
  \sum_{i=1}^n X_i \, \sim \, \M(n,p)
  \]
\end{lema}

\

Nota: esta ley  se generaliza de la misma manera que  para la binomial negativa,
dando una  ley multinomial negativa  o, de manera equivalente,  generalizando la
binomial  negativa  a  m\'as  de  dos  clases  se  obtiene  la  ley  multinomial
negativa. \SZ{Anadirlo en una seccion?}



% --------------------------------- Hipergeometrica
%\subsubseccion{Ley hipergeometrica}
\label{Sssec:MP:Hipergeometrica}

Esta ley aparece por ejemplo cuando  se hace una experiencia con una poblaci\'on
de tama\~no \  $n$ \ (ej.  $n$ bolas  en una urna), que pueden  partenecer a dos
clases, con  \ $k$ \ n\'umero  de elementos de  la primera clase (a  veces dicho
estados de \'exito; ej. $k$ \ bolas  negras), $n-k$ \ n\'umero de elementos de la
secunda clase, y se  hace \ $m$ \ tiros sin reemplazamiento.   $X$ es el n\'umero
de  tiros parteneciendo  en la  primera clase  (n\'umero de  \'exitos).  Esta ley
apareci\'o en trabajos de de Moivre en 1710~\cite{Moi10, Hal90, DavEdw01}.

Se denota \ $X \, \sim \, \H(n,k,m)$ \  con \ $n \in \Nset^*$, \quad $k \in \{ 0
\;  \ldots  \;  n  \}$,  \quad  $m  \in  \{  0 \;  \ldots  \;  m  \}$  \  y  sus
caracter\'isticas son las siguientes:

\begin{caracteristicas}
%
Dominio de definici\'on & $\X = \left\{ \max(0,k+m-n) \; \ldots \; \min(k,m)
\right\}$\\[2mm]
\hline
%
Par\'ametros & $n \in \Nset^*$ \: (poblaci\'on)\newline $k \in \{ 0 \; \ldots \;
n\}$ \ (n\'umero de estados exitosos)\newline $m \in \{ 0 \; \ldots \; n\}$ \:
(n\'umero de tiros)\\[2mm]
\hline
%
Distribuci\'on de probabilidad & \protect$p_X(x) =
\frac{\smallbino{k}{x} \smallbino{n-k}{m-x}}{\smallbino{n}{m}}$\protect\\[2mm]
\hline
%
Promedio & $\displaystyle m_X = \frac{m}{n} \, k$\\[2mm]
\hline
%
Varianza~\footnote{En el caso degenerado \ $n = 1$, o \ $m = 0$, o \ $m = 1 =
n$; en ambos casos, la variable es cierta (ver fin de la
subsecci\'on).\label{Foot:MP:HipergeometricaVarianza}} & $\displaystyle
\sigma_X^2 = \left\{ \protect\begin{array}{ccc} \frac{m \, (n-m)}{n^2 (n-1)} \,
k \, (n-k) & \mbox{si} & n > 1\\[2mm] 0 & \mbox{si} & n =
1\end{array}\protect\right.$\\[2mm]
\hline
%%
%Asimetr\'ia~\footnote{Cuando \ $m \in \{ 0 \; n \}$, tenemos \ $k \in \{ 0 \; n
%\}$. Entonces, la variable es cierta (ver fin de la subsecci\'on) as\'i que no
%hay asimetr\'ia. Si \ $n = 2$, \ y \ $k=m=1 \not\in \{0 \; n \}$, tenemos $P(X =
%0) = P(X = 1) = \frac12$ sim\'etrico: de nuevo la asimetr\'ia vale cero.} &
%$\displaystyle \gamma_X = \left\{ \!\! \begin{array}{cl} \frac{(n - 2 k) (n - 2
%m)}{n-2} % \sqrt{\frac{n-1}{m k (n-k) (n-m)}} & \mbox{si} \: n \ne 2, \quad k,l
%\not\in \{ 0 \; n \} \\[2mm] 0 & \mbox{si no} \end{array} \right.$\\[2mm]
%\hline
%%
%Curtosis por exceso & $\displaystyle \widebar{\kappa}_X = $\\[2mm]
% 0 si no
%\hline
%
Generadora de probabilidad & $G_X(z) = \frac{\smallbino{n-k}{m}}{\smallbino{n}{m}} \:
\: \hypgeom{2}{1}(-m , -k ; n-m-k+1 ; z)$ \ sobre \ $\Cset$\\[2mm]
\hline
%
Generadora de momentos & $M_X(u) = \frac{\smallbino{n-k}{m}}{\smallbino{n}{k}}  \:
\: \hypgeom{2}{1}\left( -m , -k ; n-m-k+1 ; e^u \right)$ \ sobre \
$\Cset$\\[2mm]
\hline
%%
Funci\'on caracter\'istica  & $\Phi_X(\omega) =  \frac{\smallbino{n-k}{m}}{\smallbino{n}{m}}  \:
\: \hypgeom{2}{1}\left( -m , -k ; n-m-k+1 ; e^{\imath \, \omega} \right)$
\end{caracteristicas}

Su masa  de probabilidad  y funci\'on de  repartici\'on son representadas  en la
figura Fig.~\ref{Fig:MP:Hipergeometrica}.
%
\begin{figure}[h!]
\begin{center} \begin{tikzpicture}[fixed point arithmetic]%[scale=.9]
\shorthandoff{>}
%
\pgfmathsetmacro{\sx}{.375};% x-scaling
\pgfmathsetmacro{\r}{.05};% radius arc non continuity F_X
%\pgfmathsetmacro{\p}{1/3};% probabilidad p
\pgfmathsetmacro{\n}{100};% numero n de la poblacion
\pgfmathsetmacro{\k}{12};% numero k de estados exitosos
\pgfmathsetmacro{\m}{40};% numero m de tiros
%
% Nota : con el fixed point, no anda min & max
% pero max(a,b) = (a+b+abs(a-b))/2  & min(a,b) = (a+b-abs(a-b))/2;
\pgfmathsetmacro{\f}{(\k+\m-abs(\k-\m))/2}; % ultimo indice de proba non nula
\pgfmathsetmacro{\F}{(\k+\m+abs(\k-\m))/2}; %
\pgfmathsetmacro{\d}{(abs(\m-\n+\k)+\m-\n+\k)/2}; % primer indice de proba non nula
%
% ultima proba non nula F (F-1) ... (F-f+1) / n (n-1) ... (n-f+1)
% finhiper(\F,\n,\f)
\tikzmath{function finhiper(\a,\b,\c) {
    if \c == 1 then {return (\a/\b);}
    else {return (\a/\b)*finhiper(\a-1,\b-1,\c-1);};
};};
%

\pgfmathsetmacro{\dn}{\d-1}; % proba nula hasta d-1
\pgfmathsetmacro{\fn}{\f+1}; % proba nula de nuevo a partid de f-1
\pgfmathsetmacro{\ui}{\f+3}; % ultimo indice dibujado
% f, f-1... hasta d => y de 0 hasta f-d & x = f-y
\pgfmathsetmacro{\finy}{\f-\d}
%
% masa
\begin{scope}
%
\pgfmathsetmacro{\sy}{10};% y-scaling
%
% proba nulas del principio 0 -> d-1
\foreach \y in {-2,...,\dn} {
\pgfmathsetmacro{\xl}{int(\y)};\global\let\x\xl;
\draw ({\sx*\x},0)--({\sx*\x},-.1) node[below,scale=.7]{$\x$};
\draw ({\sx*\x},0) node[scale=.6]{$\bullet$};
}
%
% proba nulas del fin f+1 -> ui
\foreach \y in {\fn,...,\ui} {
\pgfmathsetmacro{\xl}{int(\y)};\global\let\x\xl;
\draw ({\sx*\x},0)--({\sx*\x},-.1) node[below,scale=.7]{$\x$};
\draw ({\sx*\x},0) node[scale=.6]{$\bullet$};
}
%
\pgfmathsetmacro{\pr}{finhiper(\F,\n,\f)};% valor del ultima proba no nula
\pgfmathsetmacro{\maxp}{\pr};% proba maximal (inicializacion)
%
\foreach \y in {0,...,\finy} {
\pgfmathsetmacro{\xl}{int(\f-\y)};\global\let\x\xl;
\draw ({\sx*\x},0)--({\sx*\x},-.1) node[below,scale=.7]{$\x$};
\draw[dotted] ({\sx*\x},0)--({\sx*\x},{\sy*\pr}) node[scale=.6]{$\bullet$};
%
\pgfmathsetmacro{\prl}{\pr*\x*(\n-\k-\m+\x)/((\m-\x+1)*(\k-\x+1))};\global\let\pr\prl;% proba actualizado
\pgfmathsetmacro{\maxpl}{(abs(\pr-\maxp)+\pr+\maxp)/2};\global\let\maxp\maxpl;% proba max actualizado
}
%
%\pgfmathsetmacro{\maxpl}{ceil(100*\maxp)/100)};
\draw[>=stealth,->] ({-2*\sx-.25},0)--({\sx*\ui+.35},0) node[right]{\small $x$};
\draw[>=stealth,->] (0,-.15)--(0,{\sy*\maxp+.25}) node[above]{\small $p_X$};
\draw (0,{\sy/4})--(-.1,{\sy/4}) node[left]{\small $\frac14$};
%
\node at ({(\sx*(2+\f)+.25)/2},-1) [scale=.9]{(a)};
\end{scope}
%
%
% reparticion
\begin{scope}[xshift=8.25cm]
%
\pgfmathsetmacro{\sy}{2.5};% y-scaling 
%
\draw[>=stealth,->] ({-2*\sx-.25},0)--({\sx*\ui+.5},0) node[right]{\small $x$};
\draw[>=stealth,->] (0,-.15)--(0,{\sy+.25}) node[above]{\small $F_X$};
%
% proba nulas del principio 0 -> d-1
\foreach \y in {-2,...,\dn} {
\pgfmathsetmacro{\xl}{int(\y)};\global\let\x\xl;
\draw ({\sx*\x},0)--({\sx*\x},-.1) node[below,scale=.7]{$\x$};
}
\draw ({-2*\sx},0)--({\sx*\d},0);
\draw ({\sx*\d+\r},\r) arc (90:270:\r);
%
% proba nulas del fin f+1 -> ui
\foreach \y in {\fn,...,\ui} {
\pgfmathsetmacro{\xl}{int(\y)};\global\let\x\xl;
\draw ({\sx*\x},0)--({\sx*\x},-.1) node[below,scale=.7]{$\x$};
}
\draw ({\sx*\ui},\sy)--({\sx*(\f+1)},\sy);
%
\pgfmathsetmacro{\pr}{finhiper(\F,\n,\f)};% valor del ultima proba no nula
\pgfmathsetmacro{\cum}{1};% valor final de la cumulativa
% f, f-1... hasta d => y de 0 hasta f-d & x = f-y
\foreach \y in {0,...,\finy} {
\pgfmathsetmacro{\xl}{int(\f-\y)};\global\let\x\xl;
\draw ({\sx*\x},0)--({\sx*\x},-.1) node[below,scale=.7]{$\x$};
\draw ({\sx*(\x+1)},{\sy*\cum})--({\sx*\x},{\sy*\cum}) node[scale=.6]{$\bullet$};
\draw ({\sx*\x+\r},{\sy*(\cum-\pr)+\r}) arc (90:270:\r);
\draw[dotted] ({\sx*\x},{\sy*\cum})--({\sx*\x},{\sy*(\cum-\pr)});
%
\pgfmathsetmacro{\cuml}{\cum-\pr}\global\let\cum\cuml;% cumulativa actualizada
\pgfmathsetmacro{\prl}{\pr*\x*(\n-\k-\m+\x)/((\m-\x+1)*(\k-\x+1))};\global\let\pr\prl;% proba actualizado
}
\draw (0,\sy)--(-.1,\sy) node[left,scale=.7]{$1$};
%
\node at ({(\sx*(\ui+2)+.5)/2},-1) [scale=.9]{(b)};
\end{scope}
%
\end{tikzpicture} \end{center}
%
\leyenda{Ilustraci\'on  de una  distribuci\'on  de probabilidad  Hipergeometrica
  (a), y la funci\'on de repartici\'on asociada (b), con \ $n = 100$, \quad $k =
  12$, \quad $m = 40$.}
\label{Fig:MP:Hipergeometrica}
\end{figure}

\SZ{Otros ilustraciones para otros $n, k, m$?

  Poner la  asimetr\'ia (ya  lo tengo)?  El Curtosis (lo  tengo que  simplificar)? muy
  pesadas...      Momento      factorial      $f_q      =      \frac{\PocD{m}{q}
    \PocD{k}{q}}{\PocD{n}{q}}$ permitiendo calcular todo.}

Notar: la variable resuelta cierta en los casos siguientes
%
\begin{itemize}
\item  $m = 0  \: \Rightarrow  \: X  = 0$:  no se  sortean elementos,  as\'i que
  siempre se sortea $0$ elementos de la primera clase;
%
\item $m =  n \: \Rightarrow \: X =  k$: si se sortean todos los  elementos de la
  poblaci\'on, se sortean todos los \ $k$ \ de la primera clase;
%
\item $k = 0 \: \Rightarrow \: X = 0$: si la primera clase no tiene elementos, no
  se puede tirar elementos de esta clase;
%
\item $k = n  \: \Rightarrow \: X = m$: al rev\'es si  la secunda clase no tiene
  elementos, todos los sorteados partenecen a la primera clase.
\end{itemize}

La ley tiene propiedades de reflexividad del mismo tipo que para la ley binomial:
%
\begin{lema}[Reflexividad]
\label{Lem:MP:ReflexividadHipergeometrica}
%
  Sea \ $X \, \sim \, \H(n,k,m)$. Entonces
  %
  \[
  m-X \, \sim \, \H(n,n-k,m) \qquad \mbox{y} \qquad k-X \sim \H(n,k,n-m)
  \]
  %
\end{lema}
%
\begin{proof}
  El  primer   resultado  es  inmediato  de  $P(m-X   =  x)  =  P(X   =  m-x)  =
  \frac{\smallbino{k}{m-x} \smallbino{n-k}{x}}{\smallbino{n}{m}}$. El secundo de
  $P(k-X    =     x)    =     P(X    =    k-x)     =    \frac{\smallbino{k}{k-x}
    \smallbino{n-k}{m-k+x}}{\smallbino{n}{m}}      =      \frac{\smallbino{k}{x}
    \smallbino{n-k}{n-m-x}}{\smallbino{n}{n-m}}$  notando   que  $\bino{a}{b}  =
  \bino{a}{a-b}$.
\end{proof}
%
Se puede ver  que si en una urna con  bolas negras y blancas, con  \ $k$ \ bolas
negras, y \ $X$  \ es el n\'umero de bolas negras  sorteadas, $m-X$ \ representa
las bolas blancas sorteadas. Es decir que  en \ $m-X$ \ se intercambia los roles
de las bolas  negras y blancas.  De la misma manera,  $k-X$ representa las bolas
negras que quedan en la urna, entre las  \ $n-m$ \ que quedan, es decir que en \
$k-X$ \ se intercambia  los roles de las bolas sorteadas y  las que quedan en la
urna.

% Cuando  $n  = 1$,  se  recupera  la lei  de  Bernoulli  $\B(p) \equiv  \B(1,p)$.
% Ad\'emas, se muestra  sencillamente usando la generadora de  probabilidad que
% %
% De este resultado,  se puede notar que, por  ejemplo, le distribuci\'on binomial
% aparece en el conteo de eventos independientes de misma probabilidad entre $n$.

% Tambi\'en,  la ley binomial  tiene una  propiedad de  reflexividad, consecuencia
% directa de la de Bernoulli:
% %
% \begin{lema}[Reflexividad]
% \label{Lem:MP:ReflexividadBinomial}
% %
%   Sea \ $X \, \sim \, \B(n,p)$. Entonces
%   %
%   \[
%   n-X \, \sim \, \B(n,1-p)
%   \]
%   %
% \end{lema}

% Nota que cuando $p = 0$ (resp. $p = 1$) la variable es cierta $X = 0$ (resp.  $X
% = n$).



% --------------------------------- Hipergeometrica Negativa
%\subsubseccion{Ley hipergeometrica Negativa}
\label{Sssec:MP:HipergeometricaNegativa}

Esta ley aparece  por ejemplo cuando se hace una experiencia  del mismo tipo que
para la hipergeometrica, con una poblaci\'on de tama\~no \ $n$ \ (ej.  $n$ bolas
en  una urna),  que pueden  partenecer a  dos clases,  con \  $k$ \  num\'ero de
elementos de la primera clase estados  de excito; ej. $k$ \ bolas negras), $n-k$
\ num\'ero  de elementos de la  secunda clase.  Pero en  lugar de hacer  \ $m$ \
tiros  fijos,  se  hace tiros  hasta  que  $r$  elementos  de la  seconda  clase
(fracascos)  sean tiradas.   $X$ es  el n\'umero  de tiros  parteneciendo  en la
primera clase  (n\'umero de excitos). Es decir  que cuando $X =  x$, tenemos $k$
elementos de  la primera clase en  los ``primeros'' $x+r-1$  tiros, el \'ultimos
parteneciendo a la seconda clase.  Parece que se encuentran las primeras huellas
de esta ley en trabajos del marquesano de Condorcet en 1785~\cite{Con85}.

Se denota \ $X \,  \sim \, \H_-(n,k,r)$ \ con \ $n \in \Nset^*$,  \quad $k \in \{ 0 \;
\ldots \; n \}$, \quad $m \in \{  0 \; \ldots \; n-k \}$ \ y sus caracter\'isticas
son las siguientes:

\begin{caracteristicas}
%
Dominio de definici\'on & $\X = \left\{ 0 \; \ldots \; k \right\}$\\[2mm]
\hline
%
Par\'ametros & $n \in \Nset^*$ \: (poblaci\'on)\newline $k \in \{ 0 \; \ldots \;
n\}$ \ (n\'umero de estados exitosos)\newline $r \in \{ 0 \; \ldots \; n-k\}$ \:
(n\'umero de fracascos para parar)\\[2mm]
\hline
%
Distribuci\'on de probabilidad~\footnote{Para los $x+r-1$ primeros tiros, de la
primera clase hay $\smallbino{k}{x}$ combinaciones posibles, y
$\smallbino{n-k}{r-1}$ de la seconda clase, sobre los $\smallbino{n}{x+r-1}$
combinaciones posibles en total. Para el \'ultimo tiro, quedan $n-k-(r-1)$
posibilidades de la seconda clase sobre las $n-x-(r-1)$ elementos que quedan.} &
\protect$p_X(x) = \left\{ \begin{array}{ccc} \frac{\smallbino{x+r-1}{x}
\smallbino{n-r-x}{k-x}}{\smallbino{n}{k}} & \mbox{si} & r > 0\\ \un_{\{0\}}(x) & \mbox{si} & r = 0 \end{array} \right.$\protect\\[2mm]
\hline
%
Promedio & $\displaystyle m_X = \frac{r \, k}{n - k + 1}$\\[2mm]
\hline
%
Varianza & $\displaystyle \sigma_X^2 = \frac{r \, k (n+1) (n-k-r+1)}{(n-k+1)^2 (n-k+2)}$\\[2mm]
\hline
%%
%\modif{Asimetr\'ia} & \SZ{$\gamma_X =  $}\\[2mm]
%\hline
%%
%Curtosis por exceso & $\displaystyle \SZ{\widebar{\kappa}_X = ...}$\\[2mm]
%\hline
%
Generadora de probabilidad & $G_X(z) = \frac{\smallbino{n-r}{k}}{\smallbino{n}{k}} \:
\: \hypgeom{2}{1}(r , -k ; r-n ; z)$ \ sobre \ $\Cset$\\[2mm]
\hline
%
Generadora de momentos & $M_X(u) = \frac{\smallbino{n-r}{k}}{\smallbino{n}{k}} \:
\: \hypgeom{2}{1}\left(r , -k ; r-n ; e^u \right)$ \ sobre \ $\Cset$\\[2mm]
\hline
%
Funci\'on caracter\'istica  & $\Phi_X(\omega) =  \frac{\smallbino{n-r}{k}}{\smallbino{n}{k}} \:
\: \hypgeom{2}{1}\left(r , -k ; r-n ; e^{\imath \, \omega} \right)$
\end{caracteristicas}

\SZ{Poner asimetr\'ia y curtosis? Expresiones  muy pesadas... Momento factorial $f_q =
  \frac{\PocC{r}{q} \PocD{k}{q}}{\PocC{n-k+1}{q}}$ permitiendo calcular todo.}

Su masa  de probabilidad  y funci\'on de  repartici\'on son representadas  en la
figura Fig.~\ref{Fig:MP:HipergeometricaNegativa}.
%
\begin{figure}[h!]
\begin{center} \begin{tikzpicture}[fixed point arithmetic]%[scale=.9]
\shorthandoff{>}
%
\pgfmathsetmacro{\sx}{.375};% x-scaling
\pgfmathsetmacro{\r}{.05};% radius arc non continuity F_X
%\pgfmathsetmacro{\p}{1/3};% probabilidad p
\pgfmathsetmacro{\n}{100};% numero n de la poblacion
\pgfmathsetmacro{\k}{12};% numero k de estados exitosos
\pgfmathsetmacro{\rr}{40};% numero de rechazos para parar
%
% primera proba (n-k) (n-k-1) ... (n-k-r+1) / n (n-1) ... (n-r+1)
% debhiperneg(\n,\k,\rr)
\tikzmath{function debhiperneg(\a,\b,\c) {
    if \c == 0 then {return 1;}
    else {return ((\a-\b)/\a)*debhiperneg(\a-1,\b,\c-1);};
};};
%

\pgfmathsetmacro{\ui}{2}; % numeros de indices finales nulos dibujados
%
% masa
\begin{scope}
%
\pgfmathsetmacro{\sy}{10};% y-scaling
%
% proba nulas del principio 0 -> d-1
\foreach \y in {-2,...,-1} {
\pgfmathsetmacro{\xl}{int(\y)};\global\let\x\xl;
\draw ({\sx*\x},0)--({\sx*\x},-.1) node[below,scale=.7]{$\x$};
\draw ({\sx*\x},0) node[scale=.6]{$\bullet$};
}
%
% proba nulas del fin f+1 -> ui
\foreach \y in {1,...,\ui} {
\pgfmathsetmacro{\xl}{int(\k+\y)};\global\let\x\xl;
\draw ({\sx*\x},0)--({\sx*\x},-.1) node[below,scale=.7]{$\x$};
\draw ({\sx*\x},0) node[scale=.6]{$\bullet$};
}
%
\pgfmathsetmacro{\pr}{debhiperneg(\n,\k,\rr)};% valor de la primer proba no nula
\pgfmathsetmacro{\maxp}{\pr};% proba maximal (inicializacion)
%
\foreach \x in {0,...,\k} {
%\pgfmathsetmacro{\xl}{int(\f-\y)};\global\let\x\xl;
\draw ({\sx*\x},0)--({\sx*\x},-.1) node[below,scale=.7]{$\x$};
\draw[dotted] ({\sx*\x},0)--({\sx*\x},{\sy*\pr}) node[scale=.6]{$\bullet$};
%
\pgfmathsetmacro{\prl}{\pr*(\x+\rr)*(\k-\x)/((\x+1)*(\n-\rr-\x))};\global\let\pr\prl;% proba actualizado
\pgfmathsetmacro{\maxpl}{(abs(\pr-\maxp)+\pr+\maxp)/2};\global\let\maxp\maxpl;% proba max actualizado
}
%
\draw[>=stealth,->] ({-2*\sx-.25},0)--({\sx*(\k+\ui)+.35},0) node[right]{\small $x$};
\draw[>=stealth,->] (0,-.15)--(0,{\sy*\maxp+.25}) node[above]{\small $p_X$};
%\draw (0,{((1-\p)^\n)*\sy})--(-.1,{((1-\p)^\n)*\sy}) node[left,scale=.7]{$(1-p)^n$};
%\draw (0,{\n*\p*((1-\p)^(\n-1))*\sy})--(-.1,{\n*\p*((1-\p)^(\n-1))*\sy}) node[left,scale=.7]{$n \, p \, (1-p)^{n-1}$};
%
\node at ({(\sx*(\ui+\k)+.25)/2},-1) [scale=.9]{(a)};
\end{scope}
%
%
% reparticion
\begin{scope}[xshift=8.25cm]
%
\pgfmathsetmacro{\sy}{2.5};% y-scaling 
%
\draw[>=stealth,->] ({-2*\sx-.25},0)--({\sx*(\k+\ui)+.5},0) node[right]{\small $x$};
\draw[>=stealth,->] (0,-.15)--(0,{\sy+.25}) node[above]{\small $F_X$};
%
% proba nulas del principio 0 -> d-1
\foreach \y in {-2,...,1} {
\pgfmathsetmacro{\xl}{int(\y)};\global\let\x\xl;
\draw ({\sx*\x},0)--({\sx*\x},-.1) node[below,scale=.7]{$\x$};
}
\draw ({-2*\sx},0)--(0,0);
%
% proba nulas del fin f+1 -> ui
\foreach \y in {0,...,\ui} {
\pgfmathsetmacro{\xl}{int(\y+\k)};\global\let\x\xl;
\draw ({\sx*\x},0)--({\sx*\x},-.1) node[below,scale=.7]{$\x$};
}
\draw ({\sx*\k},\sy) node[scale=.6]{$\bullet$} --({\sx*(\k+\ui)},\sy);
%
\pgfmathsetmacro{\pr}{debhiperneg(\n,\k,\rr)};% valor de la primera proba no nula
\pgfmathsetmacro{\cum}{\pr};% valor inicial de la cumulativa
%
\pgfmathsetmacro{\fk}{\k-1}
\foreach \x in {0,...,\fk} {
\draw ({\sx*\x},0)--({\sx*\x},-.1) node[below,scale=.7]{$\x$};
\draw ({\sx*\x},{\sy*\cum}) node[scale=.6]{$\bullet$} --({\sx*(\x+1)},{\sy*\cum});
\draw ({\sx*(\x+1)+\r},{\sy*\cum+\r}) arc (90:270:\r);
\draw[dotted] ({\sx*\x},{\sy*(\cum-\pr)})--({\sx*\x},{\sy*\cum});
%
\pgfmathsetmacro{\prl}{\pr*(\x+\rr)*(\k-\x)/((\x+1)*(\n-\rr-\x))};\global\let\pr\prl;% proba actualizado
\pgfmathsetmacro{\cuml}{\cum+\pr}\global\let\cum\cuml;% cumulativa actualizada
}
\draw (0,\sy)--(-.1,\sy) node[left,scale=.7]{$1$};
%\draw (0,{((1-\p)^\n)*\sy})--(-.1,{((1-\p)^\n)*\sy}) node[left,scale=.7]{$(1-p)^n$};
%\draw (0,{\n*\p*((1-\p)^(\n-1))*\sy})--(-.1,{\n*\p*((1-\p)^(\n-1))*\sy}) node[left,scale=.7]{$n \, p \, (1-p)^{n-1}$};
%
\node at ({(\sx*(\ui+\k)+.5)/2},-1) [scale=.9]{(b)};
\end{scope}
%
\end{tikzpicture} \end{center}
%
\leyenda{Ilustraci\'on  de una  distribuci\'on  de probabilidad  Hipergeometrica
  (a), y la funci\'on de repartici\'on asociada (b), con \ $n = 100$, \quad $k =
  12$, \quad $r = 40$.}
\label{Fig:MP:HipergeometricaNegativa}
\end{figure}

\SZ{Otros ilustraciones para otros $n, k, r$?}

Notar: cuando  $k =  0$, la  variable es cierta  $X =  r$ (se  sortean solamente
elementos de la seconda clase, as\'i  que para siempre cuando se han tirados $r$
elementos); cuando  $r =  0$, tambi\'en  la variable es  cierta $X  = 0$  (no se
sortan bolas, as\'i que no hay de la primera clase).

% {\displaystyle NHG_{N,K,r}(k)=1-HG_{N,N-K,k}(r-1)}


% --------------------------------- Hipergeometrica multivariada
%\subsubseccion{Ley hipergeometrica multivariada}
\label{Sssec:MP:HipergeometricaMultivariada}

Esta ley aparece por ejemplo cuando  se generaliza la ley hipergeom\'etrica con $c
> 2$ clases \ con \ $k_i$ \ num\'ero de elementos de la clase \ $i$, $\sum_i k_i
= n$.   Se estudia esta ley,  entre otros, por la  primera vez, en  el ensayo de
Montmort  en  1708~\cite{Mon13},  o  m\'as   tarde,  en  1740,  en  trabajos  de
Simpson~\cite{Sim40, Hal90, DavEdw01}.

Se denota \ $X \, \sim \, \H\M(n,k,m)$ \ con \ $\displaystyle n \in \Nset$, \quad
$k = \begin{bmatrix} k_1 & \cdots & k_c\end{bmatrix}^t \in \Part{n}{c}$ (ver notaciones)
% \left\{ \{ 0 \; \ldots  \; n\}^c  \tq \sum_{i=1}^c k_i  = n \right\}$,
  \quad $m \in  \{ 0 \;  \ldots \;
n\}$.

Entonces, como en el caso de la ley multinomial, a pesar de que se escribe \ $X$
\  de manera  $c$-dimensional, el  vector partenece  a un  espacio  claramente \
$(c-1)$-dimensional.  Notar que  en el  caso \  $c  = 2$  \ se  recupera la  ley
hipergeom\'etrica.

Sus caracter\'isticas son las siguientes:

\begin{caracteristicas}
%
Dominio de definici\'on & $\displaystyle \X = \left\{ \left. x
\in \optimes_{i=1}^c \{ 0 \; \ldots \; k_i \} \, \right| \, \sum_{i=1}^c x_i = m
\right\}$\\[2mm]
\hline
%
Par\'ametros & $n \in \Nset_0$ \:
(poblaci\'on)\newline $c \in \Nset_0$ \: (n\'umero de clases)\newline
$\displaystyle k \in \Part{n,c}$ 
%\left\{ q \in \{ 0 \; \ldots \; n\}^c \tq \sum_{i=1}^c q_i
%= n \right\}$ 
\ (n\'umeros de elementos de cada clase)\newline $m \in \{ 0 \;
\ldots \; n\}$ \: (n\'umero de tiros)\\[2mm]
\hline
%
%Distribuci\'on de probabilidad
Funci\'on de masa &
\protect$\displaystyle p_X(x) = \frac{\prod_{i=1}^c
\smallbino{k_i}{x_i}}{\smallbino{n}{m}}$\protect\\[2mm]
\hline
%
Promedio & $\displaystyle m_X = \frac{m}{n} \, k$\\[2mm]
\hline
%
Covarianza
%~\footnote{Ver notas de pie~\ref{Foot:MP:HipergeometricaVarianza} y~\ref{Foot:MP::MultinomialCovarianza}.} 
& $\displaystyle \Sigma_X = \left\{
\protect\begin{array}{ccc} \frac{m (n-m)}{n^2 (n-1)} \left( n \Diag(k) - k k^t
\right) & \mbox{si} & n > 1\\ 0 & \mbox{si} & n = 1\end{array}\protect\right.$
%\\[2mm]
%\hline
%
%
%Generadora  de probabilidad~\footnote{Ver nota de pie~\ref{Foot:MP:MultinomialGeneProba}
%reemplazando $n$ por $m$.}  &  $\displaystyle  G_X(z) =  \left(  1 -  p  + p  z
%\right)^n$ \ sobre \ $\Cset$\\[2mm]
%\hline
%%
%Generadora  de momentos~\footnote{Ver nota de pie~\ref{Foot:MP:MultinomialGeneMomentos}
%reemplazando $n$ por $m$.}  &  $\displaystyle  M_X(u) =  \left(1  - p  +  p \,  e^u
%\right)^n$ \ sobre \ $\Cset$\\[2mm]
%\hline
%%
%Funci\'on caracter\'istica~\footnote{Ver nota de pie~\ref{Foot:MP:MultinomialCaracteristica}
%reemplazando $n$ por $m$.}  & $\displaystyle \Phi_X(\omega) =  \left( 1 -  p + p
%\, e^{\imath \omega} \right)^n$
\end{caracteristicas}

De hecho, se puede considerar que el vector aleatorio es \ $(c-1)$-dimensional \
$\widetilde{X}     =    \begin{bmatrix}     \widetilde{X}_1    &     \cdots    &
  \widetilde{X}_{c-1}   \end{bmatrix}^t$   \  definido   sobre   el  dominio   \
$\displaystyle    \widetilde{\X}   =    \left\{    \left.   \widetilde{x}    \in
    \optimes_{i=1}^{c-1} \{ 0 \; \ldots \; k_i\} \, \right| \, \max\left( 0 \, ,
    \,  \sum_{i=1}^{c-1}  k_i  +  m  -  n \right)  \:  \le  \:  \sum_{i=1}^{c-1}
  \widetilde{x}_i \: \le \: m \right\}$. Los par\'ametros de \ $\widetilde{X}$ \
son  entonces \  $n \in  \Nset_0$, \  $m \in  \{  0 \;  \ldots \;  n \}$  \ y  \
$\widetilde{k}     =     \protect\begin{bmatrix}      k_1     &     \cdots     &
  k_{c-1} \end{bmatrix}^t\protect \in \left\{ q \in \{ 0 \; \ldots \; n \}^{c-1}
  \tq \sum_{i=1}^{c-1} q_i  \, \le \, n \right\}$. A  continuaci\'on, la masa de
probabilidad   de  \   $\widetilde{X}$  \   es  naturalmente   \  $\displaystyle
p_{\widetilde{X}}(x)  =  \frac{\prod_{i=1}^{c-1} \bino{k_i\vspace{1mm}}{x_i}  \,
  \smallbino{n   -   \sum_{i=1}^{c-1}   k_i\vspace{2mm}}{m  -   \sum_{i=1}^{c-1}
    x_i}}{\bino{n}{m}}$.

Similarmente al  caso multinomial, se puede ver  que $\Sigma_X \uno =  0$ \ as\'i
que \ $\Sigma_X \notin \Pos_k^+(\Rset)$.  De nuevo, es la consecuencia directa del
hecho   de  que   \  $X$   \  $c$-dimensional,   vive  sobre   una   variedad  \
$(c-1)$-dimensional. Aparentemente, siendo $\Sigma_X$ no invertible, no se puede
definir  ni  asimetr\'ia, ni  curtosis.  Sin  embargo,  habr\'ia para  esta  ley
tambi\'en  que  considerar  \   $\widetilde{X}$,  de  promedio  $\frac{m}{n}  \,
\widetilde{k}$ y de covarianza el bloque $(c-1) \times (c-1)$ de $\Sigma_X$, que
es ahora  invertible. $\gamma_{\widetilde{X}}$ \ y  \ $\kappa_{\widetilde{X}}$ \
son bien definidos.  Las expresiones, demasiado pesadas, no  son dadas ac\'a (se
deja al lector como exercicio).


\SZ{Ver si se calcula Phi}

La masa  de probabilidad es representada  en la
figura Fig.~\ref{Fig:MP:HipergeometricaMultivariada}.
%
\begin{figure}[h!]
\begin{center} %\begin{tikzpicture}[fixed point arithmetic,scale=.8]
\begin{tikzpicture}[scale=.8]
\shorthandoff{>}
%
%
%\pgfmathsetmacro{\n}{5};% numeros para la multinomial
\pgfmathsetmacro{\dec}{.5};% shitf para dibujar las marginales
%
% ratio de pochammer decrecientes (b)_c / (a)_c
\tikzmath{function poc(\a,\b,\c) {
    if \c == 0 then {return 1;}
    else {return ((\b/\a)*poc(\a-1,\b-1,\c-1);};
};};
%
%
% Ejemplo
\begin{scope}
%
% c = 3 clases
\pgfmathsetmacro{\ku}{9};% k_1
\pgfmathsetmacro{\kd}{6};% k_2
\pgfmathsetmacro{\n}{18};% n
\pgfmathsetmacro{\m}{5};% m
%
\pgfmathsetmacro{\k}{\ku+\kd};% k_1+k_2
% Nota : con el fixed point, no anda min & max
% pero max(a,b) = (a+b+abs(a-b))/2  & min(a,b) = (a+b-abs(a-b))/2;
\pgfmathsetmacro{\s}{int((\m-\n+\k+abs(\m-\n+\k))/2)}; % x1+x2 min posible 
\pgfmathsetmacro{\S}{int((\m+\k-abs(\m-\k))/2)}; % x1+x2 max posible
%
\pgfmathsetmacro{\su}{int((\m-\n+\ku+abs(\m-\n+\ku))/2)}; % x1 min posible 
\pgfmathsetmacro{\Su}{int((\m+\ku-abs(\m-\ku))/2)}; % x1 max posible
%
\pgfmathsetmacro{\sd}{int((\m-\n+\kd+abs(\m-\n+\kd))/2)}; % x2 min posible 
\pgfmathsetmacro{\Sd}{int((\m+\kd-abs(\m-\kd))/2)}; % x2 max posible 

\begin{axis}[
    colormap = {whiteblack}{color(0cm)  = (white);color(1cm) = (black)},
    width=.55\textwidth,
    %height=.5\textheight,%\axisdefaultheight
    view={35}{60},
    enlargelimits=false,
    xmin={-\dec},
    xmax={\Su+\dec},
    ymin={-\dec},
    ymax={\Sd+\dec},
    zmax={.42},
    color=black,
    xtick={0,...,\Su},
    ytick={0,...,\Sd},
    xlabel=$x_1$,
    ylabel=$x_2$,
    zlabel=$p_{\widetilde{X}}$,
]
%

%
% Marginale 1
\pgfmathsetmacro{\bu}{1}; % init primer coef bino (k1 x1)
% Inicialisacion parte 2 de la probabilidad marginal, i.e. 2nd coef. binomial / (n m)
\pgfmathsetmacro{\bd}{poc(\n-\ku+\su,\m,\su)*poc(\n,\n-\ku+\su,\m)};
\foreach \xu in {0,...,\Su} { % bucla en x_1
   \ifnum\numexpr\xu > \numexpr\su-1
      \ifnum\numexpr\xu < \numexpr\Su+1
         \addplot3 [dotted,domain=0:{\bu*\bd},samples=2, samples y=0,color=black]
          (\xu,{\Sd+\dec},\x)  node[scale=.55]{$\bullet$};
      \fi
   \fi
   \pgfmathsetmacro{\blu}{\bu*(\ku-\xu)/(\xu+1)};
   \global\let\bu\blu;% parte proba en x1 actualizado
   \pgfmathsetmacro{\bld}{\bd*(\m-\xu)/(\n-\m-\ku+\xu+1)};
   \global\let\bd\bld;% parte 2 de la proba actualizado
   %
   % lineas x1 abajo
   \addplot3 [domain={-\dec}:{\Sd+\dec},samples=2, samples y=0,color=black!10] (\xu,\x,0);
}
\node at (axis cs:{3*\Su/4},{\Sd+\dec},{.25})[right]{$p_{X_1}$};
%
%
% Marginale 2
\pgfmathsetmacro{\bd}{1}; % init primer coef bino (k2 x2)
% Inicialisacion parte 2 de la probabilidad marginal, i.e. 2nd coef. binomial / (n m)
\pgfmathsetmacro{\bu}{poc(\n-\kd+\sd,\m,\sd)*poc(\n,\n-\kd+\sd,\m)};
\foreach \xd in {0,...,\Sd} { % bucla en x_2
   \ifnum\numexpr\xd > \numexpr\sd-1
      \ifnum\numexpr\xd < \numexpr\Sd+1
         \addplot3 [dotted,domain=0:{\bd*\bu},samples=2, samples y=0,color=black]
          ({-\dec},\xd,\x)  node[scale=.55]{$\bullet$};
      \fi
   \fi
   \pgfmathsetmacro{\bld}{\bd*(\kd-\xd)/(\xd+1)};
   \global\let\bd\bld;% parte proba en x2 actualizado
   \pgfmathsetmacro{\blu}{\bu*(\m-\xd)/(\n-\m-\kd+\xd+1)};
   \global\let\bu\blu;% parte 2 de la proba actualizado
   %
   % lineas x2 abajo
   \addplot3 [domain={-\dec}:{\Su+\dec},samples=2, samples y=0,color=black!10] (\x,\xd,0);
}
\node at (axis cs:{-\dec},{.6*\Sd},{.25})[right]{$p_{X_2}$};
%
% bivariada
%
% Inicialisacion parte 3 de la probabilidad, i.e. terco coef. binomial / (n m)
\pgfmathsetmacro{\bt}{poc(\n-\k+\s,\m,\s)*poc(\n,\n-\k+\s,\m)};
%\pgfmathsetmacro{\plim}{.000001};% si debajo de este valor, se pone a cero (liberar memorio) 
%\pgfmathsetmacro{\sy}{1.1};% scaling en y, para la bivariada 
%
% ahora bucla sobre x = x1+x2
\foreach \xs in {\s,...,\S} {
   \pgfmathsetmacro{\bu}{1};% inic coef. bino. parte x1 de la proba
   \foreach \xu in {0,...,\ku} { % bucla en x_1
      \pgfmathsetmacro{\bd}{1};% inic coef. bino. parte x2 de la proba
      \foreach \xd in {0,...,\kd} { % bucla en x_2
         %\pgfmathsetmacro{\tx}{\xu+\xd};
         \pgfmathparse{int(round(\xu+\xd-\xs))};\let\dif\pgfmathresult;
         \ifnum\dif=0 %\numexpr\xu+\xd = \numexpr\xs % si x_1+x_2 = x que fijamos
            %\pgfmathsetmacro{\pr}{\bu*\bd*\bt};
            \addplot3 [dotted,domain=0:{\bu*\bd*\bt},samples=2, samples y=0,color=black]
            (\xu,\xd,\x)  node[scale=.85]{$\bullet$};
         \fi
         \pgfmathsetmacro{\bld}{\bd*(\kd-\xd)/(\xd+1)};
         \global\let\bd\bld;% parte proba en x2 (x1 fijo) actualizado
      }
      \pgfmathsetmacro{\blu}{\bu*(\ku-\xu)/(\xu+1)};
      \global\let\bu\blu;% parte proba en x1 actualizado
   }
   \pgfmathsetmacro{\blt}{\bt*(\m-\xs)/(\n-\m-\k+\xs+1)};
   \global\let\bt\blt;% parte 3 de la proba actualizado
}
\end{axis}
\node at ({.6*\Su},-1)[scale=.9]{(a)};
\end{scope}
%
%
% -----------------------------------
%
% Ejemplo 
\begin{scope}[xshift = 10.5cm]
%
% c = 3 clases
\pgfmathsetmacro{\ku}{6};% k_1
\pgfmathsetmacro{\kd}{6};% k_2
\pgfmathsetmacro{\n}{18};% n
\pgfmathsetmacro{\m}{5};% m
%
\pgfmathsetmacro{\k}{\ku+\kd};% k_1+k_2
% Nota : con el fixed point, no anda min & max
% pero max(a,b) = (a+b+abs(a-b))/2  & min(a,b) = (a+b-abs(a-b))/2;
\pgfmathsetmacro{\s}{int((\m-\n+\k+abs(\m-\n+\k))/2)}; % x1+x2 min posible 
\pgfmathsetmacro{\S}{int((\m+\k-abs(\m-\k))/2)}; % x1+x2 max posible
%
\pgfmathsetmacro{\su}{int((\m-\n+\ku+abs(\m-\n+\ku))/2)}; % x1 min posible 
\pgfmathsetmacro{\Su}{int((\m+\ku-abs(\m-\ku))/2)}; % x1 max posible
%
\pgfmathsetmacro{\sd}{int((\m-\n+\kd+abs(\m-\n+\kd))/2)}; % x2 min posible 
\pgfmathsetmacro{\Sd}{int((\m+\kd-abs(\m-\kd))/2)}; % x2 max posible 

\begin{axis}[
    colormap = {whiteblack}{color(0cm)  = (white);color(1cm) = (black)},
    width=.55\textwidth,
    %height=.5\textheight,%\axisdefaultheight
    view={35}{60},
    enlargelimits=false,
    xmin={-\dec},
    xmax={\Su+\dec},
    ymin={-\dec},
    ymax={\Sd+\dec},
    zmax={.42},
    color=black,
    xtick={0,...,\Su},
    ytick={0,...,\Sd},
    xlabel=$x_1$,
    ylabel=$x_2$,
    zlabel=$p_{\widetilde{X}}$,
]
%
%
% Marginale 1
\pgfmathsetmacro{\bu}{1}; % init primer coef bino (k1 x1)
% Inicialisacion parte 2 de la probabilidad marginal, i.e. 2nd coef. binomial / (n m)
\pgfmathsetmacro{\bd}{poc(\n-\ku+\su,\m,\su)*poc(\n,\n-\ku+\su,\m)};
\foreach \xu in {0,...,\Su} { % bucla en x_1
   \ifnum\numexpr\xu > \numexpr\su-1
      \ifnum\numexpr\xu < \numexpr\Su+1
         \addplot3 [dotted,domain=0:{\bu*\bd},samples=2, samples y=0,color=black]
          (\xu,{\Sd+\dec},\x)  node[scale=.55]{$\bullet$};
      \fi
   \fi
   \pgfmathsetmacro{\blu}{\bu*(\ku-\xu)/(\xu+1)};
   \global\let\bu\blu;% parte proba en x1 actualizado
   \pgfmathsetmacro{\bld}{\bd*(\m-\xu)/(\n-\m-\ku+\xu+1)};
   \global\let\bd\bld;% parte 2 de la proba actualizado
   %
   % lineas x1 abajo
   \addplot3 [domain={-\dec}:{\Sd+\dec},samples=2, samples y=0,color=black!10] (\xu,\x,0);
}
\node at (axis cs:{\ku/2},{\kd+\dec},{.1})[right]{$p_{X_1}$};
%
%
% Marginale 2
\pgfmathsetmacro{\bd}{1}; % init primer coef bino (k2 x2)
% Inicialisacion parte 2 de la probabilidad marginal, i.e. 2nd coef. binomial / (n m)
\pgfmathsetmacro{\bu}{poc(\n-\kd+\sd,\m,\sd)*poc(\n,\n-\kd+\sd,\m)};
\foreach \xd in {0,...,\Sd} { % bucla en x_2
   \ifnum\numexpr\xd > \numexpr\sd-1
      \ifnum\numexpr\xd < \numexpr\Sd+1
         \addplot3 [dotted,domain=0:{\bd*\bu},samples=2, samples y=0,color=black]
          ({-\dec},\xd,\x)  node[scale=.55]{$\bullet$};
      \fi
   \fi
   \pgfmathsetmacro{\bld}{\bd*(\kd-\xd)/(\xd+1)};
   \global\let\bd\bld;% parte proba en x2 actualizado
   \pgfmathsetmacro{\blu}{\bu*(\m-\xd)/(\n-\m-\kd+\xd+1)};
   \global\let\bu\blu;% parte 2 de la proba actualizado
   %
   % lineas x2 abajo
   \addplot3 [domain={-\dec}:{\Su+\dec},samples=2, samples y=0,color=black!10] (\x,\xd,0);
}
\node at (axis cs:{-\dec},{.6*\Sd},{.25})[right]{$p_{X_2}$};
%
% bivariada
%
% Inicialisacion parte 3 de la probabilidad, i.e. terco coef. binomial / (n m)
\pgfmathsetmacro{\bt}{poc(\n-\k+\s,\m,\s)*poc(\n,\n-\k+\s,\m)};
%\pgfmathsetmacro{\plim}{.000001};% si debajo de este valor, se pone a cero (liberar memorio) 
%\pgfmathsetmacro{\sy}{1.1};% scaling en y, para la bivariada 
%
% ahora bucla sobre x = x1+x2
\foreach \xs in {\s,...,\S} {
   \pgfmathsetmacro{\bu}{1};% inic coef. bino. parte x1 de la proba
   \foreach \xu in {0,...,\ku} { % bucla en x_1
      \pgfmathsetmacro{\bd}{1};% inic coef. bino. parte x2 de la proba
      \foreach \xd in {0,...,\kd} { % bucla en x_2
         %\pgfmathsetmacro{\tx}{\xu+\xd};
         \pgfmathparse{int(round(\xu+\xd-\xs))};\let\dif\pgfmathresult;
         \ifnum\dif=0 %\numexpr\xu+\xd = \numexpr\xs % si x_1+x_2 = x que fijamos
            %\pgfmathsetmacro{\pr}{\bu*\bd*\bt};
            \addplot3 [dotted,domain=0:{\bu*\bd*\bt},samples=2, samples y=0,color=black]
            (\xu,\xd,\x)  node[scale=.85]{$\bullet$};
         \fi
         \pgfmathsetmacro{\bld}{\bd*(\kd-\xd)/(\xd+1)};
         \global\let\bd\bld;% parte proba en x2 (x1 fijo) actualizado
      }
      \pgfmathsetmacro{\blu}{\bu*(\ku-\xu)/(\xu+1)};
      \global\let\bu\blu;% parte proba en x1 actualizado
   }
   \pgfmathsetmacro{\blt}{\bt*(\m-\xs)/(\n-\m-\k+\xs+1)};
   \global\let\bt\blt;% parte 3 de la proba actualizado
}
\end{axis}
\node at ({.6*\Su},-1)[scale=.9]{(b)};
%
\end{scope}
%
\end{tikzpicture} \end{center}
%
\leyenda{Ilustraci\'on de  una distribuci\'on de  probabilidad hipergeom\'etrica
  multivariada  para   \  $c  =  3$   \  del  vector   \  $(c-1)$-dimensional  \
  $\widetilde{X} = \protect\begin{bmatrix}  X_1 & X_2 \protect\end{bmatrix}^t$ \
  ($X_3 = m-X_1-X_2$) \ con las marginales \ $p_{X_1}, \: p_{X_2}$.  Es dibujada
  solamente la distribuci\'on sobre $\widetilde{\X}$, siendo esta cero afuera de
  $\widetilde{\X}$.   Los par\'ametros  son  \ $n  =  18$, \  $m =  5$,  \ $k  =
  \protect\begin{bmatrix}  9  &  6   &  3  \protect\end{bmatrix}^t$  (a),  $k  =
  \protect\begin{bmatrix} 6 & 6 & 6 \protect\end{bmatrix}^t$ (b).}
\label{Fig:MP:HipergeometricaMultivariada}
\end{figure}


Notar: cuando $c = 2$ se  recupera la ley hipergeom\'etrica; adem\'as $X$ resuelta
cierta   en  los  casos   siguientes  (ver   subesecci\'on  anterior   para  las
explicaciones/ilustraciones):
%
\begin{itemize}
\item $m =  0 \: \Rightarrow \: X = 0$;
%
\item  $m = n \: \Rightarrow \: X = k$;
%
\item $k = n \uno_i \: \Rightarrow \: X = m \uno_i$.
\end{itemize}

% \SZ{casos particulares a ver; caso de propiedades de reflexividad a ver (igual
%   para la multinomial)}

Vectores  de  distribuci\'on  hipergeom\'etricas multivaluada  tienen  propiedades
notables similares a las de la  hipergeom\'etricas y de la multinomial, a saber de
tipo reflexividad, con respecto a una permutaci\'on de variable y con respecto a
una agregaci\'on.
%
\begin{lema}[Reflexividad]
\label{Lem:MP:ReflexividadHipergeomMulti}
%
  Sea \ $X \, \sim \, \H\M(n,k,m)$. Entonces
  %
  \[
  k-X \sim \H\M(n,k,n-m)
  \]
  %
\end{lema}
%
\begin{proof}
  %El  primer   resultado  es  inmediato  de  $P(m-X   =  x)  =  P(X   =  m-x)  =
  %\frac{\smallbino{k}{m-x} \smallbino{n-k}{x}}{\smallbino{n}{m}}$. El secundo de
  Sea $Y =  k-X$. De $P(Y = y) =  P(k-X = y) = P(X  = k-y) = \frac{\prod_{i=1}^c
    \smallbino{k_i}{k_i-y_i}}{\smallbino{n}{m}}       =      \frac{\prod_{i=1}^c
    \smallbino{k_i}{y_i}}{\smallbino{n}{n-m}}$   notando   que  $\bino{a}{b}   =
  \bino{a}{a-b}$. Se cierra la prueba  recordandose que \ $\sum_{i=1}^c k_i = n$
  \  y  \ $\sum_{i=1}^c  x_i  =  m$,  dando \  $\sum_{i=1}^c  k_i  =  n$ \  y  \
  $\sum_{i=1}^c y_i = n-m$.
\end{proof}
%
Como  el en  contexto  escalar, si  en una  urna  tenemos bolas  de $c$  colores
diferentes,  con  un n\'umero  $k_i$  para el  $i$-\'esimo  color,  $X_i$ es  el
num\'ero de este color que se sorte\'o y $k_i-X_i$ representan las de este color
que quedan en la urna, entre las  \ $n-m = \sum_{i=1}^c (k_i-X_i)$ \ que quedan,
es decir que en \ $k-X$ \ se  intercambia los roles de las bolas sorteadas y las
que quedan en la urna.

\begin{lema}[Efecto de una permutaci\'on]\label{Lem:MP:PermutacionHipergeomMulti}
%
  Sea  \ $X =  \begin{bmatrix} X_1  & \cdots  & X_c  \end{bmatrix}^t \,  \sim \,
  \H\M(n,k,m)$  \   y  \   $\Pi  \in  \Perm_c$   \  matriz   \  de
  permutaci\'on. Entonces
  %
  \[
  \Pi X \, \sim \, \H\M\left( n ,  \Pi k , m \right)
  \]
  %
\end{lema}
%
\begin{proof}
  La  prueba sigue  paso paso  la de  la multinomial.  Notando la  permutation \
  $\sigma$ \  tal que \ $\Pi  = \sum_{i=1}^c \uno_i  \uno_{\sigma(i)}^t$, se puede
  ver   que  \  $\displaystyle   P(\Pi  X   =  x)   =  P(X   =  \Pi^{-1}   x)  =
  \frac{\prod_{i=1}^c       \bino{k_i}{x_{\sigma^{-1}(i)}}}{\bino{n}{m}}       =
  \frac{\prod_{i=1}^c \bino{k_{\sigma(i)}}{x_i}}{\bino{n}{m}} $  \ por cambio de
  indices.
\end{proof}
%
\begin{lema}[Stabilidad por agregaci\'on]\label{Lem:MP:StabAgregacionHipergeomMulti}
%
  Sea  \ $X =  \begin{bmatrix} X_1  & \cdots  & X_c  \end{bmatrix}^t \,  \sim \,
  \H\M(n,k,m)$ \ y \ $G^{(i,j)}$ \ matriz de agrupaci\'on de las $(i,j)$-\'esima
  componentes (ver notaciones). Entonces,
  %
  \[
  G^{(i,j)} X \, \sim \, \H\M\left( n , G^{(i,j)} k , m \right)  
  \]
  %
\end{lema}
%
Este resultado es intuitivo del hecho que  vuelve a agrupar las clases \ $i$ \ e
\ $j$ \ en una clase, que tiene entonces \ $k_i + k_j$ \ elementos.
%

%
\begin{proof}
 Del  lema precediente,
  notando  que  existen  matrices  de permutaci\'on~\footnote{$\Pi_k$  pone  las
    componentes $i$ \ e \ $j$ el  las posiciones $c-1$ y $c$, sin cambiar el orden
    de  las  precedientes;  $\Pi_{k-1}$  trazlada  la  \'ultima  componente  en  la
    posici\'on $\min(i,j)$.}  \ $\Pi_k \in  \Perm_k$ \ y \ $\Pi_{k-1} \in
  \Perm_{k-1}$ \ tal que \ $G^{(i,j)} = \Pi_{k-1} \, G^{(c-1,c)} \, \Pi_k$,
  se puede concentrarse en el caso \ $(i,j) = (c-1,c)$. Ahora, claramente,
  %
  \begin{eqnarray*}
  P(G^{(c-1,c)} X = x) & = & \displaystyle P\left( \bigcap_{i=1}^{c-2} \big( X_i = x_i \big)
  \: \cap \: \big( X_{c-1} + X_c = x_{c-1} \big) \right)\\[2mm]
  %
  & = & \displaystyle \sum_{t=0}^{x_{c-1}} P\left( \bigcap_{i=1}^{c-2} \big( X_i = x_i \big)
  \: \cap \: \big( X_{c-1} = t\big) \: \cap \: \big( X_c = t-x_{c-1} \big) \right)\\[2mm]
  %
  & = & \frac{\prod_{i=1}^{c-2} \bino{k_i}{x_i}}{\bino{n}{m}}
  \: \sum_{t=0}^{x_{c-1}} \bino{k_{c-1}}{t}\bino{k_c}{x_c-t}
  \end{eqnarray*}
  %
  Se  cierra  la  prueba   de  la  identidad  de  Chu-Vandermonde~\footnote{Este
    identidad es  debido a A.-T. Vandermonde  en 1772, pero  esta conocida desde
    1303 por  el matem\'atico chino Chu Shi-Chieh,  explicando la denominaci\'on
    de  este   identidad~\cite{AndLar94}  o~\cite[p.~59-60]{Ask75}.   Se  prueba
    escribiendo $(1+x)^{r+s} = (1+x)^r  (1+x)^s$ y desallorando con la f\'ormula
    del     binomio    cada     potencia.}      $\displaystyle    \sum_{t=0}^{l}
  \bino{r}{t}\bino{s}{l-t}   =  \bino{r+s}{l}$~\cite[Ec.~(21),  p.~59]{Knu97_v1}
  o~\cite[Ec.~0.156]{GraRyz15}.
\end{proof}

De este lema, aplicado de manera recursiva, se obtiene el corolario siguiente:
%
\begin{corolario}\label{Cor:MP:MarginalMultinomial}
%
  Sea  \ $X  \,  \sim \,  \H\M(n,k,m)$, entonces  \  $\displaystyle X_i  \, \sim  \,
  \H(n,k_i,m)$.
\end{corolario}

% Cuando  $n  = 1$,  se  recupera  la lei  de  Bernoulli  $\B(p) \equiv  \B(1,p)$.
% Ad\'emas, se muestra  sencillamente usando la generadora de  probabilidad que
% %
% De este resultado,  se puede notar que, por  ejemplo, le distribuci\'on binomial
% aparece en el conteo de eventos independientes de misma probabilidad entre $n$.

% Tambi\'en,  la ley binomial  tiene una  propiedad de  reflexividad, consecuencia
% directa de la de Bernoulli:
% %
% \begin{lema}[Reflexividad]
% \label{Lem:MP:ReflexividadBinomial}
% %
%   Sea \ $X \, \sim \, \B(n,p)$. Entonces
%   %
%   \[
%   n-X \, \sim \, \B(n,1-p)
%   \]
%   %
% \end{lema}

% Nota que cuando $p = 0$ (resp. $p = 1$) la variable es cierta $X = 0$ (resp.  $X
% = n$).


\

Nota: esta  ley se  generaliza de  la misma manera  que para  la hipergeom\'etrica
negativa,  dando una  ley  hipergeom\'etrica negativa  multivariada  o, de  manera
equivalente, generalizando la hipergeom\'etrica negativa  a m\'as de dos clases se
obtiene la ley hipergeom\'etrica negativa. \SZ{Anadirlo en una seccion?}



% --------------------------------- Geometrica
%\subsubseccion{Ley Geom\'etrica}
\label{Sssec:MP:Geometrica}

Se  denota  \  $X \,  \sim  \,  \G(p)$  \  con \  $p  \in  (0  \;  1]$ \  y  sus
caracter\'isticas son las siguientes:

\begin{caracteristicas}
%
Dominio de definici\'on & $\X = \Nset^*$\\[2mm]
\hline
%
Parametro & $p \in (0 \; 1]$\\[2mm]
\hline
%
Distribuci\'on  de  probabilidad &  $\displaystyle  p_X(x)  =  (1-p)^{x-1} p$  \
(convenci\'on $0^0 = 1$)\\[2mm]
\hline
%
Promedio & $m_X = \frac1p$\\[2mm]
\hline
%
Varianza & $\displaystyle \sigma_X^2 = \frac{1-p}{p^2}$\\[2mm]
\hline
%
\modif{Sesgo} & $\displaystyle \gamma_X = \frac{2-p}{\sqrt{1-p}}$\\[2mm]
\hline
%
Curtosis por exceso & $\displaystyle \widebar{\kappa}_X = \frac{6 - 6 \, p + p^2}{1-p}$\\[2mm]
\hline
%
Generadora de  probabilidad & $\displaystyle  G_X(z) = \frac{p z}{1-(1-p)  z}$ \
para \ $|z| < \frac1{1-p}$\\[2mm]
\hline
%
Generadora de  momentos & $\displaystyle M_X(u)  = \frac{p \, e^u}{1  - (1-p) \,
e^u}$ \ para \ $\real{u} < - \ln(1-p)$\\[2mm]
\hline
%
Funci\'on caracter\'istica  & $\displaystyle \Phi_X(\omega)  = \frac{p \, e^{\imath
\omega}}{1 - (1-p) \, e^{\imath \omega}}$
\end{caracteristicas}

% Momentos & $ \Esp\left[ X^k \right] = ?$\\[2mm]
% Momento factorial & $\Esp\left[ (X)_k \right] = \frac{p^{k-1} k!}{(1-p)^k}$\\[2mm]
% Modo 1
% Mediana $\left\lceil \frac{-1}{\log_2(1-p)} \right\rceil$ 
% CDF	$1-(1-p)^k$

Su masa  de probabilidad  y funci\'on de  repartici\'on son representadas  en la
figura Fig.~\ref{Fig:MP:Geometrica}.
%
\begin{figure}[h!]
\begin{center} \begin{tikzpicture}%[scale=.9]
\shorthandoff{>}
%
\pgfmathsetmacro{\sx}{.75};% x-scaling
\pgfmathsetmacro{\r}{.05};% radius arc non continuity F_X
\pgfmathsetmacro{\p}{1/3};% probabilidad p
\pgfmathsetmacro{\n}{7};% k mas grande del plot (k in Nset^*)
%
% masa
\begin{scope}
%
\pgfmathsetmacro{\sy}{2.5/\p};% y-scaling 
\draw[>=stealth,->] (-.25,0)--({\sx*\n+.75},0) node[right]{\small $x$};
\draw[>=stealth,->] (0,-.1)--(0,{\sy*\p+.25}) node[above]{\small $p_X$};
%
\pgfmathsetmacro{\pr}{\p};% probabilidad
%
\foreach \k in {1,...,\n} {
\draw ({\k*\sx},0)--({\k*\sx},-.1) node[below,scale=.7]{\k};
\draw[dotted] ({\k*\sx},0)--({\k*\sx},{\sy*\pr}) node[scale=.7]{$\bullet$};
%
\pgfmathsetmacro{\prl}{\pr*(1-\p)};\global\let\pr\prl;% proba actualizado
}
\draw ({(\n+.5)*\sx},-.2) node[below,scale=.7]{$\ldots$};
\draw ({(\n+.5)*\sx},{(\pr/(1-\p)/2*\sy}) node[scale=.7]{$\cdots$};
\draw (0,{\p*\sy})--(-.1,{\p*\sy}) node[left,scale=.7]{$p$};
\draw (0,{\p*(1-\p)*\sy})--(-.1,{\p*(1-\p)*\sy}) node[left,scale=.7]{$p \, (1-p)$};
\draw (-.5,{\p*(1-\p)/2*\sy}) node[left,scale=.7]{$\vdots$};
%
\end{scope}
%
%
% reparticion
\begin{scope}[xshift=8.5cm]
%
\pgfmathsetmacro{\sy}{2.5};% y-scaling 
%
\draw[>=stealth,->] (-.6,0)--({\sx*\n+.75},0) node[right]{\small $x$};
\draw[>=stealth,->] (0,-.1)--(0,{\sy+.25}) node[above]{\small $F_X$};
%
\pgfmathsetmacro{\pr}{\p};% probabilidad
\pgfmathsetmacro{\c}{\p};% cumulativa
%
% cumulativa x < 1
\draw (1,0)--(1,-.1) node[below,scale=.7]{0};
\draw[thick] (-.5,0)--(\sx,0);
\draw ({\sx+\r},\r) arc (90:270:\r);
%
% cumulativa x de 1 a n
\foreach \k in {2,...,\n} {
\draw ({\k*\sx},0)--({\k*\sx},-.1) node[below,scale=.7]{\k};
\draw[thick]({(\k-1)*\sx},{\sy*\c}) node[scale=.7]{$\bullet$}--({\k*\sx},{\sy*\c});
\draw ({\k*\sx+\r},{\sy*\c+\r}) arc (90:270:\r);
\draw[dotted] ({(\k-1)*\sx},{(\c-\pr)*\sy})--({(\k-1)*\sx},{\c*\sy});
%
\pgfmathsetmacro{\prl}{\pr*(1-\p)};\global\let\pr\prl;% proba actualizado
\pgfmathsetmacro{\cl}{\c+\pr};\global\let\c\cl;% cumulativa actualizada
}
%
% cumulativa x > n
\draw ({(\n+.5)*\sx},-.2) node[below,scale=.7]{$\ldots$};
\draw ({(\n+.5)*\sx},{((\c+1)/2*\sy}) node[scale=.7]{$\cdots$};
\draw (0,{\p*\sy})--(-.1,{\p*\sy}) node[left,scale=.7]{$p$};
\draw (0,{\p*(2-\p)*\sy})--(-.1,{\p*(2-\p)*\sy}) node[left,scale=.7]{$p \, (2-p)$};
\draw (-.3,{(1+\p*(2-\p))/2*\sy}) node[left,scale=.7]{$\vdots$};
\draw (0,\sy)--(-.1,\sy) node[left,scale=.7]{$1$};
\end{scope}
%
\end{tikzpicture} \end{center}
%
\leyenda{Ilustraci\'on de una distribuci\'on de probabilidad Geom\'etrica (a), y
  la funci\'on de repartici\'on asociada (b), con $p = \frac13$.}
\label{Fig:MP:Geometrica}
\end{figure}
\SZ{Otros ilustraciones para otros $p$?}

Esta distribuci\'on  aparece en el conteo  de conteo de une  repetici\'on de una
experiencia de maneja  independiente hasta que occure un  evento de probabilidad
$p$; por ejemplo  el n\'umero de tiro de un dado  equilibriado hasta que occurre
un ``6'' sigue una ley geom\'etrica de parametro $p = \frac16$.

Nota que cuando \  $p =  1$ \ la variable es cierta \  $X = 1$.   


\SZ{?`Que propiedad mas?}


% --------------------------------- Poisson
%\subsubseccion{Ley de Poisson}
\label{Sssec:MP:Poisson}

Esta  ley fue  introducida por  Poisson en  1837 como  caso l\'imite  de  la ley
binomial     para     $n$     grande,      con     el     producto     $n     p$
fijo~\cite[Cap.~3]{Poi37},~\cite{Hal90, DavEdw01}.  Se  interes\'o Poisson en su
estudio  al  comportamentio  probabil\'istico   del  conteo  de  experiencia  de
Bernoulli bajo la hipotesis de independencia  (dando lugar a la ley binomial) en
ciencia humana, para una poblaci\'on  importante ($n$ grande), pero con un valor
promedio dado.  De hecho, se conoc\'ia esta ley, tambi\'en como caso l\'imite de
la  binomial,  por  lo  menos  desde  un  trabajo  de  de  Moivre  unas  decadas
antes~\cite{Moi10}.   Apareci\'o  tambi\'en   m\'as  tarde  en  muchos  procesos
f\'isicos, como el conteo de desintegraci\'on atomica por secundo en un material
radioactivo, o, (aproximadamente) a trav\'es del conteo de part\'iculas que caen
en una peque\~na  superficia, cuanto se tiran part\'iculas  uniformamente en una
grande superficia  en trabajos de  W. S. Gosset~\footnote{Fue connocido  bajo en
  nombre ``Student''; ver nota de pie~\ref{Foot:MP:Student}.}~\cite{Stu07}.

Se denota $X \,  \sim \, \P(\lambda)$ \ con \ $\lambda  \in \Rset_{0,+}$ \ llamada
{\em taza}, y sus caracter\'isticas son las siguientes:

\begin{caracteristicas}
%
Dominio de definici\'on & $\X = \Nset$\\[2mm]
\hline
%
Par\'ametro & $\lambda \in \Rset_{0,+}$\\[2mm]
\hline
%
%Distribuci\'on  de  probabilidad
Funci\'on de masa &  $\displaystyle  p_X(x)  =  \frac{\lambda^x
e^{-\lambda}}{x!}$\\[2mm]
\hline
%
Promedio & $ m_X = \lambda$\\[2mm]
\hline
%
Varianza & $\sigma_X^2 = \lambda$\\[2mm]
\hline
%
\modif{Asimetr\'ia} & $\displaystyle \gamma_X = \frac1{\sqrt\lambda}$\\[2mm]
\hline
%
Curtosis por exceso & $\displaystyle \widebar{\kappa}_X = \frac1\lambda$\\[2mm]
\hline
%
Generadora de probabilidad & $\displaystyle G_X(z) = e^{\lambda (z-1)}$ \quad para \
$z \in \Cset$\\[2mm]
\hline
%
Generadora  de momentos  & $\displaystyle  M_X(u) =  e^{\lambda \left(  e^u  - 1
\right)}$ \quad para \ $u \in \Cset$\\[2mm]
\hline
%
Funci\'on  caracter\'istica  &  $\displaystyle  \Phi_X(\omega) =  e^{\lambda  \,
\left( e^{\imath \omega} - 1 \right)}$
\end{caracteristicas}

% Momentos & $ \Esp\left[ X^k \right] = ?$\\[2mm]
% Momento factorial & $\Esp\left[ (X)_k \right] = \lambda^k$\\[2mm]
% modo \lfloor \lambda \rfloor 
% Mediana \approx \lfloor \lambda +1/3-0.02/\lambda \rfloor 
% CDF {\frac {\Gamma
% (\lfloor k+1\rfloor  ,\lambda )}{\lfloor k\rfloor !}} where  $\Gamma (x,y)$ is
% the upper incomplete gamma function,

Su masa  de probabilidad  y funci\'on de  repartici\'on son representadas  en la
figura Fig.~\ref{Fig:MP:Poisson}.
%
\begin{figure}[h!]
\begin{center} \begin{tikzpicture}%[scale=.9]
\shorthandoff{>}
%
\pgfmathsetmacro{\sx}{.75};% x-scaling
\pgfmathsetmacro{\r}{.05};% radius arc non continuity F_X
\pgfmathsetmacro{\l}{3};% lambda
\pgfmathsetmacro{\n}{7};% k mas grande del plot (k in Nset)
\pgfmathsetmacro{\q}{floor(\l)};% modo
\pgfmathsetmacro{\m}{(\l^\q)*exp(-\l)/factorial(\q)};% maximo
%
% masa
\begin{scope}
%
\pgfmathsetmacro{\sy}{2.75/\m};% y-scaling 
\draw[>=stealth,->] (-.25,0)--({\sx*\n+.75},0) node[right]{\small $x$};
\draw[>=stealth,->] (0,-.1)--(0,{\sy*\m+.25}) node[above]{\small $p_X$};
%
\pgfmathsetmacro{\pr}{exp(-\l)};% probabilidad
%
\foreach \k in {0,...,\n} {
\draw ({\k*\sx},0)--({\k*\sx},-.1) node[below,scale=.7]{\k};
\draw[dotted] ({\k*\sx},0)--({\k*\sx},{\sy*\pr}) node[scale=.7]{$\bullet$};
%
\pgfmathsetmacro{\prl}{\pr*\l/(\k+1)};\global\let\pr\prl;% proba actualizado
}
\draw ({(\n+.5)*\sx},-.2) node[below,scale=.7]{$\ldots$};
\draw ({(\n+.5)*\sx},{(\pr/\l*\n/2*\sy}) node[scale=.7]{$\cdots$};
\draw (0,{exp(-\l)*\sy})--(-.1,{exp(-\l)*\sy}) node[left,scale=.7]{$e^{-\lambda}$};
\draw (0,{\l*exp(-\l)*\sy})--(-.1,{\l*exp(-\l)*\sy}) node[left,scale=.7]{$\lambda e^{-\lambda}$};
\draw (0,{\l*\l*exp(-\l)/2*\sy})--(-.1,{\l*\l*exp(-\l)/2*\sy}) node[left,scale=.7]{$\frac{\lambda^2 e^{-\lambda}}{2}$};
%\draw (-.5,{\l*exp(-\l)/2*\sy}) node[left,scale=.7]{$\vdots$};
%
\end{scope}
%
%
% reparticion
\begin{scope}[xshift=8.5cm]
%
\pgfmathsetmacro{\sy}{2.75};% y-scaling 
%
\draw[>=stealth,->] (-.6,0)--({\sx*\n+.75},0) node[right]{\small $x$};
\draw[>=stealth,->] (0,-.1)--(0,{\sy+.25}) node[above]{\small $F_X$};
%
\pgfmathsetmacro{\pr}{exp(-\l)};% probabilidad
\pgfmathsetmacro{\c}{exp(-\l)};% cumulativa
%
% cumulativa x < 0
\draw (0,0)--(0,-.1) node[below,scale=.7]{0};
\draw[thick] (-.5,0)--(0,0);
\draw (\r,\r) arc (90:270:\r);
%
% cumulativa x de 0 a n
\foreach \k in {1,...,\n} {
\draw ({\k*\sx},0)--({\k*\sx},-.1) node[below,scale=.7]{\k};
\draw[thick]({(\k-1)*\sx},{\sy*\c}) node[scale=.7]{$\bullet$}--({\k*\sx},{\sy*\c});
\draw ({\k*\sx+\r},{\sy*\c+\r}) arc (90:270:\r);
\draw[dotted] ({(\k-1)*\sx},{(\c-\pr)*\sy})--({(\k-1)*\sx},{\c*\sy});
%
\pgfmathsetmacro{\prl}{\pr*\l/\k};\global\let\pr\prl;% proba actualizado
\pgfmathsetmacro{\cl}{\c+\pr};\global\let\c\cl;% cumulativa actualizada
}
%
% cumulativa x > n
\draw ({(\n+.5)*\sx},-.2) node[below,scale=.7]{$\ldots$};
\draw ({(\n+.5)*\sx},{((\c+1)/2*\sy}) node[scale=.7]{$\cdots$};
\draw (0,{exp(-\l)*\sy})--(-.1,{exp(-\l)*\sy}) node[left,scale=.7]{$e^{-\lambda}$};
\draw (0,{(1+\l)*exp(-\l)*\sy})--(-.1,{(1+\l)*exp(-\l)*\sy}) node[left,scale=.7]{$(1+\lambda) e^{-\lambda}$};
\draw (-.3,{(1+(1+\l+\l*\l/2)*exp(-\l))/2*\sy}) node[left,scale=.7]{$\vdots$};
\draw (0,\sy)--(-.1,\sy) node[left,scale=.7]{\small $1$};
\end{scope}
%
\end{tikzpicture} \end{center}
%
\leyenda{Ilustraci\'on de  una distribuci\'on de probabilidad de  Poisson (a), y
  la funci\'on de repartici\'on asociada (b), con $\lambda = 3$.}
\label{Fig:MP:Poisson}
\end{figure}

\SZ{Otras ilustraciones para otros $\lambda$?}

Ad\'emas, se muestra  sencillamente usando la generadora de  probabilidad que
%
\begin{lema}[Stabilidad]
\label{Lem:MP:StabilidadPoisson}
%
  Sean  \  $X_i  \,  \sim  \,  \P(\lambda_i),  \quad  i  =  1,  \ldots  ,  n$  \
  independientes, entonces
  %
  \[
  \sum_{i=1}^n X_i \, \sim \, \P\left( \sum_{i=1}^n \lambda_i \right)
  \]
\end{lema}


Como lo hemos introducido, la ley de Poisson esta v\'inculada a la ley binomial, como caso l\'imite:
%
\begin{lema}[V\'inculo con la ley binomial]
\label{Lem:MP:VinvuloPoissonBinomial}
%
  Sean  \  $X_n  \,  \sim  \,  \B\left( n \, , \, \frac{\lambda}{n} \right)$  \
  con $\lambda > 0$ fijo, entonces
  %
  \[
  X_n \, \limitd{n \to \infty} \, X \, \sim \, \P(\lambda)
  \]
  %
  donde  \ $\limitd{}$ \  significa que  el l\'imite  es en  distribuci\'on (ver
  notaciones).
\end{lema}
\begin{proof}
  Se  sale  de la  forma  de  la distribuci\'on  binomial  y  de  la formula  de
  Stirling~\footnote{De hecho, esta  formula es probablemente debida previamente
    a  A.  De  Moivre~\cite{Moi33, Moi56,  Pea24,  Cam86, Dut91,  Dem33}, y  fue
    mejorada por  Stirling m\'as tarde. Fue  mejorada a\'un m\'as  por el famoso
    matem\'atico                          S.                           Ramanujan
    recientemente~\cite[\S~4.1]{AndBer13}.\label{Foot:MP:Stirling}}:            \
  $\log\Gamma(z) = \left( z - \frac12 \right) \log z - z + \frac12 \log(2 \pi) +
  o(1)$ \ en \ $z \to +\infty$~\cite{Sti30, AbrSte70, GraRyz15}.
\end{proof}

Aparece  que la  ley de  Poisson esta  v\'inculada tambi\'en  a la  ley binomial
negativa, tambi\'en como caso l\'imite:
%
\begin{lema}[V\'inculo con la binomial negativa]
\label{Lem:MP:VinvuloPoissonBinomialNegativa}
%
Sean \ $X_r \, \sim \, \B_-\left( \frac{\lambda}{r+\lambda} \, , \, r \right)$ \
con $\lambda > 0$ fijo, entonces
  %
  \[
  X_r \, \limitd{r \to \infty} \, X \, \sim \, \P(\lambda)
  \]
\end{lema}
\begin{proof}
  Se  sale de nuevo  la forma  de la  distribuci\'on binomial  negativa y  de la
  formula de Stirling para probarlo.
\end{proof}

M\'as all\'a  del contexto discreto, esta  ley esta tambi\'en  v\'inculada a ley
exponencial, por  el processo dicho de  Poisson.  Si eventos  pueden aparecer de
manera aleatoria  en el tiempo tal que,  entre dos eventos, el  tiempo sigue una
ley   exponencial  de   par\'ametro   $\lambda$,  y   que   estos  tiempos   son
independientes, entonces dado  un intervalo $T$ de tiempo,  el n\'umero de estos
eventos sigue una ley de Poisson de  par\'ametro $\lambda T$.  Lo vamos a ver en
el ejemplo de la ley exponencial m\'as adelante.

Al  final, notar  que  cuando  $\lambda =  0$  la variable  es  cierta  $X =  0$
(con la convenci\'on $0^0 \igualc 1$, de $\displaystyle \lim_{x \to 0^+} x^x = 1$).



\SZ{
% --------------------------------- Familia power series
\subsubseccion{Distribuci\'on seria de potencia (power series distributions)?}

}

%%%%%%%%%%%%%%%%%%%%%%%%%%%%%%%%%%%%%%%%%%%%%%%%%%%%%%%%%%%%%%%%%%%%%%%%%%%%%%%%
\aver{ Estad\'istica  de los n\'umeros de ocupaci\'on  de niveles energ\'eticos:
distribuciones     de    Maxwell--Boltzmann,     de    Fermi--Dirac,     y    de
Bose--Einstein~\cite[p.  37-38]{Ren07}\newline  Leyes de los  grandes n\'umeros;
DeMoivre-Laplace; $F$?  inverse gamma, Rayleigh  (Gamma), Rice, chi  cuadrado no
central?

Everett ``The  Cambridge Dictionary of Statistics'', Cambridge  Univ Press, 2006
(3rd Ed.); Hazeinke Michel Ed. (2001) ``Probability Distributions'', Springer}
%
%(en) J.  C. Maxwell,  « Illustrations  of the dynamical  theory of  gases. Part
%I. On  the motions and collisions  of perfectly elastic spheres  », The London,
%Edinburgh, and Dublin Philosophical Magazine  and Journal of Science, 4e série,
%vol. 19,‎  1860, p. 19-32  (lire en  ligne [archive]) ↑  (en) J. C.  Maxwell, «
%Illustrations of  the dynamical  theory of  gases. Part II.  On the  process of
%diffusion of  two or more  kinds of moving  particles among one another  », The
%London, Edinburgh, and Dublin Philosophical Magazine and Journal of Science, 4e
%série,  vol. 20,‎  1860,  p. 21-37  (lire  en ligne  [archive])  ↑ (de)  Ludwig
%Boltzmann, « Weitere studien  über das Wärmegleichgewicht unter Gasmolekülen »,
%Sitzungsberichte  der   Kaiserlichen  Akademie  der   Wissenschaften  in  Wien,
%mathematisch-naturwissenschaftliche Classe, vol. 66,‎ 1872, p. 275–370 (lire en
%ligne [archive])  ↑ (de)  Ludwig Boltzmann, «  Über die Beziehung  zwischen dem
%zweiten     Hauptsatz     der     mechanischen     Wärmetheorie     und     der
%Wahrscheinlichkeitsrechnung respektive  den Sätzen über  das Wärmegleichgewicht
%»,  Sitzungsberichte  der Kaiserlichen  Akademie  der  Wissenschaften in  Wien,
%Mathematisch-Naturwissenschaftliche Classe, vol. 76,‎ 1877, p. 373–435 (lire en
%ligne [archive])
%%%%%%%%%%%%%%%%%%%%%%%%%%%%%%%%%%%%%%%%%%%%%%%%%%%%%%%%%%%%%%%%%%%%%%%%%%%%%%%%


% ================================= Variables continuas

\subseccion{Distribuciones de variable continua}
\label{Ssec:MP:EjemplosDistribucionescontinuas}

Antes de  ir m\'as adelante,  notamos que, tratando  de un vector  aleatorio $X$
continuo, de densidad  de probabilidad $p_X(x)$, para cualquier  $a \in \Rset^*$
el vector $Y_a = a X$ va  a ser obviamente continuo, de densidad de probabilidad
$p_{Y_a}(y) = \frac{1}{|a|} p_X\left( \frac{y}{a} \right)$. Ahora, cuando $a \to
0$, queda claro  que el vector $Y_a$ tiende al vector  \ $0$, deterministico. O,
de  un punto  de vista  de variable  aleatoria, $Y_a$  tiende al  vector cierto,
discreto. Un punto que puede parecer  sopredente es que tal vector no admite una
densidad  de probabilidad m\'as.  De hecho  la densidad  $p_{Y_a}$ tiende  a una
funci\'on  generalizada,  o distribuci\'on  de  Schwarz,  m\'as precisamente  la
``funci\'on   Dirac'',    como   lo    hemos   visto   en    el   fin    de   la
secci\'on~\ref{Ssec:MP:VAContinua}.  Como  lo  hemos  enfatizado,  en  tal  caso
preferimos  seguir trabajando con  la medida  de probabilidad,  que tiende  a la
medida de Dirac.


% --------------------------------- uniforme escalar
%\subsubseccion{Distribuci\'on uniforme sobre un intervalo}
\label{Sssec:MP:UniformeContinua}

Esta distribuci\'on es  la m\'as natural que se usa  cuando queremos modelar una
falta de  informaci\'on sobre una variable,  sabiendo que vive en  un espacio de
volumen  finito: sin  a  priori  m\'as, una  tendencia  natural/intuitiva es  de
asignar la  ``misma probabilidad''  a cada punto  del conjunto.   En particular,
aparece as\'i  naturalmente en  la inferencia bayesiana  que consiste  a modelar
como aleatorio un par\'ametro  que se quierre inferir~\footnote{En la inferencia
  bayesiana, nos interesamos al paremetro (posiblemente multivariado) \ $\theta$
  \ subyancente  a una distribuci\'on. Por ejemplo,  sabemos tener observaciones
  sorteados  de  una  distribuci\'on  de  Poisson, pero  con  el  par\'ametro  \
  $\lambda$ \  desconocido y  nos interesamos a  \ $\theta \equiv  \lambda$.  El
  enfoque bayesiano consiste a considerar el par\'ametro \ $\Theta$ \ aleatorio,
  tal que la  distribuci\'on de las observaciones sea  vista como distribuci\'on
  condicional \  $p_{X|\Theta = \theta}(x)$, llamada  distribuci\'on de sampleo.
  Dadas las  observaciones $X = x$,  la meta es de  determinar la distribuci\'on
  dicha a posteriori \  $p_{\Theta|X = x}$ \ a partir de  la cual se puede hacer
  estimaci\'on  de \ $\theta$  dadas las  observaciones, calcular  intervalos de
  confianza, etc. Se interpreta  como distribuci\'on explicando el par\'ametro a
  partir de  las observaciones.  Por eso, el  metodo se apoya sobre  la regla de
  Bayes $p_{\Theta|X=x}(\theta) \propto p_{X|\Theta=\theta}(x) p_\Theta(\theta)$
  \  as\'i que  se necesita  elegir una  distribuci\'on \  $p_\Theta$ \  dicha a
  priori.\label{Foot:MP:BayesPrior} }
%~\footnote{A  partir de una
%  distribuci\'on  parametrizada   por  un  par\'ametro   $\theta$.   El  enfoque
%  bayesiano consiste  a modelizar $\theta$ aleatorio, digamos  $\Theta$, tal que
%  la     distribuci\'on     de     observaciones     se     escribe     entonces
%  $p_{X|\Theta=\theta}(x)$.   Inferir  $\theta$ a  partir  de observaciones  $x$
%  consiste   a   determinar  la   distribuci\'on   dicha   {\it  a   posteriori}
%  $p_{\Theta|X=x}(\theta)$.  Por eso, hace  falta darse una distribuci\'on dicha
%  {\it a priori} $p_\Theta(\theta)$.\label{Foot:SZ:BayesPrior}}
~\cite{Rob07} (la
ley  es dicha  ley  {\em a  priori};  ver tambi\'en~\cite{Bay63}  o~\cite{Lap12,
  Lap14, Lap20}; tal a priori es conocido como a priori de Laplace).

Se denota $X \, \sim \, \U([a \; b])$. Las caracter\'isticas de \ $X$ \ son las
siguientes:

\begin{caracteristicas}
%
Dominio de definici\'on & $\X = [a \; b]$\\[2mm]
\hline
%
Par\'ametros & $(a,b) \in \Rset, \: b > a$\\[2mm]
\hline
%
Densidad de probabilidad & $p_X(x) = \frac{1}{b-a}$\\[2mm]
\hline
%
Promedio & $\displaystyle m_X = \frac{a+b}{2}$\\[2mm]
\hline
%
Varianza & $\displaystyle \sigma_X^2 = \frac{(b-a)^2}{12}$\\[2mm]
\hline
%
\modif{Asimetr\'ia} & $\gamma_X = 0$ \quad para \ $b \ne a$\\[2mm]
\hline
%
Curtosis por exceso & $\displaystyle \widebar{\kappa}_X = -\frac65$ \quad para \ $b \ne a$\\[2mm]
\hline
%
Generadora de momentos & $\displaystyle M_X(u) = \frac{ e^{b u} - e^{a u}}{u}$ \quad
para~\footnote{En el caso l\'imite \ $u \to  0$, \ $\lim_{u \to 0} \frac{ e^{b u}
- e^{a u}}{u} = b-a$, y similarmente para la funci\'on caracter\'istica}  \ $u \in \Cset$\\[2mm]
\hline
%
Funci\'on caracter\'istica & $\displaystyle  \Phi_X(\omega) = \frac{ e^{\imath a
\omega} - e^{\imath b \omega}}{\imath \, \omega}$
\end{caracteristicas}

% Momentos & $ \Esp\left[ X^k \right] = p^k$\\[2mm]
% Momento factorial & $\Esp\left[ (X)_k \right] = ?$\\[2mm]
% Generadora de probabilidad & $G_X(z) = e^{\lambda (z-1)}$ \ para \ $z \in \Cset$\\[2mm]
% modo 0
% Mediana \ln(2)/\lambda
% CDF 1-e^{-\lambda x}

Obviamente, se puede escribir \ $X \, \egald  \, a + (b-a) U$ \ donde \ $\egald$
\ significa que la equalidad es en distribuci\'on (las variables tienen la misma
distribuci\'on de probabilidad), con \ \ $U \, \sim \, \U \left( [ 0 \; 1 ]
\right)$ \ llamada {\em uniforme estandar}.

La densidad de probabilidad y funci\'on de repartici\'on de la variable estandar
son representadas en la figura Fig.~\ref{Fig:MP:Uniformecontinua}.
%
\begin{figure}[h!]
\begin{center} \begin{tikzpicture}%[scale=.9]
\shorthandoff{>}
%
\pgfmathsetmacro{\sx}{2.5};% x-scaling
\pgfmathsetmacro{\r}{.05};% radius arc non continuity F_X
%
% densidad
\begin{scope}
%
\pgfmathsetmacro{\sy}{2.5};% y-scaling 
\draw[>=stealth,->] ({-\sx/2-.25},0)--({\sx*1.5+.25},0) node[right]{\small $x$};
\draw[>=stealth,->] (0,-.15)--(0,{\sy+.25}) node[above]{\small $p_X$};
%
\draw[thick] ({-\sx/2},0)--(0,0);
\draw (\r,\r) arc (90:270:\r);
\draw[dotted] (0,0)--(0,\sy) node[scale=.4]{$\bullet$};
\draw[thick] plot (0,\sy)--(\sx,\sy) node[scale=.4]{$\bullet$};
\draw[dotted] (\sx,\sy)--(\sx,0);
\draw ({\sx-\r},{-\r}) arc (-90:90:\r);
\draw[thick] (\sx,0)--({1.5*\sx},0);
%
\draw (0,\sy)--(-.1,\sy) node[left,scale=.7]{$1$};
\draw (0,0)--(0,-.1) node[below,scale=.7]{$0$};
\draw (\sx,0)--(\sx,-.1) node[below,scale=.7]{$1$};
%
\node at ({(\sx*1.5+.5)/2},-1) [scale=.9]{(a)};

\end{scope}
%
%
% reparticion
\begin{scope}[xshift=8.5cm]
%
\pgfmathsetmacro{\sy}{2.5};% y-scaling 
%
\draw[>=stealth,->] ({-\sx/2-.25},0)--({\sx*1.5+.25},0) node[right]{\small $x$};
\draw[>=stealth,->] (0,-.1)--(0,{\sy+.25}) node[above]{\small $F_X$};
%
% cumulativa
\draw[thick] ({-\sx/2},0)--(0,0)--(\sx,\sy)--({1.5*\sx},\sy);
%
\draw (0,\sy)--(-.1,\sy) node[left,scale=.7]{$1$};
\draw (0,0)--(0,-.1) node[below,scale=.7]{$0$};
\draw (\sx,0)--(\sx,-.1) node[below,scale=.7]{$1$};
%
\node at ({(\sx*1.5+.5)/2},-1) [scale=.9]{(b)};
\end{scope}
%
\end{tikzpicture} \end{center}
% 
\leyenda{Ilustraci\'on  de  una densidad  de  probabilidad  uniforme  (a), y  la
funci\'on de repartici\'on asociada (b).}
\label{Fig:MP:Uniformecontinua}
\end{figure}

Una nota  importante es que cada ley  continua es v\'inculada a  la ley uniforme
sobre $(0 \; 1)$ de la manera siguiente:
%
\begin{lema}[Inversi\'on]\label{Lem:MP:InversionUniforme}
Sea $X$, continua sobre $\X \subset \Rset$, de funci\'on de repartici\'on $F_X$. Entonces
%
\[
U \equiv F_X(X) \sim \U(0 \; 1)
\]
%
Reciprocamente, definiendo la funci\'on de repartici\'on inversa (o quantile)
%
\[
F_X^{-1}(u) = \inf \{ x \tq F(x) \ge u \}
\]
%
si \ $V \sim \U( 0 \; 1 )$,
%
\[
Y = F_X^{-1}(V) \quad \Rightarrow \quad F_Y(y) = F_X(y)
\]
\end{lema}
%
Cuando  $F_X$ se  inversa  sencillamente eso  da  una manera  sencilla de  tirar
sampleos de funci\'on de repartici\'on $F_X$ a partir de sampleos tirados seg\'un
una ley uniforme.
%
\begin{proof}
Inmediatamente, $F_X$ siendo creciente,
%
\begin{eqnarray*}
P(U \le u) & = &  P( F_X(X) \le u)\\[2mm]
%
& = & P(X \le F_X^{-1}(u))\\[2mm]
%
& = & F_X\left( F_X^{-1} (u) \right)
\end{eqnarray*}
%
Similarmente
%
\begin{eqnarray*}
P(Y \le y) & = &  P( F_X^{-1}(V) \le y)\\[2mm]
%
& = & P(V \le F_X(y))\\[2mm]
%
& = & F_X(y)
\end{eqnarray*}
%
\end{proof}

De manera  general, para  cualquier ensemble $\D  \varsubsetneq \Rset^d$ de  volumen \
finito $|\D|$ \,  la variable uniforma sobre $\D$  tiene la densidad con  respecto a la
medida  ``natural'' sobre  $\D$  (Lebesque, discreta,\ldots)  constante sobre  \
$\D$,
%
\[
p_X(x) = \frac{1}{|\D|} \un_{\D}(x)
\]
%
La media va a ser el centro de gravedad de $\D$.

Vamos a ver en el cap\'itulo~\ref{Cap:SZ:Informacion} que esta distribuci\'on es
la distribuci\'on definida  sobre un conjunto de volumen  finito que maximiza la
entrop\'ia, \ie  que es la  ``menos informativa''. Por  ejemplo, si se  busca un
par\'ametro  modelizado como  aleatorio (enfoque  bayesiano), definido  sobre un
conjunto de volumen finito, sin  a priori m\'as, una tendencia natural/intuitiva
es de  asignar la ``misma probabilidad'' a  cada punto del conjunto.  Es por eso
que aparece as\'i naturalmente en la inferencia bayesiana~\cite{Rob07}.
 
Notar que  cuando $b \to a$,  la variable tiende a  una variable cierta  $X = a$
(ver principio de esta secci\'on).



% --------------------------------- Gaussiana
%\subsubseccion{Distribuci\'on normal o gaussiana multivariada real}
\label{Sssec:MP:Gaussiana}

En el caso escalar,  esta ley parece aparecer por unas de  las primeras veces en
trabajos de de  Moivre como approximaci\'on de la ley  binomial para $n$ grande,
usando la formula de Stirling~\cite{Moi30, Moi33, Moi56, Pea24, PeaMoi26, Dem33,
  Hal84,  Hal90,  JohKot95:v1, DavEdw01,  Hal06}.   Se  puede  ver tambi\'en  el
trabajo  de F.   Galton,  quien construy\'o  un  experimento, la  caja dicha  de
Galton, que  ilustra por una  parte como se  puede obtener la ley  binomial como
suma   de  Bernoulli,   y   la  convergencia   a  la   Gausiana~\cite[Figs.~7-9,
p.~63]{Gal89}  o~\cite[p.~38]{Pea20}.  Aparte  de  Moivre, la  ley gausiana  fue
desarollado mucho por los matem\'aticos  como Gauss en el estudio del movimiento
de   planetas   con   perturbaciones   (predicci\'on  de   la   trayectoria   de
C\'eres)~\cite{Gau09,  Pea24, DavEdw01,  Hal06}, basado  en trabajos  de  A.  M.
Legendre~\cite{Leg05,   DavEdw01,  Hal06},   o  Laplace   en  mismos   tipos  de
problemas~\cite{Lap09, Lap09:Supp, Lap12, Lap14, Lap20, Pea24, DavEdw01, Hal06}.
De hecho,  apoyandose en  trabajos de  de Moivre, la  formaliz\'o antes  y m\'as
claramente  Laplace,   quien  revandic\'o   entonces  su  partenidad   (ver  por
ejemplo~\cite{Pea20}).   Por eso,  esta ley  es tambi\'en  conocida como  ley de
Laplace-Gauss.

En el contexto multivariado, la extensi\'on natural de la ley binomial siendo la
ley multinomial, es sin sorpresa  que se introdujo la gausiana multivaluada como
approximaci\'on de la multinomial.  Este trabajo  es debido entre otros a J.  L.
Lagrange en  los a\~nos 1770, con  correcciones debido unas  decadas despu\'es a
A. de  Morgan~\cite{Mor38}. Pero apareci\'o  antes en el caso  bidimensional, en
particular  a  trav\'es  del  estudio  del coeficiente  de  correlaci\'on  entre
variables   aleatorias  (ver   por  ejemplo   trabajos   de  Galton~\cite{Gal77,
  Gal77:Nature, Pea20}).

A pesar de que parece menos natural en la modelisaci\'on de fenomenos aleatorios
que leyes uniformes, la ley gausiana es seguramente unas de las m\'as importante
en  probabilidad, sino  que  la m\'as  importante  y la  m\'as  expendida en  la
naturaleza.   Eso  viene sin  duda  del teorema  del  l\'imite  central. En  dos
palabras,  cuando  se  suman  un  n\'umero importante  de  variables  aleatorias
(independientes,   de  misma  ley,   admitiendo  una   varianza,  o   con  menos
restricciones~\cite[Cap.~11]{AthLah06}),  correctamente  normalizado, esta  suma
tiende a una gausiana~\footnote{De hecho,  la approximaci\'on de la ley binomial
  por una  gausiana cuando  $n$ es  grande es una  caso particular  del teorema,
  siendo la binomial una suma  de Bernoulli independientes.}.  En la naturaleza,
se puede ver el ruido (se\~nales) como suma de un n\'umero importante de fuentes
de  ruido independientes,  justificando el  modelo  gausiano~\cite{Fel71, Cam86,
  AshDol99, JacPro03, AthLah06, Ren07, Bil12}.  Ad\'emas, como lo vamos a ver en
el  cap\'itulo~\ref{Cap:SZ:Informacion}, esta  ley es  la de  incerteza m\'axima
(maximizando la entrop\'ia) teniendo  una dada varianza. Aparece naturalmente en
termod\'inamica    (gaz    perfecto,   con    un    n\'umero    muy   alto    de
particulas)~\cite{Max67, Bol96,  Bol98, Gib02, Jay65}. En  estimaci\'on, bajo la
hipotesis  gausiana,  los  estimadores  de  par\'ametros  minimizando  el  error
cuadr\'atico promedio son  generalmente lineal~\cite{Kay93, Rob07}.  Todas estas
consideraciones  dan a la  ley gausiana  un rol  central en  la teor\'ia  de las
probabilidades.

Se denota \  $X \, \sim \, \N(m,\Sigma)$ \  con \ $m \in \Rset^d$  \ y \ $\Sigma
\in  P_d^+(\Rset)$  \  conjunto  de   las  matrices  de  \  $\M_{d,d}(\Rset)$  \
s\'imetricas definidas positivas. Las  caracter\'isticas de la gaussiana son las
siguientes:

\begin{caracteristicas}
%
Dominio de definici\'on & $\X = \Rset^d$\\[2mm]
\hline
%
Par\'ametros & $m \in \Rset^d, \:\: \Sigma \in P_d^+(\Rset)$\\[2mm]
\hline
%
Densidad de probabilidad & $\displaystyle p_X(x) = \frac{1}{(2
\pi)^{\frac{d}{2}} \left| \Sigma \right|^{\frac12}} \, e^{-\frac12 (x-m)^t
\Sigma^{-1} (x-m)}$\\[2.5mm]
\hline
%
Promedio & $ m_X = m$\\[2mm]
\hline
%
Covarianza & $\Sigma_X = \Sigma$\\[2mm]
\hline
%
\modif{Asimetr\'ia} & $\gamma_X = 0$\\[2mm]
\hline
%
Curtosis por exceso & $\widebar{\kappa}_X = 0$\\[2mm]
\hline
%
Generadora de momentos & $\displaystyle M_X(u) = e^{u^t \Sigma u + u^t m}$ \
para \ $u \in \Cset^d$\\[2mm]
\hline
%
Funci\'on  caracter\'istica   &  $\displaystyle  \Phi_X(\omega)   =  e^{-\frac12
\omega^t \Sigma \omega + \imath \omega^t m}$
\end{caracteristicas}

Nota: trivialmente, se puede escribir $X  \, \egald \, \Sigma^{\frac12} N + m$ \
con \ $N \, \sim \, \N(0,I)$ \  donde \ $N$ \ es dicha {\em Gausiana estandar} o
{\em centrada-normalizada}. Las caracter\'isticas de  \ $X$ \ son v\'inculadas a
las  de  \  $N$ \  (y  vice-versa)  por  transformaci\'on afine  (ver  secciones
anteriores).


La densidad de probabilidad gausiana y  la funci\'on de repartici\'on en el caso
escalar son  representadas en la figura Fig.~\ref{Fig:MP:Gaussiana}-(a)  y (b) y
una      densidad      en       un      contexto      bi-dimensional      figura
Fig.~\ref{Fig:MP:Gaussiana}(c).
%
\begin{figure}[h!]
\begin{center} \begin{tikzpicture}%[scale=.9]
\shorthandoff{>}
%
\pgfmathsetmacro{\sx}{.75};% x-scaling
\pgfmathsetmacro{\mx}{3.5};% x maximo del plot
%
% Approximation de la cdf gaussienne
\tikzset{declare function={
normcdf(\x)=1/(1 + exp(-0.07056*(\x)^3 - 1.5976*(\x)));
}}
% densidad
\begin{scope}
%
\pgfmathsetmacro{\sy}{2.5*sqrt(2*pi)};% y-scaling 
\draw[>=stealth,->] ({-\sx*\mx-.25},0)--({\sx*\mx+.25},0) node[right]{\small $x$};
\draw[>=stealth,->] (0,-.1)--(0,2.75) node[above]{\small $p_X$};
%
\draw[thick,domain=-\mx:\mx,samples=100] plot ({\x*\sx},{\sy*exp(-.5*\x*\x)/sqrt(2*pi)});
%
\draw (0,{\sy/sqrt(2*pi)})--(-.1,{\sy/sqrt(2*pi)}) node[left,scale=.7]{$\frac1{\sqrt{2 \pi}}$};
%
\end{scope}
%
%
% reparticion
\begin{scope}[xshift=8.5cm]
%
\pgfmathsetmacro{\sy}{2.5};% y-scaling 
%
\draw[>=stealth,->] ({-\sx*\mx-.25},0)--({\sx*\mx+.25},0) node[right]{\small $x$};
\draw[>=stealth,->] (0,-.1)--(0,{\sy+.25}) node[above]{\small $F_X$};
%
% cumulativa
\draw[thick,domain=-\mx:\mx,samples=100] plot({\x*\sx},{\sy*normcdf(\x)});
%
\draw (0,\sy)--(-.1,\sy) node[left,scale=.7]{$1$};
\end{scope}
%
\end{tikzpicture} \end{center}
% 
\leyenda{Ilustraci\'on  de  una   densidad  de  probabilidad  gaussiana  escalar
  estandar  (a), y la  funci\'on de  repartici\'on asociada  (b), as\'i  que una
  densidad  de probabilidad  gaussiana bi-dimensional  centrada y  de  matriz de
  covarianza \ $\Sigma_X = R(\theta)  \Delta^2 R(\theta)^t$ \ con \ $R(\theta) =
  \protect\begin{bmatrix}   \cos\theta  &   -  \sin\theta\\[2mm]   \sin\theta  &
    \cos\theta  \protect\end{bmatrix}$ \  matriz  de rotaci\'on  y  \ $\Delta  =
  \diag\left(\protect\begin{bmatrix}  1   &  a\protect\end{bmatrix}  \right)$  \
  matriz  de   cambio  de  escala,   y  sus  marginales   \  $X_1  \,   \sim  \,
  \N\left(0,\cos^2\theta  + a^2  \sin^2\theta \right)$  \ y  \ $X_2  \,  \sim \,
  \N\left(0,\sin^2\theta + a^2 \cos^2\theta  \right)$ \ (ver m\'as adelante). En
  la figura, $a = \frac14$ \ y \ $\theta = \frac{\pi}{6}$.}
\label{Fig:MP:Gaussiana}
\end{figure}

La gaussiana tiene un par de propiedades particulares:
%
\begin{lema}[Gausiana y cumulantes]
%
  Sea \  $X$ \ vector aleatorio de  media $m$, covarianza $\Sigma$  y de secunda
  funci\'on caracter\'istica admtiendo un desarollo de Taylor. Entonces
  %
  \[
  \kappa_k[X] =  0 \quad \forall \: k  \ge 4 \quad \Longleftrightarrow  \quad X \sim
  \N(m,\Sigma)
  \]
\end{lema}
%
\begin{proof}
  La pueba es inmediata del lema~\ref{Lem:MP:CumSecFctCarac},
  %
  \[
  \kappa_k[X]  =  0  \quad  \forall  \:  k \ge  4  \quad  \Longleftrightarrow  \quad
  \Psi_X(\omega) = \imath \, \omega^t m - \frac12 \omega^t \Sigma \omega
  \]
  %
  lo que es nada m\'as que la secunda funci\'on caracter\'istica de la gausiana,
  esa determiniendo completamente la ley.
\end{proof}
%
\begin{teorema}[Stabilidad]
\label{Teo:MP:StabilidadGaussiana}
%
  Sean \ $A_i , i = 1,\ldots,n$  \ matrices de \ $\M_{d',d}(\Rset), \: d' \le d$
  \ de rango lleno, $b_i \in \Rset^{d'}$ \ y \ $X_i \, \sim \, \N(m_i,\Sigma_i)$
  \ independientes, entonces
  %
  \[
  \sum_{i=1}^n \left(  A_i X_i  + b_i \right)  \, \sim \,  \N\left( \sum_{i=1}^n
    \left( m_i + b_i \right) \, , \, \sum_{i=1}^n A_i \Sigma_i A_i^t \right)
  \]
  % 
  En particular, cualquier combinaci\'on lineal  de los componentes de un vector
  gaussiano da una gaussiana.  Reciprocamente, si cualquier combinaci\'on lineal
  de los componentes de un vector aleatorio sigue una ley gaussiana, entonces el
  vector es gaussiano.
\end{teorema}
%
\begin{proof}
  Este  resultato se proba  usando funci\'on  caracter\'istica de  la gaussiana,
  conjuntamente al teorema~\ref{Teo:MP:PropiedadesFuncionCaracteristica}.
\end{proof}
%
\begin{corolario}[Media empirica]\label{Cor:MP:MediaEmpiricaGauss}
%
  Sean \ $X_i \, \sim \, \N(m,\Sigma), \: i = 1, \ldots , n$ \ independientes. Entonces,
  %
  \[
  \overline{X} =  \frac{1}{n} \sum_{i=1}^n  X_i \,  \sim \, \N\left(  m \,  , \,
    \frac{1}{n} \Sigma \right)
  \]
   %
  $\overline{X}$  es llamada {\em  media empirica}~\footnote{Es  la estimaci\'on
    \'optima de  la media  $m$ a  partir de los  $X_i$ en  el sentido  del error
    cuadratico  promedio   m\'inimo,  o  en   el  sentido  de   la  verosimlitud
    m\'axima~\cite{Kay93, Rob07}.}, y es un estimador ``natural'' de la media de
  un vector aleatorio a partir de copias independientes de misma ley.
  %
\end{corolario}
%
\begin{teorema}[Independencia]
\label{Teo:MP:IndependenciaGaussiana}
%
  Sea   \   $X  \,   \sim   \,   \N(m,\Delta)$  \   con   \   $\Delta  =   \diag
  \left(  \begin{bmatrix}  \sigma_1^2  &  \cdots  &  \sigma_d^2  \end{bmatrix}^t
  \right)$   \  diagonal.   Entonces  los   componentes  \   $X_i  \,   \sim  \,
  \N(m_i,\sigma_i^2)$ \ son independientes.
\end{teorema}
%
\begin{proof}
  Este resultato se proba  trivialmente escribiendo la densidad de probabilidad,
  notando que se factorisa.
\end{proof}
%
Hemos visto que cuando un  vector tiene componentes independientes, la matriz de
covarianza  es   diagonal  (lema~\ref{Lem:MP:IndependenciaCov}),  pero   que  la
rec\'iproca es falsa en general.  El \'ultimo teorema muestra que la rec\'iproca
vale en el caso gausiano.

Volvemos  ahora al rol  central de  la gausiana  como modelo  probabilistico muy
frecuente  de fenomeos  aleatorios. Este  rol particular  viene del  teorema del
l\'imite  central que  ya introdujimos.  A veces,  es conocido  como  teorema de
Lindenberg-Feller (por lo menos la forma con condiciones m\'as debiles que en la
formulaci\'on original).   Para unas de  las formulaciones originales,  se puede
referirse  al trabajo  de Laplace  de  1809 o  de 1912~\cite{Lap09,  Lap09:Supp,
  Lap12,  Lap14, Lap20}.   El nombre  ``central'' viene  de un  documento  de G.
P\'olya   de   1920,  titulado   ``\"Uber   den   zentralen  Grenzwertsatz   der
Wahrscheinlichkeitsrechnung  und das Momentenproblem''  (``Sobre el  teorema del
l\'imite central del  c\'alculo probabil\'istico y el problema  de los momentos;
el  teorema  es  central\ldots~\cite{Pol20,   Cam86}).   Se  enuncia  de  manera
siguiente~\cite{Spi76, BroDav87, LehCas98, AshDol99, JacPro03, AthLah06, Bil12}:
% ~\footnote{Aparte    en~\cite{AshDol99,   JacPro03},   el    teorema   aparece
%   frecuentemente en los libros en el contexto escalar, seguramente por razones
%   historicas.    Pero,  con   el  mismo   enfoque,  se   prueba  en   el  caso
%   multivariado.}:

\begin{teorema}[Teorema del l\'imite central]\label{Teo:MP:CLT}
%
  Sea  \  $\{  X_i \}_{i  \in  \Nset^*}$  \  una  sucesi\'on de  vectores  aleatorios
  independientes, de misma ley,  y que admiten un promedio \ $m$  \ y una matriz
  de covarianza \ $\Sigma$. Entonces
  %
  \[
  \frac{1}{\sqrt{n}}  \sum_{i=1}^n  \left( X_i  -  m  \right)  \: \limitd{n  \to
    +\infty} \: Y \sim \N\left( 0 , \Sigma \right)
  \]
  %
  donde  \ $\limitd{}$ \  significa que  el l\'imite  es en  distribuci\'on (ver
  notaciones).
\end{teorema}
\begin{proof}
  Hay varias pruebas de este resultado.  Quiz\'as la m\'as simple se apoya sobre
  la funci\'on  c\'aracteristica.  Sin perdida de generalidad,  supongamos que \
  $m = 0$. Sea \ $\displaystyle Y_n = \frac{1}{\sqrt{n}} \sum_{i=1}^n X_i$.  Sea
  \    $\omega$    \    fijo.     Por    independencia    y    relaciones    del
  teorema~\ref{Teo:MP:PropiedadesFuncionCaracteristica}~\footnote{$o\left(
      n^{-1} \right)$ significa que  el termino que queda, digamos $\varepsilon$
    es tal que $n \varepsilon$ tiende a cero cuando $n$ tiende al infinito.}:
  %
  \begin{eqnarray*}
  \Phi_{Y_n}(\omega) & = & \left( \Phi_{X_i}\left( \frac{\omega}{\sqrt{n}}
  \right) \right)^n\\[2mm]
  %
& = & \left( \Phi_{X_i}(0) + \frac{1}{\sqrt{n}} \, \omega^t \, \nabla_\omega
  \Phi_{X_i}(0) + \frac{1}{2 n} \, \omega^t \, \Hess_\omega\Phi_{X_i}(0) \, \omega +
  o\left( n^{-1} \right) \right)^n\\[2mm]
  %
  & = & \left( 1 - \frac{1}{2 n} \, \omega^t \, \Sigma \, \omega +
  o\left( n^{-1} \right) \right)^n\\[2mm]
  %
  & \xrightarrow[n \to +\infty]{} & \exp\left( -\frac12 \omega^t \Sigma \omega \right)
  \end{eqnarray*}
  %
  porque \ $\Phi_{X_i}(0) = 1$, $X_i$ \  siendo de media nula el gradiente de la
  funci\'on   caracter\'istica   se  cancela   en   \   $\omega   =  0$,   y   \
  $\Hess_\omega\Phi_{X_i}(0)  =  -  \Sigma$.   Se reconoce  ahora  la  funci\'on
  caracter\'istica   de  la   gausiana,   lo  que   prueba   que  la   funci\'on
  caracter\'istica  de  \  $Y_n$  \  converge  simplemente  hacia  la  funci\'on
  caracter\'istica  de la gausiana.   Se cierra  la pruba  usando el  teorema de
  convergencia de  L\'evy, diciendo que  la convergencia simple de  la funci\'on
  caracter\'istica  implica  la  convergencia en  distribuci\'on~\cite{AshDol99,
    Bil12, AthLah06}.
\end{proof}
%
En particular, la  media empirica hechas a partir  de vectores independientes de
media  $m$,  admitiendo  una  covarianza  \  $\Sigma$  \  y  de  misma  ley  (no
necesariamente gausiana), tiende a ser gausiana  de media \ $m$ \ y covarianza \
$\frac{1}{n} \Sigma$.

Existen  varias   variantes  de  este   teorema  que  enunciamos,  sin   dar  la
prueba.  Dejamos el  lector a  libros m\'as  especializados como~\cite{AshDol99,
  Bil12, AthLah06, Lin22}.
% Lindeberg 1920

\begin{teorema}[Teorema de Lindenberg-Feller]\label{Teo:MP:LindenbergFeller}
%
  Sean  $\{  X_i \}_{i  \in  \Nset^*}$  vectores  aleatorios independientes,  no
  necesariamente de misma distribuci\'on de probabilidad, con \ $X_i$ \ de media
  \ $m_i = \Esp[X_i]$ \ y de matriz de covarianza \ $\Sigma_i \in P_d^+(\Rset)$.
  Sean \ $C_n = \sum_{i=1}^n \Sigma_i$, \ $c_n^2$ \ al autovalor m\'as peque\~na
  de $C_n$,  \ y  \ $Y_n  = C_n^{-\frac12} \sum_{i=1}^n  \left( X_i  - \Esp[X_i]
  \right)$.
  %
  \[
  \mbox{Si} \quad \lambda_n > 0 \quad \mbox{y} \quad \forall \: \varepsilon > 0,
  \quad  \lim_{n  \to  +\infty}  \sum_{i=1}^n  \Esp\left[  \left\|  \frac{X_i  -
        m_i}{c_n} \right\|^2  \un_{\left[ \varepsilon \;  +\infty \right)}\left(
      \left\| \frac{X_i - m_i}{c_n} \right\| \right) \right] = 0
  \]
  %
  entonces
  %
  \[
  Y_n \: \limitd{n \to +\infty} \: Y \sim \N\left( 0 , I \right)
  \]
\end{teorema}
%
En  numerosos libros,  este teorema  es  dado en  el caso  escalar. Se  extiende
sencillamente al caso multivariado gracia a lo que es conocido como {\it teorema
  de Cram\'er-Wold},  diciendo que una  secuencia de vectores aleatorios  \ $Y_n
\limitd{}  Y$ \  si y  solamente para  cualquier \  $u \in  \Rset^d$ \  $u^t Y_n
\limitd{} u^t Y$~\cite{AshDol99, AthLah06, Bil12}.

Sin dar  la prueba,  la condici\'on  de Lindenberg dice  que si  la suma  de las
``dispersi\'ones'' de  los vectores normalizados  por los que es  basicamente la
varianza  la   m\'as  peque\~na  de  los   componentes  de  la   suma  (una  vez
diagonalizada)  se  concentra  asintoticamente,  la  suma  renormalizada  de  lo
vectores centrados tiende a la gausiana (en distribuci\'on).

Se  puede ver  que  se  satisface la  condici\'on  de Lindeberg  en  el caso  de
variables independientes de misma ley, del hecho  que \ $C_n = n \Sigma$, lo que
da \ $c_n^2 = n c^2$ \ con  \ $c^2$ \ autovalor m\'as peque\~na de \ $\Sigma$. A
continuaci\'on da la condici\'on \ $\displaystyle \lim_{n \to \infty} \Esp\left[
  \left\| X_i - m_i \right\|^2 \un_{\left[ \varepsilon \; +\infty \right)}\left(
    \left\|  \frac{X_i  -  m_i}{\sqrt{n}  c}  \right\|  \right)  \right]  =  0$,
satisfecha  porque el  argumento de  la funci\'on  indicadora tiende  a  0 (casi
siempre).

Un otro caso  ``trivial'' aparece cuando la secuencia  es uniformamente acotada,
\ie \  $\forall \: i, \quad  \| X_i \| \le  M$. Se puede  retomar los argumentos
anteriores, remplazando las variables por la cota.

Nota: si  se satisface  la condici\'on  dicha {\it de  Lyapunov}, \ie  si existe
$\delta > 0$ tal que
%
\[
\lim_{n   \to  \infty}   \sum_{i=1}^n   \Esp\left[  \frac{\left\|   X_i  -   m_i
    \right\|^2}{c_n^{2+\delta}} \right] = 0,
\]
%
entonces         se          satisface         la         condici\'on         de
Lindeberg~\cite{AshDol99}.  Frecuentemente,  es   m\'as  sencillo  verificar  la
condici\'on m\'as fuerte de Lyapunov para  probar la convergencia de una suma de
vectores aleatorios a la gausiana.

\

Aparece que se puede a\'un  debilitar la condici\'on de independencia sin perder
la   convergencia    a   la   gausiana.    Para   m\'as   detalles,    ver   por
ejemplo~\cite[Sec.~6.4]{BroDav87}.



% --------------------------------- Gaussiana complejas
%\subsubseccion{Distribuci\'on normal o gaussiana multivariada complejas}
\label{Sssec:MP:GaussianaComplejas}

Por definici\'on, un vector aleatorio complejo $d$-dimensional \ $Z = X + \imath
Y$ \  es gaussiano significa que  el vector $2 d$-dimensional  \ $\widetilde{Z} =
\begin{bmatrix}  X^t &  Y^t \end{bmatrix}^t$  \ es  gaussiano. Se  puede entonces
referirse  en el  caso de  vectores gaussianos,  pero como  lo hemos presentado  en la
secci\'on~\ref{Ssec:MP:VAComplejos}, es frecuentemente m\'as comodo trabajar con
\  $Z$ \  en lugar  de \  $\widetilde{Z}$.
%  En  particular, en  el marco  de las
%comunicaciones en  ingeneria, se  trabaja con modulaciones  dichas en fase  y en
%cuadratura (se\~nal multiplicado  respectivamente por un seno y  un coseno) y en
%lugar  de trabajar  con dos  componente se  considera una  modulaci\'on  con una
%exponencial  compleja y  la  se\~nal/variable compleja.   Se  puede por  ejemplo
%referirse  a~\cite{Lap17} (ver en  particular el  capitulo~24) o~\cite{SchSch03,
%  EriKoi06}.

En el caso general, la gaussiana  real siendo completamente descrita por su media
y su matriz de covarianza, la  gaussiana compleja va a ser completamente definida
por   la  media,   la  matriz   de  covarianza   y  la   pseudo-covarianza  (ver
Sec.~\ref{Sec:MP:VectoresComplejosMatricesAleatorias}  por las  relaciones entre
la  covarianza  de  \ $\widetilde{Z}$  \  y  estas  matrices).

Se denota \ $Z \, \sim \, \CN(m,\Sigma,\check{\Sigma})$ \ con \ $m \in \Cset^d$,
\ $\Sigma \in \Pos_d^+(\Cset)$ \ conjunto  de las matrices de \ $\Mat_{d,d}(\Cset)$ \
herm\'iticas definidas positivas, y  \ $\check{\Sigma} \in \Sim_d(\Cset)$ \ conjunto
de las matrices de \  $\Mat_{d,d}(\Cset)$ \ symmetricas (ver notaciones).  Un caso
particular  aparece cuando  \ $Z$  \  es propio  en torno  a  \ $m$,  lo que  es
equivalente en el caso gaussiano a tener \ $Z$ \ circular (ver m\'as adelante) en
torno a  \ $m$, dado  cuando \  $\check{\Sigma} = 0$:  en este caso  usaremos la
misma notaci\'on,  $Z \,  \sim \, \CN(m,\Sigma)$.   Las caracter\'isticas  de la
gaussiana  compleja   son  las  siguientes~\cite{Lap17,   Pic96,  Goo63,  Bos95,
  SchSch03, EriKoi06}:

\begin{caracteristicas}
%
Dominio de definici\'on & $\Z = \Cset^d$\\[2mm]
\hline
%
Par\'ametros & $m \in \Cset^d, \:\: \Sigma \in \Pos_d^+(\Cset), \:\: \check{\Sigma}
\in \Sim_d(\Cset)$\\[2mm]
\hline
%
Densidad de probabilidad & \\[1mm]
%
Caso general: & $\displaystyle p_Z(z) = \frac{1}{\pi^d \left| \Sigma
 \right|^{\frac12} \left| P \right|^{\frac12}} \: e^{- (z-m)^\dag P^{-1} (z-m) +
 \real{(z-m)^t R^t P^{-1} (z-m)}}$\vspace{2.5mm}\newline
 con~\footnote{En~\cite{Pic96} la expresi\'on es ligieramente diferente, pero se
 recupera usando la simetr\'ia \ $\check{\Sigma}^* =
 \check{\Sigma}^\dag$. Recordar que \ $\cdot^{-*} = \left( \cdot^* \right)^{-1}$
 \ (ver notaciones).} \ $P = \Sigma - \check{\Sigma} \, \Sigma^{-*}
 \, \check{\Sigma}^\dag, \quad R = \check{\Sigma}^\dag \, \Sigma^{-1}$.\\[2.5mm]
%
Caso circular: & $\displaystyle p_Z(z) = \frac{1}{\pi^d \left| \Sigma \right|}
 \: e^{- (z-m)^\dag \Sigma^{-1} (z-m)}$\\[2.5mm]
\hline
%
Promedio & $ m_Z = m$\\[2mm]
\hline
%
Covarianza & $\Sigma_Z = \Sigma$\\[2mm]
\hline
%
Pseudo-covarianza & $\check{\Sigma}_Z = \check{\Sigma}$\\[2mm]
\hline
%%
%Generadora de  momentos &  $\displaystyle M_X(u) =  e^{u^t \Sigma u + u^t m}$  \ para \  $u \in
%\Cset^d$\\[2mm]
%\hline
%%
Funci\'on caracter\'istica & \\[1mm]
%
Caso general: & $\displaystyle \Phi_Z(\omega) = e^{-\frac14
\omega^\dag \Sigma \omega - \frac14 \real{\omega^\dag \check{\Sigma} \omega^*} +
\imath \real{\omega^\dag m}}, \quad \omega \in \Cset^d$\\[1mm]
%
Caso circular: & $\displaystyle \Phi_Z(\omega) = e^{-\frac14
\omega^\dag \Sigma \omega +
\imath \real{\omega^\dag m}}, \quad \omega \in \Cset^d$
\end{caracteristicas}

Notar que  en el caso  escalar propio (circular),  la varianza de  \ $Z$ \  es \
$\sigma_Z^2  = 2  \sigma^2$.  El coefficiente  2 viene  del  hecho que  \ $Z$  \
contiene dos componentes independientes de varianza $\sigma^2$.

Los vectores aleatorios complejos van a compartir las propiedades del caso real,
siendo equivalente  a un vector  $2d$-dimensional gaussiano real.

% De  manera  general,  las caracter\'isticas  de  \  $X$  \ gaussiano  real  son
% v\'inculadas a las  de \ $N$ \ (y vice-versa)  por transformaci\'on afine (ver
% secciones anteriores).

Primero, los cumulantes de orden superior o igual a $4$ valen cero:
%
\begin{lema}[Gaussiana compleja y cumulantes]
%
  Sea \ $Z$ \ vector aleatorio complejo de media $m$, de covarianza $\Sigma$, de
  pseudo-covarianze  $\check{\Sigma}$ y  de  secunda funci\'on  caracter\'istica
  admtiendo un desarollo de Taylor. Entonces para cualquier
  %
  \[
  \kappa_{i_1,\ldots,i_l,i'_1,\ldots,i'_m}[Z] = 0 \quad \forall \: ( i_1, \ldots
  , i_l  , i'_1  , i'_m  \in \{  1 , \ldots  , d  \}^{l+m}, \,  l+m \ge  4 \quad
  \Longleftrightarrow \quad X \sim \CN(m,\Sigma,\check{\sigma})
  \]
  %
\end{lema}
%
%\begin{proof}
%  La pueba sigue paso a paso la del lema~\ref{Lem:MP:CumSecFctCarac}.
%\end{proof}

Secundo, como en  el caso real, la gaussiana es  estable por combinaci\'on lineal
de vectores independientes:
%
\begin{teorema}[Stabilidad]
\label{Teo:MP:StabilidadGaussianaCompleja}
%
  Sean \ $A_i , i = 1,\ldots,n$  \ matrices de \ $\Mat_{d',d}(\Cset), \: d' \le d$
  \  de   rango  lleno,   $b_i  \in  \Cset^{d'}$   \  y   \  $Z_i  \,   \sim  \,
  \CN(m_i,\Sigma_i,\check{\Sigma}_i)$   \   $d$-dimensionales,  independientes,
  entonces
  %
  \[
  \sum_{i=1}^n \left( A_i  Z_i + b_i \right) \,  \sim \, \CN\left( \sum_{i=1}^n
    \left( m_i + b_i \right) \, ,  \, \sum_{i=1}^n A_i \Sigma_i A_i^\dag \, , \,
    \sum_{i=1}^n A_i \check{\Sigma}_i A_i^t \right)
  \]
  %
  En particular, cualquier combinaci\'on lineal  de los componentes de un vector
  gaussiano complejo  da una  gaussiana compleja.  Reciprocamente,  si cualquier
  combinaci\'on lineal de  los componentes de un vector  aleatorio sigue una ley
  gaussiana compleja, entonces el vector es gaussiano complejo.
\end{teorema}
%
El corolario~\ref{Cor:MP:MediaEmpiricaGauss} se extiende naturalmente al caso complejo:
%
\begin{corolario}[Media emp\'irica]\label{Cor:MP:MediaEmpiricaGaussCompleja}
%
  Sean \ $Z_i \, \sim \, \CN\left(  m , \Sigma , \check{\Sigma} \right), \: i =
  1, \ldots , n$ \ independientes. Entonces,
  %
  \[
  \overline{Z} =  \frac{1}{n} \sum_{i=1}^n Z_i \,  \sim \, \CN\left( m  \, , \,
    \frac{1}{n} \Sigma \, , \, \frac{1}{n} \check{\Sigma} \right)
  \]
  %
\end{corolario}

Adem\'as, en el  caso complejo se tiene  una estabilidad combinando \ $Z$  \ y \
$Z^*$:
%
\begin{teorema}%[Stabilidad]
\label{Teo:MP:StabilidadGaussianaComplejaZZestrella}
%
Sean \ $A \in \Mat_{d',d}(\Cset)$, \  $B \in \Mat_{d',d}(\Cset)$ \ tales que \ ambas
\ $A+B$ \ y \  $A-B$ \ \SZ{sean de rango lleno}, $c \in \Cset^{d'}$  \ y \ $Z \,
\sim \, \CN(m,\Sigma,\check{\Sigma})$ \ $d$-dimensonal, entonces
  %
  \[
  A Z + B Z^* + c \: \sim \: \CN\left( \mu , C , \check{C}  \right)
  \]
  %
  con
  %
  \[
  \begin{array}{lll}
  \mu & = & A m + B m^* + c\\[2.5mm]
  %
  C & = & A \Sigma A^\dag + B \Sigma^* B^\dag + A \check{\Sigma} B^\dag + B
  \check{\Sigma}^* A^\dag\\[2.5mm]
  %
  \check{C} & = & A \check{\Sigma} A^t + B \check{\Sigma} B^t + A \Sigma B^t + B
  \Sigma^* A^t
  \end{array}
  \]
\end{teorema}
\begin{proof}
  Tomando la forma real $2d$-dimensional \ $Z = X + \imath Y$ \ con $X, Y$
  reales,   es  en   biyecci\'on  con   $\widetilde{Z}  =   \begin{bmatrix}  X\\
    Y   \end{bmatrix}$  \  y   entonces  \   $Z^*$  \   en  biyecci\'on   con  \
  $\widetilde{Z^*} = \begin{bmatrix} X\\-Y \end{bmatrix}$. Eso da \ $A Z + B Z^*
  +   c$   \   en   biyecci\'on   con   $\begin{bmatrix}   A+B   &   0\\   0   &
    A-B \end{bmatrix} \begin{bmatrix} X\\ Y \end{bmatrix} + \begin{bmatrix}
    \real{c}\\ \imag{c} \end{bmatrix}$. Notando que \ $\begin{bmatrix} A+B & 0\\
    0    &   A-B    \end{bmatrix}$   \    es    de   rango    lleno,   por    el
  teorema~\ref{Teo:MP:StabilidadGaussiana} este vector es gaussiano, lo que proba
  que \  $A Z  + B  Z^* + c$  \ es  gaussiano complejo. Las  formas de  la media,
  covarianza y  pseudo-covarianza siguen de calculos directos  de la expresi\'on
  $A Z + B Z^* + c$.
\end{proof}
%
Evidentemente, se puede combinar los dos teoremas anteriores.

El teorema del l\'imite central y sus variantes se recuperan del caso real.
%
\begin{teorema}[Teorema del l\'imite central (caso complejo)]
\label{Teo:MP:CLTComplejo}
%
Sea  \ $\{  Z_i \}_{i  \in  \Nset^*}$ \  una sucesi\'on  de vectores  aleatorios
independientes, de  misma ley, y  que admiten un  promedio \ $m$, una  matriz de
covarianza   \    $\Sigma$   \   y    una   matriz   de    pseudo-covarianza   \
$\check{\Sigma}$. Entonces
  %
  \[
  \frac{1}{\sqrt{n}}  \sum_{i=1}^m  \left( Z_i  -  m  \right)  \: \limitd{n  \to
    +\infty} \: Z \sim \CN\left( 0 , \Sigma , \check{\Sigma} \right)
  \]
  %
  donde  \ $\limitd{}$ \  significa que  el l\'imite  es en  distribuci\'on (ver
  notaciones).
\end{teorema}
%
%
Como en el caso real, aparece que  la media emp\'irica hechas a partir de vectores
complejos independientes de media \  $m$, admitiendo una covarianza \ $\Sigma$ \
una  pseudo-covarianza \  $\check{\Sigma}$, y  de misma  ley  (no necesariamente
gaussiana),  tiende a  ser gaussiana  compleja  de media  \ $m$,  de covarianza  \
$\frac{1}{n} \Sigma$, y de pseudo-covarianza \ $\frac{1}{n} \check{\Sigma}$.

No   lo   presentamos,  pero   se   transpone   sencillamente   el  teorema   de
Lindenberg-Feller~\ref{Teo:MP:LindenbergFeller} al caso complejo.

Notamos tambi\'en que, en el caso  circular, se puede escribir naturalmente \ $Z
\, \egald \, \Sigma^{\frac12} N + m$ \  con \ $N \, \sim \, \CN(0,I)$ \ donde \
$N$ \  es dicha  {\em Gaussiana estandar}  o {\em centrada-normalizada}.   Eso se
generaliza en dos direcciones.  La  primera pone tambi\'en en juega una gaussiana
estandar~\cite{Lap17}:
%
\begin{teorema}
\label{Teo:MP:GaussianaComplejaWWestrella}
%
Sea \  $Z \sim  \CN(m,\Sigma,\check{\Sigma})$.  Entonces, existen  matrices (no
\'unicas) \ $A \in \Mat_{d,d}(\Cset)$, \ $B \in \Mat_{d,d}(\Cset)$ \ tales que
  %
  \[
  Z \egald A W + B W^* + m
  \]
  %
  con \ $W \sim \CN(0,I)$ \ gaussiana estandar.
  %
\end{teorema}
\begin{proof}
  Inmediatamente
  %
  \[
  Z   =  \begin{bmatrix}   I   &  \imath   I\end{bmatrix}  \begin{bmatrix}   X\\
    Y\end{bmatrix}
  \egald \begin{bmatrix} I & \imath I\end{bmatrix} M \begin{bmatrix} U\\
    V\end{bmatrix}
  \]
  %
  con \ $U \sim \N(0,I)$ \ y \  $V \sim \N(0,I)$ \ independientes, y \ $M$ \ tal
  que  \ $M  M^t =  \begin{bmatrix} \Sigma_X  & \Sigma_{X,Y}  \\  \Sigma_{X,Y}^t &
    \Sigma_Y  \end{bmatrix}$  \   (ej.  raiz  cuadrade  de  esta   matriz  de  \
  $\Pos_{2d}^+(\Rset)$,    o     descomposici\'on    de    Cholesky~\cite{HorJoh13,
    Bha07}). Ahora, volviendo a la forma compleja tenemos
  %
  \[
  Z     \egald   \begin{bmatrix}    I   &   \imath   I\end{bmatrix}
  M \begin{bmatrix} I  & I\\ -\imath I &  \imath I \end{bmatrix} \begin{bmatrix}
    \frac12 (U + \imath V)\\ \frac12 (U - \imath V)\end{bmatrix}
  \]
  %
  Se cierra la prueba denotando
  %
  \[
  \begin{bmatrix}   A   &   B\end{bmatrix}   =  \begin{bmatrix} I &
    \imath  I\end{bmatrix}  M  \begin{bmatrix}  I  &  I\\  -\imath  I  &  \imath
    I \end{bmatrix}
  \]
  %
  y notando que \ $W \equiv \frac12 (U + \imath V) \sim \CN(0,I)$.
\end{proof}
%
Notar  que, usando  la descomposici\'on  de Cholesky,  tenemos \  $M$ triangular
inferior~\footnote{Se puede hacer el  mismo razonamiento con la forma triangular
  superior;  se cambia  los roles  de \  $X$  \ e  \ $Y$  \ en  las matrices  de
  covarianza.},  y  entonces bloc-triangular  inferior  \  $M =  \begin{bmatrix}
  \alpha  &  0  \\  \beta  &  \gamma \end{bmatrix}$.   Eso  conduce,  a  $M  M^t
= \begin{bmatrix} \Sigma_X & \Sigma_{X,Y} \\ \Sigma_{X,Y}^t &
  \Sigma_Y \end{bmatrix}  = \begin{bmatrix} \alpha \alpha^t &  \alpha \beta^t \\
  \beta \alpha^t &  \beta \beta^t + \gamma \gamma^t  \end{bmatrix}$.  Eso da por
ejemplo   \   $\alpha  =   \Sigma_X^{\frac12}$,   \   $\beta  =   \Sigma_{X,Y}^t
\Sigma_X^{-\frac12}$  \   y  \  $\gamma  =  \left(   \Sigma_Y  -  \Sigma_{X,Y}^t
  \Sigma_X^{-1} \Sigma_{X,Y} \right)^{\frac12}$.   A continuaci\'on, $A = \alpha
+  \gamma +  \imath \beta$  \ y  \ $B  = \alpha  - \gamma  + \imath  \beta$. Una
soluci\'on posible es entonces
%
\[
\left\{\begin{array}{lll}
A & = & \Sigma_X^{\frac12} + \left( \Sigma_Y - \Sigma_{X,Y}^t \Sigma_X^{-1} \Sigma_{X,Y}
\right)^{\frac12}  + \imath \Sigma_{X,Y}^t
\Sigma_X^{-\frac12}\\[2.5mm]
%
B & = &  \Sigma_X^{\frac12} - \left( \Sigma_Y - \Sigma_{X,Y}^t \Sigma_X^{-1} \Sigma_{X,Y}
\right)^{\frac12} + \imath \Sigma_{X,Y}^t
\Sigma_X^{-\frac12}
\end{array}\right.
\]
%
Se  puede   re-escribir  estas  matrices   a  partir  de   \  $\Sigma$  \   y  \
$\check{\Sigma}$       \      usando       las       relaciones      de       la
secci\'on~\ref{Ssec:MP:VAComplejos}.

La  secunda  extensi\'on  pone  en  juega  una sola  gaussiana  compleja  sin  su
conjugada~\cite{EriKoi06, SchSch03}:
%
\begin{teorema}
\label{Teo:MP:GaussianaComplejaWIDiago}
%
  Sea \ $Z \sim \CN(m,\Sigma,\check{\Sigma})$. Entonces, existe una matriz \ $C
  \in \Mat_{d,d}(\Cset)$ \ tal que
  %
  \[
  Z \egald C W + m
  \]
  %
  con \ $W \sim \CN(0,I,\Delta)$ \ con \ $\Delta \in \Pos_d(\Rset)$ \ (real) diagonal.
  %
\end{teorema}
\begin{proof}
  Eso  viene  de teoremas  de  diagonalizaci\'on  conjunta. M\'as  precisamente,
  siendo \ $\Sigma  \in \Pos_d^+(\Cset)$ \ y \  $\check{\Sigma} \in \Sim_d(\Cset)$, se
  aplica el  teorema~\cite[Teo.~7.6.5]{HorJoh13} diciendo que  existe una matriz
  no  singular (invertible)  \  $C$ \  tal  que \  $\Sigma  = C  C^\dag$  \ y  \
  $\check{\Sigma} = C  \Delta C^t$ \ con $\Delta$ \  real diagonal con elementos
  positivos  ($\Delta  \in  \Pos_d(\Rset)$  \  diagonal).  Inmediatamente,  por  el
  teorema~\ref{Teo:MP:StabilidadGaussianaCompleja}, tenemos
  %
  \[
  C^{-1} (Z - m) \egald W \sim \CN(0,I,\Delta)
  \]
  lo que cierra la prueba.
\end{proof}


Al final, v\'imos en la  secci\'on~\ref{Ssec:MP:VAComplejos} que si un vector es
circular, entonces su pseudo-covarianza es  nula, pero la rec\'iproca no vale en
general. Aparece que en el contexto gaussiano tenemos la rec\'iproca:
%

\begin{teorema}[Circularidad]\label{Teo:MP:CircularidadGaussiana}
%
Sea \ $Z \, \sim \, \CN(m,\Sigma,\check{\Sigma})$.  Entonces,
  %
  \[
  Z \: \mbox{ circular  en torno a } \: m \qquad  \Longleftrightarrow \qquad Z \:
\mbox{ propio en torno a } \: m
  \]
\end{teorema}
%
\begin{proof}
  V\'imos     la    directa    en     la    secci\'on~\ref{Ssec:MP:VAComplejos},
  teorema~\ref{Teo:MP:Circularidad}.  Reciprocamente,  si \  $Z$ \ es  propio en
  torno a \ $m$, por definici\'on \ $\check{\Sigma} = 0$ \ y el resultado viene
  de la forma de  la funci\'on caracter\'istica por ejemplo: $\Phi_{Z-m}(\omega)
  = e^{-\frac14 \omega^\dag \Sigma \omega } = \Phi_{Z-m}\left( e^{\imath \theta}
    \omega \right) = \Phi_{e^{\imath \theta} (Z-m)}(\omega)$.
\end{proof}


%\SZ{Caso $X \sim \N(m,\Sigma)$; modulaci\'on $Z = e^{\imath \theta} X$}


% --------------------------------- Exponencial
%\subsubseccion{Distribuci\'on exponencial}
\label{Sssec:MP:Exponencial}

A pesar de que sea un caso particular de la distribuci\'on Gamma que vamos a ver
m\'as  adelante,  estudiada por  Pearson  desde  el  a\~no 1895~\cite{Pea95},  o
apareci\'o  quizas   un  poco   antes  en  trabajos   de  L.   Boltzmann   o  de
Whitworth~\cite{BalBas95} (como caso lim\'ite  de la ley de Poisson), apareci\'o
esta  ley  de  manera  ``propia''   mucho  m\'as  tarde,  entre  otros  en  1930
en~\cite[Ec.~(46)]{Kon30}.

Se denota $X \,  \sim \, \E(\lambda)$ \ con \ $\lambda  \in \Rset_+^*$ \ llamada
{\em  taza}  (inversa  de  {\em   escala}),  y  sus  caracter\'isticas  son  las
siguientes:

\begin{caracteristicas}
%
Dominio de definici\'on & $\X = \Rset_+$\\[2mm]
\hline
%
Par\'ametro & $\lambda \in \Rset_+^*$\\[2mm]
\hline
%
Densidad  de probabilidad &  $\displaystyle p_X(x)  = \lambda  e^{-\lambda x}$\\[2mm]
\hline
%
Promedio & $\displaystyle m_X = \frac1\lambda$\\[2mm]
\hline
%
Varianza & $\displaystyle \sigma_X^2 = \frac1{\lambda^2}$\\[2mm]
\hline
%
\modif{Asimetr\'ia} & $\gamma_X = 2$\\[2mm]
\hline
%
Curtosis por exceso & $\widebar{\kappa}_X = 6$\\[2mm]
\hline
%
Generadora de  momentos &  $\displaystyle M_X(u) =  \frac{\lambda}{\lambda-u}$ \
para \ $\real{u} < \lambda$\\[2mm]
\hline
%
Funci\'on     caracter\'istica     &     $\displaystyle     \Phi_X(\omega)     =
\frac{\lambda}{\lambda - \imath \omega}$
\end{caracteristicas}

% Momentos & $ \Esp\left[ X^k \right] = p^k$\\[2mm]
% Momento factorial & $\Esp\left[ (X)_k \right] = ?$\\[2mm]
% Generadora de probabilidad & $G_X(z) = e^{\lambda (z-1)}$ \ para \ $z \in \Cset$\\[2mm]
% modo 0
% Mediana \ln(2)/\lambda
% CDF 1-e^{-\lambda x}

Su densidad  de probabilidad  y funci\'on de  repartici\'on son representadas  en la
figura Fig.~\ref{Fig:MP:Exponencial}.
%
\begin{figure}[h!]
\begin{center} \begin{tikzpicture}%[scale=.9]
\shorthandoff{>}
%
\pgfmathsetmacro{\sx}{.75};% x-scaling
\pgfmathsetmacro{\r}{.05};% radius arc non continuity F_X
\pgfmathsetmacro{\l}{1.5};% lambda
\pgfmathsetmacro{\mx}{6};% x maximo del plot
%
% densidad
\begin{scope}
%
\pgfmathsetmacro{\sy}{2.5/\l};% y-scaling 
\draw[>=stealth,->] ({-\sx-.25},0)--({\sx*\mx+.25},0) node[right]{\small $x$};
\draw[>=stealth,->] (0,-.1)--(0,{\sy*\l+.25}) node[above]{\small $p_X$};
%
\draw[thick] ({-\sx},0)--(0,0);
\draw (\r,\r) arc (90:270:\r);
\draw[dotted] (0,0)--(0,{\sy*\l}) node[scale=.4]{$\bullet$};
\draw[thick,domain=0:\mx,samples=100] plot ({\x*\sx},{\sy*\l*exp(-\l*\x)});
%
\draw (0,{\l*\sy})--(-.1,{\l*\sy}) node[left,scale=.7]{$\lambda$};
%
\end{scope}
%
%
% reparticion
\begin{scope}[xshift=8.5cm]
%
\pgfmathsetmacro{\sy}{2.5};% y-scaling 
%
\draw[>=stealth,->] (-.6,0)--({\sx*\mx+.25},0) node[right]{\small $x$};
\draw[>=stealth,->] (0,-.1)--(0,{\sy+.25}) node[above]{\small $F_X$};
%
% cumulativa
\draw[thick,domain=0:\mx,samples=100] (-.5,0)--(0,0) plot({\x*\sx},{(1-exp(-\l*\x))*\sy});
%
\draw (0,\sy)--(-.1,\sy) node[left,scale=.7]{$1$};
\end{scope}
%
\end{tikzpicture} \end{center}
% 
\leyenda{Ilustraci\'on  de una densidad  de probabilidad  exponencial (a),  y la
funci\'on de repartici\'on asociada (b), con $\lambda = 1.5$.}
\label{Fig:MP:Exponencial}
\end{figure}
\SZ{Poner escalas; Otros ilustraciones para otros $\lambda$?}

La  ley exponencial  es conocida  como siendo  {\em sin  memoria}, es  decir, si
buscamos \  $X$ \  visto como un  tiempo (ej.  tiempo de desintegraci\'on  de un
atomo radioactivo) tal que
%
\[
\forall \: x_0 \ge 0, \, x \ge 0, \quad P( X > x+x_0 | X > x_0) = P(X > x)
\]
%
\ie  la  probabilidad  que  $X  >  x+x_0$  (extra  tiempo  despu\'es  de  $x_0$)
condicionalmente  a $X >  x_0$ es  exactamente la  de $X  > x+x_0$  (se olvid\'o
$x_0$), tenemos, por la definici\'on de la probabilidad condicional
%
\[
\forall \: x_0 \ge 0, \, x \ge 0, \quad \frac{1-F_X(x+x_0)}{1-F_X(x_0)} = 1-F_X(x)
\]
%
Por diferenciaci\'on con respeto a $x$ eso da, en $x \to 0$,
%
\[
\forall  \: x_0  \ge 0,  \quad  F_X'(x_0) +  \lambda F_X(x_0)  = \lambda  \qquad
\mbox{con} \qquad \lambda = F_X'(0)
\]
%
Teneiendo en cuenta de que $F_X$  es una funci\'on de repartici\'on, aparece que
$F_X(x) = \left( 1 - e^{-\lambda x} \right) \un_{\Rset_+}(x)$, ley exponencial.

Como  lo hemos  evocado  tratando de  la  ley de  Poisson,  esta es  v\'inculada
intimamente a la ley exponencial a trav\'es del processo dicho de poisson:
%
\begin{lema}[V\'inculo con la ley de Poisson]
\label{Lem:MP:VinculoExponencialPoisson}
%
  Sea  $T_0 =  0$ \  y \  $\forall \:  n \in  \Nset^*$ las  variables aleatorias
  positivas \ $T_n$ \ tales que $T_{n+1}  - T_n \ge 0$ \ son independientes y de
  distribuci\'on $\E(\lambda)$. Fijamos \ $T > 0$ \ y sea \ $X$ \ el n\'umero de
  variables  \  $T_n$  \  que  partenecen  a   $(0  \;  T)$,  \ie  $T_X  <  T  <
  T_{X+1}$. Entonces
  %
  \[
  X \sim \P(\lambda T)
  \]
\end{lema}
%
Dicho  de otra manera,  si tenemos  eventos que  aparecen en  tiempos aleatorios
tales  que los  incrementos de  tiempos entre  eventos son  independientes  y de
distribuci\'on  exponencial de  taza $\lambda$,  el  n\'umero de  eventos en  un
intervalo  de tiempo $T$  dado sigue  una ley  de Poisson,  de taza  $\lambda T$
proporcional al  intervalo, y proporcional a  la taza de la  ley exponencial. El
par\'ametro \ $\lambda$ \ representa la taza de evento por unidad de tiempo.
%
\begin{proof}
Por definici\'on,
%
\begin{eqnarray*}
P(X = n) & = & P(X\le n) - P(X \le n-1) \\[2mm]
%
& = & P(T_{n+1} > T) - P(T_n > T)\\[2mm]
%
& = & F_{T_n}(T) - F_{T_{n+1}}(T)
\end{eqnarray*}
%
Ahora, notando que
%
\[
T_n = \sum_{i=0}^{n-1} \left( T_{i+1} - T_i \right)
\]
%
de la  independencia de los  incrementos de tiempo,  y de las propiedades  de la
funci\'on caracter\'istica, tenemos
%
\[
\Phi_{T_n}(\omega) = \frac{\lambda^n}{(1-\imath \, \omega)^n}
\]
%
De  la  f\'ormula  de  inversion del  teorema~\ref{Teo:MP:InversionDensidad}  se
prueba  que~\footnote{Una manera  es  de  hacer una  integraci\'on  en el  plano
  complejo  y usar  los lemas  de Jordan  y teorema  de residuos~\cite{CarKro05}
  o~\cite[Cap.~4]{AblFok03}.  Nota: de hecho se  reconoce en \ $\Phi_{T_n}$ \ la
  funci\'on caracter\'istica de una ley gamma \ $\G(n,\lambda)$, ley que vamos a
  ver en la secci\'on~\ref{Sssec:MP:Gamma}.}
%
\[
p_{T_n}(x) = \frac{\lambda^n x^{n-1} e^{-\lambda x}}{(n-1)!} \un_{\Rset_+}(x)
\]
%
Con integraciones por partes, se obtiene sencillamente
%
\[
F_{T_n}(T) = 1 - \sum_{i=0}^{n-1} \frac{\lambda^i T^i e^{-\lambda T}}{i!}
\]
%
lo que cierra la prueba.
\end{proof}
%
En  f\'isica,  se  modela la  ley  de  tiempo  de desintegraci\'on  como  siendo
exponencial,  y   se  supone  que  los   desintegraciuones  son  independientes,
explicando el modelo de Poisson  para el n\'umero de desintegraci\'on durante un
tiempo dado.

Una otra caracter\'istica de esta ley  es su stabilidad con respecto al operador
no lineal m\'inimo:
%
\begin{lema}[Stabilidad por el m\'in]
\label{Lem:MP:StabilidadExponencialMinimo}
%
  Sean  $X_i \sim \E(\lambda), \: i = 1, \ldots , n$ \  independientes. Entonces,
  \[
  \min_{i=1,\ldots,n} X_i \equiv X \sim \E(n \lambda)
  \]
\end{lema}
%
\begin{proof}
Inmediatamente, para cualquief $x \ge 0$
%
\begin{eqnarray*}
1-F_X(x) & = & P(X > x) \\[2mm]
%
& = & P\left( \bigcap_{i=1}^n \big( X_i > x \big) \right)\\[2mm]
%
& = & \prod_{i=1}^n P(X_i > x)\\[2mm]
%
& = & e^{- n \lambda x}
\end{eqnarray*}
%
La   secunda   linea   viene   de   la  equivalencia   entre   los   eventos   $
\min_{i=1,\ldots,n}  X_i >  x$ y  $\bigcap_{i=1}^n  \big( X_i  > x  \big)$ y  la
tercera de la independencia de los $X_i$.
\end{proof}


% --------------------------------- Gamma
%\subsubseccion{Distribuci\'on gamma}
\label{Sssec:MP:Gamma}

Como lo introdujimos en el ejemplo  de la ley exponencial, esta familia de leyes
fue  estudiada  por primera  vez  al  fin del  siglo  XIV,  bajo  el impulso  de
Pearson~\cite{Pea95}.   De hecho,  seg\'un Lancaster~\cite{Lan66}  se encuentran
trazas  de esta  ley en  trabajos de  Laplace como  posterior  distribuci\'on en
inferencia Bayesiana (elementos conduciendo a la ley gamma) para la estimaci\'on
de  la dispersi\'on  $\frac1{\sigma^2}$  de  una ley  gaussiana.   De hecho,  la
distribuci\'on gamma aparece frecuentemente en problemas de inferencia Bayesiana
como    distribuci\'on     a    priori    conjugado~\footnote{Ver     nota    de
  pie~\ref{Foot:MP:BayesPrior}  por la explicaci\'on  del enfoque  bayesiano que
  consiste  a calcular  la distribuci\'on  a  posteriori $p_{\Theta|X=x}(\theta)
  \propto p_{X|\Theta=\theta}(x) p_\Theta(\theta)$ \  usando la ley de los datos
  parametrizado  por $\theta$  que queremos  inferir, modelizado  aleatorio. Por
  eso,  como  se  lo ve  en  la  f\'ormula  de  Bayes,  se necesita  elegir  una
  distribuci\'on a priori  \ $p_\Theta$.  V\'imos que una  elecci\'on posible es
  tomarla uniforme. Puede  ser problematico por ejemplo cuando  $\theta$ vive en
  un  espacio de  volumen inifinito  (a  priori impropio),  a\'un si  se lo  usa
  frecuentemente (en estimaci\'on es equivalente a considerar la verosimulitud).
  Una otra elecci\'on posible es tomar  el a priori en una familia parametrizada
  tal que la  distribuci\'on a posterior partenece tambi\'en  a esta familia: es
  lo que se llama {\em a  priori conjugado} para la ley de sampleos $p_{X|\Theta
    = \theta}$. La idea es que  si vienen observaciones, en lugar de re-calcular
  la ley a posteriori, se  puede actualizar solamente los par\'ametros (llamados
  hiperpar\'ametros).\label{Foot:MP:BayesPriorConjugado}}     del    par\'ametro
$\lambda$ de la  ley de Poisson~\cite{Rob07}. Se encuentren  tambi\'en trazas de
esta ley en trabajos de J.  Bienaym\'e como distribuci\'on l\'imite del promedio
centrado   y  renormalizado   de   los   componentes  de   un   vector  de   ley
multinomial~\cite{Bie38, Lan66}.

Se  denota $X \,  \sim \,  \G(a,b)$ \  con \  $a \in  \Rset_+^*$ \  llamado {\em
par\'ametro de  forma} \ y \  $b \in \Rset_+^*$  \ llamada {\em taza}  (inversa de
{\em escala}). Las caracter\'isticas son:

\begin{caracteristicas}
%
Dominio de definici\'on & $\X = \Rset_+$\\[2mm]
\hline
%
Par\'ametros & $a \in \Rset_+^*$ \ (forma), \: $b \in \Rset_+^*$ \ (taza)\\[2mm]
\hline
%
Densidad  de probabilidad  &  $\displaystyle p_X(x)  =  \frac{b^a \, x^{a-1} \,  e^{-b
x}}{\Gamma(a)}$\\[2mm]
\hline
%
Promedio & $\displaystyle m_X = \frac{a}{b}$\\[2mm]
\hline
%
Varianza & $\displaystyle \sigma_X^2 = \frac{a}{b^2}$\\[2mm]
\hline
%
\modif{Asimetr\'ia} & $\displaystyle \gamma_X = \frac2{\sqrt{a}}$\\[2mm]
\hline
%
Curtosis por exceso & $\displaystyle \widebar{\kappa}_X = \frac6{a}$\\[2mm]
\hline
%
Generadora  de momentos  & $\displaystyle  M_X(u) =  \left( 1  - \frac{u}{b}
\right)^{-a}$ \ para \ $\real{u} < b$\\[2mm]
\hline
%
Funci\'on  caracter\'istica  &  $\displaystyle   \Phi_X(\omega)  =  \left(  1  -
\frac{ \imath \omega}{b} \right)^{-a}$
\end{caracteristicas}

% Momentos & $ \Esp\left[ X^k \right] = p^k$\\[2mm]
% Momento factorial & $\Esp\left[ (X)_k \right] = ?$\\[2mm]
% Generadora de probabilidad & $G_X(z) = e^{\lambda (z-1)}$ \ para \ $z \in \Cset$\\[2mm]
% modo max(a-1,0)
% Mediana no close ver inverse gamma

Nota: trivialmente, se puede escribir $X \,  \egald \, \frac{1}{b} G$ \ con \ $G
\, \sim \, \G(a,1)$  \ donde \ $G$ \ es estandardizada  o normalizada. De nuevo,
las  caracter\'isticas  de \  $X$  \  son  v\'inculadas a  las  de  \ $G$  \  (y
vice-versa) por transformaci\'on lineal (ver secciones anteriores).

Unas densidades de probabilidad gamma y las funciones de repartici\'on asociadas
son representadas en la figura Fig.~\ref{Fig:MP:Gamma}  para varios $a$ \ y \ $b
= 1$.
%
\begin{figure}[h!]
\begin{center} \begin{tikzpicture}%[scale=.9]
\shorthandoff{>}
%
\pgfmathsetmacro{\sx}{.75};% x-scaling
\pgfmathsetmacro{\mx}{8};% x maximo del plot
%
% Approximation de la cdf gaussienne
\tikzset{declare function={
normcdf(\x)=1/(1 + exp(-0.07056*(\x)^3 - 1.5976*(\x)));
}}
%
% densidad
\begin{scope}
%
\pgfmathsetmacro{\sy}{2.5};% y-scaling 
\draw[>=stealth,->] (-.75,0)--({\sx*\mx+.25},0) node[right]{\small $x$};
\draw[>=stealth,->] (0,-.1)--(0,2.75) node[above]{\small $p_X$};
%
%\foreach \a in {1,...,3} {
\draw[thick] (-.5,0)--(0,0);
\draw[thick,dotted,domain=.175:\mx,samples=100] plot ({\x*\sx},{\sy*(\x^(-.5))*exp(-\x)/sqrt(pi)});
\draw[thick,dashed,domain=0:\mx,samples=100] plot ({\x*\sx},{\sy*exp(-\x)});
\draw[thick,dash dot,domain=0:\mx,samples=100] plot ({\x*\sx},{\sy*\x*exp(-\x)});
%\draw[thick,domain=0:\mx,samples=100] plot ({\x*\sx},{\sy*4*\x*sqrt(\x)*exp(-\x)/3/sqrt(pi)});
\draw[thick,domain=0:\mx,samples=100] plot ({\x*\sx},{\sy*\x*\x*exp(-\x)/2});
%}
%
\draw (0,\sy)--(-.1,\sy) node[left,scale=.7]{$1$};
\draw (0,{\sy*exp(-1)})--(-.1,{\sy*exp(-1)}) node[left,scale=.7]{$e^{-1}$};
\draw (0,{\sy*2*exp(-2)})--(-.1,{\sy*2*exp(-2)}) node[left,scale=.7]{$2 \, e^{-2}$};
\draw (\sx,0)--(\sx,-.1) node[below,scale=.7]{$1$};
\draw ({2*\sx},0)--({2*\sx},-.1) node[below,scale=.7]{$2$};
%
\end{scope}
%
%
% reparticion
\begin{scope}[xshift=8.5cm]
%
\pgfmathsetmacro{\sy}{2.5};% y-scaling 
%
\draw[>=stealth,->] (-.75,0)--({\sx*\mx+.25},0) node[right]{\small $x$};
\draw[>=stealth,->] (0,-.1)--(0,{\sy+.25}) node[above]{\small $F_X$};
%
% cumulativa
\draw[thick] (-.5,0)--(0,0);
\draw[thick,dotted,domain=0:\mx,samples=100] plot ({\x*\sx},{(2*normcdf(sqrt(2*\x))-1)*\sy});
\draw[thick,dashed,domain=0:\mx,samples=100] plot ({\x*\sx},{\sy*(1-exp(-\x))});
\draw[thick,dash dot,domain=0:\mx,samples=100] plot ({\x*\sx},{\sy*(1-(1+\x)*exp(-\x))});
\draw[thick,domain=0:\mx,samples=100] plot ({\x*\sx},{\sy*(1-(1+\x+\x*\x/2)*exp(-\x))});
% plot({\x*\sx},{\sy*normcdf(\x)});
%
\draw (0,\sy)--(-.1,\sy) node[left,scale=.7]{$1$};
\end{scope}
%
\end{tikzpicture} \end{center}
%
\leyenda{Ilustraci\'on de una densidad de probabilidad gamma (a), y la funci\'on
de  repartici\'on asociada  (b).   $b  = 1$  \  y \  $a  = 0.5$  (linea
punteada), $1$ (linea mixta), $2$ (linea guionada) y $3$ (linea llena).}
\label{Fig:MP:Gamma}
\end{figure}

Cuando $a \in \Nset^*$ es entero, la ley es a veces conocida como ley de Erlang,
del nombre  de un ingeniero dan\'es  trabajando en (fundador de  la) teor\'ia de
colas~\cite{Cox62, Erl09, Erl25, BroHal48}.  Si \  $a = \frac{n}{2}$ \ con \ $n$
\ entero y  \ $\beta = \frac12$, se conoce tambi\'e  como ley {\em chi-cuadrado}
con \ $n$ \ grados de libertad (ver ej.~\cite{JohKot95:v1}).
% cf archivo queueing theory "Hillier"en mi carpeta

Notar que \ $X \, \sim \,  \G(1,b)$ \ es una variable exponencial de par\'ametro
\ $b$,  \ie \ $X  \, \sim \,  \E(b)$. Cuando \  $a < 1$,  la densidad \  $p_X$ \
diverge  para \  $x  \to 0$  \  (divergencia integrable).  Adem\'as, se  muestra
tambi\'en sencillamente con las funciones caracter\'isticas que:
%
\begin{lema}[Stabilidad]
\label{Lem:MP:StabilidadGamma}
%
  Sean $X_i  \, \sim  \, \G\left( a_i  , b  \right), \: i  = 1 ,  \ldots ,  n$ \
  independientes. Entonces
  % 
  \[
  \sum_{i=1}^n X_i \, \sim \, \G\left( \sum_{i=1}^n a_i \, , \, b \right)
  \]
\end{lema}
%
En particular, la suma de  variables independientes de ley exponencial de mismos
par\'ametro sigue una distribuci\'on de Erlang de par\'ametro de forma $n$.

Adem\'as, se  muestra sencillamente por cambio  de variables y  con la funci\'on
caracter\'istica un v\'inculo con variables gaussianas:
%
\begin{lema}[V\'inculo con la gaussiana]
\label{Lem:MP:VinculoGammaGaussiana}
%
  Sean $X_i \, \sim \,  \N\left( 0 , \sigma^2 \right), \: i = 1  , \ldots , n$ \
  independientes. Entonces
  %
  \[
  \sum_{i=1}^n  X_i^2 \,  \sim \,  \G\left( \frac{n}{2}  \, ,  \,  \frac{1}{2 \,
      \sigma^2} \right)
  \]
  %
  En  esta situaci\'on,  con  $n$ entero,  la  ley es  precisamente  la ley  del
  chi-cuadrado, con \ $n$ \ grados de libertad.
\end{lema}

\begin{ejemplo}[Distibuci\'on de Maxwell-Boltzmann]
  Esta distribuci\'on apareci\'o en el  estudio de las velocidades de particulas
  en el gas perfecto, bajo  el impulso de Maxwell~\cite{Max60A, Max60B, Max67} y
  m\'as tarde de  Boltzmann~\cite{Bol77, Bol96, Bol98}.  En un  gas perfecto, se
  supone que  las particulas se  mueven libremente, sin interacciones  entre si,
  aparte  colisiones  breves con  intercambio  de  energ\'ia  entre si  (choques
  elasticos).   La energia  de cada  particula es  su energ\'ia  cinetica  $\E =
  \frac12   m  \|   v  \|^2$   donde   $v  =   \begin{bmatrix}  v_x   &  v_y   &
    v_z\end{bmatrix}^t$ es  el vector velocidad  3-dimensional.  En un  gas, hay
  tantas  particulas~\footnote{`!En   condiciones  normales  de   temperatura  y
    presi\'on, un  litro contiene  \ $2.7 \times  10^{22}$ particulas!}   que es
  imposible  describir tal  gas con  las leyes  de la  m\'ecanica.   Se modeliza
  entonces  las  velocidades  como   aleatorias.   Adem\'as,  en  este  contexto
  ``perfecto'', las particulas son  supuestas independientes entre si.  Se puede
  focalizarse sobre una  particula, que representa de una  manera el conjunto de
  particulas.          Como          lo         veremos         en         ambas
  secci\'on~\ref{Ssec:MP:FamiliaExponencial}                                    y
  cap\'itulo~\ref{Cap:SZ:Informacion},   sin   v\'inculos   adicional,   en   el
  equilibrio termodin\'amico, la ley del  vector velocidad de la particula es la
  que  maximiza la  entrop\'ia. Es  precisamente una  gausiana  3-dimensional de
  covarianza  \  $\Sigma =  \frac{m}{2  \,  k_B  T} I$,  donde  \  $T$ \  es  la
  temperatura del gas en Kelvin, y  \ $k_B \approx 1.38 \times 10^{-23}$ \ julio
  por Kelvin es la constante  de Boltzmann. En otros t\'erminos, les velocidades
  en cada direcci\'on \ $v_x, v_y,  v_z$ \ son gausianas de varianza \ $\sigma^2
  = \frac{m}{2 \,  k_B T}$ independientes.  Resuelte que la ley  de la energia \
  $\E = \frac12 m \| v \|^2$ \ es precisamente una ley chi-cuadrado con 3 grados
  de  libertad  y  la  de  \  $\|  v \|$,  la  raiz  cuadrada  de  variable  del
  chi-cuadrado, es conocida como ley  chi. En el vcaso presente, es precisamente
  conocida como ley de Maxwell-Boltzmann.
\end{ejemplo}

%\SZ{Esta distribuci\'on aparece...}
% en  el conteo  de conteo  de  une repetici\'on  de una  experiencia de  maneja
% independiente hasta que  occure un evento de probabilidad  $p$; por ejemplo el
% n\'umero de tiro de un dado  equilibriado hasta que occurre un ``6'' sigue una
% ley geometrica de par\'ametro $p = \frac16$.


% --------------------------------- Wishart
%\subsubseccion{Distribuci\'on matriz-variada de Wishart}
\label{Sssec:MP:Wishart}

Este ejemplo es una  generalizaci\'on matriz-variada de la distribuci\'on gamma.
Se puede ver  una matriz como un vector, guardando por  ejemplo sus columnas una
bajo la  precediente.  Sin embargo, tal  distribuci\'on apareciendo naturalmente
en un contexto de estimaci\'on de  matriz de covarianza (ver m\'as adelante), es
m\'as  natural  verla  matriz-variate.    Tal  distrubuci\'on  es  debido  a  J.
Wishart~\cite{Wis28, GupNag99, And03}, y se denota \ $X \, \sim \, W_d(V,\nu)$ \
donde  el  dominio  de  definici\'on  es  \  $P_d^+(\Rset)$,  conjunto  matrices
simetricas definida positivas, $V \in P_d^+(\Rset)$ par\'ametro de escala y \ $\nu
> d-1$ \ grados de libertad.  Las caracter\'isticas de la distribuci\'on son las
siguientes:

\begin{caracteristicas}
%
Dominio de definici\'on~\footnote{De hecho, se puede considerar que la matriz
aleatoria es equivalent a tener un vector \ $\frac{d (d+1)}{2}$-dimensional; por
la simetria, claramente \ $X$ \ tiene solamente \ $\frac{d (d+1)}{2}$ \
componentes diferentes; adem\'as, se puede probar que cualquier matriz \ $A \in
P_d^+(\Rset)$ \ se descompone bajo la forma \ $A = L L^t$ \ con \ $L$ \
triangular inferior con elementos no nulos sobre su diagonal, llamado
descomposici\'on de Cholesky~\cite{GupNag99, Bha07, Har08, HorJoh13} y
reciprocamente. Eso muestra que \ $A$ \ se define a partir de \ $\frac{d
(d+1)}{2}$ \ ``grados de libertad''.\label{Foot:MP:WishartXtilde}} & $\X =
P_d^+(\Rset), \: d \in \Nset^*$\\[2mm]
\hline
%
Par\'ametros & $V \in P_d^+(\Rset)$ (escala) y \ $\nu > d-1$ \ (grados de
libertad)\\[2mm]
\hline
%
Densidad de probabilidad~\footnote{La densidad de probabilidad corresponde a la
densidad conjunta de los \ $\frac{d (d+1)}{2}$ \ elementos \ $X_{i,j}, \: 1 \le
i \le j \le d$~\cite{Wis28, PedRic91, SulTra96, GupNag99,
And03}.\label{Foot:MP:WishartDensidad}} & $\displaystyle p_X(x) =
\frac{|x|^{\frac{\nu-d-1}{2}} \, e^{-\frac12 \Tr\left( V^{-1}
x\right)}}{2^{\frac{d \nu}{2}} \, |V|^{\frac{\nu}{2}} \, \Gamma_d\left(
\frac{\nu}{2} \right)}$\\[2mm]
\hline
%
Promedio & $\displaystyle m_X = \nu \, V$\\[2mm]
\hline
%
Covarianza & $\displaystyle \Sigma_X = \nu \big( J (V \otimes V) + (V \otimes I)
K (V \otimes I) \big)$\\[2mm]
% $\Cov[X_{i,j},X_{k,l}] = nu \left( V_{i,k} V_{j,l} +V_{i,l} V_{j,k} \right)$}
\hline
%
Funci\'on caracter\'istica~\footnote{\SZ{Se proba que la funci\'on generadora de
momentos no existe en
general}.\label{Foot:MP:CaracteristicaWishart}} &
$\displaystyle \Phi_X(\omega) = \left| I - 2 \imath \omega V
\right|^{-\frac{\nu}{2}}, \quad \omega \in S_d(\Rset)$
\end{caracteristicas}

(ver~\cite{PedRic91, SulTra96, And03}).

Fijense que $p_X$ no es la distribuci\'on conjuntos de los componentes de \ $X$:
el hecho de  que \ $X$ \ sea  uan matriz aleatoria de \  $P_d^+(\Rset)$ \ impone
v\'inculos sobre sus compnentes; entre otros, $X_{i,j} = X_{j,i}$.

Inmediatamente, si  $d = 1$, la  distribuci\'on de Whishart \  $W_1(V,\nu)$ \ se
reduce  a la  distribuci\'on Gamma  $\G\left(\frac{\nu}{2} \,  , \,  \frac1{2 V}
\right)$. De este  hecho, se la podr\'ia ver  como extensi\'on matriz-variada de
la  distribuci\'on  gamma.  La  distribuci\'on  de  Wishart  tiene varias  otras
propiedades como las siguientes.
%
\begin{lema}[Stabilidad por transformaci\'on lineal]
\label{Lem:MP:StabilidadWishartLineal}
%
  Sea $X \,  \sim \, W_d(V,\nu)$ \ y \  $A \in \Rset^{d \times d'}$  \ con \ $d'
  \le d$ \ y de rango lleno. Entonces
  \[
  A^t X A \, \sim \, W_{d'}\left( A^t V A , \nu \right)
  \]
  %
  En particular, si $d'  = 1$, \ $A^t X A \, \sim  \, G\left( \frac{\nu}{2} \, ,
    \, \frac1{2 \, A^t V A} \right)$. M\'as all\'a, tomando $A = \un_j$, aparece
  de que  las componentes diagonales de \  $X$ \ son de  distribuci\'on gamma, \
  $X_{j,j}  \, \sim  \,  \G\left( \frac{\nu}{2}  \,  , \,  \frac1{2 \,  V_{j,j}}
  \right)$.
\end{lema}
%
\begin{proof}
  El     resultado     es     inmediato     saliendo     de     la     funci\'on
  caract\'eristica~\footref{Foot:MP:CaracteristicaWishart} y notando de que
%
\begin{eqnarray*}
\Phi_{A^t X A}(\omega) & = & \Esp\left[ e^{\imath \Tr\left( \omega^t A^t X A
\right)}\right]\\[2mm]
%
& = & \Phi_X\left( A \omega^t A^t\right)\\[2mm]
%
& = &  \left| I - 2 \imath A \omega A^t V \right|^{-\frac{\nu}{2}}\\[2mm]
%
& = &  \left| I - 2 \imath \omega A^t V A \right|^{-\frac{\nu}{2}}
%
\end{eqnarray*}
%
de   \   $\Tr(AB)   =   \Tr(BA)$~\cite{Har08}   \   y   de   la   identidad   de
Sylvester~\cite{Syl51,  AkrAkr96}  o~\cite[\S~18.1]{Har08} \  $\left|  I  + A  B
\right| = \left| I + B A \right|$.  .
\end{proof}
%
De hecho, si los elementos diagonales son de distribuci\'on gamma, no es el caso
de         los        elementos         no-diagonales~\cite{Seb04,        And03}
o~\cite[Teo.~3.3.4]{GupNag99}.    De    eso   resuelte   delicado    llamar   la
distribuci\'on como gamma matriz-variada.

\begin{lema}[Stabilidad por suma]
\label{Lem:MP:StabilidadWishartSuma}
%
  Sea $X_i \,  \sim \, W_d(V,\nu_i), \: i = 1,\ldots,n$ independientes. Entonces
  \[
  \sum_{i=1}^n X_i \, \sim \, W_d\left( V \, , \, \sum_{i=1}^n \nu_i \right)
  \]
\end{lema}
%
\begin{proof}
  El     resultado     es     inmediato     saliendo     de     la     funci\'on
  caract\'eristica~\footref{Foot:MP:CaracteristicaWishart} y notando que como el
  el context vectorial $\Phi_{\sum_i X_i} = \prod_i \Phi_{X_i}$.
\end{proof}

La distribuci\'on  de Wishart aparece naturalmente en  problemas de estimaci\'on
de matriz de covarianza en el contexto gausiano:
%
\begin{lema}[V\'inculo con vectores gausianos~\cite{Seb04}]
  Sean \ $X_i \, \sim \, \N(0,V), \: i = 1, \ldots , n > d-1$ \ independientes y
  la  matriz  \  $S  =  \sum_{i=1}^n   X_i  X_i^t$  \  llamada  {\em  matriz  de
    dispersi\'on} (scatter matrix en ingl\'es). Entonces, \ $S \in P_d^+(\Rset)$
  (c.   s.)   \  ($S$ es  sim\'etrica  definida  positiva  casi siempre,  o  con
  probabilidad uno) y \ $S \,\sim \, W_d(V,n)$.
\end{lema}
%
Este resultado  permite tambi\'en probar  el lema~\ref{Lem:MP:StabilidadWishart}
para \ $\nu = n$ \ entero  escribiendo \ $X \egald \sum_{i=1}^n X_i X_i^t$ \ tal
que \ $A^t X A \egald \sum_{i=1}^n A^t X_i X_i^t A = \frac1n \sum_{i=1}^n \left(
  A^t X_i \right)  \left( A^t X_i \right)^t$ \  y notando que los \  $A^t X_i \,
\sim \,  \N(0, A^t V  A)$ \ son independientes~\cite{Seb04}.   Adem\'as, permite
re-obtener las  expreciones del promedio y de  las covarianzas~\footnote{Para la
  covarianza, su usa la formula \ $\Esp[Y_1 Y_2 Y_3 Y_4] = \Esp[Y_1 Y_2]\Esp[Y_3
  Y_4]  + \Esp[Y_1 Y_3]\Esp[Y_2  Y_4] +  \Esp[Y_1 Y_4]\Esp[Y_2  Y_3]$ \  para $Y
  = \begin{bmatrix}  Y_1 & Y_2 &  Y_3 & Y_4 \end{bmatrix}^t$  \ vector gausiano,
  formula que se  obtiene por ejemplo a partir  de la funci\'on caracter\'istica
  de un vecor  gausiano.}. Notar que cuando los \ $X_i$  \ tienen un promemedio,
el  lema conduce  a  lo que  es  conocido como  Wishart no  central~\cite{And03,
  Seb04}.

\SZ{Que propedad mas? Ver Gupta Nagar 1999}

La distribuci\'on  Wishart aparece as\'i naturalmente en  problema de inferencia
Bayesiana  como distribuci\'on  a  priori conjugado~\footref{Foot:MP:BayesPrior}
del par\'ametro $p$ de la ley gaussiana multivariada~\cite{Rob07}.

\SZ{Y donde aparece mas?}


% --------------------------------- Beta
%\subsubseccion{Distribuci\'on beta}
\label{Sssec:MP:Beta}

Estas   distribuciones  fueron   popularizadas   por  Pearson   en  los   a\~nos
1895~\cite{Pea95, Pea16, DavEdw01}  bajo la denominaci\'on Pearson tipo  I en su
estudio  de  la teoria  de  la evoluci\'on  y  la  modelizaci\'on con  variables
asim\'etricas.  De  hecho, apareci\'o mucho  tiempo antes, en trabajos  de Bayes
publicado en un papel postumo  por R. Price en 1763~\cite{Bay63}. Aparentemente,
la denominaci\'on  estandar ``beta'' es  debido al estad\'istico,  dem\'ografo y
soci\'ologo  italiano   C.   Gini   en  1911  su   estudio  del   ``sex  ratio''
(desequilibrio  entre los  nacimientos  de muchachos/muchachas)  con un  enfoque
bayesiano~\cite{Gin11,  For17,   DavEdw01}.   La  distribuci\'on   beta  aparece
precisamente,   entre  otros,   en   problema  de   inferencia  bayesiana   como
distribuci\'on   a   priori   conjugado   del   par\'ametro  $p$   de   la   ley
binomial~\cite{Rob07}     (ver     notas     de     pie~\ref{Foot:MP:BayesPrior}
y~\ref{Foot:MP:BayesPriorConjugado}).

Se denota $X  \sim \beta(a,b)$ \ con  \ $(a,b) \in \Rset_+^{* \,  2}$ \ llamados
{\em par\'ametros de forma}.  Las caracter\'isticas son:

\begin{caracteristicas}
%
Dominio de definici\'on & $\X = [0 \; 1]$\\[2mm]
\hline
%
Par\'ametros & $(a,b) \in \Rset_+^{* \, 2}$ (forma)\\[2mm]
\hline
%
Densidad   de    probabilidad   &   $\displaystyle    p_X(x)   =   \frac{x^{a-1}
(1-x)^{b-1}}{B(a,b)}$\\[2mm]
\hline
%
Promedio & $\displaystyle m_X = \frac{a}{a+b}$\\[2mm]
\hline
%
Varianza &  $\displaystyle \sigma_X^2  = \frac{a b}{(a  + b)^2  (a + b  + 1)}$\\[2mm]
\hline
%
\modif{Asimetr\'ia} & $\displaystyle \gamma_X = \frac{2 \, (b - a) \sqrt{a + b + 1}}{( a
+ b + 2) \sqrt{a b}}$\\[2mm]
\hline
%
Curtosis por exceso & $\displaystyle \widebar{\kappa}_X = \frac{6 \, \left( (a - b)^2 (a + b + 1) - a
b (a  + b  + 2)  \right)}{a \, b  \left( a  + b  + 2 \right)  \left( a  + b  + 3
\right)}$\\[2mm]
\hline
%
Generadora de momentos & $\displaystyle M_X(u)  = \hypgeom{1}{1}\left( a , a + b
\, ; \, u \right)$ \ para \ $u \in \Cset$\\[2mm]
\hline
%
Funci\'on     caracter\'istica     &     $\displaystyle     \Phi_X(\omega)     =
\hypgeom{1}{1}\left( a , a + b \, ; \, \imath \omega \right)$
\end{caracteristicas}

Unas  densidades de  probabilidad  y funciones  de  repartici\'on asociadas  son
representadas en la figura Fig.~\ref{Fig:MP:Beta} para varios $a$ \ y \ $b$.
%
\begin{figure}[h!]
\begin{center} \begin{tikzpicture}%[scale=.9]
\shorthandoff{>}
%
\pgfmathsetmacro{\sx}{5};% x-scaling
\pgfmathsetmacro{\b}{7};% tercera eleccion de (3,beta)
\pgfmathsetmacro{\r}{.05};% radius arc non continuity F_X
%\pgfmathsetmacro{\mb}{max(2,\b*(\b+1)*((\b-1)^(\b-1))/((\b-2)^\b))};% maximo pdf para alpha = 2
\pgfmathsetmacro{\mb}{max(3,2*\b*(\b+2)/(\b+1)*(((\b-1)/(\b+1))^(\b-1))};% maximo pdf para alpha = 3
%
%
% densidad
\begin{scope}
%
%\pgfmathsetmacro{\sy}{2.5*((\b/(\b-1))^(\b-1))/(\b+1)};% y-scaling 
\pgfmathsetmacro{\sy}{2.5/\mb};% y-scaling
\draw[>=stealth,->] (-.5,0)--({\sx+.5},0) node[right]{\small $x$};
\draw[>=stealth,->] (0,-.1)--(0,3) node[above]{\small $p_X$};
%
%\draw[thick] (-.25,0)--(0,0);\draw (\r,\r) arc (90:270:\r);
%\draw[thick] (\sx,0)--({\sx+.25},0);\draw ({\sx-\r},{-\r}) arc (-90:90:\r);
% (a,b) = ( .5 , .5 )
\draw[thick,dotted,domain=.01:.99,samples=100] plot ({\x*\sx},{\sy*1/(pi*sqrt(\x*(1-\x)))});
% (a,b) = ( 2 , 1 )
%\draw[thick,dashed,domain=0:1,samples=100] plot ({\x*\sx},{\sy*2*\x}) node[scale=.4]{$\bullet$};
%\draw[dotted] (\sx,{2*\sy})--(\sx,0);
% (a,b) = ( 3 , 1 )
\draw[thick,dash dot dot,domain=0:1,samples=100] plot ({\x*\sx},{\sy*3*(\x^2)}) node[scale=.4]{$\bullet$};
\draw[dotted] (\sx,{3*\sy})--(\sx,0);
% (a,b) = ( 2 , 2 )
%\draw[thick,dash dot,domain=0:1,samples=100] plot ({\x*\sx},{\sy*6*\x*(1-\x)});
% (a,b) = ( 3 , 2 )
\draw[thick,dash dot,domain=0:1,samples=100] plot ({\x*\sx},{\sy*12*(\x^2)*(1-\x)});
% (a,b) = ( 3 , 3 )
\draw[thick,dashed,domain=0:1,samples=100] plot ({\x*\sx},{\sy*30*(\x^2)*((1-\x)^2)});
% (a,b) = ( 2 , b_3 )
%\draw[thick,domain=0:1,samples=100] plot ({\x*\sx},{\sy*\b*(\b+1)*\x*((1-\x)^(\b-1))});
% (a,b) = ( 3 , b_4 )
\draw[thick,domain=0:1,samples=100] plot ({\x*\sx},{\sy*.5*\b*(\b+1)*(\b+2)*(\x^2)*((1-\x)^(\b-1))});
%}
%
\draw (0,{\sy*2/pi})--(-.1,{\sy*2/pi}) node[left,scale=.7]{$\frac2\pi$};
\draw (0,{\sy*3})--(-.1,{\sy*3}) node[left,scale=.7]{$3$};
\draw (0,{\sy*15/8})--(-.1,{\sy*15/8}) node[left,scale=.7]{$\frac{15}{8}$};
%\draw (0,{\sy*1.5})--(-.1,{\sy*1.5}) node[left,scale=.7]{$\frac32$};
%\draw (0,{\sy*(\b+1)*((1-1/\b)^(\b-1))})--(-.1,{\sy*(\b+1)*((1-1/\b)^(\b-1))}) node[left,scale=.7]{$6 \left(\frac{4}{5} \right)^4$};
%\draw (\sx,0)--(\sx,-.1) node[below,scale=.7]{$1$};
%
\end{scope}
%
%
% reparticion
\begin{scope}[xshift=8.5cm]
%
\pgfmathsetmacro{\sy}{2.5};% y-scaling 
%
\draw[>=stealth,->] (-.75,0)--({\sx*1.25+.25},0) node[right]{\small $x$};
\draw[>=stealth,->] (0,-.1)--(0,{\sy+.25}) node[above]{\small $F_X$};
%
% cumulativa
\draw[thick] (-.5,0)--(0,0); \draw[thick] (\sx,\sy)--({\sx*1.25},\sy);
% (a,b) = ( .5 , .5 )
\draw[thick,dotted,domain=0:1,samples=100] plot ({\x*\sx},{\sy*(.5+asin(2*\x-1)/180)});
% (a,b) = ( 2 , 1 )
%\draw[thick,dashed,domain=0:1,samples=100] plot ({\x*\sx},{\sy*(1-(1+\x)*(1-\x))});
% (a,b) = ( 3 , 1 )
\draw[thick,dash dot dot,domain=0:1,samples=100] plot ({\x*\sx},{\sy*(1-(1+\x+\x^2)*((1-\x)^2))});
% (a,b) = ( 2 , 2 )
%\draw[thick,dash dot,domain=0:1,samples=100] plot ({\x*\sx},{\sy*(1-(1+2*\x)*((1-\x)^2))});
% (a,b) = ( 3 , 2 )
\draw[thick,dash dot,domain=0:1,samples=100] plot ({\x*\sx},{\sy*(1-(1+2*\x+3*\x^2)*((1-\x)^2))});
% (a,b) = ( 2 , b_3 )
%\draw[thick,domain=0:1,samples=100] plot ({\x*\sx},{\sy*(1-(1+\b*\x)*((1-\x)^\b))});
% (a,b) = ( 3 , b_4 )
\draw[thick,domain=0:1,samples=100] plot ({\x*\sx},{\sy*(1-(1+\b*\x+.5*\b*(\b+1)*\x^2)*((1-\x)^\b))});
%
\draw (0,\sy)--(-.1,\sy) node[left,scale=.7]{$1$};
\draw (\sx,0)--(\sx,-.1) node[below,scale=.7]{$1$};
\end{scope}
%
\end{tikzpicture} \end{center}
%
\leyenda{Ilustraci\'on de una densidad de  probabilidad beta (a), y la funci\'on
de  repartici\'on asociada (b).   $(a,b) =  (0.5 \,  , \,  0.5)$ (linea
punteada), $(3 \, , \, 1)$ (linea  mixta doble punteada), $(3 \, , \, 2)$ (linea
mixta), $(3 \, , \, 3)$ (linea
guionada), $(3 \, , \, 7)$ (linea llena).}
\label{Fig:MP:Beta}
\end{figure}

Notar que  se recupera la  ley uniforme sobre  \ $[0 \;  1]$ \ para  \ $a =  b =
1$. Se  conoce la ley de $  Y = 2 \,  B - 1$ \  con \ $B \,  \sim \, \beta\left(
  \frac12 , \frac12 \right)$ \ como {\em ley arcseno}.

Variables beta  tienen tambi\'en unas propiedades notables.  Primero, por cambio
de variables, se demuestra el lema siguiente:
%
\begin{lema}[Reflexividad]
\label{Lem:MP:ReflexividadBeta}
%
  Sea \ $X \, \sim \, \beta(a,b)$. Entonces
  %
  \[
  1-X \, \sim \, \beta(b,a)
  \]
  %
\end{lema}


\begin{lema}[Un v\'inculo con la ley exponencial]
\label{Lem:MP:VinculoBetaExponencial}
%
  Sea  \   $X  \,  \sim  \,   \beta(a,1)$. Entonces
  %
  \[
  - \log X \, \sim \, \E(a)
  \]
  %
\end{lema}
%
\begin{proof}
  El   resultado  es   inmediato  de   la  f\'ormula   de   transformaci\'on  del
  corolario~\ref{Cor:MP:TransformacionInyectivaDensidadEscalar}.
\end{proof}


\begin{lema}[Un v\'inculo con la ley uniforme]
\label{Lem:MP:VinculoBetaUniforme}
%
  Sea  \   $X  \,  \sim  \,   \U([0 \; 1])$ \ y \ $a > 0$. Entonces
  %
  \[
  U^{\frac{1}{a}} \, \sim \, \beta(a,1)
  \]
  %
\end{lema}
%
\begin{proof}
  El   resultado  es   inmediato  de   la  f\'ormula   de   transformaci\'on  del
  corolario~\ref{Cor:MP:TransformacionInyectivaDensidadEscalar}.
\end{proof}

\begin{lema}[Un v\'inculo con la ley gamma]
\label{Lem:MP:VinculoBetaGamma}
%
  Sea  \   $X  \,  \sim  \,   \G(a,c)$  \  e  \   $Y  \,  \sim   \,  \G(b,c)$  \
  independientes. Entonces
  %
  \[
  \frac{X}{X+Y} \, \sim \, \beta(a,b)
  \]
  %
  (independientemente  de $c$).   Adem\'as, $\frac{X}{X+Y}$  \ y  \ $X+Y$  \ son
  independientes.
\end{lema}
%
\begin{proof}
  La independencia de \ $c$ \ es obvia del hecho de que para cualquier $\theta >
  0, \: \theta^{-1} X \, \sim \, \G(a,\theta c)$ \ e \ $\theta^{-1} Y \, \sim \,
  \G(b,\theta  c)$,  la independencia  con  respeto  a \  $c$  \  viniendo de  \
  $\frac{\theta^{-1}     X}{\theta^{-1}     X     +     \theta^{-1}     Y}     =
  \frac{X}{X+Y}$.  Entonces, se  puede considerar  \ $c  = 1$  \ sin  perdida de
  generalidad. Ahora, sea la transformaci\'on
  %
  \[
  \begin{array}{lccl}
    g\ : & \Rset_+^2 & \mapsto & [0 \; 1] \times \Rset_+\\[1.5mm]
    %
    & (x,y) & \to & (u,v) = \left( \frac{x}{x+y} \, , \, x+y \right)
  \end{array}
  \]
  %
  Entonces, la transformaci\'on inversa se escribe
  %
  \[
  g^{-1}(u,v) = \left( u v \, , \, (1-u) v \right)
  \]
  %
  de matriz Jacobiana
  %
  \[
  \Jac_{g^{-1}} = \begin{bmatrix} v & u \\[2mm] -v & 1-u \end{bmatrix}
  \]
  %
  Del          teorema          de          cambio         de          variables
  teorema~\ref{Teo:MP:TransformacionBiyectiva},      notando     que     $\left|
    \Jac_{g^{-1}} \right| =  v$ \ y de la  independencia de \ $X$ \ e  \ $Y$, se
  obtiene para el vector aleatorio \ $W = \begin{bmatrix} U & V \end{bmatrix}^t$
  \ la densidad de probabilidad, definida sobre $[0 \; 1] \times \Rset_+$, como
  %
  \begin{eqnarray*}
    p_W(u,v) & = & p_X( u v ) \, p_Y( (1-u) v ) \, v\\[2mm]
    %
    & = & \frac{\left( u v \right)^{a-1} \, e^{- u v}}{\Gamma(a)} \times
    \frac{\left( (1-u) v \right)^{b-1} \, e^{- (1-u) v}}{\Gamma(b)} \times v\\[2mm]
    %
    & = & \frac{u^{a-1} (1-u)^{b-1}}{B(a,b)} \times \frac{v^{a+b-1} e^{-v}}{\Gamma(a+b)}
  \end{eqnarray*}
  %
  Inmediatamente, factorizandose, aparece  claramente que \ $U$ \ y  \ $V$ \ son
  independientes. Adem\'as, se reconoce en  el primer factor la densidad beta de
  par\'ametros $(a,b)$.   Pasando, se recupera  el hecho que  \ $X+Y \,  \sim \,
  \G(a+b,1)$.
\end{proof}

\begin{lema}[Stabilidad por producto]
\label{Lem:StabilidadBeta}
%
  Sea  \  $X \,  \sim  \, \beta(a,b)$  \  e  \ $Y  \,  \sim  \, \beta(a+b,c)$  \
  independientes. Entonces
  %
  \[
  X Y \, \sim \, \beta(a,b+c)
  \]
  %
\end{lema}
%
\begin{proof}
  Sean \ $U  \, \sim \, \G(a,1)$, \ $V \,  \sim \, \G(b,1)$ \ y \  $W \, \sim \,
  \G(c,1)$   \   independientes  y   sean   \   $X   =  \frac{U}{U+V}$,   $Y   =
  \frac{U+V}{U+V+W}$  \ y  \ $Z  = U+V+W$.  Del lema  anterior \  $X \,  \sim \,
  \beta(a,b)$ \ y \ $Y \, \sim \, \beta(a+b,c)$. Sea la transformaci\'on
  %
  \[
  \begin{array}{lccl}
    g\ : & \Rset_+^3 & \mapsto & [0 \; 1]^2 \times \Rset_+\\[1.5mm]
    %
    & (u,v,w) & \to & (x,y,z) = \left( \frac{u}{u+v} \, , \, \frac{u+v}{u+v+w} \, , \, u+v+w \right)
  \end{array}
  \]
  %
  Entonces, la transformaci\'on inversa se escribe
  %
  \[
  g^{-1}(x,y,z) = \left( x y z \, , \, (1-x) y z \, , \, z (1-y) \right)
  \]
  %
  de matriz Jacobiana
  %
  \[
  \Jac_{g^{-1}} = \begin{bmatrix}
  %
    y z  &   x z   &   x y   \\[2mm]
  %
  - y z  & (1-x) z & (1-x) y \\[2mm]
  %
    0    &  - z    &  1-y
  %
  \end{bmatrix}
  \]
  %
  De      nuevo,      del      teorema      de     cambio      de      variables
  teorema~\ref{Teo:MP:TransformacionBiyectiva}, notando que $\left| \Jac_{g^{-1}}
  \right| = y  z^2$ \ y de la independencia  de \ $U, V, W$,  se obtiene para el
  vector aleatorio  \ $  T =  \begin{bmatrix} X &  Y &  Z \end{bmatrix}^t$  \ la
  densidad de probabilidad probabilidad
  %
  \begin{eqnarray*}
    p_T(x,y,z) & = & p_u( x y z ) \, p_V( (1-x) y z ) \, p_W( y (1-z) ) \, y z^2\\[2mm]
    %
    & = & \frac{\left( x y z \right)^{a-1} \, e^{- x y z}}{\Gamma(a)} \times
    \frac{\left( (1-x) y z \right)^{b-1} \, e^{- (1-x) y z}}{\Gamma(b)} \times
    \frac{\left( z (1-y) \right)^{c-1} \, e^{- z (1-y)}}{\Gamma(c)} \times y
    z^2\\[2mm]
    %
    & = & \frac{x^{a-1} (1-x)^{b-1}}{B(a,b)} \times \frac{y^{a+b-1}
      (1-y)^{c-1}}{B(a+b,c)} \times \frac{z^{a+b+c-1} e^{-z}}{\Gamma(a+b+c)}
  \end{eqnarray*}
  %
  Eso proba  que \  $X, \ Y$  \ y  $Z$ \ son  independientes (las  densidades se
  factorizan). Adem\'as,
  %
  \[
  X  Y =  \frac{U}{U+V} \times  \frac{U+V}{U+V+W} =  \frac{U}{U+V+W} \,  \sim \,
  \beta(a,b+c)
  \]
  %
  el      \'ultimo       resultado      como      consecuencia       de      los
  lemas~\ref{Lem:MP:VinculoBetaGamma}    y~\ref{Lem:MP:StabilidadGamma}.     Eso
  cierra la prueba.
\end{proof}


\begin{lema}[Ley gamma como caso l\'imite de beta]
\label{Lem:GamaLimiteBeta}
%
  Sea \ $X_n \, \sim \, \beta(a,n)$. Entonces
  %
  \[
  n X_n \limitd{n \to +\infty} X \, \sim \, \G(a,1)
  \]
  %
  con \ $\displaystyle \limitd{}$ \ l\'imite es en distribuci\'on
\end{lema}
%
\begin{proof}
  De la f\'ormula de transformaci\'on tenemos la distribuci\'on de $n X_n$
  %
  \begin{eqnarray*}
  p_{n X_n}(x) & = & \frac{1}{n} , \frac{\left( \frac{x}{n} \right)^{a-1} \left( 1
  - \frac{x}{n} \right)^{n-1}}{B(a,n)} \, \un_{(0 \; 1)}\left( \frac{x}{n} \right)\\[2mm]
  %
  & = & \frac{x^{a-1}}{\Gamma(a)} \: \frac{\Gamma(n+a)}{n^a \Gamma(n)} \: \left( 1 -
  \frac{x}{n} \right)^{n-1} \: \un_{(0 \; n)(x)}
  \end{eqnarray*}
  %
  El resultado sigue  notando que \ $\un_{(0 \;  n)} \to \un_{\Rset_{0,+}}$, \quad
  $\left(  1 -  \frac{x}{n} \right)^{n-1}  \to e^{-x}$  \ y  de la  f\'ormula de
  Stirling (ver secci\'on~\ref{Sssec:MP:Poisson}).
\end{proof}

\

La distribuci\'on beta  se generaliza al caso matriz-variada  $X$ definido sobre
$\X$ tal que $X$ y $I-X$ partenecen  a $\Pos_d^+(\Rset)$; se denota \ $X \, \sim
\, \beta_d(a,b)$ \ donde  \ $(a,b) \in \Rset_+^{* \, 2}$ y  la densidad est dada
por   $\displaystyle  p_X(x)   =  \frac{|x|^{a   -  \frac{d+1}{2}}   |I-x|^{b  -
    \frac{d+1}{2}}}{B_p\left([a  \quad  b]^t\right)},  \quad  (a,b)  \in  \left(
  \frac{d-1}{2} \; +\infty \right)^2$. Se refiera a~\cite[Cap.~5]{GupNag99} para
tener m\'as detalles.   Notar que esta distribuci\'on cae en  en una clase dicha
el\'iptica,      que      vamos       a      ver      brevemente      en      la
secci\'on~\ref{Ssec:MP:FamiliaElipticaMatriz},  as\'i  que  propiedades en  este
marco general.



% --------------------------------- Dirichlet
%\subsubseccion{Distribuci\'on de Dirichlet}
\label{Sssec:MP:Dirichlet}

Esta distribuci\'on  tiene su nombre de  integrales on a  simplex estudiados por
M. Lejeune-Dirichlet  y J. Liouville  en 1839~\cite{GupRic01, Dir39,  Lio39}. Es
una extensi\'on  multivariada de las variables  beta a veces  conocida como {\em
  beta multivariada}~\cite{OlkRub64}. Escribiendo  la forma de la distribuci\'on
solamente  con  la variables  $x_i$,  la  integral  permitiendo normalizarla  es
precisamente la estudiada por Lejeune-Dirichlet y Liouville.

Se nota \ $X  \, \sim \, \Dir(a)$ \ con  \ $a \in \Rset_+^{* \, k}$ \  y \ $X$ \
vive  sobre  el  $(k-1)$-simplex   estandar  \  $\Simp{k-1}$.   $a$  es  llamado
par\'ametro de forma. Como en el caso de vectores de distribuci\'on multinomial,
a pesar de que se escribe \ $X$ \ de manera $k$-dimensional, el vector partenece
a una variedad \ $d = k-1$ \ dimensional y en el caso \ $k = 2$ \ se recupera la
ley beta. A veces  se parametriza la ley con un par\'ametro  escalar \ $\alpha >
0$ \ y un vector del simplex estandar \ $\bar{a} \in \Simp{k-1}$ \ tal que
%
\[
a  = \alpha \bar{a},  \quad \mbox{\ie}  \quad \alpha  = \sum_{i=1}^k  a_i, \quad
\bar{a} = \frac{a}{\alpha}
\]
%
$\alpha$ \  es conocido como  par\'ametro de {\em  concentraci\'on} y el  vector \
$\bar{a}$ \ como {\em medida de base}.

Las caracter\'isticas de un vector de Dirichlet son:

\begin{caracteristicas}
%
Dominio de definici\'on &
$\X = \Simp{k-1}, \: k \in \Nset \setminus \{ 0 \; 1 \}$\\[2mm]
\hline
%
Par\'ametros & $a = \alpha \, \bar{a} \, \in \, \Rset_+^{* \, k}$ \ (forma) \ con
\ $\alpha \in \Rset_{0,+}$ \ (concentraci\'on) y \ $\bar{a} \in \Simp{k-1}$
(medida de base)\\[2mm]
\hline
%
Densidad de probabilidad~\footnote{La densidad de probabilidad es dada con
respeto a la medida de Lebesgue restricta al simplex $\Simp{k-1}$.\label{Foot:MP:DirichletDensidad}} & $\displaystyle
p_X(x) = \frac{\prod_{i=1}^k x_i^{a_i-1}}{B(a)}$\\[2mm]
\hline
%
Promedio & $\displaystyle m_X = \bar{a}$\\[2.5mm]
%\frac{a}{\sum_{i=1}^k a_i} \equiv \overline{a}$\\[2.5mm]
\hline
%
Covarianza~\footnote{Ver nota de pie~\ref{Foot:MP:MultinomialCov}.} &
$\displaystyle \Sigma_X = \frac{\diag\left( \bar{a} \right) - \bar{a}
\bar{a}^t}{1 + \alpha}$\\[2.5mm]
\hline
%
Generadora de momentos  &
$\displaystyle M_X(u) = \Phi_2^{(k)}( a , \alpha \, ; \, u )$ \ para \ $u \in
\Cset$\\[2mm]
\hline
%
Funci\'on caracter\'istica & $\displaystyle
\Phi_X(\omega) = \Phi_2^{(k)}( a , \alpha \, ; \, \imath \omega )$
\end{caracteristicas}

%$k$-variada~\footnote{$\Phi_2^{(k)}(a;b;z)     =     \sum_{m    \in     \Nset^k}
%  \frac{(a_1)_{(m_1)}     \ldots    (a_k)_{(m_k)}     \,     z_1^{m_1}    \ldots
%    z_k^{m_k}}{(b)_{(m_1+\cdots+m_k)} m_1!  \ldots m_k!}$  \ con \ $(x)_{(n)}$ \
%  s\'imbolo  de Pochhammer  usual o  factorial creciente,  $(x)_{(n)} =  x (x+1)
%  \ldots (x+n)$ \,  con la convenci\'ion \ $(x)_{(0)} = 1$.   De hecho, la forma

De nuevo, se puede considerar que el vector aleatorio es \ $(k-1)$-dimensional \
$\widetilde{X}     =    \begin{bmatrix}     \widetilde{X}_1    &     \cdots    &
  \widetilde{X}_{k-1}  \end{bmatrix}^t$  \ definido  sobre  el hipertriangulo  \
$\widetilde{\X}  = \Tri_{k-1}  = \left\{  \widetilde{x} \in  [0 \;  1]^{k-1} \tq
  \sum_{i=1}^{k-1}  \widetilde{x}_i \le  1 \right\}$,  proyecci\'on  del simplex
sobre el  hiperplano \ $x_k =  0$. As\'i, \ $\widetilde{X}$,  tiene una densidad
con   respeto   a   la  medida   de   Lebesgue   usual,   y   es  dada   por   \
$p_{\widetilde{X}}\left(   \widetilde{x}   \right)   =   \frac{\prod_{i=1}^{k-1}
  \widetilde{x}_i^{\,  a_i-1}  \, \left(  1  - \sum_{i=1}^{k-1}  \widetilde{x}_i
  \right)^{a_k-1}}{B(a)}$.      A      final,     se     notar\'a      que     \
$\Phi_{\widetilde{X}}\left(         \widetilde{\omega}         \right)         =
\Phi_X\left( \begin{bmatrix} \widetilde{\omega} & 0 \end{bmatrix}^t \right)$ \ y
\ $\Phi_X(u) =  e^{\omega_k} \Phi_{\widetilde{X}}\left( \begin{bmatrix} \omega_1
    - \omega_k  & \cdots &  \omega_{k-1} - \omega_k \end{bmatrix}^t  \right)$ (y
similarmente  para $G_X$  con respeto  a $G_{\widetilde{X}}$).   Notar tambi\'en
que, la  forma de la  funci\'on generadora de  momento viene directamente  de la
escritura de las series  de Taylor de $e^{u_i x_i}$ \ o  de la forma integral de
la funci\'on confluente hipergeom\'etrica~\cite{Phi88}.

Naturalmente,  $\Sigma_X \un  = 0$  \  as\'i que  de nuevo  \ $\Sigma_X  \notin
\Pos_k^+(\Rset)$, como consecuencia directa del  hecho que \ $X$ \ $k$-dimensional,
vive  sobre   \  $\Simp{k-1}$,  $(k-1)$-dimensional.  De   nuevo,  para  definir
asimetr\'ia y  curtosis habr\'ia que  considerar \ $\widetilde{X}$,  de promedio
$\begin{bmatrix}  \bar{a}_1  &  \cdots  & \bar{a}_{k-1}  \end{bmatrix}^t$  y  de
covarianza  el  bloque  $(k-1)  \times   (k-1)$  de  $\Sigma_X$,  que  es  ahora
invertible. $\gamma_{\widetilde{X}}$  \ y \ $\kappa_{\widetilde{X}}$  \ son bien
definidos. Las expresiones, demasiado pesadas, no son dadas ac\'a.

La figura Fig.~\ref{Fig:MP:Dirichlet} representa  el dominio de definici\'on del
vector (a) y la densidad de  probabildad con las marginales (ver m\'as adelante)
para $k = 3$ y dos ejemplos de par\'ametro $a$.
%
\begin{figure}[h!]
\begin{center} \begin{tikzpicture}%[scale=.8]
\shorthandoff{>}
%
\tikzset{declare function={
xplus(\x) = max(\x,0);
%ifthenelse(\x > 0 , \x , NaN);
}}
%}

% Simplex
\tdplotsetmaincoords{45}{65}
\begin{scope}[tdplot_main_coords,scale=.75]
%
% Dirichlet: \X = S_{k-1} y \widetilde{X}
\pgfmathsetmacro{\dx}{3};% scaling
%
\draw[->,>=stealth] (-.25,0,0)--({\dx+.5},0,0) node[below right,scale=.9]{$x_1$};
%\node at (\dx,0,0)[left,scale=.8]{$1$};
\draw (\dx,0,0)--(\dx,-.15,0) node[left,scale=.8]{$1$};
%
\draw[->,>=stealth] (0,-.25,0)--(0,{\dx+.5},0) node[right,scale=.9]{$x_2$};
%\node at (0,\dx,0)[below,scale=.8]{$1$};
\draw (0,\dx,0)--(.15,\dx,0) node[below,scale=.8]{$1$};
%
\draw[->,>=stealth] (0,0,-.25)--(0,0,{\dx+.5}) node[above,scale=.9]{$x_3$};
%\node at (0,0,\dx)[left,scale=.8]{$1$};
\draw (0,0,\dx)--(0,-.15,\dx) node[left,scale=.8]{$1$};
%
\node at (0,0,0)[below left,scale=.8]{$0$};
%
% tilde X
\filldraw[fill=black!50,opacity=.5] (0,0,0)--(\dx,0,0)--(0,\dx,0);
\draw[thick,color=black,dashed] (0,0,0)--(\dx,0,0)--(0,\dx,0)--(0,0,0);
\node at ({\dx/15},{\dx/20},0)[right,scale=.7]{$\Tri_2$};
%
% Simplex Delta_2
\filldraw[fill=black!75,opacity=.5] (\dx,0,0)--(0,\dx,0)--(0,0,\dx);
\draw[thick,color=black] (\dx,0,0)--(0,\dx,0)--(0,0,\dx)--(\dx,0,0);
\node at ({.05*\dx},{.05*\dx},{.8*\dx})[right,scale=.7]{$\Simp{2}$};
%
\end{scope}
%
%
% densidad (3,2,2)
\begin{scope}[xshift=4cm,yshift=-2cm,scale=.75]
%
\pgfmathsetmacro{\au}{3};% a1
\pgfmathsetmacro{\ad}{2};% a2
\pgfmathsetmacro{\at}{2};% a3
\pgfmathsetmacro{\B}{factorial(\au-1)*factorial(\ad-1)*factorial(\at-1)/factorial(\au+\ad+\at-1)};% normalizacion
\pgfmathsetmacro{\Bu}{factorial(\au-1)*factorial(\ad+\at-1)/factorial(\au+\ad+\at-1)};% normalizacion 1
\pgfmathsetmacro{\Bd}{factorial(\ad-1)*factorial(\au+\at-1)/factorial(\au+\ad+\at-1)};% normalizacion 2
\pgfmathsetmacro{\ma}{((\au-1)^(\au-1))*((\ad-1)^(\ad-1))*((\at-1)^(\at-1))/((\au+\ad+\at-3)^(\au+\ad+\at-3))/\B};
%
% Dirichlet & marginales
\begin{axis}[
    colormap = {whiteblack}{color(0cm)  = (white);color(1cm) = (black)},
    width=.5\textwidth,
    view={45}{65},
    enlargelimits=false,
    %grid=major,
    domain=0:1,
    y domain=0:1,
    %unbounded coords=jump, % para tener un dominio no cuadrado
    %filter point/.code={%
    %\pgfmathparse
    %{\pgfkeysvalueof{/data point/x} + \pgfkeysvalueof{/data point/y} > 1.0}%
    %  \ifpgfmathfloatcomparison
    %     \pgfkeyssetvalue{/data point/x}{nan}%
    %  \fi
    %},
    zmax={.8*\ma},
    color=black,
    samples=70,
    xlabel=$x_1$,
    ylabel=$x_2$,
    zlabel=$p_{\widetilde{X}}$,
]
%
% Dirichlet
\addplot3 [surf] {(x^(\au-1))*(y^(\ad-1))*(xplus(1-x-y)^(\at-1))/\B};
%
% Marginales
\addplot3 [domain=0:1,samples=50, samples y=0, thick, smooth, color=black] (x,1,{(x^(\au-1))*((1-x)^(\ad+\at-1))/\Bu});
\addplot3 [domain=0:1,samples=50, samples y=0, thick, smooth, color=black] (0,x,{(x^(\ad-1))*((1-x)^(\au+\at-1))/\Bd});
%
\node at (axis cs:.5,1,{1/(2^(\au+\ad+\at-2))/\Bu})[right]{$p_{X_1}$};
\node at (axis cs:0,.5,{1/(2^(\au+\ad+\at-2))/\Bd})[above]{$p_{X_2}$};
\end{axis}
\end{scope}
%
%
% densidad (3,2,2)
\begin{scope}[xshift=11cm,yshift=-2cm,scale=.75]
%
\pgfmathsetmacro{\au}{3};% a1
\pgfmathsetmacro{\ad}{1};% a2
\pgfmathsetmacro{\at}{2};% a3
\pgfmathsetmacro{\B}{factorial(\au-1)*factorial(\ad-1)*factorial(\at-1)/factorial(\au+\ad+\at-1)};% normalizacion
\pgfmathsetmacro{\Bu}{factorial(\au-1)*factorial(\ad+\at-1)/factorial(\au+\ad+\at-1)};% normalizacion 1
\pgfmathsetmacro{\Bd}{factorial(\ad-1)*factorial(\au+\at-1)/factorial(\au+\ad+\at-1)};% normalizacion 2
\pgfmathsetmacro{\ma}{((\au-1)^(\au-1))*((\ad-1)^(\ad-1))*((\at-1)^(\at-1))/((\au+\ad+\at-3)^(\au+\ad+\at-3))/\B};
%
\begin{axis}[
    colormap = {whiteblack}{color(0cm)  = (white);color(1cm) = (black)},
    width=.5\textwidth,
    view={45}{65},
    enlargelimits=false,
    %grid=major,
    domain=0:1,
    y domain=0:1,
    zmax={.65*\ma},
    color=black,
    samples=70,
    xlabel=$x_1$,
    ylabel=$x_2$,
    zlabel=$p_{\widetilde{X}}$,
]
%
% Dirichlet
\addplot3 [surf,opacity=.8] {(x^(\au-1))*(y^(\ad-1))*(xplus(1-x-y)^(\at-1))/\B};
%
% Marginales
\addplot3 [domain=0:1,samples=50, samples y=0, thick, smooth, color=black] (x,1,{(x^(\au-1))*((1-x)^(\ad+\at-1))/\Bu});
\addplot3 [domain=0:1,samples=50, samples y=0, thick, smooth, color=black] (0,x,{(x^(\ad-1))*((1-x)^(\au+\at-1))/\Bd});%
%
\node at (axis cs:.5,1,{1/(2^(\au+\ad+\at-2))/\Bu})[right]{$p_{X_1}$};
\node at (axis cs:0,.5,{1/(2^(\au+\ad+\at-2))/\Bd})[above]{$p_{X_2}$};
\end{axis}
\end{scope}
%
\node at (1.2,-3){(a)};
\node at (6.6,-3){(b)};
\node at (13.6,-3){(c)};
\end{tikzpicture} \end{center}
%
\leyenda{Ilustraci\'on  del dominio $\Simp{k-1}$  de definici\'on  de la  ley de
  Dirichlet   para  \   $k  =   3$   \  (grise   oscuro),  con   el  dominio   \
  $(k-1)$-dimensional   \  $\Tri_{k-1}$   \  del   vector  \   $\widetilde{X}  =
  \protect\begin{bmatrix}   X_1  &  X_2   \protect\end{bmatrix}^t$  \   ($X_3  =
  1-X_1-X_2$) \ (grise claro) (a), y densidad de probabilidad de $\widetilde{X}$
  \ con  las marginales \  $p_{X_1}, \: p_{X_2}$.   Los par\'ametros son \  $a =
  \protect\begin{bmatrix}  3 &  2  & 2  \protect\end{bmatrix}^t$  (b) y  \ $a  =
  \protect\begin{bmatrix} 3 & 1 & 2 \protect\end{bmatrix}^t$ (c).}
\label{Fig:MP:Dirichlet}
\end{figure}


Vectores  de  distribuci\'on  de  Dirichlet tienen  tambi\'en  unas  propiedades
notables, parecidas a las de la beta:
%
\begin{lema}[Reflexividad]\label{Lem:MP:ReflexividadDir}
%
  Sea  \ $X \,  \sim \,  \Dir(a), \:  a \in  \Rset_+^{* \,  k}$ \  y \  $\Pi \in
  \perm_k(\Rset)$ \ matriz \ de permutaci\'on. Entonces
  %
  \[
  \Pi X \, \sim \, \Dir\left( \Pi a \right)
  \]
  %
\end{lema}

%
\begin{proof}
  El resultado es inmediato por cambio  de variables $x \to \Pi x$, la Jacobiana
  siendo   $\Pi$,   de   valor   absoluto   determinente  igual   a   $1$   (ver
  secci\'on~\ref{Sec:MP:Transformacion}).
\end{proof}
%
Adem\'as, se muestra una stabilidad reemplazando dos componentes por su suma:
%
\begin{lema}[Stabilidad por agregaci\'on]\label{Lem:MP:StabSumaDir}
%
  Sea  \ $X =  \begin{bmatrix} X_1  & \cdots  & X_k  \end{bmatrix}^t \,  \sim \,
  \Dir(a),  \:  a =  \begin{bmatrix}  a_1 &  \cdots  &  a_k \end{bmatrix}^t  \in
  \Rset_+^{*  \,  k}$  \  y  \  $G^{(i,j)}$ \  matriz  de  agrupaci\'on  de  las
  $(i,j)$-\'esima componentes (ver notaciones). Entonces,
  %
  \[
  G^{(i,j)} X \, \sim \, \Dir\left( G^{(i,j)} a \right)  
  \]
  %
\end{lema}
%
\begin{proof}
  Se  puede probar  este resultado  a partir  de la  funci\'on caracter\'istica,
  usando      las      propiedades      de      la      funci\'on      confluent
  hipergeom\'etrica~\cite{SriKar85, Hum22, App25,  AppKam26, Erd37, Erd40}. Pero
  se  puede tambi\'en  tener un  enfoque m\'as  directo.  Del  lema precediente,
  notando  que  existen  matrices  de permutaci\'on~\footnote{$\Pi_k$  pone  las
    componentes $i$ \ e \ $j$ el  las posiciones $1$ y $2$, sin cambiar el orden
    de  las  siguientes;  $\Pi_{k-1}$  trazlada  la  primera  componente  en  la
    posici\'on $\min(i,j)$.}  \ $\Pi_k \in  \perm_k(\Rset)$ \ y \ $\Pi_{k-1} \in
  \perm_{k-1}(\Rset)$ \ tal que \ $G^{(i,j)} = \Pi_{k-1} \, G^{(1,2)} \, \Pi_k$,
  se puede concentrarse en el caso \ $(i,j) = (1,2)$. Sea el cambio de variables
  $g:    x    =    (x_1,\ldots,x_k)    \mapsto   u    =    (u_1,\ldots,u_k)    =
  (x_1,x_1+x_2,x_3,\ldots,x_k)$.        Entonces       \      $g^{-1}(u)       =
  (u_1,u_2-u_1,u_3,\ldots,u_k)$ \ es de determinente de matriz Jacobiana igual a
  \ $1$ \ dando para $U = g(X)$ \ la densidad
  %
  \[
  p_U(u)  = \frac{u_1^{a_1-1}  \left(  u_2 -  u_1 \right)^{a_2-1}  \prod_{i=3}^k
    u_i^{a_i-1}}{B(a)}
  \]
  %
  sobre $g\left(  \Simp{k-1} \right)$. Para $u_2  \in [0 \; 1]$  \ tenemos $u_1
  \in [  0 \;  u_2]$ \ as\'i  que, por  marginalizaci\'on en $u_1$  obtenemos la
  densidad
  %
  \begin{eqnarray*}
  p_{G^{(1,2)} X}(u_2,\ldots,u_k) & = & \frac{\prod_{i=3}^k u_i^{a_i-1}}{B(a)}
  \int_0^{u_2} u_1^{a_1-1} \left( u_2 - u_1 \right)^{a_2-1} \, du_1\\[2mm]
  %
  & = & \frac{\prod_{i=3}^k u_i^{a_i-1}}{B(a)} \, u_2^{a_1+a_2-1} \int_0^1
  v_1^{a_1-1} \left( 1 - v_1 \right)^{a_2-1} \, dv_1
  \end{eqnarray*}
  %
  con el cambio de variables $u_1 = u_2 v_1$. Se cierra la prueba notando que la
  integral   vale  \  $B(a_1,a_2)$   \  y   que  \   $\frac{B(a_1,a_2)}{B(a)}  =
  \frac{1}{B\left( G^{(1,2)} a \right)}$.
\end{proof}

De este lema, aplicado de manera recursiva, se obtiene en corolario siguiente:
%
\begin{corolario}
\label{Cor:MP:MarginalDirichletBeta}
%
  Sea  \ $X  \,  \sim \,  \Dir(a)$, entonces  \  $\displaystyle X_i  \, \sim  \,
  \beta\left( a_i \, , \, \alpha-a_i \right)$.
\end{corolario}

Naturalmente, la ley de Dirichelt siendo  una extensi\'on de la ley beta, existe
tambi\'en un v\'inculo entre esta ley y variables de distribuci\'on gamma:
%
\begin{lema}[V\'inculo con la ley gamma]
\label{Lem:MP:VinculoDirichletGamma}
%
Sea \ $X$  \ vector $k$-dimensional de componentes \ $X_i  \, \sim \, \G(a_i,c),
\:  i  = 1,  \ldots  , k$  \  independientes  \ y  \  $a$  vector de  componente
$i$-\'esima \ $a_i$. Entonces
  %
  \[
  \frac{X}{\sum_{i=1}^k X_i} \, \sim \, \Dir(a)
  \]
  %
  (independientemente  de $c$).   Adem\'as, $\frac{X}{\sum_{i=1}^k  X_i}$ \  y \
  $\sum_{i=1}^k X_i$ \ son independientes.
\end{lema}
%
\begin{proof}
  La    prueba   sigue    exactamente   los    mismos   pasos    que    la   del
  lema~\ref{Lem:MP:VinculoBetaGamma} \ trabajando con \ $\widetilde{X}$.
\end{proof}

Naturalmente,  la  distribuci\'on de  Dirichlet,  extensi\'on  de  la ley  beta,
aparece entre  otros en problema  de inferencia bayesiana como  distribuci\'on a
priori  conjugado  del  par\'ametro  $p$  de  la  ley  multinomial~\cite{Rob07},
extensi\'on de la ley binomial.

%\SZ{
%Polya urn schemes (ver Ash entre otros) , Chinese restaurant
%}

\

La distribuci\'on de Dirichlet se generaliza al caso matriz-variada $X$ definido
sobre $\P_{d,k}(\Rset)$,  conjunto de las  $k$-uplas de matrices  de $\Pos_d^+(\Rset)$
cumpliando la relaci\'on  de completud (ver notaciones); se denota  \ $X \, \sim
\, \Dir_d(a)$  \ donde \  $a \in \left(  \frac{d-1}{2} \; +\infty  \right)^k$ la
densidad  est dada por  $\displaystyle p_X(x)  = \frac{\prod_{i=1}^k  \left| x_i
  \right|^{a_i-\frac{d+1}{2}}}{B_d(a)}$.   Se  refiera a~\cite[Cap.~6]{GupNag99}
para tener m\'as detalles.


% --------------------------------- Student-t
%\subsubseccion{Distribuci\'on Student-$t$ multivariada}
\label{Sssec:MP:StudentT}

En   el  caso   escalar,  esta   ley   fue  introducida   inicialmente  por   F.
R. Helmert~\cite{Hel75, Hel76, She95}  y J.  L\"uroth~\cite{Lur76, Pfa96}.  Pero
es m\'as  conocida por su  introducci\'on por William  Sealy Gosset~\footnote{De
  hecho, W.  S.  Gosset fue un estudiante trabajando en  la f\'abrica de cerveza
  irlandesa  Guiness sobre  estad\'isticas  relacionadas a  la  qu\'imica de  la
  cerveza.  Hay  varias versiones sobre el  hecho que se  public\'o este trabajo
  bajo  el nombre  ``Student''.  Una  es que  fue  para que  no se  sabe que  la
  f\'abrica  estaba  trabajando  sobre  estas estad\'isticas  para  estudiar  la
  calidad   de   la   cerveza~\cite{Wen16}.\label{Foot:MP:Student}}   en   1908,
trabajando  sobre variables centradas  normalizadas por  el promedio  y varianza
emp\'iricos~\cite{Stu08}.   Fue  estudiada  entre  otros intensivamente  por  el
famoso  matem\'atico R.   Fisher~\cite{Fis25}.  En  la literatura,  esta  ley es
conocida bajo  los nombres {\em  Student}, {\em Student-$t$} o  simplemente {\em
  $t$-distribuci\'on} o  a\'un bajo el nombre  {\em Pearson tipo IV}  en el caso
escalar  y {\em  Pearson tipo  VII}  (para $\frac{\nu+d}{2}$  entero; ver  m\'as
abajo), debido  a la  familia de Pearson~\cite{Pea95,  JohKot95:v1, JohKot95:v1,
  KotBal00, FanKot90}.   Esta distribuci\'on aparece como a  priori conjugado de
la media de una gaussiana en inferencia bayesiana~\cite{Rob07, KotNad04}.

Se denota con \ $X \sim \T_\nu(m,\Sigma)$ \ con \ $m \in \Rset^d$, \ $\Sigma \in
P_d^+(\Rset)$ \ conjunto  de las matrices de \  $\M_{d,d}(\Rset)$ \ sim\'etricas
definidas positivas. $m$ \ es llamado  {\em par\'ametro de posici\'on} (no es la
media   que  puede  no   existir),  \   $\Sigma$  \   es  llamada   {\em  matriz
  caracter\'istica} (no es [proporcional a]  la covarianza que puede no existir)
y \ $\nu > 0$ \ llamado  {\em grado de libertad}.  Las caracter\'isticas de una
Student-$t$ son las siguientes:
%
\begin{caracteristicas}
%
Dominio de definici\'on & $\X = \Rset^d$\\[2mm]
\hline
%
Par\'ametro & $\nu \in \Rset_+^*$ \ (grado de libertad), \ $m \in \Rset^d$ \
(posici\'on), \ $\Sigma \in P_d^+(\Rset)$ \ (matriz caracter\'istica)\\[2mm]
\hline
%
Densidad de probabilidad & $\displaystyle p_X(x) = \frac{\Gamma\left(
\frac{\nu+d}{2} \right)}{\pi^{\frac{d}{2}} \nu^{\frac{d}{2}} \Gamma\left(
\frac{\nu}{2} \right) \, \left| \Sigma \right|^{\frac12}} \, \left( 1 +
\frac{(x-m)^t \Sigma^{-1} (x-m)}{\nu} \right)^{- \, \frac{\nu+d}{2}}$\\[2mm]
\hline
%
Promedio & $\displaystyle m_X = m$ \ si \ $\nu > 1$; \ no
existe si no~\footnote{De manera general, esta ley admite momentos de orden \ $k$ \
si y solamente si \ $\nu > k$.\label{Foot:MP:ExistenciaMomentosStudent}}.\\[2.5mm]
\hline
%
Covarianza~\footnote{Fijense de que $\Sigma$ no es la covarianza, pero es
proporcional a la covarianza\ldots cuando existe. Se podr\'ia imaginar
renormalizar la ley tal que \ $\Sigma_X$ \ y \ $\Sigma$ \ coinciden, pero no
ser\'ia posible en el caso \ $\nu \le 2$.} & $\displaystyle \Sigma_X =
\frac{\nu}{\nu-2} \, \Sigma$ \ si \ $\nu > 2$; \ no existe si
no~\footref{Foot:MP:ExistenciaMomentosStudent}.\\[2.5mm]
\hline
%
\modif{Asimetr\'ia} & $\displaystyle \gamma_X = 0$ \ si \ $\nu > 3$; \ no existe
si no~\footref{Foot:MP:ExistenciaMomentosStudent}.\\[2mm]
\hline
%
Curtosis por exceso & \modif{$\displaystyle \widebar{\kappa}_X = \frac{2}{\nu-4}
\sum_{i,j=1}^d \Big( \! \left(
    \un_i \un_i^t \right) \otimes \left(  \un_j \un_j^t \right) +  \left( \un_i
    \un_j^t \right) \otimes \left( \un_i  \un_j^t \right) + \left( \un_i \un_j^t
  \right) \otimes \left( \un_j \un_i^t \right) \! \Big)$}\newline si \ $\nu >
4$; \ no existe si no~\footref{Foot:MP:ExistenciaMomentosStudent}.\\[2mm]
\hline
%
Funci\'on caracter\'istica~\footnote{Se  muestra sencillamente que  la funci\'on
  generatriz de  momentos puede existir si y  solamente si \ $\real{u}  = 0$. La
  funci\'on generadora de momentos restricta  al producto cartesiano de rectas \
  $\real{u} =  0$ \ es nada  m\'as que la  funci\'on caracter\'istica. Adem\'as,
  esta  funci\'on  fue  calculdada,   especialmente  en  el  caso  multivariado,
  relativamente   recientemente~\cite{Sut86,  Hur95,  KibJoa06,   SonPar14}.}  &
$\displaystyle  \Phi_X(\omega)  = \frac{\nu^{\frac{\nu}{4}}}{2^{\frac{\nu}{2}-1}
  \Gamma\left(  \frac{\nu}{2}  \right)}  \,  e^{\imath  \omega^t  m}  \,  \left(
  \omega^t   \Sigma   \omega   \right)^{\frac{\nu}{4}}   K_{\frac{\nu}{2}}\left(
  \sqrt{\nu \, \omega^t \Sigma \omega} \right)$
\end{caracteristicas}

Nota: nuevamente se puede escribir $X \, \egald \, \Sigma^{\frac12} S + m$ \ con
\ $S \, \sim \, \T_\nu(0,I)$ \  donde \ $S$ \ es dicha {\em Student-$t$ estandar}
y  las caracter\'isticas  de \  $X$  \ son  v\'inculadas a  las  de \  $S$ \  (y
vice-versa) por transformaci\'on lineal (ver secciones anteriores).

Densidades  de probabilidad  Student-$t$ estandar  y funciones  de repartici\'on
asociadas   en    el   caso   escalar    son   representadas   en    la   figura
Fig.~\ref{Fig:MP:StudentT}-(a) y  (b) para  varios $\nu$, y  una densidad  en un
contexto bi-dimensional figura Fig.~\ref{Fig:MP:StudentT}(c).
%
\begin{figure}[h!]
\begin{center} \begin{tikzpicture}
\shorthandoff{>}
%
% Para el caso univariado
\pgfmathsetmacro{\sx}{.43};% x-scaling
\pgfmathsetmacro{\mu}{0};% para tomar los grados de libertad impar; 0 => Cauchy
\pgfmathsetmacro{\md}{1};
\pgfmathsetmacro{\mt}{3};
%\pgfmathsetmacro{\mq}{3};
%
%
% para el caso bi-variado
\pgfmathsetmacro{\mdd}{0};%
\pgfmathsetmacro{\nu}{2*\mdd+1};% grados de libertad
\pgfmathsetmacro{\a}{1/3};% x-scaling
\pgfmathsetmacro{\t}{30};% angulo de rotacion
\pgfmathsetmacro{\c}{cos(\t)};% coseno
\pgfmathsetmacro{\s}{sin(\t)};% seno
\pgfmathsetmacro{\su}{sqrt(\c^2+(\a*\s)^2)};% ecart-type 1
\pgfmathsetmacro{\sd}{sqrt(\s^2+(\a*\c)^2)};% ecart-type 2
\pgfmathsetmacro{\dx}{3};% dominio x del plot -dx:dx
\pgfmathsetmacro{\dy}{2.5};% dominio y del plot -dy:dy
%
%
% Approximacion de la funcion Gamma
%\tikzset{declare function={gamma(\z)=
%(2.506628274631*sqrt(1/\z) + 0.20888568*(1/\z)^(1.5) + 0.00870357*(1/\z)^(2.5) -
%(174.2106599*(1/\z)^(3.5))/25920 - (715.6423511*(1/\z)^(4.5))/1244160)*exp((-ln(1/\z)-1)*\z);}}
%
% Approximation de la cdf gaussienne
\tikzmath{function normcdf(\x) {return 1/(1 + exp(-0.07056*(\x)^3 - 1.5976*(\x)));};};
%
% coefficiente binomial, para no tener factoriales muy grandes
\tikzmath{function binocoef(\m,\k) {if \k == 0 then {return 1;} else {return ((\m-\k+1)/\k)*binocoef(\m,\k-1);};};};
%
% coefficient que aparece en la pdf y cdf (ver doubling formula GraRyz 8.335-5 con x = m+1/2)
% y coefficiente de normalizacion
%\tikzset{declare function={
\tikzmath{function coefstud(\m) {return (4^\m)/(pi*sqrt(2*\m+1)*binocoef(2*\m,\m));};}
%
%
% cdf Student que se calcula recursivamente para nu = 2 m + 1, m entero
\tikzmath{function studcdfS(\x,\k) {
    if \k == 0 then {return .5+(atan(\x))/180;}
    else {return studcdfS(\x,\k-1)+((4^\k)*(\x)/(2*pi*\k*binocoef(2*\k,\k)))/((1+((\x)^2))^\k);};
};};
% Calculo de
%  - x maximo del plot para tener pdf a 7% del max
%  - la pdf Student para nu = 2 m + 1, m entero
%  - la cdf Student para nu = 2 m + 1, m entero
%\tikzset{declare function={
\tikzmath{function maxplotpdf(\m) {return sqrt((2*\m+1)*((.03^(-1/(\m+1)))-1));};};% x maximo del plot para tener pdf a 3% del max
\tikzmath{function studpdf(\x,\m) {return coefstud(\m)*((1/(1+((\x)^2)/(2*\m+1)))^(\m+1));};};% pdf Student
\tikzmath{function studcdf(\x,\m) {return studcdfS(\x/(sqrt(2*\m+1)),\m);};};% pdf Student
%}}
%
%
%
% mismas escalas x-max para cada ejemplo
\pgfmathsetmacro{\mx}{max(maxplotpdf(\mu),maxplotpdf(\md),maxplotpdf(\mt))};

% maximo de las marginales del caso 2D
\pgfmathsetmacro{\ma}{coefstud(\mdd)/min(\su,\sd)};
%
% densidad
\begin{scope}[scale=.9]
%
\pgfmathsetmacro{\sy}{2.75*sqrt(2*pi)};% y-scaling 
\draw[>=stealth,->] ({-\sx*\mx-.1},0)--({\sx*\mx+.25},0) node[right]{\small $x$};
\draw[>=stealth,->] (0,-.15)--(0,3) node[above]{\small $p_X$};
%
\draw[thick,domain=-\mx:\mx,samples=50,smooth] plot ({\x*\sx},{\sy*studpdf(\x,\mu)});
\draw[thick,dashed,domain=-\mx:\mx,samples=50,smooth] plot ({\x*\sx},{\sy*studpdf(\x,\md)});
\draw[thick,dotted,domain=-\mx:\mx,samples=50,smooth] plot ({\x*\sx},{\sy*studpdf(\x,\mt)});
\draw[thin,domain=-\mx:\mx,samples=50,smooth] plot ({\x*\sx},{\sy*exp(-.5*((\x)^2))/sqrt(2*pi)});
%
\draw (0,{\sy/sqrt(2*pi)})--(-.2,{\sy/sqrt(2*pi)}) node[left,scale=.7]{$\displaystyle \frac1{\sqrt{2 \pi}}$};
\draw (0,0)--(0,-.1) node[below,scale=.7]{$0$};
\pgfmathsetmacro{\lm}{2*floor(\mx/2)};
\foreach \m in {2,4,...,\lm} {
\draw ({-\m*\sx},0)--({-\m*\sx},-.1) node[below,scale=.7]{$-\m$};
\draw ({\m*\sx},0)--({\m*\sx},-.1) node[below,scale=.7]{$\m$};
}
%
\node at (0,-1) [scale=.9]{(a)};
\end{scope}
%
%
% reparticion
\begin{scope}[xshift=5.75cm,scale=.9]
%
\pgfmathsetmacro{\extx}{1.1};% extnsion del dominio para la cdf (que se vea mejor) 
\pgfmathsetmacro{\sy}{2.75};% y-scaling 
%
\draw[>=stealth,->] ({-\sx*\mx*\extx-.1},0)--({\sx*\mx*\extx+.25},0) node[right]{\small $x$};
\draw[>=stealth,->] (0,-.15)--(0,{\sy+.25}) node[above]{\small $F_X$};
%
% cumulativa
%
\draw[thick,domain={-\mx*\extx}:{\mx*\extx},samples=50,smooth] plot ({\x*\sx},{\sy*studcdf(\x,\mu)});
\draw[thick,dashed,domain={-\mx*\extx}:{\mx*\extx},samples=50,smooth] plot ({\x*\sx},{\sy*studcdf(\x,\md)});
\draw[thick,dotted,domain={-\mx*\extx}:{\mx*\extx},samples=50,smooth] plot ({\x*\sx},{\sy*studcdf(\x,\mt)});
\draw[thin,domain={max(-\mx*\extx,-3.5)}:{\mx*\extx},samples=50,smooth]  plot ({\x*\sx},{\sy*normcdf(\x)});
%
\draw (0,0)--(0,-.1) node[below,scale=.7]{$0$};
\draw (0,\sy)--(-.1,\sy) node[left,scale=.7]{$1$};
\pgfmathsetmacro{\lm}{2*floor(\mx*\extx/2)};
\foreach \m in {2,4,...,\lm} {
\draw ({-\m*\sx},0)--({-\m*\sx},-.1) node[below,scale=.7]{$-\m$};
\draw ({\m*\sx},0)--({\m*\sx},-.1) node[below,scale=.7]{$\m$};
}
%
\node at (0,-1) [scale=.9]{(b)};
\end{scope}
%
%
% densidad 2D
\begin{scope}[xshift=9.5cm,yshift=-2.5mm,scale=.7]
%
\begin{axis}[
    colormap = {whiteblack}{color(0cm)  = (white);color(1cm) = (black)},
    width=.45\textwidth,
    view={45}{65},
    enlargelimits=false,
    %grid=major,
    domain=-\dx:\dx,
    y domain=-\dy:\dy,
    color=black,
    samples=80,
    xlabel=$x_1$,
    ylabel=$x_2$,
    zlabel=$p_X$,
    zmax={1.05*\ma},
]
%
% Student-t 2D
\addplot3 [surf] {1/(2*pi*\a*((1+((\c*x+\s*y)^2+((-\s*x+\c*y)/\a)^2)/(2*\mdd+1))^(1.5+\mdd)))};
%
% Marginales
\pgfmathsetmacro{\cproj}{coefstud(\mdd)};
\addplot3 [domain=-\dx:\dx,samples=50, samples y=0, thick, smooth, color=black]
(x,\dy,{\cproj*((1+((x/\su)^2)/(2*\mdd+1))^(-\mdd-1))/\su});
\addplot3 [domain=-\dy:\dy,samples=50, samples y=0, thick, smooth, color=black]
(-\dx,x,{\cproj*((1+((x/\sd)^2)/(2*\mdd+1))^(-\mdd-1))/\sd});
%\addplot3 [domain=-\dx:\dx,samples=51, thick, smooth, color=black] (x,\dy,{studpdf(x/\su,\mdd)});
%\addplot3 [domain=-\dy:\dy,samples=51, samples y=0, thick, smooth, color=black] (-\dx,x,{studpdf(x/\sd,\mdd)/\sd});
%
\node at (axis cs:{\dx/5},\dy,{\cproj*((1+((\dx/5/\su)^2)/(2*\mdd+1))^(-\mdd-1))/\su})[above right]{$p_{X_1}$};
\node at (axis cs:-\dx,{\dy/5},{\cproj*((1+((\dy/5/\sd)^2)/(2*\mdd+1))^(-\mdd-1))/\sd})[above right]{$p_{X_2}$};
%
\end{axis}
%
\node at ({\dx},-1) [scale=.9]{(c)};
\end{scope}
\end{tikzpicture} \end{center}
% 
\leyenda{Ilustraci\'on  de  una  densidad  de probabilidad  Student-$t$  escalar
  estandar (a),  y la funci\'on  de repartici\'on asociada  (b) con \ $\nu  = 1$
  (linea llena), \ $\nu = 3$ (linea  guionada), \ $\nu = 7$ (linea punteada) \ y
  \ $\nu \to +\infty$ (linea llena fina; ver m\'as adelante) grado de libertad,
  as\'i que una densidad de probabilidad Student-$t$ bi-dimensional con \ $\nu =
  1$ \  grado de libertad,  centrada, y de  matriz caracter\'istica \  $\Sigma =
  R(\theta) \Delta^2  R(\theta)^t$ \ con \  $R(\theta) = \protect\begin{bmatrix}
    \cos\theta    &     -    \sin\theta\\[2mm]    \sin\theta     &    \cos\theta
    \protect\end{bmatrix}$   \   matriz   de    rotaci\'on   y   \   $\Delta   =
  \diag\left(\protect\begin{bmatrix}  1   &  a\protect\end{bmatrix}  \right)$  \
  matriz  de   cambio  de  escala,   y  sus  marginales   \  $X_1  \,   \sim  \,
  \T_\nu\left(0,\cos^2\theta +  a^2 \sin^2\theta \right)$ \  y \ $X_2  \, \sim \,
  \T_\nu\left(0,\sin^2\theta   +   a^2  \cos^2\theta   \right)$   \  (ver   m\'as
  adelante). En la figura, $a = \frac13$ \ y \ $\theta = \frac{\pi}{6}$.}
\label{Fig:MP:StudentT}
\end{figure}

Nota: el caso  \ $\nu = 1$ \  es conocido como distribuci\'on de  {\em Cauchy} o
{\em  Cauchy-Lorentz} o  {\em Lorentzian}  o  {\em Breit-Wigner}~\cite{Cau53:07,
  Cau53,  Bien53, Bie53:07,  BreWig36, Sti74,  SamTaq94, Lorentz}.   Es  un caso
particular  tambi\'en de  distribuci\'on  $\alpha$-estables~\cite{SamTaq94}.  En
particular, una combinaci\'on lineal de variables de Cauchy independientes queda
de Cauchy. Pero, no  viola el teorema del l\'imite central del  hecho de que una
variable de Cauchy no admite covarianza.

Contrariamente al caso gaussiano, de la forma de la densidad de probabilidad, es
claro que si la matriz \ $\Sigma$ \ es diagonal, la densidad no factoriza, as\'i
que  las componentes  del vector  no son  independientes.  Este  ejemplo muestra
claramente que  la reciproca del lema~\ref{Lem:MP:IndependenciaCov}  es falsa en
general.

Sin embargo, las distribuciones Student-$t$ tienen varias propiedades notables.

\begin{lema}[Stabilidad por transformaci\'on lineal]
\label{Lem:MP:StabilidadLinealStudentT}
%
  Sea \ $X \, \sim \,  \T_\nu(m,\Sigma)$, \ $A$ \ matriz de \ $\M_{d',d}(\Rset)$
  \ con \ $d' \le d$, y de rango lleno y \ $b \in \Rset^{d'}$. Entonces
  %
  \[
  A X + b\, \sim \, \T_\nu( A m + b , A \Sigma A^t)
  \]
  %
  En particular los componentes de \ $X$ \ son student-$t$,
  %
  \[
  X_i \, \sim \, \T_\nu(m_i , \Sigma_{i,i} )
  \]
\end{lema}
\begin{proof}
  La prueba es inmediata usando  la funci\'on caracter\'istica y sus propiedades
  por  transformaci\'on lineal.  La condici\'on  sobre \  $A$ \  es  necesaria y
  suficiente para que \ $A \Sigma A^t \in P_{d'}^+(\Rset)$.
\end{proof}

\begin{lema}[V\'inculo con las distribuciones gamma y gaussiana (mezcla de escala gaussiana)]
\label{Lem:MP:MezclaGaussianaEscalaStudentT}
%
  Sea \ $V \,  \sim \, \G\left( \frac{\nu}{2} , \frac{\nu}{2} \right)$  \ y \ $G
  \,  \sim  \,  \N(0,I)$  \ con  $\nu  >  0$  \  y  \ $G$  \  $d$-dimensional  e
  independiente  de \ $V$.  Entonces, para  $\Sigma \in  P_d^+(\Rset)$ y  $m \in
  \Rset^d$,
  %
  \[
  \frac{\Sigma^{\frac12} G}{\sqrt{V}} + m  \, \sim \, \T_\nu(m,\Sigma)
  \]
  %
\end{lema}
\begin{proof}
  Sea  \  $X   =  \frac{G}{\sqrt{V}}$.   De  la  nota   siguiendo  la  tabla  de
  caracter\'isticas es  necesario y suficiente probar que  $X \sim \T_\nu(0,I)$.
  Lo  m\'as  simple  es  de  salir  de la  formula  de  probabilidad  total  del
  teorema~\ref{Teo:MP:ProbaTotalContinuo},  notando  que  de  la  independencia,
  condicionalmente  a  \  $V=v$  \   la  variable  es  gaussiana  de  covarianza
  $\frac{1}{v} I$,
  %
  \[
  p_{G|V=v}(x)  = (2  \pi)^{-\frac{d}{2}}  v^{\frac{d}{2}} e^{-  \frac{x^t x v}{2}}
  \]
  %
  Entonces, multiplicando \ $p_{G|V=v}$ \ por \ $p_V$ \ y por marginalizaci\'on,
  obtenemos
  %
  \begin{eqnarray*}
  p_X(x) & = & \frac{\nu^{\frac{\nu}{2}}}{2^{\frac{\nu+d}{2}} \pi^{\frac{d}{2}}
  \Gamma\left( \frac{\nu}{2} \right)} \, \int_{\Rset_+} v^{\frac{\nu+d}{2}-1} \,
  e^{- \frac{x^t x + \nu}{2} \, v} \, dv\\[2mm]
  %
  & = & \frac{\nu^{\frac{\nu}{2}} \left( \nu + x^t x \right)^{-
  \frac{\nu+d}{2}}}{\pi^{\frac{d}{2}} \Gamma\left( \frac{\nu}{2} \right)} \,
  \int_{\Rset_+} u^{\frac{\nu+d}{2}-1} \, e^{- u} \, du\\[2mm]
  %
  & = & \frac{\Gamma\left( \frac{\nu+d}{2} \right)}{(\pi \nu)^{\frac{d}{2}}
  \Gamma\left( \frac{\nu}{2} \right)} \, \left( 1 + \frac{x^t x}{\nu} \right)^{-
  \frac{\nu+d}{2}}
  \end{eqnarray*}
 %
  La secunda linea viene del cambio de variables \ $u = \frac{x^t x + \nu}{2} \,
  v$  \  y la  tercera  reconociendo  en la  integral  la  funci\'on gamma  (ver
  notaciones).
\end{proof}
%
Nota: este  lema permite tambi\'en  probar el lema~\ref{Lem:MP:StabilidadLineal}
escribiendo \ $A X + b \egald  \sqrt{\frac{\nu}{V}} A \Sigma^{\frac12} G + A m +
b$.

\begin{lema}[L\'imite gaussiana]
\label{Lem:MP:LimiteStudentTGaussiana}
%
  Sea \ $X_\nu \, \sim \, \T_\nu(m,\Sigma)$ \ vector Student-$t$ parametizado por
  \ $\nu$ \ su grado de libertad. Entonces
  %
  \[
  X_\nu \, \limitd{\nu \to \infty} \, = \, X \, \sim \, \N(m,\Sigma)
  \]
  %
  con \ $\displaystyle \limitd{}$ \ l\'imite es en distribuci\'on.
\end{lema}
\begin{proof}
  La prueba  es inmediata tomando el  logaritmo de la  densidad de probabilidad,
  usando la  formula de Stirling \  $\log\Gamma(z) = \left( z  - \frac12 \right)
  \log z - z  + \frac12 \log(2 \pi) + o(1)$ \  en \ $z \to +\infty$~\cite{Sti30,
    AbrSte70,  GraRyz15} \ y  \ $-\frac{d+\nu}{2}  \log\left( 1  + \frac{(x-m)^t
      \Sigma^{-1}  (x-m)}{\nu} \right)  = -\frac{d+\nu}{2}  \left( \frac{(x-m)^t
      \Sigma^{-1}   (x-m)}{\nu}  +   o\left(  \nu^{-1}   \right)  \right)   =  -
  \frac{(x-m)^t \Sigma^{-1} (x-m)}{2} + o(1)$.
\end{proof}

Las  variables   Student-$t$  tienen  varias   representaciones  estoc\'asticas,
relacionadas a la gaussiana~\cite{FanKot90, And03, KotNad04, AndKau65}:
% ej. KotNad p. 7 para la secunda
%
\begin{proof}

\end{proof}
%
Nota:       este       lema        permite       tambi\'en       probar       el
lema~\ref{Lem:MP:StabilidadLinealStudentT} escribiendo \ $A X + b \egald \frac{A
  \Sigma^{\frac12} G}{sqrt{V}} + A m + b$.

\begin{lema}[Relaci\'on con la distribuci\'on de Wishart]\label{Lem:MP:StudentWishart}
%
  Sea \ $W \, \sim \, \W( \Sigma^{-1} \, , \, \nu+d-1)$ \ $d \times d$ \ Wishart
  con \ $\Sigma \in P_d^+(\Rset)$, \ $Y \,  \sim \, \N(0,\nu I)$ \ con $\nu > 0$
  \ e \ $Y$ \ independiente de \  $W$. Entonces, para \ 
  %$R \in P_d^+(\Rset)$ \ y
  \ $m \in \Rset^d$,
  %
  \[
  W^{-\frac12} Y + m \, \sim \, \T_\nu\left( m , \Sigma \right)
  \]
  %
\end{lema}
\begin{proof}
  Sea \ $X = W^{-\frac12} Y$. De la nota siguiendo la tabla de caracter\'isticas
  es  necesario  y suficiente  probar  que $X  \sim  \T_\nu(0,\Sigma)$.  Ahora, de  la
  independencia tenemos
  %
  \[
  p_{X|W=w}(x)  = (2  \pi \nu)^{-\frac{d}{2}}  |w|^{\frac12} e^{-  \frac{x^t w  x}{2
      \nu}}
  \]
  %
  Denotamos  por \  $D =  \left\{ w_{ij},  \: 1  \le j  \le i  \le d  \tq  w \in
    P_d^+(\Rset) \right\}$ \ y, por abuso de  escritura, \ $dw = \prod_{ 1 \le j
    \le i \le d} dw_{ij}$.  Entonces,  multiplicando \ $p_{X|W=w}$ \ por \ $p_W$
  \ y por marginalizaci\'on, obtenemos
  % ($\propto$ significa ``proporcional  a'', i.e., olvidando el coefficiente de
  % normalizaci\'on)
  %
  \begin{eqnarray*}
  p_X(x) & = & \int_D \frac{|w|^{\frac{\nu-1}{2}} e^{- \frac{x^t w x}{2 \nu} -
  \frac12 \Tr\left( \Sigma w \right)}}{2^{\frac{d (\nu+d)}{2}} (\pi
  \nu)^{\frac{d}{2}} \left| \Sigma^{-1} \right|^{\frac{\nu+d-1}{2}} \Gamma_d \left(
  \frac{\nu+d-1}{2} \right)} \, dw\\[2mm]
  %
  & = & \frac{\Gamma\left( \frac{\nu+d}{2} \right)}{(\pi \nu)^{\frac{d}{2}}
  \Gamma\left( \frac{\nu}{2} \right)} \left| \Sigma + \frac{x x^t}{\nu}
  \right|^{-\frac{\nu+d}{2}} \left| \Sigma
  \right|^{\frac{\nu+d-1}{2}} \: \int_D \frac{|w|^{\frac{\nu+d-d-1}{2}} e^{-
  \frac12 \Tr\left( \left[ \Sigma + \frac{x x^t}{\nu} \right] w \right)}}{2^{\frac{d
  (\nu+d)}{2}} \left| \left( \Sigma + \frac{x x^t}{\nu} \right)^{-1}
  \right|^{\frac{\nu+d}{2}} \Gamma_d \left( \frac{\nu+d}{2} \right)} \, dw\\[2mm]
  %
  & = & \frac{\Gamma\left( \frac{\nu+d}{2} \right)}{(\pi \nu)^{\frac{d}{2}}
  \Gamma\left( \frac{\nu}{2} \right) \left| \Sigma \right|^{\frac12}} \: \left( 1
  + \frac{x^t \Sigma^{-1} x}{\nu} \right)^{-\frac{\nu+d}{2}} \, \int_D
  \frac{|w|^{\frac{\nu+d-d-1}{2}} e^{- \frac12 \Tr\left( \left[ I + \frac{x
  x^t}{\nu} \right] w \right)}}{2^{\frac{d (\nu+d)}{2}} \left| \left( I + \frac{x
  x^t}{\nu} \right)^{-1} \right|^{\frac{\nu+d}{2}} \Gamma_d \left( \frac{\nu+d}{2}
  \right)} \, dw
  \end{eqnarray*}
  %
  Para  \ $a, b  \in \Rset^d,  \: M  \in \M_{d,d}(\Rset)$,  en la  secunda linea
  usamos la  identidad \  $a^t M b  = \Tr(b a^t  M)$ \  y \ $\Gamma_d\left(  x -
    \frac12 \right) = \frac{\Gamma\left(  x - \frac{d}{2} \right)}{\Gamma(x)} \,
  \Gamma_d(x)$ \ (ver notaciones) y en  la tercera linea usamos $\left| \Sigma +
    \frac{x   x^t}{\nu}   \right|  =   \left|   \Sigma   \right|   \left|  I   +
    \frac{\Sigma^{-1}   x    x^t}{\nu}   \right|$   \   y    la   identidad   de
  Sylvester~\cite{Syl51} o~\cite[\S~18.1]{Har08} \ $\left| I + a b^t \right| = 1
  + b^t  a$. Se concluye  que \ $X  \sim \T_\nu(0,\Sigma)$ \ reconociendo  en el
  factor de  la integral como la  distribuci\'on \ $\T_\nu(0,\Sigma)$ \  y en el
  integrande  la  distribuci\'on de  Wishart  \  $\W\left(  \left( I  +  \frac{x
        x^t}{\nu}  \right) \,  , \,  \nu+d  \right)$ \  que suma  entonces a  la
  unidad.
  %  la  formula de  Sherman-Morrison-Woodbury  $\left(  I  + \frac{x  x^t}{\nu}
  % \right)^{-1} = I$~\cite{HorJoh13, Har08}
\end{proof}

Como lo hemos introducido, la distribuci\'on Student-$t$ aparece naturalmente en
el  marco  de la  estimaci\'on,  especialmente  a  trav\'es de  la  estimaci\'on
emp\'irica  de la  media  y covarianza~\cite{Mui82,  GupNag99, BilBre99,  And03,
  Seb04}:
% resp. p 80 teo 3.2.1 -- p. 92 teo 3.3.6 -- p. 87 prop. 7.1 -- p. 77 teo. 3.3.2 -- p. 63 teo. 3.1
% Nota : ver corolarios 2 y 3, p. 25 de Seber
% VER GupNag Th. 4.2.1
%
\begin{teorema}%[]
%
  Sean  \  $X_i \,  \sim  \,  \N(m,\Sigma), \:  i  =  1, \ldots  ,  n  > d-1$  \
  independientes,       y      sea       la       media      emp\'irica       (ver
  corolario~\ref{Cor:MP:MediaEmpiricaGauss})
  %
  \[
  \overline{X} = \frac{1}{n} \sum_{i=1}^n X_i
  \]
  %
  y  la  covarianza emp\'irica  construida  a partir  de  la  media emp\'irica  (ver
  corolario~\ref{Cor:MP:WishartEstimacion})
  %
  \[
  \overline{\Sigma}  =  \frac{1}{n-1}  \sum_{i=1}^n  \left( X_i  -  \overline{X}
  \right) \left( X_i - \overline{X} \right)^t
  \]
  %
  Entonces:
  %
  \begin{itemize}
  \item $\overline{X} -  m \, \sim \,  \N\left( 0 \, , \,  \frac{1}{n} \, \Sigma
    \right)$ \ y  \ $\overline{\Sigma} \, \sim \, \W(  \frac{1}{n-1} \Sigma \, ,
    \, n-1 ) $ \ son independientes;
  %
  \item $\sqrt{\frac{n (n-d)}{n-1}} \: \overline{\Sigma}^{\, -\frac12} \, \left(
      \overline{X} - m \right) \, \sim \, \T_{n-d}\left( 0 \, , \, I \right)$
  \end{itemize}
\end{teorema}
%
\begin{proof}
  Se      refiera      a     los      corolarios~\ref{Cor:MP:MediaEmpiricaGauss}
  y~\ref{Cor:MP:WishartEstimacion}   por   lo  de   las   distribuciones  de   \
  $\overline{X}-m$ \ y de \ $\overline{\Sigma}$ \ respectivamente.

  A continuaci\'on,  sean \  $\widetilde{X}_i =  X_i - m$  \ y  \ $\widetilde{X}
  =        \begin{bmatrix}        \widetilde{X}_1        &       \cdots        &
    \widetilde{X}_n \end{bmatrix}$. Obviamente
  %
  \[
  \overline{\widetilde{X}} \equiv \overline{X} - m = \frac{1}{n} \, \widetilde{X} \un
  \]
  %
  con \ $\un \in  \Rset^n$ \ vector de componentes iguales a \  $1$ \ y vimos en
  la prueba del corolario~\ref{Cor:MP:WishartEstimacion} que
  %
  \[
  \overline{\Sigma} = \frac{1}{n-1} \widetilde{X} \left( I - \frac{\un \un^t}{n}
  \right) \widetilde{X}^t
  \]
  %
  $A = I - \frac{\un \un^t}{n} \in  P_n(\Rset)$ \ es idemponenta de rango 1, con
  $A  \un = 0$,  as\'i que  por diagonalizaci\'on~\cite{HorJoh13,  Bat97, Bat07}
  tenemos
  %
  \[
  A = P \begin{bmatrix} I_{n-1} & 0\\ 0 & 0 \end{bmatrix} P^t \qquad \mbox{con} \qquad
  P = \begin{bmatrix} B & \frac{1}{\sqrt{n}} \un \end{bmatrix}
  \]
  %
  $P P^t = P P^t = I$ \ y
  %
  \[
  B \in \M_{n,n-1}(\Rset) \quad \mbox{tal que} \quad  B^t B = I \: \mbox{ y } \:
  \un^t B = 0
  \]
  %
  Ahora,   poniendo   la  descomposici\'on   diagonal   de   \   $A$  \   en   \
  $\overline{\Sigma}$ \ obtenemos (ver~corolario~\ref{Cor:MP:WishartEstimacion})
  %
  \[
  \overline{\Sigma} = \frac{1}{n-1} \, Y Y^t \qquad \mbox{con} \qquad Y = \widetilde{X} B
  \]
  %
  Luego, de la gaussianidad y independencia de los \ $\widetilde{X}_i$ \ tenemos,
  para \   $\widetilde{x}   =    \begin{bmatrix}   \widetilde{x}_1   &   \cdots   &
    \widetilde{x}_n \end{bmatrix} \in \M_{d,n}(\Rset)$
  %
  \begin{eqnarray*}
  p_{\widetilde{X}}(\widetilde{x}) & = & (2 \pi)^{-\frac{n d}{2}}
  |\Sigma|^{-\frac{n}{2}} \exp\left(- \frac12 \sum_{i=1}^n \widetilde{x}_i^t
  \Sigma^{-1} \widetilde{x}_i \right)\\[2mm]
  %
  & = & (2 \pi)^{-\frac{n d}{2}} |\Sigma|^{-\frac{n}{2}} \exp\left(- \frac12
  \sum_{i=1}^n \Tr\left( \Sigma^{-1} \widetilde{x}_i \widetilde{x}_i^t
  \right) \right)\\[2mm]
  %
  & = & (2 \pi)^{-\frac{n d}{2}} |\Sigma|^{-\frac{n}{2}} \exp\left(- \frac12
  \Tr\left( \Sigma^{-1} \widetilde{x} \, \widetilde{x}^t \right) \right)
  \end{eqnarray*}
  %
  Sea    la     transformaci\'on    \    $\begin{bmatrix}     Y    &    \sqrt{n}
    \overline{\widetilde{X}}   \end{bmatrix}   =   \widetilde{X}   P$,   \ie   \
  $\widetilde{X}        =       \begin{bmatrix}        Y        &       \sqrt{n}
    \overline{\widetilde{X}} \end{bmatrix}  P^t$.  Se nota que  \ $|P| =  1$ \ y
  por                            transformaci\'on                           (ver
  teorema~\ref{Teo:MP:TransformacionInyectivaDensidad}),    para   \    $y   \in
  \M_{d,n-1}(\Rset)$ \ y \ $x \in \Rset^d$
  %
  \begin{eqnarray*}
  p_{Y,\sqrt{n} \overline{\widetilde{X}}}(y,x) & = & (2 \pi)^{-\frac{n d}{2}}
  |\Sigma|^{-\frac{n}{2}} \exp\left(- \frac12 \Tr\left(
  \Sigma^{-1} \begin{bmatrix} y & x \end{bmatrix} P^t P \begin{bmatrix} y^t\\
  x \end{bmatrix}\right) \right)\\[2mm]
  %
  & = & (2 \pi)^{-\frac{n d}{2}} |\Sigma|^{-\frac{n}{2}} \exp\left(- \frac12
  \Tr\left( \Sigma^{-1} \left( y y^t + x x^t \right) \right) \right)\\[2mm]
  %
  & = & (2 \pi)^{-\frac{(n-1) d}{2}} |\Sigma|^{-\frac{n-1}{2}} \exp\left(-
  \frac12 \Tr\left( \Sigma^{-1} y y^t \right) \right) \times (2
  \pi)^{-\frac{d}{2}} |\Sigma|^{-\frac12} \exp\left(- \frac12 x^t \Sigma^{-1} x
  \right)
  \end{eqnarray*}
  %
  Claramente,  de la factorizaci\'on  de las  distribuciones, $Y  = X  B$ \  y \
  $\sqrt{n}  \overline{\widetilde{X}}$  \ son  independientes,  es  decir que  \
  $\frac{1}{n-1} \, Y Y^t = \overline{\Sigma}$ \ y \ $\overline{\widetilde{X}} =
  \overline{X} -  m$ \ son  independientes, lo que  cierra la prueba  del primer
  item.  Pasando, la forma  de $p_{Y,\sqrt{n}  \overline{\widetilde{X}}}(y,x)$ \
  confirma  que  \ $\overline{X}-m$  \  es  gaussiana  centrada de  covarianza  \
  $\frac{1}{n} \,  \Sigma$, y  que los \  $Y_i$ \ son  independientes gaussianos,
  dando la distribuci\'on  de Wishart del lema~\ref{Lem:MP:WishartGausiana} para
  la covarianza emp\'irica.

  A continuaci\'on,
  %
  \[
  \sqrt{\frac{n   (n-d)}{n-1}}   \,   \overline{\Sigma}^{\,   -\frac12}   \left(
    \overline{X}-m \right)  = \frac{1}{\sqrt{n-1}} \:  \Sigma^{- \frac12} \left(
    \Sigma^{-1}  \, \overline{\Sigma}  \, \Sigma^{-1}  \right)^{-\frac12} \left(
    \sqrt{n (n-d)} \: \Sigma^{-\frac12} \left( \overline{X}-m \right) \right)
  \]
  %
  Del           teorema~\ref{Teo:MP:StabilidadGaussiana}          y          del
  lema~\ref{Lem:MP:StabilidadWishartLineal} tenemos
  %
  \[
  \sqrt{n (n-d)}  \: \Sigma^{-\frac12} \left( \overline{X}-m \right)  \, \sim \,
  \N(0 ,  (n-d) I)  \qquad \mbox{y} \qquad  \Sigma^{-1} \,  \overline{\Sigma} \,
  \Sigma^{-1}  \, \sim  \, \W\left(  \left( (n-1)  \Sigma \right)^{-1}  \,  , \,
    n-d+d-1 \right)
  \]
  %
  Se   cierra   la  prueba   usando   los  lemas~\ref{Lem:MP:StudentWishart}   \
  y~\ref{Lem:MP:StabilidadLineal}.
\end{proof}

%\SZ{
% VER LO QUE PASA SI overline{Sigma} si X_i y X_i-overline{X} para estandardizar los datos.

%Sampling distribution, Gosset, Fisher 25. Applications
%}

M\'as propiedades de esta distribuci\'on se encuentran en libros especializados,
por ejemplo~\cite{KotNad04} completamente dedicado a esta distribuci\'on.

\

La distribuci\'on  Student-$t$ se generaliza al  caso complejo \  $Z$ \ definido
sobre $\Cset^d$; se denota  \ $Z \, \sim \, \CT_\nu(m,\Sigma)$ \  donde \ $m \in
\Cset^d$, \ $\Sigma \in P_d^+(\Cset)$ \ y la densidad es dada por \
%
\[
p_Z(z) = \frac{\Gamma\left( d  + \frac{\nu}{2} \right)}{\pi^d \nu^d \Gamma\left(
    \frac{\nu}{2}   \right)  \,   \left|   \Sigma  \right|}   \:   \left(  1   +
  \frac{(z-m)^\dag \, \Sigma^{-1} \, (z-m)}{\nu} \right)^{- \frac{\nu}{2}-d}
\]
%
(ver  por  ejemplo~\cite[\S~5.12  y   ref.]{KotNad04}  para  una  versi\'on  muy
parecida).
% Gupta 64, Tan 73, 69b
%Se puede referirse  a~\cite[Cap.~4]{GupNag99} para tener m\'as
%detalles.

\

Tambi\'en, la  distribuci\'on Student-$t$  se generaliza al  caso matriz-variada
$X$   definido  sobre   $M_{d,d'}(\Rset)$;   se   denota  \   $X   \,  \sim   \,
\T_\nu(M,\Sigma,\Omega)$  \  donde  \  $M  \in M_{d,d'}(\Rset),  \:  \Sigma  \in
P_d^+(\Rset),  \:  \Omega  \in  P_{d'}^+(\Rset)$  y  la  densidad  es  dada  por
$\displaystyle             p_X(x)             =             \frac{\Gamma_d\left(
    \frac{\nu+d+d'-1}{2}\right)}{\pi^{\frac{\nu       d}{2}}      \Gamma_d\left(
    \frac{\nu+d-1}{2}\right)  \,  \left|  \Sigma  \right|^{\frac{d'}{2}}  \left|
    \Omega \right|^{\frac{d}{2}}} \: \left| I + \Sigma^{-1} (x-M) \, \Omega^{-1}
  \,      (x-M)^t     \right|^{-     \frac{\nu+d+d'-1}{2}}$.      Se     refiera
a~\cite{Dic67}, \cite[Cap.~4]{GupNag99} o~\cite[\S5.11   y
ref.]{KotNag04}   para   tener   m\'as   detalles.
% Dickey 66, Cornis 54

% --------------------------------- Student-r
%\subsubseccion{Distribuci\'on Student-$r$ multivariada}
\label{Sssec:MP:StudentR}

Estas distribuciones  aparecieron esencialmente a  trav\'es el sistema  dicho de
Pearson en el fin del siglo XIX~\cite{Pea95, JohKot95:v1, JohKot95:v1, KotBal00,
  FanKot90}.   M\'as   especialmente  son  conocidos  como   {\em  Pearson  tipo
  IIIa$\alpha$}.   A veces,  se  encuentra en  f\'isica  la denominaci\'on  {\em
  Student-$r$}; aparecen como maximizantes de  la entrop\'ia de R\'enyi, o de la
de Tsalis  de la misma manera que  las Student-$t$ usual, pero  cuando el indice
entr\'opico es  mayor que la unidad  (ver capitulo~\ref{Cap:SZ:Informacion} para
tener m\'as detalles)~\cite{JohVig07, CosHer03, VigHer04, Tsa88, Tsa99}.

Se denota con \ $X \sim \R_\nu(m,\Sigma)$ \ con \ $m \in \Rset^d$, \ $\Sigma \in
\Pos_d^+(\Rset)$. $m$ \ es llamado  {\em par\'ametro de posici\'on}, \ $\Sigma$
  \ es  llamada {\em matriz  caracter\'istica} y \  $\nu > d-2$ \  llamado {\em
  grado  de  libertad}.   Las  caracter\'isticas  de una  Student-$r$  son  las
siguientes:
%
\begin{caracteristicas}
%
Dominio de definici\'on & $\begin{array}{lll} \X & = & m + \Sigma^{\frac12} \,
\Bset_d\left( 0 , \sqrt{\nu+2} \right)\\[1mm] & = & \left\{ x \in \Rset^d \tq
\Sigma^{-\frac12} (x-m) \in \Bset_d\left( 0 , \sqrt{\nu+2} \right)
\right\} \end{array}\vspace{1mm}$\\[2mm]
\hline
%
Par\'ametro & $\nu > d-2$ \ (grado de libertad),\newline $m \in \Rset^d$ \
(posici\'on), \ $\Sigma \in \Pos_d^+(\Rset)$ \ (matriz caracer\'istica)\\[2mm]
\hline
%
Densidad de probabilidad & $\displaystyle p_X(x) = \frac{\Gamma\left(
\frac{\nu}{2} + 1 \right)}{\pi^{\frac{d}{2}} (\nu+2)^{\frac{d}{2}} \Gamma\left(
\frac{\nu-d}{2} + 1 \right) \, \left| \Sigma \right|^{\frac12}} \, \left( 1 -
\frac{(x-m)^t \Sigma^{-1} (x-m)}{\nu+2} \right)_+^{\!\frac{\nu-d}{2}}$\\[2mm]
\hline
%
Promedio & $\displaystyle m_X = m$\\[2.5mm]
\hline
%
Covarianza & $\displaystyle \Sigma_X = \Sigma$\\[2.5mm]
\hline
%
\modif{Asimetr\'ia} & $\displaystyle \gamma_X = 0$\\[2mm]
\hline
%
Curtosis por exceso & $\displaystyle \widebar{\kappa}_X = \frac{- 2}{\nu+4}
\sum_{i,j=1}^d \Big( \! \left(
    \un_i \un_i^t \right) \otimes \left(  \un_j \un_j^t \right) +  \left( \un_i
    \un_j^t \right) \otimes \left( \un_i  \un_j^t \right) + \left( \un_i \un_j^t
  \right) \otimes \left( \un_j \un_i^t \right) \! \Big)$\\[2mm]
\hline
%
Funci\'on caracter\'istica & $\displaystyle
\Phi_X(\omega) = \frac{2^{\frac{\nu}{2}} \Gamma\left(
\frac{\nu}{2} +1 \right)}{(\nu+2)^{\frac{\nu}{4}}} \, e^{\imath \omega^t m} \, \left( \omega^t \Sigma \omega
\right)^{- \frac{\nu}{4}} J_{\frac{\nu}{2}}\left( \sqrt{(\nu+2) \, \omega^t \Sigma
\omega} \right)$
\end{caracteristicas}

Nota: nuevamente se puede escribir $X \, \egald \, \Sigma^{\frac12} R + m$ \ con
\ $R \, \sim \, \R_\nu(0,I)$ \ donde \ $R$ \ es dicha {\em Student-$r$ estandar}
y  las caracter\'isticas  de \  $X$  \ son  v\'inculadas a  las  de \  $R$ \  (y
vice-versa) por transformaci\'on lineal (ver secciones anteriores).

Unas  densidades  de probabilidad  Student-$r$  estandares  y  las funciones  de
repartici\'on  asociadas en  el  caso  escalar son  representadas  en la  figura
Fig.~\ref{Fig:MP:StudentR}-(a) y  (b) para varios  grado de libertad, y
una densidad en un contexto bi-dimensional figura Fig.~\ref{Fig:MP:StudentR}(c).
%%
\begin{figure}[h!]
\begin{center} \begin{tikzpicture}
\shorthandoff{>}
%
% Para el caso univariado
\pgfmathsetmacro{\sx}{.72};% x-scaling
\pgfmathsetmacro{\mu}{0};% para tomar los grados de libertad impar
\pgfmathsetmacro{\md}{1};
\pgfmathsetmacro{\mt}{5};
%\pgfmathsetmacro{\mq}{3};
%
%
% para el caso bi-variado
\pgfmathsetmacro{\mdd}{1};%
\pgfmathsetmacro{\nu}{2*\mdd};% grados de libertad
\pgfmathsetmacro{\a}{1/3};% x-scaling
\pgfmathsetmacro{\t}{30};% angulo de rotacion
\pgfmathsetmacro{\c}{cos(\t)};% coseno
\pgfmathsetmacro{\s}{sin(\t)};% seno
\pgfmathsetmacro{\su}{sqrt(\c^2+(\a*\s)^2)};% ecart-type 1
\pgfmathsetmacro{\sd}{sqrt(\s^2+(\a*\c)^2)};% ecart-type 2
\pgfmathsetmacro{\dx}{1.5*\su*sqrt(\mdd+3)};% dominio x del plot -dx:dx
\pgfmathsetmacro{\dy}{1.5*\sd*sqrt(\mdd+3)};% dominio y del plot -dy:dy
%
%
% Approximacion de la funcion Gamma
%\tikzset{declare function={gamma(\z)=
%(2.506628274631*sqrt(1/\z) + 0.20888568*(1/\z)^(1.5) + 0.00870357*(1/\z)^(2.5) -
%(174.2106599*(1/\z)^(3.5))/25920 - (715.6423511*(1/\z)^(4.5))/1244160)*exp((-ln(1/\z)-1)*\z);}}
%
\tikzset{declare function={
xplus(\x) = max(\x,0);
}}
% Approximation de la cdf gaussienne
\tikzmath{function normcdf(\x) {return 1/(1 + exp(-0.07056*(\x)^3 - 1.5976*(\x)));};};
%
% coefficiente binomial, para no tener factoriales muy grandes
\tikzmath{function binocoef(\m,\k) {if \k == 0 then {return 1;} else {return ((\m-\k+1)/\k)*binocoef(\m,\k-1);};};};
%
% coefficiente student-r, d=1, para nu = 2 m + 1, recursivamente
\tikzmath{function coefstud(\m) {if \m == 0 then {return 1/(2*sqrt(3));}
          else {return ((\m+.5)/\m)*sqrt((2*\m+1)/(2*\m+3))*coefstud(\m-1);};};};
%
% cdf Student que se calcula recursivamente para nu = 2 m + 1, m entero
\tikzmath{function studcdfS(\x,\m,\k) {
    if \k == 0 then {return \x;}
    else {return studcdfS(\x,\m,\k-1)+(binocoef(\m,\k)*((-1)^(\k))*((\x)^(2*\k+1))/(2*\k+1);};};};
% Calculo de
%  - x maximo del plot para tener pdf a 7% del max
%  - la pdf Student para nu = 2 m + 1, m entero
%  - la cdf Student para nu = 2 m + 1, m entero
%\tikzset{declare function={
%\tikzmath{function maxplotpdf(\m) {return sqrt((2*\m+1)*((.03^(-1/(\m+1)))-1));};};% x maximo del plot para tener pdf a 3% del max
\tikzmath{function studpdf(\x,\m) {return coefstud(\m)*((1-((\x)^2)/(2*\m+3))^(\m));};};% pdf Student
\tikzmath{function studcdf(\x,\m) {return .5+(sqrt(2*\m+3))*coefstud(\m)*studcdfS(\x/(sqrt(2*\m+3)),\m,\m);};};% pdf Student
%}}
%
%
%
% mismas escalas x-max para cada ejemplo
\pgfmathsetmacro{\mx}{1.05*sqrt(2*max(\mu,\md,\mt)+3)};
% x para limitar dibujo de la pdf divergente al maximo
%\pgfmathsetmacro{\my}{max(coefstud(\mu),coefstud(\md),coefstud(\mt))};

% maximo de las marginales del caso 2D
\pgfmathsetmacro{\ma}{coefstud(\mdd)/min(\su,\sd)};
%
% densidad
\begin{scope}[scale=.9]
%
\pgfmathsetmacro{\sy}{2.55*sqrt(2*pi)};% y-scaling 
\draw[>=stealth,->] ({-\sx*\mx-.1},0)--({\sx*\mx+.25},0) node[right]{\small $x$};
\draw[>=stealth,->] (0,-.15)--(0,3) node[above]{\small $p_X$};
%
\draw[thick,dash dot,domain=-sqrt(2-1.5/pi):sqrt(2-1.5/pi),samples=50,smooth]
({-\sx*\mx},0)--({-\sx*sqrt(2)},0)--({-\sx*sqrt(2)},{\sy/(pi*sqrt(1.5/pi)})
plot ({\x*\sx},{\sy/(pi*sqrt(2-\x*\x)})
({\sx*sqrt(2)},{\sy/(pi*sqrt(1.5/pi)})--({\sx*sqrt(2)},0)--({\sx*\mx},0);
%
\draw[thick,dashed,domain=-sqrt(2*\mu+3):sqrt(2*\mu+3),samples=50,smooth]
({-\sx*\mx},0)--({-\sx*sqrt(2*\mu+3)},0)--
plot ({\x*\sx},{\sy*studpdf(\x,\mu)})--
({\sx*sqrt(2*\mu+3)},0)--({\sx*\mx},0);
%
\draw[thick,domain=-sqrt(2*\md+3):sqrt(2*\md+3),samples=50,smooth]
({-\sx*\mx},0)--({-\sx*sqrt(2*\md+3)},0)--
plot ({\x*\sx},{\sy*studpdf(\x,\md)})--
({\sx*sqrt(2*\md+3)},0)--({\sx*\mx},0);
%
\draw[thick,dotted,domain=-sqrt(2*\mt+3):sqrt(2*\mt+3),samples=50,smooth]
({-\sx*\mx},0)--({-\sx*sqrt(2*\mt+3)},0)--
plot ({\x*\sx},{\sy*studpdf(\x,\mt)})--
({\sx*sqrt(2*\mt+3)},0)--({\sx*\mx},0);
%
\draw[thin,domain=-\mx:\mx,samples=50,smooth] plot ({\x*\sx},{\sy*exp(-.5*((\x)^2))/sqrt(2*pi)});
%
\draw (0,{\sy/sqrt(2*pi)})--(-.15,{\sy/sqrt(2*pi)}) node[left,scale=.625]{$\displaystyle \frac1{\sqrt{2 \pi}}$};
\draw (0,0)--(0,-.1) node[below,scale=.7]{$0$};
%\pgfmathsetmacro{\lm}{2*floor(\mx/2)};
%\foreach \m in {2,4,...,\lm} {
\draw ({-\sx*sqrt(5)},0)--({-\sx*sqrt(5)},-.1) node[below,scale=.625]{$-\sqrt{5}$};
\draw ({\sx*sqrt(5)},0)--({\sx*sqrt(5)},-.1) node[below,scale=.625]{$\sqrt{5}$};
%
\draw ({-\sx*sqrt(2)},0)--({-\sx*sqrt(2)},-.1) node[below,scale=.625]{$-\sqrt{2}$};
\draw ({\sx*sqrt(2)},0)--({\sx*sqrt(2)},-.1) node[below,scale=.625]{$\sqrt{2}$};
%\draw ({\m*\sx},0)--({\m*\sx},-.1) node[below,scale=.7]{$\m$};
%}
%
\node at (0,-1) [scale=.9]{(a)};
\end{scope}
%
%
% reparticion
\begin{scope}[xshift=5.5cm,scale=.9]
%
\pgfmathsetmacro{\extx}{.9};% dimunucion del x min-max para la cdf (que se vea mejor) 
\pgfmathsetmacro{\sy}{3};% y-scaling 
%
\draw[>=stealth,->] ({-\sx*\mx*\extx-.1},0)--({\sx*\mx*\extx+.25},0) node[right]{\small $x$};
\draw[>=stealth,->] (0,-.15)--(0,{\sy+.25}) node[above]{\small $F_X$};
%
% cumulativa
%
\draw[thick,dash dot,domain=-sqrt(2):sqrt(2),samples=50,smooth]
({-\sx*\mx*\extx},0)--({-\sx*sqrt(2)},0)--
plot ({\x*\sx},{\sy*(.5+asin(\x/sqrt(2))/180})--
({\sx*sqrt(2)},\sy)--({\sx*\mx*\extx},\sy);
%
\draw[thick,dashed,domain=-sqrt(2*\mu+3):sqrt(2*\mu+3),samples=50,smooth]
({-\sx*\mx*\extx},0)--({-\sx*sqrt(2*\mu+3)},0)--
plot ({\x*\sx},{\sy*studcdf(\x,\mu)})--
({\sx*sqrt(2*\mu+3)},\sy)--({\sx*\mx*\extx},\sy);
%
\draw[thick,domain=-sqrt(2*\md+3):sqrt(2*\md+3),samples=50,smooth]
({-\sx*\mx*\extx},0)--({-\sx*sqrt(2*\md+3)},0)--
plot ({\x*\sx},{\sy*studcdf(\x,\md)})--
({\sx*sqrt(2*\md+3)},\sy)--({\sx*\mx*\extx},\sy);
%
\draw[thick,dotted,domain=-sqrt(2*\mt+3):sqrt(2*\mt+3),samples=50,smooth]
({-\sx*\mx*\extx},0)--({-\sx*sqrt(2*\mt+3)},0)--
plot ({\x*\sx},{\sy*studcdf(\x,\mt)})--
({\sx*sqrt(2*\mt+3)},\sy)--({\sx*\mx*\extx},\sy);
%
\draw[thin,domain={max(-\mx,-3.5)}:\mx,samples=50,smooth]  plot ({\x*\sx},{\sy*normcdf(\x)});
%%
\draw (0,0)--(0,-.1) node[below,scale=.7]{$0$};
\draw (0,\sy)--(-.1,\sy) node[left,scale=.7]{$1$};
%\pgfmathsetmacro{\lm}{2*floor(\mx*\extx/2)};
%\foreach \m in {2,4,...,\lm} {
%\draw ({-\m*\sx},0)--({-\m*\sx},-.1) node[below,scale=.7]{$-\m$};
%\draw ({\m*\sx},0)--({\m*\sx},-.1) node[below,scale=.7]{$\m$};
%}
\draw ({-\sx*sqrt(5)},0)--({-\sx*sqrt(5)},-.1) node[below,scale=.625]{$-\sqrt{5}$};
\draw ({\sx*sqrt(5)},0)--({\sx*sqrt(5)},-.1) node[below,scale=.625]{$\sqrt{5}$};
%
\draw ({-\sx*sqrt(2)},0)--({-\sx*sqrt(2)},-.1) node[below,scale=.625]{$-\sqrt{2}$};
\draw ({\sx*sqrt(2)},0)--({\sx*sqrt(2)},-.1) node[below,scale=.625]{$\sqrt{2}$};
%
\node at (0,-1) [scale=.9]{(b)};
\end{scope}
%
%
% densidad 2D
\begin{scope}[xshift=9cm,yshift=-2.5mm,scale=.7]
%
\begin{axis}[
    colormap = {whiteblack}{color(0cm)  = (white);color(1cm) = (black)},
    width=.45\textwidth,
    view={45}{65},
    enlargelimits=false,
    %grid=major,
    domain=-\dx:\dx,
    y domain=-\dy:\dy,
    color=black,
    samples=80,
    xlabel=$x_1$,
    ylabel=$x_2$,
    zlabel=$p_X$,
    zmax={1.05*\ma},
]
%
% Student-r 2D
\addplot3 [surf] {5*((2*\mdd+1)/(2*pi*(2*\mdd+3)))*((xplus(1-((\c*x+\s*y)^2+((-\s*x+\c*y)/\a)^2)/(2*\mdd+3)))^(\mdd-.5))};
% el factor 5: estoy trampeando para que se vea mejor...
%
% Marginales
\pgfmathsetmacro{\cproj}{coefstud(\mdd)};
%
\addplot3 [domain=-\dx:\dx,samples=50, samples y=0, thick, smooth, color=black]
(x,\dy,{\cproj*((xplus(1-((x/\su)^2)/(2*\mdd+3)))^(\mdd-.5))/\su});
%
%
\addplot3 [domain=-\dy:\dy,samples=50, samples y=0, thick, smooth, color=black]
(-\dx,x,{\cproj*((xplus(1-((x/\sd)^2)/(2*\mdd+3)))^(\mdd-.5))/\sd});
%
\node at (axis cs:{\dx/2},\dy,{\cproj*((xplus(1-((\dx/2/\su)^2)/(2*\mdd+3)))^(\mdd-.5))/\su})[right]{$p_{X_1}$};
\node at (axis cs:-\dx,{\dy/3},{\cproj*((xplus(1-((\dy/3/\sd)^2)/(2*\mdd+3)))^(\mdd-.5))/\sd})[right]{$p_{X_2}$};
%
\end{axis}
%
\node at ({\dx},-1) [scale=.9]{(c)};
\end{scope}
\end{tikzpicture} \end{center}
% 
\leyenda{Ilustraci\'on  de  una  densidad  de probabilidad  Student-$r$  escalar
  estandar (a),  y la funci\'on  de repartici\'on asociada  (b) con \ $\nu  = 0$
  (linea mixta), \ $\nu = 1$ (linea guionada), \ $\nu = 3$ (linea llena), \ $\nu
  = 11$ (linea  punteada) \ y \  $\nu \to +\infty$ (linea llena  fina; ver m\'as
  adelante)  grado  de  libertad,   as\'i  que  una  densidad  de  probabilidad
  Student-$r$ bi-dimensional con \ $\nu = 3$ \ grado de libertad, centrada, y de
  matriz caracter\'istica \ $\Sigma = R(\theta) \Delta^2 R(\theta)^t$ \ con
  \  $R(\theta)  =  \protect\begin{bmatrix}  \cos\theta  &  -  \sin\theta\\[2mm]
    \sin\theta &  \cos\theta \protect\end{bmatrix}$ \  matriz de rotaci\'on  y \
  $\Delta   =  \diag\left(\protect\begin{bmatrix}  1   &  a\protect\end{bmatrix}
  \right)$ \  matriz de cambio  de escala,  y sus marginales  \ $X_1 \,  \sim \,
  \R_\nu\left(0,\cos^2\theta + a^2  \sin^2\theta \right)$ \ y \  $X_2 \, \sim \,
  \R_\nu\left(0,\sin^2\theta   +  a^2   \cos^2\theta  \right)$   \   (ver  m\'as
  adelante). En la figura, $a = \frac13$ \ y \ $\theta = \frac{\pi}{6}$.}
\label{Fig:MP:StudentR}
\end{figure}

Nota:  en el  caso \  $\nu =  d$ \  la ley  es uniforme  adentro del  dominio de
definici\'on  $X  \sim  \U\left(  m  + \Sigma^{\frac12}  \,  \Bset_d\left(  0  ,
    \sqrt{d+2} \right) \right)$. Para \ $d-2 < \nu < d$ \ la densidad diverge
en los bordes del dominio de definici\'on (divergencia integrable).

Contrariamente al caso gaussiano, de la forma de la densidad de probabilidad, es
claro que a\'un si la matriz \ $\Sigma$ \ es diagonal, la densidad no factoriza,
as\'i  que las  componentes  del  vector no  son  independientes.  Este  ejemplo
muestra     claramente     y     nuevamente     que     la     reciproca     del
lema~\ref{Lem:MP:IndependenciaCov} es falsa en general.

Sin embargo, las distribuciones Student-$r$ tienen varias propiedades notables.

\begin{lema}[Stabilidad por transformaci\'on lineal]
\label{Lem:MP:StabilidadLinealStudentR}
%
  Sean \ $X \, \sim \,  \R_\nu(m,\Sigma)$, \ $A$ \ matriz de \ $\M_{d',d}(\Rset))$
  \ con \ $d' \le d$, y de rango lleno y \ $b \in \Rset^{d'}$. Entonces
  %
  \[
  A X + b\, \sim \, \R_\nu( A m + b , A \Sigma A^t)
  \]
  %
  En particular los componentes de \ $X$ \ son student-$r$,
  %
  \[
  X_i \, \sim \, \R_\nu(m_i , \Sigma_{i,i} )
  \]
\end{lema}
\begin{proof}
  La prueba es inmediata usando  la funci\'on caracter\'istica y sus propiedades
  por  transformaci\'on lineal.  La condici\'on  sobre \  $A$ \  es  necesaria y
  suficiente para que \ $A \Sigma A^t \in \Pos_{d'}^+(\Rset)$.
\end{proof}

\begin{lema}[L\'imite gaussiana]
\label{Lem:MP:LimiteStudentRGaussiana}
%
  Sea \  $X_\nu \, \sim  \, \R_\nu(m,\Sigma)$ \ vector  Student-$r$ parametizado
  por \ $\nu$ \ su grado de libertad. Entonces
  %
  \[
  X_\nu \, \limitd{\nu \to \infty} \, = \, X \, \sim \, \N(m,\Sigma)
  \]
  %
  con \ $\displaystyle \limitd{}$ \ l\'imite es en distribuci\'on.
\end{lema}
\begin{proof}
  La prueba  es inmediata tomando el  logaritmo de la  densidad de probabilidad,
  usando la  formula de Stirling \  $\log\Gamma(z) = \left( z  - \frac12 \right)
  \log z - z  + \frac12 \log(2 \pi) + o(1)$ \  en \ $z \to +\infty$~\cite{Sti30,
    AbrSte70,  GraRyz15} \  y \  $\frac{\nu-d}{2} \log\left(  1  - \frac{(x-m)^t
      \Sigma^{-1} (x-m)}{\nu}  \right) = -  \frac{\nu-d}{2} \left( \frac{(x-m)^t
      \Sigma^{-1}  (x-m)}{\nu+2}   +  o\left(  \nu^{-1}  \right)   \right)  =  -
  \frac{(x-m)^t \Sigma^{-1}  (x-m)}{2} +  o(1)$. Adem\'as, se  nota que  $\X \to
  \Rset^d$.
\end{proof}

%Las   variables   Student-t   tienen   varias   representaciones   estocasticas,
%relacionadas a la gaussiana~\cite{FanKot90, And03, KotNad04, AndKau65}:
%% ej. KotNad p. 7 para la secunda
%
\begin{lema}[Relaci\'on con la distribuci\'on gamma y la ley gaussiana]\label{Lem:MP:StudentRGamma}
%
  Sea \ $V \sim \G\left( \frac{\nu-d}{2}+1 \,  , \, \frac12 \right)$ \ y \ $G \,
  \sim \,  \N(0,I)$ \  $d$-dimensional e independientes  de $V$.  Entonces, para
  $\Sigma \in \Pos_d^+(\Rset)$ y $m \in \Rset^d$,
  %
  \[
  \frac{\sqrt{\nu+2} \:  \Sigma^{\frac12} \, G}{\sqrt{V  + \|G\|^2}} + m \,  \sim \,
  \R_\nu( m , \Sigma )
  \]
  %
\end{lema}
\begin{proof}
  Sea \ $X = \frac{\sqrt{\nu+2} \,  G}{\sqrt{V + \|G\|}}$.  De la nota siguiendo
  la tabla  de caracter\'isticas  es necesario y  suficiente probar que  $X \sim
  \R_\nu(0,I)$.  Probaremos en la secci\'on~\ref{Ssec:MP:FamiliaEliptica} que $G
  \egald R U$  \ con \ $R >  0$ \ de densidad de probabilidad  \ $p_R(r) \propto
  r^{d-1} e^{-\frac{r^2}{2}}$, independiente de  \ $U$, vector de distribuci\'on
  uniforme  sobre la esfera  \ $\Sset_d$.   Probaremos tambi\'en  que \  $Y \sim
  \R_\nu(0,I)$ \ se escribe de la misma manera \ $Y \egald S U$ \ ahora con \ $S
  >  0$ \  de densidad  de probabilidad  \ $p_S(s)  \propto s^{d-1}  \left(  1 -
    \frac{s^2}{\nu+2}       \right)_+^{\frac{\nu-d}{2}}$.        Ahora,      del
  teorema~\ref{Teo:MP:IgualdadDistribucionFuncionVA},   y    de   la   escritura
  estoc\'astica de $G$ tenemos
  %
  \[
  X \egald \frac{\sqrt{d+2} \, R}{\sqrt{V + R^2}} \: U
  \]
  %
  Sea
  %
  \[
  S = \frac{\sqrt{d+2} \, R}{\sqrt{V + R^2}}
  \]
  %
  Claramente,  $R,  V$  siendo  independientes   de  $U$,  la  variable  $S$  es
  independiente de  $U$ en esta  escritura estocastica. Entonces,  sufice probar
  que    $p_S(s)    \propto     s^{d-1}    \left(    1    -    \frac{s^2}{\nu+2}
  \right)_+^{\frac{\nu-d}{2}}$. Por eso, sea
  %
  \[
  T = \sqrt{\frac{V+R^2}{\nu+2}} \qquad \mbox{y} \qquad g: (r,v) \mapsto (s,t)
  \]
  %
  con ambas  \ $(r,v) \in  \Rset_+^2$ \ y  \ $(s,t) \in \Rset_+^2$.   Se calcula
  sencillamente  la  jacobiana  de $g^{-1}:  (s,t)  \mapsto  \left(  s t  ,  t^2
    (\nu+2-s^2) \right)$ y a continuaci\'on el valor absoluto de su determinente
  que vale
  %
  \[
  \left| \Jac_{g^{-1}} \right| = 2 \, (\nu+2) \, t^2
  \]
  %
  Del teorema de transformaci\'on~\ref{Teo:MP:TransformacionInyectivaDensidad} y
  de la independencia de $R$ y $V$ tenemos
  %
  \begin{eqnarray*}
  p_{S,T}(s,t) & = & 2 \, (\nu+2) \, t^2 \, p_R(st) \, p_V\left( t^2 (\nu+2-s^2)
  \right)\\[2mm]
  %
  & \propto & t^2 \, (s t)^{d-1} \, e^{-\frac{s^2 t^2}{2}} \, t^{\nu-d} \,
  \left( \nu+2-s^2 \right)_+^{\frac{\nu-d}{2}} \, e^{-\frac{t^2 (\nu+2-s^2)}{2}}
  \end{eqnarray*}
  %
  es decir
  %
  \[
  p_{S,T}(s,t)     \propto    s^{d-1}     \left(    1     -    \frac{s^2}{\nu+2}
  \right)_+^{\frac{\nu-d}{2}} \, t^{\nu+1} \, e^{-\frac12 t^2}
    \]
  %
    Inmediatamente, de  la forma separable de  \ $p_{S,T}$, vemos que  $S$ y $T$
    son independientes, y sobre todo se recononce en el primer factor que la ley
    de  $S$ es dada  por $p_S(s)  \propto s^{d-1}  \left( 1  - \frac{s^2}{\nu+2}
    \right)_+^{\frac{\nu-d}{2}}$, lo que cierra la prueba. Pasando, de la ley de
    $T$ se notar\'a que $T^2 \sim \G\left( \frac{\nu}{2} + 1 , \frac12 \right)$,
    como suma  de dos variables  gamma independientes respectivamente \  $V \sim
    \G\left(  \frac{\nu-d}{2}+1 , \frac12  \right)$ \  y \  $R^2 =  \|G\|^2 \sim
    \G\left(        \frac{d}{2}       ,       \frac12        \right)$       (ver
    lemma~\ref{Lem:MP:StabilidadGamma}). $T$ es de distribuci\'on raiz-gamma.
 %   Por marginalisaci\'on en $t$, se  obtiene el resultado esperado. Pasando, de
 %   la forma separable de \ $p_{S,T}$,  vemos que $S$ y $T$ son independientes y
 %   que $T^2 \sim \G\left( \frac{\nu}{2} , \frac12 \right)$.
\end{proof}

M\'as propiedades de esta distribuci\'on se encuentran en libros especializados,
por ejemplo~\cite{FanKot90, KotBal00} o en~\cite[Sec.~3.2.1]{Zoz12}.

\

\

La distribuci\'on  Student-$r$ se generaliza al  caso complejo \  $Z$ \ definido
sobre $\Cset^d$; se denota  \ $Z \, \sim \, \CR_\nu(m,\Sigma)$ \  donde \ $m \in
\Cset^d$, \ $\Sigma \in \Pos_d^+(\Cset)$, \ $\nu > 2 d - 2$ \ y la densidad es dada
por \
%
\[
p_Z(z) = \frac{\Gamma\left( \frac{\nu}{2} + 1 \right)}{\pi^d (\nu+2)^d \Gamma\left(
    \frac{\nu}{2} - d + 1   \right)  \,   \left|   \Sigma  \right|}   \:   \left(  1   -
  \frac{(z-m)^\dag \, \Sigma^{-1} \, (z-m)}{\nu+2} \right)_+^{\frac{\nu}{2}-d}
\]

\

\SZ{Generalizaci\'on matriz variada?}
% \SZ{  Como  en el  contexto  Student-$t$,  se  puede imaginar  generalizar  la
%   Student-$r$ al caso matriz-variada  $X$ definido sobre $M_{d,d'}(\Rset)$; se
%   denotar\'a  \  $X \,  \sim  \, \R_\nu(M,\Sigma,\Omega)$  \  donde  \ $M  \in
%   M_{d,d'}(\Rset), \: \Sigma \in  \Pos_d^+(\Rset), \: \Omega \in \Pos_{d'}^+(\Rset)$
%   y  la  densidad  dada   por  $\displaystyle  p_X(x)  =  \frac{\Gamma_d\left(
%       \frac{\nu+d+d'-1}{2}\right)}{\pi^{\frac{\nu     d}{2}}    \Gamma_d\left(
%       \frac{\nu+d-1}{2}\right) \,  \left| \Sigma \right|^{\frac{d'}{2}} \left|
%       \Omega  \right|^{\frac{d}{2}}}  \:  \left|  I  +  \Sigma^{-1}  (x-M)  \,
%     \Omega^{-1}  \,  (x-M)^t  \right|^{-  \frac{\nu+d+d'-1}{2}}$.  Se  refiera
%   a~\cite[Cap.~4]{GupNag99} para tener m\'as detalles.
%}



\centerline{\underline{\hspace{10cm}}}

\SZ{  von Mises  en  el circulo?  Y  multivariate (von  Mises-Fisher)? Cantor  =
singular? A ver con la idea de mixta}

\

Gaussian ensembles? Wigner ensembles? Invariant matrix  ensembles? Tracy-Widom


% --------------------------------- Familia  exponencial
%% --------------------------------- Familia  exponencial
\subseccion{Familia exponencial}
\label{Ssec:MP:FamiliaExponencial}

% Familia exponencial:
% --------------------
% Kay 1993, p. 110 & 124
% Lehman & Casela 1998, p. 23
% Kotz & Balakrishnan 2000, p. 659
% Robert 2007, p. 115
% van den Bos 2007, p. 30
% Cencov 1982, p. 245, 279
% Ibarola Perez 2012, p. 174
% Mukhopadhyay 2000, p. 141


% Estadistica suficiente:
% -----------------------
% Kay 1993, p. 102-103
% Lehman & Casela 1998, p. 32-44
% Robert 2007, p. 14
% Cencov 1982, p. 28
% Ibarola Perez 2012, p. 163
% Mukhopadhyay 2000, p. 284

Muchas de las leyes que hemos visto, que sean discreta o continuas, partenecen a
una clase que comparte propriedades  particulares, y que juega un rol particular
que sea en f\'isica (ej. en  problema de maximizaci\'on de entrop\'ia de Shannon
como lo vamos a ver en  el c\'apitulo~\ref{Cap:SZ:Informacion}) o en el marco de
la inferencia bayesiana. Esta clase es la familia dicha exponencial~\cite{Dar35,
  Koo36, And70, LehCas98, IbPer12,  Muk00, KotBal00, Rob07, Bos07, Cen82, Kay93,
  NieNoc10}.  En inferencia bayesiana,  cuando una distribuci\'on de sampleo cae
en esta  familia, se  puede por  ejemplo deducir a  priori conjugados,  es decir
tales que  la distribuci\'on dicha a  posteriori caiga en la  misma familia (\ie
tenga la  misma forma  param\'etrica pero con  par\'ametros dependientes  de los
datos)~\footnote{Ver nota de pie~\footref{Foot:MP:BayesPrior}. Recordamos que si
  observaciones tienen  una distribuci\'on \ $p_{X|\Theta=\theta}(x)$  \ con una
  distribuci\'on a  priori del par\'ametro  $p_\Theta(\theta)$, por la  regla de
  Bayes el a  posteriori, es decir la ley del  parametro dados las observaciones
  es   dada  por   \   $p_{\Theta|X=x}(\theta)  \propto   p_{X|\Theta=\theta}(x)
  p_\Theta(\theta)$  \ con  \ $\propto$  \ significando  ``proporcional  a''. Si
  $p_{\Theta|X=x}$ \ tiene la misma forma param\'etrica que \ $p_\Theta(\theta)$
  \ inferir  \ $\theta$ \ con  datos que se  observan se reduce a  actualizar el
  par\'ametro de la  ley a posteriori.}. Notar que esta familia  es conocido o a
veces  {\em   familia  de  Koopman-Darmois}  debido  a   la  introducci\'on  por
Koopman~\cite{Koo36}  o  Darmois~\cite{Dar35}   en  los  a\~nos  1935-1936  (ver
tambi\'en Pitman~\cite{Pit36}) y es definida de la manera siguiente:
%
\begin{definicion}[Familia exponencial y exponencial natural]\label{Def:MP:FamiliaExponencial}
%
  Sea  \ $X$  \  vector aleatorio  definido  sobre \  $\X  \subset \Rset^d$,  de
  densidad  de probabilidad  \  $p_X$ \  \ con  respecto  a una  medida \  $\mu$
  (discreta, continua o  mixta). La distribuci\'on de probabilidad  \ $p_X$ \ es
  dicha de  la {\em familia  exponencial de orden  $k$}, $k \in \Nset^*$,  si se
  escribe de la forma
  %
  \[
  p_X(x) = C(\theta) \, h(x) \, \exp\left( \eta(\theta)^t S(x) \right)
  \]
  %
  donde
  %
  \[
  C:  \Theta \subset  \Rset^m \mapsto  \Rset_+,  \qquad h:  \X \mapsto  \Rset_+,
  \qquad \eta: \Theta \mapsto \Rset^k, \qquad S: \X \mapsto \Rset^k
  \]
  %
  con \ $m \in \Nset$.  En otras palabras, la familia exponencial es una familia
  parametrica  de la  forma as\'i  definida, con  $\Theta$ \  el espacio  de los
  par\'ametros. La familia es dicha {\em exponencial natural} si tiene la forma
  %
  \[
  p_X(x) = \frac{1}{Z(\eta)}  \, h(x) \, \exp\left( \eta^t  S(x) \right) = h(x)
  \, \exp\left( \eta^t S(x) - \varphi(\eta) \right)
  \]
  %
  con
  %
  \[
  \eta \in \mathrm{N} \subset \Rset^k, \qquad Z: \Rset^k \mapsto \Rset_+, \qquad
  \varphi = \log(Z), \qquad h: \X \mapsto \Rset_+, \qquad S: \X \mapsto \Rset^k
  \]
  %
  donde  \ $\displaystyle  \mathrm{N} =  \left\{  \eta \in  \Rset^k \tq  \int_\X h(x)  \,
    \exp\left(  \eta^t  S(x) -  \varphi(\eta)  \right)  \,  d\mu(x) <  +  \infty
  \right\}$ \ es (convexo y) llamado {\em espacio de par\'ametros naturales}.
\end{definicion}
%
Se notara  que con  la reparametrizaci\'on \  $\eta(\theta)$ \ se  puede siempre
(por lo  menos formalmente) escribir una  ley de la familia  exponencial bajo su
forma natural. Con respeto a cada termino:
%
\begin{itemize}
\item  $S(X)$ es  llamada {\em  estad\'istica suficiente}  o  {\em estad\'istica
    exaustiva}. Esta  denominaci\'on viene  del hecho que  el conocimiento  de \
  $S(X)$  \  es  suficiente para  estimar  $\eta$,  o  un resumen  exaustivo  de
  $\eta$. En particular, la estimaci\'on de verosimilitud~\footnote{El estimador
    del m\'aximo  de verosilitud \ $\eta_{\mathrm{mv}}$  \ es el \  $\eta$ \ que
    m\'aximiza \ $p_X$  \ o cualquier funci\'on creciente  como el logaritmo por
    ejemplo.    Para   la   familia   exponencial,  eso   da   sencillamente   \
    $\eta_{\mathrm{mv}}$  \  satisficiendo  \  $S(x)  =  \nabla  \varphi(\eta)$,
    dependiente solamente de  $S(x)$.\label{Foot:MP:MLE}}, o cualquier estimador
  Bayesiano~\footnote{Ver nota  de pie~\footref{Foot:SZ:BayesPrior}. Recuerdense
    que se modeliza  el par\'ametro como aleatorio, as\'i  que la distribuci\'on
    que se considera se ve como la distribuci\'on condicional \ $p_{X|N=\eta}(x)
    =   h(x)  \exp\left(   \eta^t  S(x)   -  \varphi(\eta)   \right)$,   con  la
    distribuci\'on  a  priori   \  $p_N(\eta)$.   De  la  regla   de  Bayes,  la
    distribuci\'on a  posteriori, \ie del par\'ametro dadas  las observaciones \
    $x$ \  se escribe \ $p_{N|X=x}(\eta) \propto  p_{X|N=\eta}(x) p_N(\eta)$. Si
    se da un  costo de estimaci\'on \ $C(\widehat{\eta},N)  > 0$ \ caracerizando
    la  ``distancia'' entre  la estimaci\'on  y el  parametro  ``verdadero'', se
    busca la funci\'on  $\widehat{\eta}$ que minimiza el costo  promedio, lo que
    es equivalente para cada \ $x$ \  a buscar el valor \ $\widehat{\eta}$ \ que
    minimiza \ $\displaystyle \int_N C(\widehat{\eta},\eta) \, \exp\left( \eta^t
      S(x)  - \varphi(\eta)  \right) \,  p_N(\eta)  \, d\mu(\eta)$~\cite{Rob07}:
    claramente,  el m\'inimo depende  solamente de  $S(x)$.}  de  $\eta$ depende
  solamente  de $S(X)$~\cite{Kay93,  LehCas98, Rob07,  Cen82,  IbaPer12, Muk00}.
  Formalmente, una  estadistica \ $T(X)$ \  es suficiente para  un par\'ametro \
  $\theta$ \ si la  distribuci\'on de \ $X$ \ condicionalmente a  \ $S(X) = s$ \
  no depende m\'as de \ $\theta$.   Por ejemplo, para la familia exponencial, en
  el caso discreto, tenemos,\ $\displaystyle p_{X|S(X) = s}(x) = \frac{P\big( (X
    = x) \cap (S(X) = s) \big)}{P(S(X)  = s)} = \frac{h(x) \exp\left( \eta^t s -
      \varphi(\eta)  \right) \un_{\{  S(x) \}}(s)}{\displaystyle  \sum_{x\in \X}
    h(x)  \exp\left( \eta^t  s -  \varphi(\eta) \right)  \un_{\{ S(x)  \}}(s)} =
  \frac{h(x)  \un_{\{ S(x)  \}}(s)}{\displaystyle \sum_{x  \in \X}  h(x) \un_{\{
      S(x)   \}}(s)}$   \   no   depende   de   \   $\eta$.    Veremos   en   el
  capitulo~\ref{Cap:SZ:Informacion},   secci\'on~\ref{Ssec:SZ:Fisher}   que   la
  informaci\'on de  Fisher en $\eta$,  medida informacional y apareciendo  en la
  cota del error cuadratico m\'inimo posible  de un estimador de \ $\eta$, es la
  covarianza  de  $S$,  mostrando  de  nuevo  que  \  $S$  \  es  sufiente  para
  caracterizar      \      $\eta$.        En      este      mismo      capitulo,
  secci\'on~\ref{Ssec:SZ:MaxEnt},  veremos  que,  sujeto  a \  $\Esp[S(X)]$,  la
  distribuci\'on  que m\'aximiza  la  entrop\'ia, medida  de  incerteza, es  una
  distribuci\'on  exponencial, enfatizando el  rol de  esta familia  en f\'isica
  cuando se tiene la media de una estadistica fija (ej. energ\'ia fija).
%
\item $\theta$  es el par\'ametro, escalar  o multivariado, y  $\eta$ es llamado
  {\em par\'ametro natural}.
%
\item  La funci\'on  $Z$ es  llamada {\em  funci\'on de  partici\'on};  A veces,
  $\varphi$ es as\'i llamada {\em  log-funci\'on de partici\'on}; no solo juegan
  un  rol en  la  normalizaci\'on de  la  ley, pero  tienen una  significaci\'on
  f\'isica  como lo vamos  a evocar.  Aparecio $Z$  en f\'isica  esdatistica por
  ejemplo en trabajos de Gibbs~\cite{Gib01, Gib02}.
\end{itemize}

Se notara que,  en particular, \ $Z$ \  es relacionada a los momentos  de $S$, o
$\varphi$ a los cumulantes:
%
\begin{teorema}[Funci\'on de partici\'on y generadoras]
%
  Sea \  $X$ \  de distribuci\'on exponencial  natural de par\'ametro  natural \
  $\eta$ \ y estad\'istica suficiente \ $S$  \ y denotamos \ $Z$ la funci\'on de
  partici\'on y  $\varphi$ su logaritmo.  Entonces,  las funci\'ones generadoras
  de los  momentos y de  los cumulantes \  $M_{S(X)}$ \ y  \ $C_{S(X)}$ \  de la
  estad\'istica  suficiente \  $S(X)$ \  son relacionadas  a  \ $Z$  \ y  \ a  \
  $\varphi = \log  Z$ \ por, \ $\forall \:  u \: \mbox{ tal que  } \: u+\eta \in
  N$,
  %
  \[
  M_{S(X)}(u) =  \frac{Z(u+\eta)}{Z(\eta)} \qquad \mbox{y}  \qquad C_{S(X)}(u) =
  \varphi(u+\eta) - \varphi(\eta)
  \]
  %
  En particular, si \  $\varphi$ \ es diferenciable tenemos
  %
  \[
  \nabla \varphi (\eta) = \Esp\left[ S(X) \right]
  \]
  %
  y si \ $\varphi$ \ es dos veces diferenciable tenemos
  %
  \[
  \Hess \varphi (\eta) = \Cov\left[ S(X) \right]
  \]
  %
  Pasando, de  \ $\Cov\left[ S(X)  \right] \ge 0$  \ tenemos que, hessiana  de \
  $\varphi$ \ siendo positiva, \ $\varphi$ \ es convexa~\cite{CamMar09}.
%
\end{teorema}
%
\begin{proof}
  De la definici\'on de la funci\'on generadora de \ $S(X)$, tenemos
  %
  \begin{eqnarray*}
  M_{S(X)}(u) & = &\Esp\left[ \exp\left( u^t S(X) \right) \right]\\[2mm]
  %
  & = & \int_\X \frac{1}{Z(\eta)} \, h(x) \, \exp\left( (u + \eta)^t S(x) \right) \,
  d\mu(x)\\[2mm]
  %
  & = & \frac{Z(u+\eta)}{Z(\eta)} \int_\X \frac{1}{Z(\eta)} \, h(x) \,
  \exp\left( (u + \eta)^t S(x) \right) \, d\mu(x)\\[2mm]
  %
  & = &  \frac{Z(u+\eta)}{Z(\eta)}
  \end{eqnarray*}
  %
  La secunda  relaci\'on es inmediata  de \ $C_X  = \log M_X$ \  conjuntamente a
  \ $\varphi = \log Z$.

  A continuaci\'on, los cumulantes y momentos coinciden hasta el orden $3$, dando inmediatamenta
  %
  \begin{eqnarray*}
  \Esp\left[ S(X) \right] & = & \left. \nabla_u C_{S(X)} \right|_{u=0}\\[2mm]
  %
  & = & \left. \nabla_u \left( \varphi(u+\eta) - \varphi(\eta) \right) \right|_{u=0}\\[2mm]
  %
  & = & \nabla \varphi(\eta)
  \end{eqnarray*}
  %
  y
  %
  \begin{eqnarray*}
  \Cov\left[ S(X) \right] & = & \left. \Hess_u C_{S(X)} \right|_{u=0}\\[2mm]
  %
  & = & \left. \Hess_u \left( \varphi(u+\eta) - \varphi(\eta)  \right) \right|_{u=0}\\[2mm]
  %
  & = & \Hess \varphi(\eta)
  \end{eqnarray*}
\end{proof}

En problemas  de estimaci\'on, frecuentemente, se tiene  $n$ vectores aleatorios
independiente  de misma  ley que  se  usan para  estimar un  p\'arametro. En  la
familia exponencial, resuelte que  la distribuci\'on conjunta en tal situaci\'on
queda en la familia exponencial:
%
\begin{lema}
  Sean \  $X_1 , \ldots ,  X_n$ \ vectores aleatorios,  independientes, de misma
  ley de la familia exponencial natural,  de estadistica suficiente \ $S$ \ y de
  log-funci\'on de  partici\'on \ $\varphi$. Entonces  la ley conjunta  de los \
  $X_i$ \ cae  el la familia exponencial  de mismo orden que el  de \ $p_{X_i}$,
  con  el   mismo  par\'ametro,  de  estadistica   suficiente  \  $\displaystyle
  (X_1,\ldots,X_n)  \mapsto  \sum_{i=1}^n  S(X_i)$   \  y  de  log-funci\'on  de
  partici\'on \ $n \varphi$.
\end{lema}
%
\begin{proof}
  El resulta  es inmediato siendos  los \ $X_i$  \ independientes, dando  la ley
  conjunta como el producto de las leyes de los $X_i$.
\end{proof}

Muchas  distribuciones caen  en  la  familia exponencial,  que  sean discreta  o
continua, como lo vamos a ver en dos ejemplos.
%
\begin{ejemplo}
  Sea  \ $p_X$  \ distribuci\'on  de Bernoulli  \ $\B(p)$.   Esta distribuci\'on
  partenece a la familia exponencial de  orden $1$.  De hecho, se puede escribir
  la ley \ $p_X(x) = p^x (1-p)^{1-x}$ \ bajo la forma
  %
  \[
  p_X(x) = \exp\left( x \log \left( \frac{p}{1-p} \right) + \log(1-p) \right)
  \]
  %
  Aparece  que el par\'ametro  natural es  \ $\eta  = \log  \left( \frac{p}{1-p}
  \right)$ \  y la estadistica  suficiente correspondiente es  \ $S(X) =  X$. En
  particular, para \  $X_i, \: i = 1,  \ldots , n$ \ independientes  tales que \
  $X_i  \sim  \B(p)$, una  estadistica  sufficiente de  la  ley  conjunta es  el
  promedio empirico $\displaystyle  \widebar{X} = \frac{1}{n} \sum_{i=1}^n X_i$.
  Aparece  que  el estimador  de  verosimilitud  m\'axima~\footnote{Ver nota  de
    pie~\footref{Foot:MP:MLE}}  es  precisamente  el  promedio empirico;  es  el
  estimador  \  $\widehat{p}$  \   de  error  cuadratico  \  $\Esp\left[  \left(
      \widehat{p} - p \right)^2 \right]$ \ m\'inimo (ver ej.~\cite{Kay93}).
\end{ejemplo}

\begin{ejemplo}
  Sea \ $p_X$ \ distribuci\'on gamma \ $\G(a,b)$.  Esta distribuci\'on partenece
  a la familia exponencial  de orden $2$.  De hecho, se puede  escribir la ley \
  bajo la forma
  %
  \[
  p_X(x)     =    \frac{1}{x}    \,     \exp\left(    \begin{bmatrix}     a    &
      b \end{bmatrix} \begin{bmatrix}  \log x \\ - x  \end{bmatrix} - \log\left(
      \frac{\Gamma(a)}{b^a} \right)\right)
  \]
  %
  El   par\'ametro  natural   es   as\'i   \  $\eta   =   \begin{bmatrix}  a   &
    b \end{bmatrix}^t$ \ y la  estadistica suficiente correspondiente es \ $S(X)
  = \begin{bmatrix} \log x & - x \end{bmatrix}^t$.
\end{ejemplo}

Si muchas distribuciones  partenecen a la familia exponencial,  no todas caen en
esta familia:
%
\begin{ejemplo}
  Sea  \   $p_X$  \   distribuci\'on  Student-$t$  \   $\T_\nu(m,\Sigma)$.  Esta
  distribuci\'on no cae el la  familia exponencial. De hecho, se puede todav\'ia
  escribir la ley bajo la forma
  %
  \[
  p_X(x)   =  C(\nu,\Sigma)   \exp\left(  -   \frac{d+\nu}{2}  \log\left(   1  +
      \frac{(x-m)^t \Sigma^{-1} (x-m)}{\nu} \right) \right)
  \]
  %
  A    pesar   de    que    la   ley    parece    tener   la    forma   de    la
  definici\'on~\ref{Def:MP:FamiliaExponencial}    con   \    $\eta(\nu)    =   -
  \frac{d+\nu}{2}$,  su  factor  \  $\log\left( 1  +  \frac{(x-m)^t  \Sigma^{-1}
      (x-m)}{\nu} \right)$  \ no es funci\'on  \'unicamente de \  $x$; se quedan
  todo los par\'ametros \ $\nu, \: m, \: \Sigma$. A\'un si un de esos es fijo, y
  no visto como par\'ametro, quedar\'a un parametro en este termino.
\end{ejemplo}

Se puede tambi\'en que una ley  partenece o no a la familia exponencial, seg\'un
que unos par\'ametros sean fijos  (de hecho, no on par\'ametros m\'as entonces),
o no:
%
\begin{ejemplo}
  Sea \  $p_X$ \ distribuci\'on  binomial \  $\B(n,p)$. Si \  $n$ \ es  fijo, es
  decir  no  visto  como  par\'ametro,  la  distribuci\'on  cae  el  la  familia
  exponencial  de orden $1$.  Si se  consideran ambos  \ $n$  \ y  \ $p$  \ como
  par\'ametros,  no  cea m\'as  en  la familia  exponencial.  Para  ver eso,  se
  escribir la ley bajo la forma
  %
  \[
  %begin{eqnarray*}
  p_X(x) = \frac{n!}{x! (n-x)!}  \exp\left( x \log\left( \frac{p}{1-p}
    \right) + n \, \log(1-p) \right)
  %\\[2mm]
  %
  %& = & \frac{1}{x!}  \exp\left( - \log\left( (n-x)! \right) + x \log\left(
  %\frac{p}{1-p} \right) + n \, \log(1-p) + \log(n!) \right)
  %\end{eqnarray*}
  \]
  %
  Entonces, si  \ $n$  \ es fijo,  se concluye que  \ $p_X$  \ es en  la familia
  exponencial de  orden $1$, de  par\'ametro \ $\eta =  \log\left( \frac{p}{1-p}
  \right)$  \  y estad\'istica  suficiente  correspondiente  \  $S(x) =  x$.  Al
  rev\'es,  si \  $n$ \  es un  par\'ametro  (que \  $p$ \  sea fijo  o no),  en
  $(n-x)!$, no se puede  ``separar'' \ $x$ \ de \ $n$  \ y tampoco escribir este
  termino  de la  forma \  $\exp(  f(n) g(x)  )$: la  ley  no es  de la  familia
  exponencial mas.
\end{ejemplo}

De las distribuciones que hemos visto:
%
\begin{itemize}
\item Caen en  la familia exponencial la leyes: de  Bernoulli, binomial cuando \
  $n$ \ es fijo, negative binomial cuando  \ $r$ \ es fijo, multinomial cuando \
  $n$ \  es fijo,  geometrica, de poisson,  gausiana~\footnote{En este  caso, se
    puede ver  el par\'ametro  natural como \  $\left( \Sigma^{-1} m  , -\frac12
      \Sigma^{-1}  \right)$ \  en  lugar del  vector  formado de  los \  $\left(
      \Sigma^{-1} m \right)_i$ \ y \ $-\frac12 \left( \Sigma^{-1} \right)_{i,j},
    \: 1 \le i \le j \le d$ \ y la estad\'istica suficiente como \ $\left( x , x
      x^t \right)$  \ siendo  $x^t \Sigma^{-1} x  = \Tr\left( \Sigma^{-1}  x x^t
    \right)$, en lugar  del vector formado de los  \ $x_i$ \ y \ $x_i  x_j, \: 1
    \le  i  \le  j \le  d$  (de  la  simetr\'ia).   El  orden es  $d  +  \frac{d
      (d+1)}{2}$.}, gamma, Wishart~\footnote{De nuevo, en este caso se puede ver
    el  par\'ametro  natural  formalmente  como \  $\left(  \frac{\nu-d-1}{2}  ,
      -\frac12 V^{-1}  \right)$ \ y  la estad\'istica suficiente como  \ $\left(
      \log |x|  , x \right)$.  El orden es  $1 + \frac{d (d+1)}{2}$.},  beta, de
  Dirichlet.
%
% resp. eta = log(p/(1-p)), log(p/(1-p)), log p, log p = [log p_1 ... \log p_k],
%       log(1-p), log lambda, (Sigma^{-1} m , -1/2 Sigma^{-1}), [a b],
%       ( (nu-d-1)/2 , -1/2 V^{-1}), [a b], a = [a_1 ... a_k]
% resp. S = x, x, x, x = [x_1 ... x_k], x,
%       x, (x,x x^t), [log x -x], (log |x| , x),
%       [log x  log(1-x)], [log x_1 ... log(x_k)]
%
\item No partenecen a la familia  exponencial las leyes: binomial cuando \ $n$ \
  es  un  par\'ametro, negative  binomial  cuando \  $r$  \  es un  par\'ametro,
  multinomial cuando \ $n$ \ es un par\'ametro, hipergeometrica, hipergeometrica
  negativa,     hipergeometrica      multivaluada~\footnote{En     los     casos
    hipergeometricos, har\'ia falta que sean  fijos respectivamente \ $n, m, k$,
    \ $n,  r, k$  \ y  \ $n,  m, k_1, \ldots  , k_c$  \ y  la leyes  no ser\'ian
    parametricas mas.}, Student-$t$,  uniformas~\footnote{Eso viene del hecho de
    que el soporte depende de los par\'ametros.}.
\end{itemize}

Las distribuciones exponenciales aparecen frecuentemente en f\'isica estadistica
a  trav\'es de  la teor\'ia  de Boltzmann~\cite{Bol96,  Bol98,  Gib02, LanLif80,
  MezMon09,  Mer10,  Mer18}.  Adem\'as,  cuantidad  f\'isica  se  derivan de  la
log-funci\'on partici\'on~\cite{Max67, Gib02, LanLif80, MezMon09, Mer10, Mer18}:
%
\begin{ejemplo}
  En f\'isica estadistica, se enfrente al problema de descripci\'on macroscopico
  de un sistema  de muchas particular (ej. hirviente de  un liquido). Hay tantas
  particulas que no se puede estudiar tales sistemas con las leyes usuales de la
  mec\'anica, as\'i que se usa  un enfoque probabilistico. Por eso, se considera
  un espacio  \ $\X$  \ $d$-dimensional dicho  {\em espacio  de configuraciones}
  (puede ser  discreto o  continuo). En \  $x =  \begin{bmatrix} x_1 &  \cdots &
    x_d\end{bmatrix}$,  cada  \ $x_i$  \  representa el  {\em  estado}  de la  \
  $i$-\'esima particula (posici\'on,  velocidad, esp\'in,\ldots).  Lo importante
  es que a un  tipo de sistema se asocia una funci\'on  energ\'ia \ $\E(x)$. Por
  ejemplo, en un sistema  sin interacciones, $\displaystyle \E(x) = \sum_{i=1}^d
  \E_i(x_i)$.  El  el caso  del gaz perfecto,  $\displaystyle \E(x) =  \frac12 m
  \sum_{i=1}^d x_i^2$ \ donde  \ $m$ \ es la masa de cada  particula y \ $x_i$ \
  la  velocidad  de  la  \  $i$-\'esima particula  (espacio  de  configuraciones
  continuo).   En  el  {\em   modelo  ferromagnetico  de  Ising},  se  considera
  particulas en una ret\'icula y \ $x_i = \pm 1$ \ es el esp\'in de la particula
  \ $i$ (espacio de configuraciones  discreto).  Sometido a un campo magnetico \
  $B$,  la energ\'ia  es dada  por  \ $\displaystyle  \E(x) =  - \sum_{(i,j)  \:
    \mbox{\tiny  vecinos}} x_i  x_j  - B  \sum_{i=1}^d x_i$~\cite{Len20,  Isi25,
    Ons44, LanLif80, MezMon09, Mer10, Mer18}.   Se puede poner pesos \ $J_{i,j}$
  \ en cada vecinos,  positivos para interracciones ferromagneticos, y negativos
  para  interacciones antiferromagneticos  (modelos {\em  vidrio de  esp\'in}, o
  m\'as  exactamente  de  Edwards-Anderson~\cite{EdwAnd75,  LanLif80,  MezMon09,
    Mer10,  Mer18}).  El  modelo de  Curie-Weiss~\footnote{Fue llamado  as\'i en
    relaci\'on a los trabajos  de P.  Curie~\cite{Cur95} y P. Weiss~\cite{Wei96,
      Wei07}  sobre los materiales  ferromagneticos.}  se  presenta de  la misma
  manera, con la energ\'ia $\displaystyle \E(x) = - \frac{1}{d} \sum_{i\ne j) \:
    \mbox{\tiny  pares}} x_i  x_j +  B \sum_{i=1}^d  x_i$~\cite{MezMon09, Mer10,
    Mer18}.

  Seg\'un  la  teoria  de  Gibbs-Boltzmann,  la  dicha  {\em  distribuci\'on  de
    Gibbs-Boltzman}  asociada  a  un  espacio  de configuraci\'on  y  modelo  de
  energ\'ia es dada por
  %
  \[
  p_X(x) = \frac{1}{Z(\beta)}  \exp\left( - \beta \E(x) \right),  \qquad \beta =
  \frac{1}{k_B T}
  \]
  %
  donde $k_B \approx  1.38 \times 10^{-23}$ julio por kelvin  es la constante de
  Boltzmann,  y  \ $T$  \  es la  temperatura  en  kelvin.  Esta  distribuci\'on
  partenece claramente a la familia exponencial natural de par\'ametro \ $\beta$
  \  y de  estad\'istica suficiente  \ $-\E(x)$  (ac\'a, $h  = 1$).  En f\'isica
  estadistica, la  log-funci\'on de partici\'on  aparece en varias  cantidades y
  potenciales f\'isicos:
  %
  \[
  F(\beta) = - \frac{1}{\beta} \log Z(\beta)
  \]
  %
  es  la  {\em  energ\'ia  libre}  o  {\em energ\'ia  libre  de  Helmholtz}  del
  sistema.  Es la  energ\'ia disponible  (o  que se  puede usar)  de un  sistema
  aislado.

  Luego, se define
  %
  \[
  U(\beta) =  \frac{\partial}{\partial \beta} \left( \beta F(\beta)  \right) = -
  \frac{\partial \log Z(\beta)}{\partial \beta} = \Esp\left[ \E(X) \right]
  \]
  %
  donde  \  $X$ \  ser\'ia  el  vector aleatorio  de  distribuci\'on  \ $p_X$.   \
  $U(\beta)$ \ es  la {\em energ\'ia interna} del  sistema, promedio estadistico
  de la energ\'ia a trav\'es de todas las configuraciones posibles.

  Se define  tambi\'en una medida de  incerteza llamada {\em  entrop\'ia} o {\em
    entrop\'ia  de  Gibbs}~\cite{Bol77, Bol96,  Bol98,  Gib02, Jay65,  LanLif80,
    MezMon09, Mer10,  Mer18}.  Esta medida  caracteriza las fluctuaciones  de la
  energ\'ia libre~\footnote{La  letra \ $S$ \ se  uso historicamente. Obviamente
    no corresponde a la estadica sufficiente que es ac\'a \ $\E$.},
  %
  \[
  S(\beta) = \beta^2 \frac{\partial}{\partial \beta} F(\beta)
  \]
  %
  Aparece   por   un  lado   que   \  $S(\beta)   =   \log   Z(\beta)  -   \beta
  \frac{\partial}{\partial \beta} \log Z(\beta)$  \ es decir, reconiciendo en el
  primer t\'ermino \ $- \beta F(\beta)$ \ y en el secundo \ $\beta U(\beta)$,
  %
  \[
  F = U - k_B T S
  \]
  %
  conocido como transformada de Legendre de la energ\'ia interna, y consecuencia
  de la  primera ley de  la termodynamica. Aparece tambi\'en  que $\displaystyle
  S(\beta) = \log Z(\beta) + \beta  \Esp[ \E(X) ] = \int_\X \left( \log Z(\beta)
    + \beta \E(x) \right) p_X(x) \, d\mu(x)$ \ es decir
  %
  \[
  S = - \int_\X p_X(x) \, \log p_X(x) \, d\mu(x)
  \]
  %
  Volveremos    en    esta    definici\'on    de    la    entrop\'ia    en    el
  capitulo~\ref{Cap:SZ:Informacion} en un marco m\'as general.
\end{ejemplo}

\SZ{Hablar de estadistica suficiente m\'inima?}
% Ver information geometry
% https://en.wikipedia.org/wiki/Partition_function_(mathematics)

%\SZ{Van den Bos 2007, p. 33, informaci\'on de Fisher}


% --------------------------------- Familia eliptica
%\def\izq{\mathrm{i}}
%\def\der{\mathrm{d}}

% --------------------------------- Familia  eliptica
\subseccion{Familia el\'iptica}
\label{Ssec:MP:FamiliaEliptica}


% --------------------------------- Familia  eliptica real

\subsubseccion{Caso real}
\label{Ssec:MP:FamiliaElipticaReal}

% --------------------
% Bilingsley p. 
% Kay p.
% Lehman & Casela p. 
% Kotz & Balakrishnan p.
% Robert p.
% van den Bos p.
% Cencov p.
% Ibarola Perez p.
% Mukhopadhyay p.

El estudio de  estos vectores es bastante antigua. Hace  falta volver a trabajos
de Maxwell en 1867 sobre la teor\'ia del gas para encontrar unas de las primeras
menciones  a este  formalismo~\cite{Max67}  o~\cite[pp.~377--391]{Nie52:v1}.  El
problema de Maxwell era de encontrar una distribuci\'on (tridimensional) que sea
isotr\'opica  y  separable  a  la  vez:  monstr\'o  que  tal  distribuci\'on  es
necesariamente    gaussiana    (es    ahora    conocido    como    teorema    de
Maxwell-Hershel~\footnote{Ver~\cite[Prop.~4.11]{BilBre99}.   Se notar\'a  que no
  hay muchas menciones  de este teorema bajo esta  denominaci\'on. no sabemos si
  la raz\'on es que  no tienen ni Maxwell, ni Herschel la  partenidad o si no lo
  revendicaron. Sin embargo, ver~\cite{Max67}}.   Volveremos en este teorema. La
clase de las distribuciones el\'iptical, o a simetr\'ia el\'iptica fue estudiada
intensivamente  formalmente~\cite{Bar34,  Bar34:07,  Ver64, McgWag68,  CamHua81,
  Eat81,  Kan94,  Lau75,  Yao73,  KotNad01,  FanKot90,  Mui82,  BilBre99}.   Fue
tambi\'en  usadas   en  aplicaciones  en   estad\'istica~\cite{BlaTho68,  Chu73,
  YanKot03,  ArePin06,  BauPas07,  ChiPas08},   o  procesamiento  de  se\~nal  o
imagenes~\cite{Gol76, RanWei93, RanWei95, ZozVig10, Zoz12}, entre otros.

Empezamos  por  la  definici\'on,  antes  de  ir  m\'as  all\'a  estudiando  sus
propiedades remarcables.

\begin{definicion}[Vector esfericamente invariante]
  Sea  \  $X$  \  vector   aleatorio  $d$-dimensional  real.   $X$  \  es  dicho
  esfericamente  invariante,  o   rotacionalmente  invariante,  o  a  simetr\'ia
  esf\'erica,  o  de  distribuci\'on  esf\'erica  \  si  para  cualquier  matriz
  ortogonal (o de rotaci\'on)
  % ~\footnote{Recordarse que \ $O$ \ es ortogonal o de rotaci\'on si \ $O O^t =
  %   O^t O = I$.}
  \ $O \in \Ort_d(\Rset)$ (ver notaciones),
  %
  \[
  O  X  \: \egald  \: X
  \]
  %
\end{definicion}

Tales  vectores  modelizan naturalmente  fen\'omenos  isotr\'opicos. Pero  m\'as
all\'a,  se  puede  que  haya  direcciones privilegiadas  ortogonales  pero  con
simetrias,  \ie en  lugar de  simetr\'ias esf\'ericas,  simetr\'ias como  en una
pelota de rugby. Adem\'as, se puede que eso se pasa en torno a un punto no cero.
%
\begin{definicion}[Vector a simetr\'ia el\'iptica]
  Sea  \  $X$  \ vector  aleatorio  $d$-dimensional  real.   $X$  \ es  dicho  a
  simetr\'ia  el\'iptica,   o  elipticalmente  invariante,   o  de  distribuci\'on
  el\'iptica, en torno a \ $m \in  \Rset^d$, \ si existe una matriz \ $\Sigma \in
  \Pos_d^+(\Rset)$ \ tal que para cualquier matriz ortogonal \ $O \in \Ort_d(\Rset)$,
  %
  \[
  O  \,   \Delta^{-\frac12}  \,  Q^t  \left(   X  -  m  \right)   \:  \egald  \:
  \Delta^{-\frac12} \, Q^t \left( X - m \right)
  \]
  %
  donde la  matriz diagonal \ $\Delta  > 0$ \ es  la matriz de  autovalores de \
  $\Sigma$  \  y  \  $Q  \in  \Ort_d(\Rset)$  \  la  matriz  de  los  autovectores
  correspondientes~\cite{Bha97,   Bha07,  HorJoh13},  \   $\Sigma  =   Q  \Delta
  Q^t$. Dicho de  otra manera, \ $\Delta^{-\frac12} Q^t \left( X  - m \right)$ \
  es a simetr\'ia esf\'erica.

  $m$ \ es llamado {\em par\'ametro de posici\'on} y \ la matriz \ $\Sigma$ \ es
  llamada {\em matriz caracter\'istica}.
\end{definicion}
%
Se puede inmediatamente  ver que \ $\Sigma$ \ es definida  por lo menos mediante
un factor escalar. De hecho, si un \ $\Sigma$ \ conviene, cualquier \ $a \Sigma$
\ con \ $a > 0$ \ conviene tambi\'en.

Comparativamente a un  vector esf\'ericamente invariante, \ $m$ \  es el centro de
simetr\'ia,   $P$   contiene   las   direciones   de   ``estiramientos''   y   \
$\Delta^{\frac12}$  \   los  factores  de   estiramientos,  $X  \egald  m   +  P
\Delta^{\frac12} Y$ \  con \ $Y$ \ a simetr\'ia  esf\'erica. Para \ $m =  0$ \ y \
$\Delta \propto I$, se recupera obviamente un vector a simetr\'ia esf\'erica.

Como lo hemos visto, un vector aleatorio es completamente definido por su medida
de  probabilidad,  o  equivalentemente  por su  funci\'on  car\'acteristica.  La
\'ultima tiene una forma particular en el contexto el\'iptico:
%
\begin{teorema}[Funci\'ones y generadoras caracter\'isticas]\label{Teo:MP:GeneradorasCaracteristicas}
%
  Sea \ $X$ \ vector aleatorio $d$-dimensional a simetr\'ia el\'iptica en torno a
  \   $m  \in  \Rset^d$   \  y   de  matriz   caracter\'istica  \   $\Sigma  \in
  \Pos_d^+(\Rset)$. Entonces la funci\'on caract\'eristica se escribe bajo la forma
  %
  \[
  \Phi_X(\omega)  = e^{\imath  \,  \omega^t m}  \varphi_X\left( \omega^t  \Sigma
    \omega \right)
  \]
  %
  donde  \  $\varphi_X:  \Rset_+  \mapsto  [-1 \; 1]$ \  escalar,  es  llamado  {\em
    generadora  caracter\'istica}. Tomando el  logaritmo, obviamente  la secunda
  funci\'on caracter\'istica se escribe
  %
  \[
  \Psi_X(\omega) =  \imath \, \omega^t  m + \psi_X\left( \omega^t  \Sigma \omega
  \right)
  \]
  %
  donde  \ $\psi_X  =  \log \varphi_X:  \Rset_+  \mapsto \Cset$  \ escalar.   La
  llamaremos  {\em secunda generadora  caracter\'istica}. Reciprocamente,  si la
  funci\'on caracter\'istica tiene esta forma, \ $X$ \ es a simetr\'ia el\'iptica.
\end{teorema}
%
\begin{proof}
  Sea \  $Y = \Delta^{-\frac12} Q^t  \left( X - m  \right)$ \ con \  $\Sigma = Q
  \Delta  Q^t$  \ diagonalizaci\'on  de  \  $\Sigma$.   Por definici\'on  y  del
  teorema~\ref{Teo:MP:PropiedadesFuncionCaracteristica},  para  cualquier matriz
  ortogonal \ $O \in \Ort_d(\Rset)$ \ y cualquier \ $\omega \in \Rset^d$
  %
  \[
  \Phi_Y(\omega) = \Phi_{O Y}(\omega) = \Phi_Y(O^t \omega)
  \]
  %
  En  otros  terminos,  la  funci\'on  caracter\'istica  queda  invariante  bajo
  cualquier transformaci\'on  ortogonal (rotaci\'on) sobre  $\omega$, y entonces
  depende solamente de  la norma euclideana de \ $\omega$.  Es decir, existe una
  funci\'on escalar \ $\varphi_X$ \ tal que
  %
  \[
  \Phi_Y(\omega) = \varphi_X(\omega^t \omega)
  \]
  %
  De nuevo, del teorema~\ref{Teo:MP:PropiedadesFuncionCaracteristica},
  %
  \[
  \Phi_X(\omega)  \:  = \:  \Phi_{Q  \Delta^{\frac12} Y  +  m}(\omega)  \: =  \:
  e^{\imath  \, \omega^t  m} \,  \Phi_Y( \Delta^{\frac12}  Q^t \omega)  \:  = \:
  e^{\imath \, \omega^t m} \, \varphi_x( \omega^t Q \Delta Q^t \omega)
  \]
  % 
  lo que  cierra la prueba  directa.  Reciprocamente, si  \ $\Phi_X$ \  tiene la
  forma dada, para  cualquier matriz ortogonal \ $O  \in \Ort_d(\Rset)$ \ $\Phi_{O
    Y}(\omega)  = \Phi_Y( O^t  \omega )  = \Phi_Y(\omega)$  \ y,  por relaci\'on
  uno-uno  entre  la medida  de  probabilidad de  una  variable  aleatorio y  su
  funci\'on caracter\'istica, $Y \egald O Y$.

  Al final, de la simetr\'ia herm\'itica  de la funci\'on de repartici\'on, y de
  la simetr\'ia  el\'iptica, tenemos $\varphi_X^*\left(  \| \omega \|^2  \right) =
  \varphi_X\left(  \| -\omega  \|^2  \right) =  \varphi_X\left(  \| \omega  \|^2
  \right)$,  lo que proba  que \  $\varphi_X$ \  es a  valores reales,  siendo a
  valores tambi\'en de modulo menor que $\varphi_X(0) = 1$.
\end{proof}
%
Se notar\'a que si tomamos  una matriz caracter\'istica $\Sigma$ y la generadora
correspondiente  $\varphi$, \  $a  \Sigma$ \  y  \ $\varphi_X\left(  \frac{u}{a}
\right)$ \ conviene tambi\'en, lo que  es de acuerdo con la indeterminencia de \
$\Sigma$ \ bajo un factor positivo.  Se puede a\~nadir un v\'inculo, por ejemplo
fijando  \  $\Tr \Sigma$  \  para  que  \ $\Sigma$  \  y  \ $\varphi_X$  \  sean
\'unicamente definidas. Entonces, \ $X$ \ ser\'a completamente caracterizado por
\ $m, \: \Sigma$ \ y \ $\varphi_X$, y escribiremos
%
\[
X \, \sim \, \ED \left( m , \Sigma , \varphi_X \right)
\]
%
y   los  conjuntos  de   generadoras  caracter\'isticas   que  resultan   de  la
restricci\'on  de   $\PD_d$  (y  de  $\PD$)  a   las  funciones  caracteristicas
esf\'ericamente invariante como
%
\[
\PDSI_d = \big\{ 
%\begin{array}{lll} 
\varphi: \Rset_+ \mapsto [-1 ; 1]  \:\: \mbox{  continuas con } \:\: \varphi(0) = 1  \tq \Phi: 
%& \Rset^d \to \Cset & \in \PD_d \\[-1mm] & 
x \mapsto \varphi\left( \| x \|^2 \right)
% & \end{array} 
\in \PD_d \big\}
\]
%
y
%
\[
\PDSI = \bigcap_{d=1}^{+\infty} \PDSI_d = \big\{ 
%\begin{array}{lll} 
\varphi: \Rset_+ \mapsto [-1 ; 1]  \:\: \mbox{  continuas con } \:\: \varphi(0) = 1  \tq \Phi: 
%& \Rset^d \to \Cset & \in \PD_d \\[-1mm] & 
x \mapsto \varphi\left( \| x \|^2 \right)
% & \end{array} 
\in \PD \big\}
\]
%
(ver notaciones).

Vamos a ver  m\'as adelante varios ejemplos de  vectores aleatorios a simetr\'ia
el\'iptica  que  ya  hemos  vistos  en las  subsecciones  anteriores.   Un  caso
particular que va a jugar un rol importante es el de un vector de distribuci\'on
uniforme sobre la esfera \ $\Sset_d$, \ $U \sim \U(\Sset_d)$~\cite{FanKot90}:
%
\begin{ejemplo}[Distribuci\'on uniforme sobre la esfera unitaria]\label{Ej:MP:GeneCaracUniformeEsfera}
%
  Sea \  $U \sim \U(\Sset_d)$. Entonces  \ $U \sim  \ED\left( 0 , I  , \varphi_U
  \right)$ \ con
  %
  \[
  \varphi_U(u)   =  2^{\frac{d}{2}-1}   \Gamma\left(   \frac{d}{2}  \right)   \,
  u^{-\frac{d-2}{4}} \, J_{\frac{d}{2}-1}\left( \sqrt{u} \right)
  \]
  %
  con \ $J_\nu$ \ funci\'on de Bessel~\footnote{Seg\'un Sch{\oe}nberg~\cite[Nota
    de  pie~9]{Sch38} and Watson~\cite[p.~24,  nota de  pie~*]{Wat22}, deber\'ia
    llamarse  integral  de  Poisson  proque fue  introducida  primariamente  por
    Poisson en 1823~\cite{Poi23}, pero  apareci\'o implicitamente a\'un antes en
    trabajos  de Euler~\cite[Cap.~X, \S~1036]{Eul1769}.}   primera especie  y de
  orden $\nu$ (ver notaciones).
  
  De hecho, de la definici\'on de la funci\'on caracter\'istica, tenemos
  %
  \[
  \Phi_U(\omega) =  \frac{1}{|\Sset_d|} \int_{\Sset_d} e^{\imath  \, \omega^t s}
  d\mu_H(s)
  \]
  %
  con \ $\mu_H$ \ la medida de Haar~\footnote{Para \ $S \subset \Sset$ \ $\mu(S)
    =    |S|$.}     sobre    la    esfera    y   \    $|\Sset_d|    =    \frac{2
    \pi^{\frac{d}{2}}}{\Gamma\left( \frac{d}{2} \right)}$  \ la superficia de la
  esfera  unitaria~\cite{GraRyz15}.   Ahora,  denotanto  \  $\Sset_d^+$  \  y  \
  $\Sset_d^-$  \ respectivamente  la semiesfera  superior y  inferior,  se puede
  parametrizar \  $s \in \Sset_d^{\pm}$ \  bajo la forma \  $s = \begin{bmatrix}
    b^t & \pm \sqrt{1-\|b\|^2} \end{bmatrix}^t, \quad b \in \Bset_{d-1}$, lo que
  da,  siguiendo~\cite[ec.~4.644]{GraRyz15}  (cambio  de  variables)  y  notando
  $\mu_L$ \ la medida de Lebesgue,
  %
  \begin{eqnarray*}
  \Phi_U(\omega) & = & \frac{2 \, \Gamma\left( \frac{d}{2}
  \right)}{\pi^{\frac{d}{2}}} \int_{\Bset_{d-1}} \frac{e^{\imath \, \omega^t
  s}}{\sqrt{1-\|b\|^2}} d\mu_L(b)\\[2mm]
  %
  & = & \frac{\Gamma\left( \frac{d}{2} \right)}{\sqrt{\pi} \, \Gamma\left(
  \frac{d-1}{2} \right)} \int_0^{\pi} e^{\imath \, \| \omega \| \cos\theta} \,
  \sin^{d-2}\theta \, d\theta
  \end{eqnarray*}
  %
  Al final,  de la forma de la  funci\'on de Bessel~\cite[Ec.~8.411-7]{GraRyz15}
  (ver tambi\'en~\cite{AbrSte70, Wat22, GraMat95}), se obtiene
  %
  \[
  \Phi_U(\omega) =  \frac{2^{\frac{d}{2}-1} \Gamma\left( \frac{d}{2} \right)}{\|
    \omega \|^{\frac{d}{2}-1}} \, J_{\frac{d}{2}-1}\left( \|\omega\| \right)
  \]
  %
  lo que cierra  la prueba. Volveremos a esta  funci\'on caracteristica tratando
  de las coordenadas esf\'ericas.
\end{ejemplo}

La forma de la funci\'on caracter\'istica tiene varias consecuencias. La primera
es  que se  puede escribir  estocaticamente un  vector a  simetr\'ia  el\'iptica a
partir de un vector esf\'ericamente invariante de varias maneras, entre otros:
%
\begin{corolario}
  Sea  \ $Y  \sim  \ED(0,I,\varphi_Y)$, \  $m \in  \Rset^d$  \ y  \ $\Sigma  \in
  \Pos_d^+(\Rset)$. \ Sean \ $\Sigma = Q \Delta Q^t$ \ la descomposici\'on diagonal
  de \ $\Sigma$, \ $\Sigma^{\frac12} = Q \Delta^{\frac12} Q^t$ \'unica matriz de
  \  $\Pos_d^+(\Rset)$ \  raiz cuadrada  de \  $\Sigma$,  y \  $\Sigma =  L L^t$  \
  descomposici\'on  de  Cholesky   de  \  $\Sigma$,  con  \   $L$  \  triangular
  inferior~\cite{HorJoh13, Bha07}.  Entonces
  %
  \[
  Q \Delta^{\frac12} Y + m \: \egald \:  \Sigma^{\frac12} Y + m \: \egald \: L Y
  + m \: \sim \: \ED(m,\Sigma,\varphi_Y)
  \]
\end{corolario}
%
\begin{proof}
  El             resultado            es             consecuencia            del
  teorema~\ref{Teo:MP:PropiedadesFuncionCaracteristica}  y  de  la forma  de  la
  funci\'on caracter\'istica del teorema~\ref{Teo:MP:GeneradorasCaracteristicas}.
\end{proof}

Una otra consecuencia es que, dado el orden, el tensor de los momentos tiene una
estructura dada  para cualquier ley,  bajo un factor  escalar que depende  de la
ley.  Eso vale tambi\'en para los cumulantes:
%
\begin{teorema}[Momentos centrales y cumulantes]\label{Teo:MP:MomentosCumulantesEliptica}
%
  Sea \  $X$ \  vector aleatorio $d$-dimensional  de distribuci\'on  el\'iptica en
  torno a  un  vector \  $m$,  y  de  matriz  caracter\'istica \  $\Sigma  \in
  \Pos_d^+(\Rset)$.  Entonces, si  estos momentos y cumulantes existen,  \ $m$ \ es
  la media de \ $X$, \ie
  %
  \[
  \zeta_1[X] = 0, \qquad \kappa_1[X] = m
  \]
  %
  y para cualquier orden superior a  $2$, los momentos centrales y cumulantes de
  orden impares son ceros,
  %
  \[
  \zeta_{2 k + 1}[X] = \kappa_{2 k + 1}[X] = 0, \qquad k \ge 1
  %_{i_1,\ldots,i_{2  k  +1}}[X] =  \kappa_{i_1,\ldots,i_{2  k  +1}}[X] =  0,
  %\qquad k \ge 1
  \]
  %
  y los de orden pare son dados para \ $k \ge 1$, por
  %
  \[
  \zeta_{2 k}[X] = \alpha_k\left(  \varphi_X   \right) \T_k(\Sigma)
  %{i_1,\ldots,i_{2   k}}[X]   =   \alpha_k\left(  \varphi_X   \right)   \,
  %\psi_{i_1,\ldots,i_{2 k}}(\Sigma)
 \qquad  y \qquad \kappa_{2 k}[X] = \alpha_k\left(  \psi_X   \right) \T_k(\Sigma)
%_{i_1,\ldots,i_{2 k}}[X]
%  = \alpha_k\left( \psi_X \right) \, \psi_{i_1,\ldots,i_{2 k}}(\Sigma)
  % (-2)^k \, \varphi_X^{(k)}(0)  \sum_{\pi \in \Pi_{2 k ,  2}} \prod_{(l,n) \in
  %   \pi} \Sigma_{i_l,i_n}
  \]
  %
  % y
  %  %
  % \[
  %  \kappa_{i_1,\ldots,i_{2 k}}[X]  = (-2)^k  \, \psi_X^{(k)}(0)  \sum_{\pi \in
  %   \Pi_{2 k , 2}} \prod_{(l,n) \in \pi} \Sigma_{i_l,i_n}
  % \]
  %
  con el coefficient \ $\alpha_k$ \ dado por
  %
  \[
  \alpha_k(f) = (-2)^k f^{(k)}(0)
  \]
  %
  y el tensor \ $\T_k(\Sigma)$ \ de orden \ $2 k$ \ de componentes
  %
  \[
  \T_{i_1,\ldots,i_{2k}}(\Sigma) = \sum_{\pi \in \Pi_{2 k , 2}} \prod_{(l,n) \in
    \pi} \Sigma_{i_l,i_n}
  % \qquad \mbox{y} \qquad \alpha_k(f) = (-2)^k f^{(k)}(0)
  \]
  %
  donde   \  $\Pi_{2   k  ,   2}$   \  es   el  conjunto   de  particiones   por
  pares~\footnote{Por  ejemplo, $\Pi_{4,2}  =  \Big\{ \big\{  \{1,2\} ,  \{3,4\}
    \big\} \:  , \: \big\{  \{1,3\} ,  \{2,4\} \big\} \:  , \: \big\{  \{1,4\} ,
    \{2,3\} \big\} \big\}$; cuando \ $\pi = \big\{ \{1,3\} , \{2,4\} \Big\}$, el
    t\'ermino del producto es \ $\Sigma_{i_1,i_3} \Sigma_{i_2,i_4}$.} de $\{ 1 ,
  \ldots , 2 k \}$.
\end{teorema}
%
Una prueba  es dada  por inducci\'on en~\cite{BerBen86}  para los  momentos (ver
tambi\'en~\cite[p.~44]{FanKot90} para el resultat  con los momentos).  Damos una
prueba m\'as directa, valida para ambos momentos y cumulantes.
%
\begin{proof}
  Recordamosnos que para \ $k \in \Nset^*, \quad (i_1 , \ldots , i_k) \in \{ 1 ,
  \ldots , d \}^k$,
  %
  \[
  \zeta_{i_1,\ldots,i_k}[X]     =     (-    \imath)^k     \left.\frac{\partial^k
      \Phi_{X-m_X}}{\partial          \omega_{i_1}          \cdots          \partial
      \omega_{i_k}}\right|_{\omega=0}       \qquad       \mbox{y}       \qquad
  \kappa_{i_1,\ldots,i_k}[X]     =    (-     \imath)^k    \left.\frac{\partial^k
      \Psi_X}{\partial          \omega_{i_1}          \cdots          \partial
      \omega_{i_k}}\right|_{\omega=0}
  \]
  
  Por  definici\'on  de  los  momentos  centrales  $\zeta_1  =  0$.   Luego,  de
  $\Psi_X(\omega)  =  \imath \,  \omega^t  m  +  \log\left( \varphi_X(  \omega^t
    \omega) \right)$ \ tenemos
  %
  \[
  \nabla_\omega \Psi_X(\omega) =  \imath \, m + \frac{2  \, \varphi_X'( \omega^t
    \omega)}{\varphi_X'( \omega^t \omega)} \, \omega
  \]
  %
  Recordandose que \ $\Phi_X(0) = \Esp\left[  e^{\imath \, 0^t X} \right] = 1$ \
  y \ $m =  - \imath \, \nabla_\omega \Psi_X(0)$: \ $m$ \ es  la media de \ $X$\
  lo que corresponde a la intuici\'on.

  Luego, salimos de la f\'ormula de  Hardy, extensi\'on de la f\'ormula de Fa\`a
  di   Bruno,   que   vimos   secci\'on~\ref{Ssec:MP:GeneradoraCumulantes}   que
  recordamos: Para \ $h(\omega) =  f\left( g(\omega) \right), \quad \forall \: n
  \in \Nset^*,  \quad \forall \:  (i_1 , \ldots ,  i_n ) \in  \{ 1 , \ldots  , d
  \}^n$,
  %
  \[
  \frac{\partial^n  h}{\partial  \omega_{i_1}  \cdots \partial  \omega_{i_n}}  =
  \sum_{\pi  \in \Pi_n}  f^{(|\pi|)}\left( g(\omega)  \right) \prod_{B  \in \pi}
  \frac{\partial^{|B|} g}{\displaystyle \prod_{j \in B} \partial \omega_{i_j}}
  \]
  %
  con \ $\Pi_n$ \ el conjunto de las particiones de \ $\{ 1 , \ldots , n \}$ \ y
  \ $f^{(l)}$ la $l$-\'esima derivada de $f$.  En la expresi\'on de los momentos
  centrales y cumulantes tenemos
  %
  \[
  g(\omega) = \sum_{i,j=1}^d \omega_i \omega_j \Sigma_{i,j}
  \]
  %
  as\'i que, por simetr\'ia de $\Sigma$,
  %
  \[
  \frac{\partial  g}{\partial   \omega_{j_1}}  =  2   \sum_{l=1}^d  \omega_{j_1}
  \Sigma_{j_1,l},  \qquad  \frac{\partial^2  g}{\partial  \omega_{j_1}  \partial
    \omega_{j_2}}  =  2 \Sigma_{j_1,j_2},  \qquad  \forall  \:  n \ge  3,  \quad
  \frac{\partial^n g}{\prod_{l=1}^n \partial \omega_{j_l}} = 0
  \]
  %
  Es decir que, para $n \ge 1$,
  %
  \[
  \left.      \frac{\partial^n    g}{\prod_{l=1}^n     \partial    \omega_{j_l}}
  \right|_{\omega = 0} = \left\{\begin{array}{ccl}
  %
  2   \Sigma_{j_1,j_2} & \mbox{si} & n = 2\\[2mm]
  %
  0 & \mbox{si} & n \ne 2
  %
  \end{array}\right.
  \]
  %
  Entonces,  en la formula  de Hardy  tomada en  $\omega =  0$ quedan  solas las
  particiones  que  contienen unicamente  pares  de  indices.   Eso da  momentos
  centrales y cumulantes nulos para $k$ impar (obvio por sim\'etria). Adema\'as,
  siendo \  $\Pi_{2 k , 2}$  \ el conjunto de  particiones por pares de  $\{ 1 ,
  \ldots , 2 k \}$, notando que necesariamente cada partici\'on de \ $\Pi_{2 k ,
    2}$ \ contiene \ $k$ \ pares,
  %
  \[
  \left.   \frac{\partial^{2  k}   h}{\partial   \omega_{i_1}  \cdots   \partial
      \omega_{i_{2  k}}}\right|_{\omega =  0} =  \sum_{\pi  \in \Pi_{2  k ,  2}}
  f^{(k)}(0) \prod_{(l,n) \in \pi} \left( 2 \Sigma_{i_l,i_n} \right)
  \]
  %
  La prueba  se cierra  tomando respectivamente  \ $f =  \varphi_X$ \  y \  $f =
  \psi_X$.
\end{proof}
%
Veremos m\'as adelante una prueba a\'un m\'as directa y obvia que, a orden dado,
el tensor  de los momentos  centrales tiene una  estructura dada bajo  un factor
escalar  dependiendo  de la  ley.   Lo  que es  menos  obvio,  de la  relaci\'on
momentos-cumulantes, que eso vale para el tensor de los cumulantes, y que, m\'as
que eso, estas estructuras son las mismas.

De este  resultado se puede tambi\'en  explicitar la matriz de  covariancia y el
tensor curtosis para un vector a simetr\'ia el\'iptica~\cite[p.~44]{FanKot90}:
%
\begin{corolario}\label{Lem:MP:MediaCovarianzaEliptica}
  Sea \  $X$ \  vector aleatorio $d$-dimensional  de distribuci\'on  el\'iptica de
  matriz caracter\'istica \ $\Sigma \in \Pos_d^+(\Rset)$. Entonces
  %
  \[
  \Sigma_X = - 2 \,  \varphi_X'(0) \, \Sigma
  \]
  %
  y
  %
  \[
  \kappa_X   =   \frac{\varphi_X''(0)}{4    \,   \big(   \varphi_X(0)   \big)^2}
  \sum_{i,j=1}^d \Big( \left( \un_i \un_i^t \right) \otimes \left( \un_j \un_j^t
  \right) + \left( \un_i \un_j^t  \right) \otimes \left( \un_i \un_j^t \right) +
  \left( \un_i \un_j^t \right) \otimes \left( \un_j \un_i^t \right) \Big)
  \]
  %
\end{corolario}
\begin{proof}
  El resultado es inmediato por lo  de la covarianza.  Para la curtosis, momento
  central de  orden cuatro del  vector normalizado, es equivalente  considerar \
  $\Sigma  = -  \frac{1}{2  \, \varphi_X(0)}  I$,  lo que  da  el resultado  del
  corrolario.
\end{proof}
%
Nota: de $\Tr(\Sigma_X) = \Tr\left( \Esp\left[ (X-m_X) (X-m_X)^t \right] \right)
= \Esp\left[ (X-m_X)^t  (X-m_X) \right]$ \ tenemos para  $U \sim \U(\Sset_d)$, \
$\Tr(\Sigma_U) = 1$, \ie
%
\[
\mbox{Para } \: U \sim \U(\Sset_d), \quad \Sigma_U = \frac{1}{d} I
\]
%Se  notar\'a  que  $\varphi_X(0)  =  1$  \  m\'axima,  y  que  necesariamente  \
%$\varphi_X'(0)  < 0$,  o, tambi\'en,  que $\varphi_X''(0)  > 0$:  $\phi_X$  \ es
%(localmente) concava en torno a $0$;

Se  puede  ver  de  este  corolario  que  la  covarianza  es  proporcional  a  \
$\Sigma$. Es decir que un v\'inculo que se puede poner tambi\'en para fijar \ $(
\Sigma ,  \varphi_X )$ \  es de  imponer un homotec\'ia  a \ $\varphi_X$  \ para
tener  $\varphi_X'(0) =  - \frac12$,  para que  \ $\Sigma$  \ y  \  $\Sigma_X$ \
coincidan.

De la  forma de la funci\'on  caracter\'istica se proba  tambi\'en que cualquier
transformaci\'on  lineal de un  vector a  simetr\'ia el\'iptica  da un  vector a
simetr\'ia el\'iptica:
%
\begin{teorema}\label{Teo:MP:TranformacionAfinEliptica}
  Sea  \  $X \,  \sim  \,  \ED(m,\Sigma,\varphi_X)$  \ $d$-dimensional,  $A  \in
  \Mat_{d',d}(\Rset)$  \ de  rango  lleno tal  que  \ $d'  \le  d$ \  y  \ $c  \in
  \Rset^{d'}$. Entonces
  %
  \[
  A X + c \: \sim \: \ED(A m + c , A \Sigma A^t , \varphi_X)
  \]
  %
\end{teorema}
%
\begin{proof}
  El  resultado es  inmediato  de  la forma  de  la funci\'on  caracter\'istica,
  teorema~\ref{Teo:MP:GeneradorasCaracteristicas}   y   como  consecuencia   del
  teorema~\ref{Teo:MP:PropiedadesFuncionCaracteristica}.
\end{proof}
%
En particular, la proyecci\'on de  un vector a simetr\'ia el\'iptica queda el\'iptica
en  torno a  la proyecci\'on  del vector  posici\'on, con  la  misma generadora
caracter\'istica.


Eso  v\'incula  tambi\'en  un  vector  a  simetr\'ia  el\'iptica  con  cualquier
proyecci\'on:
%
\begin{teorema}[Proyecci\'on y componentes]\label{Teo:MP:ProyeccionComponentesEliptica}
  Sea \ $X$, vector aleatorio $d$-dimensonal de componentes $X_i$. Entonces
  %
  \[
  X \sim \ED(m,\Sigma,\varphi_X) \qquad  \Longleftrightarrow \qquad \forall \: a
  \in \Rset^d, \quad a^t (X - m) \egald \sqrt{\frac{a^t \Sigma a}{\Sigma_{i,i}}}
  (X_i - m_i)
  \]
\end{teorema}
%
\begin{proof}
  Sea \ $X \sim \ED(m,\Sigma,\varphi_X)$, entonces
  %
  \[
  \Phi_{a^t (X - m)}(\omega) = \Phi_{X - m}(\omega a) = \varphi_X\left( \omega^2
    a^t \Sigma  a \right)  = \Phi_{X_i-m_i}\left( \omega  \sqrt{\frac{a^t \Sigma
        a}{\Sigma_{i,i}}}\right)  = \Phi_{\sqrt{\frac{a^t \Sigma
        a}{\Sigma_{i,i}}} (X_i-m_i)}\left( \omega  \right)
  \]
  %
  (se puede sacar el valor absoluto a $\omega$ porque, necesariamente, $X_i-m_i$
  es     a     simetr\'ia    el\'iptica,     es     decir    $(X_i-m_i)     \egald
  -(X_i-m_i)$). Reciprocamente, si para cualquier  \ $a \in \Rset^d$ \ tenemos \
  $a^t (X  - m)  \egald \sqrt{\frac{a^t \Sigma  a}{\Sigma_{i,i}}} (X_i  - m_i)$,
  necesariamente
  %
  \[
  \Phi_{X  - m}(a) =  \Esp\left[ e^{\imath  \, a^t  (X-m)} \right]  = \Esp\left[
    e^{\imath \, \sqrt{\frac{a^t \Sigma a}{\Sigma_{i,i}}} (X_i - m_i)} \right] =
  \Phi_{X_i-m_i}\left( \sqrt{\frac{a^t \Sigma a}{\Sigma_{i,i}}}\right)
  \]
  %
  Es  una  funci\'on  de  $a^t  \Sigma  a$,  lo que  cierra  la  prueba  por  el
  teorema~\ref{Teo:MP:GeneradorasCaracteristicas}.
\end{proof}

Si el  caract\'er el\'iptico se conserva  por transformaci\'on lineal,  no es el
caso en general por adici\'on:
%
\begin{teorema}
  Sean   \    $X   \sim   \ED(m_X,\Sigma_X,\varphi_X)$    \   e   \    $Y   \sim
  \ED(m_Y,\Sigma_Y,\varphi_Y)$  \ independientes  con $\varphi_X$  y $\varphi_Y$
  diferenciables (por lo menos por partes). Entonces
  %
  \[
  X + Y \sim \ED(m,\Sigma,\varphi) \quad \Leftrightarrow \quad \left\{
    \begin{array}{l}  \Sigma_Y  \propto  \Sigma_X  \quad \mbox{o}\\  X,  Y  \sim
      \N \end{array} \right.
  \]
  %
  con $\propto$ significando ``proporcional a''.
  \end{teorema}
  %
\begin{proof}
  Claramente,     si     $X$     \     e     $Y$     son     gaussianas,     del
  teorema~\ref{Teo:MP:StabilidadGaussiana}    $X+Y$    es    guassiana,   y    a
  continuaci\'on a simetr\'ia el\'iptica.  Tambi\'en, si $\Sigma_X$ y $\Sigma_Y$
  son proporcionales, por reparametrizaci\'on de $\varphi_Y$ por ejemplo, se las
  puede  considerar  iguales  (ver  indeterminancia  bajo  un  factor  escalar).
  Entonces  $\Phi_{X+Y}(\omega)  = e^{\imath  \,  \omega^t m_X}  \varphi_X\left(
    \omega^t   \Sigma_X   \omega\right)    \,   e^{\imath   \,   \omega^t   m_Y}
  \varphi_Y\left( \omega^t  \Sigma_X \omega\right) \equiv  e^{\imath \, \omega^t
    (m_X+m_Y)}  \varphi\left(  \omega^t  \Sigma_X  \omega\right)$:  $X+Y$  es  a
  simetr\'ia el\'iptica.

  Reciprocamente,  suponemos  que $X+Y$  es  a  simetr\'ia el\'iptica.  Entonces
  existe  necesariamente una matriz  $\Sigma \in  \Pos_d^+(\Rset)$ y  $\varphi \in
  \PD_d$ tal que
  %
  \[
  \varphi_X\left(  \omega^t \Sigma_X  \omega\right) \,  \varphi_Y\left( \omega^t
    \Sigma_Y \omega\right) = \varphi\left( \omega^t \Sigma \omega\right)
  \]
  %
  Introducemos la matriz
  %
  \[
  R_Y  = \Sigma_X^{-\frac12} \Sigma_Y  \Sigma_X^{-\frac12} \quad  \mbox{su forma
    diagonal} \quad = Q \Delta_Y Q^t
  \]
  %
  Siendo $R_Y \in \Pos_d^+(\Rset)$ \ se diagonaliza con $Q \in \Ort_d(\Rset)$ bajo la
  forma~\cite{Bha97, Bha07, HorJoh13}
  %
  \[
  R_Y = Q \Lambda_Y Q^t = Q \diag(\lambda) Q^t
  \]
  %
  donde    \    $\lambda    =    \begin{bmatrix}   \lambda_1    &    \cdots    &
    \lambda_d \end{bmatrix}^t$ \  es el vector conteniendo los  autovalores de \
  $R_Y$ \ y \ $Q \in \Ort_d(\Rset)$ \ los autovectores. Sean ahora
  %
  \[
  R  =  Q^t \Sigma_X^{-\frac12}  \Sigma  \Sigma_X^{-\frac12}  Q \qquad  \mbox{y}
  \qquad \omega = \Sigma_X^{-\frac12} Q, \quad v \in \Rset^d
  \]
  %
  Entonces,  si $X+Y$  es  a  simetr\'ia eli\'iptica,  la  relaci\'on entre  las
  generadoras caracter\'isticas debe tomar la forma
  %
  \[
  \varphi_X\left( v^t v \right)  \, \varphi_Y\left( \omega^t \Lambda_Y v \right)
  = \varphi\left( v^t R v \right)
  \]
  %
  Diferenciendo en $v$ se obtiene necesariamente
  %
  \[
  \left( \varphi_X'\left( v^t v  \right) \, \varphi_Y\left( \omega^t \Lambda_Y v
    \right) \,  I + \varphi_X\left(  v^t v \right) \,  \varphi_Y'\left( \omega^t
      \Lambda_Y  v \right)  \,  \Lambda_Y \right)  v  = \varphi'\left(  v^t R  v
  \right) R v
  \]
  %
  Para cualquier  $1 \le  i \le d$,  tomando $v  = w \un_i$  con $w  \in \Rset^*$
  tenemos necesariamente
  %
  \[
  \left( \varphi_X'\left( w^2  \right) \, \varphi_Y\left( w^2  \lambda_i
    \right)  + w \lambda_i \varphi_X\left(  w^2 \right) \,  \varphi_Y'\left( w^2 
      \lambda_i \right)   \right)  \un_i  = \varphi'\left(  w^2 R_{i,i}
  \right) R \un_i
  \]
  %
  De esta ecuaci\'on, queda claro que \ $R$ \ debe ser diagonal, $R = \diag(r)$.
  A continuaci\'on, para cualquier $i = 1 , \ldots , d$ tenemos
  %
  \[
  \varphi_X'\left( v^t v \right) \, \varphi_Y\left( \omega^t \Lambda_Y v \right)
  +  \lambda_i \,  \varphi_X\left( v^t  v \right)  \,  \varphi_Y'\left( \omega^t
    \Lambda_Y v \right) = r_i \, \varphi'\left( v^t R v \right)
  \]
  %
  Para cualquier $i \ne j$, por combinaci\'on se obtiene entonces
  %
  \[
  (r_j-r_i) \varphi_X'\left( v^t v \right) \, \varphi_Y\left( \omega^t \Lambda_Y
    v \right) + (\lambda_i r_j - \lambda_j r_i) \, \varphi_X\left( v^t v \right)
  \, \varphi_Y'\left( \omega^t \Lambda_Y v \right) = 0
  \]
  %
  es decir
  %
  \[
  (r_j-r_i) \, \frac{\varphi_X'\left( v^t v \right)}{\varphi_X\left( v^t v \right)}
  +  (\lambda_i r_j - \lambda_j r_i) \,  \frac{\varphi_Y'\left( \omega^t
    \Lambda_Y v \right)}{\varphi_Y\left( \omega^t
    \Lambda_Y v \right)} = 0
  \]
  %
  Tomamos ahora \ $v  \in \Sset_d, \:\: w > 0$ \  dando el prim\'er t\'ermino de
  la  suma constante  as\'i que  el  secundo debe  necesariamente ser  constante
  tambi\'en. Solamente dos casos pueden aparecen entonces: (i) $\forall \: i \ne
  j, \quad \lambda_i r_j  = \lambda_j r_i$, es decir $\forall \:  i \ne j, \quad
  \frac{\lambda_j}{\lambda_i} = \frac{r_j}{r_i}$,  \ie $R \propto \Lambda_Y$; en
  otras  palabras,  $\Sigma  \propto  \Sigma_Y$;  en adici\'on,  se  debe  tener
  tambi\'en $r_j = r_i$, \ie $R  \propto I$, es decir $\Sigma \propto \Sigma_X$;
  en  conclusi\'on, necesariamente $\Sigma_Y  \propto \Sigma_X$  y vimos  que en
  este caso $X+Y$ es efectivamente  a simetr\'ia el\'iptica.  (ii) al contrario,
  notando que  si $\Lambda_Y  \not\propto I$, \  $v^t \Delta_Y v$  describe todo
  $\Rset_+$, lo que impone que $\frac{\varphi_Y'}{\varphi_Y}$ debe ser constante
  sobre $\Rset_+$.  Es  decir que, resolviendo la ecuaci\'on  diferencial con el
  v\'inculo que $\varphi_Y(0) = 1$,  la generadora tiene necesariamente la forma
  $\varphi_Y(w) =  e^{-\alpha w}$, generadora de  una gaussiana.  Intercambiando
  los roles de $X$ e $Y$ (o tomando $v$ tal que $v^t \Lambda_Y v = w > 0$ fijo),
  lo mismo occure para $\varphi_X$, lo que cierra la prueba.
\end{proof}

Vimos   en   las  secciones~\ref{Sssec:MP:StudentT}   y~\ref{Sssec:MP:StudentR},
tratando de  vectores de  distribuci\'on Student-$t$ y  -$r$, situaciones  en la
cuales la  matriz de covarianza es  (proporcional a) la  identidad, mientras que
las componentes del  vector no son independientes, contrariamente  a lo que pasa
en  el caso  gaussiano.   De hecho  este resultado  es  general en  el marco  de
vectores a sim\'etria el\'iptica~\cite{BilBre99, Max67}:
%
\begin{teorema}[Maxwell-Hershell]\label{Teo:MP:MaxwellHershell}
  Sea \ $X \sim \ED(m,I,\varphi_X)$. Las componentes $X_i$ son independientes si
  y solamente si $X  \sim \N(m, \alpha I)$ con $\alpha >  0$. En otros terminos,
  los  solos vectores  a  simetr\'ia esf\'erica  en  torno a  un  vector $m$  son
  gaussianos de covarianza proporcional a la identidad y de media $m$.
\end{teorema}
%
\begin{proof}
  Del teorema~\cite{Teo:MP:TranformacionAfinEliptica},  \ $\begin{bmatrix} X_1 &
    X_2    \end{bmatrix}^t     \sim    \ED\left(    \begin{bmatrix}     m_1    &
      m_2 \end{bmatrix}^t, I, \varphi_X \right)$ \ y  \ $X_i \sim \ED( m_i , 1 ,
  \varphi_X)$.  Si  estas variables son  independientes (condici\'on necesaria),
  buscamos entonces  \ $\varphi_X$ \ tal  que \ $\forall \:  \omega \in \Rset^2,
  \quad  \varphi_X\left(  \omega_1^2  +  \omega_2^2  \right)  =  \varphi_X\left(
    \omega_1^2   \right)   \varphi_X\left(    \omega_2^2   \right)$,   \ie   por
  reparametrizaci\'on
  %
  \[
  \forall  \:   (u,v)  \in  \Rset_+^2,  \qquad   \varphi_X(u+v)  =  \varphi_X(u)
  \varphi_X(v)
  \]
  %
  La  sola  funci\'on  continua  satisfaciendo  este morfismo  es  la  funci\'on
  exponencial, es decir  $\varphi_X(u) = \exp(a u)$.  La  funci\'on debe ser una
  generadora  caract\'eristica, asi  que  necesariamente $a  <  0$, que  podemos
  escribir $a = - \frac{\alpha}{2}$ con $\alpha > 0$, lo que cierra la prueba.
\end{proof}

Una consecuencia importante de la  forma de la funci\'on caracter\'istica es que
sirve a probar  que un vector a simetr\'ia  esf\'erica se escribe estocasticamente
como  una mezcla de  escala de  un vector  uniforme sobre  la esfera  unitaria \
$\Sset_d$. Este resultado es  debido a Sch{\oe}nberg~\cite{Sch38, FanKot90} (ver
tambi\'en~\cite{KeiSte74, Tei60}) y se enuncia como sigue:
%
\begin{teorema}[Mezcla de escala de base uniforme]\label{Teo:MP:MezclaUniforme}
  Sea  \ $X$ \  $d$-dimensional a  simetria esf\'erica.  Entonces, este  vector se
  escribe estocasticamente como
  %
  \[
  X \egald R \, U \qquad \mbox{con} \qquad U \sim \U(\Sset_d), \quad R > 0 \quad
  \mbox{independientes}
  \]
  %
  M\'as generalmente, para \ $Y \sim \ED(m,\Sigma,\varphi_Y)$,
  %
  \[
  Y \egald \Sigma^{\frac12} R \, U + m
  \]
  %
  Adem\'as, inmediatamente
  %
  \[
  R \egald \| X \|
  \]
  %
  (obviamente, $\frac{X}{\| X \|} \egald U$).
\end{teorema}
%
\begin{proof}
  Escribimos \  $\omega = w u$ \  con \ $w \ge  0$ \ y \  $u \in \Sset_d$.
  Entonces, con  \ $\mu_H$ \ la  medida de Haar sobre  la esfera y \  $P_X$ \ la
  medida de probabilidad de \ $X$, tenemos
  %
  \begin{eqnarray*}
  \Phi_X(\omega) & = & \varphi_X\left( w^2 \right)\\[2mm]
  %
  & = & \frac{1}{|\Sset_d|} \int_{\Sset_d} \varphi_X\left( w^2 \right) \, d\mu_H(v)\\[2mm]
  %
  & = & \frac{1}{|\Sset_d|} \int_{\Sset_d} \Phi_X(w v) \, d\mu_H(v)\\[2mm]
  %
  & = & \frac{1}{|\Sset_d|} \int_{\Sset_d} \left( \int_{\Rset^d} e^{\imath \,
  w v^t x} \, dP_X(x) \right) \, d\mu_H(v)\\[2mm]
  %
  & = & \int_{\Rset^d} \left( \int_{\Sset_d} e^{\imath \, w v^t x} \,
  \frac{1}{|\Sset_d|} \, d\mu_H(v) \right) \, dP_X(x)\\[2mm]
  %
  & = & \int_{\Rset^d} \Phi_U(w x) \, dP_X(x)\\[2mm]
  %
  & = & \int_{\Rset^d} \varphi_U\left( w^2 \|x\|^2 \right) \, dP_X(x)
  \end{eqnarray*}
  %
  con \ $\Phi_U$ \ funci\'on caracteristica  de un vector \ $U \sim \U(\Sset_d)$
  \ y  \ $\varphi_U$  \ la generadora  caracter\'istica correspondiente.   Sea \
  $F_R(r) = 0$ \ para \ $r \le 0$ \ y
  %
  \[
  F_R(r) = \int_{\Bset(0,r)} dP_X(x)
  \]
  %
  si no.  Claramente  \ $F_R$ \ es creciente  de $0$ a $1$: es  una funci\'on de
  repartici\'on. Notando \ $R$ \  la variable aleatoria positiva de funci\'on de
  repartici\'on \ $F_R$ \ y \ $P_R$ \ la medida de probabilidad asociada,
  %
  \begin{eqnarray*}
  \Phi_X(\omega) & = & \int_{\Rset_+} \varphi_U( w^2 r^2) \, dP_R(r)\\[2mm]
  %
  & = & \int_{\Rset_+} \Phi_U(r \omega) \, dP_R(r)\\[2mm]
  %
  & = & \Esp\left[ \Esp\left[ \left. e^{\imath \, \omega^t R U} \right | R \right] \right]
  \end{eqnarray*}
  %
  el paso  de la ante\'ultima  a la  \'ultima linea siendo  valido para \  $R$ \
  independiente  de  \ $U$.   En  otros  terminos, con  \  $R$  \  de medida  de
  probabilidad \  $P_R$ \  y \ $U  \sim \U(\Sset_d)$  \ independiente de  \ $R$,
  tenemos del teorema de esperanza total~\ref{Teo:MP:EsperanzaTotal},
  %
  \[
  \Phi_X(\omega) = \Phi_{R U}(\omega)
  \]
  %
  La prueba se  cierra de la relaci\'on uno-uno entre  la medida de probabilidad
  de un vector aleatorio y la funci\'on caracter\'istica. De \ $X \egald R \, U$
  \ y \ $\| U \| = 1$ \ viene \ $R \egald \| X \|$.
\end{proof}
%
Se puede referirse tambi\'en a \cite[Prop.~4.10]{BilBre99} para tener una prueba
alternativa      basado     sobre      el     teorema      de     Cram\'er-Wold,
Teo.~\ref{Teo:MP:CramerWold}.

De esta escritura, se puede ir un paso m\'as all\'a~\cite[Teo~2.3]{FanKot90}:
%
\begin{corolario}\label{Cor:MP:MezclaUniforme}
  Sea  \ $X$ \  $d$-dimensional a  simetria esf\'erica.  Entonces, este  vector se
  escribe estocasticamente como
  %
  \[
  X =  \|X\| \: \frac{X}{\|X\|}
  \]
  %
  tales  que  \ $\|  X  \|  \egald  R$ \  y  \  $\frac{X}{\|X\|} \egald  U  \sim
  \U(\Sset_d)$ \ son independientes.
\end{corolario}
%
\begin{proof}
  Se  aplica  el  teorema~\ref{Teo:MP:IgualdadDistribucionFuncionVA} a  \  $f(x)
  = \begin{bmatrix} \|x\|\\ \frac{x}{\|x\|} \end{bmatrix}$ \ con \ $X \egald Y =
  R \ U$.
\end{proof}.

De la escritura  $X \egald R \,  U$, se puede ver $X  \sim \ED(0,I,\varphi_X)$ \
como mu\~necas rusas: a cada escala o  capa \ $R = r$ tenemos una distribuci\'on
uniforma sobre la esfera de este rayo  $r$. Por eso, $X$ es tambi\'en dicho {\em
  mezcla de  escala} de  una {\em base}  \ uniforme  y \ $R$  \ es  llamada {\em
  variable generadora} con respecto a  esta base. En esta situaci\'on, llamaremos
tambi\'en la variable \ $R$ \ {\em rayo}.

De  esta escritura,  queda ahora  claro que  cada tensor  de  momentos centrales
tienen  una estructura  fija: de  \ $X  = R  \Sigma U$  \ tenemos  $\zeta_l[X] =
\Esp\left[  R^l \right]  \zeta_l[\Sigma  U]$ (ver  ej.~\cite[teo.~2.8]{FanKot90}
para $\Sigma \propto I$): la estructura, com\'un a cualquier vector aleatorio de
$\ED(m,\Sigma,\varphi_X)$ es dada por \  $\zeta_k[\Sigma U]$ (cero cuando $l = 2
k + 1$, por  simetr\'ia central) y el factor, dependiente de  la ley es dada por
el  momento  del ``rayo''  $R$.   Adem\'as,  se puede  dar  una  otra forma  del
coefficiente $\alpha_k$:
%
\begin{lema}\label{Lem:MP:AlphaConR}
  El coeficiente \ $\alpha_k(\varphi_X)$ \  del tensor de los momentos centrales
  del teorema~\ref{Teo:MP:MomentosCumulantesEliptica} es tambi\'en dado por
  %
  \[
  \alpha_k(\varphi_X) = \frac{\Gamma\left( \frac{d}{2} \right)}{2^k \Gamma\left(
      \frac{d}{2}  + k\right)}  \, \Esp\left[  \left( (X-m)^t  \Sigma^{-1} (X-m)
    \right)^k \right]
  \]
\end{lema}
%
\begin{proof}
  Por  ejemplo,  del  desarollo  de  Taylor  de  la  funci\'on  de  Bessel  dado
  en~\cite{GraRyz15},   aplicado   a   \   $\varphi_U(u)   =   2^{\frac{d}{2}-1}
  \Gamma\left(  \frac{d}{2} \right) u^{-\frac{d-2}{4}}  J_{\frac{d}{2}-1} \left(
    \sqrt{u} \right)$ \ se obtiene
  %
  \[
  \varphi_U^{(k)}(0)   =  \frac{(-1)^k   \Gamma\left(   \frac{d}{2}  \right)}{4^k
  \Gamma\left( \frac{d}{2} + k\right)}
  \]
  %
  A  continuaci\'on, para  \ $X  \sim \ED(m,\Sigma,\varphi_X)$,  de $X-m  =  R \
  \Sigma  \ U$  \  ($R$ es  escalar)  y de  la formula  del  tensor de  momentos
  centrales se obtiene
  %
  \[
  \zeta_{2    k}[X]     =    \Esp\left[    R^{2     k}    \right]
  \zeta_{2   k}[\Sigma   U]   =   \Esp\left[  R^{2   k}   \right]
  \alpha_k(\varphi_U) \T_k(\Sigma)
  \]
  %
  es decir
  %
  \[
  \alpha_k(\varphi_X) = (-2)^k \varphi_U^{(k)}(0) \Esp\left[ R^{2 k} \right]
  \]
  %
  Ahora, $(X-m)^t \Sigma^{-1} (X-m) = R^2 U^t U = R^2$, lo que cierra la prueba.
\end{proof}

Fijense  que  un vector  a  simetr\'ia  el\'iptica  no admite  necesariamente  una
densidad con respecto a la medida de  Lebesgue, como por ejemplo en el caso de un
vector uniforme  sobre $\Sset_d$  (pero esa tiene  una densidad,  constante, con
respecto a la medida de Haar sobre la esfera). Un otro ejemplo puede ser \ $X = B
G$  \ con  \  $B \sim  \B(p)$  \  y \  $G  \sim \N(0,I)$  \  independiente de  \
$B$. $\Phi_X(\omega)  = p \, e^{-\frac{\|  \omega \|^2}{2}} + 1  - p$ \  no es \
$L_1$,  \ie no tiene  una transformada  de Fourier  inversa usual.  Sin embargo,
cuando un vector el\'iptico admite una densidad, esa tiene propiedades remarcables
tambi\'en.
%
\begin{teorema}
%
  Sea \ $X \sim \ED(m,\Sigma,\varphi_X)$. Si  admite una  densidad~\footnote{De hecho, sufice  que admita
    una densidad  con respecto  a una medida  que depende solamente  del volumen,
    como la de Lebesgue pero  tambi\'en, de Haar.}, entonces esta densidad tiene
  la forma
  %
  \[
  p_X(x)  = |\Sigma|^{-\frac12} d_X\left( (x-m)^t \Sigma^{-1} (x-m) \right)
  \]
  %
  donde \ $d_X:  \Rset_+ \mapsto \Rset_+$ \ escalar,  es llamada {\em generadora
    de densidad}. Reciprocamente, si la densidad  tiene esta forma, \ $X$ \ es a
  simetr\'ia el\'iptica.
\end{teorema}
%
\begin{proof}
  Sin perdida de generalidad, se  puede considerar \ $X \sim \ED(0,I,\varphi_X)$
  \ y recuperar  el caso general por cambio de  variables.  Entonces, por cambio
  de   variables   (ver   secci\'on~\ref{Sec:MP:Transformacion})  tenemos   para
  cualquier matriz ortogonal \ $O \in \Ort_d(\Rset)$ \ y cualquier \ $x$
  %
  \[
  p_X(x)  = p_{O  X}(x)  =  |O|^{-\frac12} p_X\left(  O^{-\frac12}  x \right)  =
  p_X\left( O^{-\frac12} x \right)
  \]
  %
  siendo \ $|O|  = 1$~\cite{Bha97, HorJoh13}. $O^{-\frac12}$ \  es tambi\'en una
  matriz de rotaci\'on,  probando que \ $p_X$ \  queda invariante bajo cualquier
  rotaci\'on de su argumento, es decir que depende solamente de la norma de $x$.
\end{proof}

De este resultado, cuando \ $X$ \ admite una densidad de probablidad, se escribe tambien
%
\[
X \sim \ED(m,\Sigma,d_X)
\]
%
a\'un que puede ser confuso.
% (el uso de la letra $d$ o $\varphi$ permite diferenciar).

Claramente, para \ $X \sim  \ED(m,\Sigma,d_X)$, los niveles de probabildad \ $\{
x \tq p_X(x)  = c \} = \left\{  x \tq (x-m)^t \Sigma^{-1} (x-m) =  c \right\}$ \
son   elipsiodes  centrados   en   \  $m$,   de   direcciones  y   estiramientos
respectivamente  dados  por los  autovectores  y  autovalores  de $\Sigma$.  Eso
justifica tambi\'en la denominaci\'on ``$X$ \ a simetr\'ia el\'iptica''.

Como lo vimos, de una forma, todas las caracter\'isticas est\'adisticas de \ $X$
\ es  en el ``rayo''  \ $R \egald  \sqrt{(X-m)^t \Sigma^{-1} (X-m)}$. En  lo que
sigue,  sin   perdida  de  generalidad,   se  puede  concentrarse  en   $X  \sim
\ED(0,I,d_X)$.

Primero, se puede escribir la ley de $R$ a partir de $d_X$:
%
\begin{teorema}[Densidad del rayo]\label{Teo:MP:DensidadRayo}
  Sea \ $X \sim \ED(0,I,d_X)$ \ y \ $R \egald \|X\|$. \ $R$ admite tambi\'en una
  densidad que se escribe
  %
  \[
  p_R(r)  =  \frac{2  \pi^{\frac{d}{2}}}{\Gamma\left(  \frac{d}{2}  \right)}  \,
  r^{d-1} \, d_X\left( r^2 \right)
  \]
\end{teorema}
%
\begin{proof}
  Como lo  hemos visto en la prueba  del teorema~\ref{Teo:MP:MezclaUniforme}, la
  funci\'on de repartici\'on de $R$ se escribe
  %
  \[
  F_R(r) = \int_{\Bset_d(0,r)} dP_X(x)
  \]
  %
  es decir, $P_X$ admitiendo una densidad
  %
  \[
  F_R(r) = \int_{\Bset_d(0,r)} d_X\left( x^t x \right) \, dx
  \]
  %
  lo que da, de~\cite[Ec.~4.642]{GraRyz15}
  %
  \[
  F_R(r)  =  \frac{2  \pi^{\frac{d}{2}}}{\Gamma\left(  \frac{d}{2}  \right)}  \,
  \int_0^r \rho^{d-1} d_X\left( \rho^2 \right) \, d\rho
  \]
  %
  $F_R$   es    continua   y   diferenciable,   llegando    al   resultado   del
  teorema.  Volveremos   a  este  resultado  m\'as  adelante   tratando  de  las
  coordenadas esf\'ericas.
\end{proof}

De  este  resultado,  se puede  ahora  dar  la  relaci\'on  que existe  entre  \
$\varphi_X$ \  y \ $d_X$  \ siendos \  $\Phi_X$ \ y  \ $p_X$ \  relacionados por
transformada de Fourier.
%
\begin{teorema}\label{Teo:MP:TransformadaDeHankel}
%
  Sea  \  $X  \sim   \ED(m,\Sigma,\varphi_X)  \equiv  \ED(m,\Sigma,d_X)$  \  las
  generadoras caracter\'isticas y de densidad son relacionadas por
  %
  \[
  \varphi_X\left(   w^2   \right)   =   \left(   2   \pi   \right)^{\frac{d}{2}}
  w^{1-\frac{d}{2}}  \int_{\Rset_+}  r^{\frac{d}{2}}  d_X\left( r^2  \right)  \,
  J_{\frac{d}{2}-1}( r w) \, dr
  \]
  %
  y reciprocamente
  %
  \[
  d_X\left( r^2 \right) =  \left( 2 \pi \right)^{-\frac{d}{2}} r^{1-\frac{d}{2}}
  \int_{\Rset_+}     w^{\frac{d}{2}}    \varphi_X\left(    w^2     \right)    \,
  J_{\frac{d}{2}-1}( r w) \, dw
  \]
  %
  La transformaci\'on dando \  $w^{\frac{d}{2}-1} \varphi_X\left( w^2 \right)$ \
  a partir  de \  $r^{\frac{d}{2}-1} d_X\left( r^2  \right)$ \ es  conocida como
  {\em transformada  de Hankel}  de orden \  $\frac{d}{2}-1$~\cite{Sch38, Sch69,
    Sch71,  Lor54,   Pou99,  Pou10}.  A  veces,  la   transformaci\'on  dando  \
  $\varphi_X\left( w^2  \right)$ \ a  partir de \  $d_X\left( r^2 \right)$  \ es
  llamada {\em transformada de Hankel modificada} de orden \ $\frac{d}{2}-1$.

  M\'as generalmente, \ $\varphi_X$ \ es relacionada a la medida de probabilidad
  \   $P_R$  \  del   rayo  por   transformaci\'on  dicha   de  Hankel-Stieltjes
  (modificada)~\cite{Sch38, Nus73, ChoHai70, Sch71}
  %
  \[
  \varphi_X(w^2)   =    2^{\frac{d}{2}-1}   \Gamma\left(   \frac{d}{2}   \right)
  w^{1-\frac{d}{2}}  \int_{\Rset_+} r^{1-\frac{d}{2}} J_{\frac{d}{2}-1}(r  w) \,
  dP_R(r)
  % \equiv      2^{\frac{d}{2}-1}      \Gamma\left(     \frac{d}{2}      \right)
  % \HS_{\frac{d}{2}}[P_R](w) \qquad \mbox{y} \qquad P_R\left( (0 \; r ) \right)
  % =       \frac{2^{1-\frac{d}{2}}}{\Gamma\left(      \frac{d}{2}      \right)}
  % \HS_{\frac{d}{2}}^{-1}[w^{\frac{d}{2}} \varphi_X(w^2)]
  \]
\end{teorema}
%
\begin{proof}
  Hay varias  pruebas de este  resultado. Una prueba  elegante se basa  sobre la
  generadora caracteristica  de la variable $U \sim  \U(\Sset_d)$.  Primero, sin
  perdida de generalidad,  consideramos $m = 0, \: \Sigma =  I$. Entonces, de la
  escritura \ $X \egald R \, U$ \ tenemos
  %
  \begin{eqnarray*}
  \Phi_X(\omega) & = & \Esp\left[ e^{\imath \, R \omega^t U} \right]\\[2mm]
  %
  & = & \Esp\left[ \Esp\left[ \left. e^{\imath \, R \omega^t U} \right| R \right]
  \right]\\[2mm]
  %
  & = & \Esp\left[ \Phi_U\left( R^2 \| \omega \|^2 \right)  \right]
  \end{eqnarray*}
  %
  Es    decir,    del    ejemplo~\ref{Ej:MP:GeneCaracUniformeEsfera} se obtiene
  %
  \[
  \varphi_X\left(  w^2  \right)  =  2^{\frac{d}{2}-1}  \Gamma\left(
    \frac{d}{2}  \right)  \int_{\Rset_+}  (r w)^{-
    \frac{d-2}{2}}  J_{\frac{d}{2}-1}  \left( r  w \right)  \, dP_R(r)
  \]
  %
  transformada de Hankel-Stieltes (modificada)  de \ $P_R$. A continuaci\'on del
  teorema~\ref{Teo:MP:DensidadRayo} se obtiene
  %
  \[
  \varphi_X\left(  \|  \omega  \|^2  \right)  =  2^{\frac{d}{2}-1}  \Gamma\left(
    \frac{d}{2}  \right)  \int_{\Rset_+}  \left(   r  \|  \omega  \|  \right)^{-
    \frac{d-2}{2}}  J_{\frac{d}{2}-1}  \left( r  \|  \omega  \| \right)  \frac{2
    \pi^{\frac{d}{2}}}{\Gamma\left(  \frac{d}{2} \right)}  r^{d-1}  \, d_X\left(
    r^2 \right) \, dr
  \]
  %
  Eso da  la expresi\'on  de \ $\varphi_X$  \ como  transformada de Hankel  de \
  $d_X$. Para la transformaci\'on inversa, sufice recordarse que $\Phi_X$ siendo
  transformada  de  Fourier  de  $p_X$,  tenemos $p_X$  tranformada  inversa  de
  $\Phi_X$.  Luego,  por  simetr\'ia  (cambio  de variables  $w  \to  -w$,  esta
  transformada inversa  es nada mas que  la transformada directa,  por un factor
  $(2 \pi)^{-d}$ (ver teoremas~\ref{Teo:MP:InversionDensidad}).
\end{proof}

Tratando  de un vector  a simetr\'ia  esf\'erica, resuelte  frecuentemente m\'as
comodo  tratarlo  en su  representaci\'on  en  coordenadas hiperesf\'ericas,  es
decir, para $d \ge 2$
%
\[
x_i = r \left( \prod_{k=1}^{i-1} \sin\theta_k \right) \cos \theta_i, \quad 1 \le
i \le d
\]
%
con
%
\[
(r,\theta_1,\ldots,\theta_{d-1}) \in \Rset_+ \times [0 \; \pi)^{d-2} \times [0 \; 2 \pi)
\]
%
y las convenciones
%
\[
\theta_d = 0, \quad \prod_{k=1}^{0} = 1
\]

Los  par\'ametros   de  las  coordenadas  hiperesf\'ericas   (rayo,  angulos)  son
representadas en las figuras~\ref{Fig:MP:CoordenadasHiperesfericas}(a) para $d =
2$, y~\ref{Fig:MP:CoordenadasHiperesfericas}(b) para $d = 3$.

\begin{figure}[h!]
\begin{center} \begin{tikzpicture}[scale=.7]
\shorthandoff{>}

% CUIDADO
% Para el dibujo de la esfera, todo era hecho  en coordenadas esfericas de fisica (theta,phi)
% Para el punto x, se usan las coordenadas como en el libro, (theta_1,theta_2)
% (on dos parametrizaciones de la misma cosa...)
%
% Proyeccion para diburar un punto (x_1,x_2,x_3) teniendo en cuenta el angulo de vision
\pgfmathdeclarefunction{projx}{2}{\pgfmathparse{#1*cos(\Az)-#2*sin(\Az)}}
\pgfmathdeclarefunction{projy}{3}{\pgfmathparse{#1*sin(\Az)*sin(\El)+#2*cos(\Az)*sin(\El)+#3*cos(\El)}}


% unas definiciones para elegir el angolu de vision
\pgfmathsetmacro{\El}{35} % Elevacion
\pgfmathsetmacro{\Az}{-105}% Azimuth
%
\pgfmathsetmacro\r{3.5} % radio del circulo (d=2) y de la esfera (d=3)
%
%
% punto x para el caso d = 2
\pgfmathsetmacro\tpbi{55} % theta_1
\pgfmathsetmacro\xpbi{\r*cos(\tpbi)} % x_1
\pgfmathsetmacro\ypbi{\r*sin(\tpbi)} % x_2
%
%
% punto x para el caso d = 3
\pgfmathsetmacro\tu{60} % theta_1 punto x
\pgfmathsetmacro\td{45} %theta_2 punto x
\pgfmathsetmacro\xp{\r*cos(\tu)} % x_1 de x
\pgfmathsetmacro\yp{\r*sin(\tu)*cos(\td)} % x_2 de x
\pgfmathsetmacro\zp{\r*sin(\tu)*sin(\td)} % x_3 de x
%\pgfmathsetmacro\xp{\rp*cos(\tp)*cos(\pp)} % x_1 de x
%\pgfmathsetmacro\yp{\rp*sin(\tp)*cos(\pp)} % x_2 de x
%\pgfmathsetmacro\zp{\rp*sin(\pp)} % x_3 de x
%
%
% funcion permitiendo calcular phi(theta) dando la linea de cumbre
\pgfmathdeclarefunction{phicr}{1}{\pgfmathparse{
atan((sin(#1)*cos(\Az)+cos(#1)*sin(\Az))/(tan(\El)))
}}
%
% ====================================================================
%
% -----------------
% | Circulo d = 2 |
% -----------------
%
\begin{scope}
%
%  axes
% ------
%
\draw[>=stealth, ->] ({-1.2*\r},0)--({1.2*\r},0) node[right] {\footnotesize $x_1$};
\draw[>=stealth, ->] (0,{-1.2*\r})--(0,{1.2*\r}) node[above] {\footnotesize $x_2$};
%
%
%  Circulo
% --------
%
\draw[thick] (0,0) circle (\r);
%
%\draw (\r,0) node[below]{$r$};
%
%
%  punto x, r, theta_1
% --------------------
%
\draw[thick] (0,0) -- (\xpbi,\ypbi) node[scale=.9]{$\bullet$};
\draw[thick] (0,0) -- (\xpbi,\ypbi) node[above right]{$\boldsymbol{x}$};
\draw ({\xpbi/2},{1.05*\ypbi/2}) node[above,scale=.75]{$\boldsymbol{r}$};
\draw[->,>=stealth,thick] ({\r/2},0)  arc (0:\tpbi:{\r/2});
\draw ({\r*cos(\tpbi/2)/2},{\r*sin(\tpbi/2)/2}) node[right,scale=.75] {$\boldsymbol{\theta_1}$};
%
\node at (0,{-\r-1.25}) [scale=.9]{(a)};
\end{scope}
%
%
% ====================================================================
%
% ----------
% | Esfera |
% ----------
%
\begin{scope}[xshift=10.5cm]
%
%
%  angulo de visibilidad
% -----------------------
%
\pgfmathsetmacro\tv{-\Az}
%
%
% Axes
% -----
%
\draw[>=stealth, ->] (0,0)--({projx(\r*2,0)},{projy(\r*2,0,0)}) node[below]
{\footnotesize $x_1$};
\draw[>=stealth, ->] (0,0)--({projx(0,\r*1.2)},{projy(0,\r*1.2,0)}) node[right]
{\footnotesize $x_2$};
\draw[>=stealth, ->] (0,0)--(0,{projy(0,0,\r*1.5)}) node[above]
{\footnotesize $x_3$};
%
%
% Esfera
% ------
%
% interior de la esfera / linea de cumbre
\filldraw[very thick, domain=-\Az:-\Az+360, variable=\t, samples=180, fill opacity=.1]
plot ({projx(\r*cos(\t)*cos(phicr(\t)),\r*sin(\t)*cos(phicr(\t)))} , 
      {projy(\r*cos(\t)*cos(phicr(\t)),\r*sin(\t)*cos(phicr(\t)),\r*sin(phicr(\t)))});
%
% Ecuador (para visualisar la esfera), dado por phi = 0:
% parte visible
\draw[domain=-\Az+180:-\Az+360, variable=\t, samples=90]
plot ({projx(\r*cos(\t),\r*sin(\t))},{projy(\r*cos(\t),\r*sin(\t),0)});
%
% partie invisible
\draw[dashed, domain=-\Az:-\Az+180, variable=\t, samples=45]
plot ({projx(\r*cos(\t),\r*sin(\t))},{projy(\r*cos(\t),\r*sin(\t),0)});
%
% Linea de longitud para que se vea bien la surfaza: theta = 0 y 90
% parte visible theta = 0
\draw[domain=phicr(0):phicr(0)+180, variable=\p, samples=90]
plot ({projx(\r*cos(\p),0)},{projy(\r*cos(\p),0,\r*sin(\p))});
%
% partie escondida theta = 0
\draw[dashed, domain=phicr(0)+180:phicr(0)+360, variable=\p, samples=90]
plot ({projx(\r*cos(\p),0)},{projy(\r*cos(\p),0,\r*sin(\p))});
%
% parte visible theta = 90
\draw[domain=phicr(90):phicr(90)+180, variable=\p, samples=90]
plot ({projx(0,\r*cos(\p))},{projy(0,\r*cos(\p),\r*sin(\p))});
%
% partie escondida theta = 90
\draw[dashed, domain=phicr(90)+180:phicr(90)+360, variable=\p, samples=90]
plot ({projx(0,\r*cos(\p))},{projy(0,\r*cos(\p),\r*sin(\p))});
%
%
% Punto
% ------
%
%
% Posicion del punto x
\draw[thick] (0,0) --
({projx(\xp,\yp},{projy(\xp,\yp,\zp)})
node[scale=.75]{$\bullet$};
%
\draw[thick]
({projx(\xp,\yp},{projy(\xp,\yp,\zp)})
node[right]{$\boldsymbol{x}$};
%
%
% proyecciones para poner los angulos
% x1 = 0
%\draw[thick,dashed] ({projx(\xp,\yp},{projy(\xp,\yp,\zp})--({projx(0,\yp},{projy(0,\yp,\zp});
%\draw[thick,dashed] (0,0)--({projx(0,\yp},{projy(0,\yp,\zp});
%\draw[thick,dashed] ({projx(0,\yp},{projy(0,\yp,0})--({projx(0,\yp},{projy(0,\yp,\zp});
%
%
% notacion del radio
\pgfmathsetmacro{\propr}{.6} % distancia del radio para anotar "r"
\draw ({projx(\propr*\xp,\propr*\yp},{projy(\propr*\xp,\propr*\yp,\propr*\zp})
node[below,scale=.75] {$\boldsymbol{r}$};
%
%
% notacion del angulo theta_1
\pgfmathsetmacro{\proptu}{.45} % "radio" para dibujar el angulo
\draw[thick,dotted] ({projx(\xp,\yp},{projy(\xp,\yp,\zp})--({projx(\xp,0},{projy(\xp,0,\zp});;% proy x2=0
\draw[thick,dotted] (0,0)--({projx(\xp,0},{projy(\xp,0,\zp}) node[scale=.7]{$\bullet$};% proy x2=0
\draw[->,>=stealth]
({projx(\proptu*\xp,0)},{projy(\proptu*\xp,0,0)}) to [bend left=20]
node[pos=0,left,scale=.75] {$\boldsymbol{\theta_1}$}
({projx(\proptu*\xp,0)},{projy(\proptu*\xp,0,\proptu*\zp)});
%
%
% notacion del angulo theta_2
\pgfmathsetmacro{\propp}{.55} % "radio" para dibujar el angulo
\draw[thick,dashed] ({projx(\xp,\yp},{projy(\xp,\yp,\zp})--({projx(0,\yp},{projy(0,\yp,\zp});;% proy x1=0
\draw[thick,dashed] (0,0)--({projx(0,\yp},{projy(0,\yp,\zp}) node[scale=.7]{$\bullet$};% proy x1=0
\draw[->,>=stealth]
({projx(0,\propp*\yp)},{projy(0,\propp*\yp,0)}) to [bend right=20]
node[pos=0.3,right,scale=.75] {$\boldsymbol{\theta_2}$}
({projx(0,\propp*\yp)},{projy(0,\propp*\yp,\propp*\zp)});
%
\node at (0,{-\r-1.25}) [scale=.9]{(b)};
\end{scope}
%
\end{tikzpicture}
 \end{center}
% 
\leyenda{Coordenadas hiperesf\'ericas para un punto dado de $\Rset^d$. (a) caso $d
  = 2$ y (b) caso $d = 3$.}
%
\label{Fig:MP:CoordenadasHiperesfericas}
\end{figure}

En  el caso  de vector  a simetr\'ia  esf\'erica, aparece  que el  rayo $R$  y los
angulos     son     independientes      y     se     puede     calcular     cada
distribuci\'on~\cite{FanKot90, Lor54, Zoz, toto, titi}:
%
\begin{teorema}
  Sea  \  $X  \sim  \ED(0,I,d_X)$  \  y \  su  representaci\'on  en  coordenadas
  hiperesf\'ericas \  $X_i = R \left( \prod_{k=1}^{i-1}  \sin\Theta_k \right) \cos
  \Theta_i, \quad 1 \le  i \le d$. Entonces $R$ y los $\Theta_i,  \: 1 \le i \le
  d-1$ \ son independientes y
  %
  \[
  p_R(r) = \frac{2  \pi^{\frac{d}{2}}}{\Gamma\left( \frac{d}{2} \right)} r^{d-1}
  d_X\left( r^2 \right)
  \]
  %
  \[
  p_{\Theta_i}(\theta_i)        =       \frac{\Gamma\left(       \frac{d-j+1}{2}
    \right)}{\sqrt{\pi}   \,  \Gamma\left(   \frac{d-j}{2}  \right)}   \,  \big(
  \sin\theta_i \big)^{d-i-1}  \, \un_{[0 \;  \pi)}(\theta_i), \quad 1 \le  i \le
  d-2 , \qquad p_{\Theta_{d-1}}(\theta_{d-1}) =  \frac{1}{2 \pi} \, \un_{[0 \; 2
    \pi)}(\theta_{d-1})
  \]
  %
%  y
%  %
%  \[
%  p_{\Theta_{d-1}}(\theta_{d-1})   =   \frac{1}{2   \pi}   \,   \un_{[0   \;   2
%    \pi)}(\theta_{d-1})
%  \]
\end{teorema}
%
\begin{proof}
  Sea      la       transformaci\'on      $g:      (x_1,\ldots,x_d)      \mapsto
  (r,\theta_1,\ldots,\theta_{d-1})$. La jacobiana de $g^{-1}$ tiene la forma
  %
  \[
  \Jac_{g^{-1}} = \begin{bmatrix}
  %
    \frac{\partial x_1}{\partial r} & \frac{\partial x_1}{\partial \theta_1} & 0 & \cdots & 0\\
  %
    \vdots & \vdots & \ddots & \ddots & \vdots\\
  %
    \vdots & \vdots &  & \ddots  & 0\\
  %
    \frac{\partial x_{d-1}}{\partial r} & \frac{\partial x_{d-1}}{\partial
    \theta_1} & \cdots & & \frac{\partial x_{d-1}}{\partial \theta_{d-1}}\\
  %
    \frac{\partial x_d}{\partial  r} & \frac{\partial  x_d}{\partial \theta_1} &
    \cdots & \cdots & \frac{\partial x_d}{\partial \theta_{d-1}}
  %
  \end{bmatrix}
  \]
  %
  Desarrolando   el   determinante  por   la   \'ultima   columna  por   ejemplo
  (ver~\cite{Bha97, HorJoh13}), y as\'i por inducci\'on se obtiene
  %
  \[
  \left|   \Jac_{g^{-1}}  \right|  =   r^{d-1}  \prod_{j=1}^{d-2}   \left(  \sin
    \theta_i\right)^{d-j-1}
  \]
  %
  Entonces,  por  transformaci\'on  (ver  secci\'on~\ref{Sec:MP:Transformacion})
  tenemos
  %
  \[
  f_{R,\Theta_1,\ldots,\Theta_{d-1}}(r,\theta_1,\ldots,\theta_{d-1}) = d_X\left(
    r^2    \right)     r^{d-1}    \prod_{j=1}^{d-2}    \left(     \left(    \sin
      \theta_i\right)^{d-j-1} \un_{[0 \; \pi)}(\theta_i) \right) \, \un_{[0 \; 2
    \pi)}(\theta_{d-1})
  \]
  %
  Claramente  se  factoriza  probando  la  independencia  y  la  forma  de  cada
  distribuci\'on   marginal.    El   factor   viene    de   la   normalizaci\'on
  (ver~\cite[Ec.~8.380-2]{GraRyz15}   para  los   angulos,   lo  que   determina
  necesariamente el del rayo).
\end{proof}
%
Obviamente,   se   recupera   la   ley   de   \   $R   \egald   \|X\|$   \   que
encontramos.  Ad\'emas, estas coordenadas  hiperesf\'ericas, a  $r=1$, parametriza
$\Sset_d$  y permite  por ejemplo  de probar  la independencia  entre  $\|X\|$ y
$\frac{X}{\|X\|}$ (y entonces la  escritura en mezcla), de calcular directamente
\ $\Phi_U(\omega)$ \ para \ $U  \sim \U(\Sset_d)$, o de vincular las generadoras
caracter\'istica y de densidad entre otros.

En la familia  el\'iptica, hay una subfamilia particular  que queda invariante por
marginalizaci\'on, y extensi\'on, como la gaussiana por ejemplo:
%
\begin{definicion}[Consistencia~\cite{Yao73, Kan94}]
%
  Sea \  $X = \begin{bmatrix} X_1 &  \cdots & X_d \end{bmatrix}$  \ a simetr\'ia
  el\'iptica  de  generadora  de  densidad  $d_X$.   Este  vector  es  dicho  {\em
    consistente}   si  la   generadora  de   densidad  \   $d_{X'}$  \   de  $X'
  = \begin{bmatrix} X_1 & \cdots &  X_{d'} \end{bmatrix}$ \ tiene la misma forma
  que \ $d_X$  \ donde el par\'ametro  dimensional \ $d$ \ es  reemplazado por \
  $d'$.  En  otros termoinos tenemos  invarianza por marginalizaci\'on si  $d' <
  d$, pero se puede ver $X$ como marginal de un vector m\'as grande con la misma
  generadora de probabilidad re-emplazando $d$ por $d'$.

  Se proba  que, equivalentemente, la generadora caracteristica  \ $\varphi_X$ \
  no  es  relacionada a  \  $d$  (ver~\cite{Kan94,  FanKot90} para  la  prueba).
  Entonces, $\varphi_X$  va a ser  una generadora caracter\'istica de  un vector
  aleatorio   de  cualquier   dimensi\'on  \   $d  \in   \Nset^*$,  o,   con  la
  caracterisaci\'on por functiones definida no negativa (ver teorema de Bocher),
  \  $\varphi_X  \in  \PDSI$  (ver   notaciones,  y  las  a  continuaci\'on  del
  teorema~\ref{Teo:MP:GeneradorasCaracteristicas}).
%$\omega \mapsto \varphi_X\left(
%    \| \omega  \|^2 \right) \, \in  \, \PD$ \ conjuntos  de funci\'on hermiticas
%  definidas    positivas,     unidad    al    origen     (ver    notaciones    o
%  secci\'on~\ref{Ssec:MP:Generadoras}).
\end{definicion}


Si \  $X$ \ a simetr\'a  esf\'erica (y por transformaci\'on  af\'in a simetr\'ia
el\'iptica)  se escribe  siempre  como mezcla  de  escala de  base uniforme,  esta
escritura estoc\'astica  no es \'unica. Por  ejemplo, si consideramos $X  = A \,
G$, con \
%$m \in \Rset^d$, , $\Sigma \in \Pos_d^+(\Rset)$, 
$A$ \ variable  aleatoria positiva \ y \ $G \sim  \N(0,I)$ independiente de $G$,
queda claro  que \ $X$  \ es a  simetr\'ia esf\'erica. Estos  vectores, llamados
{\em mezcla  de escala de base  gaussiana}, o simplemente {\em  mezcla de escala
  gaussiana}  (GSM para gaussian  scale mixture  en ingl\'es)  fueron estudiados
intensivamente de manera formal~\cite{Kan94,  Yao73, Ver64, Pic70, Kel71, Kin72,
  KeiSte74, Tei60, AndMal74}.  Ecuentra aplicaciones en varias area, modelizando
la  textura de  imagenes  o datos  de  radar (ej.   suma  aleatoria de  vectores
gaussianos)~\cite{PorStr03,  BomBea08,   Sel08,  ShiSel07,  ZozVig10,  TisNic04,
  TodTab07}.

Una  pregunta natural  es  de saber  si  se puede  escribir  cualquier vector  a
simetr\'ia esf\'erica como mezcla de  escala de base gaussiana.  La respuesta es
negativa. Por ejemplo,  del dominio de definici\'on de  la variable, queda claro
que un  vector uniforme sobre la  esfera no puede  ser mezcla de escala  de base
gaussiana (ver un otro  contra-ejemplo en~\cite{Pic70}).  Escribir un vecor como
tal mezcla  resuelte posible solamente bajo  restricciones.  M\'as precisamente,
una condici\'on necesaria y suficiente  es que la variable debe ser consistente,
como       lo        prob\'o       Sch{\oe}nberg~\cite[Teo.~2]{Sch38}       (ver
tambi\'en~\cite[Teo.~2]{SteVan05},            \cite[Lem.~2.2]{Yao73}           o
\cite[Teo.~1]{Kan94}).
%
\begin{teorema}[Sch{\oe}nberg'38 -- a partir de la generadora caracter\'istica]
\label{Teo:MP:SchoenbergCaracteristica}
%
  Sea \ $X \sim  \ED\left( m , \Sigma , \varphi_X \right)$.  Entonces $X$ es una
  mezcla de  gaussiana si y  solamente si $X$  es consistente, es decir
  % lo que se escribe tambi\'en, con $Y = \sigma^{-\frac12} (X-m)$,
  %
  \[
  X \,  \egald \,  A \, \Sigma^{\frac12}  \, G  + m \quad  \Leftrightarrow \quad
  \varphi_X \in \PDSI
  \]
  %
  con \ $A > 0$ \ independiente de \ $G \sim \N(0,I)$.
\end{teorema}
%
\begin{proof}
  La directa es inmediata de la formula de esperanza total
  %
  \[
  \varphi_X(w)  =  \Esp\big[ \,  \Esp\left[  \varphi_G\left( A w \right)  \left.  \right|
      A\right] \, \big]
  \]
  %
  conjuntamente a \ $\varphi_G: w \mapsto e^{-\frac{w}{2}} \: \in \PDSI$.

  La rec\'iproca es  m\'as dificil a probar. Para los  detalles, dejamos el lector
  a~\cite{Sch38}. Los  elementos de prueba  son los siguientes. De  la escritura
  como mezcla de escala uniforme tenemos entonces para cualquier \ $d$
  %
  \[
  \varphi_X\left(  w^2  \right)   =  2^{\frac{d}{2}-1}  \Gamma\left(  \frac{d}{2}
  \right) \int_{\Rset_+} \left(  w r \right)^{1-\frac{d}{2}} J_{\frac{d}{2}-1}(w
  r) \, dP_{R_d}(r)
  \]
  %
  con \  $R_d$ \ la  variable generadora de  base uniforme correspondiente  a la
  dimension   $d$   (ver   prueba  del   teorema~\ref{Teo:MP:MezclaUniforme}   y
  ejemplo~\ref{Ej:MP:GeneCaracUniformeEsfera}).  Por   cambio  de  variables  se
  escribe tambi\'en
  %
  \[
  \varphi_X\left(  w^2 \right)  = \int_{\Rset_+}  2^{\frac{d}{2}-1} \Gamma\left(
    \frac{d}{2}   \right)   \left(    w   r   \sqrt{d}   \right)^{1-\frac{d}{2}}
  J_{\frac{d}{2}-1}\left(  w r  \sqrt{d}  \right) \,  dP_{R_d}\left( r  \sqrt{d}
  \right)
  \]
  %
  Eso siendo  valid para cualquier orden,  se nota $P_R$ el  l\'imite de $P_{R_d}$
  cuando $d$ tiende al infinito.  Luego,  el desarollo de Taylor de la funci\'on
  de Bessel  y de  la f\'ormula de  Stirling~\cite[Ec.~8.402~y~8.327]{GraRyz} se
  proba que
  %
  \[
  \lim_{d \to +\infty} 2^{\frac{d}{2}-1} \Gamma\left( \frac{d}{2} \right) \left(
    w  r \sqrt{d} \right)^{1-\frac{d}{2}}  J_{\frac{d}{2}-1}\left( w  r \sqrt{d}
  \right) \: = \: e^{-\frac{w^2 r^2}{2}}
  \]
  %
  Todo el  juego consiste a probar  que la convergencia es  uniforme, para poder
  intercambiar   l\'imite   e    integral.    Dio   una   prueba   Sch{\oe}nberg
  en~\cite{Sch38},  y  luego propuso  una  ``m\'as  moderna''  Steerneman y  van
  Perlo-ten-Kleij  en~\cite{SteVan05}.  Basicamente  se reconoce  en  $w \mapsto
  2^{\frac{d}{2}-1}   \Gamma\left(  \frac{d}{2}   \right)   \left(  w   \sqrt{d}
  \right)^{1-\frac{d}{2}}   J_{\frac{d}{2}-1}\left(  w   \sqrt{d}   \right)$  la
  funci\'on  caracter\'istica  de  $q_d(x)  \propto  \left(  1  -  \frac{x^2}{d}
  \right)_+^{\frac{d-3}{2}}$  \  que tiende  a  la  gaussiana  (ver por  ejemplo
  secci\'on~\ref{Sssec:MP:StudentR}).   Se usa el  lema de  Scheff\'e o  de Riez
  (ver~\cite{Rie28,  Sch47, Nov72,  Kus10} o  \cite{AthLah06,  Bog07:v1, Bil12})
  para probar la convergencia de la integral.
% (es similar al uso del teorema de convergencia dominada).
\end{proof}
%
\noindent Se encuentra una prueba alternativa tambi\'en en~\cite{FanKot90, Kin72}.

Este  teorema pone  en  juega la  condici\'on  necesaria y  suficiente sobre  la
funci\'on caracter\'istica para tener una  mezcla de escala gaussiana, as\'i que
no es necesario que \ $X$ \ admita una densidad de probabilidad. Sin embargo, al
imagen de la  consistencia que se exprime tambi\'en a  trav\'es de la generadora
de  densidad,  el teorema  de  Sch{\oe}nberg  tiene  una versi\'on  usando  esta
generadora.   Por  eso,  introducimos  el conjunto  de  funciones  complatemente
monotonas   sobre  $\Rset_+$   hasta   un  cierto   orden,   y  para   cualquier
orden~\footnote{Una funci\'on  \ $f: I \subset  \Rset \mapsto \Rset$  \ es dicha
  {\em absolutamente monotona} si es continua, diferenciable a todos los ordenes
  en el interior de $I$, y todas sus derivadas son positivas, $ \forall \: k \in
  \Nset, \quad f^{(k)} \ge 0$.  $f$ \ es dicha {\em completamente monotona} si \
  $f(-x)$ \ es absolutamente monotona~\cite{Ber29}.} (ver notaciones):
%
\[
\CM_n = \left\{ f \in C^{n-1}(\Rset_+) \tq \forall \, k = 0 , \ldots , n-1, \:\:
  (-1)^k  f^{(k)}  \ge 0  \:  \et \:(-1)^k  f^{(k)}  \:  \mbox{ es  decreciente}
\right\}
\]
%
y
%
\[
\CM = \bigcap_{n=0}^{+\infty} \CM_n =  \left\{ f \in
  C^{\infty}(\Rset_+)  \tq \forall \,  k \in  \Nset, \:\:  (-1)^k f^{(k)}  \ge 0
\right\}
\]
%
\begin{teorema}[Sch{\oe}nberg'38 -- a partir de la generadora de densidad]
\label{Teo:MP:SchoenbergDensidad}
%
  Sea \ $X \sim \ED\left( m , \Sigma , d_X \right)$.  Entonces $X$ es una mezcla
  de  gaussiana  si  y  solamente   si  $d_X$  es  completamente  monotona  sobre
  $\Rset_+$
  % lo que se escribe tambi\'en, con $Y = \sigma^{-\frac12} (X-m)$,
  %
  \[
  X \, \egald \, A \, \Sigma^{\frac12}  \, G + m \quad \Leftrightarrow \quad d_X
  \in \CM
  \]
  %
  con \ $A > 0$ \ independiente de \ $G \sim \N(0,I)$.
\end{teorema}
%
\begin{proof}
  La prueba se apoya sobre el teorema de Haussdorf-Bernstein-Widder que proba la
  equivalencia entre la  clase de funciones completamente monotonas  y la de las
  funciones   \   $f$   \   que   se   escriben   como   una   transformada   de
  Laplace-Stieltjes~\footnote{Por     definici\'on,    la     transformada    de
    Laplace-Stieltjes de una medida \ $\mu$  \ definida sobre $\Rset_+$ \ es una
    funci\'on   de  un   n\'umero  complejo   \  $\displaystyle   \LS[\mu](s)  =
    \int_{\Rset_+} e^{-s t}  \, d\mu(t)$. Se proba que un  real \ $x_0$, llamado
    abscisa de convergencia tal que la convergencia es uniforma en el semi-plano
    \ $\real{s} > x_0$, y a  continuaci\'on tal que \ $\displaystyle \LS[\mu]$ \
    es  anal\'itica  en este  semi-plano.   Si  \  $\displaystyle \mu  =  \sum_i
    \alpha_i \delta_{t_i}$ \ es una  medida discreta se obtiene \ $\displaystyle
    \LS[\mu](s)  =  \sum_i  \alpha_i  e^{-s  t_i}$  \  conocido  como  serie  de
    Dirichlet,  y  si  \ $\mu$  \  admite  una  densidad \  $g$,  $\displaystyle
    \LS[\mu](s) \equiv  \L[g](s) = \int_{\Rset_+} e^{-s  t} \, g(t) \,  dt$ \ es
    dicha transformada de Laplace ordinaria de  \ $g$.  Para $s = \imath \omega$
    \ con  $\omega \in  \Rset$, las transformaciones  de Laplace-Stieltjes  y de
    Laplace coinciden con las transformaciones $\FS$ de Fourier-Stieltjes y $\F$
    de           Fourier            ordinaria           (ver           tambi\'en
    secci\'on~\ref{Ssec:MP:FuncionCaracterisica} para esta transformaci\'on). Al
    imagen  de   la  transformada  de  Fourier  inversa   que  vimos  brevemente
    secci\'on~\ref{Ssec:MP:FuncionCaracteristica},    existen    f\'ormulas   de
    inversi\'on  de   las  transformadas  de  Laplace-Stieltjes   y  de  Laplace
    ordinaria, denotadas  \ $\LS^{-1}$  \ y \  $\L^{-1}$ \  respectivamente.  Se
    referir\'a     por    ejemplo     a~\cite{Wid46}     para    tener     m\'as
    detalles. \label{Foot:MP:LaplaceStieltjes}} (en el eje real)
  %
  \[
  f(t) = \int_{\Rset_+} e^{-t u} d\mu(u)
  \]
  %
  donde  $\mu$  es  una  medida finita  (ver~\cite[Teo.~3]{Sch38},  \cite{Ber29,
    Haus21:I, Haus21:II,  Wid32}, \cite[\S~12]{Wid46}, \cite[\S~XIII.4]{Fel71}).
  %
  % Feller v2, p.  439; Widder p. 160; Schoenberg th. 3  y sec. 4 (Hausdorff'21,
  % Bernstein'29, Carlson'21),
  % 
  Con el  cambio de variables  $u = \frac{1}{2  a^2}$ \ y  \ $\mu\big( (0  \; u)
  \big) =  (2 u)^{\frac{d}{2}} P_A\left( \left( \frac{1}{\sqrt{2  u}} \; +\infty
    \right) \right)$,  se aplica entonces este teorema  a (formula de probabilidad
  total)
  %
  \[
  d_X(r)  =  (2 \pi)^{-\frac{d}{2}} \int_{\Rset_+} a^{-d} e^{-\frac{r}{2 a^2}} dP_A(a) 
  \]

  Basicamente la  directa viene  de esta  formula de probabilidad  total y  de \
  $\displaystyle     \frac{d^k    e^{-\frac{r}{2    a^2}}}{dr^k}     =    (-1)^k
  \frac{e^{-\frac{r}{2 a^2}}}{2^k  a^{2 k}}$ \  (se usa para  poder intercambiar
  derivada    e     integral    el    teorema     de    convergencia    dominada
  sec.~\ref{Ssec:MP:VAPreliminaria}).

  De  nuevo,  la  rec\'iproca  es  m\'as  d\'ificil  a  probar  y  es  detallada
  en~\cite[\S~12]{Wid46} por ejemplo.  Basicamente, se  muestra que si \ $d_X$ \
  es  completamente  monotona,  se  puede  escribir \  $\displaystyle  d_X(n)  =
  \int_{\Rset_+}  v^n  d\mu(v)  =  \int_{\Rset_+}  e^{-n  u}  d\mu\left(  e^{-u}
  \right)$, momento de  orden \ $n$ con respecto a una  medida de probabilidad \
  $\mu$~\cite{Hau21:I, Hau21:II},  visto como transformada  de Laplace-Stieltjes
  en \ $t  = n$.  Se define la funci\'on \  $\displaystyle f(r) = \int_{\Rset_+}
  e^{- r  u} d\mu\left( e^{-u}  \right)$ \  que coincide con  \ $d_X$ \  sobre \
  $\Nset$.   Se  proba   finalmente  que  \  $d_X(r)  -   f(r)$  \  se  extiende
  analiticamente  en el  semi  plano  complejo \  $\real{s}  \ge 0$  \  y de  la
  analiticidad  que  \ $d_X$  \  y  \ $f$  \  coinciden  entonces  en todo  este
  semi-plano~\cite{Car21, CarKro05}.
\end{proof}

Se  notar\'a  que  la consistencia  se  manifesta  tambi\'en  a trav\'es  de  la
generadora de la mezcla gaussiana:
%
\begin{lema}\label{Lem:MP:AIndependienteD}
  Sea  \ $\varphi_X  \in  \PDSI$ (o  $d_X  \in \CM$).  Entonces, para  cualquier
  dimension \ $d$ \ y  $X \sim \ED(m,\Sigma,\varphi_X)$ \ $d$-dimensional, en la
  escritura  \ $X  \egald  A \,  \Sigma^{\frac12} G  +  m$ \  con  \ $A  > 0$  \
  independiente de \ $G  \sim \N(0,I)$ la ley de $A$ no  depende de la dimension
  $d$.
\end{lema}
%
\begin{proof}
  La  prueba  es  inmediata  del   hecho  que,  para  cualquier  matriz  $M  \in
  \Mat_{d,n}(\Rset)$ de rango lleno con $d  \le n$, \ $X \egald A \Sigma^{\frac12}
  G $ \ con  \ $G \sim \N(0,I)$ \ $n$-dimensional, tenemos $X' =  M X \egald A M
  \Sigma^{\frac12} G \egald A \left( M  \Sigma M^t \right)^{\frac12} G'$ \ con \
  $G'         \sim        \N(0,I)$         \         $d$-dimensional        (ver
  teorema~\ref{Teo:MP:StabilidadGaussiana}).  Una   mezcla  de  gaussiana  queda
  mezcla por proyecci\'on,  con la misma generadora $A$, y  al rev\'es puede ser
  vista como  proyecci\'on de una  mezcla de dimensi\'on  m\'as grande (no  va a
  cambiar la generadora)~\footnote{Cuidense  de que no anda si  salimos a partir
    de una mezcla de base uniforme: como lo vamos a ver, las proyecciones no son
    uniforme m\'as si $d  < n$, lo que se intuite por  razones de dominio imagen
    (bolas por proyecci\'on de la esfera).}.
\end{proof}

Si  en  la  escritura  como  mezcla  de  escala  de  base  uniforme  se  escribe
sencillamente  la  ley  del rayo  \  $R$,  se  puede tambi\'en  caracterizar  la
generadora en  el caso de mezcla de  base gaussiana. Viene de  la escritura como
mezcla, donde  se reconoce  une transformada de  Laplace-Stieltjes (ver  nota de
pie~\ref{Foot:MP:LaplaceStieltjes}).

\begin{lema}\label{Lem:MP:PaLaplaceInversa}
  Sea \ $X \sim \ED(m,\Sigma,\varphi_X)$ \ mezcla de escala gaussiana, con \ $A$
  la  variable aleatoria  generadora.  $\varphi_X$  \ admite  una continuaci\'on
  anal\'itica \ $s \mapsto \varphi_X(s)$ \ en el semi-plano complejo \ $\real{s}
  > 0$.  Notando \ $\LS$ \ la transformada de Laplace-Stieltjes y \ $\LS^{-1}$ \
  la transformada  inversa (ver nota  de pie~\ref{Foot:MP:LaplaceStieltjes}), la
  funci\'on de repartici\'on de \ $A$ \ es dada por
  %
  \[
  F_A(a) = \LS^{-1}[\varphi_X] \left( \left( 0 \; \frac{a^2}{2} \right) \right)
  \]
  %
  Si la medida \ $P_A$ \ admite una densidad \ $p_A$, se obtiene inmediatamente
  %
  \[
  p_A(a) = a \, \L^{-1}[\varphi_X] \left(\frac{a^2}{2} \right)
  \]

  Similarmente, si \ $X$ \ admite una  generadora de densidad \ $d_X$, \ $d_X$ \
  se extiende tambi\'en anal\'iticamente en el semi-plano complejo \ $\real{s} >
  0$ \ y se obtiene
  %
  \[
  \int_0^a u^{-d} dP_A(u) = (2 \pi)^{\frac{d}{2}} \, \LS^{-1}[d_X] \left( \left(
      \frac{1}{2 a^2} \; +\infty \right) \right) \qquad \mbox{y} \qquad f_A(a) =
  (2 \pi)^{\frac{d}{2}} \, a^{d-3} \, \L^{-1}[d_X]\left( \frac{1}{2 a^2} \right)
  \]
  %
  respectivamente.
\end{lema}
%
\begin{proof}
  Para \ $U$ \ uniforme sobre la  esfera, del desarollo en serie de la funci\'on
  de   Bessel~\cite[Ec.~8.402]{GraRyz15},  es  claro   que  \   $\varphi_U(w)  =
  2^{\frac{d}{2}-1}  \Gamma\left(   \frac{d}{2}  \right)  \,  w^{-\frac{d-2}{4}}
  J_{\frac{d}{2}-1}(\sqrt{w})$  \ admite  una continuaci\'on  anal\'itica  en el
  semi-plano  complejo $\real{s}  >  0$.   De la  mezcla  uniforme, tambi\'en  \
  $\varphi_X(w)  =   \int_{\Rset_+}  \varphi_X(r  w)  \,   dP_R(r)$  admite  una
  continuaci\'on anal\'itica en el semi-plano complejo $\real{s} > 0$. Luego, de
  la escritura de mezcla gaussiana tenemos
  %
  \[
  \varphi_X(w) = \int_{\Rset_+} e^{-\frac{w a^2}{2}} \, dP_A(a) = \int_{\Rset_+}
  e^{- w t} \, dP_A(\sqrt{2 t})
  \]
  %
  por cambio  de variables. Se  reconoce la transformada de  Laplace-Stieltjes \
  $\L$ \ de la medida \ $\mu$  \ (definida sobre $\Rset_+$) dada por \ $\mu\big(
  (0  \;  t)  \big) =  P_A\left(  \left(  0  \;  \sqrt{2  t} \right)  \right)  =
  F_A(\sqrt{2  t})$.  La  funci\'on  de repartici\'on  en  \ $\sqrt{2  t}$ \  es
  entonces dada por la transformada  inversa de Laplace-Stieljes de \ $s \mapsto
  \varphi_X(s)$, lo  que cierra  la prueba para  \ $F_A$.  Ahora,  para escribir
  $p_A$ \ se puede diferenciar $F_A$ obtenido, o simplemente escribir $dP_A(a) =
  p_A(a) \, da$, \ie \ $dP_A(\sqrt{2 t}) = \frac{p_A(\sqrt{2 t})}{\sqrt{2 t}} \,
  dt$ \ y reconocer en \  $\varphi(s)$ \ la transformada ordinaria de Laplace de
  $t \mapsto \frac{p_A(\sqrt{2 t})}{\sqrt{2 t}}$.

  La prueba que \ $d_X$ \  se extiende analiticamente en el semi-plano $\real{s}
  >0 $ es dada en~\cite[Cap.~IV, Teo.~3a]{Wid46}~\footnote{Se proba de hecho que
    una  funci\'on absolutamente  monotona  sobre  $\Rset_-$, como  lo  es \  $r
    \mapsto d_X(-r)$,  se extiende analiticamente  en el semi-plano  $\real{s} <
    0$.}. Basicamente, siendo  \ $d_X$ \ $C^\infty$, se  escribe su desarollo de
  Taylor en torno a un  $r_0 > 0$ y se proba que sobre $r  \in (0 \; r_0)$, esta
  serie de terminos todos positivos converge uniformamente. Por consecuencia, se
  tiene la convergencia uniforme tambi\'en para \  $r = s$ \ en la bola compleja
  \ $|s-r_0|  < r_0$. Se cierra  la prueba de  la validez para cualquier  $r_0 >
  0$. A continuaci\'on, la forma de \ $P_A$ \  y \ $p_A$ \ a partir de \ $d_X$ \
  viene de los mismos pasos, saliendo de
  %
  \[
  d_X(r) =  \int_{\Rset_+} (2 \pi)^{-\frac{d}{2}} \,  a^{-d} e^{-\frac{r}{2}} \,
  dP_A(a) = (2 \pi)^{\frac{d}{2}}  \, \int_{\Rset_+} (2 t)^{\frac{d}{2}} \, e^{-
    w t} \, dP_A\left( (2 t)^{-\frac12} \right)
  \]
  %
  y introduciendo la medida \ $\mu_A$ \ definida por \ $\displaystyle \mu_A(B) =
  \int_B (2  t)^{\frac{d}{2}} \, dP_A\left(  (2 t)^{-\frac12} \right)$ \  por un
  lado, y escribiendo \ $dP_A(a) = p_A(a) da$ \ por el otro lado.
\end{proof}

Cuando se puede extender \ $\varphi_X$ \  o \ $d_X$ \ al eje complejo imaginario
puro, $w  = \imath \, \omega$ \  o \ $r =  \imath \, \omega$ \  se reemplazan las
transformadas de Laplace-Stieltjes y Laplace ordinarias (y sus inversas) por las
de  Fourier-Stieltjes   \  $\FS$\   y  Fourier  ordinaria   \  $\F$  \   (y  sus
inversas)~\footnote{Se   puede  tambi\'en  pensar   a  la   transformaci\'on  de
  Mellin-Stieltjes   y  la   de  Mellin   ordinario  dada   respectivamente  por
  $\displaystyle  \MS[\mu](s) =  \int_{\Rset_+} t^s  \, d\mu(t)$  \quad  y \quad
  $\displaystyle  \M[g](s) = \int_{\Rset_+}  t^{s-1} \,  g(t) \,  dt$, definidas
  sobre una  franja del  plano complejo del  tipo \  $\real{s} \in (x_m  \; x_M)$
  (abscisas de convergencia), \  y usar sus propiedades remarcables~\cite{Zol57,
    Pou99, Pou10,  Wid46, ParKam01}.   En el caso  con densidad por  ejemplo, se
  reconoce en \  $g_X(r) = r^{d-1} d_X\left( r^2 \right)$ \  (ley del rayo, bajo
  un factor) lo  que se llama convoluci\'on af\'in, entre \  $g_G(r)$ \ (ley del
  rayo  de  la gaussiana)  \  y  la medida  de  \  $A$,  $\displaystyle g_X(r)  =
  \int_{\Rset_+} \frac{1}{a} g_G\left( \frac{r}{a} \right) \, p_A(a) \, da$.  Al
  imagen de las  propiedades de la transformada de  Laplace, la transformaci\'on
  de Mellin de  una convoluci\'on af\'in es el  producto de las transformaciones
  de Mellin, y por un lado \ $\M[t^\alpha g(t)](s) = \M[g](s+\alpha)$ \ y por el
  otro lado  \ $\M[g(r^\alpha)](s) = |\alpha|^{-1}  \M[g](s/\alpha)$.  As\'i, se
  obtiene  sencillamente,  de  \  $\M[d_G](s)  =  2^s  \Gamma(s)$~\cite[Cap.~12,
  Tabla~12.1]{Pou10}   o~\cite[Cap.~18,  Tabla~18.1]{Pou99}   \   $\M[p_A](s)  =
  \frac{\pi^{\frac{d}{2}}              \M[d_X]\left(             \frac{s+d-1}{2}
    \right)}{2^{\frac{s-1}{2}} \Gamma\left( \frac{s+d-1}{2} \right)}$.
  % \ para \ $\real(s) > 1-d$.
  Similarmente,  se  muestra  en  el  caso  con densidad  que  \  $\M[p_A](s)  =
  \frac{\M[\varphi_X]\left(       \frac{1-s}{2}       \right)}{2^{\frac{1-s}{2}}
    \Gamma\left( \frac{1-s}{2} \right)}$,
  % \ para \ $\real{s} < 1$,
  \   y   \   obviamente   \  $\M[\varphi_X]   =   \frac{\pi^{\frac{d}{2}}   4^s
    \Gamma(s)}{\Gamma\left( \frac{d}{2} - s \right)} \M[d_X]\left( \frac{d}{2} -
    s  \right)$ \  (lo que  se obtiene  tambien a  partir de  la  relaci\'on via
  transformada de  Hankel entre esta generadoras). Se  puede referirse tambi\'en
  a~\cite[\S~3.2.1]{Zoz12}).  }
%
\cite{Zol57, Pou99, Pou10, Wid46, ParKam01}.

\

Varios  vectores aleatorios que  hemos visto  caen en  la familia  el\'iptica.  No
admiten todas  una densidad con respecto a  la medida de Lebesgue,  como lo hemos
visto con la ley uniforme sobre la esfera (admite una con respecto a la medida de
Haar), tampoco no  son todas consistentes (y entonces no  se escriben todas como
mezcla de escala de base gaussiana).
%
\begin{ejemplo}[Distribuci\'on gaussiana]
%
  Sea \ $X \sim \N(m,\Sigma)$.  Entonces, \ $X \sim \ED(m,\Sigma,\varphi_X)$ \ o
  equivalentemente \ $X \sim \ED(m,\Sigma,d_X)$ con
  %
  \[
  \varphi_X(u) =  e^{-\frac{u}{2}} \qquad \mbox{y}  \qquad , \qquad d_X(r)  = (2
  \pi)^{-\frac{d}{2}} e^{- \frac{r}{2}}
  \]
  
  Eso da la ley del rayo
  %
  \[
  p_R(r)  =  \frac{1}{2^{\frac{d}{2}-1}  \Gamma\left(  \frac{d}{2}  \right)}  \:
  r^{d-1} e^{- \frac{r^2}{2}}
  \]
  %
  Aparece   que  \   $R^2  \sim   \G\left(  \frac{d}{2}   ,   \frac12\right)$  \
  distribuci\'on gamma.
  
  Adem\'as,   la  generadora   caracter\'istica  siendo   independiente   de  la
  dimensi\'on \  $d$, \  $\Phi_X \in \PD$,  o equivalentemente la  generadora de
  densidad  \ $d_X$ \  teniendo la  forma que  damos para  cualquier dimensi\'on
  (cambiar  de  dimensi\'on es  equivalente  a  cambiar  $d$), la  gaussiana  es
  consistente;  Las marginales  de  una gaussiana  son  gaussianas, y  cualquier
  gaussiana puede ser vista como  marginal de gaussiana de cualquier dimensi\'on
  m\'as grande.

  Obviamente, se  escribe como  mezcla de  escala de gaussiana  con \  $A =  1$ \
  variable cierta.
\end{ejemplo}

\begin{ejemplo}[Distribuci\'on Student-$t$]
%
  Sea \ $X \sim \T_\nu(m,\Sigma)$.  Entonces, \ $X \sim \ED(m,\Sigma,\varphi_X)$
  \ o equivalentemente \ $X \sim \ED(m,\Sigma,d_X)$ con
  %
  \[
  \varphi_X(u)   =  \frac{\nu^{\frac{\nu}{4}}}{2^{\frac{\nu}{2}-1}  \Gamma\left(
      \frac{\nu}{2}   \right)}   \,  u^{\frac{\nu}{4}}   K_{\frac{\nu}{2}}\left(
    \sqrt{\nu \,  u} \right) \qquad \mbox{y} \qquad  d_X(r) = \frac{\Gamma\left(
      \frac{d+\nu}{2}     \right)}{(\pi     \nu)^{\frac{d}{2}}     \Gamma\left(
      \frac{\nu}{2}    \right)}   \left(    1    +   \frac{r}{\nu}    \right)^{-
    \frac{d+\nu}{2}}
  \]

  Se obtiene de \ $d_X$ \ la ley de rayo
  %
  \[
  p_R(r)  =  \frac{2}{\nu^{\frac{d}{2}}   B\left(  \frac{\nu}{2}  ,  \frac{d}{2}
    \right)} \:
  % \frac{2    \,   \Gamma\left(    \frac{d+\nu}{2}   \right)}{\nu^{\frac{d}{2}}
  %   \Gamma\left( \frac{\nu}{2} \right) \Gamma\left( \frac{d}{2} \right)} \,
  \frac{r^{d-1}}{\left( 1 + \frac{r^2}{\nu} \right)^{\frac{d+\nu}{2}}}
  \]
  %
  Con \ $F = \frac{R^2}{d}$ \ se obtiene por transformaci\'on
  %
  \[
  p_F(f)   \propto   \frac{f^{\frac{d}{2}-1}}{\left(   1   +   \frac{d}{\nu}   f
    \right)^{\frac{d+\nu}{2}}}
  \]
  %
  conocido como ley de Fisher-Snedecor (o $F$), de grados de libertad $(d,\nu)$,
  o tipo Pearson VI, o  beta de secunda especie~\cite{JohKot95:v2, Muk00, Bre88,
    IbaPer12} (ratio de gamma independientes).

  La Student-$t$ es tambi\'en consistence de la independencia de \ $\varphi_X$ \
  con la dimensi\'on \ $d$, \  $\Phi_X \in \PD$, o equivalentemente del hecho de
  que  la generadora de  densidad \  $d_X$ \  teniendo la  forma que  damos para
  cualquier dimensi\'on  (cambiar de dimensi\'on es equivalente  a cambiar $d$);
  Las  marginales de  una Student-$t$  con  \ $\nu$  \ grado  de libertad  queda
  Student-$t$  con $\nu$  grado de  libertad y  cualquier Student-$t$  con $\nu$
  grado de  libertad pued ser  vista como marginal  de un Student-$t$  con $\nu$
  grado de libertad de cualquier dimensi\'on m\'as grande.

  V\'imos           en           la           secci\'on~\ref{Sssec:MP:StudentT},
  lema~\ref{Lem:MP:MezclaGaussianaEscalaStudentT},    que    se   escribe    una
  Student-$t$   como   mezcla   de   escala   gaussiana  donde   \   $A   \egald
  \frac{1}{\sqrt{V}},  \: V \sim  \G\left( \frac{\nu}{2}  \, ,  \, \frac{\nu}{2}
  \right)$ ley gamma, \ie \ $A$ \  es la raiz cuadrada de una gamma inversa (ver
  tambi\'en~\cite{FanKot90,  KotNad04}).   Se  puede re-obtener  este  resultado
  buscando directamente $p_A$ por transformaci\'on de Laplace inversa de \ $d_X$
  \  o  de  \  $\varphi_X$,  lema~\ref{Lem:MP:PaLaplaceInversa}  y~\cite[Cap.~5,
  Tab.~A.5.1, Ec.~14]{Pou10} o~\cite[Cap.~2, Tab.~2.3, Ec.~14]{Pou99}.
%
% con Mellin \cite[Cap.~2, Tab.~2.5]{Pou10}
\end{ejemplo}

\begin{ejemplo}[Distribuci\'on student-$r$]
%
  Sea \ $X \sim \R_\nu(m,\Sigma)$.  Entonces, \ $X \sim \ED(m,\Sigma,\varphi_X)$
  \ o equivalentemente \ $X \sim \ED(m,\Sigma,d_X)$ con
  %
  \[
  \varphi_X(u)   =   \frac{2^{\frac{\nu}{2}}   \Gamma\left(   \frac{\nu}{2}   +1
    \right)}{(\nu+2)^{\frac{\nu}{4}}}            \,           u^{-\frac{\nu}{4}}
  J_{\frac{\nu}{2}}\left(  \sqrt{(\nu+2) \,  u} \right)  \qquad  \mbox{y} \qquad
  d_X(r)  =  \frac{\Gamma\left(  \frac{\nu}{2}  +  1  \right)}{\pi^{\frac{d}{2}}
    (\nu+2)^{\frac{d}{2}}  \Gamma\left(  \frac{\nu-d}{2}  \right)}  \left(  1  -
    \frac{r}{\nu+2} \right)_+^{\frac{\nu-d}{2}}
  \]
  
  Se obtiene as\'i la ley del rayo bajo la forma
  %
  \[
  p_R(r) = \frac{2}{(\nu+2)^{\frac{d}{2}}} \:  r^{d-1} \left( 1 - \frac{r^2}{\nu+2}
  \right)^{\frac{\nu-d}{2}} \un_{(0 \; 1)}\left( \frac{r^2}{\nu+2} \right)
  \]
  %
  Se  concluye   que  \   $\frac{R^2}{\nu+2}  \sim  \beta\left(   \frac{d}{2}  ,
    \frac{\nu-d}{2}+1 \right)$ \ distribuci\'on beta.

  Pero $X \sim \R_\nu(m,\Sigma)$ no es consistente. Puede parecer contradictorio
  porque  $\varphi_X$ parece no  depender de  la dimensi\'on  y $d_X$  tiene una
  forma tal  que cambiar  de dimensi\'on parece  equivalente a cambiar  $d$. Sin
  embargo, la  dependencia de  \ $\varphi_X$  \ en \  $d$ \  es escondida  en el
  v\'inculo sobre el  grado de libertad \  $\nu > d-2$. Dicho de  otra manera, a
  $\nu$  dado, si  aumentamos  la dimensi\'on,  se  lo puede  hacer solamente  a
  condici\'on que $d < \nu+2$, \ie $\Phi_X \not\in \PD$ \ ($\Phi_X \in \PD_n, \:
  n < \nu+2$).   Tratando de \ $d_X$, las marginales de  $X$ son Student-$r$ con
  grado  de  libertad  $\nu$, pero  una  Student-$r$  no  puede ser  vista  como
  marginales  de  cualquier  dimensi\'on  m\'as  grande;  es  posible  hasta  la
  dimensi\'on m\'axima $\lceil \nu+2 \rceil-1$.

  Obviamente, sin calculos, se ve del soporte acotado de la densidad que \ $X$ \
  no  puede   ser  mezcla  de  escala   de  gaussiana.   Pero   recordar  de  la
  secci\'on~\ref{Sssec:MP:StudentR},  lema~\ref{Lem:MP:StudentRGamma}, que  \ $X
  \egald \frac{\sqrt{\nu+2} \: \Sigma^{\frac12} \, G}{\sqrt{V + \|G\|^2}} + m \,
  \sim \, \R_\nu( m , \Sigma )$  \ con \ $V \sim \G\left( \frac{\nu-d}{2}+1 \, ,
    \,  \frac12 \right)$  \  y \  $G \,  \sim  \, \N(0,I)$  \ $d$-dimensional  e
  independientes de $V$.  Parece una mezcla de gaussiana, pero con la generadora
  que ser\'ia dependiente de \ $G$ \ en este caso.
\end{ejemplo}

\begin{ejemplo}[Distribuci\'on uniforme sobre la esfera - continuaci\'on]\label{Ej:MP:UniformeCnt}
%
  Claramente, $U \sim  \U(\Sset_d)$ tampoco es consistente: las  marginales de $U$ no
  pueden ser uniformes sobre $\Sset_{d'}$;  sin calculo, se queda claro que para
  $d' =  d-1$, el  vector es  definido en la  bola $\Bset_{d-1}(\Rset)$,  y, por
  inducci\'on es lo mismo para cualquier dimensi\'on m\'as baja.

  En  calculos, recuendense del  ejemplo~\ref{Ej:MP:GeneCaracUniformeEsfera} que
  la generadora caracter\'istica toma la forma
  %
  \[
  \varphi_X\left(  w^2  \right)  =  2^{\frac{d}{2}-1}  \Gamma\left(  \frac{d}{2}
  \right) \, w^{1-\frac{d}{2}} \, J_{\frac{d}{2}-1}\left( w \right)
  \]
  %
  depende  explicitamente  de $d$.  M\'as  all\'a,  para $d'  <  d$  \  y \  $X'
  =  \begin{bmatrix} X_1 &  \cdots &  X_{d'} \end{bmatrix}^t$,  por transformada
  inversa de Hankel y~\cite[Ec.~6.575-1]{GraRyz15}, se obtiene
  %
  \[
  d_{X'}\left(     r^2      \right)     =     \frac{\Gamma\left(     \frac{d}{2}
    \right)}{\pi^{\frac{d'}{2}} \Gamma\left( \frac{d-d'}{2}  \right)} \left( 1 -
    r^2 \right)^{\frac{d-2-d'}{2}} \un_{(0 \; 1)}(r)
  \]
  %
  y no existe para  $d' > d$.  Claramente lar marginales depende  de ambas $d$ y
  $d'$;  no son  uniforma, como  lo vimos  sin  calculos y  no se  puede ver  la
  uniforma sobre la  esfera como marginal de un  vector aleatorio de dimensi\'on
  m\'as  alta.  De  hecho, se  puede notar  que las  marginales de  \ $U$  \ son
  Student-$r$ con \ $\nu = d-2$ \ el grado de libertad, a condici\'on que $\nu >
  d'-2$, es decir solamente cuando $d' < d$ (ver~\cite{titi, toto, Zoz}). Cuando
  \  $d'  =  d-1$, la  distribuci\'on  diverge  en  los  bordes del  dominio  de
  definici\'on.   Notar que si  no cambia  el grado  de libertad  para cualquier
  marginales,  depende de  la  dimensi\'on  $d$: la  generadora  de densidad  no
  depende solamente de $d'$.  Obviamente, del soporte acotado de la densidad, se
  vio que \ $X$ \ no pod\'ia ser mezcla de escala de gaussiana.
\end{ejemplo}

Si un vector a simetr\'ia el\'iptica no se escribe siempre como mezcla de escala
de   base   gaussiana,   existe   una   situaci\'on   que   podemos   ver   como
``intermediaria'': a veces se escribe como mezcla de escala de base Student-$r$.
Este  resultado  fue  probado  por  ejemplo   en  el  caso  escalar  \  $d=1$  \
en~\cite{Wil56, Kem71,  KeiSte74} (ver unos elementos  en~\cite{Dan51}), pero se
extiende sencillamente al caso multivariado:
%
\begin{teorema}[Mezcla de escala de base Student-$r$ -- a partir de la generadora caracter\'istica]
\label{Teo:MP:MezclaStudentRCaracteristica}
%
  Sea \  $X \sim \ED(m,\Sigma,\varphi_X)$  \ $d$-dimensional. Si  $\varphi_X \in
  \PDSI_n, \: n  > d$, entonces se  puede escribir $X$ como mezcla  de escala de
  Student-$r$ con $\nu = n-2$ el grado de libertad, y reciprocamente
  %
  \[
  \varphi_X \in  \PDSI_n, \:  n > d  \quad \Leftrightarrow  \quad X \egald  B \,
  \Sigma^{\frac12} \, S_{n-2} + m
  \]
  %
  con \ $B > 0$ \ independiente de \ $S_{n-2} \sim \R_{n-2}(0,I)$.
\end{teorema}
%
\begin{proof}
%Sin perdida de generalidad, consideramos $m = 0, \: \Sigma = I$.
  La prueba la  m\'as sencilla se apoya  sobre el hecho que si  \ $\varphi_X \in
  \PDSI_n$,   se  puede  considerar   un  vector   $n$-dimensional  \   $Y  \sim
  \ED(0,I,\varphi_X)$  \  de  generadora  caracter\'istica  \  $\varphi_X$  \  y
  escribir \ $Y \egald  R \, U$ \ con \ $U  \sim \U(\Sset_n)$, y reciprocamente.
  Ahora, del ejemplo~\ref{Ej:MP:UniformeCnt}, la proyecci\'on $d$-dimensional de
  \ $U$  \ es Student-$r$  con $n-2$ el  grado de libertad, y  reciprocamente se
  puede  ver  cualquier  Student-$r$  con   $n-2$  el  grado  de  libertad  como
  proyecci\'on de un uniforme  m\'as grande.  Adem\'as, $U$ siendo independiente
  de  $R$, la  proyecci\'on es  tambi\'en independiente  de $R$.  Eso  cierra la
  prueba.

  Alternativamente, escribiendo  la generadora  caracter\'istica a partir  de la
  mezcla        uniforme         $n$-dimensional        como        en        el
  teorema~\ref{Teo:MP:SchoenbergCaracteristica}, en
  %
  \[
  \varphi_X\left(  w^2 \right)  = \int_{\Rset_+}  2^{\frac{n}{2}-1} \Gamma\left(
    \frac{n}{2}   \right)   \left(    w   r   \sqrt{n}   \right)^{1-\frac{n}{2}}
  J_{\frac{n}{2}-1}\left(  w r  \sqrt{n}  \right) \,  dP_{R_d}\left( r  \sqrt{n}
  \right)
  \]
  %
  se  reconoce  en \  $  w  \mapsto  2^{\frac{d}{2}-1} \Gamma\left(  \frac{d}{2}
  \right)  \left(  \sqrt{d  w}  \right)^{1-\frac{d}{2}}  J_{\frac{d}{2}-1}\left(
    \sqrt{d w} \right)$  la generadora caracter\'istica de la  Student-$r$ con \
  $\nu = n-2$ \ el grado de libertad.
\end{proof}

Pasando, se notar\'a que para buscar la  ley de \ $B$, sufice ver la Student-$r$
como  proyecci\'on  de una  uniforme  de  dimensi\'on \  $d+2$,  \ie  \ $X  \sim
\ED(m,\Sigma,\varphi_X) \egald B \Sigma^{\frac12} S_{n-2} + m$, $d$-dimensional:
%
\[
\mbox{Construimos     }     \:\:     Y     \sim     \ED(0,I,\varphi_X)     \quad
(d+2)\mbox{-dimensional}  \qquad   \mbox{y}  \qquad  Y   \egald  R  \,   U  \:\:
\Leftrightarrow \:\: B \egald R
\]


De nuevo, existe un teorema equivalente tratando de la generadora de densidad:
%
\begin{teorema}[Mezcla de escala de base Student-$r$ -- a partir de la generadora de densidad]
\label{Teo:MP:MezclaStudentRDensidad}
%
  Sea \  $X \sim \ED(m,\Sigma,d_X)$ \  $d$-dimensional. Si $d_X \in  \CM_n, \: n
  \in  \Nset^*$,  entonces  se puede  escribir  $X$  como  mezcla de  escala  de
  Student-$r$ con $\nu = 2 n + d - 2$ el grado de libertad, y reciprocamente,
  %
  \[
  d_X \in  \CM_n, \:  n \in \Nset^*  \quad \Leftrightarrow  \quad X \egald  B \,
  \Sigma^{\frac12} \, S_{2 n + d - 2} + m
  \]
  %
  con \ $B > 0$ \ independiente de \ $S_{2 n + d - 2} \sim \R_{2 n + d - 2}(0,I)$.
\end{teorema}
%
\begin{proof}
  La rec\'iproca es inmediata de
  %
  \[
  d_X(r) = \frac{\Gamma\left( \frac{d}{2}  + n \right)}{\pi^{\frac{d}{2}} (2 n +
    d)^{\frac{d}{2}} \Gamma(n)} \, \int_{\Rset_+}  b^{-d} \left( 1 - \frac{r}{(2
      n + d) b^2} \right)_+^{\! n - 1} dP_B(b)
  \]
  %
  Si  $n   =  1$,  claramente  \  $\displaystyle   d_X(r)  =  \frac{\Gamma\left(
      \frac{d}{2}   +  1   \right)}{\pi^{\frac{d}{2}}   (d+2)^{\frac{d}{2}}}  \,
  \int_{\left( \sqrt{\frac{r}{2  n + d}}  \; +\infty \right)} b^{-d}  dP_B(b)$ \
  representa  una  medida de  \  $\left( \sqrt{\frac{r}{2  n  +  d}} \;  +\infty
  \right)$: es decreciente con \ $r$. Si $n \ge 2$, se muestra que se puede usar
  el teorema de  convergencia dominada para intercambiar derivada  e integral, y
  eso hasta el orden $n-1$, dando, para $0 \le k \le n-1$
  %
 \[
 (-1)^k     d_X^{(k)}(r)     =     \frac{\Gamma\left(    \frac{d}{2}     +     n
   \right)}{\pi^{\frac{d}{2}}   (2  n   +  d)^{\frac{d}{2}+k}   \Gamma(n-k)}  \,
 \int_{\Rset_+} b^{-d-2k} \left( 1 - \frac{r}{(2 n + d) b^2} \right)_+^{\! n - k
   - 1} dP_B(b) \: \ge \: 0
  \]
  %
  y  de nuevo  \ $\displaystyle  (-1)^{n-1} d_X^{(n-1)}(r)  = \frac{\Gamma\left(
      \frac{d}{2} + n \right)}{\pi^{\frac{d}{2}} (2 n + d)^{\frac{d}{2}+n-1}} \,
  \int_{\left( \sqrt{\frac{r}{2 n + d}} \; +\infty \right)} b^{-d-2n+2} dP_B(b)$
  \ es decreciente.

  Una prueba detallada  de la propiedad directa se  encuentra en~\cite{Wil56} en
  el  contexto  escalar,  pero  se  extiende  al  caso  multivariado  sin  costo
  adicional.  Basicamente, se proba que para \  $f \in \CM_n$, \ $\forall \: k =
  1,  \ldots ,  n-1$ \  y \  $\alpha >  0$, \  (i) \  $t^{k-1} f^{(k)}(t)$  \ es
  integrable sobre \  $(\alpha \; +\infty)$ \ y \  (ii) \ $\displaystyle \lim_{t
    \to +\infty} t^k f^{(k)}(t) = 0$. \SZ{Luego, sea \ $P_n$  \ tal que
  %
  \[
  P_n\big( (a \; b) \big) = (-1)^n \left( f^{(n-1)}(b) - f^{(n-1)}(a) \right)
  \]
  %
  Es una  medida por la decrecencia de  \ $(-1)^{n-1} f^{(n-1)}$ \  y finita del
  resultado (i)}.   Entonces, se  escribe el  desarollo de Taylor  en torno  a \
  $\alpha$, hasta el  orden \ $n-1$ \  con resto de la forma  integral, es decir
  cuando \ $\alpha \ge t$
  %
  \[
  f(t) =  \sum_{k=0}^{n-1} \frac{(t-\alpha)^k}{k!}  \,  f^{(k)}(\alpha) - (-1)^n
  \int_{(t \; \alpha)} \frac{(t-u)^{n-1}}{(n-1)!}  \, dP_n(u)
  \]
  %
  Escribiendo  \  $(t-u)^{n-1}  =  (-1)^{n-1}  u^{n-1} \left(  1  -  \frac{t}{u}
  \right)^{n-1}$ \  y aplicando  el desarollo anterior  a \ $d_X(r)$,  con el
  cambio de variables $u = (2 n + d) b^2$ \ se obtiene
  %
  \[
  d_X(r)  =  \sum_{k=0}^{n-1}  \frac{(r-\alpha)^k}{k!}  \,  d_X^{(k)}(\alpha)  +
  \frac{(2  n + d)^{n-1}}{\Gamma(n)}  \int_{\left( \sqrt{\frac{t}{2  n +  d}} \;
      \sqrt{\frac{\alpha}{2 n + d}} \right)} \left(  1 - \frac{r}{(2 n + d) b^2}
  \right)^{n-1} \, dP_n\big( (2 n + d) b^2 \big)
  \]
  %
  Ahora, dejando $\alpha$ tender al infinito, se nota que: \ $d_X(\alpha) \to 0$
  \ (generadora de densidad $d$-dimensional, $\alpha^{d-1} d_X(\alpha^2)$ \ debe
  tender a 0); \ del punto (ii), cada  termino de la suma tiende a 0. Queda solo
  el termino integral, que se escribe tambi\'en
  %
  \[
  d_X(r) =  \frac{(2 n +  d)^{n-1}}{\Gamma(n)} \int_{\Rset_+} b^{-d} \left(  1 -
    \frac{r}{(2 n + d) b^2} \right)_+^{\!n-1} \, b^{2 n +d - 2} dP_n\big( (2 n +
  d) b^2 \big)
  \]
  %
  Tiene precisamente la forma de mezcla de Student-$r$ $d$-dimensional con grado
  de libertad \ $2 n + d - 2$.
  % , con \ $P_B$ \ definida por \
  %  %
  % \[
  %   P_B(A)  =   \frac{\pi^{\frac{d}{2}}  (2   n  +   d)^{\frac{2  n   +   d  -
  %       2}{2}}}{\Gamma\left( \frac{d}{2} +  n \right)} \int_A b^{2 n  + d - 2}
  % \, dP_n\big( (2 n + d) b^2 \big) \qquad \mbox{con} \qquad \SZ{P_n\big( (a \;
  %   b) \big) = (-1)^n \left( f^{(n-1)}(b) - f^{(n-1)}(a) \right)}
  % \]
\SZ{Bien fijar que todo listo.}
\end{proof}

Pasando, se notar\'a que la ley de \ $B$ \ es tambi\'en dada por
%
\[
P_B(A) = \frac{\pi^{\frac{d}{2}} (2 n + d)^{\frac{2 n + d - 2}{2}}}{\Gamma\left(
    \frac{d}{2} + n \right)}  \int_A b^{2 n + d - 2} \, dP_n\big(  (2 n + d) b^2
\big)  \qquad \mbox{con}  \qquad \SZ{P_n\big(  (a \;  b) \big)  =  (-1)^n \left(
    f^{(n-1)}(b) - f^{(n-1)}(a) \right)}
\]


Nota: obviamente, si denotamos \  $\mathfrak{S}_{n,d}$ \ el conjunto de vectores
aleatorios  $d$-dimensjonal, mezcla de  Student-$r$ con  \ $n-2$  \ el  grado de
libertad, $\mathfrak{S}_{d+1,d} \supset \mathfrak{S}_{d+2,d} \supset \cdots$ \ y
\    $\displaystyle    \mathfrak{S}_{+\infty,d}   =    \bigcap_{n=d+1}^{+\infty}
\mathfrak{S}_{n,d}$    \   es    el    conjunto   de    mezcla   de    gaussiana
$d$-dimensional. As\'i, se puede imaginar  tambi\'en buscar la ley de una mezcla
de gaussiana pasando por la de mezcla de Student-$r$, tomando el l\'imite cuando
$n$ tiende al infinito de la ley de $B$.


% --------------------------------- Familia  eliptica compleja

\subsubseccion{Caso complejo en unas palabras}
\label{Ssec:MP:FamiliaElipticaCompleja}


V\'imos en la secci\'on~\ref{Sec:MP:VectoresComplejosMatricesAleatorias} el caso
de vectores aleatorio reales, y la noci\'on de circularidad. No es equivalente a
la de esf\'ericidad, a\'un que hay v\'inculos, como lo vamos a ver. De hecho, la
necesidad de  trabajar con vector a  simetr\'ia el\'iptica se  justifica con los
mismos argumentos que  llevaron al estudio de vectores  aleatorios complejos. Se
encontrar\'a estudios  de esta familia en  referencias tales que~\cite{KriLin86,
  MicDey06,  OllEri11,  OllTyl12, FanKot90,  BesAbr13,  BauPas07, ChiPas08}  por
ejemplo. Damos en esta secci\'on lo esencial.

Antes de  ir m\'as  all\'a, empezamos  con la definici\'on  de tales  vectores a
simetr\'ia el\'iptica.
%
\begin{definicion}[Vector complejo esf\'ericamente invariante]
  Sea  \  $Z$ \  vector  aleatorio $d$-dimensional  complejo.   $X$  \ es  dicho
  esfericamente  invariante,  o   rotacionalmente  invariante,  o  a  simetr\'ia
  esf\'erica,  o  de  distribuci\'on  esf\'erica  \  si  para  cualquier  matriz
  unitaria~\footnote{Recordarse que \ $V$ \ es  unitaria si \ $V V^\dag = V^\dag
    V = I$.} \ $V$,
  %
  \[
  V  Z  \: \egald  \: Z
  \]
  %
\end{definicion}

A continuaci\'on,  como en  el caso real,  se extiende haciendo  estiramientos y
transformaci\'on unitaria (equivalente de una rotaci\'on) de manera siguiente:
%
\begin{definicion}[Vector complejo a simetr\'ia el\'iptica]\label{Def:MP:ElipticoComplejo}
  Sea  \ $Z$  \ vector  aleatorio $d$-dimensional  complejo.  $Z$  \ es  dicho a
  simetr\'ia  el\'iptica,   o  elipticalmente  invariante,   o  de  distribuci\'on
  el\'iptica, en torno a \ $m \in  \Cset^d$, \ si existe una matriz \ $\Sigma \in
  \Pos_d^+(\Cset)$ \ tal que para cualquier matriz unitaria \ $V \in \Unit_d(\Cset)$,
  %
  \[
  V  \,   \Delta^{-\frac12}  \,  Q^\dag  \left(   X  -  m  \right)   \:  \egald  \:
  \Delta^{-\frac12} \, Q^\dag \left( X - m \right)
  \]
  %
  donde la matriz real diagonal \ $\Delta  > 0$ \ es la matriz de autovalores de
  \  $\Sigma$  \ y  \  $Q  \in \Unit_d(\Cset)$  \  la  matriz  de los  autovectores
  correspondientes~\cite{Bha97,   Bha07,  HorJoh13},  \   $\Sigma  =   Q  \Delta
  Q^\dag$. Es  decir, \ $\Delta^{-\frac12} Q^\dag \left(  X - m \right)$  \ es a
  simetr\'ia esf\'erica.

  De  nuevo, $m$  \  es  un par\'ametro  de  posici\'on y  \  $\Sigma$ \  matriz
  caracter\'istica.
\end{definicion}
%
Obviamente, se queda en el caso complejo la indeterminaci\'on mediante un factor
escalar.

El v\'inculo entre la simetr\'ia el\'iptica y la circularidad es tambi\'en obvia:
%
\begin{lema}
  Sea \ $Z$ \ vector complejo  $d$-dimensional a simetr\'ia el\'iptica en torno a
  \ $m \in \Cset^d$. Entonces \ $Z-m$ \ es circular.
\end{lema}
%
\begin{proof}
  La prueba es  inmediata de \ $Z-m \egald  Q \, \Delta^{\frac12} \, Y$  \ con \
  $Y$ \  a simetr\'ia  esf\'erica. Se  cierra la prueba  de, \  $e^{\imath \theta}
  (Z-m) \egald Q \, \Delta^{\frac12} \,  \big( e^{\imath \theta} I \big) \, Y$ \
  conjuntamente a \ $e^{\imath \theta} I \in \Unit_d(\Cset)$.
\end{proof}
%Como antes,  para estudiar  los vectores a  simetr\'ia el\'iptica, sin  perdida de
%generalidad se puede restringirnos al caso a simetr\'ia esf\'erica.

Es importante notar  las inmersiones biyectivas de \ $\Cset^d$  \ en \ $\Rset^{2
  d}$ \  y de $\Mat_{d',d}(\Cset)$ \  en un subconjunto de  \ $\Mat_{2 d', 2d}(\Rset)$
siguientes:
%
\[
\forall \: z  \in \Cset^d \qquad \mbox{en biyecci\'on  con} \qquad \widetilde{z}
= \begin{bmatrix} \real{z}\\ \imag{z} \end{bmatrix} \in \Rset^{2 d}
\]
%
\[
\forall  \:  M \in  \Mat_{d',d}(\Cset)  \qquad  \mbox{en  biyecci\'on con}  \qquad
\overline{M}   =   \begin{bmatrix}   \real{M}   &  -   \imag{M}\\   \imag{M}   &
  \real{M} \end{bmatrix} \in \Mat_{2d',2d}(\Rset)
\]
%
Claramente para matrices $M, N$ y un vector $z$,
%
\[
M z \quad \mbox{es en biyecci\'on con} \quad \overline{M} \, \widetilde{z}, \qquad
M N \quad \mbox{es en biyecci\'on con} \quad \overline{M} \, \overline{N}
\]
%
Llamaremos   {\em    forma   real}   ambas   inmersiones    y   las   escrituras
precedientes. Adem\'as, se ve sencillamente que
%
\begin{itemize}
\item  $\displaystyle   \overline{M^\dag}  =  \overline{M}^{\,   t}$,
%  \quad  con  consecuencia
%
\item $\displaystyle M \in \Her_d(\Cset) \:\: \Leftrightarrow \:\:
  \overline{M}  \in  \Sim_{2d}(\Rset)  \:\:  \Leftrightarrow  \:\:  \real{M}^t  =
  \real{M} \, \et \, \imag{M}^t = - \imag{M}$,
%
\item  $\forall \:  M \in  \Her_d(\Cset), \:  z  \in \Cset^d,  \quad z^\dag  M z  =
  \widetilde{z}^{\, t} \, \overline{M} \, \widetilde{z}$,
%
\item $M \in \Unit_d(\Cset) \quad \Leftrightarrow \quad \overline{M} \in \Ort_{2d}(\Rset)$.
\end{itemize}
%
%\[
%\overline{M} \:\: \mbox{unitaria}  \quad \Leftrightarrow \quad \overline{M} \:\:
%\mbox{ortogonal}
%\]

Ahora, se v\'incula naturalmente la noci\'on de elipticidad del caso complejo al
caso real:
%
\begin{lema}
  Sea \ $Z$ \ vector complejo  $d$-dimensional a simetr\'ia el\'iptica en torno a
  \  $m$.  Entonces,  $\widetilde{Z}$ \  \ vector  real $2d$-dimensionales  es a
  simetr\'ia el\'iptica en torno a \ $\widetilde{m}$.
\end{lema}
%
\begin{proof}
  Se obtiene la forma real de la descomposici\'on \ $\Sigma = Q \Delta Q^\dag$ \
  bajo  la forma  \ $\overline{\Sigma}  = \overline{Q}  \,  \overline{\Delta} \,
  \overline{Q}^t$ \ con \ $\overline{\Delta}  = \begin{bmatrix} \Delta & 0\\ 0 &
    \Delta  \end{bmatrix}  \in \Pos_{2d}^+(\Rset)$  \  diagonal, $\overline{Q}  \in
  \Ort_{2d}(\Rset)$.      La     prueba    se     cierra     saliendo    de     la
  definici\'on~\ref{Def:MP:ElipticoComplejo} escrita bajo su forma real, notando
  que $\overline{\Delta^{-\frac12}} = \overline{\Delta}^{\: -\frac12}$.
\end{proof}

De esta equivalencia, todos  los resultados anteriores se extienden naturalmente
al caso  complejo. Para \ $Z$ \  a simetr\'a el\'iptica en  torno a \ $m$  \ y de
matriz caracter\'istica \ $\Sigma$:
%
\begin{itemize}
\item  De la  secci\'on~\ref{Ssec:MP:VAComplejos} se  obtiene  $\Phi_Z(\omega) =
  \Phi_{\widetilde{Z}}\left(    \widetilde{\omega}     \right)    =    e^{\imath
    \widetilde{\omega}^t       \widetilde{m}}      \varphi_{\widetilde{Z}}\left(
    \widetilde{\omega}^t  \, \overline{\Sigma}  \, \widetilde{\omega}  \right) =
  e^{\imath  \real{\omega^\dag m}} \varphi_{\widetilde{Z}}\left(  \omega^\dag \,
    \Sigma \omega  \right)$;\newline Denotaremos $\varphi_{\widetilde{Z}} \equiv
  \varphi_Z$ \ quien, con  $m$ y $\Sigma$ caracteriza completamente $Z$;\newline
  Escribiremos $Z \sim \CED\left( m , \Sigma , \varphi_Z \right)$.
%
\item Similarmente, si \ $Z$ \  admite una densidad, se obtiene la densidad bajo
  la  forma~\footnote{Se  notar\'a  que,  escribiendo las  formas  diagonales  \
    $\Sigma  =  Q  \Delta  Q^\dag$  \ y  $\overline{\Sigma}  =  \overline{Q}  \,
    \overline{\Delta}  \, \overline{Q}^t$ \  con \  $Q$ \  y \  $\overline{Q}$ \
    respectivamente  unitaria y ortogonal,  tenemos \  $\left| \overline{\Sigma}
    \right|  = \left|  \overline{\Delta}  \right| =  \left|  \Delta \right|^2  =
    \left| \Sigma  \right|^2$.}  \ $p_Z(z) =  |\Sigma|^{-1} d_Z\left( (z-m)^\dag
    \Sigma^{-1} (z-m)  \right)$ \ con  \ $d_Z \equiv  d_{\widetilde{Z}}: \Rset_+
  \mapsto \Rset_+$.
%
\item  De eso,  se extiende  naturalmente la mayoria de los teoremas:
%
  \begin{itemize}
  %
  \item  \SZ{ Momentos  y  cumulantes~\ref{Teo:MP:MomentosCumulantesEliptica}, y
      corrolario~\ref{Lem:MP:MediaCovarianzaEliptica}}~\cite{Kri76}
  %
  \item  Teorema~\ref{Teo:MP:TranformacionAfinEliptica}  en:  \  $Z \,  \sim  \,
    \CED(m,\Sigma,\varphi_Z)$ \  $d$-dimensional, $A \in  \Mat_{d',d}(\Cset)$ \ de
    rango lleno tal que  \ $d' \le d$ \ y \  $c \in \Cset^{d'} \quad \Rightarrow
    \quad A X + c \: \sim \: \CED(A m + c , A \Sigma A^\dag , \varphi_Z)$;
  %
  \item Teorema~\ref{Teo:MP:ProyeccionComponentesEliptica} en: Sea \ $Z$, vector
    aleatorio   complejo   $d$-dimensonal   de   componentes   $Z_i$.   Entonces
    $\displaystyle Z  \sim \CED(m,\Sigma,\varphi_Z) \quad  \Leftrightarrow \quad
    \forall \:  a \in  \Cset^d, \quad a^\dag  (Z - m)  \egald \sqrt{\frac{a^\dag
        \Sigma a}{\Sigma_{i,i}}} (Z_i - m_i)$;
  %
  \item Teorema~\ref{Teo:MP:MaxwellHershell} en:  \ $Z \sim \CED(m,I,\varphi_Z)$
    \ tiene  sus componentes independientes si  y solamente si \  $Z \sim \CN(m,
    \alpha I)$ con $\alpha > 0$;
  %
  \item  Teorema~\ref{Teo:MP:MezclaUniforme} en:  \ $Z  \sim \CED(0,I,\varphi_Z)
    \:\: \Leftrightarrow  \:\: Z \egald R \,  U$ \:\: con \:\:  $U \sim \U\left(
      \SCset_d \right)$  \ uniforme sobre  la esfera $d$-dimensional  compleja y
    m\'as generalmente, para \ $Z \sim \ED(m,\Sigma,\varphi_Z)$, \quad $Y \egald
    \Sigma^{\frac12} R \, U + m$.
  %
  \item  Corolario~\ref{Cor:MP:MezclaUniforme} en: $Z  \sim \CED(0,I,\varphi_Z)$
    da $\| Z \|  \egald R$ \ y \ $\frac{Z}{\|Z\|} \egald  U \sim \U(\SCset_d)$ \
    son independientes;
  %
  \item \SZ{Lema para alpha momentos \ref{Lem:MP:AlphaConR}};
  %
  \item Teorema~\ref{Teo:MP:DensidadRayo} en: \ $Z  \sim \CED(0,I,d_Z)$ \ y \ $R
    \egald \|Z\|$ \  admite una densidad que se  escribe $\displaystyle p_R(r) =
    \frac{2 \pi^d}{\Gamma(d)} \, r^{2 d-1} \, d_X\left( r^2 \right)$;
  %
  \item   Teorema~\ref{Teo:MP:TransformadaDeHankel}   en:   para   \   $Z   \sim
    \CED(m,\Sigma,\varphi_Z)   \equiv  \CED(m,\Sigma,d_Z)$  \   las  generadoras
    caracter\'isticas y de  densidad son relacionadas por $\displaystyle
    \varphi_Z\left( w^2 \right) =  \left( 2 \pi \right)^d w^{1-d} \int_{\Rset_+}
    r^d  d_Z\left(  r^2  \right)  \,  J_{d-1}(  r  w)  \,  dr  \quad$  y  $\quad
    \displaystyle  d_Z\left( r^2  \right) =  \left( 2  \pi  \right)^{-d} r^{1-d}
    \int_{\Rset_+} w^d \varphi_Z\left( w^2 \right) \, J_{d-1}( r w) \, dw$;
  %
  % \item Coordenadas hiperesf\'ericas
  %
  \item Teorema~\ref{Teo:MP:SchoenbergCaracteristica} en: \ $Z \sim \CED\left( m
      , \Sigma , \varphi_Z \right) \egald \, A \, \Sigma^{\frac12} \, G + m \:\:
    \Leftrightarrow \:\: \varphi_Z  \in \PDSI$, \ con \ $A  > 0$ \ independiente
    de \ $G \sim \CN(0,I)$;
  %
  \item  Teorema~\ref{Teo:MP:SchoenbergDensidad} en:  \ $Z  \sim \CED\left(  m ,
      \Sigma  , d_Z  \right)  \egald \,  A \,  \Sigma^{\frac12}  \, G  + m  \:\:
    \Leftrightarrow \:\: d_Z \in  \CM$, \ con \ $A > 0$  \ independiente de \ $G
    \sim \CN(0,I)$;
  %
  \item Lema~\ref{Lem:MP:AIndependienteD} en: \ $\varphi_Z \in \PDSI$ (o $d_Z \in
    \CM$)   \   y    para   cualquier   dimension   \   $d$    \   y   $Z   \sim
    \CED(m,\Sigma,\varphi_Z)$ \  $d$-dimensional, en la escritura \  $X \egald A
    \, \Sigma^{\frac12}  G + m$  \ con \  $A > 0$ \  independiente de \  $G \sim
    \CN(0,I)$ la ley de \ $A$ \ no depende de la dimension $d$;
  %
  \item   Teorema~\ref{Teo:MP:MezclaStudentRCaracteristica}   en:   \  $Z   \sim
    \CED(m,\Sigma,\varphi_Z)$ \ $d$-dimensional con \ $\varphi_z \in \PDSI_n, \:
    n > 2 d$, dando la escritura  estoc\'astica \ $Z$ \ como mezcla de escala de
    Student-$r$ compleja con $\nu = n-2$ el grado de libertad, y reciprocamente:
    \ $\varphi_Z \in \PDSI_n, \: n >  2 d \quad \Leftrightarrow \quad X \egald B
    \, \Sigma^{\frac12}  \, S_{n-2} + m$  \ con \ $B  > 0$ \  independiente de \
    $S_{n-2} \sim \CR_{n-2}(0,I)$ (ver secci\'on~\ref{Sssec:MP:StudentR});
  %
  \item    Teorema~\ref{Teo:MP:MezclaStudentRDensidad}    en:    \    $Z    \sim
    \CED(m,\Sigma,d_Z)$  \  $d$-dimensional con  \  $d_X  \in  \CM_n, \:  n  \in
    \Nset^*$, dando la escritura estoc\'astica \  $Z$ \ como mezcla de escala de
    Student-$r$  con  $\nu  =  2  n  +  2  d  -  2$  el  grado  de  libertad,  y
    reciprocamente: \  $d_Z \in  \CM_n, \: n  \in \Nset^*  \quad \Leftrightarrow
    \quad Z \egald B \, \Sigma^{\frac12} \, S_{2 n +  2 d - 2} + m$ \ con \ $B >
    0$ \ independiente de \ $S_{2 n + 2 d - 2} \sim \CR_{2 n + 2 d - 2}(0,I)$.
  \end{itemize}
%
\end{itemize}


% --------------------------------- Familia  eliptica matricial

\subsubseccion{Caso matriz variado en unas palabras}
\label{Ssec:MP:FamiliaElipticaMatriz}

% Fang Zang: Generalized Multivariate Analysis
La  extensi\'on  matriz  variada  de   vector  a  simetr\'ia  el\'iptica  no  es
trivial. De hecho, hay pocas resultados en la literatura y aparece esencialmente
en  contexto de estimaci\'on  de matriz  de covarianza~\cite[\S.~13.2]{BilBre99}
o~\cite{Mal61,   Tyl82,  GruRoc90,   GupVar94,   GupVar95,  gupNag99,   FanLi99,
  CarGon16}.   Tipicamente, si  uno  sale de  \  $X_1, \ldots,  X_n$ \  vectores
aleatorios $d$-dimensionales independientes, de  misma ley, y suponemos la media
cero  para  simplificar  el  ejemplo,  un  estimador natural  de  la  matriz  de
covarianza \  $\Sigma_X$ \ es \ $\displaystyle  \widehat{\Sigma}_X = \frac{1}{n}
\sum_{i=1}^n X_i X_i^t$, matriz  de $\Sim_d(\Rset)$ (y a\'un de $\Pos_d(\Rset)$,
y de  $\Pos_d^+(\Rset)$ casi siempre si  $n \le d$).  Ahora, si los $X_i$  son a
sim\'etria elipt\'ica, $\widehat{\Sigma}_X$ va  a tener simetr\'ia por cambio de
escala  y rotaciones.  M\'as precisamente,  si  se supone  $X_i$ esf\'erica  por
ejemplo, para  cualquier rotaci\'on $O$  se tiene $O  X_i \egald X_i$,  dando $O
\widehat{\Sigma}_X O^t  \egald \widehat{\Sigma}_X$.  Esta propiedad  dio lugar a
une definici\'on de  una matriz a simetr\'ia esf\'erica aparece  por unas de las
primeras veces  en~\cite{Mal61}.  Ahora,  si uno quierre  estimar una  matriz de
covarianza  $\Sigma_{X,Y}$  con  $X$   vector  $d$-dimensional  e  $Y_i$  vector
$d'$-dimensional   a    partir   de   copias    independientes,   $\displaystyle
\widehat{\Sigma}_{X,Y}  =   \frac{1}{n}  \sum_{i=1}^n  X_i   Y_i^t$,  matriz  de
$\Mat_{d,d'}(\Rset)$. Se observa que cuando  los $X_i$ e $Y_i$ son esfericas por
ejemplo, para cualquier  par de rotaciones $O$ de  $\Mat_{d,d}(\Rset)$ y $O'$ de
$\Mat_{d',d'}(\Rset)$     se    tiene    $\widehat{\Sigma}_{X,Y}     \egald    O
\widehat{\Sigma}_{X,Y}    \egald     \widehat{\Sigma}_{X,Y}    O'    \egald    O
\widehat{\Sigma}_{X,Y} O'$.

Empezamos con las  definiciones de~\cite[\S~9.3]{GupNag99} o~\cite{Daw77, Daw78,
  Daw81, FanChe84}, que contiene la de~\cite{Mal61, Tyl82}.
%Es  precisamente lo que va  a dar la noci\'on  de matriz a
%simetr\'ia esf\'erica, y luego el\'iptica.

\begin{definicion}[Matriz a simetr\'ia esf\'erica]
  Sea  $X$ matriz  aleatoria definida  sobre  $\X =  \Mat_{d,d'}(\Rset)$.  $X$  es
  dicha:
  %
  \begin{enumerate}%[label={\normalsize (\roman*)}]
  \item[(i)] a simetr\'ia esf\'erica a la  derecha si para cualquier matriz \ $O'
    \in \Ort_{d'}(\Rset)$ \ se tiene
    %
    \[
    X O' \: \egald \: X;
    \]
  %
  \item[(ii)] A simetr\'ia esf\'erica a  la izquierda si para cualquier matriz \
    $O \in \Ort_d(\Rset)$ \ ortogonal se tiene
    %
    \[
    O X \: \egald \: X;
    \]
  %
  \item[(iii)] A  simetr\'ia esf\'erica si para  cualquier par de  matrices \ $O
    \in \Ort_d(\Rset)$ \ y \ $O' \in \Ort_{d'}(\Rset)$ \ se tiene
    %
    \[
    O X O' \: \egald \: X;
    \]
  %
    (ambas derecha y izquierda).
  %
  \item[(iv)]  cuando $d'  = d$  y $\X  = \Sim_d(\Rset)$,  a  simetr\'ia esf\'erica
    fuerte si para cualquier matriz \ $O \in \Ort_d(\Rset)$ \ se tiene
    %
    \[
    O X O^t \: \egald \: X
    \]
   %
    (ambas derecha y  izquierda, con los v\'inculos adicionales  de simetr\'ia y
    $O'=O^t$).
\end{enumerate}
\end{definicion}
%

Naturlamente, se  extiende como en el  caso de vectores con  estiramientos y una
translaci\'on:
%
\begin{definicion}[Vector a simetr\'ia el\'iptica]
%
  Sea $X$ matriz aleatoria definida  sobre $\X = \Mat_{d,d'}(\Rset)$.  Si existe
  una matriz $m \in  \Mat_{d,d'}(\Rset)$ y matrices $\Sigma \in \Pos_d^+(\Rset)$
  y $\Sigma' \in \Pos_{d'}^+(\Rset)$ tales que:
  %
  \begin{enumerate}%[label={\normalsize (\roman*)}]
  \item[(i)] si para cualquier matriz \ $O' \in \Ort_{d'}(\Rset)$ se tiene
    %
    \[
    (X-m) Q' \Delta'^{-\frac12} O' \: \egald \: (X-m) Q' \Delta'^{-\frac12}
    \]
    %
    $X$ es dicha a a simetr\'ia el\'iptica a la derecha en torno de $m$.
  %
  \item[(ii)] si para cualquier matriz \ $O \in \Ort_d(\Rset)$ se tiene
    %
    \[
    O \, \Delta^{-\frac12} Q^t (X-m) \: \egald \: \Delta^{-\frac12} Q^t (X-m)
    \]
    %
    $X$ es dicha a a simetr\'ia el\'iptica a la izquierda en torno de $m$.
  %
  \item[(iii)] si para  cualquier par de matrices \ $O \in  \Ort_d(\Rset)$ \ y \
    $O' \in \Ort_{d'}(\Rset)$ se tiene
    %
    \[
    O  \, \Delta^{-\frac12}  Q^t (X-m)  Q'  \Delta'^{-\frac12} O'  \: \egald  \:
    \Delta^{-\frac12} Q^t (X-m) Q' \Delta'^{-\frac12}
    \]
    %
    $X$ es dicha a simetr\'ia el\'iptica en torno de $m$.
  %
  \item[(iv)] si  cuando $d' = d$  y $\X = \Sim_d(\Rset)$,  $m = \Sim_d(\Rset)$,
    para cualquier matriz \ $O \in \Ort_d(\Rset)$ \ se tiene
    %
    \[
    O \Delta^{-\frac12} Q^t (X-m) Q \Delta^{-\frac12} O^t \: \egald \: \Delta^{-\frac12} Q^t (X-m) Q \Delta^{-\frac12}
    \]
    %
    $X$ es dicha a simetr\'ia fuertemente el\'iptica en torno de $m$.
  \end{enumerate}
  %
  donde la  matriz diagonal \ $\Delta  > 0$ \ es  la matriz de  autovalores de \
  $\Sigma$ \ y \ $Q$ \ la matriz de los autovectores correspondientes, \ $\Sigma
  = Q  \Delta Q^t$ y  similarmente la matriz  diagonal \ $\Delta'  > 0$ \  es la
  matriz  de  autovalores  de \  $\Sigma'$  \  y  \  $Q'$  \ la  matriz  de  los
  autovectores correspondientes, \ $\Sigma' = Q' \Delta' Q'^t$.

  Llamaremos tambi\'en a $m$ \  {\em par\'ametro  de posici\'on} y  \ las  matrices \
  $\Sigma, \Sigma'$ \ {\em matrices caracter\'isticas}.

  % Se denotar\'a  respectivamente \ $X  \sim \DED(m,S,\varphi_X)$ \ en  el caso
  % (i),  \ $X  \sim \IED(m,\Sigma,\varphi_X)$  \  en el  caso (ii),  \ $X  \sim
  % \ED(m,\Sigma,S,\varphi_X)$    \   en    el   caso    (iii),   \    $X   \sim
  % \FED(m,\Sigma,\varphi_X)$ \ en el caso (iv).}
\end{definicion}

Si denotamos  $\DED, \: \IED, \:  \ED$ y $\ED$ \  las clases de los  casos (i) a
(iv) respectivamente, se tiene obviamente
%
\[
\DED \, \cap \, \IED \: = \: \ED \quad \mbox{y, para } \: d'=d, \quad \FED \subset \ED
\]


\SZ{ADELANTAR funcion caracteristica, covarianzas empiricas media cero cae, y sacando la media empirica cae en tal clase}
 
\vspace{2cm}

\centerline{\underline{\hspace{10cm}}}

\SZ{
\begin{itemize}

\item Volver a los cumulantes $2 k$: se puede decir m\'as?

\item Citar mi HDR~\cite[Sec.~3.2.1]{Zoz12}

\item Ver BilBre def 13 o \cite{Kri76}: extension matricial

\item Ver Shiryayev, \SZ{Fang and Anderson, ``Statistical inference in elliptically contoured and related distributions'' a buscar}

\end{itemize}

}





\centerline{\underline{\hspace{10cm}}}

\SZ{hablar de simulaci\'on? Metoto inverso,  mezcla (aparece en la concavidad de
  la entropia), rejeccion, a traves de la condicional para el caso vectorial?}

