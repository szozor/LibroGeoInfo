\subsubseccion{Distribuci\'on exponencial}
\label{Sssec:MP:Exponencial}

A pesar de que sea un caso particular de la distribuci\'on Gamma que vamos a ver
m\'as  adelante,  estudiada por  Pearson  desde  el  a\~no 1895~\cite{Pea95},  o
apareci\'o  quizas   un  poco   antes  en  trabajos   de  L.   Boltzmann   o  de
Whitworth~\cite{BalBas95} (como caso lim\'ite  de la ley de Poisson), apareci\'o
esta  ley  de  manera  ``propia''   mucho  m\'as  tarde,  entre  otros  en  1930
en~\cite[Ec.~(46)]{Kon30}.

Se denota $X \,  \sim \, \E(\lambda)$ \ con \ $\lambda  \in \Rset_{0,+}$ \ llamada
{\em  taza}  (inversa  de  {\em   escala}),  y  sus  caracter\'isticas  son  las
siguientes:

\begin{caracteristicas}
%
Dominio de definici\'on & $\X = \Rset_+$\\[2mm]
\hline
%
Par\'ametro & $\lambda \in \Rset_{0,+}$\\[2mm]
\hline
%
Densidad  de probabilidad &  $\displaystyle p_X(x)  = \lambda  e^{-\lambda x}$\\[2mm]
\hline
%
Promedio & $\displaystyle m_X = \frac1\lambda$\\[2mm]
\hline
%
Varianza & $\displaystyle \sigma_X^2 = \frac1{\lambda^2}$\\[2mm]
\hline
%
\modif{Asimetr\'ia} & $\gamma_X = 2$\\[2mm]
\hline
%
Curtosis por exceso & $\widebar{\kappa}_X = 6$\\[2mm]
\hline
%
Generadora de  momentos &  $\displaystyle M_X(u) =  \frac{\lambda}{\lambda-u}$ \
para \ $\real{u} < \lambda$\\[2mm]
\hline
%
Funci\'on     caracter\'istica     &     $\displaystyle     \Phi_X(\omega)     =
\frac{\lambda}{\lambda - \imath \omega}$
\end{caracteristicas}

% Momentos & $ \Esp\left[ X^k \right] = p^k$\\[2mm]
% Momento factorial & $\Esp\left[ (X)_k \right] = ?$\\[2mm]
% Generadora de probabilidad & $G_X(z) = e^{\lambda (z-1)}$ \ para \ $z \in \Cset$\\[2mm]
% modo 0
% Mediana \ln(2)/\lambda
% CDF 1-e^{-\lambda x}

Unas densidades  de probabilidad  y funciones de  repartici\'on asociadas son representadas  en la
figura Fig.~\ref{Fig:MP:Exponencial} para varios par\'ametros.
%
\begin{figure}[h!]
\begin{center} \begin{tikzpicture}%[scale=.9]
\shorthandoff{>}
%
\pgfmathsetmacro{\sx}{.75};% x-scaling
\pgfmathsetmacro{\r}{.05};% radius arc non continuity F_X
\pgfmathsetmacro{\l}{1.5};% lambda
\pgfmathsetmacro{\mx}{6};% x maximo del plot
%
% densidad
\begin{scope}
%
\pgfmathsetmacro{\sy}{2.5/\l};% y-scaling 
\draw[>=stealth,->] ({-\sx-.25},0)--({\sx*\mx+.25},0) node[right]{\small $x$};
\draw[>=stealth,->] (0,-.1)--(0,{\sy*\l+.25}) node[above]{\small $p_X$};
%
\draw[thick] ({-\sx},0)--(0,0);
\draw (\r,\r) arc (90:270:\r);
\draw[dotted] (0,0)--(0,{\sy*\l}) node[scale=.4]{$\bullet$};
\draw[thick,domain=0:\mx,samples=100] plot ({\x*\sx},{\sy*\l*exp(-\l*\x)});
%
%\draw (0,{((1-\p)^\n)*\sy})--(-.1,{((1-\p)^\n)*\sy}) node[left,scale=.7]{$(1-p)^n$};
%\draw (0,{\n*\p*((1-\p)^(\n-1))*\sy})--(-.1,{\n*\p*((1-\p)^(\n-1))*\sy}) node[left,scale=.7]{$n p (1-p)^{n-1}$};
%
\end{scope}
%
%
% reparticion
\begin{scope}[xshift=8.5cm]
%
\pgfmathsetmacro{\sy}{2.5};% y-scaling 
%
\draw[>=stealth,->] (-.6,0)--({\sx*\mx+.25},0) node[right]{\small $x$};
\draw[>=stealth,->] (0,-.1)--(0,{\sy+.25}) node[above]{\small $F_X$};
%
% cumulativa
\draw[thick,domain=0:\mx,samples=100] (-.5,0)--(0,0) plot({\x*\sx},{(1-exp(-\l*\x))*\sy});
%
%\draw (0,{((1-\p)^\n)*\sy})--(-.1,{((1-\p)^\n)*\sy}) node[left,scale=.7]{$(1-p)^n$};
%%\draw (0,{(\n*\p+1-\p)*((1-\p)^(\n-1))*\sy})--(-.1,{(\n*\p+1-\p)*((1-\p)^(\n-1))*\sy}) node[left,scale=.7]{$(1-p+np) (1-p)^{n-1}$};
%\draw (-.2,{((\n*\p+1-\p)*((1-\p)^(\n-1))+1)/2*\sy}) node[scale=.7]{$\vdots$};
%\draw (0,\sy)--(-.1,\sy) node[left,scale=.7]{\small $1$};
\end{scope}
%
\end{tikzpicture} \end{center}
% 
\leyenda{Ilustraci\'on  de una densidad  de probabilidad  exponencial (a),  y la
  funci\'on  de  repartici\'on asociada  (b),  con  $\lambda  = \frac32$  (linea
  llena),  $\lambda  =  1$  (linea   guionada)  y  $\lambda  =  \frac12$  (linea
  punteada).}
\label{Fig:MP:Exponencial}
\end{figure}
%\SZ{Poner escalas; Otros ilustraciones para otros $\lambda$?}

La  ley exponencial  es conocida  como siendo  {\em sin  memoria}, es  decir, si
buscamos \  $X$ \  visto como un  tiempo (ej.  tiempo de desintegraci\'on  de un
at\'omo radioactivo) tal que
%
\[
\forall \: x_0 \ge 0, \, x \ge 0, \quad P( X > x+x_0 | X > x_0) = P(X > x)
\]
%
\ie  la  probabilidad  que  $X  >  x+x_0$  (extra  tiempo  despu\'es  de  $x_0$)
condicionalmente  a $X >  x_0$ es  exactamente la  de $X  > x$  (se olvid\'o
$x_0$). De hecho, tenemos de la definici\'on de la probabilidad condicional
%
\[
\forall \: x_0 \ge 0, \, x \ge 0, \quad P(X > x+x_0 | X > x_0) = \frac{P\big( (X > x+x_0) \cap (X > x_0) \big)}{P(X > x_0)}
% = \frac{1-F_X(x+x_0)}{1-F_X(x_0)} = 1-F_X(x)
\]
%
Ahora, de $(X > x+x_0) \subset (X > x_0)$ se obtiene
%
\[
\forall \: x_0 \ge 0, \, x \ge 0, \quad P(X > x+x_0 | X > x_0) = \frac{1-F_X(x+x_0)}{1-F_X(x_0)} = 1-F_X(x)
\]
%
De la  densidad de  probabilidad, tenemos  para $x >  0$, $F_X(x)  = \left(  1 -
  e^{-\lambda x} \right) \uno_{\Rset_+}(x)$, as\'i  que $1 - F_X(x) = e^{-\lambda
  x}$. Poniendo este resultado el la expresi\'on anterior se obtiene finalmente
%
\[
\forall \: x_0 \ge 0, \, x \ge 0,  \quad P(X > x+x_0 | X > x_0) = e^{-\lambda x}
= 1-F_X(x) = P(X > x)
\]
%
%Por diferenciaci\'on con respecto a $x$ eso da, en $x \to 0$,
%%
%\[
%%\forall  \: x_0  \ge 0,  \quad  F_X'(x_0) +  \lambda F_X(x_0)  = \lambda  \qquad
%\mbox{con} \qquad \lambda = F_X'(0)
%\]
%%
%Teneiendo en cuenta de que $F_X$  es una funci\'on de repartici\'on, aparece que
%$F_X(x) = \left( 1 - e^{-\lambda x} \right) \uno_{\Rset_+}(x)$, ley exponencial.

Como  lo hemos  evocado  tratando de  la  ley de  Poisson,  esta es  v\'inculada
intimamente a la ley exponencial a trav\'es del processo dicho de poisson:
%
\begin{lema}[V\'inculo con la ley de Poisson]
\label{Lem:MP:VinculoExponencialPoisson}
%
  Sea  $T_0 =  0$ \  y \  $\forall \:  n \in  \Nset_0$ las  variables aleatorias
  positivas \ $T_n$ \ tales que $T_{n+1}  - T_n \ge 0$ \ son independientes y de
  distribuci\'on $\E(\lambda)$. Fijamos \ $T > 0$ \ y sea \ $X$ \ el n\'umero de
  variables  \  $T_n$  \  que  partenecen  a   $(0  \;  T)$,  \ie  $T_X  <  T  <
  T_{X+1}$. Entonces
  %
  \[
  X \sim \P(\lambda T)
  \]
\end{lema}
%
Dicho  de otra manera,  si tenemos  eventos que  aparecen en  tiempos aleatorios
tales  que los  incrementos de  tiempos entre  eventos son  independientes  y de
distribuci\'on  exponencial de  taza $\lambda$,  el  n\'umero de  eventos en  un
intervalo  de tiempo $T$  dado sigue  una ley  de Poisson,  de taza  $\lambda T$
proporcional al  intervalo, y proporcional a  la taza de la  ley exponencial. El
par\'ametro \ $\lambda$ \ representa la taza de evento por unidad de tiempo.
%
\begin{proof}
Por definici\'on,
%
\begin{eqnarray*}
P(X = n) & = & P(X\le n) - P(X \le n-1) \\[2mm]
%
& = & P(T_{n+1} > T) - P(T_n > T)
\end{eqnarray*}
%
Es decir
%
\[
P(X = n) = F_{T_n}(T) - F_{T_{n+1}}(T)
\]
%
Ahora, notando que
%
\[
T_n = \sum_{i=0}^{n-1} \left( T_{i+1} - T_i \right)
\]
%
de la  independencia de los  incrementos de tiempo,  y de las propiedades  de la
funci\'on caracter\'istica, tenemos
%
\[
\Phi_{T_n}(\omega) = \frac{\lambda^n}{(1-\imath \, \omega)^n}
\]
%
De  la  f\'ormula  de  inversion del  teorema~\ref{Teo:MP:InversionDensidad}  se
prueba  que~\footnote{Una manera  es  de  hacer una  integraci\'on  en el  plano
  complejo  y usar  los lemas  de Jordan  y teorema  de residuos~\cite{CarKro05}
  o~\cite[Cap.~4]{AblFok03}.  Nota: de hecho se  reconoce en \ $\Phi_{T_n}$ \ la
  funci\'on caracter\'istica de una ley gamma \ $\G(n,\lambda)$, ley que vamos a
  ver en la secci\'on~\ref{Sssec:MP:Gamma}.}
%
\[
p_{T_n}(x) = \frac{\lambda^n x^{n-1} e^{-\lambda x}}{(n-1)!} \uno_{\Rset_+}(x)
\]
%
Con integraciones por partes, se obtiene sencillamente
%
\[
F_{T_n}(T) = 1 - \sum_{i=0}^{n-1} \frac{\lambda^i T^i e^{-\lambda T}}{i!}
\]
%
Se concluye poniendo  este resultado en la expresi\'on $P(X =  n) = F_{T_n}(T) -
F_{T_{n+1}}(T)$ que obtuvimos.
\end{proof}
%
En  f\'isica,  se  modela la  ley  de  tiempo  de desintegraci\'on  como  siendo
exponencial,  y  se supone  que  las  desintegraciones  son independientes.  Eso
justifica  modelo de  Poisson  para modelizar  el  n\'umero de  desintegraci\'on
durante un tiempo dado.

Una otra caracter\'istica de esta ley  es su stabilidad con respecto al operador
no lineal m\'inimo:
%
\begin{lema}[Stabilidad por el m\'in]
\label{Lem:MP:StabilidadExponencialMinimo}
%
  Sean  $X_i \sim \E(\lambda), \: i = 1, \ldots , n$ \  independientes. Entonces,
  \[
  \min_{i=1,\ldots,n} X_i \equiv X \sim \E(n \lambda)
  \]
\end{lema}
%
\begin{proof}
Inmediatamente, para cualquier $x \ge 0$
%
\begin{eqnarray*}
1-F_X(x) & = & P(X > x) \\[2mm]
%
& = & P\left( \bigcap_{i=1}^n \big( X_i > x \big) \right)\\[2mm]
%
& = & \prod_{i=1}^n P(X_i > x)\\[2mm]
%
& = & e^{- n \lambda x}
\end{eqnarray*}
%
La   secunda   linea   viene   de   la  equivalencia   entre   los   eventos   $
\min_{i=1,\ldots,n}  X_i >  x$ y  $\bigcap_{i=1}^n  \big( X_i  > x  \big)$ y  la
tercera de la independencia de los $X_i$.
\end{proof}
