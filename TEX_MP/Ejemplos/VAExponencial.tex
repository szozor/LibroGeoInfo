\subsubseccion{Distribuci\'on exponencial}
\label{Sssec:MP:Exponencial}

Se denota $X \,  \sim \, \E(\lambda)$ \ con \ $\lambda  \in \Rset_+^*$ \ llamada
{\em  taza}  (inversa  de  {\em   escala}),  y  sus  caracter\'isticas  son  las
siguientes:

\begin{caracteristicas}
%
Dominio de definici\'on & $\X = \Rset_+$\\[2mm]
\hline
%
Parametro & $\lambda \in \Rset_+^*$\\[2mm]
\hline
%
Densidad  de probabilidad &  $\displaystyle p_X(x)  = \lambda  e^{-\lambda x}$\\[2mm]
\hline
%
Promedio & $\displaystyle m_X = \frac1\lambda$\\[2mm]
\hline
%
Varianza & $\displaystyle \sigma_X^2 = \frac1{\lambda^2}$\\[2mm]
\hline
%
\modif{Sesgo} & $\gamma_X = 2$\\[2mm]
\hline
%
Curtosis por exceso & $\widebar{\kappa}_X = 6$\\[2mm]
\hline
%
Generadora de  momentos &  $\displaystyle M_X(u) =  \frac{\lambda}{\lambda-u}$ \
para \ $\real{u} < \lambda$\\[2mm]
\hline
%
Funci\'on     caracter\'istica     &     $\displaystyle     \Phi_X(\omega)     =
\frac{\lambda}{\lambda - \imath \omega}$
\end{caracteristicas}

% Momentos & $ \Esp\left[ X^k \right] = p^k$\\[2mm]
% Momento factorial & $\Esp\left[ (X)_k \right] = ?$\\[2mm]
% Generadora de probabilidad & $G_X(z) = e^{\lambda (z-1)}$ \ para \ $z \in \Cset$\\[2mm]
% modo 0
% Mediana \ln(2)/\lambda
% CDF 1-e^{-\lambda x}

Su densidad  de probabilidad  y funci\'on de  repartici\'on son representadas  en la
figura Fig.~\ref{Fig:MP:Exponencial}.
%
\begin{figure}[h!]
\begin{center} \begin{tikzpicture}%[scale=.9]
\shorthandoff{>}
%
\pgfmathsetmacro{\sx}{.75};% x-scaling
\pgfmathsetmacro{\r}{.05};% radius arc non continuity F_X
\pgfmathsetmacro{\l}{1.5};% lambda
\pgfmathsetmacro{\mx}{6};% x maximo del plot
%
% densidad
\begin{scope}
%
\pgfmathsetmacro{\sy}{2.5/\l};% y-scaling 
\draw[>=stealth,->] ({-\sx-.25},0)--({\sx*\mx+.25},0) node[right]{\small $x$};
\draw[>=stealth,->] (0,-.1)--(0,{\sy*\l+.25}) node[above]{\small $p_X$};
%
\draw[thick] ({-\sx},0)--(0,0);
\draw (\r,\r) arc (90:270:\r);
\draw[dotted] (0,0)--(0,{\sy*\l}) node[scale=.4]{$\bullet$};
\draw[thick,domain=0:\mx,samples=100] plot ({\x*\sx},{\sy*\l*exp(-\l*\x)});
%
\draw (0,{\l*\sy})--(-.1,{\l*\sy}) node[left,scale=.7]{$\lambda$};
%
\end{scope}
%
%
% reparticion
\begin{scope}[xshift=8.5cm]
%
\pgfmathsetmacro{\sy}{2.5};% y-scaling 
%
\draw[>=stealth,->] (-.6,0)--({\sx*\mx+.25},0) node[right]{\small $x$};
\draw[>=stealth,->] (0,-.1)--(0,{\sy+.25}) node[above]{\small $F_X$};
%
% cumulativa
\draw[thick,domain=0:\mx,samples=100] (-.5,0)--(0,0) plot({\x*\sx},{(1-exp(-\l*\x))*\sy});
%
\draw (0,\sy)--(-.1,\sy) node[left,scale=.7]{$1$};
\end{scope}
%
\end{tikzpicture} \end{center}
% 
\leyenda{Ilustraci\'on  de una densidad  de probabilidad  exponencial (a),  y la
funci\'on de repartici\'on asociada (b), con $\lambda = 1.5$.}
\label{Fig:MP:Exponencial}
\end{figure}
\SZ{Poner escalas; Otros ilustraciones para otros $\lambda$?}

Cuando $\lambda \to +\infty$ la variable tiende a una variable cierta $X = 0$.
\SZ{Esta distribuci\'on aparece... Propiedades}
% en  el conteo  de conteo  de  une repetici\'on  de una  experiencia de  maneja
% independiente hasta que  occure un evento de probabilidad  $p$; por ejemplo el
% n\'umero de tiro de un dado  equilibriado hasta que occurre un ``6'' sigue una
% ley geometrica de parametro $p = \frac16$.