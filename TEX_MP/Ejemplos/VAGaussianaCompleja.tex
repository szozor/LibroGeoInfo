\subsubseccion{Distribuci\'on normal o gaussiana multivariada complejas}
\label{Sssec:MP:GaussianaComplejas}

Por definici\'on, un vector aleatorio complejo $d$-dimensional \ $Z = X + \imath
Y$ \  es gaussiano significa que  el vector $2 d$-dimensional  \ $\widetilde{Z} =
\begin{bmatrix}  X^t &  Y^t \end{bmatrix}^t$  \ es  gaussiano. Se  puede entonces
referirse  en el  caso de  vectores gaussianos,  pero como  lo hemos presentado  en la
secci\'on~\ref{Ssec:MP:VAComplejos}, es frecuentemente m\'as comodo trabajar con
\  $Z$ \  en lugar  de \  $\widetilde{Z}$.
%  En  particular, en  el marco  de las
%comunicaciones en  ingeneria, se  trabaja con modulaciones  dichas en fase  y en
%cuadratura (se\~nal multiplicado  respectivamente por un seno y  un coseno) y en
%lugar  de trabajar  con dos  componente se  considera una  modulaci\'on  con una
%exponencial  compleja y  la  se\~nal/variable compleja.   Se  puede por  ejemplo
%referirse  a~\cite{Lap17} (ver en  particular el  capitulo~24) o~\cite{SchSch03,
%  EriKoi06}.

En el caso general, la gaussiana  real siendo completamente descrita por su media
y su matriz de covarianza, la  gaussiana compleja va a ser completamente definida
por   la  media,   la  matriz   de  covarianza   y  la   pseudo-covarianza  (ver
Sec.~\ref{Sec:MP:VectoresComplejosMatricesAleatorias}  por las  relaciones entre
la  covarianza  de  \ $\widetilde{Z}$  \  y  estas  matrices).

Se denota \ $Z \, \sim \, \CN(m,\Sigma,\check{\Sigma})$ \ con \ $m \in \Cset^d$,
\ $\Sigma \in P_d^+(\Cset)$ \ conjunto  de las matrices de \ $\M_{d,d}(\Cset)$ \
herm\'iticas definidas positivas, y  \ $\check{\Sigma} \in S_d(\Cset)$ \ conjunto
de las matrices de \  $\M_{d,d}(\Cset)$ \ symmetricas (ver notaciones).  Un caso
particular  aparece cuando  \ $Z$  \  es propio  en torno  a  \ $m$,  lo que  es
equivalente en el caso gaussiano a tener \ $Z$ \ circular (ver m\'as adelante) en
torno a  \ $m$, dado  cuando \  $\check{\Sigma} = 0$:  en este caso  usaremos la
misma notaci\'on,  $Z \,  \sim \, \CN(m,\Sigma)$.   Las caracter\'isticas  de la
gaussiana  compleja   son  las  siguientes~\cite{Lap17,   Pic96,  Goo63,  Bos95,
  SchSch03, EriKoi06}:

\begin{caracteristicas}
%
Dominio de definici\'on & $\Z = \Cset^d$\\[2mm]
\hline
%
Par\'ametros & $m \in \Cset^d, \:\: \Sigma \in P_d^+(\Cset), \:\: \check{\Sigma}
\in S_d(\Cset)$\\[2mm]
\hline
%
Densidad de probabilidad & \\[1mm]
%
Caso general: & $\displaystyle p_Z(z) = \frac{1}{\pi^d \left| \Sigma
 \right|^{\frac12} \left| P \right|^{\frac12}} \: e^{- (z-m)^\dag P^{-1} (z-m) +
 \real{(z-m)^t R^t P^{-1} (z-m)}}$\vspace{2.5mm}\newline
 con~\footnote{En~\cite{Pic96} la expresi\'on es ligieramente diferente, pero se
 recupera usando la simetr\'ia \ $\check{\Sigma}^* =
 \check{\Sigma}^\dag$. Recordar que \ $\cdot^{-*} = \left( \cdot^* \right)^{-1}$
 \ (ver notaciones).} \ $P = \Sigma - \check{\Sigma} \, \Sigma^{-*}
 \, \check{\Sigma}^\dag, \quad R = \check{\Sigma}^\dag \, \Sigma^{-1}$.\\[2.5mm]
%
Caso circular: & $\displaystyle p_Z(z) = \frac{1}{\pi^d \left| \Sigma \right|}
 \: e^{- (z-m)^\dag \Sigma^{-1} (z-m)}$\\[2.5mm]
\hline
%
Promedio & $ m_Z = m$\\[2mm]
\hline
%
Covarianza & $\Sigma_Z = \Sigma$\\[2mm]
\hline
%
Pseudo-covarianza & $\check{\Sigma}_Z = \check{\Sigma}$\\[2mm]
\hline
%%
%Generadora de  momentos &  $\displaystyle M_X(u) =  e^{u^t \Sigma u + u^t m}$  \ para \  $u \in
%\Cset^d$\\[2mm]
%\hline
%%
Funci\'on caracter\'istica & \\[1mm]
%
Caso general: & $\displaystyle \Phi_Z(\omega) = e^{-\frac14
\omega^\dag \Sigma \omega - \frac14 \real{\omega^\dag \check{\Sigma} \omega^*} +
\imath \real{\omega^\dag m}}, \quad \omega \in \Cset^d$\\[1mm]
%
Caso circular: & $\displaystyle \Phi_Z(\omega) = e^{-\frac14
\omega^\dag \Sigma \omega +
\imath \real{\omega^\dag m}}, \quad \omega \in \Cset^d$
\end{caracteristicas}

Notar que  en el caso  escalar propio (circular),  la varianza de  \ $Z$ \  es \
$\sigma_Z^2  = 2  \sigma^2$.  El coefficiente  2 viene  del  hecho que  \ $Z$  \
contiene dos componentes independientes de varianza $\sigma^2$.

Los vectores aleatorios complejos van a compartir las propiedades del caso real,
siendo equivalente  a un vector  $2d$-dimensional gaussiano real.

% De  manera  general,  las caracter\'isticas  de  \  $X$  \ gaussiano  real  son
% v\'inculadas a las  de \ $N$ \ (y vice-versa)  por transformaci\'on afine (ver
% secciones anteriores).

Primero, los cumulantes de orden superior o igual a $4$ valen cero:
%
\begin{lema}[Gaussiana compleja y cumulantes]
%
  Sea \ $Z$ \ vector aleatorio complejo de media $m$, de covarianza $\Sigma$, de
  pseudo-covarianze  $\check{\Sigma}$ y  de  secunda funci\'on  caracter\'istica
  admtiendo un desarollo de Taylor. Entonces para cualquier
  %
  \[
  \kappa_{i_1,\ldots,i_l,i'_1,\ldots,i'_m}[Z] = 0 \quad \forall \: ( i_1, \ldots
  , i_l  , i'_1  , i'_m  \in \{  1 , \ldots  , d  \}^{l+m}, \,  l+m \ge  4 \quad
  \Longleftrightarrow \quad X \sim \CN(m,\Sigma,\check{\sigma})
  \]
  %
\end{lema}
%
%\begin{proof}
%  La pueba sigue paso a paso la del lema~\ref{Lem:MP:CumSecFctCarac}.
%\end{proof}

Secundo, como en  el caso real, la gaussiana es  estable por combinaci\'on lineal
de vectores independientes:
%
\begin{teorema}[Stabilidad]
\label{Teo:MP:StabilidadGaussianaCompleja}
%
  Sean \ $A_i , i = 1,\ldots,n$  \ matrices de \ $\M_{d',d}(\Cset), \: d' \le d$
  \  de   rango  lleno,   $b_i  \in  \Cset^{d'}$   \  y   \  $Z_i  \,   \sim  \,
  \CN(m_i,\Sigma_i,\check{\Sigma}_i)$   \   $d$-dimensionales,  independientes,
  entonces
  %
  \[
  \sum_{i=1}^n \left( A_i  Z_i + b_i \right) \,  \sim \, \CN\left( \sum_{i=1}^n
    \left( m_i + b_i \right) \, ,  \, \sum_{i=1}^n A_i \Sigma_i A_i^\dag \, , \,
    \sum_{i=1}^n A_i \check{\Sigma}_i A_i^t \right)
  \]
  %
  En particular, cualquier combinaci\'on lineal  de los componentes de un vector
  gaussiano complejo  da una  gaussiana compleja.  Reciprocamente,  si cualquier
  combinaci\'on lineal de  los componentes de un vector  aleatorio sigue una ley
  gaussiana compleja, entonces el vector es gaussiano complejo.
\end{teorema}
%
El corolario~\ref{Cor:MP:MediaEmpiricaGauss} se extiende naturalmente al caso complejo:
%
\begin{corolario}[Media emp\'irica]\label{Cor:MP:MediaEmpiricaGaussCompleja}
%
  Sean \ $Z_i \, \sim \, \CN\left(  m , \Sigma , \check{\Sigma} \right), \: i =
  1, \ldots , n$ \ independientes. Entonces,
  %
  \[
  \overline{Z} =  \frac{1}{n} \sum_{i=1}^n Z_i \,  \sim \, \CN\left( m  \, , \,
    \frac{1}{n} \Sigma \, , \, \frac{1}{n} \check{\Sigma} \right)
  \]
  %
\end{corolario}

Adem\'as, en el  caso complejo se tiene  una estabilidad combinando \ $Z$  \ y \
$Z^*$:
%
\begin{teorema}%[Stabilidad]
\label{Teo:MP:StabilidadGaussianaComplejaZZestrella}
%
Sean \ $A \in \M_{d',d}(\Cset)$, \  $B \in \M_{d',d}(\Cset)$ \ tales que \ ambas
\ $A+B$ \ y \  $A-B$ \ \SZ{sean de rango lleno}, $c \in \Cset^{d'}$  \ y \ $Z \,
\sim \, \CN(m,\Sigma,\check{\Sigma})$ \ $d$-dimensonal, entonces
  %
  \[
  A Z + B Z^* + c \: \sim \: \CN\left( \mu , C , \check{C}  \right)
  \]
  %
  con
  %
  \[
  \begin{array}{lll}
  \mu & = & A m + B m^* + c\\[2.5mm]
  %
  C & = & A \Sigma A^\dag + B \Sigma^* B^\dag + A \check{\Sigma} B^\dag + B
  \check{\Sigma}^* A^\dag\\[2.5mm]
  %
  \check{C} & = & A \check{\Sigma} A^t + B \check{\Sigma} B^t + A \Sigma B^t + B
  \Sigma^* A^t
  \end{array}
  \]
\end{teorema}
\begin{proof}
  Tomando la forma real $2d$-dimensional \ $Z = X + \imath Y$ \ con $X, Y$
  reales,   es  en   biyecci\'on  con   $\widetilde{Z}  =   \begin{bmatrix}  X\\
    Y   \end{bmatrix}$  \  y   entonces  \   $Z^*$  \   en  biyecci\'on   con  \
  $\widetilde{Z^*} = \begin{bmatrix} X\\-Y \end{bmatrix}$. Eso da \ $A Z + B Z^*
  +   c$   \   en   biyecci\'on   con   $\begin{bmatrix}   A+B   &   0\\   0   &
    A-B \end{bmatrix} \begin{bmatrix} X\\ Y \end{bmatrix} + \begin{bmatrix}
    \real{c}\\ \imag{c} \end{bmatrix}$. Notando que \ $\begin{bmatrix} A+B & 0\\
    0    &   A-B    \end{bmatrix}$   \    es    de   rango    lleno,   por    el
  teorema~\ref{Teo:MP:StabilidadGaussiana} este vector es gaussiano, lo que proba
  que \  $A Z  + B  Z^* + c$  \ es  gaussiano complejo. Las  formas de  la media,
  covarianza y  pseudo-covarianza siguen de calculos directos  de la expresi\'on
  $A Z + B Z^* + c$.
\end{proof}
%
Evidentemente, se puede combinar los dos teoremas anteriores.

El teorema del l\'imite central y sus variantes se recuperan del caso real.
%
\begin{teorema}[Teorema del l\'imite central (caso complejo)]
\label{Teo:MP:CLTComplejo}
%
Sea  \ $\{  Z_i \}_{i  \in  \Nset^*}$ \  una sucesi\'on  de vectores  aleatorios
independientes, de  misma ley, y  que admiten un  promedio \ $m$, una  matriz de
covarianza   \    $\Sigma$   \   y    una   matriz   de    pseudo-covarianza   \
$\check{\Sigma}$. Entonces
  %
  \[
  \frac{1}{\sqrt{n}}  \sum_{i=1}^m  \left( Z_i  -  m  \right)  \: \limitd{n  \to
    +\infty} \: Z \sim \CN\left( 0 , \Sigma , \check{\Sigma} \right)
  \]
  %
  donde  \ $\limitd{}$ \  significa que  el l\'imite  es en  distribuci\'on (ver
  notaciones).
\end{teorema}
%
%
Como en el caso real, aparece que  la media emp\'irica hechas a partir de vectores
complejos independientes de media \  $m$, admitiendo una covarianza \ $\Sigma$ \
una  pseudo-covarianza \  $\check{\Sigma}$, y  de misma  ley  (no necesariamente
gaussiana),  tiende a  ser gaussiana  compleja  de media  \ $m$,  de covarianza  \
$\frac{1}{n} \Sigma$, y de pseudo-covarianza \ $\frac{1}{n} \check{\Sigma}$.

No   lo   presentamos,  pero   se   transpone   sencillamente   el  teorema   de
Lindenberg-Feller~\ref{Teo:MP:LindenbergFeller} al caso complejo.

Notamos tambi\'en que, en el caso  circular, se puede escribir naturalmente \ $Z
\, \egald \, \Sigma^{\frac12} N + m$ \  con \ $N \, \sim \, \CN(0,I)$ \ donde \
$N$ \  es dicha  {\em Gaussiana estandar}  o {\em centrada-normalizada}.   Eso se
generaliza en dos direcciones.  La  primera pone tambi\'en en juega una gaussiana
estandar~\cite{Lap17}:
%
\begin{teorema}
\label{Teo:MP:GaussianaComplejaWWestrella}
%
Sea \  $Z \sim  \CN(m,\Sigma,\check{\Sigma})$.  Entonces, existen  matrices (no
\'unicas) \ $A \in \M_{d,d}(\Cset)$, \ $B \in \M_{d,d}(\Cset)$ \ tales que
  %
  \[
  Z \egald A W + B W^* + m
  \]
  %
  con \ $W \sim \CN(0,I)$ \ gaussiana estandar.
  %
\end{teorema}
\begin{proof}
  Inmediatamente
  %
  \[
  Z   =  \begin{bmatrix}   I   &  \imath   I\end{bmatrix}  \begin{bmatrix}   X\\
    Y\end{bmatrix}
  \egald \begin{bmatrix} I & \imath I\end{bmatrix} M \begin{bmatrix} U\\
    V\end{bmatrix}
  \]
  %
  con \ $U \sim \N(0,I)$ \ y \  $V \sim \N(0,I)$ \ independientes, y \ $M$ \ tal
  que  \ $M  M^t =  \begin{bmatrix} \Sigma_X  & \Sigma_{X,Y}  \\  \Sigma_{X,Y}^t &
    \Sigma_Y  \end{bmatrix}$  \   (ej.  raiz  cuadrade  de  esta   matriz  de  \
  $P_{2d}^+(\Rset)$,    o     descomposici\'on    de    Cholesky~\cite{HorJoh13,
    Bha07}). Ahora, volviendo a la forma compleja tenemos
  %
  \[
  Z     \egald   \begin{bmatrix}    I   &   \imath   I\end{bmatrix}
  M \begin{bmatrix} I  & I\\ -\imath I &  \imath I \end{bmatrix} \begin{bmatrix}
    \frac12 (U + \imath V)\\ \frac12 (U - \imath V)\end{bmatrix}
  \]
  %
  Se cierra la prueba denotando
  %
  \[
  \begin{bmatrix}   A   &   B\end{bmatrix}   =  \begin{bmatrix} I &
    \imath  I\end{bmatrix}  M  \begin{bmatrix}  I  &  I\\  -\imath  I  &  \imath
    I \end{bmatrix}
  \]
  %
  y notando que \ $W \equiv \frac12 (U + \imath V) \sim \CN(0,I)$.
\end{proof}
%
Notar  que, usando  la descomposici\'on  de Cholesky,  tenemos \  $M$ triangular
inferior~\footnote{Se puede hacer el  mismo razonamiento con la forma triangular
  superior;  se cambia  los roles  de \  $X$  \ e  \ $Y$  \ en  las matrices  de
  covarianza.},  y  entonces bloc-triangular  inferior  \  $M =  \begin{bmatrix}
  \alpha  &  0  \\  \beta  &  \gamma \end{bmatrix}$.   Eso  conduce,  a  $M  M^t
= \begin{bmatrix} \Sigma_X & \Sigma_{X,Y} \\ \Sigma_{X,Y}^t &
  \Sigma_Y \end{bmatrix}  = \begin{bmatrix} \alpha \alpha^t &  \alpha \beta^t \\
  \beta \alpha^t &  \beta \beta^t + \gamma \gamma^t  \end{bmatrix}$.  Eso da por
ejemplo   \   $\alpha  =   \Sigma_X^{\frac12}$,   \   $\beta  =   \Sigma_{X,Y}^t
\Sigma_X^{-\frac12}$  \   y  \  $\gamma  =  \left(   \Sigma_Y  -  \Sigma_{X,Y}^t
  \Sigma_X^{-1} \Sigma_{X,Y} \right)^{\frac12}$.   A continuaci\'on, $A = \alpha
+  \gamma +  \imath \beta$  \ y  \ $B  = \alpha  - \gamma  + \imath  \beta$. Una
soluci\'on posible es entonces
%
\[
\left\{\begin{array}{lll}
A & = & \Sigma_X^{\frac12} + \left( \Sigma_Y - \Sigma_{X,Y}^t \Sigma_X^{-1} \Sigma_{X,Y}
\right)^{\frac12}  + \imath \Sigma_{X,Y}^t
\Sigma_X^{-\frac12}\\[2.5mm]
%
B & = &  \Sigma_X^{\frac12} - \left( \Sigma_Y - \Sigma_{X,Y}^t \Sigma_X^{-1} \Sigma_{X,Y}
\right)^{\frac12} + \imath \Sigma_{X,Y}^t
\Sigma_X^{-\frac12}
\end{array}\right.
\]
%
Se  puede   re-escribir  estas  matrices   a  partir  de   \  $\Sigma$  \   y  \
$\check{\Sigma}$       \      usando       las       relaciones      de       la
secci\'on~\ref{Ssec:MP:VAComplejos}.

La  secunda  extensi\'on  pone  en  juega  una sola  gaussiana  compleja  sin  su
conjugada~\cite{EriKoi06, SchSch03}:
%
\begin{teorema}
\label{Teo:MP:GaussianaComplejaWIDiago}
%
  Sea \ $Z \sim \CN(m,\Sigma,\check{\Sigma})$. Entonces, existe una matriz \ $C
  \in \M_{d,d}(\Cset)$ \ tal que
  %
  \[
  Z \egald C W + m
  \]
  %
  con \ $W \sim \CN(0,I,\Delta)$ \ con \ $\Delta \in P_d(\Rset)$ \ (real) diagonal.
  %
\end{teorema}
\begin{proof}
  Eso  viene  de teoremas  de  diagonalizaci\'on  conjunta. M\'as  precisamente,
  siendo \ $\Sigma  \in P_d^+(\Cset)$ \ y \  $\check{\Sigma} \in S_d(\Cset)$, se
  aplica el  teorema~\cite[Teo.~7.6.5]{HorJoh13} diciendo que  existe una matriz
  no  singular (invertible)  \  $C$ \  tal  que \  $\Sigma  = C  C^\dag$  \ y  \
  $\check{\Sigma} = C  \Delta C^t$ \ con $\Delta$ \  real diagonal con elementos
  positivos  ($\Delta  \in  P_d(\Rset)$  \  diagonal).  Inmediatamente,  por  el
  teorema~\ref{Teo:MP:StabilidadGaussianaCompleja}, tenemos
  %
  \[
  C^{-1} (Z - m) \egald W \sim \CN(0,I,\Delta)
  \]
  lo que cierra la prueba.
\end{proof}


Al final, v\'imos en la  secci\'on~\ref{Ssec:MP:VAComplejos} que si un vector es
circular, entonces su pseudo-covarianza es  nula, pero la rec\'iproca no vale en
general. Aparece que en el contexto gaussiano tenemos la rec\'iproca:
%

\begin{teorema}[Circularidad]\label{Teo:MP:CircularidadGaussiana}
%
Sea \ $Z \, \sim \, \CN(m,\Sigma,\check{\Sigma})$.  Entonces,
  %
  \[
  Z \: \mbox{ circular  en torno a } \: m \qquad  \Longleftrightarrow \qquad Z \:
\mbox{ propio en torno a } \: m
  \]
\end{teorema}
%
\begin{proof}
  V\'imos     la    directa    en     la    secci\'on~\ref{Ssec:MP:VAComplejos},
  teorema~\ref{Teo:MP:Circularidad}.  Reciprocamente,  si \  $Z$ \ es  propio en
  torno a \ $m$, por definici\'on \ $\check{\Sigma} = 0$ \ y el resultado viene
  de la forma de  la funci\'on caracter\'istica por ejemplo: $\Phi_{Z-m}(\omega)
  = e^{-\frac14 \omega^\dag \Sigma \omega } = \Phi_{Z-m}\left( e^{\imath \theta}
    \omega \right) = \Phi_{e^{\imath \theta} (Z-m)}(\omega)$.
\end{proof}


%\SZ{Caso $X \sim \N(m,\Sigma)$; modulaci\'on $Z = e^{\imath \theta} X$}