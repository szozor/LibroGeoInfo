\subsubseccion{Distribuci\'on normal o Gaussiana multivariada complejas}
\label{Sssec:MP:GaussianaComplejas}

Por definici\'on, un vector aleatorio complejo $d$-dimensional \ $Z = X + \imath
Y$ \  es gausiano significa que  el vector $2 d$-dimensional  \ $\widetilde{Z} =
\begin{bmatrix}  X^t &  Y^t \end{bmatrix}^t$  \ es  gausiano. Se  puede entonces
referirse  en el  caso de  vectores gausianos,  pero como  lo presentamos  en la
secci\'on~\ref{Ssec:MP:VAComplejos}, es frecuentemente m\'as comodo trabajar con
\  $Z$ \  en lugar  de \  $\widetilde{Z}$.  En particular,  en el  marco de  las
comunicaciones en  ingeneria, se  trabaja con modulaciones  dichas en fase  y en
cuadratura (se\~nal multiplicado  respectivamente por un seno y  un coseno) y en
lugar  de trabajar  con dos  componente se  considera una  modulaci\'on  con una
exponencial  compleja  y la  se\~nal/variable  compleja.  Se  puede por  ejemplo
referirse a~\cite{Lap17} (ver en particular el capitulo~24).

En el caso general, la gausiana real siendo completamente descritar por su media
y su matriz de covarianza, la  gausiana compleja va a ser completamente definida
por   la  media,   la  matriz   de  covarianza   y  la   pseudo-covarianza  (ver
Sec.~\ref{Sec:MP:VectoresComplejosMatricesAleatorias}  por las  relaciones entre
la  covarianza  de  \ $\widetilde{Z}$  \  y  estas  matrices).   $Z \,  \sim  \,
\C\N(m,\Sigma,\check{\Sigma})$  \   con  \  $m  \in  \Cset^d$,   \  $\Sigma  \in
P_d^+(\Cset)$ \  conjunto de  las matrices de  \ $\M_{d,d}(\Cset)$  \ hermiticas
definida  positivas, y  \  $\check{\Sigma}  \in S_d(\Cset)$  \  conjunto de  las
matrices  de  \  $\M_{d,d}(\Cset)$  \  symmetricas (ver  notaciones).   Un  caso
particular  aparece cuando  \ $Z$  \ es  propio en  torno de  \ $m$,  lo  que es
equivalente en el caso gausiano a tener \ $Z$ \ circular (ver m\'as adelante) en
torno de  \ $0$, dado cuando  \ $\check{\Sigma} =  0$: en este caso  usaremos la
misma notaci\'on,  $Z \,  \sim \, \C\N(m,\Sigma)$.  Las caracter\'isticas  de la
Gaussiana compleja son las siguientes~\cite{Lap07, Pic96, Bos95}:

\begin{caracteristicas}
%
Dominio de definici\'on & $\Z = \Cset^d$\\[2mm]
\hline
%
Par\'ametros & $m \in \Cset^d, \:\: \Sigma \in P_d^+(\Cset), \:\: \check{\Sigma}
\in S_d(\Cset)$\\[2mm]
\hline
%
Densidad de probabilidad & \\[1mm]
%
Caso general: & $\displaystyle p_Z(z) = \frac{1}{\pi^d \left| \Sigma
 \right|^{\frac12} \left| P \right|^{\frac12}} \: e^{- (z-m)^\dag P^{-1} (z-m) +
 \real{(z-m)^t R^t P^{-1} (z-m)}}$\vspace{2.5mm}\newline
 con~\footnote{En~\cite{Pic96} la expresi\'on es ligieramente diferente, pero se
 recupera usando la simetr\'ia \ $\check{\Sigma}^* =
 \check{\Sigma}^\dag$. Recordar que \ $\cdot^{-*} = \left( \cdot^* \right)^{-1}$
 \ (ver notaciones).} \ $P = \Sigma - \check{\Sigma} \Sigma^{-*}
 \check{\Sigma}^\dag, \quad R = \check{\Sigma}^\dag \Sigma^{-1}$.\\[2.5mm]
%
Caso circular: & $\displaystyle p_Z(z) = \frac{1}{\pi^d \left| \Sigma \right|}
 \: e^{- (z-m)^\dag \Sigma^{-1} (z-m)}$\\[2.5mm]
\hline
%
Promedio & $ m_Z = m$\\[2mm]
\hline
%
Covarianza & $\Sigma_Z = \Sigma$\\[2mm]
\hline
%
Pseudo-covarianza & $\check{\Sigma}_Z = \check{\Sigma}$\\[2mm]
\hline
%%
%Generadora de  momentos &  $\displaystyle M_X(u) =  e^{u^t \Sigma u + u^t m}$  \ para \  $u \in
%\Cset^d$\\[2mm]
%\hline
%%
Funci\'on caracter\'istica & \\[1mm]
%
Caso general: & $\displaystyle \Phi_Z(\omega) = e^{-\frac14
\omega^\dag \Sigma \omega - \frac14 \real{\omega^\dag \check{\Sigma} \omega^*} +
\imath \real{\omega^\dag m}}, \quad \omega \in \Cset^d$\\[1mm]
%
Caso circular: & $\displaystyle \Phi_Z(\omega) = e^{-\frac14
\omega^\dag \Sigma \omega +
\imath \real{\omega^\dag m}}, \quad \omega \in \Cset^d$
\end{caracteristicas}

Notar que  en el caso  escalar propio (circular),  la varianza de  \ $Z$ \  es \
$\sigma_Z^2 =  2 \sigma^2$. El  coefficiente 2  viene del hecho  de que \  $Z$ \
contiene dos componentes independientes de varianza $\sigma^2$.

Los vectores aleatorios complejos van a compartir las propiedades del caso real,
siendo equivalente  a un vector  $2d$-dimensional gausiano real. Primero,  en el
caso circular, se puede escribir $Z \, \egald \, \Sigma^{\frac12} N + m$ \ con \
$N \, \sim \, \C\N(0,I)$ \ donde \ $N$ \ es dicha {\em Gausiana estandar} o {\em
centrada-normalizada}. Las caracter\'isticas  de \ $X$ \ son  v\'inculadas a las
de \ $N$ \ (y vice-versa) por transformaci\'on afine (ver secciones anteriores).

Como  en el  caso  real, la  gausiana  es estable  por  combinaci\'on lineal  de
vectores independientes:
%
\begin{teorema}[Stabilidad]
\label{Teo:MP:StabilidadGaussianaCompleja}
%
  Sean \ $A_i , i = 1,\ldots,n$ \  matrices de \ $\Cset^{d' \times d}, d' \le d$ \
  de   rango  lleno,   $b_i   \in  \Cset^{d'}$   \   y  \   $Z_i   \,  \sim   \,
  \C\N(m_i,\Sigma_i,\check{\Sigma}_i)$ \ independientes, entonces
  %
  \[
  \sum_{i=1}^n \left( A_i  Z_i + b_i \right) \,  \sim \, \C\N\left( \sum_{i=1}^n
    \left( m_i + b_i \right) \, ,  \, \sum_{i=1}^n A_i \Sigma_i A_i^\dag \, , \,
    \sum_{i=1}^n A_i \check{\Sigma}_i A_i^t \right)
  \]
  % En  particular, cualquier  combinaci\'on  lineal de  los  componentes de  un
  vector  gaussiano  complejo da  una  gaussiana  compleja.  Reciprocamente,  si
  cualquier combinaci\'on lineal de los componentes de un vector aleatorio sigue
  una ley gaussiana compleja, entonces el vector es gaussiano complejo.
\end{teorema}

El teorema del l\'imite central y sus variantes se recuperan del caso real.
%
\begin{teorema}[Teorema del l\'imite central (caso complejo)]
\label{Teo:MP:CLTComplejo}
%
  Sea  \  $\{  Z_i \}_{i  \in  \Nset^*}$  \  una  serie de  vectores  aleatorios
  independientes, de misma  ley, y que admiten un promedio \  $m$, una matriz de
  covarianza   \   $\Sigma$   \    y   una   matriz   de   pseudo-covarianza   \
  $\check{\Sigma}$. Entonces
  %
  \[
  \frac{1}{\sqrt{n}}  \sum_{i=1}^m  \left( Z_i  -  m  \right)  \: \limitd{n  \to
    +\infty} \: W \sim \C\N\left( 0 , \Sigma , \check{\Sigma} \right)
  \]
  %
  donde  \ $\limitd{}$ \  significa que  el l\'imite  es en  distribuci\'on (ver
  notaciones).
\end{teorema}
%
No   lo   presentamos,  pero   se   transpone   sencillamente   el  teorema   de
Lindenberg-Feller~\ref{Teo:MP:LindenbergFeller} al caso complejo.

Al final, v\'imos en la  secci\'on~\ref{Ssec:MP:VAComplejos} que si un vector es
circular, entonces  su pseudo-covarianza es nula,  pero la reciproca  no vale en
general. Aparece que en el contexto gausiano tenemos la reciproca:
%

\begin{teorema}[Circularidad]\label{Teo:MP:CircularidadGaussiana}
%
Sea \ $Z \, \sim \, \C\N(m,\Sigma,\check{\Sigma})$.  Entonces,
  %
  \[
  Z \: \mbox{ circular  en torno de } \: m \qquad  \Longleftrightarrow \qquad Z \:
\mbox{ propio en torno de } \: m
  \]
\end{teorema}
%
\begin{proof}
  V\'imos     la    directa    en     la    secci\'on~\ref{Ssec:MP:VAComplejos},
  teorema~\ref{Teo:MP:Circularidad}.  Reciprocamente,  si \  $Z$ \ es  propio en
  torno de \ $m$, por definici\'on \ $\check{\Sigma} = 0$ \ y el resultado viene
  de la forma de  la funci\'on caracter\'istica por ejemplo: $\Phi_{Z-m}(\omega)
  = e^{-\frac14 \omega^\dag \Sigma \omega } = \Phi_{Z-m}\left( e^{\imath \theta}
    \omega \right) = \Phi_{e^{\imath \theta} (Z-m)}(\omega)$.
\end{proof}
