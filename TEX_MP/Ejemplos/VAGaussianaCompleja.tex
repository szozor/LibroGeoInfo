\subsubseccion{Distribuci\'on normal o Gaussiana multivariada complejas}
\label{Sssec:MP:GaussianaComplejas}

Por definici\'on, un vector aleatorio complejo $d$-dimensional \ $Z = X + \imath
Y$ \  es gausiano significa que  el vector $2 d$-dimensional  \ $\widetilde{Z} =
\begin{bmatrix}  X^t &  Y^t \end{bmatrix}^t$  \ es  gausiano. Se  puede entonces
referirse  en el  caso de  vectores gausianos,  pero como  lo presentamos  en la
secci\'on~\ref{Sec:MP:VectoresComplejosMatricesAleatorias},   es  frecuentemente
m\'as comodo trabajar con \ $Z$  \ en lugar de \ $\widetilde{Z}$. En particular,
en  el marco de  las comunicaciones  en ingeneria,  se trabaja  con modulaciones
dichas en fase y en cuadratura (se\~nal multiplicado respectivamente por un seno
y  un coseno)  y  en  lugar de  trabajar  con dos  componente  se considera  una
modulaci\'on  con una exponencial  compleja y  la se\~nal/variable  compleja. Se
puede por ejemplo referirse a~\cite{Lap17} (ver en particular el capitulo~24).

\SZ{
caso general y caso circular}