\subsubseccion{Distribuci\'on uniforme sobre un intervalo}
\label{Sssec:MP:UniformeContinua}

Se denota $X \, \sim \, \U([a \; b])$. Las caracter\'isticas de \ $X$ \ son las
siguientes:

\begin{caracteristicas}
%
Dominio de definici\'on & $\X = [a \; b]$\\[2mm]
\hline
%
Parametros & $(a,b) \in \Rset, \: b > a$\\[2mm]
\hline
%
Densidad de probabilidad & $p_X(x) = \frac{1}{b-a}$\\[2mm]
\hline
%
Promedio & $\displaystyle m_X = \frac{a+b}{2}$\\[2mm]
\hline
%
Varianza & $\displaystyle \sigma_X^2 = \frac{(b-a)^2}{12}$\\[2mm]
\hline
%
\modif{Sesgo} & $\gamma_X = 0$\\[2mm]
\hline
%
Curtosis por exceso & $\displaystyle \widebar{\kappa}_X = -\frac65$\\[2mm]
\hline
%
Generadora de momentos & $\displaystyle M_X(u) = \frac{ e^{b u} - e^{a u}}{u}$ \
para~\footnote{En el caso l\'imite \ $u \to  0$, \ $\lim_{u \to 0} \frac{ e^{b u}
- e^{a u}}{u} = b-a$, y similarmente para la funci\'on caracter\'istica}  \ $u \in \Cset^d$\\[2mm]
\hline
%
Funci\'on caracter\'istica & $\displaystyle  \Phi_X(\omega) = \frac{ e^{\imath a
\omega} - e^{\imath b \omega}}{\omega}$
\end{caracteristicas}

% Momentos & $ \Esp\left[ X^k \right] = p^k$\\[2mm]
% Momento factorial & $\Esp\left[ (X)_k \right] = ?$\\[2mm]
% Generadora de probabilidad & $G_X(z) = e^{\lambda (z-1)}$ \ para \ $z \in \Cset$\\[2mm]
% modo 0
% Mediana \ln(2)/\lambda
% CDF 1-e^{-\lambda x}

Obviamente, se puede escribir \ $X \, \egald  \, a + (b-a) U$ \ donde \ $\egald$
\ significa que la equalidad es en distribuci\'on (las variables tienen la misma
distribuci\'on de probabilidad), con \ \ $U \, \sim \, \U \left( [ 0 \; 1 ]
\right)$ \ llamada {\em uniforme estandar}.

La densidad de probabilidad y funci\'on de repartici\'on de la variable estandar
son representadas en la figura Fig.~\ref{Fig:MP:Uniformecontinua}.
%
\begin{figure}[h!]
\begin{center} \input{TIKZ_MP/UniformeContinua} \end{center}
% 
\leyenda{Ilustraci\'on  de  una densidad  de  probabilidad  uniforme  (a), y  la
funci\'on de repartici\'on asociada (b).}
\label{Fig:MP:Uniformecontinua}
\end{figure}

De manera  general, para  cualquier ensemble $\D  \subset \Rset^d$ de  volumen \
$|\D|$ \,  la variable uniforma sobre $\D$  tiene la densidad con  respecto a la
medida  ``natural'' sobre  $\D$  (Lebesque, discreta,\ldots)  constante sobre  \
$\D$,
%
\[
p_X(x) = \frac{1}{|\D|} \un_{\D}(x)
\]
%
La media va a ser el centro de gravedad de $\D$.

%Cuando $\lambda \to +\infty$ la variable tiende a una variable cierta $X = 0$.
\SZ{Esta distribuci\'on aparece..., propiedades}
% en  el conteo  de conteo  de  une repetici\'on  de una  experiencia de  maneja
% independiente hasta que  occure un evento de probabilidad  $p$; por ejemplo el
% n\'umero de tiro de un dado  equilibriado hasta que occurre un ``6'' sigue una
% ley geometrica de parametro $p = \frac16$.