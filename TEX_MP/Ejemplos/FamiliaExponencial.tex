% --------------------------------- Familia  exponencial
\subseccion{Familia exponencial}
\label{Ssec:MP:FamiliaExponencial}

% Familia exponencial:
% --------------------
% Kay 1993, p. 110 & 124
% Lehman & Casela 1998, p. 23
% Kotz & Balakrishnan 2000, p. 659
% Robert 2007, p. 115
% van den Bos 2007, p. 30
% Cencov 1982, p. 245, 279
% Ibarola Perez 2012, p. 174
% Mukhopadhyay 2000, p. 141


% Estadistica suficiente:
% -----------------------
% Kay 1993, p. 102-103
% Lehman & Casela 1998, p. 32-44
% Robert 2007, p. 14
% Cencov 1982, p. 28
% Ibarola Perez 2012, p. 163
% Mukhopadhyay 2000, p. 284

Muchas de las leyes que hemos visto, que sean discreta o continuas, partenecen a
una clase que comparte propriedades  particulares, y que juega un rol particular
que sea en f\'isica (ej. en  problema de maximizaci\'on de entrop\'ia de Shannon
como lo vamos a ver en  el c\'apitulo~\ref{Cap:SZ:Informacion}) o en el marco de
la inferencia bayesiana. Esta clase es la familia dicha exponencial~\cite{Dar35,
  Koo36, And70, LehCas98, IbPer12,  Muk00, KotBal00, Rob07, Bos07, Cen82, Kay93,
  NieNoc10}.  En inferencia bayesiana,  cuando una distribuci\'on de sampleo cae
en esta  familia, se  puede por  ejemplo deducir a  priori conjugados,  es decir
tales que  la distribuci\'on dicha a  posteriori caiga en la  misma familia (\ie
tenga la  misma forma  param\'etrica pero con  par\'ametros dependientes  de los
datos)~\footnote{Ver nota de pie~\footref{Foot:MP:BayesPrior}. Recordamos que si
  observaciones tienen  una distribuci\'on \ $p_{X|\Theta=\theta}(x)$  \ con una
  distribuci\'on a  priori del par\'ametro  $p_\Theta(\theta)$, por la  regla de
  Bayes el a  posteriori, es decir la ley del  parametro dados las observaciones
  es   dada  por   \   $p_{\Theta|X=x}(\theta)  \propto   p_{X|\Theta=\theta}(x)
  p_\Theta(\theta)$  \ con  \ $\propto$  \ significando  ``proporcional  a''. Si
  $p_{\Theta|X=x}$ \ tiene la misma forma param\'etrica que \ $p_\Theta(\theta)$
  \ inferir  \ $\theta$ \ con  datos que se  observan se reduce a  actualizar el
  par\'ametro de la  ley a posteriori.}. Notar que esta familia  es conocido o a
veces  {\em   familia  de  Koopman-Darmois}  debido  a   la  introducci\'on  por
Koopman~\cite{Koo36}  o  Darmois~\cite{Dar35}   en  los  a\~nos  1935-1936  (ver
tambi\'en Pitman~\cite{Pit36}) y es definida de la manera siguiente:
%
\begin{definicion}[Familia exponencial y exponencial natural]\label{Def:MP:FamiliaExponencial}
%
  Sea  \ $X$  \  vector aleatorio  definido  sobre \  $\X  \subset \Rset^d$,  de
  densidad  de probabilidad  \  $p_X$ \  \ con  respecto  a una  medida \  $\mu$
  (discreta, continua o  mixta). La distribuci\'on de probabilidad  \ $p_X$ \ es
  dicha de  la {\em familia  exponencial de orden  $k$}, $k \in \Nset^*$,  si se
  escribe de la forma
  %
  \[
  p_X(x) = C(\theta) \, h(x) \, \exp\left( \eta(\theta)^t S(x) \right)
  \]
  %
  donde
  %
  \[
  C:  \Theta \subset  \Rset^m \mapsto  \Rset_+,  \qquad h:  \X \mapsto  \Rset_+,
  \qquad \eta: \Theta \mapsto \Rset^k, \qquad S: \X \mapsto \Rset^k
  \]
  %
  con \ $m \in \Nset$.  En otras palabras, la familia exponencial es una familia
  parametrica  de la  forma as\'i  definida, con  $\Theta$ \  el espacio  de los
  par\'ametros. La familia es dicha {\em exponencial natural} si tiene la forma
  %
  \[
  p_X(x) = \frac{1}{Z(\eta)}  \, h(x) \, \exp\left( \eta^t  S(x) \right) = h(x)
  \, \exp\left( \eta^t S(x) - \varphi(\eta) \right)
  \]
  %
  con
  %
  \[
  \eta \in \mathrm{N} \subset \Rset^k, \qquad Z: \Rset^k \mapsto \Rset_+, \qquad
  \varphi = \log(Z), \qquad h: \X \mapsto \Rset_+, \qquad S: \X \mapsto \Rset^k
  \]
  %
  donde  \ $\displaystyle  \mathrm{N} =  \left\{  \eta \in  \Rset^k \tq  \int_\X h(x)  \,
    \exp\left(  \eta^t  S(x) -  \varphi(\eta)  \right)  \,  d\mu(x) <  +  \infty
  \right\}$ \ es (convexo y) llamado {\em espacio de par\'ametros naturales}.
\end{definicion}
%
Se notara  que con  la reparametrizaci\'on \  $\eta(\theta)$ \ se  puede siempre
(por lo  menos formalmente) escribir una  ley de la familia  exponencial bajo su
forma natural. Con respeto a cada termino:
%
\begin{itemize}
\item  $S(X)$ es  llamada {\em  estad\'istica suficiente}  o  {\em estad\'istica
    exaustiva}. Esta  denominaci\'on viene  del hecho que  el conocimiento  de \
  $S(X)$  \  es  suficiente para  estimar  $\eta$,  o  un resumen  exaustivo  de
  $\eta$. En particular, la estimaci\'on de verosimilitud~\footnote{El estimador
    del m\'aximo  de verosilitud \ $\eta_{\mathrm{mv}}$  \ es el \  $\eta$ \ que
    m\'aximiza \ $p_X$  \ o cualquier funci\'on creciente  como el logaritmo por
    ejemplo.    Para   la   familia   exponencial,  eso   da   sencillamente   \
    $\eta_{\mathrm{mv}}$  \  satisficiendo  \  $S(x)  =  \nabla  \varphi(\eta)$,
    dependiente solamente de  $S(x)$.\label{Foot:MP:MLE}}, o cualquier estimador
  Bayesiano~\footnote{Ver nota  de pie~\footref{Foot:SZ:BayesPrior}. Recuerdense
    que se modeliza  el par\'ametro como aleatorio, as\'i  que la distribuci\'on
    que se considera se ve como la distribuci\'on condicional \ $p_{X|N=\eta}(x)
    =   h(x)  \exp\left(   \eta^t  S(x)   -  \varphi(\eta)   \right)$,   con  la
    distribuci\'on  a  priori   \  $p_N(\eta)$.   De  la  regla   de  Bayes,  la
    distribuci\'on a  posteriori, \ie del par\'ametro dadas  las observaciones \
    $x$ \  se escribe \ $p_{N|X=x}(\eta) \propto  p_{X|N=\eta}(x) p_N(\eta)$. Si
    se da un  costo de estimaci\'on \ $C(\widehat{\eta},N)  > 0$ \ caracerizando
    la  ``distancia'' entre  la estimaci\'on  y el  parametro  ``verdadero'', se
    busca la funci\'on  $\widehat{\eta}$ que minimiza el costo  promedio, lo que
    es equivalente para cada \ $x$ \  a buscar el valor \ $\widehat{\eta}$ \ que
    minimiza \ $\displaystyle \int_N C(\widehat{\eta},\eta) \, \exp\left( \eta^t
      S(x)  - \varphi(\eta)  \right) \,  p_N(\eta)  \, d\mu(\eta)$~\cite{Rob07}:
    claramente,  el m\'inimo depende  solamente de  $S(x)$.}  de  $\eta$ depende
  solamente  de $S(X)$~\cite{Kay93,  LehCas98, Rob07,  Cen82,  IbaPer12, Muk00}.
  Formalmente, una  estadistica \ $T(X)$ \  es suficiente para  un par\'ametro \
  $\theta$ \ si la  distribuci\'on de \ $X$ \ condicionalmente a  \ $S(X) = s$ \
  no depende m\'as de \ $\theta$.   Por ejemplo, para la familia exponencial, en
  el caso discreto, tenemos,\ $\displaystyle p_{X|S(X) = s}(x) = \frac{P\big( (X
    = x) \cap (S(X) = s) \big)}{P(S(X)  = s)} = \frac{h(x) \exp\left( \eta^t s -
      \varphi(\eta)  \right) \un_{\{  S(x) \}}(s)}{\displaystyle  \sum_{x\in \X}
    h(x)  \exp\left( \eta^t  s -  \varphi(\eta) \right)  \un_{\{ S(x)  \}}(s)} =
  \frac{h(x)  \un_{\{ S(x)  \}}(s)}{\displaystyle \sum_{x  \in \X}  h(x) \un_{\{
      S(x)   \}}(s)}$   \   no   depende   de   \   $\eta$.    Veremos   en   el
  capitulo~\ref{Cap:SZ:Informacion},   secci\'on~\ref{Ssec:SZ:Fisher}   que   la
  informaci\'on de  Fisher en $\eta$,  medida informacional y apareciendo  en la
  cota del error cuadratico m\'inimo posible  de un estimador de \ $\eta$, es la
  covarianza  de  $S$,  mostrando  de  nuevo  que  \  $S$  \  es  sufiente  para
  caracterizar      \      $\eta$.        En      este      mismo      capitulo,
  secci\'on~\ref{Ssec:SZ:MaxEnt},  veremos  que,  sujeto  a \  $\Esp[S(X)]$,  la
  distribuci\'on  que m\'aximiza  la  entrop\'ia, medida  de  incerteza, es  una
  distribuci\'on  exponencial, enfatizando el  rol de  esta familia  en f\'isica
  cuando se tiene la media de una estadistica fija (ej. energ\'ia fija).
%
\item $\theta$  es el par\'ametro, escalar  o multivariado, y  $\eta$ es llamado
  {\em par\'ametro natural}.
%
\item  La funci\'on  $Z$ es  llamada {\em  funci\'on de  partici\'on};  A veces,
  $\varphi$ es as\'i llamada {\em  log-funci\'on de partici\'on}; no solo juegan
  un  rol en  la  normalizaci\'on de  la  ley, pero  tienen una  significaci\'on
  f\'isica  como lo vamos  a evocar.  Aparecio $Z$  en f\'isica  esdatistica por
  ejemplo en trabajos de Gibbs~\cite{Gib01, Gib02}.
\end{itemize}

Se notara que,  en particular, \ $Z$ \  es relacionada a los momentos  de $S$, o
$\varphi$ a los cumulantes:
%
\begin{teorema}[Funci\'on de partici\'on y generadoras]
%
  Sea \  $X$ \  de distribuci\'on exponencial  natural de par\'ametro  natural \
  $\eta$ \ y estad\'istica suficiente \ $S$  \ y denotamos \ $Z$ la funci\'on de
  partici\'on y  $\varphi$ su logaritmo.  Entonces,  las funci\'ones generadoras
  de los  momentos y de  los cumulantes \  $M_{S(X)}$ \ y  \ $C_{S(X)}$ \  de la
  estad\'istica  suficiente \  $S(X)$ \  son relacionadas  a  \ $Z$  \ y  \ a  \
  $\varphi = \log  Z$ \ por, \ $\forall \:  u \: \mbox{ tal que  } \: u+\eta \in
  N$,
  %
  \[
  M_{S(X)}(u) =  \frac{Z(u+\eta)}{Z(\eta)} \qquad \mbox{y}  \qquad C_{S(X)}(u) =
  \varphi(u+\eta) - \varphi(\eta)
  \]
  %
  En particular, si \  $\varphi$ \ es diferenciable tenemos
  %
  \[
  \nabla \varphi (\eta) = \Esp\left[ S(X) \right]
  \]
  %
  y si \ $\varphi$ \ es dos veces diferenciable tenemos
  %
  \[
  \Hess \varphi (\eta) = \Cov\left[ S(X) \right]
  \]
  %
  Pasando, de  \ $\Cov\left[ S(X)  \right] \ge 0$  \ tenemos que, hessiana  de \
  $\varphi$ \ siendo positiva, \ $\varphi$ \ es convexa~\cite{CamMar09}.
%
\end{teorema}
%
\begin{proof}
  De la definici\'on de la funci\'on generadora de \ $S(X)$, tenemos
  %
  \begin{eqnarray*}
  M_{S(X)}(u) & = &\Esp\left[ \exp\left( u^t S(X) \right) \right]\\[2mm]
  %
  & = & \int_\X \frac{1}{Z(\eta)} \, h(x) \, \exp\left( (u + \eta)^t S(x) \right) \,
  d\mu(x)\\[2mm]
  %
  & = & \frac{Z(u+\eta)}{Z(\eta)} \int_\X \frac{1}{Z(\eta)} \, h(x) \,
  \exp\left( (u + \eta)^t S(x) \right) \, d\mu(x)\\[2mm]
  %
  & = &  \frac{Z(u+\eta)}{Z(\eta)}
  \end{eqnarray*}
  %
  La secunda  relaci\'on es inmediata  de \ $C_X  = \log M_X$ \  conjuntamente a
  \ $\varphi = \log Z$.

  A continuaci\'on, los cumulantes y momentos coinciden hasta el orden $3$, dando inmediatamenta
  %
  \begin{eqnarray*}
  \Esp\left[ S(X) \right] & = & \left. \nabla_u C_{S(X)} \right|_{u=0}\\[2mm]
  %
  & = & \left. \nabla_u \left( \varphi(u+\eta) - \varphi(\eta) \right) \right|_{u=0}\\[2mm]
  %
  & = & \nabla \varphi(\eta)
  \end{eqnarray*}
  %
  y
  %
  \begin{eqnarray*}
  \Cov\left[ S(X) \right] & = & \left. \Hess_u C_{S(X)} \right|_{u=0}\\[2mm]
  %
  & = & \left. \Hess_u \left( \varphi(u+\eta) - \varphi(\eta)  \right) \right|_{u=0}\\[2mm]
  %
  & = & \Hess \varphi(\eta)
  \end{eqnarray*}
\end{proof}

En problemas  de estimaci\'on, frecuentemente, se tiene  $n$ vectores aleatorios
independiente  de misma  ley que  se  usan para  estimar un  p\'arametro. En  la
familia exponencial, resuelte que  la distribuci\'on conjunta en tal situaci\'on
queda en la familia exponencial:
%
\begin{lema}
  Sean \  $X_1 , \ldots ,  X_n$ \ vectores aleatorios,  independientes, de misma
  ley de la familia exponencial natural,  de estadistica suficiente \ $S$ \ y de
  log-funci\'on de  partici\'on \ $\varphi$. Entonces  la ley conjunta  de los \
  $X_i$ \ cae  el la familia exponencial  de mismo orden que el  de \ $p_{X_i}$,
  con  el   mismo  par\'ametro,  de  estadistica   suficiente  \  $\displaystyle
  (X_1,\ldots,X_n)  \mapsto  \sum_{i=1}^n  S(X_i)$   \  y  de  log-funci\'on  de
  partici\'on \ $n \varphi$.
\end{lema}
%
\begin{proof}
  El resulta  es inmediato siendos  los \ $X_i$  \ independientes, dando  la ley
  conjunta como el producto de las leyes de los $X_i$.
\end{proof}

Muchas  distribuciones caen  en  la  familia exponencial,  que  sean discreta  o
continua, como lo vamos a ver en dos ejemplos.
%
\begin{ejemplo}
  Sea  \ $p_X$  \ distribuci\'on  de Bernoulli  \ $\B(p)$.   Esta distribuci\'on
  partenece a la familia exponencial de  orden $1$.  De hecho, se puede escribir
  la ley \ $p_X(x) = p^x (1-p)^{1-x}$ \ bajo la forma
  %
  \[
  p_X(x) = \exp\left( x \log \left( \frac{p}{1-p} \right) + \log(1-p) \right)
  \]
  %
  Aparece  que el par\'ametro  natural es  \ $\eta  = \log  \left( \frac{p}{1-p}
  \right)$ \  y la estadistica  suficiente correspondiente es  \ $S(X) =  X$. En
  particular, para \  $X_i, \: i = 1,  \ldots , n$ \ independientes  tales que \
  $X_i  \sim  \B(p)$, una  estadistica  sufficiente de  la  ley  conjunta es  el
  promedio empirico $\displaystyle  \widebar{X} = \frac{1}{n} \sum_{i=1}^n X_i$.
  Aparece  que  el estimador  de  verosimilitud  m\'axima~\footnote{Ver nota  de
    pie~\footref{Foot:MP:MLE}}  es  precisamente  el  promedio empirico;  es  el
  estimador  \  $\widehat{p}$  \   de  error  cuadratico  \  $\Esp\left[  \left(
      \widehat{p} - p \right)^2 \right]$ \ m\'inimo (ver ej.~\cite{Kay93}).
\end{ejemplo}

\begin{ejemplo}
  Sea \ $p_X$ \ distribuci\'on gamma \ $\G(a,b)$.  Esta distribuci\'on partenece
  a la familia exponencial  de orden $2$.  De hecho, se puede  escribir la ley \
  bajo la forma
  %
  \[
  p_X(x)     =    \frac{1}{x}    \,     \exp\left(    \begin{bmatrix}     a    &
      b \end{bmatrix} \begin{bmatrix}  \log x \\ - x  \end{bmatrix} - \log\left(
      \frac{\Gamma(a)}{b^a} \right)\right)
  \]
  %
  El   par\'ametro  natural   es   as\'i   \  $\eta   =   \begin{bmatrix}  a   &
    b \end{bmatrix}^t$ \ y la  estadistica suficiente correspondiente es \ $S(X)
  = \begin{bmatrix} \log x & - x \end{bmatrix}^t$.
\end{ejemplo}

Si muchas distribuciones  partenecen a la familia exponencial,  no todas caen en
esta familia:
%
\begin{ejemplo}
  Sea \ $p_X$ \ distribuci\'on  Student-t \ $\T_\nu(m,\Sigma)$. Esta distribuci\'on no
  cae el  la familia exponencial. De  hecho, se puede todav\'ia  escribir la ley
  bajo la forma
  %
  \[
  p_X(x)   =  C(\nu,\Sigma)   \exp\left(  -   \frac{d+\nu}{2}  \log\left(   1  +
      \frac{(x-m)^t \Sigma^{-1} (x-m)}{\nu} \right) \right)
  \]
  %
  A    pesar   de    que    la   ley    parece    tener   la    forma   de    la
  definici\'on~\ref{Def:MP:FamiliaExponencial}    con   \    $\eta(\nu)    =   -
  \frac{d+\nu}{2}$,  su  factor  \  $\log\left( 1  +  \frac{(x-m)^t  \Sigma^{-1}
      (x-m)}{\nu} \right)$  \ no es funci\'on  \'unicamente de \  $x$; se quedan
  todo los par\'ametros \ $\nu, \: m, \: \Sigma$. A\'un si un de esos es fijo, y
  no visto como par\'ametro, quedar\'a un parametro en este termino.
\end{ejemplo}

Se puede tambi\'en que una ley  partenece o no a la familia exponencial, seg\'un
que unos par\'ametros sean fijos  (de hecho, no on par\'ametros m\'as entonces),
o no:
%
\begin{ejemplo}
  Sea \  $p_X$ \ distribuci\'on  binomial \  $\B(n,p)$. Si \  $n$ \ es  fijo, es
  decir  no  visto  como  par\'ametro,  la  distribuci\'on  cae  el  la  familia
  exponencial  de orden $1$.  Si se  consideran ambos  \ $n$  \ y  \ $p$  \ como
  par\'ametros,  no  cea m\'as  en  la familia  exponencial.  Para  ver eso,  se
  escribir la ley bajo la forma
  %
  \[
  %begin{eqnarray*}
  p_X(x) = \frac{n!}{x! (n-x)!}  \exp\left( x \log\left( \frac{p}{1-p}
    \right) + n \, \log(1-p) \right)
  %\\[2mm]
  %
  %& = & \frac{1}{x!}  \exp\left( - \log\left( (n-x)! \right) + x \log\left(
  %\frac{p}{1-p} \right) + n \, \log(1-p) + \log(n!) \right)
  %\end{eqnarray*}
  \]
  %
  Entonces, si  \ $n$  \ es fijo,  se concluye que  \ $p_X$  \ es en  la familia
  exponencial de  orden $1$, de  par\'ametro \ $\eta =  \log\left( \frac{p}{1-p}
  \right)$  \  y estad\'istica  suficiente  correspondiente  \  $S(x) =  x$.  Al
  rev\'es,  si \  $n$ \  es un  par\'ametro  (que \  $p$ \  sea fijo  o no),  en
  $(n-x)!$, no se puede  ``separar'' \ $x$ \ de \ $n$  \ y tampoco escribir este
  termino  de la  forma \  $\exp(  f(n) g(x)  )$: la  ley  no es  de la  familia
  exponencial mas.
\end{ejemplo}

De las distribuciones que hemos visto:
%
\begin{itemize}
\item Caen en  la familia exponencial la leyes: de  Bernoulli, binomial cuando \
  $n$ \ es fijo, negative binomial cuando  \ $r$ \ es fijo, multinomial cuando \
  $n$ \  es fijo,  geometrica, de poisson,  gausiana~\footnote{En este  caso, se
    puede ver  el par\'ametro  natural como \  $\left( \Sigma^{-1} m  , -\frac12
      \Sigma^{-1}  \right)$ \  en  lugar del  vector  formado de  los \  $\left(
      \Sigma^{-1} m \right)_i$ \ y \ $-\frac12 \left( \Sigma^{-1} \right)_{i,j},
    \: 1 \le i \le j \le d$ \ y la estad\'istica suficiente como \ $\left( x , x
      x^t \right)$  \ siendo  $x^t \Sigma^{-1} x  = \Tr\left( \Sigma^{-1}  x x^t
    \right)$, en lugar  del vector formado de los  \ $x_i$ \ y \ $x_i  x_j, \: 1
    \le  i  \le  j \le  d$  (de  la  simetr\'ia).   El  orden es  $d  +  \frac{d
      (d+1)}{2}$.}, gamma, Wishart~\footnote{De nuevo, en este caso se puede ver
    el  par\'ametro  natural  formalmente  como \  $\left(  \frac{\nu-d-1}{2}  ,
      -\frac12 V^{-1}  \right)$ \ y  la estad\'istica suficiente como  \ $\left(
      \log |x|  , x \right)$.  El orden es  $1 + \frac{d (d+1)}{2}$.},  beta, de
  Dirichlet.
%
% resp. eta = log(p/(1-p)), log(p/(1-p)), log p, log p = [log p_1 ... \log p_k],
%       log(1-p), log lambda, (Sigma^{-1} m , -1/2 Sigma^{-1}), [a b],
%       ( (nu-d-1)/2 , -1/2 V^{-1}), [a b], a = [a_1 ... a_k]
% resp. S = x, x, x, x = [x_1 ... x_k], x,
%       x, (x,x x^t), [log x -x], (log |x| , x),
%       [log x  log(1-x)], [log x_1 ... log(x_k)]
%
\item No partenecen a la familia  exponencial las leyes: binomial cuando \ $n$ \
  es  un  par\'ametro, negative  binomial  cuando \  $r$  \  es un  par\'ametro,
  multinomial cuando \ $n$ \ es un par\'ametro, hipergeometrica, hipergeometrica
  negativa,     hipergeometrica      multivaluada~\footnote{En     los     casos
    hipergeometricos, har\'ia falta que sean  fijos respectivamente \ $n, m, k$,
    \ $n,  r, k$  \ y  \ $n,  m, k_1, \ldots  , k_c$  \ y  la leyes  no ser\'ian
    parametricas  mas.}, Student-t, uniformas~\footnote{Eso  viene del  hecho de
    que el soporte depende de los par\'ametros.}.
\end{itemize}

Las distribuciones exponenciales aparecen frecuentemente en f\'isica estadistica
a  trav\'es de  la teor\'ia  de Boltzmann~\cite{Bol96,  Bol98,  Gib02, LanLif80,
  MezMon09,  Mer10,  Mer18}.  Adem\'as,  cuantidad  f\'isica  se  derivan de  la
log-funci\'on partici\'on~\cite{Max67, Gib02, LanLif80, MezMon09, Mer10, Mer18}:
%
\begin{ejemplo}
  En f\'isica estadistica, se enfrente al problema de descripci\'on macroscopico
  de un sistema  de muchas particular (ej. hirviente de  un liquido). Hay tantas
  particulas que no se puede estudiar tales sistemas con las leyes usuales de la
  mec\'anica, as\'i que se usa  un enfoque probabilistico. Por eso, se considera
  un espacio  \ $\X$  \ $d$-dimensional dicho  {\em espacio  de configuraciones}
  (puede ser  discreto o  continuo). En \  $x =  \begin{bmatrix} x_1 &  \cdots &
    x_d\end{bmatrix}$,  cada  \ $x_i$  \  representa el  {\em  estado}  de la  \
  $i$-\'esima particula (posici\'on,  velocidad, esp\'in,\ldots).  Lo importante
  es que a un  tipo de sistema se asocia una funci\'on  energ\'ia \ $\E(x)$. Por
  ejemplo, en un sistema  sin interacciones, $\displaystyle \E(x) = \sum_{i=1}^d
  \E_i(x_i)$.  El  el caso  del gaz perfecto,  $\displaystyle \E(x) =  \frac12 m
  \sum_{i=1}^d x_i^2$ \ donde  \ $m$ \ es la masa de cada  particula y \ $x_i$ \
  la  velocidad  de  la  \  $i$-\'esima particula  (espacio  de  configuraciones
  continuo).   En  el  {\em   modelo  ferromagnetico  de  Ising},  se  considera
  particulas en una ret\'icula y \ $x_i = \pm 1$ \ es el esp\'in de la particula
  \ $i$ (espacio de configuraciones  discreto).  Sometido a un campo magnetico \
  $B$,  la energ\'ia  es dada  por  \ $\displaystyle  \E(x) =  - \sum_{(i,j)  \:
    \mbox{\tiny  vecinos}} x_i  x_j  - B  \sum_{i=1}^d x_i$~\cite{Len20,  Isi25,
    Ons44, LanLif80, MezMon09, Mer10, Mer18}.   Se puede poner pesos \ $J_{i,j}$
  \ en cada vecinos,  positivos para interracciones ferromagneticos, y negativos
  para  interacciones antiferromagneticos  (modelos {\em  vidrio de  esp\'in}, o
  m\'as  exactamente  de  Edwards-Anderson~\cite{EdwAnd75,  LanLif80,  MezMon09,
    Mer10,  Mer18}).  El  modelo de  Curie-Weiss~\footnote{Fue llamado  as\'i en
    relaci\'on a los trabajos  de P.  Curie~\cite{Cur95} y P. Weiss~\cite{Wei96,
      Wei07}  sobre los materiales  ferromagneticos.}  se  presenta de  la misma
  manera, con la energ\'ia $\displaystyle \E(x) = - \frac{1}{d} \sum_{i\ne j) \:
    \mbox{\tiny  pares}} x_i  x_j +  B \sum_{i=1}^d  x_i$~\cite{MezMon09, Mer10,
    Mer18}.

  Seg\'un  la  teoria  de  Gibbs-Boltzmann,  la  dicha  {\em  distribuci\'on  de
    Gibbs-Boltzman}  asociada  a  un  espacio  de configuraci\'on  y  modelo  de
  energ\'ia es dada por
  %
  \[
  p_X(x) = \frac{1}{Z(\beta)}  \exp\left( - \beta \E(x) \right),  \qquad \beta =
  \frac{1}{k_B T}
  \]
  %
  donde $k_B \approx  1.38 \times 10^{-23}$ julio por kelvin  es la constante de
  Boltzmann,  y  \ $T$  \  es la  temperatura  en  kelvin.  Esta  distribuci\'on
  partenece claramente a la familia exponencial natural de par\'ametro \ $\beta$
  \  y de  estad\'istica suficiente  \ $-\E(x)$  (ac\'a, $h  = 1$).  En f\'isica
  estadistica, la  log-funci\'on de partici\'on  aparece en varias  cantidades y
  potenciales f\'isicos:
  %
  \[
  F(\beta) = - \frac{1}{\beta} \log Z(\beta)
  \]
  %
  es  la  {\em  energ\'ia  libre}  o  {\em energ\'ia  libre  de  Helmholtz}  del
  sistema.  Es la  energ\'ia disponible  (o  que se  puede usar)  de un  sistema
  aislado.

  Luego, se define
  %
  \[
  U(\beta) =  \frac{\partial}{\partial \beta} \left( \beta F(\beta)  \right) = -
  \frac{\partial \log Z(\beta)}{\partial \beta} = \Esp\left[ \E(X) \right]
  \]
  %
  donde  \  $X$ \  seria  el  vector aleatorio  de  distribuci\'on  \ $p_X$.   \
  $U(\beta)$ \ es  la {\em energ\'ia interna} del  sistema, promedio estadistico
  de la energ\'ia a trav\'es de todas las configuraciones posibles.

  Se define  tambi\'en una medida de  incerteza llamada {\em  entrop\'ia} o {\em
    entrop\'ia  de  Gibbs}~\cite{Bol77, Bol96,  Bol98,  Gib02, Jay65,  LanLif80,
    MezMon09, Mer10,  Mer18}.  Esta medida  caracteriza las fluctuaciones  de la
  energ\'ia libre~\footnote{La  letra \ $S$ \ se  uso historicamente. Obviamente
    no corresponde a la estadica sufficiente que es ac\'a \ $\E$.},
  %
  \[
  S(\beta) = \beta^2 \frac{\partial}{\partial \beta} F(\beta)
  \]
  %
  Aparece   por   un  lado   que   \  $S(\beta)   =   \log   Z(\beta)  -   \beta
  \frac{\partial}{\partial \beta} \log Z(\beta)$  \ es decir, reconiciendo en el
  primer t\'ermino \ $- \beta F(\beta)$ \ y en el secundo \ $\beta U(\beta)$,
  %
  \[
  F = U - k_B T S
  \]
  %
  conocido como transformada de Legendre de la energ\'ia interna, y consecuencia
  de la  primera ley de  la termodynamica. Aparece tambi\'en  que $\displaystyle
  S(\beta) = \log Z(\beta) + \beta  \Esp[ \E(X) ] = \int_\X \left( \log Z(\beta)
    + \beta \E(x) \right) p_X(x) \, d\mu(x)$ \ es decir
  %
  \[
  S = - \int_\X p_X(x) \, \log p_X(x) \, d\mu(x)
  \]
  %
  Volveremos    en    esta    definici\'on    de    la    entrop\'ia    en    el
  capitulo~\ref{Cap:SZ:Informacion} en un marco m\'as general.
\end{ejemplo}

\SZ{Hablar de estadistica suficiente m\'inima?}
% Ver information geometry
% https://en.wikipedia.org/wiki/Partition_function_(mathematics)

%\SZ{Van den Bos 2007, p. 33, informaci\'on de Fisher}