\subsubseccion{Ley Uniforme sobre un ``intervalo'' de $\Zset$}
\label{Sssec:MP:UniformeDiscreta}

Se denota $X \, \sim \, \U\{ a \; b \}$ \ con $(a,b) \in \Zset^2, \: b \ge
a$.  Las caracter\'isticas de \ $X$ \ son las siguientes:

\begin{caracteristicas}
%
Parametros & $(a,b) \in \Zset^2, \: b \ge a$\\[2mm]
\hline
%
Dominio de definici\'on & $\X = \{ a  \; a+1 \; \ldots \; b \}$\\[2mm]
\hline
%
Distribuci\'on de probabilidad & $p_X(x) = \frac1{b-a+1}$\\[2mm]
\hline
%
Promedio & $\displaystyle m_X = \frac{a+b}{2}$\\[2mm]
\hline
%
Varianza & $\displaystyle \sigma_X^2 = \frac{(b-a) (b-a+2)}{12}$\\[2mm]
\hline
%%
\modif{Sesgo} & $\gamma_X = 0$\\[2mm]
\hline
%
Curtosis por exceso & $\displaystyle \widebar{\kappa}_X = -\frac65 \frac{(b-a)
(b-a+2)+2}{(b-a) (b-a+2)}$\\[2mm]
\hline
%
Generadora de probabilidad & $\displaystyle G_X(z) = \frac{z^a-z^{b+1}}{1-z}$ \
para~\footnote{En el caso l\'imite \ $z \to 1$, \ $\lim_{z \to 1} \frac{ z^a -
z^{b+1}}{1-z} = b+1-a$} \ $z \in \Cset$ \ si $a \ge 0$ \ y \ $\Cset^*$ \ sino\\[2mm]
\hline
%
Generadora de momentos & $\displaystyle M_X(u) = \frac{ e^{a u} - e^{(b+1)
u}}{1-e^u}$ \ para~\footnote{En el caso l\'imite \ $u \to 0$, \ $\lim_{u \to 0}
\frac{ e^{a u} - e^{(b+1) u}}{1-e^u} = b+1-a$, y similarmente para la funci\'on
caracter\'istica.}  \ $u \in \Cset$\\[2mm]
\hline
%
Funci\'on caracter\'istica & $\displaystyle  \Phi_X(\omega) = \frac{ e^{\imath a
\omega} - e^{\imath (b+1) \omega}}{1-e^{\imath \omega}}$
\end{caracteristicas}

% Momentos & $ \Esp\left[ X^k \right] = p^k$\\[2mm]
% Momento factorial & $\Esp\left[ (X)_k \right] = ?$\\[2mm]
% modo 0
% Mediana \ln(2)/\lambda
% CDF 1-e^{-\lambda x}

La distribuci\'on  de masa de probabilidad  y funci\'on de  repartici\'on de una
variable uniforme  \ $\U\{  a \; b  \}$ \ son  representadas en  la figura
Fig.~\ref{Fig:MP:UniformeDiscreta}.
%
\begin{figure}[h!]
\begin{center} \begin{tikzpicture}%[scale=.9]
\shorthandoff{>}
%
\pgfmathsetmacro{\sx}{.75};% x-scaling
\pgfmathsetmacro{\r}{.05};% radius arc non continuity F_X
\pgfmathsetmacro{\n}{6};% n de la uniforme
\pgfmathsetmacro{\m}{\n-1};
%
% masa
\begin{scope}
%
%
\pgfmathsetmacro{\sy}{2.5};% y-scaling 
\draw[>=stealth,->] (-.25,0)--({\sx*(\n+.5)+.25},0) node[right]{\small $x$};
\draw[>=stealth,->] (0,-.1)--(0,{\sy+.25}) node[above]{\small $p_X$};
%
\foreach \k in {1,...,\n} {
\draw[dotted] ({\k*\sx},0)--({\k*\sx},\sy) node[scale=.4]{$\bullet$};
\draw ({\k*\sx},0)--({\k*\sx},-.1) node[below,scale=.7]{$\k$};
}
\draw (0,\sy)--(-.1,\sy) node[left,scale=.7]{$\frac1\n$};
%%
\end{scope}
%
%
% reparticion
\begin{scope}[xshift=8.5cm]
%
\pgfmathsetmacro{\sy}{2.5};% y-scaling 
%
\draw[>=stealth,->] ({-\sx/2-.25},0)--({\sx*(\n+1.5)+.25},0) node[right]{\small $x$};
\draw[>=stealth,->] (0,-.1)--(0,{\sy+.25}) node[above]{\small $F_X$};
%
% cumulativa
\draw[thick] ({-\sx/2},0)--(\sx,0);
\draw ({\sx+\r},\r) arc (90:270:\r);
\draw (0,0)--(0,-.1) node[below,scale=.7]{$0$};
%
\foreach \k in {1,...,\m} {
\draw ({\k*\sx},0)--({\k*\sx},-.1) node[below,scale=.7]{$\k$};
\draw[thick] ({\k*\sx},{\k*\sy/\n}) node[scale=.4]{$\bullet$}--({(\k+1)*\sx},{\k*\sy/\n});
\draw ({(\k+1)*\sx+\r},{\k*\sy/\n+\r}) arc (90:270:\r);
\draw[dotted] ({\k*\sx},{(\k-1)*\sy/\n})--({\k*\sx},{\k*\sy/\n});
}
\draw ({\n*\sx},0)--({\n*\sx},-.1) node[below,scale=.7]{$\n$};
\draw[thick] ({\n*\sx},\sy) node[scale=.4]{$\bullet$}--({(\n+1.5)*\sx},\sy);
\draw[dotted] ({\n*\sx},{(\n-1)*\sy/\n})--({\n*\sx},\sy);
%%
\draw (0,\sy)--(-.1,\sy) node[left,scale=.7]{$1$};
\end{scope}
%
\end{tikzpicture} \end{center}
% 
\leyenda{Ilustraci\'on  de  una densidad  de  probabilidad  uniforme  (a), y  la
  funci\'on  de repartici\'on  asociada (b).  $a =  1,  \: b  = 6$  \ (ej.  dado
  equilibriado).}
\label{Fig:MP:UniformeDiscreta}
\end{figure}

Cuando \ $b = a$, la variable tiende a una variable cierta \ $X = a$.

La  distribuci\'on  uniforme  aparece  por   ejemplo  en  el  tiro  de  un  dado
equilibriado con \ $a = 1, \: b = 6$.
%% en el conteo de
%%conteo de une repetici\'on de  una experiencia de maneja independiente hasta que
%%occure un evento de probabilidad $p$; por ejemplo el n\'umero de tiro de un dado
%%equilibriado hasta que occurre un ``6'' sigue una ley geometrica de parametro $p
%%= \frac16$.
