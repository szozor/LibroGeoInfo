\subsubseccion{Distribuci\'on matriz-variada de Wishart}
\label{Sssec:MP:Wishart}

Este ejemplo es una  generalizaci\'on matriz-variada de la distribuci\'on gamma.
Se puede ver  una matriz como un vector, guardando por  ejemplo sus columnas una
bajo la  precediente.  Sin embargo, tal  distribuci\'on apareciendo naturalmente
en un contexto de estimaci\'on de  matriz de covarianza (ver m\'as adelante), es
m\'as  natural  verla  matriz-variate.    Tal  distrubuci\'on  es  debido  a  J.
Wishart~\cite{Wis28, GupNag99, And03}, y  aparece as\'i naturalmente en problema
de      inferencia     Bayesiana      como      distribuci\'on     a      priori
conjugado~\footref{Foot:MP:BayesPrior} del  par\'ametro $p$ de  la ley gaussiana
multivariada~\cite{Rob07}.

Se denota \  $X \, \sim \, W_d(V,\nu)$  \ donde el dominio de  definici\'on es \
$P_d^+(\Rset)$,  conjunto   matrices  simetricas  definida   positivas,  $V  \in
P_d^+(\Rset)$ par\'ametro de  escala y \ $\nu > d-1$ \  grados de libertad.  Las
caracter\'isticas de la distribuci\'on son las siguientes:

\begin{caracteristicas}
%
Dominio de definici\'on~\footnote{De hecho, se puede considerar que la matriz
aleatoria es equivalent a tener un vector \ $\frac{d (d+1)}{2}$-dimensional; por
la simetria, claramente \ $X$ \ tiene solamente \ $\frac{d (d+1)}{2}$ \
componentes diferentes; adem\'as, se puede probar que cualquier matriz \ $A \in
P_d^+(\Rset)$ \ se descompone bajo la forma \ $A = L L^t$ \ con \ $L$ \
triangular inferior con elementos no nulos sobre su diagonal, llamado
descomposici\'on de Cholesky~\cite{GupNag99, Bha07, Har08, HorJoh13} y
reciprocamente. Eso muestra que \ $A$ \ se define a partir de \ $\frac{d
(d+1)}{2}$ \ ``grados de libertad''.\label{Foot:MP:WishartXtilde}} & $\X =
P_d^+(\Rset), \: d \in \Nset^*$\\[2mm]
\hline
%
Par\'ametros & $V \in P_d^+(\Rset)$ (escala) y \ $\nu > d-1$ \ (grados de
libertad)\\[2mm]
\hline
%
Densidad de probabilidad~\footnote{La densidad de probabilidad corresponde a la
densidad conjunta de los \ $\frac{d (d+1)}{2}$ \ elementos \ $X_{i,j}, \: 1 \le
i \le j \le d$~\cite{Wis28, PedRic91, SulTra96, GupNag99,
And03}.\label{Foot:MP:WishartDensidad}} & $\displaystyle p_X(x) =
\frac{|x|^{\frac{\nu-d-1}{2}} \, e^{-\frac12 \Tr\left( V^{-1}
x\right)}}{2^{\frac{d \nu}{2}} \, |V|^{\frac{\nu}{2}} \, \Gamma_d\left(
\frac{\nu}{2} \right)}$\\[2mm]
\hline
%
Promedio & $\displaystyle m_X = \nu \, V$\\[2mm]
\hline
%
Covarianza & $\displaystyle \Sigma_X = \nu \big( J (V \otimes V) + (V \otimes I)
K (V \otimes I) \big)$\\[2mm]
% $\Cov[X_{i,j},X_{k,l}] = nu \left( V_{i,k} V_{j,l} +V_{i,l} V_{j,k} \right)$}
\hline
%
Funci\'on caracter\'istica~\footnote{\SZ{Se proba que la funci\'on generadora de
momentos no existe en
general}.\label{Foot:MP:CaracteristicaWishart}} &
$\displaystyle \Phi_X(\omega) = \left| I - 2 \imath \omega V
\right|^{-\frac{\nu}{2}}, \quad \omega \in S_d(\Rset)$
\end{caracteristicas}

(ver~\cite{PedRic91, SulTra96, And03}).

Fijense que $p_X$ no es la distribuci\'on conjuntos de los componentes de \ $X$:
el hecho de  que \ $X$ \ sea  uan matriz aleatoria de \  $P_d^+(\Rset)$ \ impone
v\'inculos sobre sus compnentes; entre otros, $X_{i,j} = X_{j,i}$.

Inmediatamente, si  $d = 1$, la  distribuci\'on de Whishart \  $W_1(V,\nu)$ \ se
reduce  a la  distribuci\'on Gamma  $\G\left(\frac{\nu}{2} \,  , \,  \frac1{2 V}
\right)$. De este  hecho, se la podr\'ia ver  como extensi\'on matriz-variada de
la  distribuci\'on  gamma.  La  distribuci\'on  de  Wishart  tiene varias  otras
propiedades como las siguientes.
%
\begin{lema}[Stabilidad por transformaci\'on lineal]
\label{Lem:MP:StabilidadWishartLineal}
%
  Sea $X \,  \sim \, W_d(V,\nu)$ \ y \  $A \in \Rset^{d \times d'}$  \ con \ $d'
  \le d$ \ y de rango lleno. Entonces
  \[
  A^t X A \, \sim \, W_{d'}\left( A^t V A , \nu \right)
  \]
  %
  En particular, si $d'  = 1$, \ $A^t X A \, \sim  \, G\left( \frac{\nu}{2} \, ,
    \, \frac1{2 \, A^t V A} \right)$. M\'as all\'a, tomando $A = \un_j$, aparece
  de que  las componentes diagonales de \  $X$ \ son de  distribuci\'on gamma, \
  $X_{j,j}  \, \sim  \,  \G\left( \frac{\nu}{2}  \,  , \,  \frac1{2 \,  V_{j,j}}
  \right)$.
\end{lema}
%
\begin{proof}
  El     resultado     es     inmediato     saliendo     de     la     funci\'on
  caract\'eristica~\footref{Foot:MP:CaracteristicaWishart} y notando de que
%
\begin{eqnarray*}
\Phi_{A^t X A}(\omega) & = & \Esp\left[ e^{\imath \Tr\left( \omega^t A^t X A
\right)}\right]\\[2mm]
%
& = & \Phi_X\left( A \omega^t A^t\right)\\[2mm]
%
& = &  \left| I - 2 \imath A \omega A^t V \right|^{-\frac{\nu}{2}}\\[2mm]
%
& = &  \left| I - 2 \imath \omega A^t V A \right|^{-\frac{\nu}{2}}
%
\end{eqnarray*}
%
de   \   $\Tr(AB)   =   \Tr(BA)$~\cite{Har08}   \   y   de   la   identidad   de
Sylvester~\cite{Syl51,  AkrAkr96}  o~\cite[\S~18.1]{Har08} \  $\left|  I  + A  B
\right| = \left| I + B A \right|$.  .
\end{proof}
%
De hecho, si los elementos diagonales son de distribuci\'on gamma, no es el caso
de         los        elementos         no-diagonales~\cite{Seb04,        And03}
o~\cite[Teo.~3.3.4]{GupNag99}.    De    eso   resuelte   delicado    llamar   la
distribuci\'on como gamma matriz-variada.

\begin{lema}[Stabilidad por suma]
\label{Lem:MP:StabilidadWishartSuma}
%
  Sea $X_i \,  \sim \, W_d(V,\nu_i), \: i = 1,\ldots,n$ independientes. Entonces
  \[
  \sum_{i=1}^n X_i \, \sim \, W_d\left( V \, , \, \sum_{i=1}^n \nu_i \right)
  \]
\end{lema}
%
\begin{proof}
  El     resultado     es     inmediato     saliendo     de     la     funci\'on
  caract\'eristica~\footref{Foot:MP:CaracteristicaWishart} y notando que como el
  el context vectorial $\Phi_{\sum_i X_i} = \prod_i \Phi_{X_i}$.
\end{proof}

La distribuci\'on  de Wishart aparece naturalmente en  problemas de estimaci\'on
de matriz de covarianza en el contexto gausiano:
%
\begin{lema}[V\'inculo con vectores gausianos~\cite{Seb04}]
  Sean \ $X_i \, \sim \, \N(0,V), \: i = 1, \ldots , n > d-1$ \ independientes y
  la  matriz  \  $S  =  \sum_{i=1}^n   X_i  X_i^t$  \  llamada  {\em  matriz  de
    dispersi\'on} (scatter matrix en ingl\'es). Entonces, \ $S \in P_d^+(\Rset)$
  (c.   s.)   \  ($S$ es  sim\'etrica  definida  positiva  casi siempre,  o  con
  probabilidad uno) y \ $S \,\sim \, W_d(V,n)$.
\end{lema}
%
Este resultado  permite tambi\'en probar  el lema~\ref{Lem:MP:StabilidadWishart}
para \ $\nu = n$ \ entero  escribiendo \ $X \egald \sum_{i=1}^n X_i X_i^t$ \ tal
que \ $A^t X A \egald \sum_{i=1}^n A^t X_i X_i^t A = \frac1n \sum_{i=1}^n \left(
  A^t X_i \right)  \left( A^t X_i \right)^t$ \  y notando que los \  $A^t X_i \,
\sim \,  \N(0, A^t V  A)$ \ son independientes~\cite{Seb04}.   Adem\'as, permite
re-obtener las  expreciones del promedio y de  las covarianzas~\footnote{Para la
  covarianza, su usa la formula \ $\Esp[Y_1 Y_2 Y_3 Y_4] = \Esp[Y_1 Y_2]\Esp[Y_3
  Y_4]  + \Esp[Y_1 Y_3]\Esp[Y_2  Y_4] +  \Esp[Y_1 Y_4]\Esp[Y_2  Y_3]$ \  para $Y
  = \begin{bmatrix}  Y_1 & Y_2 &  Y_3 & Y_4 \end{bmatrix}^t$  \ vector gausiano,
  formula que se  obtiene por ejemplo a partir  de la funci\'on caracter\'istica
  de un vecor  gausiano.}. Notar que cuando los \ $X_i$  \ tienen un promemedio,
el  lema conduce  a  lo que  es  conocido como  Wishart no  central~\cite{And03,
  Seb04}.

\SZ{Que propedad mas? Ver Gupta Nagar 1999}

\SZ{Y donde aparece mas?}