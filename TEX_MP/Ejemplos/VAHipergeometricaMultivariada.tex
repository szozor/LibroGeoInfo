\subsubseccion{Ley hipergeometrica multivariada}
\label{Sssec:MP:HipergeometricaMultivariada}

Esta ley aparece por ejemplo cuando  se se generaliza la ley hipergeometrica con
$c > 2$ clases \ con \ $K_i$ \ estados en la clase $i$, $\sum_i K_i = N$.

Se denota \ $X \, \sim \, \H\M(N,K,n)$ \ con \ $\displaystyle N \in \Nset, \quad
K = (K_1  \, , \, \ldots  \, , \, K_c) \in  \left\{ \{ 0 \; \ldots  \; N\}^c \tq
  \sum_{i=1}^c K_i = N \right\}, \quad n \in \{ 0 \; \ldots \; N\}$.

Entonces, como en el caso de la ley multinomial, a pesar de que se escribe \ $X$
\ de manera $c$-dimensional, el vector  partenece a una variedad claramente \ $d
=  c-1$  \  dimensional  y  en  el  caso  \  $c  =  2$  \  se  recupera  la  ley
hipergeometrica.

Sus caracter\'isticas son las siguientes:

\begin{caracteristicas}
%
Dominio de definici\'on & $\displaystyle \X = \left\{ k \in \optimes_{i=1}^c \{
0 \; \ldots \; K_i \} \tq \sum_{i=1}^c k_i = n \right\}$\\[2mm]
\hline
%
Parametros & $N \in \Nset^*$ \: (poblaci\'on)\newline $c \in \Nset^*$ \: (n\'umero de clases)\newline $\displaystyle K \in \left\{
\{ 0 \; \ldots \; N\}^c \tq \sum_{i=1}^c K_i = N \right\}$ \ (n\'umero de
estados en cada clase)\newline $n \in \{ 0 \; \ldots \; N\}$ \: (n\'umero de
tiros)\\[2mm]
\hline
%
Distribuci\'on de probabilidad & \protect$\displaystyle p_X(k) =
\frac{\prod_{i=1}^c \bino{K_i}{k_i}}{\bino{N}{n}}$\protect\\[2mm]
\hline
%
Promedio & $\displaystyle m_X = \frac{n}{N} \, K$\\[2mm]
\hline
%
Covarianza & $\displaystyle \Sigma_X = \frac{n (N-n)}{N^2 (N-1)} \left( N \diag K - K K^t \right)$\\[2mm]
\hline
%
%
%Generadora  de probabilidad  &  $\displaystyle  G_X(z) =  \left(  1 -  p  + p  z
%\right)^n$ \ sobre \ $\Cset$\\[2mm]
%\hline
%%
%Generadora  de momentos  &  $\displaystyle  M_X(u) =  \left(1  - p  +  p \,  e^u
%\right)^n$ \ sobre \ $\Cset$\\[2mm]
%\hline
%%
%Funci\'on caracter\'istica  & $\displaystyle \Phi_X(\omega) =  \left( 1 -  p + p
%\, e^{\imath \omega} \right)^n$
\end{caracteristicas}

Cuando $c = 2$ se recupera 

Su masa  de probabilidad  y funci\'on de  repartici\'on son representadas  en la
figura Fig.~\ref{Fig:MP:Hipergeometrica}.
%
\begin{figure}[h!]
% \begin{center} %\begin{tikzpicture}[fixed point arithmetic,scale=.8]
\begin{tikzpicture}[scale=.8]
\shorthandoff{>}
%
%
%\pgfmathsetmacro{\n}{5};% numeros para la multinomial
\pgfmathsetmacro{\dec}{.5};% shitf para dibujar las marginales
%
% ratio de pochammer decrecientes (b)_c / (a)_c
\tikzmath{function poc(\a,\b,\c) {
    if \c == 0 then {return 1;}
    else {return ((\b/\a)*poc(\a-1,\b-1,\c-1);};
};};
%
%
% Ejemplo
\begin{scope}
%
% c = 3 clases
\pgfmathsetmacro{\ku}{9};% k_1
\pgfmathsetmacro{\kd}{6};% k_2
\pgfmathsetmacro{\n}{18};% n
\pgfmathsetmacro{\m}{5};% m
%
\pgfmathsetmacro{\k}{\ku+\kd};% k_1+k_2
% Nota : con el fixed point, no anda min & max
% pero max(a,b) = (a+b+abs(a-b))/2  & min(a,b) = (a+b-abs(a-b))/2;
\pgfmathsetmacro{\s}{int((\m-\n+\k+abs(\m-\n+\k))/2)}; % x1+x2 min posible 
\pgfmathsetmacro{\S}{int((\m+\k-abs(\m-\k))/2)}; % x1+x2 max posible
%
\pgfmathsetmacro{\su}{int((\m-\n+\ku+abs(\m-\n+\ku))/2)}; % x1 min posible 
\pgfmathsetmacro{\Su}{int((\m+\ku-abs(\m-\ku))/2)}; % x1 max posible
%
\pgfmathsetmacro{\sd}{int((\m-\n+\kd+abs(\m-\n+\kd))/2)}; % x2 min posible 
\pgfmathsetmacro{\Sd}{int((\m+\kd-abs(\m-\kd))/2)}; % x2 max posible 

\begin{axis}[
    colormap = {whiteblack}{color(0cm)  = (white);color(1cm) = (black)},
    width=.55\textwidth,
    %height=.5\textheight,%\axisdefaultheight
    view={35}{60},
    enlargelimits=false,
    xmin={-\dec},
    xmax={\Su+\dec},
    ymin={-\dec},
    ymax={\Sd+\dec},
    zmax={.42},
    color=black,
    xtick={0,...,\Su},
    ytick={0,...,\Sd},
    xlabel=$x_1$,
    ylabel=$x_2$,
    zlabel=$p_{\widetilde{X}}$,
]
%

%
% Marginale 1
\pgfmathsetmacro{\bu}{1}; % init primer coef bino (k1 x1)
% Inicialisacion parte 2 de la probabilidad marginal, i.e. 2nd coef. binomial / (n m)
\pgfmathsetmacro{\bd}{poc(\n-\ku+\su,\m,\su)*poc(\n,\n-\ku+\su,\m)};
\foreach \xu in {0,...,\Su} { % bucla en x_1
   \ifnum\numexpr\xu > \numexpr\su-1
      \ifnum\numexpr\xu < \numexpr\Su+1
         \addplot3 [dotted,domain=0:{\bu*\bd},samples=2, samples y=0,color=black]
          (\xu,{\Sd+\dec},\x)  node[scale=.55]{$\bullet$};
      \fi
   \fi
   \pgfmathsetmacro{\blu}{\bu*(\ku-\xu)/(\xu+1)};
   \global\let\bu\blu;% parte proba en x1 actualizado
   \pgfmathsetmacro{\bld}{\bd*(\m-\xu)/(\n-\m-\ku+\xu+1)};
   \global\let\bd\bld;% parte 2 de la proba actualizado
   %
   % lineas x1 abajo
   \addplot3 [domain={-\dec}:{\Sd+\dec},samples=2, samples y=0,color=black!10] (\xu,\x,0);
}
\node at (axis cs:{3*\Su/4},{\Sd+\dec},{.25})[right]{$p_{X_1}$};
%
%
% Marginale 2
\pgfmathsetmacro{\bd}{1}; % init primer coef bino (k2 x2)
% Inicialisacion parte 2 de la probabilidad marginal, i.e. 2nd coef. binomial / (n m)
\pgfmathsetmacro{\bu}{poc(\n-\kd+\sd,\m,\sd)*poc(\n,\n-\kd+\sd,\m)};
\foreach \xd in {0,...,\Sd} { % bucla en x_2
   \ifnum\numexpr\xd > \numexpr\sd-1
      \ifnum\numexpr\xd < \numexpr\Sd+1
         \addplot3 [dotted,domain=0:{\bd*\bu},samples=2, samples y=0,color=black]
          ({-\dec},\xd,\x)  node[scale=.55]{$\bullet$};
      \fi
   \fi
   \pgfmathsetmacro{\bld}{\bd*(\kd-\xd)/(\xd+1)};
   \global\let\bd\bld;% parte proba en x2 actualizado
   \pgfmathsetmacro{\blu}{\bu*(\m-\xd)/(\n-\m-\kd+\xd+1)};
   \global\let\bu\blu;% parte 2 de la proba actualizado
   %
   % lineas x2 abajo
   \addplot3 [domain={-\dec}:{\Su+\dec},samples=2, samples y=0,color=black!10] (\x,\xd,0);
}
\node at (axis cs:{-\dec},{.6*\Sd},{.25})[right]{$p_{X_2}$};
%
% bivariada
%
% Inicialisacion parte 3 de la probabilidad, i.e. terco coef. binomial / (n m)
\pgfmathsetmacro{\bt}{poc(\n-\k+\s,\m,\s)*poc(\n,\n-\k+\s,\m)};
%\pgfmathsetmacro{\plim}{.000001};% si debajo de este valor, se pone a cero (liberar memorio) 
%\pgfmathsetmacro{\sy}{1.1};% scaling en y, para la bivariada 
%
% ahora bucla sobre x = x1+x2
\foreach \xs in {\s,...,\S} {
   \pgfmathsetmacro{\bu}{1};% inic coef. bino. parte x1 de la proba
   \foreach \xu in {0,...,\ku} { % bucla en x_1
      \pgfmathsetmacro{\bd}{1};% inic coef. bino. parte x2 de la proba
      \foreach \xd in {0,...,\kd} { % bucla en x_2
         %\pgfmathsetmacro{\tx}{\xu+\xd};
         \pgfmathparse{int(round(\xu+\xd-\xs))};\let\dif\pgfmathresult;
         \ifnum\dif=0 %\numexpr\xu+\xd = \numexpr\xs % si x_1+x_2 = x que fijamos
            %\pgfmathsetmacro{\pr}{\bu*\bd*\bt};
            \addplot3 [dotted,domain=0:{\bu*\bd*\bt},samples=2, samples y=0,color=black]
            (\xu,\xd,\x)  node[scale=.85]{$\bullet$};
         \fi
         \pgfmathsetmacro{\bld}{\bd*(\kd-\xd)/(\xd+1)};
         \global\let\bd\bld;% parte proba en x2 (x1 fijo) actualizado
      }
      \pgfmathsetmacro{\blu}{\bu*(\ku-\xu)/(\xu+1)};
      \global\let\bu\blu;% parte proba en x1 actualizado
   }
   \pgfmathsetmacro{\blt}{\bt*(\m-\xs)/(\n-\m-\k+\xs+1)};
   \global\let\bt\blt;% parte 3 de la proba actualizado
}
\end{axis}
\node at ({.6*\Su},-1)[scale=.9]{(a)};
\end{scope}
%
%
% -----------------------------------
%
% Ejemplo 
\begin{scope}[xshift = 10.5cm]
%
% c = 3 clases
\pgfmathsetmacro{\ku}{6};% k_1
\pgfmathsetmacro{\kd}{6};% k_2
\pgfmathsetmacro{\n}{18};% n
\pgfmathsetmacro{\m}{5};% m
%
\pgfmathsetmacro{\k}{\ku+\kd};% k_1+k_2
% Nota : con el fixed point, no anda min & max
% pero max(a,b) = (a+b+abs(a-b))/2  & min(a,b) = (a+b-abs(a-b))/2;
\pgfmathsetmacro{\s}{int((\m-\n+\k+abs(\m-\n+\k))/2)}; % x1+x2 min posible 
\pgfmathsetmacro{\S}{int((\m+\k-abs(\m-\k))/2)}; % x1+x2 max posible
%
\pgfmathsetmacro{\su}{int((\m-\n+\ku+abs(\m-\n+\ku))/2)}; % x1 min posible 
\pgfmathsetmacro{\Su}{int((\m+\ku-abs(\m-\ku))/2)}; % x1 max posible
%
\pgfmathsetmacro{\sd}{int((\m-\n+\kd+abs(\m-\n+\kd))/2)}; % x2 min posible 
\pgfmathsetmacro{\Sd}{int((\m+\kd-abs(\m-\kd))/2)}; % x2 max posible 

\begin{axis}[
    colormap = {whiteblack}{color(0cm)  = (white);color(1cm) = (black)},
    width=.55\textwidth,
    %height=.5\textheight,%\axisdefaultheight
    view={35}{60},
    enlargelimits=false,
    xmin={-\dec},
    xmax={\Su+\dec},
    ymin={-\dec},
    ymax={\Sd+\dec},
    zmax={.42},
    color=black,
    xtick={0,...,\Su},
    ytick={0,...,\Sd},
    xlabel=$x_1$,
    ylabel=$x_2$,
    zlabel=$p_{\widetilde{X}}$,
]
%
%
% Marginale 1
\pgfmathsetmacro{\bu}{1}; % init primer coef bino (k1 x1)
% Inicialisacion parte 2 de la probabilidad marginal, i.e. 2nd coef. binomial / (n m)
\pgfmathsetmacro{\bd}{poc(\n-\ku+\su,\m,\su)*poc(\n,\n-\ku+\su,\m)};
\foreach \xu in {0,...,\Su} { % bucla en x_1
   \ifnum\numexpr\xu > \numexpr\su-1
      \ifnum\numexpr\xu < \numexpr\Su+1
         \addplot3 [dotted,domain=0:{\bu*\bd},samples=2, samples y=0,color=black]
          (\xu,{\Sd+\dec},\x)  node[scale=.55]{$\bullet$};
      \fi
   \fi
   \pgfmathsetmacro{\blu}{\bu*(\ku-\xu)/(\xu+1)};
   \global\let\bu\blu;% parte proba en x1 actualizado
   \pgfmathsetmacro{\bld}{\bd*(\m-\xu)/(\n-\m-\ku+\xu+1)};
   \global\let\bd\bld;% parte 2 de la proba actualizado
   %
   % lineas x1 abajo
   \addplot3 [domain={-\dec}:{\Sd+\dec},samples=2, samples y=0,color=black!10] (\xu,\x,0);
}
\node at (axis cs:{\ku/2},{\kd+\dec},{.1})[right]{$p_{X_1}$};
%
%
% Marginale 2
\pgfmathsetmacro{\bd}{1}; % init primer coef bino (k2 x2)
% Inicialisacion parte 2 de la probabilidad marginal, i.e. 2nd coef. binomial / (n m)
\pgfmathsetmacro{\bu}{poc(\n-\kd+\sd,\m,\sd)*poc(\n,\n-\kd+\sd,\m)};
\foreach \xd in {0,...,\Sd} { % bucla en x_2
   \ifnum\numexpr\xd > \numexpr\sd-1
      \ifnum\numexpr\xd < \numexpr\Sd+1
         \addplot3 [dotted,domain=0:{\bd*\bu},samples=2, samples y=0,color=black]
          ({-\dec},\xd,\x)  node[scale=.55]{$\bullet$};
      \fi
   \fi
   \pgfmathsetmacro{\bld}{\bd*(\kd-\xd)/(\xd+1)};
   \global\let\bd\bld;% parte proba en x2 actualizado
   \pgfmathsetmacro{\blu}{\bu*(\m-\xd)/(\n-\m-\kd+\xd+1)};
   \global\let\bu\blu;% parte 2 de la proba actualizado
   %
   % lineas x2 abajo
   \addplot3 [domain={-\dec}:{\Su+\dec},samples=2, samples y=0,color=black!10] (\x,\xd,0);
}
\node at (axis cs:{-\dec},{.6*\Sd},{.25})[right]{$p_{X_2}$};
%
% bivariada
%
% Inicialisacion parte 3 de la probabilidad, i.e. terco coef. binomial / (n m)
\pgfmathsetmacro{\bt}{poc(\n-\k+\s,\m,\s)*poc(\n,\n-\k+\s,\m)};
%\pgfmathsetmacro{\plim}{.000001};% si debajo de este valor, se pone a cero (liberar memorio) 
%\pgfmathsetmacro{\sy}{1.1};% scaling en y, para la bivariada 
%
% ahora bucla sobre x = x1+x2
\foreach \xs in {\s,...,\S} {
   \pgfmathsetmacro{\bu}{1};% inic coef. bino. parte x1 de la proba
   \foreach \xu in {0,...,\ku} { % bucla en x_1
      \pgfmathsetmacro{\bd}{1};% inic coef. bino. parte x2 de la proba
      \foreach \xd in {0,...,\kd} { % bucla en x_2
         %\pgfmathsetmacro{\tx}{\xu+\xd};
         \pgfmathparse{int(round(\xu+\xd-\xs))};\let\dif\pgfmathresult;
         \ifnum\dif=0 %\numexpr\xu+\xd = \numexpr\xs % si x_1+x_2 = x que fijamos
            %\pgfmathsetmacro{\pr}{\bu*\bd*\bt};
            \addplot3 [dotted,domain=0:{\bu*\bd*\bt},samples=2, samples y=0,color=black]
            (\xu,\xd,\x)  node[scale=.85]{$\bullet$};
         \fi
         \pgfmathsetmacro{\bld}{\bd*(\kd-\xd)/(\xd+1)};
         \global\let\bd\bld;% parte proba en x2 (x1 fijo) actualizado
      }
      \pgfmathsetmacro{\blu}{\bu*(\ku-\xu)/(\xu+1)};
      \global\let\bu\blu;% parte proba en x1 actualizado
   }
   \pgfmathsetmacro{\blt}{\bt*(\m-\xs)/(\n-\m-\k+\xs+1)};
   \global\let\bt\blt;% parte 3 de la proba actualizado
}
\end{axis}
\node at ({.6*\Su},-1)[scale=.9]{(b)};
%
\end{scope}
%
\end{tikzpicture} \end{center}
%
\leyenda{Ilustraci\'on de una distribuci\'on  de probabilidad Hipergeometrica (a) \SZ{con $n = 6, \quad p = \frac13$.}}
\label{Fig:MP:HipergeometricaMultivariada}
\end{figure}
\SZ{CERRAR}

\SZ{Reflexibilidad? Stabilidad por agregaci\'on, Marginales hipegeometrica}
% Cuando  $n  = 1$,  se  recupera  la lei  de  Bernoulli  $\B(p) \equiv  \B(1,p)$.
% Ad\'emas, se muestra  sencillamente usando la generadora de  probabilidad que
% %
% De este resultado,  se puede notar que, por  ejemplo, le distribuci\'on binomial
% aparece en el conteo de eventos independientes de misma probabilidad entre $n$.

% Tambi\'en,  la ley binomial  tiene una  propiedad de  reflexividad, consecuencia
% directa de la de Bernoulli:
% %
% \begin{lema}[Reflexividad]
% \label{Lem:MP:ReflexividadBinomial}
% %
%   Sea \ $X \, \sim \, \B(n,p)$. Entonces
%   %
%   \[
%   n-X \, \sim \, \B(n,1-p)
%   \]
%   %
% \end{lema}

% Nota que cuando $p = 0$ (resp. $p = 1$) la variable es cierta $X = 0$ (resp.  $X
% = n$).
