\subsubseccion{Ley hipergeometrica multivariada}
\label{Sssec:MP:HipergeometricaMultivariada}

Esta ley aparece por ejemplo cuando  se se generaliza la ley hipergeometrica con
$c > 2$ clases \ con \ $k_i$  \ num\'ero de elementos de la clase \ $i$, $\sum_i
k_i = n$.  Se estudia esta ley, entre otros, por la primera vez, en el ensayo de
Montmort  en  1708~\cite{Mon13},  o  m\'as   tarde,  en  1740,  en  trabajos  de
Simpson~\cite{Sim40, Hal90, DavEdw01}.

Se denota \ $X \, \sim \, \H\M(n,k,m)$ \ con \ $\displaystyle n \in \Nset$, \quad
$k = \begin{bmatrix} k_1 & \cdots & k_c\end{bmatrix}^t \in \Part{n}{c}$ (ver notaciones)
% \left\{ \{ 0 \; \ldots  \; n\}^c  \tq \sum_{i=1}^c k_i  = n \right\}$,
  \quad $m \in  \{ 0 \;  \ldots \;
n\}$.

Entonces, como en el caso de la ley multinomial, a pesar de que se escribe \ $X$
\  de manera  $c$-dimensional, el  vector partenece  a un  espacio  claramente \
$(c-1)$-dimensional.  Notar que  en el  caso \  $c  = 2$  \ se  recupera la  ley
hipergeometrica.

Sus caracter\'isticas son las siguientes:

\begin{caracteristicas}
%
Dominio de definici\'on~\footnote{De hecho, se puede considerar que el vector
aleatorio es \ $(c-1)$-dimensional \ $\widetilde{X} = \begin{bmatrix}
\widetilde{X}_1 & \cdots & \widetilde{X}_{c-1} \end{bmatrix}^t$ \ definido sobre
el dominio \ $\displaystyle \widetilde{\X} = \left\{ x \in \optimes_{i=1}^{c-1}
\{ 0 \; \ldots \; k_i\} \tq \max\left( 0 \, , \, \sum_{i=1}^{c-1} k_i + m - n
\right) \: \le \: \sum_{i=1}^{c-1} x_i \: \le \: m
\right\}$.\label{Foot:MP:HipergeomMultiDominio}} & $\displaystyle \X = \left\{ x
\in \optimes_{i=1}^c \{ 0 \; \ldots \; k_i \} \tq \sum_{i=1}^c x_i = m
\right\}$\\[2mm]
\hline
%
Par\'ametros~\footnote{Los par\'ametros de \ $\widetilde{X}$ \ son \ $n \in
\Nset^*$, \ $m \in \{ 0 \; \ldots \; n \}$ \ y \ $\widetilde{k} =
\protect\begin{bmatrix} k_1 & \cdots & k_{c-1} \end{bmatrix}^t\protect \in
\left\{ q \in \{ 0 \; \ldots \; n \}^{c-1} \tq \sum_{i=1}^{c-1} q_i \, \le \, n
\right\}$.\label{Foot:MP:MultinomialParametro}} & $n \in \Nset^*$ \:
(poblaci\'on)\newline $c \in \Nset^*$ \: (n\'umero de clases)\newline
$\displaystyle k \in \Part{n}{c}$ 
%\left\{ q \in \{ 0 \; \ldots \; n\}^c \tq \sum_{i=1}^c q_i
%= n \right\}$ 
\ (n\'umeros de elementos de cada clase)\newline $m \in \{ 0 \;
\ldots \; n\}$ \: (n\'umero de tiros)\\[2mm]
\hline
%
Distribuci\'on de probabilidad~\footnote{La masa de probabilidad de \
$\widetilde{X}$ \ es \ $\displaystyle p_{\widetilde{X}}(x) =
\frac{\prod_{i=1}^{c-1} \bino{k_i\vspace{1mm}}{x_i} \, \smallbino{n -
\sum_{i=1}^{c-1} k_i\vspace{2mm}}{m - \sum_{i=1}^{c-1}
x_i}}{\bino{n}{m}}$.\label{Foot:MP:HipergeomMultiMasa}.} &
\protect$\displaystyle p_X(x) = \frac{\prod_{i=1}^c
\smallbino{k_i}{x_i}}{\smallbino{n}{m}}$\protect\\[2mm]
\hline
%
Promedio & $\displaystyle m_X = \frac{m}{n} \, k$\\[2mm]
\hline
%
Covarianza~\footnote{Ver notas de pie~\ref{Foot:MP:HipergeometricaVarianza}
y~\ref{Foot:MP::MultinomialCovarianza}.} & $\displaystyle \Sigma_X = \left\{
\protect\begin{array}{ccc} \frac{m (n-m)}{n^2 (n-1)} \left( n \diag(k) - k k^t
\right) & \mbox{si} & n > 1\\ 0 & \mbox{si} & n = 1\end{array}\protect\right.$
%\\[2mm]
%\hline
%
%
%Generadora  de probabilidad~\footnote{Ver nota de pie~\ref{Foot:MP:MultinomialGeneProba}
%reemplazando $n$ por $m$.}  &  $\displaystyle  G_X(z) =  \left(  1 -  p  + p  z
%\right)^n$ \ sobre \ $\Cset$\\[2mm]
%\hline
%%
%Generadora  de momentos~\footnote{Ver nota de pie~\ref{Foot:MP:MultinomialGeneMomentos}
%reemplazando $n$ por $m$.}  &  $\displaystyle  M_X(u) =  \left(1  - p  +  p \,  e^u
%\right)^n$ \ sobre \ $\Cset$\\[2mm]
%\hline
%%
%Funci\'on caracter\'istica~\footnote{Ver nota de pie~\ref{Foot:MP:MultinomialCaracteristica}
%reemplazando $n$ por $m$.}  & $\displaystyle \Phi_X(\omega) =  \left( 1 -  p + p
%\, e^{\imath \omega} \right)^n$
\end{caracteristicas}

\SZ{Ver si se calcula Phi}

La masa  de probabilidad es representada  en la
figura Fig.~\ref{Fig:MP:HipergeometricaMultivariada}.
%
\begin{figure}[h!]
\begin{center} %\begin{tikzpicture}[fixed point arithmetic,scale=.8]
\begin{tikzpicture}[scale=.8]
\shorthandoff{>}
%
%
%\pgfmathsetmacro{\n}{5};% numeros para la multinomial
\pgfmathsetmacro{\dec}{.5};% shitf para dibujar las marginales
%
% ratio de pochammer decrecientes (b)_c / (a)_c
\tikzmath{function poc(\a,\b,\c) {
    if \c == 0 then {return 1;}
    else {return ((\b/\a)*poc(\a-1,\b-1,\c-1);};
};};
%
%
% Ejemplo
\begin{scope}
%
% c = 3 clases
\pgfmathsetmacro{\ku}{9};% k_1
\pgfmathsetmacro{\kd}{6};% k_2
\pgfmathsetmacro{\n}{18};% n
\pgfmathsetmacro{\m}{5};% m
%
\pgfmathsetmacro{\k}{\ku+\kd};% k_1+k_2
% Nota : con el fixed point, no anda min & max
% pero max(a,b) = (a+b+abs(a-b))/2  & min(a,b) = (a+b-abs(a-b))/2;
\pgfmathsetmacro{\s}{int((\m-\n+\k+abs(\m-\n+\k))/2)}; % x1+x2 min posible 
\pgfmathsetmacro{\S}{int((\m+\k-abs(\m-\k))/2)}; % x1+x2 max posible
%
\pgfmathsetmacro{\su}{int((\m-\n+\ku+abs(\m-\n+\ku))/2)}; % x1 min posible 
\pgfmathsetmacro{\Su}{int((\m+\ku-abs(\m-\ku))/2)}; % x1 max posible
%
\pgfmathsetmacro{\sd}{int((\m-\n+\kd+abs(\m-\n+\kd))/2)}; % x2 min posible 
\pgfmathsetmacro{\Sd}{int((\m+\kd-abs(\m-\kd))/2)}; % x2 max posible 

\begin{axis}[
    colormap = {whiteblack}{color(0cm)  = (white);color(1cm) = (black)},
    width=.55\textwidth,
    %height=.5\textheight,%\axisdefaultheight
    view={35}{60},
    enlargelimits=false,
    xmin={-\dec},
    xmax={\Su+\dec},
    ymin={-\dec},
    ymax={\Sd+\dec},
    zmax={.42},
    color=black,
    xtick={0,...,\Su},
    ytick={0,...,\Sd},
    xlabel=$x_1$,
    ylabel=$x_2$,
    zlabel=$p_{\widetilde{X}}$,
]
%

%
% Marginale 1
\pgfmathsetmacro{\bu}{1}; % init primer coef bino (k1 x1)
% Inicialisacion parte 2 de la probabilidad marginal, i.e. 2nd coef. binomial / (n m)
\pgfmathsetmacro{\bd}{poc(\n-\ku+\su,\m,\su)*poc(\n,\n-\ku+\su,\m)};
\foreach \xu in {0,...,\Su} { % bucla en x_1
   \ifnum\numexpr\xu > \numexpr\su-1
      \ifnum\numexpr\xu < \numexpr\Su+1
         \addplot3 [dotted,domain=0:{\bu*\bd},samples=2, samples y=0,color=black]
          (\xu,{\Sd+\dec},\x)  node[scale=.55]{$\bullet$};
      \fi
   \fi
   \pgfmathsetmacro{\blu}{\bu*(\ku-\xu)/(\xu+1)};
   \global\let\bu\blu;% parte proba en x1 actualizado
   \pgfmathsetmacro{\bld}{\bd*(\m-\xu)/(\n-\m-\ku+\xu+1)};
   \global\let\bd\bld;% parte 2 de la proba actualizado
   %
   % lineas x1 abajo
   \addplot3 [domain={-\dec}:{\Sd+\dec},samples=2, samples y=0,color=black!10] (\xu,\x,0);
}
\node at (axis cs:{3*\Su/4},{\Sd+\dec},{.25})[right]{$p_{X_1}$};
%
%
% Marginale 2
\pgfmathsetmacro{\bd}{1}; % init primer coef bino (k2 x2)
% Inicialisacion parte 2 de la probabilidad marginal, i.e. 2nd coef. binomial / (n m)
\pgfmathsetmacro{\bu}{poc(\n-\kd+\sd,\m,\sd)*poc(\n,\n-\kd+\sd,\m)};
\foreach \xd in {0,...,\Sd} { % bucla en x_2
   \ifnum\numexpr\xd > \numexpr\sd-1
      \ifnum\numexpr\xd < \numexpr\Sd+1
         \addplot3 [dotted,domain=0:{\bd*\bu},samples=2, samples y=0,color=black]
          ({-\dec},\xd,\x)  node[scale=.55]{$\bullet$};
      \fi
   \fi
   \pgfmathsetmacro{\bld}{\bd*(\kd-\xd)/(\xd+1)};
   \global\let\bd\bld;% parte proba en x2 actualizado
   \pgfmathsetmacro{\blu}{\bu*(\m-\xd)/(\n-\m-\kd+\xd+1)};
   \global\let\bu\blu;% parte 2 de la proba actualizado
   %
   % lineas x2 abajo
   \addplot3 [domain={-\dec}:{\Su+\dec},samples=2, samples y=0,color=black!10] (\x,\xd,0);
}
\node at (axis cs:{-\dec},{.6*\Sd},{.25})[right]{$p_{X_2}$};
%
% bivariada
%
% Inicialisacion parte 3 de la probabilidad, i.e. terco coef. binomial / (n m)
\pgfmathsetmacro{\bt}{poc(\n-\k+\s,\m,\s)*poc(\n,\n-\k+\s,\m)};
%\pgfmathsetmacro{\plim}{.000001};% si debajo de este valor, se pone a cero (liberar memorio) 
%\pgfmathsetmacro{\sy}{1.1};% scaling en y, para la bivariada 
%
% ahora bucla sobre x = x1+x2
\foreach \xs in {\s,...,\S} {
   \pgfmathsetmacro{\bu}{1};% inic coef. bino. parte x1 de la proba
   \foreach \xu in {0,...,\ku} { % bucla en x_1
      \pgfmathsetmacro{\bd}{1};% inic coef. bino. parte x2 de la proba
      \foreach \xd in {0,...,\kd} { % bucla en x_2
         %\pgfmathsetmacro{\tx}{\xu+\xd};
         \pgfmathparse{int(round(\xu+\xd-\xs))};\let\dif\pgfmathresult;
         \ifnum\dif=0 %\numexpr\xu+\xd = \numexpr\xs % si x_1+x_2 = x que fijamos
            %\pgfmathsetmacro{\pr}{\bu*\bd*\bt};
            \addplot3 [dotted,domain=0:{\bu*\bd*\bt},samples=2, samples y=0,color=black]
            (\xu,\xd,\x)  node[scale=.85]{$\bullet$};
         \fi
         \pgfmathsetmacro{\bld}{\bd*(\kd-\xd)/(\xd+1)};
         \global\let\bd\bld;% parte proba en x2 (x1 fijo) actualizado
      }
      \pgfmathsetmacro{\blu}{\bu*(\ku-\xu)/(\xu+1)};
      \global\let\bu\blu;% parte proba en x1 actualizado
   }
   \pgfmathsetmacro{\blt}{\bt*(\m-\xs)/(\n-\m-\k+\xs+1)};
   \global\let\bt\blt;% parte 3 de la proba actualizado
}
\end{axis}
\node at ({.6*\Su},-1)[scale=.9]{(a)};
\end{scope}
%
%
% -----------------------------------
%
% Ejemplo 
\begin{scope}[xshift = 10.5cm]
%
% c = 3 clases
\pgfmathsetmacro{\ku}{6};% k_1
\pgfmathsetmacro{\kd}{6};% k_2
\pgfmathsetmacro{\n}{18};% n
\pgfmathsetmacro{\m}{5};% m
%
\pgfmathsetmacro{\k}{\ku+\kd};% k_1+k_2
% Nota : con el fixed point, no anda min & max
% pero max(a,b) = (a+b+abs(a-b))/2  & min(a,b) = (a+b-abs(a-b))/2;
\pgfmathsetmacro{\s}{int((\m-\n+\k+abs(\m-\n+\k))/2)}; % x1+x2 min posible 
\pgfmathsetmacro{\S}{int((\m+\k-abs(\m-\k))/2)}; % x1+x2 max posible
%
\pgfmathsetmacro{\su}{int((\m-\n+\ku+abs(\m-\n+\ku))/2)}; % x1 min posible 
\pgfmathsetmacro{\Su}{int((\m+\ku-abs(\m-\ku))/2)}; % x1 max posible
%
\pgfmathsetmacro{\sd}{int((\m-\n+\kd+abs(\m-\n+\kd))/2)}; % x2 min posible 
\pgfmathsetmacro{\Sd}{int((\m+\kd-abs(\m-\kd))/2)}; % x2 max posible 

\begin{axis}[
    colormap = {whiteblack}{color(0cm)  = (white);color(1cm) = (black)},
    width=.55\textwidth,
    %height=.5\textheight,%\axisdefaultheight
    view={35}{60},
    enlargelimits=false,
    xmin={-\dec},
    xmax={\Su+\dec},
    ymin={-\dec},
    ymax={\Sd+\dec},
    zmax={.42},
    color=black,
    xtick={0,...,\Su},
    ytick={0,...,\Sd},
    xlabel=$x_1$,
    ylabel=$x_2$,
    zlabel=$p_{\widetilde{X}}$,
]
%
%
% Marginale 1
\pgfmathsetmacro{\bu}{1}; % init primer coef bino (k1 x1)
% Inicialisacion parte 2 de la probabilidad marginal, i.e. 2nd coef. binomial / (n m)
\pgfmathsetmacro{\bd}{poc(\n-\ku+\su,\m,\su)*poc(\n,\n-\ku+\su,\m)};
\foreach \xu in {0,...,\Su} { % bucla en x_1
   \ifnum\numexpr\xu > \numexpr\su-1
      \ifnum\numexpr\xu < \numexpr\Su+1
         \addplot3 [dotted,domain=0:{\bu*\bd},samples=2, samples y=0,color=black]
          (\xu,{\Sd+\dec},\x)  node[scale=.55]{$\bullet$};
      \fi
   \fi
   \pgfmathsetmacro{\blu}{\bu*(\ku-\xu)/(\xu+1)};
   \global\let\bu\blu;% parte proba en x1 actualizado
   \pgfmathsetmacro{\bld}{\bd*(\m-\xu)/(\n-\m-\ku+\xu+1)};
   \global\let\bd\bld;% parte 2 de la proba actualizado
   %
   % lineas x1 abajo
   \addplot3 [domain={-\dec}:{\Sd+\dec},samples=2, samples y=0,color=black!10] (\xu,\x,0);
}
\node at (axis cs:{\ku/2},{\kd+\dec},{.1})[right]{$p_{X_1}$};
%
%
% Marginale 2
\pgfmathsetmacro{\bd}{1}; % init primer coef bino (k2 x2)
% Inicialisacion parte 2 de la probabilidad marginal, i.e. 2nd coef. binomial / (n m)
\pgfmathsetmacro{\bu}{poc(\n-\kd+\sd,\m,\sd)*poc(\n,\n-\kd+\sd,\m)};
\foreach \xd in {0,...,\Sd} { % bucla en x_2
   \ifnum\numexpr\xd > \numexpr\sd-1
      \ifnum\numexpr\xd < \numexpr\Sd+1
         \addplot3 [dotted,domain=0:{\bd*\bu},samples=2, samples y=0,color=black]
          ({-\dec},\xd,\x)  node[scale=.55]{$\bullet$};
      \fi
   \fi
   \pgfmathsetmacro{\bld}{\bd*(\kd-\xd)/(\xd+1)};
   \global\let\bd\bld;% parte proba en x2 actualizado
   \pgfmathsetmacro{\blu}{\bu*(\m-\xd)/(\n-\m-\kd+\xd+1)};
   \global\let\bu\blu;% parte 2 de la proba actualizado
   %
   % lineas x2 abajo
   \addplot3 [domain={-\dec}:{\Su+\dec},samples=2, samples y=0,color=black!10] (\x,\xd,0);
}
\node at (axis cs:{-\dec},{.6*\Sd},{.25})[right]{$p_{X_2}$};
%
% bivariada
%
% Inicialisacion parte 3 de la probabilidad, i.e. terco coef. binomial / (n m)
\pgfmathsetmacro{\bt}{poc(\n-\k+\s,\m,\s)*poc(\n,\n-\k+\s,\m)};
%\pgfmathsetmacro{\plim}{.000001};% si debajo de este valor, se pone a cero (liberar memorio) 
%\pgfmathsetmacro{\sy}{1.1};% scaling en y, para la bivariada 
%
% ahora bucla sobre x = x1+x2
\foreach \xs in {\s,...,\S} {
   \pgfmathsetmacro{\bu}{1};% inic coef. bino. parte x1 de la proba
   \foreach \xu in {0,...,\ku} { % bucla en x_1
      \pgfmathsetmacro{\bd}{1};% inic coef. bino. parte x2 de la proba
      \foreach \xd in {0,...,\kd} { % bucla en x_2
         %\pgfmathsetmacro{\tx}{\xu+\xd};
         \pgfmathparse{int(round(\xu+\xd-\xs))};\let\dif\pgfmathresult;
         \ifnum\dif=0 %\numexpr\xu+\xd = \numexpr\xs % si x_1+x_2 = x que fijamos
            %\pgfmathsetmacro{\pr}{\bu*\bd*\bt};
            \addplot3 [dotted,domain=0:{\bu*\bd*\bt},samples=2, samples y=0,color=black]
            (\xu,\xd,\x)  node[scale=.85]{$\bullet$};
         \fi
         \pgfmathsetmacro{\bld}{\bd*(\kd-\xd)/(\xd+1)};
         \global\let\bd\bld;% parte proba en x2 (x1 fijo) actualizado
      }
      \pgfmathsetmacro{\blu}{\bu*(\ku-\xu)/(\xu+1)};
      \global\let\bu\blu;% parte proba en x1 actualizado
   }
   \pgfmathsetmacro{\blt}{\bt*(\m-\xs)/(\n-\m-\k+\xs+1)};
   \global\let\bt\blt;% parte 3 de la proba actualizado
}
\end{axis}
\node at ({.6*\Su},-1)[scale=.9]{(b)};
%
\end{scope}
%
\end{tikzpicture} \end{center}
%
\leyenda{Ilustraci\'on  de una  distribuci\'on  de probabilidad  hipergeometrica
  multivariada  para   \  $c  =  3$   \  del  vector   \  $(c-1)$-dimensional  \
  $\widetilde{X} = \protect\begin{bmatrix}  X_1 & X_2 \protect\end{bmatrix}^t$ \
  ($X_3 = m-X_1-X_2$) \ con las  marginales \ $p_{X_1}, \: p_{X_2}$ \ (ver notas
  de                                      pie~\ref{Foot:MP:HipergeomMultiDominio}
  y~\ref{Foot:MP:HipergeomMultiMasa}).  Es  dibujada solamente la distribuci\'on
  sobre $\widetilde{\X}$, siendo esta  nula afuera de $\widetilde{\X}$.  Los
    par\'ametros son \ $n = 18$, \ $m = 5$, \ $k = \protect\begin{bmatrix} 9 & 6 &
      3 \protect\end{bmatrix}^t$  (a), $k  = \protect\begin{bmatrix} 6  & 6  & 6
      \protect\end{bmatrix}^t$ (b).}
\label{Fig:MP:HipergeometricaMultivariada}
\end{figure}


Notar: cuando $c = 2$ se  recupera la ley hipergeometrica; adem\'as $X$ resuelta
cierta   en  los  casos   siguientes  (ver   subesecci\'on  anterior   para  las
explicaciones/ilustraciones):
%
\begin{itemize}
\item $m =  0 \: \Rightarrow \: X = 0$;
%
\item  $m = n \: \Rightarrow \: X = k$;
%
\item $k = n \un_i \: \Rightarrow \: X = m \un_i$.
\end{itemize}

% \SZ{casos particulares a ver; caso de propiedades de reflexividad a ver (igual
%   para la multinomial)}

Vectores  de  distribuci\'on  hipergeometricas multivaluada  tienen  propiedades
notables similares a las de la  hipergeometricas y de la multinomial, a saber de
tipo reflexividad, con respecto a una permutaci\'on de variable y con respecto a
una agregaci\'on.
%
\begin{lema}[Reflexividad]
\label{Lem:MP:ReflexividadHipergeomMulti}
%
  Sea \ $X \, \sim \, \H\M(n,k,m)$. Entonces
  %
  \[
  k-X \sim \H\M(n,k,n-m)
  \]
  %
\end{lema}
%
Como  el en  contexto  escalar, si  en una  urna  tenemos bolas  de $c$  colores
diferentes,  con  un n\'umero  $k_i$  para el  $i$-\'esimo  color,  $X_i$ es  el
num\'ero de este color que se sorte\'o y $k_i-X_i$ representan las de este color
que quedan en la urna, entre las  \ $n-m = \sum_{i=1}^c (k_i-X_i)$ \ que quedan,
es decir que en \ $k-X$ \ se  intercambia los roles de las bolas sorteadas y las
que quedan en la urna. M\'as formalmente:
%
\begin{proof}
  %El  primer   resultado  es  inmediato  de  $P(m-X   =  x)  =  P(X   =  m-x)  =
  %\frac{\smallbino{k}{m-x} \smallbino{n-k}{x}}{\smallbino{n}{m}}$. El secundo de
  Sea $Y =  k-X$. De $P(Y = y) =  P(k-X = y) = P(X  = k-y) = \frac{\prod_{i=1}^c
    \smallbino{k_i}{k_i-y_i}}{\smallbino{n}{m}}       =      \frac{\prod_{i=1}^c
    \smallbino{k_i}{y_i}}{\smallbino{n}{n-m}}$   notando   que  $\bino{a}{b}   =
  \bino{a}{a-b}$. Se cierra la prueba  recordandose que \ $\sum_{i=1}^c k_i = n$
  \  y  \ $\sum_{i=1}^c  x_i  =  m$,  dando \  $\sum_{i=1}^c  k_i  =  n$ \  y  \
  $\sum_{i=1}^c y_i = n-m$.
\end{proof}
%
\begin{lema}[Efecto de una permutaci\'on]\label{Lem:MP:PermutacionHipergeomMulti}
%
  Sea  \ $X =  \begin{bmatrix} X_1  & \cdots  & X_c  \end{bmatrix}^t \,  \sim \,
  \H\M(n,k,m)$  \   y  \   $\Pi  \in  \perm_c(\Rset)$   \  matriz   \  de
  permutaci\'on. Entonces
  %
  \[
  \Pi X \, \sim \, \H\M\left( n ,  \Pi k , m \right)
  \]
  %
\end{lema}
%
\begin{proof}
  La  prueba sigue  paso paso  la de  la multinomial.  Notando la  permutation \
  $\sigma$ \  tal que \ $\Pi  = \sum_{i=1}^c \un_i  \un_{\sigma(i)}^t$, se puede
  ver   que  \  $\displaystyle   P(\Pi  X   =  x)   =  P(X   =  \Pi^{-1}   x)  =
  \frac{\prod_{i=1}^c       \bino{k_i}{x_{\sigma^{-1}(i)}}}{\bino{n}{m}}       =
  \frac{\prod_{i=1}^c \bino{k_{\sigma(i)}}{x_i}}{\bino{n}{m}} $  \ por cambio de
  indices.
\end{proof}
%
\begin{lema}[Stabilidad por agregaci\'on]\label{Lem:MP:StabAgregacionHipergeomMulti}
%
  Sea  \ $X =  \begin{bmatrix} X_1  & \cdots  & X_c  \end{bmatrix}^t \,  \sim \,
  \H\M(n,k,m)$ \ y \ $G^{(i,j)}$ \ matriz de agrupaci\'on de las $(i,j)$-\'esima
  componentes (ver notaciones). Entonces,
  %
  \[
  G^{(i,j)} X \, \sim \, \H\M\left( n , G^{(i,j)} k , m \right)  
  \]
  %
\end{lema}
%
Este resultado es  intuitivo en el hecho  de que vuelve a agrupar  las clases \
$i$ \ e \ $j$ \ en una clase, que tiene entonces  \ $k_i + k_j$ \ elementos.
%

%
\begin{proof}
 Del  lema precediente,
  notando  que  existen  matrices  de permutaci\'on~\footnote{$\Pi_k$  pone  las
    componentes $i$ \ e \ $j$ el  las posiciones $c-1$ y $c$, sin cambiar el orden
    de  las  precedientes;  $\Pi_{k-1}$  trazlada  la  \'ultima  componente  en  la
    posici\'on $\min(i,j)$.}  \ $\Pi_k \in  \perm_k(\Rset)$ \ y \ $\Pi_{k-1} \in
  \perm_{k-1}(\Rset)$ \ tal que \ $G^{(i,j)} = \Pi_{k-1} \, G^{(c-1,c)} \, \Pi_k$,
  se puede concentrarse en el caso \ $(i,j) = (c-1,c)$. Ahora, claramente,
  %
  \begin{eqnarray*}
  P(G^{(c-1,c)} X = x) & = & \displaystyle P\left( \bigcap_{i=1}^{c-2} \big( X_i = x_i \big)
  \: \cap \: \big( X_{c-1} + X_c = x_{c-1} \big) \right)\\[2mm]
  %
  & = & \displaystyle \sum_{t=0}^{x_{c-1}} P\left( \bigcap_{i=1}^{c-2} \big( X_i = x_i \big)
  \: \cap \: \big( X_{c-1} = t\big) \: \cap \: \big( X_c = t-x_{c-1} \big) \right)\\[2mm]
  %
  & = & \frac{\prod_{i=1}^{c-2} \bino{k_i}{x_i}}{\bino{n}{m}}
  \: \sum_{t=0}^{x_{c-1}} \bino{k_{c-1}}{t}\bino{k_c}{x_c-t}
  \end{eqnarray*}
  %
  Se  cierra  la  prueba   de  la  identidad  de  Chu-Vandermonde~\footnote{Este
    identidad es  debido a A.-T. Vandermonde  en 1772, pero  esta conocida desde
    1303 por el matematico chino  Chu Shi-Chieh, explicando la denominaci\'on de
    este    identidad~\cite{AndLar94}   o~\cite[p.~59-60]{Ask75}.    Se   prueba
    escribiendo $(1+x)^{r+s} = (1+x)^r (1+x)^s$  y desorrando con la formula del
    binomio       cada      potencia.}        $\displaystyle      \sum_{t=0}^{l}
  \bino{r}{t}\bino{s}{l-t}   =  \bino{r+s}{l}$~\cite[Ec.~(21),  p.~59]{Knu97_v1}
  o~\cite[Ec.~0.156]{GraRyz15}.
\end{proof}

De este lema, aplicado de manera recursiva, se obtiene el corolario siguiente:
%
\begin{corolario}\label{Cor:MP:MarginalMultinomial}
%
  Sea  \ $X  \,  \sim \,  \H\M(n,k,m)$, entonces  \  $\displaystyle X_i  \, \sim  \,
  \H(n,k_i,m)$.
\end{corolario}

% Cuando  $n  = 1$,  se  recupera  la lei  de  Bernoulli  $\B(p) \equiv  \B(1,p)$.
% Ad\'emas, se muestra  sencillamente usando la generadora de  probabilidad que
% %
% De este resultado,  se puede notar que, por  ejemplo, le distribuci\'on binomial
% aparece en el conteo de eventos independientes de misma probabilidad entre $n$.

% Tambi\'en,  la ley binomial  tiene una  propiedad de  reflexividad, consecuencia
% directa de la de Bernoulli:
% %
% \begin{lema}[Reflexividad]
% \label{Lem:MP:ReflexividadBinomial}
% %
%   Sea \ $X \, \sim \, \B(n,p)$. Entonces
%   %
%   \[
%   n-X \, \sim \, \B(n,1-p)
%   \]
%   %
% \end{lema}

% Nota que cuando $p = 0$ (resp. $p = 1$) la variable es cierta $X = 0$ (resp.  $X
% = n$).


\

Nota: esta  ley se  generaliza de  la misma manera  que para  la hipergeometrica
negativa,  dando una  ley  hipergeometrica negativa  multivariada  o, de  manera
equivalente, generalizando la hipergeometrica negativa  a m\'as de dos clases se
obtiene la ley hipergeometrica negativa. \SZ{Anadirlo en una seccion?}
