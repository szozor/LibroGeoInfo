\subsubseccion{Ley de Poisson}
\label{Sssec:MP:Poisson}

Esta  ley fue  introducida por  Poisson en  1837 como  caso l\'imite  de  la ley
binomial     para     $n$     grande,      con     el     producto     $n     p$
fijo~\cite[Cap.~3]{Poi37},~\cite{Hal90, DavEdw01}.  Se  interes\'o Poisson en su
estudio  al  comportamentio  probabil\'istico   del  conteo  de  experiencia  de
Bernoulli bajo la hipotesis de independencia  (dando lugar a la ley binomial) en
ciencia humana, para una poblaci\'on  importante ($n$ grande), pero con un valor
promedio dado.  De hecho, se conoc\'ia esta ley, tambi\'en como caso l\'imite de
la  binomial,  por  lo  menos  desde  un  trabajo  de  de  Moivre  unas  decadas
antes~\cite{Moi10}.   Apareci\'o  tambi\'en   m\'as  tarde  en  muchos  procesos
f\'isicos, como el conteo de desintegraci\'on atomica por secundo en un material
radioactivo, o, (aproximadamente) a trav\'es del conteo de part\'iculas que caen
en una peque\~na  superficia, cuanto se tiran part\'iculas  uniformamente en una
grande superficia  en trabajos de  W. S. Gosset~\footnote{Fue connocido  bajo en
  nombre ``Student''; ver nota de pie~\ref{Foot:MP:Student}.}~\cite{Stu07}.

Se denota $X \,  \sim \, \P(\lambda)$ \ con \ $\lambda  \in \Rset_{0,+}$ \ llamada
{\em taza}, y sus caracter\'isticas son las siguientes:

\begin{caracteristicas}
%
Dominio de definici\'on & $\X = \Nset$\\[2mm]
\hline
%
Par\'ametro & $\lambda \in \Rset_{0,+}$\\[2mm]
\hline
%
Distribuci\'on  de  probabilidad   &  $\displaystyle  p_X(x)  =  \frac{\lambda^x
e^{-\lambda}}{x!}$\\[2mm]
\hline
%
Promedio & $ m_X = \lambda$\\[2mm]
\hline
%
Varianza & $\sigma_X^2 = \lambda$\\[2mm]
\hline
%
\modif{Asimetr\'ia} & $\displaystyle \gamma_X = \frac1{\sqrt\lambda}$\\[2mm]
\hline
%
Curtosis por exceso & $\displaystyle \widebar{\kappa}_X = \frac1\lambda$\\[2mm]
\hline
%
Generadora de probabilidad & $\displaystyle G_X(z) = e^{\lambda (z-1)}$ \quad para \
$z \in \Cset$\\[2mm]
\hline
%
Generadora  de momentos  & $\displaystyle  M_X(u) =  e^{\lambda \left(  e^u  - 1
\right)}$ \quad para \ $u \in \Cset$\\[2mm]
\hline
%
Funci\'on  caracter\'istica  &  $\displaystyle  \Phi_X(\omega) =  e^{\lambda  \,
\left( e^{\imath \omega} - 1 \right)}$
\end{caracteristicas}

% Momentos & $ \Esp\left[ X^k \right] = ?$\\[2mm]
% Momento factorial & $\Esp\left[ (X)_k \right] = \lambda^k$\\[2mm]
% modo \lfloor \lambda \rfloor 
% Mediana \approx \lfloor \lambda +1/3-0.02/\lambda \rfloor 
% CDF {\frac {\Gamma
% (\lfloor k+1\rfloor  ,\lambda )}{\lfloor k\rfloor !}} where  $\Gamma (x,y)$ is
% the upper incomplete gamma function,

Su masa  de probabilidad  y funci\'on de  repartici\'on son representadas  en la
figura Fig.~\ref{Fig:MP:Poisson}.
%
\begin{figure}[h!]
\begin{center} \begin{tikzpicture}%[scale=.9]
\shorthandoff{>}
%
\pgfmathsetmacro{\sx}{.75};% x-scaling
\pgfmathsetmacro{\r}{.05};% radius arc non continuity F_X
\pgfmathsetmacro{\l}{3};% lambda
\pgfmathsetmacro{\n}{7};% k mas grande del plot (k in Nset)
\pgfmathsetmacro{\q}{floor(\l)};% modo
\pgfmathsetmacro{\m}{(\l^\q)*exp(-\l)/factorial(\q)};% maximo
%
% masa
\begin{scope}
%
\pgfmathsetmacro{\sy}{2.75/\m};% y-scaling 
\draw[>=stealth,->] (-.25,0)--({\sx*\n+.75},0) node[right]{\small $x$};
\draw[>=stealth,->] (0,-.1)--(0,{\sy*\m+.25}) node[above]{\small $p_X$};
%
\pgfmathsetmacro{\pr}{exp(-\l)};% probabilidad
%
\foreach \k in {0,...,\n} {
\draw ({\k*\sx},0)--({\k*\sx},-.1) node[below,scale=.7]{\k};
\draw[dotted] ({\k*\sx},0)--({\k*\sx},{\sy*\pr}) node[scale=.7]{$\bullet$};
%
\pgfmathsetmacro{\prl}{\pr*\l/(\k+1)};\global\let\pr\prl;% proba actualizado
}
\draw ({(\n+.5)*\sx},-.2) node[below,scale=.7]{$\ldots$};
\draw ({(\n+.5)*\sx},{(\pr/\l*\n/2*\sy}) node[scale=.7]{$\cdots$};
\draw (0,{exp(-\l)*\sy})--(-.1,{exp(-\l)*\sy}) node[left,scale=.7]{$e^{-\lambda}$};
\draw (0,{\l*exp(-\l)*\sy})--(-.1,{\l*exp(-\l)*\sy}) node[left,scale=.7]{$\lambda e^{-\lambda}$};
\draw (0,{\l*\l*exp(-\l)/2*\sy})--(-.1,{\l*\l*exp(-\l)/2*\sy}) node[left,scale=.7]{$\frac{\lambda^2 e^{-\lambda}}{2}$};
%\draw (-.5,{\l*exp(-\l)/2*\sy}) node[left,scale=.7]{$\vdots$};
%
\end{scope}
%
%
% reparticion
\begin{scope}[xshift=8.5cm]
%
\pgfmathsetmacro{\sy}{2.75};% y-scaling 
%
\draw[>=stealth,->] (-.6,0)--({\sx*\n+.75},0) node[right]{\small $x$};
\draw[>=stealth,->] (0,-.1)--(0,{\sy+.25}) node[above]{\small $F_X$};
%
\pgfmathsetmacro{\pr}{exp(-\l)};% probabilidad
\pgfmathsetmacro{\c}{exp(-\l)};% cumulativa
%
% cumulativa x < 0
\draw (0,0)--(0,-.1) node[below,scale=.7]{0};
\draw[thick] (-.5,0)--(0,0);
\draw (\r,\r) arc (90:270:\r);
%
% cumulativa x de 0 a n
\foreach \k in {1,...,\n} {
\draw ({\k*\sx},0)--({\k*\sx},-.1) node[below,scale=.7]{\k};
\draw[thick]({(\k-1)*\sx},{\sy*\c}) node[scale=.7]{$\bullet$}--({\k*\sx},{\sy*\c});
\draw ({\k*\sx+\r},{\sy*\c+\r}) arc (90:270:\r);
\draw[dotted] ({(\k-1)*\sx},{(\c-\pr)*\sy})--({(\k-1)*\sx},{\c*\sy});
%
\pgfmathsetmacro{\prl}{\pr*\l/\k};\global\let\pr\prl;% proba actualizado
\pgfmathsetmacro{\cl}{\c+\pr};\global\let\c\cl;% cumulativa actualizada
}
%
% cumulativa x > n
\draw ({(\n+.5)*\sx},-.2) node[below,scale=.7]{$\ldots$};
\draw ({(\n+.5)*\sx},{((\c+1)/2*\sy}) node[scale=.7]{$\cdots$};
\draw (0,{exp(-\l)*\sy})--(-.1,{exp(-\l)*\sy}) node[left,scale=.7]{$e^{-\lambda}$};
\draw (0,{(1+\l)*exp(-\l)*\sy})--(-.1,{(1+\l)*exp(-\l)*\sy}) node[left,scale=.7]{$(1+\lambda) e^{-\lambda}$};
\draw (-.3,{(1+(1+\l+\l*\l/2)*exp(-\l))/2*\sy}) node[left,scale=.7]{$\vdots$};
\draw (0,\sy)--(-.1,\sy) node[left,scale=.7]{\small $1$};
\end{scope}
%
\end{tikzpicture} \end{center}
%
\leyenda{Ilustraci\'on de  una distribuci\'on de probabilidad de  Poisson (a), y
  la funci\'on de repartici\'on asociada (b), con $\lambda = 3$.}
\label{Fig:MP:Poisson}
\end{figure}

\SZ{Otras ilustraciones para otros $\lambda$?}

Ad\'emas, se muestra  sencillamente usando la generadora de  probabilidad que
%
\begin{lema}[Stabilidad]
\label{Lem:MP:StabilidadPoisson}
%
  Sean  \  $X_i  \,  \sim  \,  \P(\lambda_i),  \quad  i  =  1,  \ldots  ,  n$  \
  independientes, entonces
  %
  \[
  \sum_{i=1}^n X_i \, \sim \, \P\left( \sum_{i=1}^n \lambda_i \right)
  \]
\end{lema}


Como lo hemos introducido, la ley de Poisson esta v\'inculada a la ley binomial, como caso l\'imite:
%
\begin{lema}[V\'inculo con la ley binomial]
\label{Lem:MP:VinvuloPoissonBinomial}
%
  Sean  \  $X_n  \,  \sim  \,  \B\left( n \, , \, \frac{\lambda}{n} \right)$  \
  con $\lambda > 0$ fijo, entonces
  %
  \[
  X_n \, \limitd{n \to \infty} \, X \, \sim \, \P(\lambda)
  \]
  %
  donde  \ $\limitd{}$ \  significa que  el l\'imite  es en  distribuci\'on (ver
  notaciones).
\end{lema}
\begin{proof}
  Se  sale  de la  forma  de  la distribuci\'on  binomial  y  de  la formula  de
  Stirling~\footnote{De hecho, esta  formula es probablemente debida previamente
    a  A.  De  Moivre~\cite{Moi33, Moi56,  Pea24,  Cam86, Dut91,  Dem33}, y  fue
    mejorada por  Stirling m\'as tarde. Fue  mejorada a\'un m\'as  por el famoso
    matem\'atico                          S.                           Ramanujan
    recientemente~\cite[\S~4.1]{AndBer13}.\label{Foot:MP:Stirling}}:            \
  $\log\Gamma(z) = \left( z - \frac12 \right) \log z - z + \frac12 \log(2 \pi) +
  o(1)$ \ en \ $z \to +\infty$~\cite{Sti30, AbrSte70, GraRyz15}.
\end{proof}

Aparece  que la  ley de  Poisson esta  v\'inculada tambi\'en  a la  ley binomial
negativa, tambi\'en como caso l\'imite:
%
\begin{lema}[V\'inculo con la binomial negativa]
\label{Lem:MP:VinvuloPoissonBinomialNegativa}
%
Sean \ $X_r \, \sim \, \B_-\left( \frac{\lambda}{r+\lambda} \, , \, r \right)$ \
con $\lambda > 0$ fijo, entonces
  %
  \[
  X_r \, \limitd{r \to \infty} \, X \, \sim \, \P(\lambda)
  \]
\end{lema}
\begin{proof}
  Se  sale de nuevo  la forma  de la  distribuci\'on binomial  negativa y  de la
  formula de Stirling para probarlo.
\end{proof}

M\'as all\'a  del contexto discreto, esta  ley esta tambi\'en  v\'inculada a ley
exponencial, por  el processo dicho de  Poisson.  Si eventos  pueden aparecer de
manera aleatoria  en el tiempo tal que,  entre dos eventos, el  tiempo sigue una
ley   exponencial  de   par\'ametro   $\lambda$,  y   que   estos  tiempos   son
independientes, entonces dado  un intervalo $T$ de tiempo,  el n\'umero de estos
eventos sigue una ley de Poisson de  par\'ametro $\lambda T$.  Lo vamos a ver en
el ejemplo de la ley exponencial m\'as adelante.

Al final, notar que  cuando $\lambda = 0$ la variable es  cierta $X = 0$ (usando
la convenci\'on $0^0 = 1$).
