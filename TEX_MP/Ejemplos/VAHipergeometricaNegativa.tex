\subsubseccion{Ley hipergeometrica Negativa}
\label{Sssec:MP:HipergeometricaNegativa}

Esta ley aparece  por ejemplo cuando se hace una experiencia  del mismo tipo que
para la hipergeometrica, con una poblaci\'on de tama\~no \ $n$ \ (ej.  $n$ bolas
en  una urna),  que pueden  partenecer a  dos clases,  con \  $k$ \  num\'ero de
elementos de la primera clase estados  de excito; ej. $k$ \ bolas negras), $n-k$
\ num\'ero  de elementos de la  secunda clase.  Pero en  lugar de hacer  \ $m$ \
tiros  fijos,  se  hace tiros  hasta  que  $r$  elementos  de la  seconda  clase
(fracascos)  sean tiradas.   $X$ es  el n\'umero  de tiros  parteneciendo  en la
primera clase  (n\'umero de excitos). Es decir  que cuando $X =  x$, tenemos $k$
elementos de  la primera clase en  los ``primeros'' $x+r-1$  tiros, el \'ultimos
parteneciendo a la seconda clase.  Parece que se encuentran las primeras huellas
de esta ley en trabajos del marquesano de Condorcet en 1785~\cite{Con85}.

Se denota \ $X \,  \sim \, \H_-(n,k,r)$ \ con \ $n \in \Nset^*$,  \quad $k \in \{ 0 \;
\ldots \; n \}$, \quad $m \in \{  0 \; \ldots \; n-k \}$ \ y sus caracter\'isticas
son las siguientes:

\begin{caracteristicas}
%
Dominio de definici\'on & $\X = \left\{ 0 \; \ldots \; k \right\}$\\[2mm]
\hline
%
Parametros & $n \in \Nset^*$ \: (poblaci\'on)\newline $k \in \{ 0 \; \ldots \;
n\}$ \ (n\'umero de estados exitosos)\newline $r \in \{ 0 \; \ldots \; n-k\}$ \:
(n\'umero de fracascos para parar)\\[2mm]
\hline
%
Distribuci\'on de probabilidad~\footnote{Para los $x+r-1$ primeros tiros, de la
primera clase hay $\smallbino{k}{x}$ combinaciones posibles, y
$\smallbino{n-k}{r-1}$ de la seconda clase, sobre los $\smallbino{n}{x+r-1}$
combinaciones posibles en total. Para el \'ultimo tiro, quedan $n-k-(r-1)$
posibilidades de la seconda clase sobre las $n-x-(r-1)$ elementos que quedan.} &
\protect$p_X(x) = \left\{ \begin{array}{ccc} \frac{\smallbino{x+r-1}{x}
\smallbino{n-r-x}{k-x}}{\smallbino{n}{k}} & \mbox{si} & r > 0\\ \un_{\{0\}}(x) & \mbox{si} & r = 0 \end{array} \right.$\protect\\[2mm]
\hline
%
Promedio & $\displaystyle m_X = \frac{r \, k}{n - k + 1}$\\[2mm]
\hline
%
Varianza & $\displaystyle \sigma_X^2 = \frac{r \, k (n+1) (n-k-r+1)}{(n-k+1)^2 (n-k+2)}$\\[2mm]
\hline
%%
%\modif{Sesgo} & \SZ{$\gamma_X =  $}\\[2mm]
%\hline
%%
%Curtosis por exceso & $\displaystyle \SZ{\widebar{\kappa}_X = ...}$\\[2mm]
%\hline
%
Generadora de probabilidad & $G_X(z) = \frac{\smallbino{n-r}{k}}{\smallbino{n}{k}} \:
\: \hypgeom{2}{1}(r , -k ; r-n ; z)$ \ sobre \ $\Cset$\\[2mm]
\hline
%
Generadora de momentos & $M_X(u) = \frac{\smallbino{n-r}{k}}{\smallbino{n}{k}} \:
\: \hypgeom{2}{1}\left(r , -k ; r-n ; e^u \right)$ \ sobre \ $\Cset$\\[2mm]
\hline
%
Funci\'on caracter\'istica  & $\Phi_X(\omega) =  \frac{\smallbino{n-r}{k}}{\smallbino{n}{k}} \:
\: \hypgeom{2}{1}\left(r , -k ; r-n ; e^{\imath \, \omega} \right)$
\end{caracteristicas}

\SZ{Poner sesgo y curtosis? Expresiones  muy pesadas... Momento factorial $f_q =
  \frac{\PocC{r}{q} \PocD{k}{q}}{\PocC{n-k+1}{q}}$ permitiendo calcular todo.}

Su masa  de probabilidad  y funci\'on de  repartici\'on son representadas  en la
figura Fig.~\ref{Fig:MP:HipergeometricaNegativa}.
%
\begin{figure}[h!]
\begin{center} \begin{tikzpicture}[fixed point arithmetic]%[scale=.9]
\shorthandoff{>}
%
\pgfmathsetmacro{\sx}{.375};% x-scaling
\pgfmathsetmacro{\r}{.05};% radius arc non continuity F_X
%\pgfmathsetmacro{\p}{1/3};% probabilidad p
\pgfmathsetmacro{\n}{100};% numero n de la poblacion
\pgfmathsetmacro{\k}{12};% numero k de estados exitosos
\pgfmathsetmacro{\rr}{40};% numero de rechazos para parar
%
% primera proba (n-k) (n-k-1) ... (n-k-r+1) / n (n-1) ... (n-r+1)
% debhiperneg(\n,\k,\rr)
\tikzmath{function debhiperneg(\a,\b,\c) {
    if \c == 0 then {return 1;}
    else {return ((\a-\b)/\a)*debhiperneg(\a-1,\b,\c-1);};
};};
%

\pgfmathsetmacro{\ui}{2}; % numeros de indices finales nulos dibujados
%
% masa
\begin{scope}
%
\pgfmathsetmacro{\sy}{10};% y-scaling
%
% proba nulas del principio 0 -> d-1
\foreach \y in {-2,...,-1} {
\pgfmathsetmacro{\xl}{int(\y)};\global\let\x\xl;
\draw ({\sx*\x},0)--({\sx*\x},-.1) node[below,scale=.7]{$\x$};
\draw ({\sx*\x},0) node[scale=.6]{$\bullet$};
}
%
% proba nulas del fin f+1 -> ui
\foreach \y in {1,...,\ui} {
\pgfmathsetmacro{\xl}{int(\k+\y)};\global\let\x\xl;
\draw ({\sx*\x},0)--({\sx*\x},-.1) node[below,scale=.7]{$\x$};
\draw ({\sx*\x},0) node[scale=.6]{$\bullet$};
}
%
\pgfmathsetmacro{\pr}{debhiperneg(\n,\k,\rr)};% valor de la primer proba no nula
\pgfmathsetmacro{\maxp}{\pr};% proba maximal (inicializacion)
%
\foreach \x in {0,...,\k} {
%\pgfmathsetmacro{\xl}{int(\f-\y)};\global\let\x\xl;
\draw ({\sx*\x},0)--({\sx*\x},-.1) node[below,scale=.7]{$\x$};
\draw[dotted] ({\sx*\x},0)--({\sx*\x},{\sy*\pr}) node[scale=.6]{$\bullet$};
%
\pgfmathsetmacro{\prl}{\pr*(\x+\rr)*(\k-\x)/((\x+1)*(\n-\rr-\x))};\global\let\pr\prl;% proba actualizado
\pgfmathsetmacro{\maxpl}{(abs(\pr-\maxp)+\pr+\maxp)/2};\global\let\maxp\maxpl;% proba max actualizado
}
%
\draw[>=stealth,->] ({-2*\sx-.25},0)--({\sx*(\k+\ui)+.35},0) node[right]{\small $x$};
\draw[>=stealth,->] (0,-.15)--(0,{\sy*\maxp+.25}) node[above]{\small $p_X$};
%\draw (0,{((1-\p)^\n)*\sy})--(-.1,{((1-\p)^\n)*\sy}) node[left,scale=.7]{$(1-p)^n$};
%\draw (0,{\n*\p*((1-\p)^(\n-1))*\sy})--(-.1,{\n*\p*((1-\p)^(\n-1))*\sy}) node[left,scale=.7]{$n \, p \, (1-p)^{n-1}$};
%
\node at ({(\sx*(\ui+\k)+.25)/2},-1) [scale=.9]{(a)};
\end{scope}
%
%
% reparticion
\begin{scope}[xshift=8.25cm]
%
\pgfmathsetmacro{\sy}{2.5};% y-scaling 
%
\draw[>=stealth,->] ({-2*\sx-.25},0)--({\sx*(\k+\ui)+.5},0) node[right]{\small $x$};
\draw[>=stealth,->] (0,-.15)--(0,{\sy+.25}) node[above]{\small $F_X$};
%
% proba nulas del principio 0 -> d-1
\foreach \y in {-2,...,1} {
\pgfmathsetmacro{\xl}{int(\y)};\global\let\x\xl;
\draw ({\sx*\x},0)--({\sx*\x},-.1) node[below,scale=.7]{$\x$};
}
\draw ({-2*\sx},0)--(0,0);
%
% proba nulas del fin f+1 -> ui
\foreach \y in {0,...,\ui} {
\pgfmathsetmacro{\xl}{int(\y+\k)};\global\let\x\xl;
\draw ({\sx*\x},0)--({\sx*\x},-.1) node[below,scale=.7]{$\x$};
}
\draw ({\sx*\k},\sy) node[scale=.6]{$\bullet$} --({\sx*(\k+\ui)},\sy);
%
\pgfmathsetmacro{\pr}{debhiperneg(\n,\k,\rr)};% valor de la primera proba no nula
\pgfmathsetmacro{\cum}{\pr};% valor inicial de la cumulativa
%
\pgfmathsetmacro{\fk}{\k-1}
\foreach \x in {0,...,\fk} {
\draw ({\sx*\x},0)--({\sx*\x},-.1) node[below,scale=.7]{$\x$};
\draw ({\sx*\x},{\sy*\cum}) node[scale=.6]{$\bullet$} --({\sx*(\x+1)},{\sy*\cum});
\draw ({\sx*(\x+1)+\r},{\sy*\cum+\r}) arc (90:270:\r);
\draw[dotted] ({\sx*\x},{\sy*(\cum-\pr)})--({\sx*\x},{\sy*\cum});
%
\pgfmathsetmacro{\prl}{\pr*(\x+\rr)*(\k-\x)/((\x+1)*(\n-\rr-\x))};\global\let\pr\prl;% proba actualizado
\pgfmathsetmacro{\cuml}{\cum+\pr}\global\let\cum\cuml;% cumulativa actualizada
}
\draw (0,\sy)--(-.1,\sy) node[left,scale=.7]{$1$};
%\draw (0,{((1-\p)^\n)*\sy})--(-.1,{((1-\p)^\n)*\sy}) node[left,scale=.7]{$(1-p)^n$};
%\draw (0,{\n*\p*((1-\p)^(\n-1))*\sy})--(-.1,{\n*\p*((1-\p)^(\n-1))*\sy}) node[left,scale=.7]{$n \, p \, (1-p)^{n-1}$};
%
\node at ({(\sx*(\ui+\k)+.5)/2},-1) [scale=.9]{(b)};
\end{scope}
%
\end{tikzpicture} \end{center}
%
\leyenda{Ilustraci\'on  de una  distribuci\'on  de probabilidad  Hipergeometrica
  (a), y la funci\'on de repartici\'on asociada (b), con \ $n = 100$, \quad $k =
  12$, \quad $r = 40$.}
\label{Fig:MP:HipergeometricaNegativa}
\end{figure}

\SZ{Otros ilustraciones para otros $n, k, r$?}

Notar: cuando  $k =  0$, la  variable es cierta  $X =  r$ (se  sortean solamente
elementos de la seconda clase, as\'i  que para siempre cuando se han tirados $r$
elementos); cuando  $r =  0$, tambi\'en  la variable es  cierta $X  = 0$  (no se
sortan bolas, as\'i que no hay de la primera clase).

% {\displaystyle NHG_{N,K,r}(k)=1-HG_{N,N-K,k}(r-1)}