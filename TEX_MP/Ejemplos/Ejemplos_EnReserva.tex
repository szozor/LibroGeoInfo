
% % --------------------------------- uniforme producto cartesiano

% \subsubseccion{Distribuci\'on uniforme sobre un producto cartesiano de intervalos}

% Se denota $X  \, \sim \, \U(\D)$ \  con, en este caso, \  $\D = \optimes_{i=1}^d
% [a_i \, ; \, b_i] \subset \Rset^d$.   De nuevo, se puede escribir \ $X \, \egald
% \,  a +  \diag(b-a)  U$ \  donde  \ $a  =  \begin{bmatrix} a_1  &  \cdots &  a_d
% \end{bmatrix}^d$, \ $b = \begin{bmatrix} b_1 & \cdots & b_d \end{bmatrix}^d$ \ y
% \ $U \, \sim \, \U \left( [ 0  \, ; \, 1 ]^d \right)$ \ {\em uniforme estandar}.
% Las caracteristicas de  \ $X \, \sim \, \U  \left( [ 0 \, ; \,  1 ]^d \right)$ \
% son  las  siguientes  (se  deducen   para  cualquier  uniforme  sobre  $\D$  por
% transformaci\'on lineal; ver secciones anteriores):

% \begin{caracteristicas}
% %
% Dominio de definici\'on & $\D = [0 \, ; \, 1 ]^d$\\[2mm]
% \hline
% %
% Densidad de probabilidad & $p_X(x) = 1$\\[2mm]
% \hline
% %
% %Momentos & $ \Esp\left[ X^k \right] = p^k$\\[2mm]
% %\hline
% %
% Promedio & $\displaystyle m_X = \frac12 \begin{bmatrix} 1 & \cdots & 1
% \end{bmatrix}^t$\\[2mm]
% \hline
% %
% Covarianza & $\displaystyle \Sigma_X = \frac1{12} \, I$\\[2mm]
% \hline
% %
% %Generadora de probabilidad & $G_X(z) = e^{\lambda (z-1)}$ \ para \ $z \in \Cset$\\[2mm]
% %\hline
% %
% Generadora de momentos & $\displaystyle  M_X(u) = \prod_{k=1}^d \frac{ e^{u_k} -
% 1}{u_k}$ \  para~\footnote{Con el  l\'imite $\lim_{u_k \to  0} \frac{  e^{u_k} -
% 1}{u_k} = 1$.}  \ $u \in \Cset^d$\\[2mm]
% \hline
% %
% Funci\'on  caracter\'istica   &  $\displaystyle  \Phi_X(\omega)   =  (-\imath)^d
% \prod_{k=1}^d \frac{ e^{\imath \omega_k} - 1}{\omega_k}$
% \end{caracteristicas}
% %
% % modo 0
% % Mediana \ln(2)/\lambda
% % CDF 1-e^{-\lambda x}

% \SZ{Hacer el caso $d = 2$}

% %Cuando $\lambda \to +\infty$ la variable tiende a una variable cierta $X = 0$.
% \SZ{Esta distribuci\'on aparece..., propiedades}
% % en el conteo de
% %conteo de une repetici\'on de  una experiencia de maneja independiente hasta que
% %occure un evento de probabilidad $p$; por ejemplo el n\'umero de tiro de un dado
% %equilibriado hasta que occurre un ``6'' sigue una ley geometrica de parametro $p
% %= \frac16$.


\SZ{
% --------------------------------- Laplace
\subsubseccion{Distribuci\'on de Laplace o doble exponencial}
\label{Sssec:MP:Laplace}
}

% --------------------------------- Dirichlet matriz-variada
\subsubseccion{Distribuci\'on de Dirichlet matriz-variada}
\label{Sssec:MP:DirichletMatrizVariada}

Este  ejemplo es  una generalizaci\'on  matriz-variada de  la  distribuci\'on de
Dirichlet, la distribuci\'on beta matriz  variada siendo un caso particular. Tal
como en el caso de la distribuci\'on de Wishart, se puede ver una matriz como un
vector,  guardando  por ejemplo  sus  columnas  una  bajo la  precediente.   Sin
embargo, tal distribuci\'on apareciendo \SZ{ naturalmente
%en un contexto de estimaci\'on de  matriz de covarianza (ver m\'as adelante), es
%m\'as  natural  verla  matriz-variate.    
\SZ{Tal  distrubuci\'on  es  debido  a
%  J.
%Wishart~\cite{Wis28, GupNag99, And03
}, y se denota \ $X \, \sim \, \Dir_d(a)$ \ donde \ $a \in \Rset_+^{* \, k}$ \ y
el dominio  de definici\'on es  \ $\P_{d,k}^+(\Rset)$, conjunto de  $k$-uplet de
matrices simetricas definida positivas que satisfacen la relaci\'on de completud
(ver  notaciones).    Las  caracter\'isticas   de  la  distribuci\'on   son  las
siguientes~\cite{GupNag99}:

\begin{caracteristicas}
%
Dominio de definici\'on
%~\footnote{De hecho, se puede considerar que el vector
%aleatorio es \ $(k-1)$-dimensional \ $\widetilde{X} = \begin{bmatrix}
%\widetilde{X}_1 & \cdots & \widetilde{X}_{k-1} \end{bmatrix}^t$ \ definido sobre
%el hipertriangulo \ $\widetilde{\X} = \Tri_{k-1} = \left\{ \widetilde{x} \in [0
%\; 1]^{k-1}, \: \sum_{i=1}^{k-1} \widetilde{x}_i \le 1 \right\}$, proyecci\'on
%del simplex sobre el hiperplano \ $x_k = 0$.\label{Foot:MP:DirichletXtilde}} 
& $\X = \P_{d,k}^+(\Rset)$}\\[2mm]
\hline
%
Parametros & $a = \alpha \, \bar{a} \, \in \, \Rset_+^{* \, k}$ \ (forma) \ con
\ $\alpha \in \Rset_+^*$ \ (concentraci\'on) y \ $\bar{a} \in \Simp{k-1}$
(medida de base)\\[2mm]
\hline
%
Densidad de probabilidad
%~\footnote{La densidad de probabilidad es dada con
%respeto a la medida de Lebesgue restricta al simplex \ $\Simp{k-1}$. Tratando
%de \ $\widetilde{X}$, el vector tiene una densidad con respeto a la medida de
%Lebesgue usual, y es dada por \ $p_{\widetilde{X}}\left( \widetilde{x} \right) =
%\frac{\prod_{i=1}^{k-1} \widetilde{x}_i^{\, a_i-1} \, \left( 1 -
%\sum_{i=1}^{k-1} \widetilde{x}_i
%\right)^{a_k-1}}{B(a)}$.\label{Foot:MP:DirichletDensidad}} 
& $\displaystyle
p_X(x) = \frac{\prod_{i=1}^k \left| x_i \right|^{a_i-\frac{d+1}{2}}}{B_d(a)}$\\[2mm]
\hline
%%
%%Momentos & $ \Esp\left[ X^k \right] = p^k$\\[2mm]
%%\hline
%%
%%Momento factorial & $\Esp\left[ (X)_k \right] = ?$\\[2mm]
%%\hline
%%
%Promedio & $\displaystyle m_X = \bar{a}$\\[2.5mm]
%\frac{a}{\sum_{i=1}^k a_i} \equiv \overline{a}$\\[2.5mm]
%\hline
%
%Covarianza & $\displaystyle \Sigma_X = \frac{\diag\left( \bar{a} \right) -
%\bar{a} \bar{a}^t}{1 + \alpha}$\\[2.5mm]
%%\sum_{i=1}^k a_k}$\\[2.5mm]
%\hline
%%
%Asimetr\'ia & $\displaystyle \gamma_X = \frac{2 \, (b - a) \sqrt{a + b + 1}}{( a
%+ b + 2) \sqrt{a b}}$\\[2mm]
%\hline
%%
%Curtosis por exceso & $\displaystyle \widebar{\kappa}_X = \frac{6 \, \left( (a - b)^2 (a + b + 1) - a
%b (a  + b  + 2)  \right)}{a \, b  \left( a  + b  + 2 \right)  \left( a  + b  + 3
%\right)}$\\[2mm]
%\hline
%%
%Generadora de probabilidad & $G_X(z) = e^{\lambda (z-1)}$ \ para \ $z \in \Cset$\\[2mm]
%\hline
%
%Generadora de momentos~\footnote{El v\'inculo entre las funciones generadoras de
%momento de \ $X$ \ y \ $\widetilde{X}$ \ es trivialmente \
%$M_{\widetilde{X}}\left( \widetilde{u} \right) = M_X\left( \begin{bmatrix}
%\widetilde{u} & 0 \end{bmatrix}^t \right)$ \ o \ $M_X(u) = e^{\imath u_k}
%M_{\widetilde{X}}\left( \begin{bmatrix} u_1 - u_k & \cdots & u_{k-1} -
%u_k \end{bmatrix}^t \right)$, y similarmente para la funci\'on
%caracter\'istica.\label{Foot:MP:Dirichlet}} & $\displaystyle M_X(u) =
%\Phi_2^{(k)}( a , \alpha \, ; \, u )$ \ para \ $u \in \Cset$\\[2mm]
%\hline
%%%
%Funci\'on caracter\'istica~\footnote{La forma de la funci\'on generadora de
%momento viene directamente de la escritura de las series de Taylor de $e^{u_i
%x_i}$ \ o de la forma integral de la funci\'on confluente
%hipergeom\'etrica~\cite{Phi88}.} & $\displaystyle \Phi_X(\omega) = \Phi_2^{(k)}(
%a , \alpha \, ; \, \imath \omega )$
\end{caracteristicas}

%\footnotetext{De  hecho,  se puede  considerar  que  el  vector aleatorio  es  \
%  $(k-1)$-dimensional  \  $\widetilde{X}  =  \begin{bmatrix}  \widetilde{X}_1  &
%    \cdots   &  \widetilde{X}_{k-1}   \end{bmatrix}^t$  \   definido   sobre  el
%  hipertriangulo \  $\widetilde{\X} = \Tri_{k-1} = \left\{  \widetilde{x} \in [0
%    \;   1]^{k-1},  \:   \sum_{i=1}^{k-1}  \widetilde{x}_i   \le   1  \right\}$,
%  proyecci\'on del  simplex sobre  el hiperplano  \ $x_k =  0$.  Su  densidad de
%  probabilidad,  con   respeto  a  la  medida   de  Lebesgue,  es   dada  por  \
%  $p_{\widetilde{X}}\left(   \widetilde{x}  \right)   =  \frac{\prod_{i=1}^{k-1}
%    \widetilde{x}_i^{\,  a_i-1} \, \left(  1 -  \sum_{i=1}^{k-1} \widetilde{x}_i
%    \right)^{a_k-1}}{B(a)}$.   El v\'inculo entre  las funciones  generadoras de
%  momento  es trivialmente  \ $M_{\widetilde{X}}\left(  \widetilde{u}  \right) =
%  M_X\left(  \begin{bmatrix} \widetilde{u} &  0 \end{bmatrix}^t  \right)$ \  o \
%  $M_X(u) =  e^{\imath u_k} M_{\widetilde{X}}\left( \begin{bmatrix} u_1  - u_k &
%      \cdots &  u_{k-1} - u_k  \end{bmatrix}^t \right)$, y similarmente  para la
%  funci\'on caracter\'istica.}
%%
%(ver~\cite{PedRic91, SulTra96, And03}).

%Fijense que $p_X$ no es la distribuci\'on conjuntos de los componentes de \ $X$:
%el hecho de  que \ $X$ \ sea  uan matriz aleatoria de \  $P_d^+(\Rset)$ \ impone
%v\'inculos sobre sus compnentes; entre otros, $X_{i,j} = X_{j,i}$

%Inmediatamente, si  $d = 1$, la  distribuci\'on de Whishart \  $W_1(V,\nu)$ \ se
%reduce  a la  distribuci\'on Gamma  $\G\left(\frac{\nu}{2} \,  , \,  \frac1{2 V}
%\right)$. De este  hecho, se la podr\'ia ver  como extensi\'on matriz-variada de
%la  distribuci\'on  gamma.  La  distribuci\'on  de  Wishart  tiene varias  otras
%propiedades como las siguientes.
%%
%\begin{lema}[Stabilidad por transformaci\'on lineal]
%\label{Lem:MP:StabilidadWishartLineal}
%%%
%  Sea $X \,  \sim \, W_d(V,\nu)$ \ y \  $A \in \Rset^{d \times d'}$  \ con \ $d'
%  \le d$ \ y de rango lleno. Entonces
%  \[
%  A^t X A \, \sim \, W_{d'}\left( A^t V A , \nu \right)
%  \]
%  %
%  En particular, si $d'  = 1$, \ $A^t X A \, \sim  \, G\left( \frac{\nu}{2} \, ,
%    \, \frac1{2 \, A^t V A} \right)$. M\'as all\'a, tomando $A = \un_j$, aparece
%  de que  las componentes diagonales de \  $X$ \ son de  distribuci\'on gamma, \
%  $X_{j,j}  \, \sim  \,  \G\left( \frac{\nu}{2}  \,  , \,  \frac1{2 \,  V_{j,j}}
%  \right)$.
%\end{lema}
%%%
%\begin{proof}
%  El     resultado     es     inmediato     saliendo     de     la     funci\'on
%  caract\'eristica~\footref{Foot:MP:CaracteristicaWishart} y notando de que
%
%%\begin{eqnarray*}
%\Phi_{A^t X A}(\omega) & = & \Esp\left[ e^{\imath \Tr\left( \omega^t A^t X A
%\right)}\right]\\[2mm]
%
%%& = & \Phi_X\left( A \omega^t A^t\right)\\[2mm]
%%
%& = &  \left| I - 2 \imath A \omega A^t V \right|^{-\frac{\nu}{2}}\\[2mm]
%
%& = &  \left| I - 2 \imath \omega A^t V A \right|^{-\frac{\nu}{2}}
%%
%\end{eqnarray*}
%%%
%de   \   $\Tr(AB)   =   \Tr(BA)$~\cite{Har08}   \   y   de   la   identidad   de
%Sylvester~\cite{Syl51,  AkrAkr96}  o~\cite[\S~18.1]{Har08} \  $\left|  I  + A  B
%\right| = \left| I + B A \right|$.  .
%\end{proof}
%%%
%De hecho, si los elementos diagonales son de distribuci\'on gamma, no es el caso
%de         los        elementos         no-diagonales~\cite{Seb04,        And03}
%o~\cite[Teo.~3.3.4]{GupNag99}.    De    eso   resuelte   delicado    llamar   la
%distribuci\'on como gamma matriz-variada.

%\begin{lema}[Stabilidad por suma]
%\label{Lem:MP:StabilidadWishartSuma}
%%
%  Sea $X_i \,  \sim \, W_d(V,\nu_i), \: i = 1,\ldots,n$ independientes. Entonces
%  \[
%  \sum_{i=1}^n X_i \, \sim \, W_d\left( V \, , \, \sum_{i=1}^n \nu_i \right)
%  \]
%\end{lema}
%%
%\begin{proof}
%  El     resultado     es     inmediato     saliendo     de     la     funci\'on
%  caract\'eristica~\footref{Foot:MP:CaracteristicaWishart} y notando que como el
%  el context vectorial $\Phi_{\sum_i X_i} = \prod_i \Phi_{X_i}$.
%\end{proof}

%La distribuci\'on  de Wishart aparece naturalmente en  problemas de estimaci\'on
%de matriz de covarianza en el contexto gausiano:
%%
%\begin{lema}[V\'inculo con vectores gausianos~\cite{Seb04}]
%  Sean \ $X_i \, \sim \, \N(0,V), \: i = 1, \ldots , n > d-1$ \ independientes y
%  la  matriz  \  $S  =  \sum_{i=1}^n   X_i  X_i^t$  \  llamada  {\em  matriz  de
%    dispersi\'on} (scatter matrix en ingl\'es). Entonces, \ $S \in P_d^+(\Rset)$
%  (c.   s.)   \  ($S$ es  sim\'etrica  definida  positiva  casi siempre,  o  con
%  probabilidad uno) y \ $S \,\sim \, W_d(V,n)$.
%\end{lema}
%%
%Este resultado  permite tambi\'en probar  el lema~\ref{Lem:MP:StabilidadWishart}
%para \ $\nu = n$ \ entero  escribiendo \ $X \egald \sum_{i=1}^n X_i X_i^t$ \ tal
%que \ $A^t X A \egald \sum_{i=1}^n A^t X_i X_i^t A = \frac1n \sum_{i=1}^n \left(
%%  A^t X_i \right)  \left( A^t X_i \right)^t$ \  y notando que los \  $A^t X_i \,
%\sim \,  \N(0, A^t V  A)$ \ son independientes~\cite{Seb04}.   Adem\'as, permite
%re-obtener las  expreciones del promedio y de  las covarianzas~\footnote{Para la
%  covarianza, su usa la formula \ $\Esp[Y_1 Y_2 Y_3 Y_4] = \Esp[Y_1 Y_2]\Esp[Y_3
%  Y_4]  + \Esp[Y_1 Y_3]\Esp[Y_2  Y_4] +  \Esp[Y_1 Y_4]\Esp[Y_2  Y_3]$ \  para $Y
%  = \begin{bmatrix}  Y_1 & Y_2 &  Y_3 & Y_4 \end{bmatrix}^t$  \ vector gausiano,
%  formula que se  obtiene por ejemplo a partir  de la funci\'on caracter\'istica
%  de un vecor  gausiano.}. Notar que cuando los \ $X_i$  \ tienen un promemedio,
%el  lema conduce  a  lo que  es  conocido como  Wishart no  central~\cite{And03,
%  Seb04}.

La distribuci\'on  de Dirichlet matriz-variada aparece as\'i naturalmente en  problema
% de inferencia
%Bayesiana  como distribuci\'on  a  priori conjugado~\footref{Foot:MP:BayesPrior}
%del parametro $p$ de la ley gaussiana multivariada~\cite{Rob07}
Que mas?}