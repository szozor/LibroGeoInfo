\subsubseccion{Ley Binomial}
\label{Sssec:MP:Binomial}

Se denota \ $X \, \sim \, \B(n,p)$ \ con \ $n \in \Nset \setminus \{ 0 \; 1 \}$,
\quad $p \in [0 \; 1]$ \ y sus caracter\'isticas son las siguientes:

\begin{caracteristicas}
%
Dominio de definici\'on & $\X = \{ 0 \; \ldots \; n \}$\\[2mm]
\hline
%
Parametros & $n  \in \Nset \setminus \{0  \; 1 \},  \quad p \in [0  \;
1]$\\[2mm]
\hline
%
Distribuci\'on de probabilidad & \protect$\displaystyle p_X(x) = \bino{n}{x} \, p^x
(1-p)^{n-x}$\protect\\[2mm]
\hline
%
Promedio & $ m_X = n \, p$\\[2mm]
\hline
%
Varianza & $\sigma_X^2 = n \, p \, (1-p)$\\[2mm]
\hline
%
\modif{Sesgo} & $\displaystyle \gamma_X = \frac{1 - 2 \, p}{\sqrt{n \, p \, (1-p)}}$\\[2mm]
\hline
%
Curtosis por exceso & $\displaystyle \widebar{\kappa}_X = \frac{1 - 6 \, p + 6
\, p^2}{n \, p \, (1-p)} $\\[2mm]
\hline
%
Generadora  de probabilidad  &  $\displaystyle  G_X(z) =  \left(  1 -  p  + p  z
\right)^n$ \ sobre \ $\Cset$\\[2mm]
\hline
%
Generadora  de momentos  &  $\displaystyle  M_X(u) =  \left(1  - p  +  p \,  e^u
\right)^n$ \ sobre \ $\Cset$\\[2mm]
\hline
%
Funci\'on caracter\'istica  & $\displaystyle \Phi_X(\omega) =  \left( 1 -  p + p
\, e^{\imath \omega} \right)^n$
\end{caracteristicas}

% Momentos & $ \Esp\left[ X^k \right] = ??\\[2mm]
% Momento factorial & $\Esp\left[ (X)_k \right] = 
% \frac{n!}{(n-k)!} p^k \un_{\{ 0 \, , \, \ldots \, , \, n \}}(k)$\\[2mm]
% Modo $\left\lfloor (n+1) p \right\rfloor$
% Mediana $\left\lfloor n p \right\rfloor$ o $\left\lceil n p \right\rceil
% CDF	$I_{1-p}(n-k,k+1)$ regularized incomplete beta function

Su masa  de probabilidad  y funci\'on de  repartici\'on son representadas  en la
figura Fig.~\ref{Fig:MP:Binomial}.
%
\begin{figure}[h!]
\begin{center} \begin{tikzpicture}%[scale=.9]
\shorthandoff{>}
%
\pgfmathsetmacro{\sx}{.75};% x-scaling
\pgfmathsetmacro{\r}{.05};% radius arc non continuity F_X
\pgfmathsetmacro{\p}{1/3};% probabilidad p
\pgfmathsetmacro{\n}{6};% numero n de la binomial
\pgfmathsetmacro{\q}{floor((\n+1)*\p)};% modo de la binomial
\pgfmathsetmacro{\m}{factorial(\n)/factorial(\q)/factorial(\n-\q)*(\p^\q)*((1-\p)^(\n-\q))};% maximo de la binomial
% masa
\begin{scope}
%
\pgfmathsetmacro{\sy}{2.5/\m};% y-scaling 
\draw[>=stealth,->] (-.25,0)--({\sx*\n+.25},0) node[right]{\small $x$};
\draw[>=stealth,->] (0,-.1)--(0,{\sy*\m+.25}) node[above]{\small $p_X$};
%
\pgfmathsetmacro{\b}{(1-\p)^\n};% coeficiente binomial por la probabilidad
%
\foreach \k in {0,...,\n} {
\draw ({\k*\sx},0)--({\k*\sx},-.1) node[below,scale=.7]{\k};
\draw[dotted] ({\k*\sx},0)--({\k*\sx},{\sy*\b}) node[scale=.7]{$\bullet$};
%
\pgfmathsetmacro{\bl}{\b*\p*(\n-\k)/((\k+1)*(1-\p))};\global\let\b\bl;% proba actualizado
}
\draw (0,{((1-\p)^\n)*\sy})--(-.1,{((1-\p)^\n)*\sy}) node[left,scale=.7]{$(1-p)^n$};
\draw (0,{\n*\p*((1-\p)^(\n-1))*\sy})--(-.1,{\n*\p*((1-\p)^(\n-1))*\sy}) node[left,scale=.7]{$n p (1-p)^{n-1}$};
%
\end{scope}
%
%
% reparticion
\begin{scope}[xshift=8.5cm]
%
\pgfmathsetmacro{\sy}{2.5};% y-scaling 
%
\draw[>=stealth,->] (-.6,0)--({\sx*(\n+.5)+.5},0) node[right]{\small $x$};
\draw[>=stealth,->] (0,-.1)--(0,{\sy+.25}) node[above]{\small $F_X$};
%
\pgfmathsetmacro{\b}{(1-\p)^\n};% coeficiente binomial por la probabilidad
\pgfmathsetmacro{\c}{(1-\p)^\n};% cumulativa binomial por la probabilidad
%
% cumulativa x < 0
\draw (0,0)--(0,-.1) node[below,scale=.7]{0};
\draw[thick] (-.5,0)--(0,0);
\draw (\r,\r) arc (90:270:\r);
%
% cumulativa x de 0 a n-1
\foreach \k in {1,...,\n} {
\draw ({\k*\sx},0)--({\k*\sx},-.1) node[below,scale=.7]{\k};
\draw[thick]({(\k-1)*\sx},{\sy*\c}) node[scale=.7]{$\bullet$}--({\k*\sx},{\sy*\c});
\draw ({\k*\sx+\r},{\sy*\c+\r}) arc (90:270:\r);
\draw[dotted] ({(\k-1)*\sx},{(\c-\b)*\sy})--({(\k-1)*\sx},{\c*\sy});
%
\pgfmathsetmacro{\bl}{\b*\p*(\n-\k+1)/(\k*(1-\p))};\global\let\b\bl;% proba actualizado
\pgfmathsetmacro{\cl}{\c+\b};\global\let\c\cl;% cumulativa actualizada
}
%
% cumulativa x > n
\draw[dotted] ({\n*\sx},{(1-\b)*\sy})--({\n*\sx},\sy);
\draw[thick]({\n*\sx},\sy) node[scale=.7]{$\bullet$}--({(\n+.5)*\sx},\sy);
%
\draw (0,{((1-\p)^\n)*\sy})--(-.1,{((1-\p)^\n)*\sy}) node[left,scale=.7]{$(1-p)^n$};
\draw (0,{(\n*\p+1-\p)*((1-\p)^(\n-1))*\sy})--(-.1,{(\n*\p+1-\p)*((1-\p)^(\n-1))*\sy}) node[left,scale=.7]{$(1-p+np) (1-p)^{n-1}$};
\draw (-.2,{((\n*\p+1-\p)*((1-\p)^(\n-1))+1)/2*\sy}) node[scale=.7]{$\vdots$};
\draw (0,\sy)--(-.1,\sy) node[left,scale=.7]{$1$};
\end{scope}
%
\end{tikzpicture} \end{center}
%
\leyenda{Ilustraci\'on de una distribuci\'on  de probabilidad Binomial (a), y la
  funci\'on de repartici\'on asociada (b), con $n = 6$, \quad $p = \frac13$.}
\label{Fig:MP:Binomial}
\end{figure}

\SZ{Otros ilustraciones para otros $p$?}

Cuando  $n  = 1$,  se  recupera  la lei  de  Bernoulli  $\B(p) \equiv  \B(1,p)$.
Ad\'emas, se muestra  sencillamente usando la generadora de  probabilidad que
%
\begin{lema}
\label{Lem:BinomialSumaBernoulli}
%
  Sean \  $X_i \,  \sim \, \B(p),  \quad i  = 1, \ldots  , n$  \ independientes,
  entonces
  %
  \[
  \sum_{i=1}^n X_i \, \sim \, \B(n,p)
  \]
\end{lema}
%
De este resultado,  se puede notar que, por  ejemplo, le distribuci\'on binomial
aparece en el conteo de eventos independientes de misma probabilidad entre $n$.

Tambi\'en,  la ley binomial  tiene una  propiedad de  reflexividad, consecuencia
directa de la de Bernoulli:
%
\begin{lema}[Reflexividad]
\label{Lem:MP:ReflexividadBinomial}
%
  Sea \ $X \, \sim \, \B(n,p)$. Entonces
  %
  \[
  n-X \, \sim \, \B(n,1-p)
  \]
  %
\end{lema}
%
Si  tomamos  el  ejemplo  de  una  moneda  que se  tira  $n$  veces  de  maneras
independientes, con  probabilidad $p$ que  aparezca una cara, $X$  representa el
n\'umero de caras tiradas. Entonces, $n-X$  es el n\'umero de secas: en $n-X$ se
intercambian los roles de la cara y seca. M\'as formalmente:
%
\begin{proof}
  El  resultado es  inmediato  de la  propiedad  de reflexividad  de  la ley  de
  Bernoulli,                           conjuntamente                          al
  lema~\ref{Lem:BinomilaSumaBernoulli}. Alternativamente,  se nota que  $P(n-X =
  x) = P(X = n-x) = \bino{n}{n-x} p^{n-x} (1-p)^x = \bino{n}{x} (1-p)^x p^{n-x}$
  \ notando que $\bino{n}{n-x} = \bino{n}{x}$.
\end{proof}

Nota que cuando $p = 0$ (resp. $p = 1$) la variable es cierta $X = 0$ (resp.  $X
= n$).
