\subsubseccion{Variable real con certeza}
\label{Sssec:MP:Certeza}

El caso \ $X = a \in \Rset^d$ \ deterministico ($\forall \, \omega, \: X(\omega)
= a$)  puede ser ver  visto como un  caso degenerado de vector  aleatorio. Visto
as\'i, sus caracter\'isticas principales  vistas en las secciones anteriores son
resimidas en la tabla siguiente:

\begin{caracteristicas}
%
Dominio de definici\'on & $\X = \{ a \}, \quad a \in \Rset^d$\\[2mm]
\hline
%
Distribuci\'on de probabilidad & $p_X(x) = \un_{\{a\}}(x)$\\[2mm]
\hline
%
Promedio & $\displaystyle m_X = a$\\[2mm]
\hline
%
Covarianza~\footnote{Siendo cero la covarianza, no se define ni la asimetr\'ia,
ni la curtosis. Sin embargo, de una manera se puede decir que la ley no es
asim\'etrica, y con cola livianas (no hay colas).} & $\displaystyle \Sigma_X =
0$\\[2mm]
\hline
%
%\modif{Asimetr\'ia} & $\gamma_X = 0$\\[2mm]
%\hline
%%
%Curtosis por exceso & $\displaystyle \widebar{\kappa}_X = - \sum_{i,j=1}^d \Big( \! \left(
%    \un_i \un_i^t \right) \otimes \left(  \un_j \un_j^t \right) +  \left( \un_i
%    \un_j^t \right) \otimes \left( \un_i  \un_j^t \right) + \left( \un_i \un_j^t
%  \right) \otimes \left( \un_j \un_i^t \right) \! \Big)$\\[2mm]
%\hline
%
Generadora de probabilidad & $\displaystyle G_X(z) = \prod_{i=1}^d z_i^{a_i}$ \ para \ $z_i \in \Cset$
\ si $a_i \ge 0$ \ y \ $\Cset^*$ \ si no\\[2mm]
\hline
%
Generadora de momentos & $\displaystyle M_X(u) = e^{a^t u}$ \ para \ $u \in
\Cset^d$\\[2mm]
\hline
%
Funci\'on caracter\'istica & $\displaystyle \Phi_X(\omega) = e^{\imath \, a^t
\omega}$
\end{caracteristicas}

% Momentos & $ \Esp\left[ X^k \right] = p^k$\\[2mm]
% Momento factorial & $\Esp\left[ (X)_k \right] = ?$\\[2mm]

La funci\'on de masa y funci\'on de repartici\'on son representadas en la figura
Fig.~\ref{Fig:MP:Certeza} en el caso escalar.
%
\begin{figure}[h!]
\begin{center} \begin{tikzpicture}%[scale=.9]
\shorthandoff{>}
%
\pgfmathsetmacro{\a}{2};% a
\pgfmathsetmacro{\sy}{2.5};% y-scaling
\pgfmathsetmacro{\r}{.05};% radius arc non continuity F_X
%
% masa
\begin{scope}
%
%
\draw[>=stealth,->] (-.25,0)--({\a+1.75},0) node[right]{\small $x$};
\draw[>=stealth,->] (0,-.1)--(0,{\sy+.25}) node[above]{\small $p_X$};
%
\draw[dotted] (\a,0)--(\a,\sy) node[scale=.4]{$\bullet$};
\draw (0,\sy)--(-.1,\sy) node[left,scale=.7]{$1$};
\draw (\a,0)--(\a,-.1) node[below,scale=.7]{$a$};
%%
\end{scope}
%
%
% reparticion
\begin{scope}[xshift=8.5cm]
%
\draw[>=stealth,->] (-.75,0)--({\a+1.75},0) node[right]{\small $x$};
\draw[>=stealth,->] (0,-.1)--(0,{\sy+.25}) node[above]{\small $F_X$};
%
% cumulativa
\draw[thick] (-.5,0)--(\a,0);
\draw ({\a+\r},\r) arc (90:270:\r);
\draw[dotted] (\a,0)--(\a,\sy);
\draw[thick] (\a,\sy) node[scale=.4]{$\bullet$}--({\a+1.5},\sy);
%
\draw (0,\sy)--(-.1,\sy) node[left,scale=.7]{$1$};
\draw (\a,0)--(\a,-.1) node[below,scale=.7]{$a$};
\end{scope}
%
\end{tikzpicture} \end{center}
% 
\leyenda{Ilustraci\'on  de una  distribuci\'on  cierta (a),  y  la funci\'on  de
  repartici\'on asociada (b).}
\label{Fig:MP:Certeza}
\end{figure}

\

Notar que todo se extiende al caso complejo sin costo adicional.

\SZ{Poner ac\'a la ley de los gran n\'umeros? M\'as notas historicas.}