\subsubseccion{Variable real con certeza}
\label{Sssec:MP:Certeza}

El caso \ $X = a \in \Rset^d$ \ deterministico ($\forall \, \omega, \: X(\omega)
= a$)  puede ser ver  visto como un  caso degenerado de vector  aleatorio. Visto
as\'i, sus caracter\'isticas principales  vistas en las secciones anteriores son
resumidas en la tabla siguiente:

\begin{caracteristicas}
%
Dominio de definici\'on & $\X = \{ a \}, \quad a \in \Rset^d$\\[2mm]
\hline
%
%Distribuci\'on de probabilidad
Funci\'on de masa & $p_X(x) = \uno_{\{a\}}(x)$\\[2mm]
\hline
%
Promedio & $\displaystyle m_X = a$\\[2mm]
\hline
%
Covarianza~\footnote{Siendo cero la covarianza, no se define ni la asimetr\'ia,
ni la curtosis. Sin embargo, de una manera se puede decir que la ley no es
asim\'etrica, y con cola livianas (no hay colas).} & $\displaystyle \Sigma_X =
0$\\[2mm]
\hline
%
%\modif{Asimetr\'ia} & $\gamma_X = 0$\\[2mm]
%\hline
%%
%Curtosis por exceso & $\displaystyle \widebar{\kappa}_X = - \sum_{i,j=1}^d \Big( \! \left(
%    \uno_i \uno_i^t \right) \otimes \left(  \uno_j \uno_j^t \right) +  \left( \uno_i
%    \uno_j^t \right) \otimes \left( \uno_i  \uno_j^t \right) + \left( \uno_i \uno_j^t
%  \right) \otimes \left( \uno_j \uno_i^t \right) \! \Big)$\\[2mm]
%\hline
%
Generadora de probabilidad & $\displaystyle G_X(z) = \prod_{i=1}^d z_i^{a_i}$ \ para \ $z_i \in \Cset$
\ si $a_i \ge 0$ \ y \ $\Cset_0$ \ si no\\[2mm]
\hline
%
Generadora de momentos & $\displaystyle M_X(u) = e^{a^t u}$ \ para \ $u \in
\Cset^d$\\[2mm]
\hline
%
Funci\'on caracter\'istica & $\displaystyle \Phi_X(\omega) = e^{\imath \, a^t
\omega}$
\end{caracteristicas}

% Momentos & $ \Esp\left[ X^k \right] = p^k$\\[2mm]
% Momento factorial & $\Esp\left[ (X)_k \right] = ?$\\[2mm]

La funci\'on de masa y funci\'on de repartici\'on son representadas en la figura
Fig.~\ref{Fig:MP:Certeza} en el caso escalar.
%
\begin{figure}[h!]
\begin{center} \begin{tikzpicture}%[scale=.9]
\shorthandoff{>}
%
\pgfmathsetmacro{\a}{2};% a
\pgfmathsetmacro{\sy}{2.5};% y-scaling
\pgfmathsetmacro{\r}{.05};% radius arc non continuity F_X
%
% masa
\begin{scope}
%
%
\draw[>=stealth,->] (-.25,0)--({\a+1.75},0) node[right]{\small $x$};
\draw[>=stealth,->] (0,-.1)--(0,{\sy+.25}) node[above]{\small $p_X$};
%
\draw[dotted] (\a,0)--(\a,\sy) node[scale=.4]{$\bullet$};
\draw (0,\sy)--(-.1,\sy) node[left,scale=.7]{$1$};
\draw (\a,0)--(\a,-.1) node[below,scale=.7]{$a$};
%%
\end{scope}
%
%
% reparticion
\begin{scope}[xshift=8.5cm]
%
\draw[>=stealth,->] (-.75,0)--({\a+1.75},0) node[right]{\small $x$};
\draw[>=stealth,->] (0,-.1)--(0,{\sy+.25}) node[above]{\small $F_X$};
%
% cumulativa
\draw[thick] (-.5,0)--(\a,0);
\draw ({\a+\r},\r) arc (90:270:\r);
\draw[dotted] (\a,0)--(\a,\sy);
\draw[thick] (\a,\sy) node[scale=.4]{$\bullet$}--({\a+1.5},\sy);
%
\draw (0,\sy)--(-.1,\sy) node[left,scale=.7]{$1$};
\draw (\a,0)--(\a,-.1) node[below,scale=.7]{$a$};
\end{scope}
%
\end{tikzpicture} \end{center}
% 
\leyenda{Ilustraci\'on  de una  distribuci\'on  cierta (a),  y  la funci\'on  de
  repartici\'on asociada (b).}
\label{Fig:MP:Certeza}
\end{figure}

\

Notar que todo se extiende al caso complejo sin costo adicional.

\

\index{Ley de gran n\'umeros}
El caso  de variables  deterministicas puede ser  visto como caso  degenerado de
variables aleatorias, pero aparecen de vez a cuando tambi\'en como caso l\'imite
de  sucesiones o  series  de  variables aleatorias.   En  particular, aparece  a
trav\'es de  la ley  de gran n\'umeros,  un de  los primeros casos  de l\'imites
estudiado tratando  de variables aleatorias. Historicamente, un  de los primeros
que  estudio  la convergencia  (sin  prueba  e  implicitamente) de  un  promedio
empirico a esta ``ley'' es el  matem\'atico italiano y jugador de dados y cartas
Gerolamo Cardano en el siglo~{XVI}, en su libro sobre los juegos de azar escrito
en  1564   (ver  introducci\'on  y~\cite{Car63,   Bel05}  o~\cite[Cap.~4]{Hal90}
o~\cite[Cpa.~3]{Mlo08}).  En  otras palabras,  explic\'o que la  precisi\'on las
estadisticas empiricos se mejora con el  n\'umero de datos, lo que es nada m\'as
que, en palabras,  el resultado de la ley dicha de  gran n\'umeros. En palabras,
saliendo  de  variables  aleatoria  independientes  de misma  ley,  el  promedio
empirico tiende a  la media donde enfatisaremos en que  sentido hay que entender
``tiende a''. Tal  convergencia fue estudiada y probada  mucho m\'as tarde, bajo
el  impulso  del suizo  Jacob  Bernoulli~\cite[Pars  4]{Ber1713} (ver  tambi\'en
Montmort~\cite{Mon13, Pea25})  en el  contexto de variables  binarias, conocidos
hoy  como  variables  de Bernoulli  (ver  subecci\'on~\ref{Sssec:MP:Bernoulli}).
Luego,  el  teorema  fue   mejorado  por  ejemplo  por  de  Moivre~\cite{Moi56},
Laplace~\cite{Lap20} o Poisson~\cite{Poi37}, yendo  m\'as all\'a de solamente la
convergencia del  promedio empirico a la  media.  El teorema  fue ampliado m\'as
all\'a  de la  ley  binomial como  suma  de variables  de  Bernoulli (ver  m\'as
adelante),    por    varios    autores   tales    que    Chebyshev~\cite{Tch46},
Markov~\cite{Mar13},    Borel~\cite{Bor09:12}     ,    Kinchin~\cite{Kin29}    o
Kolmogoroff~\cite{Kol30} entre otros (ver~\cite{Sen13} y referencias).

Formalmente,  las dos  versiones usuales  del  teoremas de  formaliza de  manera
siguiente  (ver  tambi\'en~\cite{Fel71,  Shi84,  AshDol99,  JacPro03,  AthLah06,
  Bil12, Coh13}).

\begin{teorema}[Ley debil de los gran n\'umero]
  Sea  \ $\left\{ X_k  \right\}_{k \in  \Nset_0}$ \  una sucesi\'on  de vectores
  aleatorios  independientes e identicamente  distribuidas (iid),  admitiendo una
  media  $m =  \Esp[X_k]$  \ y  sea  \ $\displaystyle  \widebar{X}_n =  \frac1n
  \sum_{k=1}^n X_k$ \ el promedio empirico. Entonces
  %
  \[
  \widebar{X}_n \limitP{n \to +\infty} m
  \]
  %
  donde $\limitP{}$ significa que el l\'imite es en probabilidad, \ie
  %
  \[
  \forall  \:  \varepsilon  >0,  \quad  \lim_{n  \to  +\infty}  P\left(  \left\|
      \widebar{X}_n - m \right\| > \varepsilon \right) = 0
  \]
\end{teorema}
\begin{proof}
  Una    prueba    sencilla   se    apoya    en    el    teorema   de    Markov,
  Cor.~\ref{Cor:MP:Markov},  cuando  los $X_k$  admiten  una  covarianza. De  la
  independencia, es  sencillo ver que  \ $\Cov\left[ \widebar{X}_n \,  \right] =
  \frac1n \Cov\left[ X_1 \right]$. Entonces,
  %
  \[
  P\left(  \left\|  \widebar{X}_n  -  m  \right\|  >  \varepsilon  \right)  \le
  \frac{\Esp\left[ \left\|  \widebar{X}_n - m\right\|^2 \right]}{\varepsilon^2}
  =  \frac{\Tr\left(   \Cov\left[  X_1  \right]   \right)}{n  \,  \varepsilon^2}
  \xrightarrow[n \to \infty]{} 0
  \]
  %
  lo que cierra la prueba.

  De hecho,  no es necesario  que los $X_k$  admitan una covarianza.  Una prueba
  alternativa    se   apoya   sobre    la   funci\'on    caracter\'istica.   Del
  teorema~\ref{Teo:MP:PropiedadesFuncionCaracteristica},   se   obtiene  de   la
  independencia
  %
  \[
  \Phi_{\widebar{X}_n}(\omega)   =  \left(   \Phi_{X_1}\left(  \frac{\omega}{n}
    \right) \right)^n = \left( 1 + \frac{\imath}{n} m^t \omega + o\left( \left\|
        \frac{\omega}{n} \right\| \right) \right)^n \xrightarrow[n \to \infty]{}
  e^{\imath m^t \omega}
  \]
  %
  En otros t\'erminos, la funci\'on caracter\'istica de $\widebar{X}_n$ tiende a
  la   de  $m$   punto  a   punto.  Se   usa  el   teorema  de   continuidad  de
  L\'evy~\cite{AshDol99, AthLah06, Bil12, Coh13}, no probado en este libro, para
  concluir  que  \  $\widebar{X}_n$  \  tiende  en distribuci\'on  a  \  $m$,  y
  equivalentemente tiende en probabilidad.
\end{proof}
%
Pasando,  de  la primera  prueba,  se  puede notar  que  se  puede debilitar  la
hypotesis  de  independencia,  y  a\'un  la  de misma  ley  para  los  $X_k$,  a
condici\'on de que $\Cov\left[ \widebar{X}_n \right]$ \ tiende a cero cuando $n
\to +\infty$ (por ejemplo, queda valide con la independencia y varianza acotada).

En palabras, el teorema traduce el  pensamiento de Cardano, que es que cualquier
sea el radio de la bola centrada en $m$, cuando crece el n\'umero de variables en
el promedio  empirico, la probabilidad de  que este promedio sea  afuera de esta
bola tiende a cero.

De hecho, como  para series de funciones (lo que  son las variables aleatorias),
hay varias manera  de converger. Una m\'as fuerte  es conocido como convergencia
casi  siempre,  dando  lugar a  la  ley  dicha  fuerte  de los  gran  n\'umeros.
Historicamente, este teorema es dada en  el caso escalar, pero se extiende en el
caso  vectorial.    No  daremos  la   prueba,  que  se  encuentra   por  ejemplo
en~\cite[Teo.~6.4.2]{Gre63}    en   el   caso    vectorial,   o    entre   otros
en~\cite[Teo.~22.1]{Bil12} en el caso escalar.
%
\begin{teorema}[Ley fuert de los gran n\'umero o teorema de Kolmogorov-Khintchine]
%
  Sea  \ $\left\{ X_k  \right\}_{k \in  \Nset_0}$ \  una sucesi\'on  de vectores
  aleatorios independientes  e identicamente distribuidas  (iid), admitiendo una
  media  $m =  \Esp[X_k]$ \  y  tales que  tambi\'en \  $\Esp\left[ \left\|  X_k
    \right\| \right] <  \infty$, y sea \ $\displaystyle  \widebar{X}_n = \frac1n
  \sum_{k=1}^n X_k$ \ el promedio empirico. Entonces
  %
  \[
  \widebar{X}_n \limitcs{n \to +\infty} m
  \]
  %
  donde $\limitcs{}$ significa que el l\'imite  es casi siempre (o a veces dicho
  ``con probabiludad uno''), \ie
  %
  \[
  P\left(    \lim_{n \to +\infty}   \widebar{X}_n =  m \right) = 1
  \]
  %
  o, dicho  de otra  manera, la medida  del conjuto  $\{ \omega \tq  \lim_{n \to
    +\infty} \widebar{X}_n \ne m \}$ es cero.
\end{teorema}
%\begin{proof}
%Ver~\cite[Teo.~6.4.2]{Gre63}
%\end{proof}
%

Esta versi\'on es dicha fuerte porque la convergencia casi siempre implica la en
probabilidad~\cite{Fel71, Shi84, AshDol99, JacPro03, AthLah06, Bil12, Coh13}. Se
puede debilitar un paso m\'as  las condiciones (ej. indemendencia, etc.) pero va
m\'as all\'a  de la  meta de esta  secci\'on. El  lector se podr\'ea  referir en
libros  especializados,  por  ejemplo~\cite{Fel71,  Shi84,  AshDol99,  JacPro03,
  AthLah06, Bil12, Coh13}.

Una consecuencia de la ley fuerte de gran n\'umeros es conocido como theorema de
Borel. Dice  que, en el  contexto de variables  discretas, si una  experienca se
repite de  manera independiente  un gran n\'umero  de veces, la  proporci\'on de
ocurencia de un  estado tiendo a su probabilidad  de ocurencia (con probabilidad
uno).  Se podr\'a  referir  por  ejemplo a~\cite{Wen91}  para  tener una  prueba
``moderna''.

%\SZ{Poner ac\'a la ley de los gran n\'umeros? M\'as notas historicas.}
