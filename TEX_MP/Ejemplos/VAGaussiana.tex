\subsubseccion{Distribuci\'on normal o gaussiana multivariada real}
\label{Sssec:MP:Gaussiana}

En el caso escalar,  esta ley parece aparecer por unas de  las primeras veces en
trabajos de de  Moivre como approximaci\'on de la ley  binomial para $n$ grande,
usando la formula de Stirling~\cite{Moi30, Moi33, Moi56, Pea24, PeaMoi26, Dem33,
  Hal84,  Hal90,  JohKot95:v1, DavEdw01,  Hal06}.   Se  puede  ver tambi\'en  el
trabajo  de F.   Galton,  quien construy\'o  un  experimento, la  caja dicha  de
Galton, que  ilustra por una  parte como se  puede obtener la ley  binomial como
suma   de  Bernoulli,   y   la  convergencia   a  la   gaussiana~\cite[Figs.~7-9,
p.~63]{Gal89}  o~\cite[p.~38]{Pea20}.  Aparte  de  Moivre, la  ley gaussiana  fue
desarollado mucho por los matem\'aticos  como Gauss en el estudio del movimiento
de   planetas   con   perturbaciones   (predicci\'on  de   la   trayectoria   de
C\'eres)~\cite{Gau09,  Pea24, DavEdw01,  Hal06}, basado  en trabajos  de  A.  M.
Legendre~\cite{Leg05,   DavEdw01,  Hal06},   o  Laplace   en  mismos   tipos  de
problemas~\cite{Lap09, Lap09:Supp, Lap12, Lap14, Lap20, Pea24, DavEdw01, Hal06}.
De hecho,  apoyandose en  trabajos de  de Moivre, la  formaliz\'o antes  y m\'as
claramente  Laplace,   quien  revandic\'o   entonces  su  partenidad   (ver  por
ejemplo~\cite{Pea20}).   Por eso,  esta ley  es tambi\'en  conocida como  ley de
Laplace-Gauss.

En el contexto multivariado, la extensi\'on natural de la ley binomial siendo la
ley multinomial, es sin sorpresa  que se introdujo la gaussiana multivaluada como
approximaci\'on de la multinomial.  Este trabajo  es debido entre otros a J.  L.
Lagrange en  los a\~nos 1770, con  correcciones debido unas  decadas despu\'es a
A. de  Morgan~\cite{Mor38}. Pero apareci\'o  antes en el caso  bidimensional, en
particular  a  trav\'es  del  estudio  del coeficiente  de  correlaci\'on  entre
variables   aleatorias  (ver   por  ejemplo   trabajos   de  Galton~\cite{Gal77,
  Gal77:Nature, Pea20}).

A pesar de que parece menos natural en la modelisaci\'on de fenomenos aleatorios
que leyes uniformes, la ley gaussiana es seguramente unas de las m\'as importante
en  probabilidad, sino  que  la m\'as  importante  y la  m\'as  expendida en  la
naturaleza.   Eso  viene sin  duda  del teorema  del  l\'imite  central. En  dos
palabras,  cuando  se  suman  un  n\'umero importante  de  variables  aleatorias
(independientes,   de  misma  ley,   admitiendo  una   varianza,  o   con  menos
restricciones~\cite[Cap.~11]{AthLah06}),  correctamente  normalizado, esta  suma
tiende a una gaussiana~\footnote{De hecho,  la approximaci\'on de la ley binomial
  por una  gaussiana cuando  $n$ es  grande es una  caso particular  del teorema,
  siendo la binomial una suma  de Bernoulli independientes.}.  En la naturaleza,
se puede ver el ruido (se\~nales) como suma de un n\'umero importante de fuentes
de  ruido independientes,  justificando el  modelo  gaussiano~\cite{Fel71, Cam86,
  AshDol99, JacPro03, AthLah06, Ren07, Bil12}.  Ad\'emas, como lo vamos a ver en
el  cap\'itulo~\ref{Cap:SZ:Informacion}, esta  ley es  la de  incerteza m\'axima
(maximizando la entrop\'ia) teniendo  una dada varianza. Aparece naturalmente en
termod\'inamica    (gaz    perfecto,   con    un    n\'umero    muy   alto    de
particulas)~\cite{Max67, Bol96,  Bol98, Gib02, Jay65}. En  estimaci\'on, bajo la
hipotesis  gaussiana,  los  estimadores  de  par\'ametros  minimizando  el  error
cuadr\'atico promedio son  generalmente lineal~\cite{Kay93, Rob07}.  Todas estas
consideraciones  dan a la  ley gaussiana  un rol  central en  la teor\'ia  de las
probabilidades.

Se denota \  $X \, \sim \, \N(m,\Sigma)$ \  con \ $m \in \Rset^d$  \ y \ $\Sigma
\in  \Pos_d^+(\Rset)$  \  conjunto  de   las  matrices  de  \  $\M_{d,d}(\Rset)$  \
s\'imetricas definidas positivas. Las  caracter\'isticas de la gaussiana son las
siguientes:

\begin{caracteristicas}
%
Dominio de definici\'on & $\X = \Rset^d$\\[2mm]
\hline
%
Par\'ametros & $m \in \Rset^d, \:\: \Sigma \in \Pos_d^+(\Rset)$\\[2mm]
\hline
%
Densidad de probabilidad & $\displaystyle p_X(x) = \frac{1}{(2
\pi)^{\frac{d}{2}} \left| \Sigma \right|^{\frac12}} \, e^{-\frac12 (x-m)^t
\Sigma^{-1} (x-m)}$\\[2.5mm]
\hline
%
Promedio & $ m_X = m$\\[2mm]
\hline
%
Covarianza & $\Sigma_X = \Sigma$\\[2mm]
\hline
%
\modif{Asimetr\'ia} & $\gamma_X = 0$\\[2mm]
\hline
%
Curtosis por exceso & $\widebar{\kappa}_X = 0$\\[2mm]
\hline
%
Generadora de momentos & $\displaystyle M_X(u) = e^{u^t \Sigma u + u^t m}$ \
para \ $u \in \Cset^d$\\[2mm]
\hline
%
Funci\'on  caracter\'istica   &  $\displaystyle  \Phi_X(\omega)   =  e^{-\frac12
\omega^t \Sigma \omega + \imath \omega^t m}$
\end{caracteristicas}

Nota: trivialmente, se puede escribir $X  \, \egald \, \Sigma^{\frac12} N + m$ \
con \ $N \, \sim \, \N(0,I)$ \  donde \ $N$ \ es dicha {\em gaussiana estandar} o
{\em centrada-normalizada}. Las caracter\'isticas de  \ $X$ \ son v\'inculadas a
las  de  \  $N$ \  (y  vice-versa)  por  transformaci\'on afine  (ver  secciones
anteriores).


La densidad de probabilidad gaussiana y  la funci\'on de repartici\'on en el caso
escalar son  representadas en la figura Fig.~\ref{Fig:MP:Gaussiana}-(a)  y (b) y
una      densidad      en       un      contexto      bi-dimensional      figura
Fig.~\ref{Fig:MP:Gaussiana}(c).
%
\begin{figure}[h!]
\begin{center} \begin{tikzpicture}%[scale=.9]
\shorthandoff{>}
%
\pgfmathsetmacro{\sx}{.75};% x-scaling
\pgfmathsetmacro{\mx}{3.5};% x maximo del plot
%
% Approximation de la cdf gaussienne
\tikzset{declare function={
normcdf(\x)=1/(1 + exp(-0.07056*(\x)^3 - 1.5976*(\x)));
}}
% densidad
\begin{scope}
%
\pgfmathsetmacro{\sy}{2.5*sqrt(2*pi)};% y-scaling 
\draw[>=stealth,->] ({-\sx*\mx-.25},0)--({\sx*\mx+.25},0) node[right]{\small $x$};
\draw[>=stealth,->] (0,-.1)--(0,2.75) node[above]{\small $p_X$};
%
\draw[thick,domain=-\mx:\mx,samples=100] plot ({\x*\sx},{\sy*exp(-.5*\x*\x)/sqrt(2*pi)});
%
\draw (0,{\sy/sqrt(2*pi)})--(-.1,{\sy/sqrt(2*pi)}) node[left,scale=.7]{$\frac1{\sqrt{2 \pi}}$};
%
\end{scope}
%
%
% reparticion
\begin{scope}[xshift=8.5cm]
%
\pgfmathsetmacro{\sy}{2.5};% y-scaling 
%
\draw[>=stealth,->] ({-\sx*\mx-.25},0)--({\sx*\mx+.25},0) node[right]{\small $x$};
\draw[>=stealth,->] (0,-.1)--(0,{\sy+.25}) node[above]{\small $F_X$};
%
% cumulativa
\draw[thick,domain=-\mx:\mx,samples=100] plot({\x*\sx},{\sy*normcdf(\x)});
%
\draw (0,\sy)--(-.1,\sy) node[left,scale=.7]{$1$};
\end{scope}
%
\end{tikzpicture} \end{center}
% 
\leyenda{Ilustraci\'on  de  una   densidad  de  probabilidad  gaussiana  escalar
  estandar  (a), y la  funci\'on de  repartici\'on asociada  (b), as\'i  que una
  densidad  de probabilidad  gaussiana bi-dimensional  centrada y  de  matriz de
  covarianza \ $\Sigma_X = R(\theta)  \Delta^2 R(\theta)^t$ \ con \ $R(\theta) =
  \protect\begin{bmatrix}   \cos\theta  &   -  \sin\theta\\[2mm]   \sin\theta  &
    \cos\theta  \protect\end{bmatrix}$ \  matriz  de rotaci\'on  y  \ $\Delta  =
  \diag\left(\protect\begin{bmatrix}  1   &  a\protect\end{bmatrix}  \right)$  \
  matriz  de   cambio  de  escala,   y  sus  marginales   \  $X_1  \,   \sim  \,
  \N\left(0,\cos^2\theta  + a^2  \sin^2\theta \right)$  \ y  \ $X_2  \,  \sim \,
  \N\left(0,\sin^2\theta + a^2 \cos^2\theta  \right)$ \ (ver m\'as adelante). En
  la figura, $a = \frac14$ \ y \ $\theta = \frac{\pi}{6}$.}
\label{Fig:MP:Gaussiana}
\end{figure}

La gaussiana tiene un par de propiedades particulares:
%
\begin{lema}[Gaussiana y cumulantes]
%
  Sea \  $X$ \ vector aleatorio de  media $m$, covarianza $\Sigma$  y de secunda
  funci\'on caracter\'istica admtiendo un desarollo de Taylor. Entonces
  %
  \[
  \kappa_k[X] =  0 \quad \forall \: k  \ge 4 \quad \Longleftrightarrow  \quad X \sim
  \N(m,\Sigma)
  \]
\end{lema}
%
\begin{proof}
  La pueba es inmediata del lema~\ref{Lem:MP:CumSecFctCarac},
  %
  \[
  \kappa_k[X]  =  0  \quad  \forall  \:  k \ge  4  \quad  \Longleftrightarrow  \quad
  \Psi_X(\omega) = \imath \, \omega^t m - \frac12 \omega^t \Sigma \omega
  \]
  %
  lo que es nada m\'as que la secunda funci\'on caracter\'istica de la gaussiana,
  esa determiniendo completamente la ley.
\end{proof}
%
\begin{teorema}[Stabilidad]
\label{Teo:MP:StabilidadGaussiana}
%
  Sean \ $A_i , i = 1,\ldots,n$  \ matrices de \ $\M_{d',d}(\Rset), \: d' \le d$
  \ de rango lleno, $b_i \in \Rset^{d'}$ \ y \ $X_i \, \sim \, \N(m_i,\Sigma_i)$
  \ independientes, entonces
  %
  \[
  \sum_{i=1}^n \left(  A_i X_i  + b_i \right)  \, \sim \,  \N\left( \sum_{i=1}^n
    \left( m_i + b_i \right) \, , \, \sum_{i=1}^n A_i \Sigma_i A_i^t \right)
  \]
  % 
  En particular, cualquier combinaci\'on lineal  de los componentes de un vector
  gaussiano da una gaussiana.  Reciprocamente, si cualquier combinaci\'on lineal
  de los componentes de un vector aleatorio sigue una ley gaussiana, entonces el
  vector es gaussiano.
\end{teorema}
%
\begin{proof}
  Este  resultato se proba  usando funci\'on  caracter\'istica de  la gaussiana,
  conjuntamente al teorema~\ref{Teo:MP:PropiedadesFuncionCaracteristica}.
\end{proof}
%
\begin{corolario}[Media emp\'irica]\label{Cor:MP:MediaEmpiricaGauss}
%
  Sean \ $X_i \, \sim \, \N(m,\Sigma), \: i = 1, \ldots , n$ \ independientes. Entonces,
  %
  \[
  \overline{X} =  \frac{1}{n} \sum_{i=1}^n  X_i \,  \sim \, \N\left(  m \,  , \,
    \frac{1}{n} \Sigma \right)
  \]
   %
  $\overline{X}$  es llamada {\em  media emp\'irica}~\footnote{Es  la estimaci\'on
    \'optima de  la media  $m$ a  partir de los  $X_i$ en  el sentido  del error
    cuadratico  promedio   m\'inimo,  o  en   el  sentido  de   la  verosimlitud
    m\'axima~\cite{Kay93, Rob07}.}, y es un estimador ``natural'' de la media de
  un vector aleatorio a partir de copias independientes de misma ley.
  %
\end{corolario}
%
\begin{teorema}[Independencia]
\label{Teo:MP:IndependenciaGaussiana}
%
  Sea   \   $X  \,   \sim   \,   \N(m,\Delta)$  \   con   \   $\Delta  =   \diag
  \left(  \begin{bmatrix}  \sigma_1^2  &  \cdots  &  \sigma_d^2  \end{bmatrix}^t
  \right)$   \  diagonal.   Entonces  los   componentes  \   $X_i  \,   \sim  \,
  \N(m_i,\sigma_i^2)$ \ son independientes.
\end{teorema}
%
\begin{proof}
  Este resultato se proba  trivialmente escribiendo la densidad de probabilidad,
  notando que se factorisa.
\end{proof}
%
Hemos visto que cuando un  vector tiene componentes independientes, la matriz de
covarianza  es   diagonal  (lema~\ref{Lem:MP:IndependenciaCov}),  pero   que  la
rec\'iproca es falsa en general.  El \'ultimo teorema muestra que la rec\'iproca
vale en el caso gaussiano.

Volvemos  ahora al rol  central de  la gaussiana  como modelo  probabilistico muy
frecuente  de fenomeos  aleatorios. Este  rol particular  viene del  teorema del
l\'imite  central que  ya introdujimos.  A veces,  es conocido  como  teorema de
Lindenberg-Feller (por lo menos la forma con condiciones m\'as debiles que en la
formulaci\'on original).   Para unas de  las formulaciones originales,  se puede
referirse  al trabajo  de Laplace  de  1809 o  de 1912~\cite{Lap09,  Lap09:Supp,
  Lap12,  Lap14, Lap20}.   El nombre  ``central'' viene  de un  documento  de G.
P\'olya   de   1920,  titulado   ``\"Uber   den   zentralen  Grenzwertsatz   der
Wahrscheinlichkeitsrechnung  und das Momentenproblem''  (``Sobre el  teorema del
l\'imite central del  c\'alculo probabil\'istico y el problema  de los momentos;
el  teorema  es  central\ldots~\cite{Pol20,   Cam86}).   Se  enuncia  de  manera
siguiente~\cite{Spi76, BroDav87, LehCas98, AshDol99, JacPro03, AthLah06, Bil12}:
% ~\footnote{Aparte    en~\cite{AshDol99,   JacPro03},   el    teorema   aparece
%   frecuentemente en los libros en el contexto escalar, seguramente por razones
%   historicas.    Pero,  con   el  mismo   enfoque,  se   prueba  en   el  caso
%   multivariado.}:

\begin{teorema}[Teorema del l\'imite central]\label{Teo:MP:CLT}
%
  Sea  \  $\{  X_i \}_{i  \in  \Nset^*}$  \  una  sucesi\'on de  vectores  aleatorios
  independientes, de misma ley,  y que admiten un promedio \ $m$  \ y una matriz
  de covarianza \ $\Sigma$. Entonces
  %
  \[
  \frac{1}{\sqrt{n}}  \sum_{i=1}^n  \left( X_i  -  m  \right)  \: \limitd{n  \to
    +\infty} \: Y \sim \N\left( 0 , \Sigma \right)
  \]
  %
  donde  \ $\limitd{}$ \  significa que  el l\'imite  es en  distribuci\'on (ver
  notaciones).
\end{teorema}
\begin{proof}
  Hay varias pruebas de este resultado.  Quiz\'as la m\'as simple se apoya sobre
  la funci\'on  c\'aracteristica.  Sin perdida de generalidad,  supongamos que \
  $m = 0$. Sea \ $\displaystyle Y_n = \frac{1}{\sqrt{n}} \sum_{i=1}^n X_i$.  Sea
  \    $\omega$    \    fijo.     Por    independencia    y    relaciones    del
  teorema~\ref{Teo:MP:PropiedadesFuncionCaracteristica}~\footnote{$o\left(
      n^{-1} \right)$ significa que  el termino que queda, digamos $\varepsilon$
    es tal que $n \varepsilon$ tiende a cero cuando $n$ tiende al infinito.}:
  %
  \begin{eqnarray*}
  \Phi_{Y_n}(\omega) & = & \left( \Phi_{X_i}\left( \frac{\omega}{\sqrt{n}}
  \right) \right)^n\\[2mm]
  %
& = & \left( \Phi_{X_i}(0) + \frac{1}{\sqrt{n}} \, \omega^t \, \nabla_\omega
  \Phi_{X_i}(0) + \frac{1}{2 n} \, \omega^t \, \Hess_\omega\Phi_{X_i}(0) \, \omega +
  o\left( n^{-1} \right) \right)^n\\[2mm]
  %
  & = & \left( 1 - \frac{1}{2 n} \, \omega^t \, \Sigma \, \omega +
  o\left( n^{-1} \right) \right)^n\\[2mm]
  %
  & \xrightarrow[n \to +\infty]{} & \exp\left( -\frac12 \omega^t \Sigma \omega \right)
  \end{eqnarray*}
  %
  porque \ $\Phi_{X_i}(0) = 1$, $X_i$ \  siendo de media nula el gradiente de la
  funci\'on   caracter\'istica   se  cancela   en   \   $\omega   =  0$,   y   \
  $\Hess_\omega\Phi_{X_i}(0)  =  -  \Sigma$.   Se reconoce  ahora  la  funci\'on
  caracter\'istica   de  la   gaussiana,   lo  que   prueba   que  la   funci\'on
  caracter\'istica  de  \  $Y_n$  \  converge  simplemente  hacia  la  funci\'on
  caracter\'istica  de la gaussiana.   Se cierra  la pruba  usando el  teorema de
  convergencia de  L\'evy, diciendo que  la convergencia simple de  la funci\'on
  caracter\'istica  implica  la  convergencia en  distribuci\'on~\cite{AshDol99,
    Bil12, AthLah06}.
\end{proof}
%
En particular, la  media emp\'irica hechas a partir  de vectores independientes de
media  $m$,  admitiendo  una  covarianza  \  $\Sigma$  \  y  de  misma  ley  (no
necesariamente gaussiana), tiende a ser gaussiana  de media \ $m$ \ y covarianza \
$\frac{1}{n} \Sigma$.

Existen  varias   variantes  de  este   teorema  que  enunciamos,  sin   dar  la
prueba.  Dejamos el  lector a  libros m\'as  especializados como~\cite{AshDol99,
  Bil12, AthLah06, Lin22}.
% Lindeberg 1920

\begin{teorema}[Teorema de Lindenberg-Feller]\label{Teo:MP:LindenbergFeller}
%
  Sean  $\{  X_i \}_{i  \in  \Nset^*}$  vectores  aleatorios independientes,  no
  necesariamente de misma distribuci\'on de probabilidad, con \ $X_i$ \ de media
  \ $m_i = \Esp[X_i]$ \ y de matriz de covarianza \ $\Sigma_i \in \Pos_d^+(\Rset)$.
  Sean \ $C_n = \sum_{i=1}^n \Sigma_i$, \ $c_n^2$ \ al autovalor m\'as peque\~na
  de $C_n$,  \ y  \ $Y_n  = C_n^{-\frac12} \sum_{i=1}^n  \left( X_i  - \Esp[X_i]
  \right)$.
  %
  \[
  \mbox{Si} \quad \lambda_n > 0 \quad \mbox{y} \quad \forall \: \varepsilon > 0,
  \quad  \lim_{n  \to  +\infty}  \sum_{i=1}^n  \Esp\left[  \left\|  \frac{X_i  -
        m_i}{c_n} \right\|^2  \un_{\left[ \varepsilon \;  +\infty \right)}\left(
      \left\| \frac{X_i - m_i}{c_n} \right\| \right) \right] = 0
  \]
  %
  entonces
  %
  \[
  Y_n \: \limitd{n \to +\infty} \: Y \sim \N\left( 0 , I \right)
  \]
\end{teorema}
%
En  numerosos libros,  este teorema  es  dado en  el caso  escalar. Se  extiende
sencillamente al caso multivariado gracia a lo que es conocido como {\it teorema
  de Cram\'er-Wold},  diciendo que una  secuencia de vectores aleatorios  \ $Y_n
\limitd{}  Y$ \  si y  solamente para  cualquier \  $u \in  \Rset^d$ \  $u^t Y_n
\limitd{} u^t Y$~\cite{AshDol99, AthLah06, Bil12}.

Sin dar  la prueba,  la condici\'on  de Lindenberg dice  que si  la suma  de las
``dispersi\'ones'' de  los vectores normalizados  por los que es  basicamente la
varianza  la   m\'as  peque\~na  de  los   componentes  de  la   suma  (una  vez
diagonalizada)  se  concentra  asintoticamente,  la  suma  renormalizada  de  lo
vectores centrados tiende a la gaussiana (en distribuci\'on).

Se  puede ver  que  se  satisface la  condici\'on  de Lindeberg  en  el caso  de
variables independientes de misma ley, del hecho  que \ $C_n = n \Sigma$, lo que
da \ $c_n^2 = n c^2$ \ con  \ $c^2$ \ autovalor m\'as peque\~na de \ $\Sigma$. A
continuaci\'on da la condici\'on \ $\displaystyle \lim_{n \to \infty} \Esp\left[
  \left\| X_i - m_i \right\|^2 \un_{\left[ \varepsilon \; +\infty \right)}\left(
    \left\|  \frac{X_i  -  m_i}{\sqrt{n}  c}  \right\|  \right)  \right]  =  0$,
satisfecha  porque el  argumento de  la funci\'on  indicadora tiende  a  0 (casi
siempre).

Un otro caso  ``trivial'' aparece cuando la secuencia  es uniformamente acotada,
\ie \  $\forall \: i, \quad  \| X_i \| \le  M$. Se puede  retomar los argumentos
anteriores, remplazando las variables por la cota.

Nota: si  se satisface  la condici\'on  dicha {\it de  Lyapunov}, \ie  si existe
$\delta > 0$ tal que
%
\[
\lim_{n   \to  \infty}   \sum_{i=1}^n   \Esp\left[  \frac{\left\|   X_i  -   m_i
    \right\|^2}{c_n^{2+\delta}} \right] = 0,
\]
%
entonces         se          satisface         la         condici\'on         de
Lindeberg~\cite{AshDol99}.  Frecuentemente,  es   m\'as  sencillo  verificar  la
condici\'on m\'as fuerte de Lyapunov para  probar la convergencia de una suma de
vectores aleatorios a la gaussiana.

\

Aparece que se puede a\'un  debilitar la condici\'on de independencia sin perder
la   convergencia    a   la   gaussiana.    Para   m\'as   detalles,    ver   por
ejemplo~\cite[Sec.~6.4]{BroDav87}.
