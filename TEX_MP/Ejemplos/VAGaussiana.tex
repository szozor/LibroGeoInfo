\subsubseccion{Distribuci\'on normal o Gaussiana multivariada real}
\label{Sssec:MP:Gaussiana}

Se denota $X \, \sim \, \N(m,\Sigma)$ \  con \ $m \in \Rset^d$ \ y \ $\Sigma \in
P_d^+(\Rset)$  \  conjunto  de  las   matrices  de  \  $\Rset^{d  \times  d}$  \
s\'imetricas definidas positivas. Las  caracter\'isticas de la Gaussiana son las
siguientes:

\begin{caracteristicas}
%
Dominio de definici\'on & $\X = \Rset^d$\\[2mm]
\hline
%
Parametros & $m \in \Rset^d, \: \Sigma \in P_d^+(\Rset)$\\[2mm]
\hline
%
Densidad de probabilidad & $\displaystyle p_X(x) = \frac{1}{(2
\pi)^{\frac{d}{2}} \left| \Sigma \right|^{\frac12}} \, e^{-\frac12 (x-m)^t
\Sigma^{-1} (x-m)}$\\[2.5mm]
\hline
%
Promedio & $ m_X = m$\\[2mm]
\hline
%
Covarianza & $\Sigma_X = \Sigma$\\[2mm]
\hline
%
\modif{Sesgo} (caso escalar) & $\gamma_X = 0$\\[2mm]
\hline
%
Curtosis por exceso (caso escalar) & $\widebar{\kappa}_X = 0$\\[2mm]
\hline
%
Generadora de  momentos &  $\displaystyle M_X(u) =  e^{u^t \Sigma u + u^t m}$  \ para \  $u \in
\Cset^d$\\[2mm]
\hline
%
Funci\'on  caracter\'istica   &  $\displaystyle  \Phi_X(\omega)   =  e^{-\frac12
\omega^t \Sigma \omega + \imath \omega^t m}$
\end{caracteristicas}

% Momentos & $ \Esp\left[ X^k \right] = p^k$\\[2mm]
% Momento factorial & $\Esp\left[ (X)_k \right] = ?$\\[2mm]
% Generadora de probabilidad & $G_X(z) = e^{\lambda (z-1)}$ \ para \ $z \in \Cset$\\[2mm]
% modo 0
% Mediana 0

Nota: trivialmente, se puede escribir $X  \, \egald \, \Sigma^{\frac12} N + m$ \
con \ $N \, \sim \, \N(0,I)$ \  donde \ $N$ \ es dicha {\em Gausiana estandar} o
{\em centrada-normalizada}. Las caracter\'isticas de  \ $X$ \ son v\'inculadas a
las  de  \  $N$ \  (y  vice-versa)  por  transformaci\'on afine  (ver  secciones
anteriores).


La densidad de probabilidad gausiana y  la funci\'on de repartici\'on en el caso
escalar son  representadas en la figura Fig.~\ref{Fig:MP:Gaussiana}-(a)  y (b) y
una      densidad      en       un      contexto      bi-dimensional      figura
Fig.~\ref{Fig:MP:Gaussiana}(c).
%
\begin{figure}[h!]
\begin{center} \begin{tikzpicture}%[scale=.9]
\shorthandoff{>}
%
\pgfmathsetmacro{\sx}{.75};% x-scaling
\pgfmathsetmacro{\mx}{3.5};% x maximo del plot
%
% Approximation de la cdf gaussienne
\tikzset{declare function={
normcdf(\x)=1/(1 + exp(-0.07056*(\x)^3 - 1.5976*(\x)));
}}
% densidad
\begin{scope}
%
\pgfmathsetmacro{\sy}{2.5*sqrt(2*pi)};% y-scaling 
\draw[>=stealth,->] ({-\sx*\mx-.25},0)--({\sx*\mx+.25},0) node[right]{\small $x$};
\draw[>=stealth,->] (0,-.1)--(0,2.75) node[above]{\small $p_X$};
%
\draw[thick,domain=-\mx:\mx,samples=100] plot ({\x*\sx},{\sy*exp(-.5*\x*\x)/sqrt(2*pi)});
%
\draw (0,{\sy/sqrt(2*pi)})--(-.2,{\sy/sqrt(2*pi)}) node[left,scale=.7]{$\frac1{\sqrt{2 \pi}}$};
%
\end{scope}
%
%
% reparticion
\begin{scope}[xshift=8.5cm]
%
\pgfmathsetmacro{\sy}{2.5};% y-scaling 
%
\draw[>=stealth,->] ({-\sx*\mx-.25},0)--({\sx*\mx+.25},0) node[right]{\small $x$};
\draw[>=stealth,->] (0,-.1)--(0,{\sy+.25}) node[above]{\small $F_X$};
%
% cumulativa
\draw[thick,domain=-\mx:\mx,samples=100] plot({\x*\sx},{\sy*normcdf(\x)});
%
\draw (0,\sy)--(-.1,\sy) node[left,scale=.7]{$1$};
\end{scope}
%
\end{tikzpicture} \end{center}
% 
\leyenda{Ilustraci\'on  de  una   densidad  de  probabilidad  gaussiana  escalar
  estandar  (a), y la  funci\'on de  repartici\'on asociada  (b), as\'i  que una
  densidad  de probabilidad  gaussiana bi-dimensional  centrada y  de  matriz de
  covarianza \ $\Sigma_X = R(\theta)  \Delta^2 R(\theta)^t$ \ con \ $R(\theta) =
  \protect\begin{bmatrix}   \cos\theta  &   -  \sin\theta\\[2mm]   \sin\theta  &
    \cos\theta  \protect\end{bmatrix}$ \  matriz  de rotaci\'on  y  \ $\Delta  =
  \diag\left(\protect\begin{bmatrix}  1   &  a\protect\end{bmatrix}  \right)$  \
  matriz  de   cambio  de  escala,   y  sus  marginales   \  $X_1  \,   \sim  \,
  \N\left(0,\cos^2\theta  + a^2  \sin^2\theta \right)$  \ y  \ $X_2  \,  \sim \,
  \N\left(0,\sin^2\theta + a^2 \cos^2\theta  \right)$ \ (ver m\'as adelante). En
  la figura, $a = \frac14$ \ y \ $\theta = \frac{\pi}{6}$.}
\label{Fig:MP:Gaussiana}
\end{figure}

La gaussiana tiene un par de propiedades particulares:
%
\begin{teorema}[Stabilidad]
\label{Teo:MP:StabilidadGaussiana}
%
  Sean \ $A_i , i = 1,\ldots,n$ \  matrices de \ $\Rset^{d' \times d}, d' \le d$
  \ de rango lleno, $b_i \in \Rset^{d'}$ \ y \ $X_i \, \sim \, \N(m_i,\Sigma_i)$
  \ independientes, entonces
  %
  \[
  \sum_{i=1}^n \left(  A_i X_i  + b_i \right)  \, \sim \,  \N\left( \sum_{i=1}^n
    \left( m_i + b_i \right) \, , \, \sum_{i=1}^n A_i \Sigma_i A_i^t \right)
  \]
  % 
  En particular, cualquier combinaci\'on lineal  de los componentes de un vector
  Gaussiano da una gaussiana.  Reciprocamente, si cualquier combinaci\'on lineal
  de los componentes de un vector aleatorio sigue una ley gaussiana, entonces el
  vector es gaussiano.
\end{teorema}
%
\begin{proof}
  Este  resultato se proba  usando funci\'on  caracter\'istica de  la gaussiana,
  conjuntalmente al teorema~\ref{Teo:MP:PropiedadesFuncionCaracteristica}.
\end{proof}

%
\begin{teorema}[Independencia]
\label{Teo:MP:IndependenciaGaussiana}
%
  Sea   \   $X  \,   \sim   \,   \N(m,\Delta)$  \   con   \   $\Delta  =   \diag
  \left(  \begin{bmatrix}  \sigma_1^2  &  \cdots  &  \sigma_d^2  \end{bmatrix}^t
  \right)$   \  diagonal.   Entonces  las   componentes  \   $X_i  \,   \sim  \,
  \N(m_i,\sigma_i^2)$ \ son independientes.
\end{teorema}
%
\begin{proof}
  Este resultato se proba  trivialmente escribiendo la densidad de probabilidad,
  notando que se factorisa.
\end{proof}
%
Hemos visto que cuando un  vector tiene componentes independientes, la matriz de
covarianza  es   diagonal  (lema~\ref{Lem:MP:IndependenciaCov}),  pero   que  la
reciproca es falsa en general. El \'ultimo teorema muestra que la reciproca vale
en el caso gausiano.

%Cuando $\lambda \to +\infty$ la variable tiende a una variable cierta $X = 0$.
\SZ{Esta distribuci\'on aparece...}
% en  el conteo  de conteo  de  une repetici\'on  de una  experiencia de  maneja
% independiente hasta que  occure un evento de probabilidad  $p$; por ejemplo el
% n\'umero de tiro de un dado  equilibriado hasta que occurre un ``6'' sigue una
% ley geometrica de parametro $p = \frac16$.