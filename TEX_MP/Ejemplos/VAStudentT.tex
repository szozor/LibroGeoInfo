\subsubseccion{Distribuci\'on Student-$t$ multivariada}
\label{Sssec:MP:StudentT}

En   el  caso   escalar,  esta   ley   fue  introducida   inicialmente  por   F.
R. Helmert~\cite{Hel75, Hel76, She95}  y J.  L\"uroth~\cite{Lur76, Pfa96}.  Pero
es m\'as  conocida por su  introducci\'on por William  Sealy Gosset~\footnote{De
  hecho,  Gosset  fue  un  estudiante  trabajando en  la  f\'abrica  de  cerveza
  irlandesa  Guiness  sobre estad\'istica  relacionada  a  la  qui\'imica de  la
  cerveza.   A pesar  que hay  varias  explicaciones sobre  el hecho  de que  se
  public\'o este trabajo bajo el nombre ``Student''. Unas es que fue para que no
  se  sabe que  la f\'abrica  estaba  trabajando sobre  estas estadisticas  para
  estudiar  la calidad  de  la cerveza~\cite{Wen16}.\label{Foot:MP:Student}}  en
1908,  trabajando  sobre variables  centradas  normalizadas  por  el promedio  y
varianza empiricos~\cite{Stu08}.   Fue estudiada entre  otros intensivamente por
el  famoso matematico  R. Fisher~\cite{Fis25}.   En la  literatura, esta  ley es
conocida bajo  los nombres {\em  Student}, {\em Student-$t$} o  simplemente {\em
  $t$-distribuci\'on} o  a\'un bajo el nombre  {\em Pearson tipo IV}  en el caso
escalar  y {\em  Pearson tipo  VII}  (para $\frac{\nu+d}{2}$  entero; ver  m\'as
abajo), debido  a la  familia de Pearson~\cite{Pea95,  JohKot95:v1, JohKot95:v1,
  KotBal00, FanKot90}.  Esta distribuci\'on aparece como a  priori conjugado del
promedio de una gausiana en inferencia bayesiana~\cite{Rob07, KotNad04}.

Se denota con \ $X \sim \T_\nu(m,\Sigma)$  \ con \ $m \in \Rset^d$, \ $\Sigma \in
P_d^+(\Rset)$ \ conjunto  de las matrices de \  $\M_{d,d}(\Rset)$ \ s\'imetricas
definidas positivas. $m$ \ es llamado  {\em par\'ametro de posici\'on} (no es la
media   que  puede  no   existir),  \   $\Sigma$  \   es  llamada   {\em  matriz
  caracter\'istica} (no es [proporcional a]  la covarianza que puede no existir)
y \ $\nu > 0$ \ llamado  {\em grados de libertad}.  Las caracter\'isticas de una
Student-$t$ son las siguientes:
%
\begin{caracteristicas}
%
Dominio de definici\'on & $\X = \Rset^d$\\[2mm]
\hline
%
Par\'ametro & $\nu \in \Rset_+^*$ \ (grados de libertad), \ $m \in \Rset^d$ \
(posici\'on), \ $\Sigma \in P_d^+(\Rset)$ \ (matriz caracer\'istica)\\[2mm]
\hline
%
Densidad de probabilidad & $\displaystyle p_X(x) = \frac{\Gamma\left(
\frac{\nu+d}{2} \right)}{\pi^{\frac{d}{2}} \nu^{\frac{d}{2}} \Gamma\left(
\frac{\nu}{2} \right) \, \left| \Sigma \right|^{\frac12}} \, \left( 1 +
\frac{(x-m)^t \Sigma^{-1} (x-m)}{\nu} \right)^{- \, \frac{\nu+d}{2}}$\\[2mm]
\hline
%
Promedio & $\displaystyle m_X = m$ \ si \ $\nu > 1$; \ no
existe si no~\footnote{De manera general, esta ley admite momentos de orden \ $k$ \
si y solamente si \ $\nu > k$.\label{Foot:MP:ExistenciaMomentosStudent}}.\\[2.5mm]
\hline
%
Covarianza~\footnote{Fijense de que $\Sigma$ no es la covarianza, pero es
proporcional a la covarianza\ldots cuando existe. Se podr\'ia imaginar
renormalizar la ley tal que \ $\Sigma_X$ \ y \ $\Sigma$ \ coinciden, pero no
ser\'ia posible en el caso \ $\nu \le 2$.} & $\displaystyle \Sigma_X =
\frac{\nu}{\nu-2} \, \Sigma$ \ si \ $\nu > 2$; \ no existe si
no~\footref{Foot:MP:ExistenciaMomentosStudent}.\\[2.5mm]
\hline
%
\modif{Asimetr\'ia} & $\displaystyle \gamma_X = 0$ \ si \ $\nu > 3$; \ no existe
si no~\footref{Foot:MP:ExistenciaMomentosStudent}.\\[2mm]
\hline
%
Curtosis por exceso & \modif{$\displaystyle \widebar{\kappa}_X = \frac{2}{\nu-4}
\sum_{i,j=1}^d \Big( \! \left(
    \un_i \un_i^t \right) \otimes \left(  \un_j \un_j^t \right) +  \left( \un_i
    \un_j^t \right) \otimes \left( \un_i  \un_j^t \right) + \left( \un_i \un_j^t
  \right) \otimes \left( \un_j \un_i^t \right) \! \Big)$}\newline si \ $\nu >
4$; \ no existe si no~\footref{Foot:MP:ExistenciaMomentosStudent}.\\[2mm]
\hline
%
Funci\'on caracter\'istica~\footnote{Se muestra sencillamente que la funci\'on
generatriz de momentos puede existir si y solamente si \ $\real{u} = 0$. La
funci\'on generadora de momentos restricta al producto cartesiano de bandas \
$\real{u} = 0$ \ es nada m\'as que la funci\'on caracter\'istica. Adem\'as, esta
funci\'on fue calculdada, especialmente en el caso multivariado, relativamente
recientemente~\cite{Sut86, Hur95, KibJoa06, SonPar14}.} & $\displaystyle
\Phi_X(\omega) = \frac{\nu^{\frac{\nu}{4}}}{2^{\frac{\nu}{2}-1} \Gamma\left(
\frac{\nu}{2} \right)} \, e^{\imath \omega^t m} \, \left( \omega^t \Sigma \omega
\right)^{\frac{\nu}{4}} K_{\frac{\nu}{2}}\left( \sqrt{\nu \, \omega^t \Sigma
\omega} \right)$
\end{caracteristicas}

Nota: nuevamente se puede escribir $X \, \egald \, \Sigma^{\frac12} S + m$ \ con
\ $S \, \sim \, \T_\nu(0,I)$ \  donde \ $S$ \ es dicha {\em Student-$t$ estandar}
y  las caracter\'isticas  de \  $X$  \ son  v\'inculadas a  las  de \  $S$ \  (y
vice-versa) por transformaci\'on lineal (ver secciones anteriores).

La densidad de probabilidad Student-$t$ estandar y la funci\'on de repartici\'on
en el caso escalar son representadas en la figura Fig.~\ref{Fig:MP:StudentT}-(a)
y    (b)   y    una   densidad    en   un    contexto    bi-dimensional   figura
Fig.~\ref{Fig:MP:StudentT}(c).
%
\begin{figure}[h!]
\begin{center} \begin{tikzpicture}
\shorthandoff{>}
%
% Para el caso univariado
\pgfmathsetmacro{\sx}{.43};% x-scaling
\pgfmathsetmacro{\mu}{0};% para tomar los grados de libertad impar; 0 => Cauchy
\pgfmathsetmacro{\md}{1};
\pgfmathsetmacro{\mt}{3};
%\pgfmathsetmacro{\mq}{3};
%
%
% para el caso bi-variado
\pgfmathsetmacro{\mdd}{0};%
\pgfmathsetmacro{\nu}{2*\mdd+1};% grados de libertad
\pgfmathsetmacro{\a}{1/3};% x-scaling
\pgfmathsetmacro{\t}{30};% angulo de rotacion
\pgfmathsetmacro{\c}{cos(\t)};% coseno
\pgfmathsetmacro{\s}{sin(\t)};% seno
\pgfmathsetmacro{\su}{sqrt(\c^2+(\a*\s)^2)};% ecart-type 1
\pgfmathsetmacro{\sd}{sqrt(\s^2+(\a*\c)^2)};% ecart-type 2
\pgfmathsetmacro{\dx}{3};% dominio x del plot -dx:dx
\pgfmathsetmacro{\dy}{2.5};% dominio y del plot -dy:dy
%
%
% Approximacion de la funcion Gamma
%\tikzset{declare function={gamma(\z)=
%(2.506628274631*sqrt(1/\z) + 0.20888568*(1/\z)^(1.5) + 0.00870357*(1/\z)^(2.5) -
%(174.2106599*(1/\z)^(3.5))/25920 - (715.6423511*(1/\z)^(4.5))/1244160)*exp((-ln(1/\z)-1)*\z);}}
%
% Approximation de la cdf gaussienne
\tikzmath{function normcdf(\x) {return 1/(1 + exp(-0.07056*(\x)^3 - 1.5976*(\x)));};};
%
% coefficiente binomial, para no tener factoriales muy grandes
\tikzmath{function binocoef(\m,\k) {if \k == 0 then {return 1;} else {return ((\m-\k+1)/\k)*binocoef(\m,\k-1);};};};
%
% coefficient que aparece en la pdf y cdf (ver doubling formula GraRyz 8.335-5 con x = m+1/2)
% y coefficiente de normalizacion
%\tikzset{declare function={
\tikzmath{function coefstud(\m) {return (4^\m)/(pi*sqrt(2*\m+1)*binocoef(2*\m,\m));};}
%
%
% cdf Student que se calcula recursivamente para nu = 2 m + 1, m entero
\tikzmath{function studcdfS(\x,\k) {
    if \k == 0 then {return .5+(atan(\x))/180;}
    else {return studcdfS(\x,\k-1)+((4^\k)*(\x)/(2*pi*\k*binocoef(2*\k,\k)))/((1+((\x)^2))^\k);};
};};
% Calculo de
%  - x maximo del plot para tener pdf a 7% del max
%  - la pdf Student para nu = 2 m + 1, m entero
%  - la cdf Student para nu = 2 m + 1, m entero
%\tikzset{declare function={
\tikzmath{function maxplotpdf(\m) {return sqrt((2*\m+1)*((.03^(-1/(\m+1)))-1));};};% x maximo del plot para tener pdf a 3% del max
\tikzmath{function studpdf(\x,\m) {return coefstud(\m)*((1/(1+((\x)^2)/(2*\m+1)))^(\m+1));};};% pdf Student
\tikzmath{function studcdf(\x,\m) {return studcdfS(\x/(sqrt(2*\m+1)),\m);};};% pdf Student
%}}
%
%
%
% mismas escalas x-max para cada ejemplo
\pgfmathsetmacro{\mx}{max(maxplotpdf(\mu),maxplotpdf(\md),maxplotpdf(\mt))};

% maximo de las marginales del caso 2D
\pgfmathsetmacro{\ma}{coefstud(\mdd)/min(\su,\sd)};
%
% densidad
\begin{scope}[scale=.9]
%
\pgfmathsetmacro{\sy}{2.75*sqrt(2*pi)};% y-scaling 
\draw[>=stealth,->] ({-\sx*\mx-.1},0)--({\sx*\mx+.25},0) node[right]{\small $x$};
\draw[>=stealth,->] (0,-.15)--(0,3) node[above]{\small $p_X$};
%
\draw[thick,domain=-\mx:\mx,samples=50,smooth] plot ({\x*\sx},{\sy*studpdf(\x,\mu)});
\draw[thick,dashed,domain=-\mx:\mx,samples=50,smooth] plot ({\x*\sx},{\sy*studpdf(\x,\md)});
\draw[thick,dotted,domain=-\mx:\mx,samples=50,smooth] plot ({\x*\sx},{\sy*studpdf(\x,\mt)});
\draw[thin,domain=-\mx:\mx,samples=50,smooth] plot ({\x*\sx},{\sy*exp(-.5*((\x)^2))/sqrt(2*pi)});
%
\draw (0,{\sy/sqrt(2*pi)})--(-.2,{\sy/sqrt(2*pi)}) node[left,scale=.7]{$\displaystyle \frac1{\sqrt{2 \pi}}$};
\draw (0,0)--(0,-.1) node[below,scale=.7]{$0$};
\pgfmathsetmacro{\lm}{2*floor(\mx/2)};
\foreach \m in {2,4,...,\lm} {
\draw ({-\m*\sx},0)--({-\m*\sx},-.1) node[below,scale=.7]{$-\m$};
\draw ({\m*\sx},0)--({\m*\sx},-.1) node[below,scale=.7]{$\m$};
}
%
\node at (0,-1) [scale=.9]{(a)};
\end{scope}
%
%
% reparticion
\begin{scope}[xshift=5.75cm,scale=.9]
%
\pgfmathsetmacro{\extx}{1.1};% extnsion del dominio para la cdf (que se vea mejor) 
\pgfmathsetmacro{\sy}{2.75};% y-scaling 
%
\draw[>=stealth,->] ({-\sx*\mx*\extx-.1},0)--({\sx*\mx*\extx+.25},0) node[right]{\small $x$};
\draw[>=stealth,->] (0,-.15)--(0,{\sy+.25}) node[above]{\small $F_X$};
%
% cumulativa
%
\draw[thick,domain={-\mx*\extx}:{\mx*\extx},samples=50,smooth] plot ({\x*\sx},{\sy*studcdf(\x,\mu)});
\draw[thick,dashed,domain={-\mx*\extx}:{\mx*\extx},samples=50,smooth] plot ({\x*\sx},{\sy*studcdf(\x,\md)});
\draw[thick,dotted,domain={-\mx*\extx}:{\mx*\extx},samples=50,smooth] plot ({\x*\sx},{\sy*studcdf(\x,\mt)});
\draw[thin,domain={max(-\mx*\extx,-3.5)}:{\mx*\extx},samples=50,smooth]  plot ({\x*\sx},{\sy*normcdf(\x)});
%
\draw (0,0)--(0,-.1) node[below,scale=.7]{$0$};
\draw (0,\sy)--(-.1,\sy) node[left,scale=.7]{$1$};
\pgfmathsetmacro{\lm}{2*floor(\mx*\extx/2)};
\foreach \m in {2,4,...,\lm} {
\draw ({-\m*\sx},0)--({-\m*\sx},-.1) node[below,scale=.7]{$-\m$};
\draw ({\m*\sx},0)--({\m*\sx},-.1) node[below,scale=.7]{$\m$};
}
%
\node at (0,-1) [scale=.9]{(b)};
\end{scope}
%
%
% densidad 2D
\begin{scope}[xshift=9.5cm,yshift=-2.5mm,scale=.7]
%
\begin{axis}[
    colormap = {whiteblack}{color(0cm)  = (white);color(1cm) = (black)},
    width=.45\textwidth,
    view={45}{65},
    enlargelimits=false,
    %grid=major,
    domain=-\dx:\dx,
    y domain=-\dy:\dy,
    color=black,
    samples=80,
    xlabel=$x_1$,
    ylabel=$x_2$,
    zlabel=$p_X$,
    zmax={1.05*\ma},
]
%
% Student-t 2D
\addplot3 [surf] {1/(2*pi*\a*((1+((\c*x+\s*y)^2+((-\s*x+\c*y)/\a)^2)/(2*\mdd+1))^(1.5+\mdd)))};
%
% Marginales
\pgfmathsetmacro{\cproj}{coefstud(\mdd)};
\addplot3 [domain=-\dx:\dx,samples=50, samples y=0, thick, smooth, color=black]
(x,\dy,{\cproj*((1+((x/\su)^2)/(2*\mdd+1))^(-\mdd-1))/\su});
\addplot3 [domain=-\dy:\dy,samples=50, samples y=0, thick, smooth, color=black]
(-\dx,x,{\cproj*((1+((x/\sd)^2)/(2*\mdd+1))^(-\mdd-1))/\sd});
%\addplot3 [domain=-\dx:\dx,samples=51, thick, smooth, color=black] (x,\dy,{studpdf(x/\su,\mdd)});
%\addplot3 [domain=-\dy:\dy,samples=51, samples y=0, thick, smooth, color=black] (-\dx,x,{studpdf(x/\sd,\mdd)/\sd});
%
\node at (axis cs:{\dx/5},\dy,{\cproj*((1+((\dx/5/\su)^2)/(2*\mdd+1))^(-\mdd-1))/\su})[above right]{$p_{X_1}$};
\node at (axis cs:-\dx,{\dy/5},{\cproj*((1+((\dy/5/\sd)^2)/(2*\mdd+1))^(-\mdd-1))/\sd})[above right]{$p_{X_2}$};
%
\end{axis}
%
\node at ({\dx},-1) [scale=.9]{(c)};
\end{scope}
\end{tikzpicture} \end{center}
% 
\leyenda{Ilustraci\'on  de  una  densidad  de probabilidad  Student-$t$  escalar
  estandar (a),  y la funci\'on  de repartici\'on asociada  (b) con \ $\nu  = 1$
  (linea llena), \ $\nu = 3$ (linea  guionada), \ $\nu = 7$ (linea punteada) \ y
  \ $\nu \to +\infty$ (linea llena fina; ver m\'as adelante) grados de libertad,
  as\'i que una densidad de probabilidad Student-$t$ bi-dimensional con \ $\nu =
  1$ \  grado de libertad,  centrada, y de  matriz caracter\'istica \  $\Sigma =
  R(\theta) \Delta^2  R(\theta)^t$ \ con \  $R(\theta) = \protect\begin{bmatrix}
    \cos\theta    &     -    \sin\theta\\[2mm]    \sin\theta     &    \cos\theta
    \protect\end{bmatrix}$   \   matriz   de    rotaci\'on   y   \   $\Delta   =
  \diag\left(\protect\begin{bmatrix}  1   &  a\protect\end{bmatrix}  \right)$  \
  matriz  de   cambio  de  escala,   y  sus  marginales   \  $X_1  \,   \sim  \,
  \T_\nu\left(0,\cos^2\theta +  a^2 \sin^2\theta \right)$ \  y \ $X_2  \, \sim \,
  \T_\nu\left(0,\sin^2\theta   +   a^2  \cos^2\theta   \right)$   \  (ver   m\'as
  adelante). En la figura, $a = \frac13$ \ y \ $\theta = \frac{\pi}{6}$.}
\label{Fig:MP:StudentT}
\end{figure}

Nota: el caso  \ $\nu = 1$ \  es conocido como distribuci\'on de  {\em Cauchy} o
{\em  Cauchy-Lorentz} o  {\em Lorentzian}  o  {\em Breit-Wigner}~\cite{Cau53:07,
  Cau53,  Bien53, Bie53:07,  BreWig36, Sti74,  SamTaq94, Lorentz}.   Es  un caso
particular  tambi\'en de  distrubuci\'on  $\alpha$-estables~\cite{SamTaq94}.  En
particular, una combinaci\'on lineal de variables de Cauchy independientes queda
de Cauchy. Pero, no  viola el teorema del l\'imite central del  hecho de que una
variable de Cauchy no admite covarianza.

Contrariamente al caso gaussiano, de la forma de la densidad de probabilidad, es
claro que si la matriz \ $\Sigma$ \ es diagonal, la densidad no factoriza, as\'i
que  las componentes  del vector  no son  independientes.  Este  ejemplo muestra
claramente que  la reciproca del lema~\ref{Lem:MP:IndependenciaCov}  es falsa en
general.

Sin embargo, las distribuciones Student-$t$ tienen varias propiedades notables.

\begin{lema}[Stabilidad por transformaci\'on lineal]
\label{Lem:MP:StabilidadLinealStudentT}
%
  Sea \ $X \, \sim \,  \T_\nu(m,\Sigma)$, \ $A$ \ matriz de \ $\M_{d',d}(\Rset))$
  \ con \ $d' \le d$, y de rango lleno y \ $b \in \Rset^{d'}$. Entonces
  %
  \[
  A X + b\, \sim \, \T_\nu( A m + b , A \Sigma A^t)
  \]
  %
  En particular los componentes de \ $X$ \ son student-$t$,
  %
  \[
  X_i \, \sim \, \T_\nu(m_i , \Sigma_{i,i} )
  \]
\end{lema}
\begin{proof}
  La prueba es inmediata usando  la funci\'on caracter\'istica y sus propiedades
  por  transformaci\'on lineal.  La condici\'on  sobre \  $A$ \  es  necesaria y
  suficiente para que \ $A \Sigma A^t \in P_{d'}^+(\Rset)$.
\end{proof}

\begin{lema}[V\'inculo con las distribuciones Gamma y Gausiana (mezcla Gaussiana de escala)]
\label{Lem:MP:MezclaGaussianaEscalaStudentT}
%
  Sea \ $V \sim \G\left( \frac{\nu}{2} \, ,  \, \frac{\nu}{2} \right)$ \ y \ $G \, \sim \,
  \N(0,I)$ \ independientes. Entonces
  %
  \[
  \frac{G}{\sqrt{V}} \, \sim \, \T_\nu( 0 , I )
  \]
  %
  Dicho de  otra manera, se puede escribir  \ $X \, \sim  \, \T_\nu(m,\Sigma)$ \
  esticasticamente bajo la forma \ $X \egald \frac{\Sigma^{\frac12} G}{\sqrt{V}}
  + m $ \ donde \ $\egald$ \ significa que la igualdad es en distribuci\'on.
\end{lema}
\begin{proof}
  Lo  m\'as  simple  es  de  salir  de la  formula  de  probabilidad  total  del
  teorema~\ref{Teo:MP:ProbaTotalContinuo},  notando  que  condicionalmente  a  \
  $V=v$ \ la variable es gausiana de covarianza $\frac{1}{v} I$,
  %
  \begin{eqnarray*}
  p_X(x) & = & \int_\Rset p_{X|V=v}(x) \, p_V(v) \, dv\\[2mm]
  %
  & \propto & \int_0^{+\infty} v^{\frac{d}{2}} e^{-\frac{v}{2} x^t x}
  v^{\frac{\nu}{2}-1} e^{-\frac{\nu}{2} v} \, dv\\[2mm]
  %
  & \propto & \left( 1 + \frac{x^t x}{\nu} \right)^{- \frac{d+\nu}{2}}
  \int_0^{+\infty} u^{\frac{d+\nu}{2}-1} e^{-u} \, du\\[2mm]
  %
  & \propto & \left( 1 + \frac{x^t x}{\nu} \right)^{- \frac{d+\nu}{2}}
  \end{eqnarray*}
  %
  con \  $\propto$ \  significando ``proporcional a''  (el coeficiente es  lo de
  normalizaci\'on) y el  cambio de variables $v  = \frac{2 \, u}{\nu +  x^t x} =
  \frac{\frac{2}{\nu}}{1 + \frac{x^t x}{\nu}} \, u$.
\end{proof}
%
Nota: este  lema permite tambi\'en  probar el lema~\ref{Lem:MP:StabilidadLineal}
escribiendo \ $A X + b \egald  \sqrt{\frac{\nu}{V}} A \Sigma^{\frac12} G + A m +
b$.

\begin{lema}[L\'imite Gausiana]
\label{Lem:MP:LimiteStudentTGaussiana}
%
  Sea \ $X_\nu \, \sim \, \T_\nu(m,\Sigma)$ \ vector Student-$t$ parametizado por
  \ $\nu$ \ sus grados de libertad. Entonces
  %
  \[
  X_\nu \, \limitd{\nu \to \infty} \, = \, X \, \sim \, \N(m,\Sigma)
  \]
  %
  con \ $\displaystyle \limitd{}$ \ l\'imite es en distribuci\'on.
\end{lema}
\begin{proof}
  La prueba  es inmediata tomando el  logaritmo de la  densidad de probabilidad,
  usando      la     formula      de     Stirling~\footnote{Ver      nota     de
    pie~\footref{Foot:MP:Stirling}} para  \ $\log\Gamma(z) = \left(  z - \frac12
  \right)  \log  z  -  z  +  \frac12   \log(2  \pi)  +  o(1)$  \  en  \  $z  \to
  +\infty$~\cite{Sti30, AbrSte70, GraRyz15} \ y \ $-\frac{d+\nu}{2} \log\left( 1
    +  \frac{(x-m)^t \Sigma^{-1} (x-m)}{\nu}  \right) =  -\frac{d+\nu}{2} \left(
    \frac{(x-m)^t \Sigma^{-1} (x-m)}{\nu} + o\left( \nu^{-1} \right) \right) = -
  \frac{(x-m)^t \Sigma^{-1} (x-m)}{2} + o(1)$.
\end{proof}

Las   variables  Student-$t$   tienen   varias  representaciones   estocasticas,
relacionadas a la gausiana~\cite{FanKot90, And03, KotNad04, AndKau65}:
% ej. KotNad p. 7 para la secunda
%
\begin{lema}[Relaci\'on con la distribuci\'on Gamma]\label{Lem:MP:StudentTGamma}
%
  Sea \ $V \,  \sim \, \G\left( \frac{\nu}{2} , \frac{\nu}{2} \right)$  \ y \ $G
  \,  \sim  \, \N(0,I)$  \  con  $\nu >  0$  \  y \  $G$  \  $d$-dimensional e independiente de  \
  $V$. Entonces, para $\Sigma \in P_d^+(\Rset)$ y $m \in \Rset^d$,
  %
  \[
  \frac{\Sigma^{\frac12} G}{\sqrt{V}} + m  \, \sim \, \T_\nu(m,\Sigma)
  \]
  %
\end{lema}
\begin{proof}
  Sea \ $X = \frac{G}{\sqrt{V}}$. De la nota siguiendo la tabla de caracter\'isticas
  es  necesario  y suficiente  probar  que $X  \sim  \T_\nu(0,I)$.  Ahora, de  la
  independencia tenemos
  %
  \[
  p_{G|V=v}(x)  = (2  \pi)^{-\frac{d}{2}}  v^{\frac{d}{2}} e^{-  \frac{x^t x v}{2}}
  \]
  %
  Entonces, multiplicando \ $p_{G|V=v}$ \ por \ $p_V$ \ y por marginalizaci\'on,
  obtenemos
  %
  \begin{eqnarray*}
  p_X(x) & = & \frac{\nu^{\frac{\nu}{2}}}{2^{\frac{\nu+d}{2}} \pi^{\frac{d}{2}}
  \Gamma\left( \frac{\nu}{2} \right)} \, \int_{\Rset_+} v^{\frac{\nu+d}{2}-1} \,
  e^{- \frac{x^t x + \nu}{2} \, v} \, dv\\[2mm]
  %
  & = & \frac{\nu^{\frac{\nu}{2}} \left( \nu + x^t x \right)^{-
  \frac{\nu+d}{2}}}{\pi^{\frac{d}{2}} \Gamma\left( \frac{\nu}{2} \right)} \,
  \int_{\Rset_+} u^{\frac{\nu+d}{2}-1} \, e^{- u} \, du\\[2mm]
  %
  & = & \frac{\Gamma\left( \frac{\nu+d}{2} \right)}{(\pi \nu)^{\frac{d}{2}}
  \Gamma\left( \frac{\nu}{2} \right)} \, \left( 1 + \frac{x^t x}{\nu} \right)^{-
  \frac{\nu+d}{2}}
  \end{eqnarray*}
 %
  La secunda linea viene del cambio de variables \ $u = \frac{x^t x + \nu}{2} \,
  v$  \  y la  tercera  reconociendo  en la  integral  la  funci\'on Gamma  (ver
  notaciones).
\end{proof}
%
Nota:       este       lema        permite       tambi\'en       probar       el
lema~\ref{Lem:MP:StabilidadLinealStudentT} escribiendo \ $A X + b \egald \frac{A
  \Sigma^{\frac12} G}{sqrt{V}} + A m + b$.

\begin{lema}[Relaci\'on con la distribuci\'on de Wishart]\label{Lem:MP:StudentWishart}
%
  Sea \ $W \, \sim \, \W( \Sigma^{-1} \, , \, \nu+d-1)$ \ $d \times d$ \ Wishart
  con \ $\Sigma \in P_d^+(\Rset)$, \ $Y \,  \sim \, \N(0,\nu I)$ \ con $\nu > 0$
  \ e \ $Y$ \ independiente de \  $W$. Entonces, para \ 
  %$R \in P_d^+(\Rset)$ \ y
  \ $m \in \Rset^d$,
  %
  \[
  W^{-\frac12} Y + m \, \sim \, \T_\nu\left( m , \Sigma \right)
  \]
  %
\end{lema}
\begin{proof}
  Sea \ $X = W^{-\frac12} Y$. De la nota siguiendo la tabla de caracter\'isticas
  es  necesario  y suficiente  probar  que $X  \sim  \T_\nu(0,\Sigma)$.  Ahora, de  la
  independencia tenemos
  %
  \[
  p_{X|W=w}(x)  = (2  \pi \nu)^{-\frac{d}{2}}  |w|^{\frac12} e^{-  \frac{x^t w  x}{2
      \nu}}
  \]
  %
  Denotamos  por \  $D =  \left\{ w_{ij},  \: 1  \le j  \le i  \le d  \tq  w \in
    P_d^+(\Rset) \right\}$ \ y, por abuso de  escritura, \ $dv = \prod_{ 1 \le j
    \le i \le d} dw_{ij}$.  Entonces,  multiplicando \ $p_{X|W=w}$ \ por \ $p_W$
  \ y por marginalizaci\'on, obtenemos
  % ($\propto$ significa ``proporcional  a'', i.e., olvidando el coefficiente de
  % normalizaci\'on)
  %
  \begin{eqnarray*}
  p_X(x) & = & \int_D \frac{|w|^{\frac{\nu-1}{2}} e^{- \frac{x^t w x}{2 \nu} -
  \frac12 \Tr\left( \Sigma w \right)}}{2^{\frac{d (\nu+d)}{2}} (\pi
  \nu)^{\frac{d}{2}} \left| \Sigma^{-1} \right|^{\frac{\nu+d-1}{2}} \Gamma_d \left(
  \frac{\nu+d-1}{2} \right)} \, dw\\[2mm]
  %
  & = & \frac{\Gamma\left( \frac{\nu+d}{2} \right)}{(\pi \nu)^{\frac{d}{2}}
  \Gamma\left( \frac{\nu}{2} \right)} \left| \Sigma + \frac{x x^t}{\nu}
  \right|^{-\frac{\nu+d}{2}} \left| \Sigma
  \right|^{\frac{\nu+d-1}{2}} \: \int_D \frac{|w|^{\frac{\nu+d-d-1}{2}} e^{-
  \frac12 \Tr\left( \left[ \Sigma + \frac{x x^t}{\nu} \right] w \right)}}{2^{\frac{d
  (\nu+d)}{2}} \left| \left( \Sigma + \frac{x x^t}{\nu} \right)^{-1}
  \right|^{\frac{\nu+d}{2}} \Gamma_d \left( \frac{\nu+d}{2} \right)} \, dw\\[2mm]
  %
  & = & \frac{\Gamma\left( \frac{\nu+d}{2} \right)}{(\pi \nu)^{\frac{d}{2}}
  \Gamma\left( \frac{\nu}{2} \right) \left| \Sigma \right|^{\frac12}} \: \left( 1
  + \frac{x^t \Sigma^{-1} x}{\nu} \right)^{-\frac{\nu+d}{2}} \, \int_D
  \frac{|w|^{\frac{\nu+d-d-1}{2}} e^{- \frac12 \Tr\left( \left[ I + \frac{x
  x^t}{\nu} \right] w \right)}}{2^{\frac{d (\nu+d)}{2}} \left| \left( I + \frac{x
  x^t}{\nu} \right)^{-1} \right|^{\frac{\nu+d}{2}} \Gamma_d \left( \frac{\nu+d}{2}
  \right)} \, dw
  \end{eqnarray*}
  %
  Para  \ $a, b  \in \Rset^d,  \: M  \in \M_{d,d}(\Rset)$,  en la  secunda linea
  usamos la  identidad \  $a^t M b  = \Tr(b a^t  M)$ \  y \ $\Gamma_d\left(  x -
    \frac12 \right) = \frac{\Gamma\left(  x - \frac{d}{2} \right)}{\Gamma(x)} \,
  \Gamma_d(x)$ \ (ver notaciones) y en  la tercera linea usamos $\left| \Sigma +
    \frac{x   x^t}{\nu}   \right|  =   \left|   \Sigma   \right|   \left|  I   +
    \frac{\Sigma^{-1}   x    x^t}{\nu}   \right|$   \   y    la   identidad   de
  Sylvester~\cite{Syl51} o~\cite[\S~18.1]{Har08} \ $\left| I + a b^t \right| = 1
  + b^t  a$. Se  concluye que \  $X \sim  \T_\nu(0,\Sigma)$ \ reconociendo  en el
  factor de  la integral como  la distribuci\'on \  $\T_\nu(0,\Sigma)$ \ y  en el
  integrande la  distribuci\'on de Wishart \  $\W( \left( I  + \frac{x x^t}{\nu}
  \right),\nu+d)$ \ que suma entonces a la unidad.
  %  la  formula de  Sherman-Morrison-Woodbury  $\left(  I  + \frac{x  x^t}{\nu}
  % \right)^{-1} = I$~\cite{HorJoh13, Har08}
\end{proof}

Como lo hemos introducido, la distribuci\'on Student-$t$ aparece naturalmente en
el  marco  de la  estimaci\'on,  especialmente  a  trav\'es de  la  estimaci\'on
empirica  de  la  media  y covarianza~\cite{Mui82,  GupNag99,  BilBre99,  And03,
  Seb04}:
% resp. p 80 teo 3.2.1 -- p. 92 teo 3.3.6 -- p. 87 prop. 7.1 -- p. 77 teo. 3.3.2 -- p. 63 teo. 3.1
% Nota : ver corolarios 2 y 3, p. 25 de Seber
% VER GupNag Th. 4.2.1
%
\begin{teorema}%[]
%
  Sean  \  $X_i \,  \sim  \,  \N(m,\Sigma), \:  i  =  1, \ldots  ,  n  > d-1$  \
  independientes,       y      sea       la       media      empirica       (ver
  corolario~\ref{Cor:MP:MediaEmpiricaGauss})
  %
  \[
  \overline{X} = \frac{1}{n} \sum_{i=1}^n X_i
  \]
  %
  y  la  covarianza empirica  construida  a partir  de  la  media empirica  (ver
  corolario~\ref{Cor:MP:WishartEstimacion})
  %
  \[
  \overline{\Sigma}  =  \frac{1}{n-1}  \sum_{i=1}^n  \left( X_i  -  \overline{X}
  \right) \left( X_i - \overline{X} \right)^t
  \]
  %
  Entonces:
  %
  \begin{itemize}
  \item $\overline{X} -  m \, \sim \,  \N\left( 0 \, , \,  \frac{1}{n} \, \Sigma
    \right)$ \ y  \ $\overline{\Sigma} \, \sim \, \W(  \frac{1}{n-1} \Sigma \, ,
    \, n-1 ) $ \ son independientes;
  %
  \item $\sqrt{\frac{n (n-d)}{n-1}} \: \overline{\Sigma}^{\, -\frac12} \, \left(
      \overline{X} - m \right) \, \sim \, \T_{n-d}\left( 0 \, , \, I \right)$
  \end{itemize}
\end{teorema}
%
\begin{proof}
  Se      refiera      a     los      corolarios~\ref{Cor:MP:MediaEmpiricaGauss}
  y~\ref{Cor:MP:WishartEstimacion}   por   lo  de   las   distribuciones  de   \
  $\overline{X}-m$ \ y de \ $\overline{\Sigma}$ \ respectivamente.

  A continuaci\'on,  sean \  $\widetilde{X}_i =  X_i - m$  \ y  \ $\widetilde{X}
  =        \begin{bmatrix}        \widetilde{X}_1        &       \cdots        &
    \widetilde{X}_n \end{bmatrix}$. Obviamente
  %
  \[
  \overline{\widetilde{X}} \equiv \overline{X} - m = \frac{1}{n} \, \widetilde{X} \un
  \]
  %
  con \ $\un \in  \Rset^n$ \ vector de componentes iguales a \  $1$ \ y vimos en
  la prueba del corolario~\ref{Cor:MP:WishartEstimacion} que
  %
  \[
  \overline{\Sigma} = \frac{1}{n-1} \widetilde{X} \left( I - \frac{\un \un^t}{n}
  \right) \widetilde{X}^t
  \]
  %
  $A = I - \frac{\un \un^t}{n} \in  P_n(\Rset)$ \ es idemponenta de rango 1, con
  $A  \un = 0$,  as\'i que  por diagonalizaci\'on~\cite{HorJoh13,  Bat97, Bat07}
  tenemos
  %
  \[
  A = P \begin{bmatrix} I_{n-1} & 0\\ 0 & 0 \end{bmatrix} P^t \qquad \mbox{con} \qquad
  P = \begin{bmatrix} B & \frac{1}{\sqrt{n}} \un \end{bmatrix}
  \]
  %
  $P P^t = P P^t = I$ \ y
  %
  \[
  B \in \M_{n,n-1}(\Rset) \quad \mbox{tal que} \quad  B^t B = I \: \mbox{ y } \:
  \un^t B = 0
  \]
  %
  Ahora,   poniendo   la  descomposici\'on   diagonal   de   \   $A$  \   en   \
  $\overline{\Sigma}$ \ obtenemos (ver~corolario~\ref{Cor:MP:WishartEstimacion})
  %
  \[
  \overline{\Sigma} = \frac{1}{n-1} \, Y Y^t \qquad \mbox{con} \qquad Y = \widetilde{X} B
  \]
  %
  Luego, de la gausianidad y independencia de los \ $\widetilde{X}_i$ \ tenemos,
  para \   $\widetilde{x}   =    \begin{bmatrix}   \widetilde{x}_1   &   \cdots   &
    \widetilde{x}_n \end{bmatrix} \in \M_{d,n}(\Rset)$
  %
  \begin{eqnarray*}
  p_{\widetilde{X}}(\widetilde{x}) & = & (2 \pi)^{-\frac{n d}{2}}
  |\Sigma|^{-\frac{n}{2}} \exp\left(- \frac12 \sum_{i=1}^n \widetilde{x}_i^t
  \Sigma^{-1} \widetilde{x}_i \right)\\[2mm]
  %
  & = & (2 \pi)^{-\frac{n d}{2}} |\Sigma|^{-\frac{n}{2}} \exp\left(- \frac12
  \sum_{i=1}^n \Tr\left( \Sigma^{-1} \widetilde{x}_i \widetilde{x}_i^t
  \right) \right)\\[2mm]
  %
  & = & (2 \pi)^{-\frac{n d}{2}} |\Sigma|^{-\frac{n}{2}} \exp\left(- \frac12
  \Tr\left( \Sigma^{-1} \widetilde{x} \widetilde{x}^t \right) \right)
  \end{eqnarray*}
  %
  Sea    la     transformaci\'on    \    $\begin{bmatrix}     Y    &    \sqrt{n}
    \overline{\widetilde{X}}   \end{bmatrix}   =   \widetilde{X}   P$,   \ie   \
  $\widetilde{X}        =       \begin{bmatrix}        Y        &       \sqrt{n}
    \overline{\widetilde{X}} \end{bmatrix}  P^t$.  Se nota que  \ $|P| =  1$ \ y
  por                            transformaci\'on                           (ver
  teorema~\ref{Teo:MP:TransformacionInyectivaDensidad}),    para   \    $y   \in
  \M_{d,n-1}(\Rset)$ \ y \ $x \in \Rset^d$
  %
  \begin{eqnarray*}
  p_{Y,\sqrt{n} \overline{\widetilde{X}}}(y,x) & = & (2 \pi)^{-\frac{n d}{2}}
  |\Sigma|^{-\frac{n}{2}} \exp\left(- \frac12 \Tr\left(
  \Sigma^{-1} \begin{bmatrix} y & x \end{bmatrix} P^t P \begin{bmatrix} y^t\\
  x \end{bmatrix}\right) \right)\\[2mm]
  %
  & = & (2 \pi)^{-\frac{n d}{2}} |\Sigma|^{-\frac{n}{2}} \exp\left(- \frac12
  \Tr\left( \Sigma^{-1} \left( y y^t + x x^t \right) \right) \right)\\[2mm]
  %
  & = & (2 \pi)^{-\frac{(n-1) d}{2}} |\Sigma|^{-\frac{n-1}{2}} \exp\left(-
  \frac12 \Tr\left( \Sigma^{-1} y y^t \right) \right) \times (2
  \pi)^{-\frac{d}{2}} |\Sigma|^{-\frac12} \exp\left(- \frac12 x^t \Sigma^{-1} x
  \right)
  \end{eqnarray*}
  %
  Claramente,  de la factorizaci\'on  de las  distribuciones, $Y  = X  B$ \  y \
  $\sqrt{n}  \overline{\widetilde{X}}$  \ son  independientes,  es  decir que  \
  $\frac{1}{n-1} \, Y Y^t = \overline{\Sigma}$ \ y \ $\overline{\widetilde{X}} =
  \overline{X} -  m$ \ son  independientes, lo que  cierra la prueba  del primer
  item.  Pasando, la forma  de $p_{Y,\sqrt{n}  \overline{\widetilde{X}}}(y,x)$ \
  confirma  que  \ $\overline{X}-m$  \  es  gausiana  centrada de  covarianza  \
  $\frac{1}{n} \,  \Sigma$, y  que los \  $Y_i$ \ son  independientes gausianos,
  dando la distribuci\'on  de Wishart del lema~\ref{Lem:MP:WishartGausiana} para
  la covarianza empirica.

  A continuaci\'on,
  %
  \[
  \sqrt{\frac{n   (n-d)}{n-1}}   \,   \overline{\Sigma}^{\,   -\frac12}   \left(
    \overline{X}-m \right)  = \frac{1}{\sqrt{n-1}} \:  \Sigma^{- \frac12} \left(
    \Sigma^{-1}  \, \overline{\Sigma}  \, \Sigma^{-1}  \right)^{-\frac12} \left(
    \sqrt{n (n-d)} \: \Sigma^{-\frac12} \left( \overline{X}-m \right) \right)
  \]
  %
  Del           teorema~\ref{Teo:MP:StabilidadGaussiana}          y          del
  lema~\ref{Lem:MP:StabilidadWishartLineal} tenemos
  %
  \[
  \sqrt{n (n-d)}  \: \Sigma^{-\frac12} \left( \overline{X}-m \right)  \, \sim \,
  \N(0 ,  (n-d) I)  \qquad \mbox{y} \qquad  \Sigma^{-1} \,  \overline{\Sigma} \,
  \Sigma^{-1}  \, \sim  \, \W\left(  \left( (n-1)  \Sigma \right)^{-1}  \,  , \,
    n-d+d-1 \right)
  \]
  %
  Se   cierra   la  prueba   usando   los  lemas~\ref{Lem:MP:StudentWishart}   \
  y~\ref{Lem:MP:StabilidadLineal}.
\end{proof}

%\SZ{
% VER LO QUE PASA SI overline{Sigma} si X_i y X_i-overline{X} para estandardizar los datos.

%Sampling distribution, Gosset, Fisher 25. Applications
%}

M\'as propiedades de esta distribuci\'on se encuentran en libros especializados,
por ejemplo~\cite{KotNad04} completamente dedicado a esta distribuci\'on.

\

La distribuci\'on  Student-$t$ se generaliza al  caso complejo \  $Z$ \ definido
sobre $\Cset^d$; se denota  \ $Z \, \sim \, \CT_\nu(m,\Sigma)$ \  donde \ $m \in
\Cset^d$, \ $\Sigma \in P_d^+(\Cset)$ \ y la densidad es dada por \
%
\[
p_Z(z) = \frac{\Gamma\left( d  + \frac{\nu}{2} \right)}{\pi^d \nu^d \Gamma\left(
    \frac{\nu}{2}   \right)  \,   \left|   \Sigma  \right|}   \:   \left(  1   +
  \frac{(z-m)^\dag \, \Sigma^{-1} \, (z-m)}{\nu} \right)^{- \frac{\nu}{2}-d}
\]
%
(ver  por  ejemplo~\cite[\S~5.12  y   ref.]{KotNad04}  para  una  versi\'on  muy
parecida).
% Gupta 64, Tan 73, 69b
%Se puede referirse  a~\cite[Cap.~4]{GupNag99} para tener m\'as
%detalles.

\

Tambi\'en, la  distribuci\'on Student-$t$  se generaliza al  caso matriz-variada
$X$   definido  sobre   $M_{d,d'}(\Rset)$;   se   denota  \   $X   \,  \sim   \,
\T_\nu(M,\Sigma,\Omega)$  \  donde  \  $M  \in M_{d,d'}(\Rset),  \:  \Sigma  \in
P_d^+(\Rset),  \:  \Omega  \in  P_{d'}^+(\Rset)$  y  la  densidad  es  dada  por
$\displaystyle             p_X(x)             =             \frac{\Gamma_d\left(
    \frac{\nu+d+d'-1}{2}\right)}{\pi^{\frac{\nu       d}{2}}      \Gamma_d\left(
    \frac{\nu+d-1}{2}\right)  \,  \left|  \Sigma  \right|^{\frac{d'}{2}}  \left|
    \Omega \right|^{\frac{d}{2}}} \: \left| I + \Sigma^{-1} (x-M) \, \Omega^{-1}
  \,      (x-M)^t     \right|^{-     \frac{\nu+d+d'-1}{2}}$.      Se     refiera
a~\cite{Dic67}, \cite[Cap.~4]{GupNag99} o~\cite[\S5.11   y
ref.]{KotNag04}   para   tener   m\'as   detalles.
% Dickey 66, Cornis 54