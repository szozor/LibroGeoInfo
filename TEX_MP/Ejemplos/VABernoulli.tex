\subsubseccion{Ley de Bernoulli}
\label{Sssec:MP:Bernoulli}

Esta ley aparece  cuando se hace una experiencia con dos  estados posibles, tipo un
tiro  de  moneda.   Apareci\'o en  trabajos  muy  antiguos,  entre otros  el  de
J.  Bernoulli  tratando  de  la  ley  de  gran  n\'umeros~\cite{Ber1713,  Hal90,
  DavEdw01}.

Se  denota \  $X \,  \sim \,  \B(p)$ \  con \  $p \in  [0 \;  1]$ \  y sus
caracter\'isticas son las siguientes:

\begin{caracteristicas}
%
Dominio de definici\'on & $\X = \{ 0 \; 1 \}$\\[2mm]
\hline
%
Par\'ametro & $p \in [ 0 \; 1 ]$\\[2mm]
\hline
%
Distribuci\'on de probabilidad & $p_X(1) = 1 - p_X (0) = p$\\[2mm]
\hline
%
Promedio & $ m_X = p$\\[2mm]
\hline
%
Varianza & $\sigma_X^2 = p \, (1-p)$\\[2mm]
\hline
%
\modif{Asimetr\'ia} & $\displaystyle \gamma_X =  \frac{1 - 2 \, p}{\sqrt{p \, (1-p)}}$ \quad para \ $p \notin \{ 0 \; 1 \}$ (ver m\'as adelante)\\[2mm]
\hline
%
Curtosis por exceso & $\displaystyle \widebar{\kappa}_X = \frac{1 - 6 \, p + 6
\, p^2}{p \, (1-p)}$ \quad para \ $p \notin \{ 0 \; 1 \}$ (ver m\'as adelante)\\[2mm]
\hline
%
Generadora de probabilidad & $G_X(z) = 1 - p + p z$ \quad sobre \ $\Cset$\\[2mm]
\hline
%
Generadora de momentos & $M_X(u) = 1 - p + p \, e^u$ \quad sobre \ $\Cset$\\[2mm]
\hline
%
Funci\'on caracter\'istica & $\Phi_X(\omega) = 1 - p + p \, e^{\imath \omega}$
\end{caracteristicas}


% Momentos & $ \Esp\left[ X^k \right] = p^k\\[2mm]
% Momento factorial & $\Esp\left[ (X)_k \right] = p^k \un_{\{0 \, , \, 1 \}}(k)$\\[2mm]

Su masa  de probabilidad  y funci\'on de  repartici\'on son representadas  en la
figura Fig.~\ref{Fig:MP:Bernoulli}.
%
\begin{figure}[h!]
\begin{center} \begin{tikzpicture}%[scale=.9]
\shorthandoff{>}
%
\pgfmathsetmacro{\sx}{2};% x-scaling
\pgfmathsetmacro{\r}{.05};% radius arc non continuity F_X
\pgfmathsetmacro{\p}{1/3};% probabilidad p
% masa
\begin{scope}
%
\pgfmathsetmacro{\sy}{2/max(\p,1-\p)};% y-scaling
%
\pgfmathsetmacro{\ss}{\sy*(1-\p)};
\draw[>=stealth,->] (-.5,0)--({\sx+.75},0) node[right]{\small $x$};
\draw[>=stealth,->] (0,-.15)--(0,2.5) node[above]{\small $p_X$};
%
\draw (0,-.1) node[below,scale=.7]{$0$} --(0,0);
\draw[dotted] (0,0)--(0,{\sy*(1-\p)}) node[scale=.7]{$\bullet$};
\draw (0,{\sy*(1-\p)})--(-.1,{\sy*(1-\p)}) node[left,scale=.7]{$1-p$};
%
\draw (\sx,-.1) node[below,scale=.8]{\small $1$} --(\sx,0);
\draw[dotted] (\sx,0)--(\sx,{\sy*\p}) node[scale=.7]{$\bullet$};
\draw (0,{\sy*\p})--(-.1,{\sy*\p}) node[left,scale=.7]{\small $p$};
%
\node at ({(\sx+.75)/2},-1) [scale=.9]{(a)};
\end{scope}
%
%
% reparticion
\begin{scope}[xshift=7cm]
%
\pgfmathsetmacro{\sy}{2};% y-scaling 
%
\draw[>=stealth,->] (-.5,0)--({\sx+1.5},0) node[right]{\small $x$};
\draw[>=stealth,->] (0,-.15)--(0,{\sy+.5}) node[above]{\small $F_X$};
%
\draw (0,0)--(0,-.1) node[below,scale=.7]{$0$};
\draw (\sx,0)--(\sx,-.1) node[below,scale=.7]{$1$};
\draw (0,{\sy*(1-\p)})--(-.1,{\sy*(1-\p)}) node[left,scale=.7]{$1-p$};
\draw (0,\sy)--(-.1,\sy) node[left,scale=.7]{$1$};
%
\draw[thick](-.25,0)--(0,0);
\draw ({0+\r},\r) arc (90:270:\r);
%
\draw[dotted] (0,0)--(0,{\sy*(1-\p)});
\draw[thick](0,{\sy*(1-\p)}) node[scale=.7]{$\bullet$}--(\sx,{\sy*(1-\p)});
\draw ({\sx+\r},{\r+\sy*(1-\p)}) arc (90:270:\r);
%
\draw[dotted] (\sx,{\sy*(1-\p)})--(\sx,\sy);
\draw[thick](\sx,\sy) node[scale=.7]{$\bullet$}--({\sx+1},\sy);
%\draw ({\sx+\r},{\r+\sy*(1-\p)}) arc (90:270:\r);
%
\node at ({(\sx+1.5)/2},-1) [scale=.9]{(b)};
\end{scope}
%
\end{tikzpicture} \end{center}
%
\leyenda{Ilustraci\'on de una distribuci\'on de probabilidad de Bernoulli (a), y
  la funci\'on de repartici\'on asociada (b), con $p = \frac13$.}
\label{Fig:MP:Bernoulli}
\end{figure}

Notar que cuando $p = 0$ (resp. $p = 1$) la variable es cierta $X = 0$ (resp. $X
= 1$). En estos casos, nuevamente,  siendo la varianza cero, no se puede definir
ni asimetr\'ia  (pero no  hay asimetr\'ia),  ni curtosis (pero  la ley  no tiene
colas, \ie colas livianas), como ya lo hemos visto anterioramente.

Se  notar\'a  tambi\'en  que  la   ley  de  Bernoulli  tiene  una  propiedad  de
reflexividad trivial:
%
\begin{lema}[Reflexividad]
\label{Lem:MP:ReflexividadBernoulli}
%
  Sea \ $X \, \sim \, \B(p)$. Entonces
  %
  \[
  1-X \, \sim \, \B(1-p)
  \]
  %
\end{lema}
\begin{proof}
El resultado es inmediato de $P(1-X = 1) = P(X = 0) = 1-p$.
\end{proof}