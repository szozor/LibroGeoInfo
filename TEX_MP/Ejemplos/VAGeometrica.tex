\subsubseccion{Ley Geom\'etrica}
\label{Sssec:MP:Geometrica}

Se  denota  \  $X \,  \sim  \,  \G(p)$  \  con \  $p  \in  (0  \;  1]$ \  y  sus
caracter\'isticas son las siguientes:

\begin{caracteristicas}
%
Dominio de definici\'on & $\X = \Nset^*$\\[2mm]
\hline
%
Parametro & $p \in (0 \; 1]$\\[2mm]
\hline
%
Distribuci\'on  de  probabilidad &  $\displaystyle  p_X(x)  =  (1-p)^{x-1} p$  \
(convenci\'on $0^0 = 1$)\\[2mm]
\hline
%
Promedio & $m_X = \frac1p$\\[2mm]
\hline
%
Varianza & $\displaystyle \sigma_X^2 = \frac{1-p}{p^2}$\\[2mm]
\hline
%
\modif{Sesgo} & $\displaystyle \gamma_X = \frac{2-p}{\sqrt{1-p}}$\\[2mm]
\hline
%
Curtosis por exceso & $\displaystyle \widebar{\kappa}_X = \frac{6 - 6 \, p + p^2}{1-p}$\\[2mm]
\hline
%
Generadora de  probabilidad & $\displaystyle  G_X(z) = \frac{p z}{1-(1-p)  z}$ \
para \ $|z| < \frac1{1-p}$\\[2mm]
\hline
%
Generadora de  momentos & $\displaystyle M_X(u)  = \frac{p \, e^u}{1  - (1-p) \,
e^u}$ \ para \ $\real{u} < - \ln(1-p)$\\[2mm]
\hline
%
Funci\'on caracter\'istica  & $\displaystyle \Phi_X(\omega)  = \frac{p \, e^{\imath
\omega}}{1 - (1-p) \, e^{\imath \omega}}$
\end{caracteristicas}

% Momentos & $ \Esp\left[ X^k \right] = ?$\\[2mm]
% Momento factorial & $\Esp\left[ (X)_k \right] = \frac{p^{k-1} k!}{(1-p)^k}$\\[2mm]
% Modo 1
% Mediana $\left\lceil \frac{-1}{\log_2(1-p)} \right\rceil$ 
% CDF	$1-(1-p)^k$

Su masa  de probabilidad  y funci\'on de  repartici\'on son representadas  en la
figura Fig.~\ref{Fig:MP:Geometrica}.
%
\begin{figure}[h!]
\begin{center} \begin{tikzpicture}%[scale=.9]
\shorthandoff{>}
%
\pgfmathsetmacro{\sx}{.75};% x-scaling
\pgfmathsetmacro{\r}{.05};% radius arc non continuity F_X
\pgfmathsetmacro{\p}{1/3};% probabilidad p
\pgfmathsetmacro{\n}{7};% k mas grande del plot (k in Nset^*)
%
% masa
\begin{scope}
%
\pgfmathsetmacro{\sy}{2.5/\p};% y-scaling 
\draw[>=stealth,->] (-.25,0)--({\sx*\n+.75},0) node[right]{\small $x$};
\draw[>=stealth,->] (0,-.1)--(0,{\sy*\p+.25}) node[above]{\small $p_X$};
%
\pgfmathsetmacro{\pr}{\p};% probabilidad
%
\foreach \k in {1,...,\n} {
\draw ({\k*\sx},0)--({\k*\sx},-.1) node[below,scale=.7]{\k};
\draw[dotted] ({\k*\sx},0)--({\k*\sx},{\sy*\pr}) node[scale=.7]{$\bullet$};
%
\pgfmathsetmacro{\prl}{\pr*(1-\p)};\global\let\pr\prl;% proba actualizado
}
\draw ({(\n+.5)*\sx},-.2) node[below,scale=.7]{$\ldots$};
\draw ({(\n+.5)*\sx},{(\pr/(1-\p)/2*\sy}) node[scale=.7]{$\cdots$};
\draw (0,{\p*\sy})--(-.1,{\p*\sy}) node[left,scale=.7]{$p$};
\draw (0,{\p*(1-\p)*\sy})--(-.1,{\p*(1-\p)*\sy}) node[left,scale=.7]{$p \, (1-p)$};
\draw (-.5,{\p*(1-\p)/2*\sy}) node[left,scale=.7]{$\vdots$};
%
\end{scope}
%
%
% reparticion
\begin{scope}[xshift=8.5cm]
%
\pgfmathsetmacro{\sy}{2.5};% y-scaling 
%
\draw[>=stealth,->] (-.6,0)--({\sx*\n+.75},0) node[right]{\small $x$};
\draw[>=stealth,->] (0,-.1)--(0,{\sy+.25}) node[above]{\small $F_X$};
%
\pgfmathsetmacro{\pr}{\p};% probabilidad
\pgfmathsetmacro{\c}{\p};% cumulativa
%
% cumulativa x < 1
\draw (1,0)--(1,-.1) node[below,scale=.7]{0};
\draw[thick] (-.5,0)--(\sx,0);
\draw ({\sx+\r},\r) arc (90:270:\r);
%
% cumulativa x de 1 a n
\foreach \k in {2,...,\n} {
\draw ({\k*\sx},0)--({\k*\sx},-.1) node[below,scale=.7]{\k};
\draw[thick]({(\k-1)*\sx},{\sy*\c}) node[scale=.7]{$\bullet$}--({\k*\sx},{\sy*\c});
\draw ({\k*\sx+\r},{\sy*\c+\r}) arc (90:270:\r);
\draw[dotted] ({(\k-1)*\sx},{(\c-\pr)*\sy})--({(\k-1)*\sx},{\c*\sy});
%
\pgfmathsetmacro{\prl}{\pr*(1-\p)};\global\let\pr\prl;% proba actualizado
\pgfmathsetmacro{\cl}{\c+\pr};\global\let\c\cl;% cumulativa actualizada
}
%
% cumulativa x > n
\draw ({(\n+.5)*\sx},-.2) node[below,scale=.7]{$\ldots$};
\draw ({(\n+.5)*\sx},{((\c+1)/2*\sy}) node[scale=.7]{$\cdots$};
\draw (0,{\p*\sy})--(-.1,{\p*\sy}) node[left,scale=.7]{$p$};
\draw (0,{\p*(2-\p)*\sy})--(-.1,{\p*(2-\p)*\sy}) node[left,scale=.7]{$p \, (2-p)$};
\draw (-.3,{(1+\p*(2-\p))/2*\sy}) node[left,scale=.7]{$\vdots$};
\draw (0,\sy)--(-.1,\sy) node[left,scale=.7]{$1$};
\end{scope}
%
\end{tikzpicture} \end{center}
%
\leyenda{Ilustraci\'on de una distribuci\'on de probabilidad Geom\'etrica (a), y
  la funci\'on de repartici\'on asociada (b), con $p = \frac13$.}
\label{Fig:MP:Geometrica}
\end{figure}
\SZ{Otros ilustraciones para otros $p$?}

Esta distribuci\'on  aparece en el conteo  de conteo de une  repetici\'on de una
experiencia de maneja  independiente hasta que occure un  evento de probabilidad
$p$; por ejemplo  el n\'umero de tiro de un dado  equilibriado hasta que occurre
un ``6'' sigue una ley geom\'etrica de parametro $p = \frac16$.

Esta ley esta v\'inculada con la binomial negativa, siendo un caso particular:

% En el caso \ $r = 1$ \
%aparece que \ $N \sim \G(1-p)$ dando:
%%
\begin{lema}[V\'inculo con la ley Binomial negativa]\label{Lem:MP:VinculoGeomBinoNegativa}
%
  Sea \ $X \sim \B_-(1,p)$, entonces tenemos tambi\'en
  %
  \[
  X \sim \G(1-p).
  \]
\end{lema}
%
Si volvemos  a la representation  de $X \sim  \B_-(1,p)$ como $X  = \sum_{i=1}^N
X_i$  con $X_i  \sim \B(p)$  independientes,  $N$ tal  que \  $X_N  = 0$  \ y  \
$\sum_{i=1}^N (1-X_i) = 1$, aparece que, tambi\'en, $N \sim \G(1-p)$.

Nota que cuando \  $p =  1$ \ la variable es cierta \  $X = 1$.