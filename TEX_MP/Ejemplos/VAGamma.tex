\subsubseccion{Distribuci\'on gamma}
\label{Sssec:MP:Gamma}

Como lo introdujimos en el ejemplo  de la ley exponencial, esta familia de leyes
fue  estudiada  por primera  vez  al  fin del  siglo  XIV,  bajo  el impulso  de
Pearson~\cite{Pea95}.   De hecho,  seg\'un Lancaster~\cite{Lan66}  se encuentran
trazas  de esta  ley en  trabajos de  Laplace como  posterior  distribuci\'on en
inferencia Bayesiana (elementos conduciendo a la ley gamma) para la estimaci\'on
de  la dispersi\'on  $\frac1{\sigma^2}$  de  una ley  gaussiana.   De hecho,  la
distribuci\'on gamma aparece frecuentemente en problemas de inferencia Bayesiana
como    distribuci\'on     a    priori    conjugado~\footnote{Ver     nota    de
  pie~\ref{Foot:MP:BayesPrior}  por la explicaci\'on  del enfoque  bayesiano que
  consiste  a calcular  la distribuci\'on  a  posteriori $p_{\Theta|X=x}(\theta)
  \propto p_{X|\Theta=\theta}(x) p_\Theta(\theta)$ \  usando la ley de los datos
  parametrizado  por $\theta$  que queremos  inferir, modelizado  aleatorio. Por
  eso,  como  se  lo ve  en  la  f\'ormula  de  Bayes,  se necesita  elegir  una
  distribuci\'on a priori  \ $p_\Theta$.  V\'imos que una  elecci\'on posible es
  tomarla uniforme. Puede  ser problematico por ejemplo cuando  $\theta$ vive en
  un  espacio de  volumen inifinito  (a  priori impropio),  a\'un si  se lo  usa
  frecuentemente (en estimaci\'on es equivalente a considerar la verosimulitud).
  Una otra elecci\'on posible es tomar  el a priori en una familia parametrizada
  tal que la  distribuci\'on a posterior partenece tambi\'en  a esta familia: es
  lo que se llama {\em a  priori conjugado} para la ley de sampleos $p_{X|\Theta
    = \theta}$. La idea es que  si vienen observaciones, en lugar de re-calcular
  la ley a posteriori, se  puede actualizar solamente los par\'ametros (llamados
  hiperpar\'ametros).\label{Foot:MP:BayesPriorConjugado}}     del    par\'ametro
$\lambda$ de la  ley de Poisson~\cite{Rob07}. Se encuentren  tambi\'en trazas de
esta ley en trabajos de J.  Bienaym\'e como distribuci\'on l\'imite del promedio
centrado   y  renormalizado   de   los   componentes  de   un   vector  de   ley
multinomial~\cite{Bie38, Lan66}.

Se  denota $X \,  \sim \,  \G(a,b)$ \  con \  $a \in  \Rset_{0,+}$ \  llamado {\em
par\'ametro de  forma} \ y \  $b \in \Rset_+^*$  \ llamada {\em taza}  (inversa de
{\em escala}). Las caracter\'isticas son:

\begin{caracteristicas}
%
Dominio de definici\'on & $\X = \Rset_+$\\[2mm]
\hline
%
Par\'ametros & $a \in \Rset_+^*$ \ (forma), \: $b \in \Rset_+^*$ \ (taza)\\[2mm]
\hline
%
Densidad  de probabilidad  &  $\displaystyle p_X(x)  =  \frac{b^a \, x^{a-1} \,  e^{-b
x}}{\Gamma(a)}$\\[2mm]
\hline
%
Promedio & $\displaystyle m_X = \frac{a}{b}$\\[2mm]
\hline
%
Varianza & $\displaystyle \sigma_X^2 = \frac{a}{b^2}$\\[2mm]
\hline
%
\modif{Asimetr\'ia} & $\displaystyle \gamma_X = \frac2{\sqrt{a}}$\\[2mm]
\hline
%
Curtosis por exceso & $\displaystyle \widebar{\kappa}_X = \frac6{a}$\\[2mm]
\hline
%
Generadora  de momentos  & $\displaystyle  M_X(u) =  \left( 1  - \frac{u}{b}
\right)^{-a}$ \ para \ $\real{u} < b$\\[2mm]
\hline
%
Funci\'on  caracter\'istica  &  $\displaystyle   \Phi_X(\omega)  =  \left(  1  -
\frac{ \imath \omega}{b} \right)^{-a}$
\end{caracteristicas}

% Momentos & $ \Esp\left[ X^k \right] = p^k$\\[2mm]
% Momento factorial & $\Esp\left[ (X)_k \right] = ?$\\[2mm]
% Generadora de probabilidad & $G_X(z) = e^{\lambda (z-1)}$ \ para \ $z \in \Cset$\\[2mm]
% modo max(a-1,0)
% Mediana no close ver inverse gamma

Nota: trivialmente, se puede escribir $X \,  \egald \, \frac{1}{b} G$ \ con \ $G
\, \sim \, \G(a,1)$  \ donde \ $G$ \ es estandardizada  o normalizada. De nuevo,
las  caracter\'isticas  de \  $X$  \  son  v\'inculadas a  las  de  \ $G$  \  (y
vice-versa) por transformaci\'on lineal (ver secciones anteriores).

Unas densidades de probabilidad gamma y las funciones de repartici\'on asociadas
son representadas en la figura Fig.~\ref{Fig:MP:Gamma}  para varios $a$ \ y \ $b
= 1$.
%
\begin{figure}[h!]
\begin{center} \begin{tikzpicture}%[scale=.9]
\shorthandoff{>}
%
\pgfmathsetmacro{\sx}{.75};% x-scaling
\pgfmathsetmacro{\mx}{8};% x maximo del plot
%
% Approximation de la cdf gaussienne
\tikzset{declare function={
normcdf(\x)=1/(1 + exp(-0.07056*(\x)^3 - 1.5976*(\x)));
}}
%
% densidad
\begin{scope}
%
\pgfmathsetmacro{\sy}{2.5};% y-scaling 
\draw[>=stealth,->] (-.75,0)--({\sx*\mx+.25},0) node[right]{\small $x$};
\draw[>=stealth,->] (0,-.1)--(0,2.75) node[above]{\small $p_X$};
%
%\foreach \a in {1,...,3} {
\draw[thick] (-.5,0)--(0,0);
\draw[thick,dotted,domain=.175:\mx,samples=100] plot ({\x*\sx},{\sy*(\x^(-.5))*exp(-\x)/sqrt(pi)});
\draw[thick,dashed,domain=0:\mx,samples=100] plot ({\x*\sx},{\sy*exp(-\x)});
\draw[thick,dash dot,domain=0:\mx,samples=100] plot ({\x*\sx},{\sy*\x*exp(-\x)});
%\draw[thick,domain=0:\mx,samples=100] plot ({\x*\sx},{\sy*4*\x*sqrt(\x)*exp(-\x)/3/sqrt(pi)});
\draw[thick,domain=0:\mx,samples=100] plot ({\x*\sx},{\sy*\x*\x*exp(-\x)/2});
%}
%
\draw (0,\sy)--(-.1,\sy) node[left,scale=.7]{$1$};
\draw (0,{\sy*exp(-1)})--(-.1,{\sy*exp(-1)}) node[left,scale=.7]{$e^{-1}$};
\draw (0,{\sy*2*exp(-2)})--(-.1,{\sy*2*exp(-2)}) node[left,scale=.7]{$2 \, e^{-2}$};
\draw (\sx,0)--(\sx,-.1) node[below,scale=.7]{$1$};
\draw ({2*\sx},0)--({2*\sx},-.1) node[below,scale=.7]{$2$};
%
\end{scope}
%
%
% reparticion
\begin{scope}[xshift=8.5cm]
%
\pgfmathsetmacro{\sy}{2.5};% y-scaling 
%
\draw[>=stealth,->] (-.75,0)--({\sx*\mx+.25},0) node[right]{\small $x$};
\draw[>=stealth,->] (0,-.1)--(0,{\sy+.25}) node[above]{\small $F_X$};
%
% cumulativa
\draw[thick] (-.5,0)--(0,0);
\draw[thick,dotted,domain=0:\mx,samples=100] plot ({\x*\sx},{(2*normcdf(sqrt(2*\x))-1)*\sy});
\draw[thick,dashed,domain=0:\mx,samples=100] plot ({\x*\sx},{\sy*(1-exp(-\x))});
\draw[thick,dash dot,domain=0:\mx,samples=100] plot ({\x*\sx},{\sy*(1-(1+\x)*exp(-\x))});
\draw[thick,domain=0:\mx,samples=100] plot ({\x*\sx},{\sy*(1-(1+\x+\x*\x/2)*exp(-\x))});
% plot({\x*\sx},{\sy*normcdf(\x)});
%
\draw (0,\sy)--(-.1,\sy) node[left,scale=.7]{$1$};
\end{scope}
%
\end{tikzpicture} \end{center}
%
\leyenda{Ilustraci\'on de una densidad de probabilidad gamma (a), y la funci\'on
de  repartici\'on asociada  (b).   $b  = 1$  \  y \  $a  = 0.5$  (linea
punteada), $1$ (linea mixta), $2$ (linea guionada) y $3$ (linea llena).}
\label{Fig:MP:Gamma}
\end{figure}

Cuando $a \in \Nset^*$ es entero, la ley es a veces conocida como ley de Erlang,
del nombre  de un ingeniero dan\'es  trabajando en (fundador de  la) teor\'ia de
colas~\cite{Cox62, Erl09, Erl25, BroHal48}.  Si \  $a = \frac{n}{2}$ \ con \ $n$
\ entero y  \ $\beta = \frac12$, se conoce tambi\'e  como ley {\em chi-cuadrado}
con \ $n$ \ grados de libertad (ver ej.~\cite{JohKot95:v1}).
% cf archivo queueing theory "Hillier"en mi carpeta

Notar que \ $X \, \sim \,  \G(1,b)$ \ es una variable exponencial de par\'ametro
\ $b$,  \ie \ $X  \, \sim \,  \E(b)$. Cuando \  $a < 1$,  la densidad \  $p_X$ \
diverge  para \  $x  \to 0$  \  (divergencia integrable).  Adem\'as, se  muestra
tambi\'en sencillamente con las funciones caracter\'isticas que:
%
\begin{lema}[Stabilidad]
\label{Lem:MP:StabilidadGamma}
%
  Sean $X_i  \, \sim  \, \G\left( a_i  , b  \right), \: i  = 1 ,  \ldots ,  n$ \
  independientes. Entonces
  % 
  \[
  \sum_{i=1}^n X_i \, \sim \, \G\left( \sum_{i=1}^n a_i \, , \, b \right)
  \]
\end{lema}
%
En particular, la suma de  variables independientes de ley exponencial de mismos
par\'ametro sigue una distribuci\'on de Erlang de par\'ametro de forma $n$.

Adem\'as, se  muestra sencillamente por cambio  de variables y  con la funci\'on
caracter\'istica un v\'inculo con variables gaussianas:
%
\begin{lema}[V\'inculo con la gaussiana]
\label{Lem:MP:VinculoGammaGaussiana}
%
  Sean $X_i \, \sim \,  \N\left( 0 , \sigma^2 \right), \: i = 1  , \ldots , n$ \
  independientes. Entonces
  %
  \[
  \sum_{i=1}^n  X_i^2 \,  \sim \,  \G\left( \frac{n}{2}  \, ,  \,  \frac{1}{2 \,
      \sigma^2} \right)
  \]
  %
  En  esta situaci\'on,  con  $n$ entero,  la  ley es  precisamente  la ley  del
  chi-cuadrado, con \ $n$ \ grados de libertad.
\end{lema}

\begin{ejemplo}[Distibuci\'on de Maxwell-Boltzmann]\label{Ej:MP:MaxwellBoltzmann}
  Esta distribuci\'on apareci\'o en el  estudio de las velocidades de particulas
  en el gas perfecto, bajo  el impulso de Maxwell~\cite{Max60A, Max60B, Max67} y
  m\'as tarde de  Boltzmann~\cite{Bol77, Bol96, Bol98}.  En un  gas perfecto, se
  supone que  las particulas se  mueven libremente, sin interacciones  entre si,
  aparte  colisiones  breves con  intercambio  de  energ\'ia  entre si  (choques
  elasticos).   La energia  de cada  particula es  su energ\'ia  cinetica  $\E =
  \frac12   m  \|   v  \|^2$   donde   $v  =   \begin{bmatrix}  v_x   &  v_y   &
    v_z\end{bmatrix}^t$ es  el vector velocidad  3-dimensional.  En un  gas, hay
  tantas  particulas~\footnote{`!En   condiciones  normales  de   temperatura  y
    presi\'on, un  litro contiene  \ $2.7 \times  10^{22}$ particulas!}   que es
  imposible  describir tal  gas con  las leyes  de la  m\'ecanica.   Se modeliza
  entonces  las  velocidades  como   aleatorias.   Adem\'as,  en  este  contexto
  ``perfecto'', las particulas son  supuestas independientes entre si.  Se puede
  focalizarse sobre una  particula, que representa de una  manera el conjunto de
  particulas.          Como          lo         veremos         en         ambas
  secci\'on~\ref{Ssec:MP:FamiliaExponencial}                                    y
  cap\'itulo~\ref{Cap:SZ:Informacion},   sin   v\'inculos   adicional,   en   el
  equilibrio termodin\'amico, la ley del  vector velocidad de la particula es la
  que  maximiza la  entrop\'ia. Es  precisamente una  gausiana  3-dimensional de
  covarianza  \  $\Sigma =  \frac{m}{2  \,  k_B  T} I$,  donde  \  $T$ \  es  la
  temperatura del gas en Kelvin, y  \ $k_B \approx 1.38 \times 10^{-23}$ \ julio
  por Kelvin es la constante  de Boltzmann. En otros t\'erminos, les velocidades
  en cada direcci\'on \ $v_x, v_y,  v_z$ \ son gausianas de varianza \ $\sigma^2
  = \frac{m}{2 \,  k_B T}$ independientes.  Resuelte que la ley  de la energia \
  $\E = \frac12 m \| v \|^2$ \ es precisamente una ley chi-cuadrado con 3 grados
  de libertad, conocida como ley de Maxwell-boltzmann de la energ\'ia, y la de \
  $\| v \|$, la raiz cuadrada de variable del chi-cuadrado, es conocida como ley
  chi;   en  el   caso  presente,   es   precisamente  conocida   como  ley   de
  Maxwell-Boltzmann de la velocidad.
\end{ejemplo}

%\SZ{Esta distribuci\'on aparece...}
% en  el conteo  de conteo  de  une repetici\'on  de una  experiencia de  maneja
% independiente hasta que  occure un evento de probabilidad  $p$; por ejemplo el
% n\'umero de tiro de un dado  equilibriado hasta que occurre un ``6'' sigue una
% ley geometrica de par\'ametro $p = \frac16$.