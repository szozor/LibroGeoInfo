% --------------------------------- Familia  eliptica
\subseccion{Familia eliptica real}

% --------------------
% Bilingsley p. 
% Kay p.
% Lehman & Casela p. 
% Kotz & Balakrishnan p.
% Robert p.
% van den Bos p.
% Cencov p.
% Ibarola Perez p.
% Mukhopadhyay p.

\begin{definicion}
  Sea  \  $X$  \ vector  aleatorio  $d$-dimensional  real.  $X$  \ es  dicho  de
  distribuci\'on eliptica al  torno de $m \in \Rset^d$ \ si  existe una matriz \
  $R \in P_d^+(\Rset)$ \ tal que para cualquier matriz ortogonal \ $O$,
  %
  \[
  O R^{-\frac12}  \left( X - m  \right) \: \egald  \: R^{-\frac12} \left( X  - m
  \right)
  \]
  %
  $m$  \  es  llamado  {\em  par\'ametro   de  posici\'on}  y  \  $R$  \  matriz
  caracter\'istica.
\end{definicion}
%
Se  puede  inmediatamente  ver que  \  $R$  \  es  definida mediante  un  factor
escalar. De hecho, si un  \ $R$ \ conviene, cualquier \ $a R$ \  con \ $a > 0$ \
conviene tambi\'en. Se puede  a\~nadir un v\'inculo, por ejemplo \ $Tr  R = d$ \
para que \ $R$ \ sea \'unica.

\begin{teorema}
  Sea \  $X$ \  vector aleatorio $d$-dimensional  de distribuci\'on  eliptica al
  torno  de  $m   \in  \Rset^d$  \  y  de  matriz   caracter\'istica  \  $R  \in
  P_d^+(\Rset)$. Entonces la funci\'on caract\'eristica se escribe bajo la forma
  %
  \[
  \Phi_X(\omega) = e^{\imath \, \omega^t m} \varphi_X\left( \omega^t R \omega \right)
  \]
  %
  donde  \  $\varphi_X:  \Rset_+  \mapsto  \Rset$ \  escalar,  es  llamado  {\em
    generadora  caracter\'istica}. Tomando el  logaritmo, obviamente  la secunda
  funci\'on caracteristica se escribe
  %
  \[
  \Psi_X(\omega) = \imath \, \omega^t m + \psi_X\left( \omega^t R \omega \right)
  \]
  %
  donde  \  $\psi_X = \log \varphi_X:  \Rset_+  \mapsto  \Rset$ \  escalar. La llamaremos  {\em
    secunda generadora caracter\'istica}.
\end{teorema}
%
\begin{proof}
  Sea  \ $Y  =  R^{-\frac12} \left(  X -  m  \right)$.  Por  definici\'on y  del
  teorema~\ref{Teo:MP:PropiedadesFuncionCaracteristica},  por  cualquier  matriz
  ortogonal $O$ y cualquier $\omega \in \Rset^d$
  %
  \[
  \Phi_Y(\omega) = \Phi_{O Y}(\omega) = \Phi_Y(O^t \omega)
  \]
  %
  En  otros  terminos,  la  funci\'on  caracter\'istica  queda  invariante  bajo
  cualquier transformaci\'on  ortogonal (rotaci\'on) sobre  $\omega$, y entonces
  depende solamente de  la norma euclideana de \ $\omega$.  Es decir, existe una
  funci\'on escalar \ $\varphi_X$ \ tal que
  %
  \[
  \Phi_Y(\omega) = \varphi_X(\omega^t \omega)
  \]
  %
  De nuevo, del teorema~\ref{Teo:MP:PropiedadesFuncionCaracteristica},
  %
  \[
  \Phi_X(\omega) = \Phi_{R^{\frac12}  Y + m}(\omega) = e^{\imath  \, \omega^t m}
  \Phi_Y( R^{\frac12} \omega)
  \]
  % 
  lo que cierra la prueba.
\end{proof}
%
Se  notar\'a que  si tomamos  una matriz  caracter\'istica $R$  y  la generadora
correspondiente $\varphi$, \ $a R$  \ y \ $\varphi_X\left( \frac{u}{a} \right)$ \
conviene tambi\'en, lo  que es de acuerdo con la indeterminencia  de $R$ bajo un
factor positivo.

\begin{teorema}
  Sea \  $X$ \  vector aleatorio $d$-dimensional  de distribuci\'on  eliptica al
  torno de $0$  \ y de matriz caracter\'istica \  $R \in P_d^+(\Rset)$. Entonces
  los momentos  centrales y cumulantes impares  son nulos y los  pares son dados
  por
  %
  \[
  \zeta_{i_1,\ldots,i_{2  k}}[X] =  (-2)^k \,  \varphi_X^{(k)}(0)  \sum_{\pi \in
    \Pi_{2 k , 2}} \prod_{(l,n) \in \pi} R_{i_l,i_n}
  \]
  %
  y
  %
  \[
  \kappa_{i_1,\ldots,i_{2  k}}[X]  =  (-2)^k  \, \psi_X^{(k)}(0)  \sum_{\pi  \in
    \Pi_{2 k , 2}} \prod_{(l,n) \in \pi} R_{i_l,i_n}
  \]
  %
  donde   \  $\Pi_{2   k  ,   2}$   \  es   el  conjunto   de  particiones   por
  pares~\footnote{Por ejemplo,  $\Pi_{4,2} =  \big\{ \{1,2\} ,  \{3,4\} \:  , \:
    \{1,3\} , \{2,4\} \: , \: \{1,4\}  , \{2,3\} \big\}$, un \ $\pi$ \ puede ser
    $\big\{ \{1,3\} , \{2,4\} \big\}$ \ y en el producto va a dar \ $R_{i_1,i_3}
    R_{i_2,i_4}$.} de $\{ 1 , \ldots , 2 k \}$.
\end{teorema}
%
\begin{proof}
  Recordamosnos que para \ $k \in \Nset^*, \quad (i_1 , \ldots , i_k) \in \{ 1 ,
  \ldots , d \}^k$,
  %
  \[
  \zeta_{i_1,\ldots,i_k}[X]     =     (-    \imath)^k     \left.\frac{\partial^k
      \Phi_X}{\partial          \omega_{i_1}          \cdots          \partial
      \omega_{i_k}}\right|_{\omega=0}       \qquad       \mbox{y}       \qquad
  \kappa_{i_1,\ldots,i_k}[X]     =    (-     \imath)^k    \left.\frac{\partial^k
      \Psi_X}{\partial          \omega_{i_1}          \cdots          \partial
      \omega_{i_k}}\right|_{\omega=0}
  \]
  %
  Salimos de la formula de Hardy~\cite[Prop.~1]{Har06}, generalizando la formula
  de Fa\`a di  Bruno~\cite{Faa55, Faa57}: Para \ $h(\omega)  = f\left( g(\omega)
  \right), \quad \forall \: k \in \Nset^*,  \quad \forall \: (i_1 , \ldots , i_k
  ) \in \{ 1 , \ldots , d \}^k$,
  %
  \[
  \frac{\partial^k h}{\partial \omega_{i_1} \cdots  \partial \omega_{i_k}} = \sum_{\pi \in
    \Pi_k}  f^{(|\pi|)}\left( g(\omega)  \right) \prod_{B  \in  \pi} \frac{\partial^{|B|}
    g}{\displaystyle \prod_{j \in B} \partial \omega_{i_j}}
  \]
  %
  donde \ $\Pi_k$ \  es el  conjunto de las  particiones de \ $\{ 1 ,  \ldots ,  k \}$,
  $f^{(n)}$  la $n$-esima derivada  de $f$.  En la  expresi\'on de  los momentos
  centrales y cumulantes tenemos
  %
  \[
  g(\omega) = \sum_{i,j=1}^d \omega_i \omega_i R_{i,j}
  \]
  %
  as\'i que, por simetr\'ia de $R$,
  %
  \[
  \frac{\partial  g}{\partial   \omega_{j_1}}  =  2   \sum_{l=1}^d  \omega_{j_1}
  R_{j_1,l},   \qquad   \frac{\partial^2   g}{\partial   \omega_{j_1}   \partial
    \omega_{j_2}}   =  2   R_{j_1,j_2},  \qquad   \forall   \:  k   \ge  3,   \:
  \frac{\partial^n g}{\prod_{l=1}^n \partial \omega_{j_l}} = 0
  \]
  %
  Es decir que, para $n \ge 1$,
  %
  \[
  \left.     \frac{\partial^n     g}{\prod_{l=1}^n    \partial     \omega_{j_l}}
  \right|_{\omega = 0} = \left\{\begin{array}{ccl}
  %
  2   R_{j_1,j_2} & \mbox{si} & n = 2\\[2mm]
  %
  0 & \mbox{si} & n \ne 2
  %
  \end{array}\right.
  \]
  %
  Entonces,  en la formula  de Hardy  tomada en  $\omega =  0$ quedan  solas las
  particiones  que  contienen  unicamente  pares  de indices.  Eso  da  momentos
  centrales y cumulantes nulos para $k$ impar (obvio por sim\'etria). Adema\'as,
  siendo \  $\Pi_{2 k , 2}$  \ el conjunto de  particiones por pares de  $\{ 1 ,
  \ldots , 2 k \}$, notando que necesariamente cada partici\'on de \ $\Pi_{2 k ,
    2}$ \ contiene \ $k$ \ pares,
  %
  \[
  \left.   \frac{\partial^{2  k}   h}{\partial   \omega_{i_1}  \cdots   \partial
      \omega_{i_{2  k}}}\right|_{\omega =  0} =  \sum_{\pi  \in \Pi_{2  k ,  2}}
  f^{(k)}(0) \prod_{(l,n) \in \pi} \left( 2 R_{i_l,i_n} \right)
  \]
  %
  La prueba  se cierra  tomando respectivamente  \ $f =  \varphi_X$ \  y \  $f =
  \psi_X$.
\end{proof}

\begin{corolario}
  Sea \  $X$ \  vector aleatorio $d$-dimensional  de distribuci\'on  eliptica al
  torno de $0$  \ y de matriz caracter\'istica \  $R \in P_d^+(\Rset)$. Entonces
  %
  \[
  \Sigma_X = - 2 \,  \varphi_X'(0) R
  \]
  %
  y
  %
  \[
  \kappa_X =  - \frac{  2 \, \varphi_X''(0)}{\varphi_X(0)}  \sum_{i,j=1}^d \Big(
  \left( \un_i  \un_i^t \right)  \otimes \left( \un_j  \un_j^t \right)  + \left(
    \un_i \un_j^t  \right) \otimes \left(  \un_i \un_j^t \right) +  \left( \un_i
    \un_j^t \right) \otimes \left( \un_j \un_i^t \right) \Big)
  \]
  %
\SZ{Escribir con la secunda}
\end{corolario}

\SZ{Invariante por rotacion... GSM

Ver BilBre def 13.3


Ver 1.5 de KotNad04 entre otros, Mui82, m\'as lis ref.

Hablar de nuevo de la Student-t como GSM~\cite{toto, titi, FanKot90, KotNad04}

% Salimos de Nota:  $m_X = m$, \  $\Sigma_X = 2 \varphi_X'(0) R$,  \ $\gamma_X =
% 0$,  \  $\kappa_X =  -  \frac{2 \varphi_X''(0)}{\varphi_X'(0)}  \sum_{i,j=1}^d
% \Big(  \left( \un_i  \un_i^t \right)  \otimes \left(  \un_j \un_j^t  \right) +
% \left( \un_i  \un_j^t \right)  \otimes \left( \un_i  \un_j^t \right)  + \left(
%   \un_i \un_j^t \right) \otimes \left( \un_j \un_i^t \right) \Big)$

}

\SZ{Caso complejo, con matriz unitaria

Ver Shiryayev}