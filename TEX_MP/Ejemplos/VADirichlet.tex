\subsubseccion{Distribuci\'on de Dirichlet}
\label{Sssec:MP:Dirichlet}

Esta distribuci\'on  tiene su nombre de  integrales on a  simplex estudiados por
M. Lejeune-Dirichlet  y J. Liouville  en 1839~\cite{GupRic01, Dir39,  Lio39}. Es
una extensi\'on  multivariada de las variables  beta a veces  conocida como {\em
  beta multivariada}~\cite{OlkRub64}. Escribiendo  la forma de la distribuci\'on
solamente  con  la variables  $x_i$,  la  integral  permitiendo normalizarla  es
precisamente la estudiada por Lejeune-Dirichlet y Liouville.

Se nota \ $X  \, \sim \, \Dir(a)$ \ con  \ $a \in \Rset_+^{* \, k}$ \  y \ $X$ \
vive  sobre  el  $(k-1)$-simplex   estandar  \  $\Simp{k-1}$.   $a$  es  llamado
par\'ametro de forma. Como en el caso de vectores de distribuci\'on multinomial,
a pesar de que se escribe \ $X$ \ de manera $k$-dimensional, el vector partenece
a una variedad \ $d = k-1$ \ dimensional y en el caso \ $k = 2$ \ se recupera la
ley beta. A veces  se parametriza la ley con un par\'ametro  escalar \ $\alpha >
0$ \ y un vector del simplex estandar \ $\bar{a} \in \Simp{k-1}$ \ tal que
%
\[
a  = \alpha \bar{a},  \quad \mbox{\ie}  \quad \alpha  = \sum_{i=1}^k  a_i, \quad
\bar{a} = \frac{a}{\alpha}
\]
%
$\alpha$ \  es conocido como  par\'ametro de {\em  concentraci\'on} y el  vector \
$\bar{a}$ \ como {\em medida de base}.

Las caracter\'isticas de un vector de Dirichlet son:

\begin{caracteristicas}
%
Dominio de definici\'on &
$\X = \Simp{k-1}, \: k \in \Nset \setminus \{ 0 \; 1 \}$\\[2mm]
\hline
%
Par\'ametros & $a = \alpha \, \bar{a} \, \in \, \Rset_+^{* \, k}$ \ (forma) \ con
\ $\alpha \in \Rset_{0,+}$ \ (concentraci\'on) y \ $\bar{a} \in \Simp{k-1}$
(medida de base)\\[2mm]
\hline
%
Densidad de probabilidad~\footnote{La densidad de probabilidad es dada con
respecto a la medida de Lebesgue restricta al simplex $\Simp{k-1}$.\label{Foot:MP:DirichletDensidad}} & $\displaystyle
p_X(x) = \frac{\prod_{i=1}^k x_i^{a_i-1}}{B(a)}$\\[2mm]
\hline
%
Promedio & $\displaystyle m_X = \bar{a}$\\[2.5mm]
%\frac{a}{\sum_{i=1}^k a_i} \equiv \overline{a}$\\[2.5mm]
\hline
%
Covarianza~\footnote{Ver nota de pie~\ref{Foot:MP:MultinomialCov}.} &
$\displaystyle \Sigma_X = \frac{\diag\left( \bar{a} \right) - \bar{a}
\bar{a}^t}{1 + \alpha}$\\[2.5mm]
\hline
%
Generadora de momentos  &
$\displaystyle M_X(u) = \Phi_2^{(k)}( a , \alpha \, ; \, u )$ \ para \ $u \in
\Cset$\\[2mm]
\hline
%
Funci\'on caracter\'istica & $\displaystyle
\Phi_X(\omega) = \Phi_2^{(k)}( a , \alpha \, ; \, \imath \omega )$
\end{caracteristicas}

%$k$-variada~\footnote{$\Phi_2^{(k)}(a;b;z)     =     \sum_{m    \in     \Nset^k}
%  \frac{(a_1)_{(m_1)}     \ldots    (a_k)_{(m_k)}     \,     z_1^{m_1}    \ldots
%    z_k^{m_k}}{(b)_{(m_1+\cdots+m_k)} m_1!  \ldots m_k!}$  \ con \ $(x)_{(n)}$ \
%  s\'imbolo  de Pochhammer  usual o  factorial creciente,  $(x)_{(n)} =  x (x+1)
%  \ldots (x+n)$ \,  con la convenci\'ion \ $(x)_{(0)} = 1$.   De hecho, la forma

De nuevo, se puede considerar que el vector aleatorio es \ $(k-1)$-dimensional \
$\widetilde{X}     =    \begin{bmatrix}     \widetilde{X}_1    &     \cdots    &
  \widetilde{X}_{k-1}  \end{bmatrix}^t$  \ definido  sobre  el hipertriangulo  \
$\widetilde{\X}  = \Tri_{k-1}  = \left\{  \widetilde{x} \in  [0 \;  1]^{k-1} \tq
  \sum_{i=1}^{k-1}  \widetilde{x}_i \le  1 \right\}$,  proyecci\'on  del simplex
sobre el  hiperplano \ $x_k =  0$. As\'i, \ $\widetilde{X}$,  tiene una densidad
con   respecto   a   la  medida   de   Lebesgue   usual,   y   es  dada   por   \
$p_{\widetilde{X}}\left(   \widetilde{x}   \right)   =   \frac{\prod_{i=1}^{k-1}
  \widetilde{x}_i^{\,  a_i-1}  \, \left(  1  - \sum_{i=1}^{k-1}  \widetilde{x}_i
  \right)^{a_k-1}}{B(a)}$.      A      final,     se     notar\'a      que     \
$\Phi_{\widetilde{X}}\left(         \widetilde{\omega}         \right)         =
\Phi_X\left( \begin{bmatrix} \widetilde{\omega} & 0 \end{bmatrix}^t \right)$ \ y
\ $\Phi_X(u) =  e^{\omega_k} \Phi_{\widetilde{X}}\left( \begin{bmatrix} \omega_1
    - \omega_k  & \cdots &  \omega_{k-1} - \omega_k \end{bmatrix}^t  \right)$ (y
similarmente  para $G_X$  con respecto  a $G_{\widetilde{X}}$).   Notar tambi\'en
que, la  forma de la  funci\'on generadora de  momento viene directamente  de la
escritura de las series  de Taylor de $e^{u_i x_i}$ \ o  de la forma integral de
la funci\'on confluente hipergeom\'etrica~\cite{Phi88}.

Naturalmente,  $\Sigma_X \un  = 0$  \  as\'i que  de nuevo  \ $\Sigma_X  \notin
\Pos_k^+(\Rset)$, como consecuencia directa del  hecho que \ $X$ \ $k$-dimensional,
vive  sobre   \  $\Simp{k-1}$,  $(k-1)$-dimensional.  De   nuevo,  para  definir
asimetr\'ia y  curtosis habr\'ia que  considerar \ $\widetilde{X}$,  de promedio
$\begin{bmatrix}  \bar{a}_1  &  \cdots  & \bar{a}_{k-1}  \end{bmatrix}^t$  y  de
covarianza  el  bloque  $(k-1)  \times   (k-1)$  de  $\Sigma_X$,  que  es  ahora
invertible. $\gamma_{\widetilde{X}}$  \ y \ $\kappa_{\widetilde{X}}$  \ son bien
definidos. Las expresiones, demasiado pesadas, no son dadas ac\'a.

La figura Fig.~\ref{Fig:MP:Dirichlet} representa  el dominio de definici\'on del
vector (a) y la densidad de  probabildad con las marginales (ver m\'as adelante)
para $k = 3$ y dos ejemplos de par\'ametro $a$.
%
\begin{figure}[h!]
\begin{center} \begin{tikzpicture}%[scale=.8]
\shorthandoff{>}
%
\tikzset{declare function={
xplus(\x) = max(\x,0);
%ifthenelse(\x > 0 , \x , NaN);
}}
%}

% Simplex
\tdplotsetmaincoords{45}{65}
\begin{scope}[tdplot_main_coords,scale=.75]
%
% Dirichlet: \X = S_{k-1} y \widetilde{X}
\pgfmathsetmacro{\dx}{3};% scaling
%
\draw[->,>=stealth] (-.25,0,0)--({\dx+.5},0,0) node[below right,scale=.9]{$x_1$};
%\node at (\dx,0,0)[left,scale=.8]{$1$};
\draw (\dx,0,0)--(\dx,-.15,0) node[left,scale=.8]{$1$};
%
\draw[->,>=stealth] (0,-.25,0)--(0,{\dx+.5},0) node[right,scale=.9]{$x_2$};
%\node at (0,\dx,0)[below,scale=.8]{$1$};
\draw (0,\dx,0)--(.15,\dx,0) node[below,scale=.8]{$1$};
%
\draw[->,>=stealth] (0,0,-.25)--(0,0,{\dx+.5}) node[above,scale=.9]{$x_3$};
%\node at (0,0,\dx)[left,scale=.8]{$1$};
\draw (0,0,\dx)--(0,-.15,\dx) node[left,scale=.8]{$1$};
%
\node at (0,0,0)[below left,scale=.8]{$0$};
%
% tilde X
\filldraw[fill=black!50,opacity=.5] (0,0,0)--(\dx,0,0)--(0,\dx,0);
\draw[thick,color=black,dashed] (0,0,0)--(\dx,0,0)--(0,\dx,0)--(0,0,0);
\node at ({\dx/15},{\dx/20},0)[right,scale=.7]{$\Tri_2$};
%
% Simplex Delta_2
\filldraw[fill=black!75,opacity=.5] (\dx,0,0)--(0,\dx,0)--(0,0,\dx);
\draw[thick,color=black] (\dx,0,0)--(0,\dx,0)--(0,0,\dx)--(\dx,0,0);
\node at ({.05*\dx},{.05*\dx},{.8*\dx})[right,scale=.7]{$\Simp{2}$};
%
\end{scope}
%
%
% densidad (3,2,2)
\begin{scope}[xshift=4cm,yshift=-2cm,scale=.75]
%
\pgfmathsetmacro{\au}{3};% a1
\pgfmathsetmacro{\ad}{2};% a2
\pgfmathsetmacro{\at}{2};% a3
\pgfmathsetmacro{\B}{factorial(\au-1)*factorial(\ad-1)*factorial(\at-1)/factorial(\au+\ad+\at-1)};% normalizacion
\pgfmathsetmacro{\Bu}{factorial(\au-1)*factorial(\ad+\at-1)/factorial(\au+\ad+\at-1)};% normalizacion 1
\pgfmathsetmacro{\Bd}{factorial(\ad-1)*factorial(\au+\at-1)/factorial(\au+\ad+\at-1)};% normalizacion 2
\pgfmathsetmacro{\ma}{((\au-1)^(\au-1))*((\ad-1)^(\ad-1))*((\at-1)^(\at-1))/((\au+\ad+\at-3)^(\au+\ad+\at-3))/\B};
%
% Dirichlet & marginales
\begin{axis}[
    colormap = {whiteblack}{color(0cm)  = (white);color(1cm) = (black)},
    width=.5\textwidth,
    view={45}{65},
    enlargelimits=false,
    %grid=major,
    domain=0:1,
    y domain=0:1,
    %unbounded coords=jump, % para tener un dominio no cuadrado
    %filter point/.code={%
    %\pgfmathparse
    %{\pgfkeysvalueof{/data point/x} + \pgfkeysvalueof{/data point/y} > 1.0}%
    %  \ifpgfmathfloatcomparison
    %     \pgfkeyssetvalue{/data point/x}{nan}%
    %  \fi
    %},
    zmax={.8*\ma},
    color=black,
    samples=70,
    xlabel=$x_1$,
    ylabel=$x_2$,
    zlabel=$p_{\widetilde{X}}$,
]
%
% Dirichlet
\addplot3 [surf] {(x^(\au-1))*(y^(\ad-1))*(xplus(1-x-y)^(\at-1))/\B};
%
% Marginales
\addplot3 [domain=0:1,samples=50, samples y=0, thick, smooth, color=black] (x,1,{(x^(\au-1))*((1-x)^(\ad+\at-1))/\Bu});
\addplot3 [domain=0:1,samples=50, samples y=0, thick, smooth, color=black] (0,x,{(x^(\ad-1))*((1-x)^(\au+\at-1))/\Bd});
%
\node at (axis cs:.5,1,{1/(2^(\au+\ad+\at-2))/\Bu})[right]{$p_{X_1}$};
\node at (axis cs:0,.5,{1/(2^(\au+\ad+\at-2))/\Bd})[above]{$p_{X_2}$};
\end{axis}
\end{scope}
%
%
% densidad (3,2,2)
\begin{scope}[xshift=11cm,yshift=-2cm,scale=.75]
%
\pgfmathsetmacro{\au}{3};% a1
\pgfmathsetmacro{\ad}{1};% a2
\pgfmathsetmacro{\at}{2};% a3
\pgfmathsetmacro{\B}{factorial(\au-1)*factorial(\ad-1)*factorial(\at-1)/factorial(\au+\ad+\at-1)};% normalizacion
\pgfmathsetmacro{\Bu}{factorial(\au-1)*factorial(\ad+\at-1)/factorial(\au+\ad+\at-1)};% normalizacion 1
\pgfmathsetmacro{\Bd}{factorial(\ad-1)*factorial(\au+\at-1)/factorial(\au+\ad+\at-1)};% normalizacion 2
\pgfmathsetmacro{\ma}{((\au-1)^(\au-1))*((\ad-1)^(\ad-1))*((\at-1)^(\at-1))/((\au+\ad+\at-3)^(\au+\ad+\at-3))/\B};
%
\begin{axis}[
    colormap = {whiteblack}{color(0cm)  = (white);color(1cm) = (black)},
    width=.5\textwidth,
    view={45}{65},
    enlargelimits=false,
    %grid=major,
    domain=0:1,
    y domain=0:1,
    zmax={.65*\ma},
    color=black,
    samples=70,
    xlabel=$x_1$,
    ylabel=$x_2$,
    zlabel=$p_{\widetilde{X}}$,
]
%
% Dirichlet
\addplot3 [surf,opacity=.8] {(x^(\au-1))*(y^(\ad-1))*(xplus(1-x-y)^(\at-1))/\B};
%
% Marginales
\addplot3 [domain=0:1,samples=50, samples y=0, thick, smooth, color=black] (x,1,{(x^(\au-1))*((1-x)^(\ad+\at-1))/\Bu});
\addplot3 [domain=0:1,samples=50, samples y=0, thick, smooth, color=black] (0,x,{(x^(\ad-1))*((1-x)^(\au+\at-1))/\Bd});%
%
\node at (axis cs:.5,1,{1/(2^(\au+\ad+\at-2))/\Bu})[right]{$p_{X_1}$};
\node at (axis cs:0,.5,{1/(2^(\au+\ad+\at-2))/\Bd})[above]{$p_{X_2}$};
\end{axis}
\end{scope}
%
\node at (1.2,-3){(a)};
\node at (6.6,-3){(b)};
\node at (13.6,-3){(c)};
\end{tikzpicture} \end{center}
%
\leyenda{Ilustraci\'on  del dominio $\Simp{k-1}$  de definici\'on  de la  ley de
  Dirichlet   para  \   $k  =   3$   \  (grise   oscuro),  con   el  dominio   \
  $(k-1)$-dimensional   \  $\Tri_{k-1}$   \  del   vector  \   $\widetilde{X}  =
  \protect\begin{bmatrix}   X_1  &  X_2   \protect\end{bmatrix}^t$  \   ($X_3  =
  1-X_1-X_2$) \ (grise claro) (a), y densidad de probabilidad de $\widetilde{X}$
  \ con  las marginales \  $p_{X_1}, \: p_{X_2}$.   Los par\'ametros son \  $a =
  \protect\begin{bmatrix}  3 &  2  & 2  \protect\end{bmatrix}^t$  (b) y  \ $a  =
  \protect\begin{bmatrix} 3 & 1 & 2 \protect\end{bmatrix}^t$ (c).}
\label{Fig:MP:Dirichlet}
\end{figure}


Vectores  de  distribuci\'on  de  Dirichlet tienen  tambi\'en  unas  propiedades
notables, parecidas a las de la beta:
%
\begin{lema}[Reflexividad]\label{Lem:MP:ReflexividadDir}
%
  Sea  \ $X \,  \sim \,  \Dir(a), \:  a \in  \Rset_+^{* \,  k}$ \  y \  $\Pi \in
  \Perm_k$ \ matriz \ de permutaci\'on. Entonces
  %
  \[
  \Pi X \, \sim \, \Dir\left( \Pi a \right)
  \]
  %
\end{lema}

%
\begin{proof}
  El resultado es inmediato por cambio  de variables $x \to \Pi x$, la Jacobiana
  siendo   $\Pi$,   de   valor   absoluto   determinente  igual   a   $1$   (ver
  secci\'on~\ref{Sec:MP:Transformacion}).
\end{proof}
%
Adem\'as, se muestra una stabilidad reemplazando dos componentes por su suma:
%
\begin{lema}[Stabilidad por agregaci\'on]\label{Lem:MP:StabSumaDir}
%
  Sea  \ $X =  \begin{bmatrix} X_1  & \cdots  & X_k  \end{bmatrix}^t \,  \sim \,
  \Dir(a),  \:  a =  \begin{bmatrix}  a_1 &  \cdots  &  a_k \end{bmatrix}^t  \in
  \Rset_+^{*  \,  k}$  \  y  \  $G^{(i,j)}$ \  matriz  de  agrupaci\'on  de  las
  $(i,j)$-\'esima componentes (ver notaciones). Entonces,
  %
  \[
  G^{(i,j)} X \, \sim \, \Dir\left( G^{(i,j)} a \right)  
  \]
  %
\end{lema}
%
\begin{proof}
  Se  puede probar  este resultado  a partir  de la  funci\'on caracter\'istica,
  usando      las      propiedades      de      la      funci\'on      confluent
  hipergeom\'etrica~\cite{SriKar85, Hum22, App25,  AppKam26, Erd37, Erd40}. Pero
  se  puede tambi\'en  tener un  enfoque m\'as  directo.  Del  lema precediente,
  notando  que  existen  matrices  de permutaci\'on~\footnote{$\Pi_k$  pone  las
    componentes $i$ \ e \ $j$ el  las posiciones $1$ y $2$, sin cambiar el orden
    de  las  siguientes;  $\Pi_{k-1}$  trazlada  la  primera  componente  en  la
    posici\'on $\min(i,j)$.}  \ $\Pi_k \in  \Perm_k$ \ y \ $\Pi_{k-1} \in
  \Perm_{k-1}$ \ tal que \ $G^{(i,j)} = \Pi_{k-1} \, G^{(1,2)} \, \Pi_k$,
  se puede concentrarse en el caso \ $(i,j) = (1,2)$. Sea el cambio de variables
  $g:    x    =    (x_1,\ldots,x_k)    \mapsto   u    =    (u_1,\ldots,u_k)    =
  (x_1,x_1+x_2,x_3,\ldots,x_k)$.        Entonces       \      $g^{-1}(u)       =
  (u_1,u_2-u_1,u_3,\ldots,u_k)$ \ es de determinente de matriz Jacobiana igual a
  \ $1$ \ dando para $U = g(X)$ \ la densidad
  %
  \[
  p_U(u)  = \frac{u_1^{a_1-1}  \left(  u_2 -  u_1 \right)^{a_2-1}  \prod_{i=3}^k
    u_i^{a_i-1}}{B(a)}
  \]
  %
  sobre $g\left(  \Simp{k-1} \right)$. Para $u_2  \in [0 \; 1]$  \ tenemos $u_1
  \in [  0 \;  u_2]$ \ as\'i  que, por  marginalizaci\'on en $u_1$  obtenemos la
  densidad
  %
  \begin{eqnarray*}
  p_{G^{(1,2)} X}(u_2,\ldots,u_k) & = & \frac{\prod_{i=3}^k u_i^{a_i-1}}{B(a)}
  \int_0^{u_2} u_1^{a_1-1} \left( u_2 - u_1 \right)^{a_2-1} \, du_1\\[2mm]
  %
  & = & \frac{\prod_{i=3}^k u_i^{a_i-1}}{B(a)} \, u_2^{a_1+a_2-1} \int_0^1
  v_1^{a_1-1} \left( 1 - v_1 \right)^{a_2-1} \, dv_1
  \end{eqnarray*}
  %
  con el cambio de variables $u_1 = u_2 v_1$. Se cierra la prueba notando que la
  integral   vale  \  $B(a_1,a_2)$   \  y   que  \   $\frac{B(a_1,a_2)}{B(a)}  =
  \frac{1}{B\left( G^{(1,2)} a \right)}$.
\end{proof}

De este lema, aplicado de manera recursiva, se obtiene en corolario siguiente:
%
\begin{corolario}
\label{Cor:MP:MarginalDirichletBeta}
%
  Sea  \ $X  \,  \sim \,  \Dir(a)$, entonces  \  $\displaystyle X_i  \, \sim  \,
  \beta\left( a_i \, , \, \alpha-a_i \right)$.
\end{corolario}

Naturalmente, la ley de Dirichelt siendo  una extensi\'on de la ley beta, existe
tambi\'en un v\'inculo entre esta ley y variables de distribuci\'on gamma:
%
\begin{lema}[V\'inculo con la ley gamma]
\label{Lem:MP:VinculoDirichletGamma}
%
Sea \ $X$  \ vector $k$-dimensional de componentes \ $X_i  \, \sim \, \G(a_i,c),
\:  i  = 1,  \ldots  , k$  \  independientes  \ y  \  $a$  vector de  componente
$i$-\'esima \ $a_i$. Entonces
  %
  \[
  \frac{X}{\sum_{i=1}^k X_i} \, \sim \, \Dir(a)
  \]
  %
  (independientemente  de $c$).   Adem\'as, $\frac{X}{\sum_{i=1}^k  X_i}$ \  y \
  $\sum_{i=1}^k X_i$ \ son independientes.
\end{lema}
%
\begin{proof}
  La    prueba   sigue    exactamente   los    mismos   pasos    que    la   del
  lema~\ref{Lem:MP:VinculoBetaGamma} \ trabajando con \ $\widetilde{X}$.
\end{proof}

Naturalmente,  la  distribuci\'on de  Dirichlet,  extensi\'on  de  la ley  beta,
aparece entre  otros en problema  de inferencia bayesiana como  distribuci\'on a
priori  conjugado  del  par\'ametro  $p$  de  la  ley  multinomial~\cite{Rob07},
extensi\'on de la ley binomial.

%\SZ{
%Polya urn schemes (ver Ash entre otros) , Chinese restaurant
%}

\

La distribuci\'on de Dirichlet se generaliza al caso matriz-variada $X$ definido
sobre $\P_{d,k}(\Rset)$,  conjunto de las  $k$-uplas de matrices  de $\Pos_d^+(\Rset)$
cumpliando la relaci\'on  de completud (ver notaciones); se denota  \ $X \, \sim
\, \Dir_d(a)$  \ donde \  $a \in \left(  \frac{d-1}{2} \; +\infty  \right)^k$ la
densidad  est dada por  $\displaystyle p_X(x)  = \frac{\prod_{i=1}^k  \left| x_i
  \right|^{a_i-\frac{d+1}{2}}}{B_d(a)}$.   Se  refiera a~\cite[Cap.~6]{GupNag99}
para tener m\'as detalles.
