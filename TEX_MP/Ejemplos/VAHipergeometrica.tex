\subsubseccion{Ley hipergeometrica}
\label{Sssec:MP:Hipergeometrica}

Esta ley aparece por ejemplo cuando  se hace una experiencia con una poblaci\'on
de tama\~no  \ $N$ \ (ej.  $N$ bolas en una  urna), que pueden  partenecer a dos
clases, con  \ $K$  \ estados parteneciendo  a la  primera clase (a  veces dicho
estados de excito; ej. $K$ \ bolas  negras), $N-K$ a la secunda clase, y se hace
\ $n$  \ tiros. $X$ es  el n\'umero de  tiros parteneciendo en la  primera clase
(n\'umero de excitos).

Se denota \ $X \,  \sim \, \H(N,K,n)$ \ con \ $N \in \Nset,  \quad K \in \{ 0 \;
\ldots \; N\}, \quad n \in \{ 0  \; \ldots \; N\}$ \ y sus caracter\'isticas son
las siguientes:

\begin{caracteristicas}
%
Dominio de definici\'on & $\X = \left\{ \max(0,n+K-N) \; \ldots \; \min(n,K)
\right\}$\\[2mm]
\hline
%
Parametros & $N \in \Nset$ \: (poblaci\'on)\newline $K \in \{ 0 \; \ldots \;
N\}$ \ (n\'umero de estados exitosos)\newline $n \in \{ 0 \; \ldots \; N\}$ \:
(n\'umero de tiros)\\[2mm]
\hline
%
Distribuci\'on de probabilidad & \protect$p_X(k) =
\frac{\bino{K}{k} \bino{N-K}{n-k}}{\bino{N}{n}}$\protect\\[2mm]
\hline
%
Promedio & $\displaystyle m_X = \frac{n}{N} \, K$\\[2mm]
\hline
%
Varianza & $\displaystyle \sigma_X^2 = \frac{n \, (N-n)}{N^2 (N-1)} \, K \,
(N-K)$\\[2mm]
\hline
%%
\modif{Sesgo} & $\displaystyle \gamma_X = \frac{(N - 2 \, K) (N - 2 \, n)}{N-2}
\, \sqrt{\frac{N-1}{n K (N-K) (N-n)}}$\\[2mm]
\hline
%%
Curtosis por exceso & $\displaystyle \SZ{\widebar{\kappa}_X = ...}$\\[2mm]
\hline
%
Generadora de probabilidad & $G_X(z) = \frac{\bino{N-K}{n}}{\bino{N}{n}} \:
\hypgeom{2}{1}(-n , -K ; N-K-n+1 ; z)$ \ sobre \ $\Cset$\\[2mm]
\hline
%
Generadora de momentos & $M_X(u) = \frac{\bino{N-K}{n}}{\bino{N}{n}}  \:
\hypgeom{2}{1}\left( -n , -K ; N-K-n+1 ; e^u \right)$ \ sobre \
$\Cset$\\[2mm]
\hline
%%
Funci\'on caracter\'istica  & $\Phi_X(\omega) =  \frac{\bino{N-K}{n}}{\bino{N}{n}}  \:
\hypgeom{2}{1}\left( -n , -K ; N-K-n+1 ; e^{\imath \omega} \right)$
\end{caracteristicas}

Su masa  de probabilidad  y funci\'on de  repartici\'on son representadas  en la
figura Fig.~\ref{Fig:MP:Hipergeometrica}.
%
\begin{figure}[h!]
% \begin{center} \begin{tikzpicture}[fixed point arithmetic]%[scale=.9]
\shorthandoff{>}
%
\pgfmathsetmacro{\sx}{.375};% x-scaling
\pgfmathsetmacro{\r}{.05};% radius arc non continuity F_X
%\pgfmathsetmacro{\p}{1/3};% probabilidad p
\pgfmathsetmacro{\n}{100};% numero n de la poblacion
\pgfmathsetmacro{\k}{12};% numero k de estados exitosos
\pgfmathsetmacro{\m}{40};% numero m de tiros
%
% Nota : con el fixed point, no anda min & max
% pero max(a,b) = (a+b+abs(a-b))/2  & min(a,b) = (a+b-abs(a-b))/2;
\pgfmathsetmacro{\f}{(\k+\m-abs(\k-\m))/2}; % ultimo indice de proba non nula
\pgfmathsetmacro{\F}{(\k+\m+abs(\k-\m))/2}; %
\pgfmathsetmacro{\d}{(abs(\m-\n+\k)+\m-\n+\k)/2}; % primer indice de proba non nula
%
% ultima proba non nula F (F-1) ... (F-f+1) / n (n-1) ... (n-f+1)
% finhiper(\F,\n,\f)
\tikzmath{function finhiper(\a,\b,\c) {
    if \c == 1 then {return (\a/\b);}
    else {return (\a/\b)*finhiper(\a-1,\b-1,\c-1);};
};};
%

\pgfmathsetmacro{\dn}{\d-1}; % proba nula hasta d-1
\pgfmathsetmacro{\fn}{\f+1}; % proba nula de nuevo a partid de f-1
\pgfmathsetmacro{\ui}{\f+3}; % ultimo indice dibujado
% f, f-1... hasta d => y de 0 hasta f-d & x = f-y
\pgfmathsetmacro{\finy}{\f-\d}
%
% masa
\begin{scope}
%
\pgfmathsetmacro{\sy}{10};% y-scaling
%
% proba nulas del principio 0 -> d-1
\foreach \y in {-2,...,\dn} {
\pgfmathsetmacro{\xl}{int(\y)};\global\let\x\xl;
\draw ({\sx*\x},0)--({\sx*\x},-.1) node[below,scale=.7]{$\x$};
\draw ({\sx*\x},0) node[scale=.6]{$\bullet$};
}
%
% proba nulas del fin f+1 -> ui
\foreach \y in {\fn,...,\ui} {
\pgfmathsetmacro{\xl}{int(\y)};\global\let\x\xl;
\draw ({\sx*\x},0)--({\sx*\x},-.1) node[below,scale=.7]{$\x$};
\draw ({\sx*\x},0) node[scale=.6]{$\bullet$};
}
%
\pgfmathsetmacro{\pr}{finhiper(\F,\n,\f)};% valor del ultima proba no nula
\pgfmathsetmacro{\maxp}{\pr};% proba maximal (inicializacion)
%
\foreach \y in {0,...,\finy} {
\pgfmathsetmacro{\xl}{int(\f-\y)};\global\let\x\xl;
\draw ({\sx*\x},0)--({\sx*\x},-.1) node[below,scale=.7]{$\x$};
\draw[dotted] ({\sx*\x},0)--({\sx*\x},{\sy*\pr}) node[scale=.6]{$\bullet$};
%
\pgfmathsetmacro{\prl}{\pr*\x*(\n-\k-\m+\x)/((\m-\x+1)*(\k-\x+1))};\global\let\pr\prl;% proba actualizado
\pgfmathsetmacro{\maxpl}{(abs(\pr-\maxp)+\pr+\maxp)/2};\global\let\maxp\maxpl;% proba max actualizado
}
%
%\pgfmathsetmacro{\maxpl}{ceil(100*\maxp)/100)};
\draw[>=stealth,->] ({-2*\sx-.25},0)--({\sx*\ui+.35},0) node[right]{\small $x$};
\draw[>=stealth,->] (0,-.15)--(0,{\sy*\maxp+.25}) node[above]{\small $p_X$};
\draw (0,{\sy/4})--(-.1,{\sy/4}) node[left]{\small $\frac14$};
%
\node at ({(\sx*(2+\f)+.25)/2},-1) [scale=.9]{(a)};
\end{scope}
%
%
% reparticion
\begin{scope}[xshift=8.25cm]
%
\pgfmathsetmacro{\sy}{2.5};% y-scaling 
%
\draw[>=stealth,->] ({-2*\sx-.25},0)--({\sx*\ui+.5},0) node[right]{\small $x$};
\draw[>=stealth,->] (0,-.15)--(0,{\sy+.25}) node[above]{\small $F_X$};
%
% proba nulas del principio 0 -> d-1
\foreach \y in {-2,...,\dn} {
\pgfmathsetmacro{\xl}{int(\y)};\global\let\x\xl;
\draw ({\sx*\x},0)--({\sx*\x},-.1) node[below,scale=.7]{$\x$};
}
\draw ({-2*\sx},0)--({\sx*\d},0);
\draw ({\sx*\d+\r},\r) arc (90:270:\r);
%
% proba nulas del fin f+1 -> ui
\foreach \y in {\fn,...,\ui} {
\pgfmathsetmacro{\xl}{int(\y)};\global\let\x\xl;
\draw ({\sx*\x},0)--({\sx*\x},-.1) node[below,scale=.7]{$\x$};
}
\draw ({\sx*\ui},\sy)--({\sx*(\f+1)},\sy);
%
\pgfmathsetmacro{\pr}{finhiper(\F,\n,\f)};% valor del ultima proba no nula
\pgfmathsetmacro{\cum}{1};% valor final de la cumulativa
% f, f-1... hasta d => y de 0 hasta f-d & x = f-y
\foreach \y in {0,...,\finy} {
\pgfmathsetmacro{\xl}{int(\f-\y)};\global\let\x\xl;
\draw ({\sx*\x},0)--({\sx*\x},-.1) node[below,scale=.7]{$\x$};
\draw ({\sx*(\x+1)},{\sy*\cum})--({\sx*\x},{\sy*\cum}) node[scale=.6]{$\bullet$};
\draw ({\sx*\x+\r},{\sy*(\cum-\pr)+\r}) arc (90:270:\r);
\draw[dotted] ({\sx*\x},{\sy*\cum})--({\sx*\x},{\sy*(\cum-\pr)});
%
\pgfmathsetmacro{\cuml}{\cum-\pr}\global\let\cum\cuml;% cumulativa actualizada
\pgfmathsetmacro{\prl}{\pr*\x*(\n-\k-\m+\x)/((\m-\x+1)*(\k-\x+1))};\global\let\pr\prl;% proba actualizado
}
\draw (0,\sy)--(-.1,\sy) node[left,scale=.7]{$1$};
%
\node at ({(\sx*(\ui+2)+.5)/2},-1) [scale=.9]{(b)};
\end{scope}
%
\end{tikzpicture} \end{center}
%
\leyenda{Ilustraci\'on de una distribuci\'on  de probabilidad Hipergeometrica (a), y la
  funci\'on de repartici\'on asociada (b), \SZ{con $n = 6, \quad p = \frac13$.}}
\label{Fig:MP:Hipergeometrica}
\end{figure}
\SZ{Otros ilustraciones para otros $p$? CERRAR}

% Cuando  $n  = 1$,  se  recupera  la lei  de  Bernoulli  $\B(p) \equiv  \B(1,p)$.
% Ad\'emas, se muestra  sencillamente usando la generadora de  probabilidad que
% %
% De este resultado,  se puede notar que, por  ejemplo, le distribuci\'on binomial
% aparece en el conteo de eventos independientes de misma probabilidad entre $n$.

% Tambi\'en,  la ley binomial  tiene una  propiedad de  reflexividad, consecuencia
% directa de la de Bernoulli:
% %
% \begin{lema}[Reflexividad]
% \label{Lem:MP:ReflexividadBinomial}
% %
%   Sea \ $X \, \sim \, \B(n,p)$. Entonces
%   %
%   \[
%   n-X \, \sim \, \B(n,1-p)
%   \]
%   %
% \end{lema}

% Nota que cuando $p = 0$ (resp. $p = 1$) la variable es cierta $X = 0$ (resp.  $X
% = n$).
