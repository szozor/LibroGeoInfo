\subsubseccion{Ley hipergeometrica}
\label{Sssec:MP:Hipergeometrica}

Esta ley aparece por ejemplo cuando  se hace una experiencia con una poblaci\'on
de tama\~no \  $n$ \ (ej.  $n$ bolas  en una urna), que pueden  partenecer a dos
clases, con  \ $k$ \ num\'ero  de elementos de  la primera clase (a  veces dicho
estados de excito; ej. $k$ \ bolas  negras), $n-k$ \ num\'ero de elementos de la
secunda clase, y se  hace \ $m$ \ tiros si reemplazamiento.   $X$ es el n\'umero
de  tiros parteneciendo  en la  primera clase  (n\'umero de  excitos).  Esta ley
apareci\'o en trabajos de de Moivre en 1710~\cite{Moi10, Hal90, DavEdw01}.

Se denota \ $X \, \sim \, \H(n,k,m)$ \  con \ $n \in \Nset^*$, \quad $k \in \{ 0
\;  \ldots  \;  n  \}$,  \quad  $m  \in  \{  0 \;  \ldots  \;  m  \}$  \  y  sus
caracter\'isticas son las siguientes:

\begin{caracteristicas}
%
Dominio de definici\'on & $\X = \left\{ \max(0,k+m-n) \; \ldots \; \min(k,m)
\right\}$\\[2mm]
\hline
%
Par\'ametros & $n \in \Nset^*$ \: (poblaci\'on)\newline $k \in \{ 0 \; \ldots \;
n\}$ \ (n\'umero de estados exitosos)\newline $m \in \{ 0 \; \ldots \; n\}$ \:
(n\'umero de tiros)\\[2mm]
\hline
%
Distribuci\'on de probabilidad & \protect$p_X(x) =
\frac{\smallbino{k}{x} \smallbino{n-k}{m-x}}{\smallbino{n}{m}}$\protect\\[2mm]
\hline
%
Promedio & $\displaystyle m_X = \frac{m}{n} \, k$\\[2mm]
\hline
%
Varianza~\footnote{En el caso degenerado \ $n = 1$, o \ $m = 0$, o \ $m = 1 =
n$; en ambos casos, la variable es cierta (ver fin de la
subsecci\'on).\label{Foot:MP:HipergeometricaVarianza}} & $\displaystyle
\sigma_X^2 = \left\{ \protect\begin{array}{ccc} \frac{m \, (n-m)}{n^2 (n-1)} \,
k \, (n-k) & \mbox{si} & n > 1\\[2mm] 0 & \mbox{si} & n =
1\end{array}\protect\right.$\\[2mm]
\hline
%%
%Sesgo~\footnote{Cuando \ $m \in \{ 0 \; n \}$, tenemos \ $k \in \{ 0 \; n
%\}$. Entonces, la variable es cierta (ver fin de la subsecci\'on) as\'i que no
%hay asimetr\'ia. Si \ $n = 2$, \ y \ $k=m=1 \not\in \{0 \; n \}$, tenemos $P(X =
%0) = P(X = 1) = \frac12$ sim\'etrico: de nuevo el sesgo vale cero.} &
%$\displaystyle \gamma_X = \left\{ \!\! \begin{array}{cl} \frac{(n - 2 k) (n - 2
%m)}{n-2} % \sqrt{\frac{n-1}{m k (n-k) (n-m)}} & \mbox{si} \: n \ne 2, \quad k,l
%\not\in \{ 0 \; n \} \\[2mm] 0 & \mbox{si no} \end{array} \right.$\\[2mm]
%\hline
%%
%Curtosis por exceso & $\displaystyle \widebar{\kappa}_X = $\\[2mm]
% 0 si no
%\hline
%
Generadora de probabilidad & $G_X(z) = \frac{\smallbino{n-k}{m}}{\smallbino{n}{m}} \:
\: \hypgeom{2}{1}(-m , -k ; n-m-k+1 ; z)$ \ sobre \ $\Cset$\\[2mm]
\hline
%
Generadora de momentos & $M_X(u) = \frac{\smallbino{n-k}{m}}{\smallbino{n}{k}}  \:
\: \hypgeom{2}{1}\left( -m , -k ; n-m-k+1 ; e^u \right)$ \ sobre \
$\Cset$\\[2mm]
\hline
%%
Funci\'on caracter\'istica  & $\Phi_X(\omega) =  \frac{\smallbino{n-k}{m}}{\smallbino{n}{m}}  \:
\: \hypgeom{2}{1}\left( -m , -k ; n-m-k+1 ; e^{\imath \, \omega} \right)$
\end{caracteristicas}

Su masa  de probabilidad  y funci\'on de  repartici\'on son representadas  en la
figura Fig.~\ref{Fig:MP:Hipergeometrica}.
%
\begin{figure}[h!]
\begin{center} \begin{tikzpicture}[fixed point arithmetic]%[scale=.9]
\shorthandoff{>}
%
\pgfmathsetmacro{\sx}{.375};% x-scaling
\pgfmathsetmacro{\r}{.05};% radius arc non continuity F_X
%\pgfmathsetmacro{\p}{1/3};% probabilidad p
\pgfmathsetmacro{\n}{100};% numero n de la poblacion
\pgfmathsetmacro{\k}{12};% numero k de estados exitosos
\pgfmathsetmacro{\m}{40};% numero m de tiros
%
% Nota : con el fixed point, no anda min & max
% pero max(a,b) = (a+b+abs(a-b))/2  & min(a,b) = (a+b-abs(a-b))/2;
\pgfmathsetmacro{\f}{(\k+\m-abs(\k-\m))/2}; % ultimo indice de proba non nula
\pgfmathsetmacro{\F}{(\k+\m+abs(\k-\m))/2}; %
\pgfmathsetmacro{\d}{(abs(\m-\n+\k)+\m-\n+\k)/2}; % primer indice de proba non nula
%
% ultima proba non nula F (F-1) ... (F-f+1) / n (n-1) ... (n-f+1)
% finhiper(\F,\n,\f)
\tikzmath{function finhiper(\a,\b,\c) {
    if \c == 1 then {return (\a/\b;}
    else {return (\a/\b)*finhiper(\a-1,\b-1,\c-1);};
};};
%

\pgfmathsetmacro{\dn}{\d-1}; % proba nula hasta d-1
\pgfmathsetmacro{\fn}{\f+1}; % proba nula de nuevo a partid de f-1
\pgfmathsetmacro{\ui}{\f+3}; % ultimo indice dibujado
% f, f-1... hasta d => y de 0 hasta f-d & x = f-y
\pgfmathsetmacro{\finy}{\f-\d}
%
% masa
\begin{scope}
%
\pgfmathsetmacro{\sy}{10};% y-scaling
%
% proba nulas del principio 0 -> d-1
\foreach \y in {-2,...,\dn} {
\pgfmathsetmacro{\xl}{int(\y)};\global\let\x\xl;
\draw ({\sx*\x},0)--({\sx*\x},-.1) node[below,scale=.7]{$\x$};
\draw ({\sx*\x},0) node[scale=.6]{$\bullet$};
}
%
% proba nulas del fin f+1 -> ui
\foreach \y in {\fn,...,\ui} {
\pgfmathsetmacro{\xl}{int(\y)};\global\let\x\xl;
\draw ({\sx*\x},0)--({\sx*\x},-.1) node[below,scale=.7]{$\x$};
\draw ({\sx*\x},0) node[scale=.6]{$\bullet$};
}
%
\pgfmathsetmacro{\pr}{finhiper(\F,\n,\f)};% valor del ultima proba no nula
\pgfmathsetmacro{\maxp}{\pr};% proba maximal (inicializacion)
%
\foreach \y in {0,...,\finy} {
\pgfmathsetmacro{\xl}{int(\f-\y)};\global\let\x\xl;
\draw ({\sx*\x},0)--({\sx*\x},-.1) node[below,scale=.7]{$\x$};
\draw[dotted] ({\sx*\x},0)--({\sx*\x},{\sy*\pr}) node[scale=.6]{$\bullet$};
%
\pgfmathsetmacro{\prl}{\pr*\x*(\n-\k-\m+\x)/((\m-\x+1)*(\k-\x+1))};\global\let\pr\prl;% proba actualizado
\pgfmathsetmacro{\maxpl}{(abs(\pr-\maxp)+\pr+\maxp)/2};\global\let\maxp\maxpl;% proba max actualizado
}
%
\draw[>=stealth,->] ({-2*\sx-.25},0)--({\sx*\ui+.35},0) node[right]{\small $x$};
\draw[>=stealth,->] (0,-.15)--(0,{\sy*\maxp+.25}) node[above]{\small $p_X$};
%\draw (0,{((1-\p)^\n)*\sy})--(-.1,{((1-\p)^\n)*\sy}) node[left,scale=.7]{$(1-p)^n$};
%\draw (0,{\n*\p*((1-\p)^(\n-1))*\sy})--(-.1,{\n*\p*((1-\p)^(\n-1))*\sy}) node[left,scale=.7]{$n \, p \, (1-p)^{n-1}$};
%
\node at ({(\sx*(2+\f)+.25)/2},-1) [scale=.9]{(a)};
\end{scope}
%
%
% reparticion
\begin{scope}[xshift=8.25cm]
%
\pgfmathsetmacro{\sy}{2.5};% y-scaling 
%
\draw[>=stealth,->] ({-2*\sx-.25},0)--({\sx*\ui+.5},0) node[right]{\small $x$};
\draw[>=stealth,->] (0,-.15)--(0,{\sy+.25}) node[above]{\small $F_X$};
%
% proba nulas del principio 0 -> d-1
\foreach \y in {-2,...,\dn} {
\pgfmathsetmacro{\xl}{int(\y)};\global\let\x\xl;
\draw ({\sx*\x},0)--({\sx*\x},-.1) node[below,scale=.7]{$\x$};
}
\draw ({-2*\sx},0)--({\sx*\d},0);
%
% proba nulas del fin f+1 -> ui
\foreach \y in {\fn,...,\ui} {
\pgfmathsetmacro{\xl}{int(\y)};\global\let\x\xl;
\draw ({\sx*\x},0)--({\sx*\x},-.1) node[below,scale=.7]{$\x$};
}
\draw ({\sx*\ui},\sy)--({\sx*(\f+1)},\sy);
%
\pgfmathsetmacro{\pr}{finhiper(\F,\n,\f)};% valor del ultima proba no nula
\pgfmathsetmacro{\cum}{1};% valor final de la cumulativa
% f, f-1... hasta d => y de 0 hasta f-d & x = f-y
\foreach \y in {0,...,\finy} {
\pgfmathsetmacro{\xl}{int(\f-\y)};\global\let\x\xl;
\draw ({\sx*\x},0)--({\sx*\x},-.1) node[below,scale=.7]{$\x$};
\draw ({\sx*(\x+1)},{\sy*\cum})--({\sx*\x},{\sy*\cum}) node[scale=.6]{$\bullet$};
\draw ({\sx*\x+\r},{\sy*(\cum-\pr)+\r}) arc (90:270:\r);
\draw[dotted] ({\sx*\x},{\sy*\cum})--({\sx*\x},{\sy*(\cum-\pr)});
%
\pgfmathsetmacro{\cuml}{\cum-\pr}\global\let\cum\cuml;% cumulativa actualizada
\pgfmathsetmacro{\prl}{\pr*\x*(\n-\k-\m+\x)/((\m-\x+1)*(\k-\x+1))};\global\let\pr\prl;% proba actualizado
}
%\draw (0,{((1-\p)^\n)*\sy})--(-.1,{((1-\p)^\n)*\sy}) node[left,scale=.7]{$(1-p)^n$};
%\draw (0,{\n*\p*((1-\p)^(\n-1))*\sy})--(-.1,{\n*\p*((1-\p)^(\n-1))*\sy}) node[left,scale=.7]{$n \, p \, (1-p)^{n-1}$};
%
\node at ({(\sx*(\ui+2)+.5)/2},-1) [scale=.9]{(b)};
\end{scope}
%
\end{tikzpicture} \end{center}
%
\leyenda{Ilustraci\'on  de una  distribuci\'on  de probabilidad  Hipergeometrica
  (a), y la funci\'on de repartici\'on asociada (b), con \ $n = 100$, \quad $k =
  12$, \quad $m = 40$.}
\label{Fig:MP:Hipergeometrica}
\end{figure}

\SZ{Otros ilustraciones para otros $n, k, m$?

  Poner el  Sesgo (ya  lo tengo)?  El Curtosis (lo  tengo que  simplificar)? muy
  pesadas...      Momento      factorial      $f_q      =      \frac{\PocD{m}{q}
    \PocD{k}{q}}{\PocD{n}{q}}$ permitiendo calcular todo.}

Notar: la variable resuelta cierta en los casos siguientes
%
\begin{itemize}
\item  $m = 0  \: \Rightarrow  \: X  = 0$:  no se  sortean elementos,  as\'i que
  siempre se sortea $0$ elementos de la primera clase;
%
\item $m =  n \: \Rightarrow \: X =  k$: si se sortean todos los  elementos de la
  poblaci\'on, se sortean todos los \ $k$ \ de la primera clase;
%
\item $k = 0 \: \Rightarrow \: X = 0$: si la primera clase no tiene elementos, no
  se puede tirar elementos de esta clase;
%
\item $k = n  \: \Rightarrow \: X = m$: al rev\'es si  la secunda clase no tiene
  elementos, todos los sorteados partenecen a la primera clase.
\end{itemize}

La ley tiene propiedades de reflexividad del mismo tipo que para la ley binomial:
%
\begin{lema}[Reflexividad]
\label{Lem:MP:ReflexividadHipergeometrica}
%
  Sea \ $X \, \sim \, \H(n,k,m)$. Entonces
  %
  \[
  m-X \, \sim \, \H(n,n-k,m) \qquad \mbox{y} \qquad k-X \sim \H(n,k,n-m)
  \]
  %
\end{lema}
%
Se puede ver  que si en una urna con  bolas negras y blancas, con  \ $k$ \ bolas
negras, y \ $X$  \ es el n\'umero de bolas negras  sorteadas, $m-X$ \ representa
las bolas blancas sorteadas. Es decir que  en \ $m-X$ \ se intercambia los roles
de las bolas  negras y blancas.  De la misma manera,  $k-X$ representa las bolas
negras que quedan en la urna, entre las  \ $n-m$ \ que quedan, es decir que en \
$k-X$ \ se intercambia  los roles de las bolas sorteadas y  las que quedan en la
urna. M\'as formalmente:
%
\begin{proof}
  El  primer   resultado  es  inmediato  de  $P(m-X   =  x)  =  P(X   =  m-x)  =
  \frac{\smallbino{k}{m-x} \smallbino{n-k}{x}}{\smallbino{n}{m}}$. El secundo de
  $P(k-X    =     x)    =     P(X    =    k-x)     =    \frac{\smallbino{k}{k-x}
    \smallbino{n-k}{m-k+x}}{\smallbino{n}{m}}      =      \frac{\smallbino{k}{x}
    \smallbino{n-k}{n-m-x}}{\smallbino{n}{n-m}}$  notando   que  $\bino{a}{b}  =
  \bino{a}{a-b}$.
\end{proof}

% Cuando  $n  = 1$,  se  recupera  la lei  de  Bernoulli  $\B(p) \equiv  \B(1,p)$.
% Ad\'emas, se muestra  sencillamente usando la generadora de  probabilidad que
% %
% De este resultado,  se puede notar que, por  ejemplo, le distribuci\'on binomial
% aparece en el conteo de eventos independientes de misma probabilidad entre $n$.

% Tambi\'en,  la ley binomial  tiene una  propiedad de  reflexividad, consecuencia
% directa de la de Bernoulli:
% %
% \begin{lema}[Reflexividad]
% \label{Lem:MP:ReflexividadBinomial}
% %
%   Sea \ $X \, \sim \, \B(n,p)$. Entonces
%   %
%   \[
%   n-X \, \sim \, \B(n,1-p)
%   \]
%   %
% \end{lema}

% Nota que cuando $p = 0$ (resp. $p = 1$) la variable es cierta $X = 0$ (resp.  $X
% = n$).
