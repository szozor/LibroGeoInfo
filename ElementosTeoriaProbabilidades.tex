%%Mariela:
% \cup = union, \cap = interseccion
%% 

\capitulo{Elementos de teor\'ia de probabilidades}{}
%Mariela A. Portesi}
\label{Cap:MP:TeoriaProbabilidades}

% Epigrafe de capitulo
\begin{epigrafe}
%{\flushright \emph{
  While writing my book I had an argument with Feller.\\
  He  asserted that everyone  said ``random  variable''\\
  and I asserted that  everyone said ``chance variable.''\\
  We obviously had to use the same name in our books,\\
  so we decided the issue  by a stochastic procedure.\\
  That is, we tossed for it and he won.
%
  \autortituloepigrafe{J. L. Doob, Statistical Science (1953)}
\end{epigrafe}
% \\ J. L. DOOB (cita del libro   \emph{Statistical Science}, 1953)
%\\ }


\SZ{Anadir por lo menos un poco
\begin{itemize}
\item Las  nociones de convergencia  (aparece en casos  limites de leyes,  en el
  TCL).       Ver~\cite[Cap.~2.8,       6]{AshDol99},      \cite[Cap.~5]{Bil12},
  \cite[Sec.~9,~p.~287]{AthLah06},                       \cite[Cap.~6]{BroDav87},
  \cite[Caps.~17,~18]{JacPro03}.
%
%\item      Ver~\cite[Cap.~7]{AshDol99},       \cite[Sec.~29,~p.~409]{Bil12},
%  \cite[Sec.~11,~p.~344]{AthLah06},                      \cite[Cap.~6]{BroDav87},
%  \cite[Cap.~21]{JacPro03}
%https://fr.wikipedia.org/wiki/Loi_normale
%
\item Definicion de los cumulentes (ver momentos)
%
\item Cotas de Chernoff con la MGF o PGF
%
\item Descomposici\'on de Bartlett caso matriz variado
%
\item Dibujar en el plano complejo $\Phi_X$? Unas curvas son lindas.
%
\item Ejemplos: von Mises y vonMises-Fisher? Cantor (singular...)?
%
\item Hablar  de simulaci\'on  (inversion OK; mezcla?  rejecci\'on? multivariada
  via la condicional?) VEr Kotz vol 2 por ejemplo
\end{itemize}
}

% ================================ Introduccion ================================ %

%\seccion{Introducci\'on}
\label{Sec:MP:Introduccion}

\SZ{Hacer (ver cap. 2 si conviene, con notas historicas).}

% =============================== Probabilidades =============================== %

%\seccion{Probabilidades}
\label{s:MP:Probabilidad}

El  concepto  de  {\it  probabilidad}  es importante  en  situaciones  donde  el
resultado (o {\it outcome}) de un  dado proceso o medici\'on es incierto, cuando
la salida de una experiencia no  es totalmente previsible. La probabilidad de un
evento es una medida que se asocia con cu\'an probable es el evento o resultado.

Una  definici\'on de  probabilidad puede  obtenerse en  base a  la enumeraci\'on
exhaustiva de los resultados posibles de un experimento o proceso,
%lo que no siempre es factible
suponiendo que el conjunto de posibilidades es completo en el sentido de que una
de  ellas debe ser  verdad. Si  el proceso  tiene $K$  resultados distinguibles,
mutuamente  excluyentes e  igualmente probables  (esto  es, no  se prefiere  una
posibilidad frente a  otras), y si $k$  de esos $K$ tienen un  dado atributo, la
probabilidad asociada a dicho atributo en un dado procesos es $\frac{k}{K}$. Por
ejemplo, sorteando un n\'umero entre los  naturales del 1 al 10, la probabilidad
de ``obtener un n\'umero par'' es $\frac5{10} = \frac12$.

Otra  definici\'on  de  probabilidad  se  basa  en  la  frecuencia  relativa  de
ocurrencia  de  un evento.   Si  en  una cantidad  $K$  muy  grande de  procesos
independientes  cierto   atributo  aparece  $k$   veces,  se  identifica   a  la
probabilidad  asociada a  un  proceso o  ensayo  con la  frecuencia relativa  de
ocurrencia    $\frac{k}{K}$    del    atributo~\cite[\&   Ref.]{Bra76,    Hal90,
  ShaVov06}~\footnote{A pesar de que la  noci\'on de azar (viniendo del arabe) o
  de  alea  (en  latin)  es   muy  antiguo,  el  italiano  Gerolamo  Cardano  es
  ``probablemente'' un de los  primeros tratando matematicamente del concepto de
  probabilidad en el siglo XVI, escribiando un libro sobre los juegos de azar en
  1564~\cite{Bel05}  o~\cite[Cap.~4]{Hal90}.   Entre  los numerosos  matematicos
  desarollando  la teoria  de las  probabildades, (en  particular  los franceses
  Pierre de  Fermat y Blaise  Pascal~\cite[Cap.~5]{Hal90}) hay que  mencionar el
  suizo  Jacob Bernoulli  y el  franc\'es Abraham  de Moivre,  quizas un  de los
  primeros  llevandos un  aporte importante  al desarollo  de la  teoria  de las
  probabilidades  en  el  siglo  XVIII   a  trav\'es  de  este  punto  de  vista
  ``frequencista''  y combinatorial~\cite[en  latin]{Ber1713} o~\cite{Ber1713:2,
    Dem56} y~\cite[Cap.~13, 15 \&~22]{Hal90}.}.
%lim N->infty n/N indef.

Los axiomas de Kolmogorov~\footnote{Un paso importante es debido a Kolmogorov en
  1933  que  se apoy\'o  sobre  trabajos  de  Richard von  Mises~\cite{Mis32}  y
  tambi\'en sobre  la teoria de  la medida y  de la integraci\'ion  debido entre
  otros a Emile Borel y Henri-L\'eon Lebesgues~\cite{Bor98, Bor09, Leb04, Leb18,
    Hal50}    para    formalizar    anal\'iticamente    la   teoria    de    las
  probabilidades~\cite{Kol56,    BarNov78,   JacPro03}.}    proveen   requisitos
suficientes  para  determinar completamente  las  propiedades  de  la medida  de
probabilidad $P(A)$  que se puede asociar a  un evento $A$ entre  un conjunto de
resultados o eventos de un proceso.

Llamemos $\Omega$ al {\it espacio  muestral} o {\it espacio fundamental}, que es
el espacio de  {\it muestras (outcomes en ingl\'es)} \  $\omega \in \Omega$.  Se
asocia \ $\A$ \ una colecci\'on  de conjuntos de \ $\Omega$, donde los elementos
de $\A$ son llamados {\it eventos}.
% total de eventos.
Por ejemplo, $\Omega$ puede  ser las caras de un dado de  6 caras (los n\'umeros
naturales del 1 al  6, o las letras {\it a, b, c, d,  e, f}, u otro etiquetado),
$\A$ teniendo los eventos
% si $A$ es el evento
$A$ ``es  un n\'umero  natural par''  y $B$ indicando  ``es un  n\'umero natural
impar''.
% ,  el espacio  muestral  $\Omega  = \{  A  , B  \}$  indica  ``es un  n\'umero
% natural'';
En el caso de analizar el tiempo de vida de un aparato, $\Omega \equiv \Rset_+$.
% en el  lanzamiento de un  dado de 6  caras es $\Omega$  es el conjunto  de las
% etiquetas que se asigne a cada una de las caras (los n\'umeros naturales del 1
% al 6, o las letras {\it a, b, c, d, e, f}, u otro etiquetado).
El  conjunto  de  resultados  posibles  se  supone  conocido,  a\'un  cuando  se
desconozca de antemano el resultado de una prueba.

Entre  los eventos  se  pueden considerar  operaciones  an\'alogas a  las de  la
teor\'ia de  conjuntos (ej.~\cite{Spi76,  Bre88, ManWol95, Sie75,  Sie76, Bor98,
  Bor09}):
%
\begin{itemize}
\item Combinaci\'on  o uni\'on de  eventos: \ $A  \cup B$, implicando que  se da
  $A$, \'o $B$, o ambos (ej. por  un dado, $A$ eventos ``cara par'' y $B$ evento
  ``cara menor o igual a 3'' tal que $A \cup B = \{1 \, , \, 2 \, , \, 3 \, , \,
  4 \,  , \, 6\}$); Seg\'un la  literatura, se denota a  veces \ $A+B$ \  o \ $A
  \wedge B$.
%
\item  Intersecci\'on de  eventos:  \ $A\cap  B$,  implicando que  se dan  ambos
  $A$~y~$B$ (con  el ejemplo precediente,  $A \cap  B = \{  2 \}$); Se  denota a
  veces \ $(A,B)$ \ o \ $A \vee B$.
%
\item Complemento de un evento: \ $\bar{A}$ e indica que no se da $A$; Se denota
  a veces \ $-A$  \ o \ $A^c$ (con el ejemplo precediente, $\bar{A}  = \{ 1 \, ,
  \, 3 \, , \, 5 \}$).
%
\item  Eventos  {\it  disjuntos}  o   {\it  mutuamente  excluyentes}  o  {\it  o
    incompatibles}: \ son aquellos que no se  superponen, se anota \ $A \cap B =
  \emptyset$  \ donde  \  $\emptyset =  \bar{\Omega}$  \ denota  el evento  nulo
  (evento que no  puede ocurrir, es el complemento de  $\Omega$, por ejemplo $A$
  ``cara par'' y $B$ ``cara impar'').
\end{itemize}
%
\noindent Eso es ilustrado  en la figura Fig.~\ref{fig:MP:Ensembles}. La uni\'on
e intersecci\'on satisfacen a las mismas reglas que en la teoria ensemblista, es
decir cada una es comutativa \ $A \cup B = B \cup A$, \ $A \cap B = B \cap A$, \
asociativa \ $(A \cup B)  \cup C = A \cup (B \cup C)$, \ $(A  \cap B) \cap C = A
\cap (B \cap C)$, \ distributiva con respeto a la otra \ $(A \cup B) \cap C = (A
\cap C) \cup (B \cap C)$ \ y \ $(A  \cap B) \cup C = (A \cup C) \cap (B \cup C)$
\  (ver  ej.~\cite{Jef48,  Jef73,   Hal50,  Fel71,  Bre88,  ManWol95,  IbaPar97,
  LehCas98, AthLah06, Coh13}).

\begin{figure}[h!]
\begin{center} \begin{tikzpicture}[scale=.725]
\shorthandoff{>}
%
% Union y interseccion:
%
% Omega: .25*(x-.25)^2 + (y/1.5)^2 = 1
% A: x^2 + 4 y^2 = 1
% B: (x-1)^2 + 4 (y-1/4)^2 = 1
% A y B se cruzan cuando x = 1 \pm sqrt(55)/10 =>
% theta = acos(.5 \pm sqrt(55)/20) para A
% theta = acos(-.5 \pm sqrt(55)/20) para A
\pgfmathsetmacro{\s}{acos(.5-sqrt(55)/20)};
\pgfmathsetmacro{\t}{-acos(.5+sqrt(55)/20)};
\pgfmathsetmacro{\u}{-acos(-.5+sqrt(55)/20)};
\pgfmathsetmacro{\v}{acos(-.5-sqrt(55)/20)-360};
%
%
%Union
\begin{scope}
%
\fill[opacity=.1]
plot[domain=\s:\t+360,samples=200] ({cos(\x)},{.5*sin(\x)})
-- plot[domain=\u:\v+360,samples=200] ({cos(\x)+1},{.5*sin(\x)+.25})
-- cycle;
%
% borders A, B y Omega
\draw[domain=0:360,samples=200,thick] plot ({cos(\x)},{.5*sin(\x)});
\draw[dashed,domain=0:360,samples=200,thick] plot ({cos(\x)+1},{.5*sin(\x)+.25});
\draw[domain=0:360,samples=200] plot ({2*cos(\x)+.5},{1.5*sin(\x)});
%
% A, B, Omega
\draw (-.75,0) node[scale=.85]{\small $A$};
\draw(1.75,.25) node[scale=.85]{\small $B$};
\draw(-.5,1.375) node[left,scale=.9]{\small $\Omega$};
%
\draw (.5,-2.5) node{(a)};
\end{scope}
%
%
% Interseccion
\begin{scope}[xshift=4.8cm]
%
\fill[opacity=.1]
plot[domain=\s:\t,samples=200] ({cos(\x)},{.5*sin(\x)})
-- plot[domain=\u:\v,samples=200] ({cos(\x)+1},{.5*sin(\x)+.25})
-- cycle;
%
% borders A, B y Omega
\draw[domain=0:360,samples=200,thick] plot ({cos(\x)},{.5*sin(\x)});
\draw[dashed,domain=0:360,samples=200,thick] plot ({cos(\x)+1},{.5*sin(\x)+.25});
\draw[domain=0:360,samples=200] plot ({2*cos(\x)+.5},{1.5*sin(\x)});
%
% A, B, Omega
\draw (-.75,0) node[scale=.85]{\small $A$};
\draw(1.75,.25) node[scale=.85]{\small $B$};
\draw(-.5,1.375) node[left,scale=.9]{\small $\Omega$};
%
\draw (.5,-2.5) node{(b)};
\end{scope}
%
%
% Complemento
\begin{scope}[xshift=9.6cm]
%
\fill[opacity=.1]
plot[domain=0:360,samples=200] ({2*cos(\x)+.5},{1.5*sin(\x)})
-- plot[domain=0:360,samples=200] ({cos(\x)},{-.5*sin(\x)})
-- cycle;
%
% borders A y Omega
\draw[domain=0:360,samples=200,thick] plot ({cos(\x)},{.5*sin(\x)});
\draw[domain=0:360,samples=200] plot ({2*cos(\x)+.5},{1.5*sin(\x)});
%
% A, Omega
\draw (-.75,0) node[scale=.85]{\small $A$};
\draw(-.5,1.375) node[left,scale=.9]{\small $\Omega$};
%
\draw (.5,-2.5) node{(c)};
\end{scope}
%
%
% Excluyentes
\begin{scope}[xshift=14.6cm]
%
% borders A, B (con un shift...) y Omega
\draw[domain=0:360,samples=200,thick] plot ({cos(\x)},{.5*sin(\x)-.5});
\draw[dashed,domain=0:360,samples=200,thick] plot ({cos(\x)+1},{.5*sin(\x)+.5});
\draw[domain=0:360,samples=200] plot ({2*cos(\x)+.5},{1.5*sin(\x)});
%
% A, B, Omega
\draw (-.75,-.5) node[scale=.85]{\small $A$};
\draw (1.75,.5) node[scale=.85]{\small $B$};
\draw(-.5,1.375) node[left,scale=.9]{\small $\Omega$};
%
\draw (.5,-2.5) node{(d)};
\end{scope}
%
%
% privado
\begin{scope}[xshift=19.4cm]
%
\fill[opacity=.1]
plot[domain=\s:\t+360,samples=200] ({cos(\x)},{.5*sin(\x)})
-- plot[domain=\u:\v,samples=200] ({cos(\x)+1},{.5*sin(\x)+.25})
-- cycle;
%
% borders A, B y Omega
\draw[domain=0:360,samples=200,thick] plot ({cos(\x)},{.5*sin(\x)});
\draw[dashed,domain=0:360,samples=200,thick] plot ({cos(\x)+1},{.5*sin(\x)+.25});
\draw[domain=0:360,samples=200] plot ({2*cos(\x)+.5},{1.5*sin(\x)});
%
% A, B, Omega
\draw (-.75,0) node[scale=.85]{\small $A$};
\draw(1.75,.25) node[scale=.85]{\small $B$};
\draw(-.5,1.375) node[left,scale=.9]{\small $\Omega$};
%
\draw (.5,-2.5) node{(e)};
\end{scope}
%
\end{tikzpicture} \end{center}
%
\leyenda{Ilustraci\'on   de  la  operaciones   de  uni\'on   $A  \cup   B$  (a),
  intersecci\'on $A \cap B$ (b), complemento $\bar{A}$ (c), enventos excluyentes
  $A \cap B =  \emptyset$ (d). $A$ es representado en linea  llena, $B$ en linea
  discontinua; (a)-(c)  el resultado de  la operaci\'on es  la zona en  grise. A
  veces, esta representaci\'on ensemblista se  denota {\it diagrama de Venn o de
    Euler}.}
\label{fig:MP:Ensembles}
\end{figure}


Formalmente, se define  de manera abstracta un espacio  medible $(\Omega,\A)$ de
la  manera   siguiente~\cite{Hal50,  Bre88,  IbaPar97,   AthLah06,  Coh13}  (ver
tambi\'en~\cite[\&  Ref.]{BarNov78,  Bor98,  Sie18,  Sie75,  Sie76}  para  notas
hist\'oricas):
%
\begin{definicion}[Espacio medible]
  $(\Omega, \A)$ \ formado de un espacio muestral \ $\Omega$ \ y una colecci\'on
  \ $\A$  \ de conjuntos  de \  $\Omega$ \ es  llamado {\it espacio  medible} si
  satisface a los requisitos
  %
  \begin{enumerate}%[label={(\Roman*)}]
  \item $\emptyset \in \A$,
  %
  \item si $A \in \A$, entonces \ $\bar{A} \in \A$,
  %
  \item la uni\'on numerable de conjuntos de $\A$ queda en $\A$ ($\A$ es cerrado
    por la un\'ion numerable).
  %
  \end{enumerate}
  %
  Con esta propiedades, $\A$  es llamado {\it $\sigma$-\'algebra}. Los elementos
  de $\A$ son dichos {\it medibles}.
\end{definicion}
%
\noindent Es  sencillo mostrar de  que $\Omega$ tambi\'en  es en $\A$, y  de que
$\A$   est   cerrado  por   la   intersecci\'on   numerable.    Un  ejemplo   de
$\sigma$-\'algebra sobre \ $\Omega = \{ 1 \, , \,  2 \, , \, 3 \, , \, 4 \, , \,
5 \, , \, 6 \}$ \ puede ser \ $\big\{ \emptyset \, , \, \Omega \, , \, \{ 1 \, ,
\, 2 \, , \, 3 \} \, , \, \{ 4 \, , \, 5 \, , \, 6 \} \big\}$.

Las propiedades de la probabilidad $P$ de un dado evento quedan determinadas por
los siguientes (ej.~\cite{Spi76, Kol56, ShaVov06, Pla05}):

\noindent {\it Axiomas de Kolmogorov}
%
\begin{enumerate}
\item $P(A_i) \geq 0 \ \ \forall \ A_i \A$
%
\item  Si $\{ A_i  \}_i$ son  eventos mutuamente  excluyentes de  $\A$, entonces
  $\displaystyle P\left( \bigcup_i A_i \right) = \sum_i P(A_i)$
%
\item $P(\Omega) = 1$
\end{enumerate}
%
Formalmente,  se  define  un  {\it  espacio  de  probabilidad}  o  {\it  espacio
  probabil\'istico}  de   la  manera  siguiente~\cite{Hal50,   Bre88,  IbaPar97,
  AthLah06, JacPro03, Coh13}:
%
\begin{definicion}[Espacio probabil\'istico]
  Sea $(\Omega,\A)$ un espacio medible.  Una funci\'on $\mu: \A \mapsto \Rset_+$
  tal que
  %
  \begin{enumerate}
  \item $\mu(\emptyset) = 0$, y
  \item para cualquier  conjunto numerable $\{ A_i \}$  de elementos mutualmente
    excluyentes  de $\A$  se tiene  $\mu\left(  \bigcup_i A_i  \right) =  \sum_i
    \mu(A_i)$
  \end{enumerate}
  %
  es  llamada {\it  funci\'on  medida}  o {\it  medida  $\sigma$-aditiva} y  el
  espacio $(\Omega,\A,\mu)$ es  llamado {\it espacio de medida}.

  Cuando $\mu$  es acotada por  arriba, $\mu(\Omega) <  + \infty$, la  medida es
  dicha  {\it  finita} y  el  espacio tambi\'n  es  dicho  finito. Adem\'as,  si
  \[ P \equiv  \mu, \quad \P(\Omega) = 1  \], la medida es dicha  medida de {\it
    probabilidad}, $\mu \equiv P$.  En  este caso, el espacio $(\Omega,\A,P)$ es
  llamado {\it espacio probabil\'istico}.
\end{definicion}


A  partir de  los axiomas  de Kolmogorov  se pueden  probar varios  corolarios y
propiedades:
%
\begin{itemize}
\item la probabilidad de un evento seguro o cierto es 1;
%
\item  la   probabilidad  de   un  evento   que  no  puede   ocurrir  es   0:  \
  $P(\emptyset) = 0$;
%
\item el rango  de las probabilidades est\'a  acotado: \ $0 \leq P(A)  \leq 1\ \
  \forall \ A \in \A$;
%
\item condici\'on de  normalizaci\'on: \ si $\Omega =  \bigcup_{i=1}^n A_i$, con
  $A_i$  mutuamente  excluyentes,  entonces  \  $\sum_{i=1}^n P(A_i)  =  1$;  el
  conjunto  $\{ A_i  \}_{i=1}^n$  es  dicho {\it  conjunto  completo de  eventos
    posibles      excluyentes      entre      s\'i}     y      es      ilustrado
  figure~\ref{fig:MP:CompletoSub};
%
\item si $A$ es subconjunto de $B$,  lo que escribiremos $A \subset B$, es decir
  si  $B$  se realiza,  $A$  se realiza  tambi\'en  (pero  no necesariamente  al
  rev\'es),     entonces     \     $P(A)     \leq    P(B)$;     Es     ilustrado
  figure~\ref{fig:MP:CompletoSub};
\end{itemize}

\begin{figure}[h!]
\begin{center} \begin{tikzpicture}%[scale=.9]
\shorthandoff{>}
%
% Particion
\begin{scope}
%
% Omega y A_i
\draw[domain=0:360,samples=200] plot ({2*cos(\x)+.5},{1.5*sin(\x)});
\draw[domain=-90:90,samples=200] plot ({cos(\x)+.5},{1.5*sin(\x)});
\draw[domain=90:180,samples=200] plot ({.5*cos(\x)+.5},{1.5*sin(\x)});
\draw[domain=0:3,samples=200] plot (\x-1.5,{.5*sin(120*\x)});
%
%
% Omega y A_i's
\draw(-.5,1.375) node[left,scale=.9]{\small $\Omega$};
\draw(-.5,.8) node[scale=.9]{\small $A_1$};
\draw(.75,.5) node[scale=.9]{\small $A_2$};
\draw(0,-.75) node[scale=.9]{\small $A_3$};
\draw(2,0) node[scale=.9]{\small $A_4$};
%
\draw (.5,-2.5) node{(a)};
\end{scope}
%
%
% Inclusion
\begin{scope}[xshift=7.5cm]
%
\fill[opacity=.15] plot[domain=0:360,samples=200] ({1.25*cos(\x)+.25},{.75*sin(\x)});
\fill[opacity=.3] plot[domain=0:360,samples=200] ({.75*cos(\x)+.25},{.5*sin(\x)});
%
% borders A, B y Omega
\draw[domain=0:360,samples=200] plot ({1.25*cos(\x)+.25},{.75*sin(\x)});
\draw[domain=0:360,samples=200] plot ({.75*cos(\x)+.25},{.5*sin(\x)});
\draw[domain=0:360,samples=200] plot ({2*cos(\x)+.5},{1.5*sin(\x)});
%
% A, B, Omega
\draw (-.8,0) node[scale=.9]{\small $A$};
\draw(.8,0) node[scale=.9]{\small $B$};
\draw(-.5,1.375) node[left,scale=.9]{\small $\Omega$};
%
\draw (.5,-2.5) node{(b)};
\end{scope}
%
\end{tikzpicture} \end{center}
%
\leyenda{Ilustraci\'on  de  conjunto completo  de  eventos posibles  excluyentes
  entre s\'i  (a), y de  la inclusi\'on  (b) donde $A$  es en grise  (claro como
  oscuro) mientras de que $B$ es en grise oscuro.}
\label{fig:MP:CompletoSub}
\end{figure}

Nota:  la probabilidad  \ $P(A  \cap B)$  \ del  evento $A  \cap B$  \  se llama
tambi\'en {\it probabilidad conjunta} de \ $A$ \ y \ $B$.
%$P(A  \cap B)  = P(B  \cap A)$}  es la
%probabilidad del evento conjunto dado por  la composici\'on de los eventos $A$ y
%$B$. 

Se demuestra que
%
\begin{itemize}
\item $P(A  \cap B)$ est\'a acotada:  \ $0 \leq P(A  \cap B) \leq  \min\{ P(A) ,
  P(B)\}$; (viene de \ $A \cap B \subset A$ \ y \ $A \cap B \subset B$).
%
\item Si  $A$ y $B$ son  mutuamente excluyentes, entonces  \ $p(A \cap B)  = 0$;
  (viene de \ $A \cap B = \emptyset$).
%
\item  si  $\{ B_j  \}_{j=1}^m$  es un  conjunto  completo  de eventos  posibles
  excluyentes entre s\'i, entonces \ $\sum_{j=1}^m P(A \cap B_j) = P(A)$; (viene
  de  $\{ A  \cap B_j  \}$ mutualmente  excluyentes y  $\bigcup_j \left(  A \cap
    B_j\right) = A \cap \left( \bigcup_j B_j \right) = A \cap \Omega = A$).
\end{itemize}

En el caso de eventos no necesariamente mutuamente excluyentes, se prueba que la
{\it ley de composici\'on} o {\it formula de inclusi\'on-exclusi\'on} es
%
\[
P(A \cup B) = P(A) + P(B) - P(A \cap B) \leq P(A) + P(B), 
\]
%
y que para $n$ eventos resulta
%
\[
P\left( \bigcup_{i=1}^n A_i \right) \leq \sum_{i=1}^n P\left( A_i \right).
\]
%
La  igualdad  vale  en  el  caso  especial  de  eventos  mutuamente  excluyentes
(recuperando el segundo axioma de Kolmogorov).

Se  prueba tambi\'en  de que  si $\{  A_i \}_{i=1}^{+\infty}$  es  una secuencia
creciente  de eventos,  \ie $\forall  \, i  \ge 1,  \quad A_i  \subset A_{i+1}$,
entonces
%
\[
P\left( \bigcup_{i=1}^{+\infty} A_i \right) = \lim_{i \to +\infty} P(A_i)
\]
%
Similarmente,  si $\{ A_i  \}_{i=1}^{+\infty}$ es  una secuencia  decreciente de
eventos, \ie $\forall \, i \ge 1, \quad A_{i+1} \subset A_i$, entonces
%
\[
P\left( \bigcap_{i=1}^{+\infty} A_i \right) = \lim_{i \to +\infty} P(A_i)
\]


Se puede preguntarse de cual es la probabilidad de un evento $A$, si sabemos que
tenemos un evento $B$, dado.  Por  ejemplo, por un dado de 6 caras equilibriado,
cual es la probabilidad de tener  un n\'umero par sabiendo que tenemos un numero
menor  a igual  a  3.   La respuesta  es  en la  noci\'on  de {\it  probabilidad
  condicional}~\cite{Hau01, Jef48, Jef73, Bre88, ManWol95, JacPro03, ShaVov06}:
%
\begin{definicion}[Probabilidad condicional]
  \underline{Por definici\'on},  la {\it  probabilidad condicional} de  $A$ dado
  $B$ es la raz\'on entre la  probabilidad del evento conjunto y la probabilidad
  de que se d\'e $B$ (cuando \'este es un evento no nulo):
  %
  \[
  P(A|B) = \frac{P(A \cap B)}{P(B)}.
  \]
\end{definicion}
%
En  el  ejemplo precediente,  la  probabilidad  va a  ser  $P(A|B)  = \frac13  =
\frac{\frac16}{\frac12}  = \frac{P(A  \cup B)}{P(B)}$. 

Es f\'acil demostrar %% <-- ejercicio
que esta  cantidad toma valores  entre 0 y  1, con $P(\Omega|B)  = 1$, y  que es
aditiva  para  una  uni\'on  de  eventos  mutuamente  excluyentes  referidos  al
cumplimiento de $B$.  Luego, $P(A|B)$ es una medida de probabilidad~\footnote{Se
  puede  definir un  espacio de  probabilidad $(  \Omega_B, \A_B  ,  P_B)$ donde
  $P_B(A)  \equiv P(A|B)$.};  Por eso,  a veces  en la  literatura se  la denota
$P_B(A)$.  Diversas  situaciones de probabilidades  condicionales son ilustradas
en la figura siguiente, Fig.~\ref{fig:MP:Condicional}.

\begin{figure}[h!]
\begin{center} \begin{tikzpicture}%[scale=.9]
\shorthandoff{>}
%
\begin{scope}
%
% Omega, A, B_i
\draw[domain=0:360,samples=200] plot ({2*cos(\x)+.5},{1.5*sin(\x)});
\draw[domain=0:360,samples=200] plot ({1.2*cos(\x)},{.9*sin(\x)});
\draw[dashed,domain=0:360,samples=200] plot ({.5*cos(\x)+.3},{.3*sin(\x)+.25});
\draw[dashed,domain=0:360,samples=200] plot ({.4*cos(\x)+.6},{.6*sin(\x)-.8});
\draw[dashed,domain=0:360,samples=200] plot ({.4*cos(\x)+1.8},{.4*sin(\x)+.25});
%
%
% Omega y A_i's
\draw(-.5,1.375) node[left,scale=.9]{\small $\Omega$};
\draw(-.5,.7) node[scale=.9]{\small $A$};
\draw(.5,.3) node[scale=.9]{\small $B_1$};
\draw(.6,-1.2) node[scale=.9]{\small $B_2$};
\draw(1.8,.25) node[scale=.9]{\small $B_3$};
%
\end{scope}
%
\end{tikzpicture} \end{center}
%
\leyenda{Ilustraci\'on  de  la probabilidad  condicional  con  $A$ interior  del
  elipse  en linea  llena  y unos  $B_i$  interiores de  los  elipses en  lineas
  discontinuas.   \ $\omega  \in B_1  \Rightarrow \omega  \in A$  \ as\'i  que \
  $P(A|B_1) = 1$. Al rev\'es, \  $\omega \in B_3 \Rightarrow \omega \not\in A$ \
  as\'i que \  $P(A|B_3) = 0$.  Entre estas  situaciones extremas, si $P(\bar{A}
  \cap B_2)  \ne 0$ \ y  \ $P(A \cap  B_2) \ne 0$ \  tenemos $0 < P(A|B_2)  < 1$
  (ej.  con   probabilidades  iguales  a   las  superficias  relativas   de  los
  conjuntos).}
\label{fig:MP:Condicional}
\end{figure}

Algunas propiedades interesantes son las siguientes:
%
\begin{itemize}
\item condici\'on  de normalizaci\'on: \  $\sum_{i=1}^n P(A_i|B) = 1$,  siendo \
  $\{ A_i \}_{i=1}^n$  \ un conjunto completo de  resultados posibles mutuamente
  excluyentes;
%
\item relaci\'on  entre probabilidades condicionales  inversas: \ $\displaystyle
  P(B|A)  = \frac{P(B)}{P(A)}  P(A|B)$, de  donde \  $p(A|B)$ \  y \  $p(B|A)$ \
  coinciden s\'olo cuando \ $A$ \ y \ $B$ \ tienen la misma probabilidad;
%
\item {\it f\'ormula de probabilidades totales}: \ si $\{ B_j \}$ es un conjunto
  completo de eventos no nulos mutuamente excluyentes, entonces
  %
  \[
  P(A) = \sum_j P(A|B_j) P(B_j)
  \]
  %
  Viene de \ $A = A \cap  \left( \bigcup_j B_j \right) = \bigcup_j \left( A \cap
    B_j \right)$  \ donde  los \ $A  \cap B_j$  \ son mutuamente  excluyentes, y
  $P\left( A \cap B_j \right) = P(A|B_j) P(B_j)$.
%
\item {\it  f\'ormula de Bayes}:  \ si  $\{ B_j \}$  es un conjunto  completo de
  eventos no nulos mutuamente excluyentes, entonces
  %
  \[
  P(B_i|A) = \frac{P(A \cap B_i)}{P(A)} = \frac{P(A|B_i) P(B_i)}{\sum_j P(A|B_j)
    P(B_j)} .
  \]
  %
  (ver~\cite{Bre88, JacPro03, Bay63, Bar58}).
\end{itemize}

Terminamos  esta   secci\'on  por  la   noci\'on  de  independencia   entre  dos
eventos. Por ejemplo, si dos dados son tirado sobre dos mesas diferentes, no hay
ninguna raz\'on de que la muestra de  uno ``influye'' la del otro. Dicho de otra
manera,  dos  eventos  son  inde�ndientes  si conciendo  uno  no  lleva  ninguna
``informaci\'on'' sobre el otro~\cite{Bre88, ManWol95, Hau01, JacPro03, Bor09}:
%
\begin{definicion}[Independencia estad\'istica]
  Dos eventos \ $A$ \ y \ $B$ \ se dicen {\it estad\'isticamente independientes}
  si la  probabilidad condicional  de $A$  dado $B$ es  igual a  la probabilidad
  incondicional de $A$:
  %
  \[
   P(A|B) = P(A).
   \]
  %
   Es equivalente al hecho de que la probabilidad conjunta se factoriza,
  %
  \[
   P(A \cap B) = P(A) P(B).
   \]
\end{definicion}
%
Por  inducci\'on, la  condici\'on necesaria  y suficiente  para que  $n$ eventos
$A_1,\ldots,A_n$ sean  estad\'isticamente independientes es  que la probabilidad
conjunta se factorice como
%
\[
P\left( \bigcap_{i=1}^n A_i \right) = \prod_{i=1}^n P(A_i).
\]
%
Se  deduce que  los  eventos mutuamente  excluyentes  no son  estad\'isticamente
independientes.

% ============ Variables aleatorias y distribuciones de probabilidad =========== %

%\seccion{Variables aleatorias y distribuciones de probabilidad}
\label{s:variablealeatoria}


\SZ{\cite[\& Ref.]{BarNov78}}



%%%%%%%%%%%%%%%%%%%%%%%%%%%%%%%%%%%%%%%%%%%%%%%%%%%%%%%%%%%%%%%%%%%%%%%%%%

\emph{En  un  experimento  o  un  dado  proceso,  los  posibles  resultados  son
  t\'ipicamente  n\'umeros reales, siendo  cada n\'umero  un evento.   Luego los
  resultados  son mutuamente  excluyentes. Se  considera a  esos  n\'umeros como
  valores de una  \emph{variable aleatoria} $X$ a valores  reales, que puede ser
  discreta  (cuando  el espacio  muestral  es  finito  o infinito  numerable)  o
  continua.  La ley  de la variable aleatoria $X$ es  una medida de probabilidad
  definida por  \ $P_X(x) =  \Pr(X=x)$ o, en  general, por $P_X(A)  = \Pr(X=x\in
  A)$.  Para indicar  que la variable~$X$ sigue la ley  de distribuci\'on $p$ se
  escribe  \  $X\sim p  $.   Puede  ser  \'util tambi\'en  considerar  variables
  aleatorias  complejas  $Z=X+iY$, donde  $X$  e  $Y$  son variables  aleatorias
  reales. }

%%%%%%%%%%%%%%%%%%%%%%%%%%%%%%%%%%%%%%%%%%%%%%%%%%%%%%%%%%%%%%%%%%%%%%%%%%
\subseccion{Variable aleatoria discreta}

Los  posibles valores de  una variable  aleatoria discreta  $X$ consisten  en un
conjunto  contable  (finito  o   infinito  numerable)  de  n\'umeros  reales:  \
$x\in\Omega=\{x_1,   x_2,  \ldots\}$.   A   cada  uno   de  los   valores  $x_n$
($n=1,2,\ldots$) se puede asociar una  probabilidad $p_n=p(x_n)$, de modo que se
satisface la condici\'on de normalizaci\'on:
%
$$\sum_n p_n = 1 . $$ 
%
La \emph{funci\'on (de masa) de probabilidad} es  de la forma: 
%
$$
p(x) = \left\{
\begin{array}{cl}
\Pr(X=x) & \mbox{si} \ x=x_1, x_2, \ldots \\ 0 &
\mbox{en~todo~otro~punto}
\end{array} \right.
$$
%
En  la   Fig.~\ref{fig:distribprobdiscreta}  se  muestra   una  representaci\'on
gr\'afica de una distribuci\'on de probabilidad discreta.
%
\begin{figure}[h!] %%ojo numeracion de figs !
%\centerline{\includegraphics[width=3cm]{distribprobdiscreta.jpg}} %%rehacer
%
\leyenda{Una distribuci\'on de probabilidad discreta.}
\label{fig:distribprobdiscreta}
\end{figure}

Tambi\'en, se puede caracterizar la ley de la variable discreta $X$ por medio de
su \emph{funci\'on de repartici\'on}:
%
$$
F_X(x)= \Pr(X\in(-\infty,x]) = \Pr (X\leq x) = \sum_{\forall
  n \,:\, x_n\leq x} p(x_n)
$$
%
que es una funci\'on discontinua, con saltos finitos, y no decreciente. 

Sin  p\'erdida de  generalidad, el  conjunto de  valores que  toma  una variable
aleatoria discreta $X$ puede considerarse como $\{0,1,2,\ldots,N\}$ para alg\'un
$N$ natural, o todo $\Nset$. Entonces la ley de una variable aleatoria a valores
naturales est\'a dada por  \ $\{p_n = \Pr(X=n), \ n\in \Nset  \}$.  Luego \ $\Pr
(X\in  A)=\sum_{n\in A\cap  \Nset}  p_n$,  y la  funci\'on  de repartici\'on  se
calcula como  \ $\Pr(X\leq x) = \sum_{n  \leq x} \Pr(X=n)$ que  es una funci\'on
que presenta un  salto finito en cada n\'umero natural.  En  general un salto de
la funci\'on  de repartici\'on corresponde a  la presencia de  una \emph{masa de
  Dirac} en el entorno del salto. %%

Un caso especial se tiene cuando un  valor $x_j$ es cierto o seguro, y no ocurre
ninguno de  los otros valores $x_i \  (i\neq j)$. La forma  de la distribuci\'on
es: \ $p_n=\delta_{nj}$, donde
%
$$
\delta_{ij} = \left\{
\begin{array}{cl}
1 & \mbox{si} \ i=j \\
0 & \mbox{si} \ i\neq j 
\end{array} \right.
$$
%
es el s\'imbolo \emph{delta de  Kronecker}. Cuando el espacio muestral es finito
de dimensi\'on $N$, la ley de  distribuci\'on se puede representar por medio del
siguiente vector columna:
%
$$
p = \begin{pmatrix}
% \left(  
%\begin{array}{c}
0 \\ \vdots \\ 0 \\ 1  \\ 0 \\ \vdots \\ 0
\end{pmatrix}
%\end{array} \right)
$$
%
con  un  1  en  el  lugar  $j$-\'esimo,  que tambi\'en  se  escribe  como  \  $p
= \begin{pmatrix} 0  & \cdots & 0 &  1 & 0 & \cdots &  0 \end{pmatrix}^t$, donde
$t$ indica transposici\'on.  La funci\'on de repartici\'on resulta una funci\'on
escal\'on o de Heaviside: \ $F(x)=\Theta(x-x_j)$.

Otra    situaci\'on   particular    es   la    de    \emph{equiprobabilidad}   o
\emph{distribuci\'on uniforme}.  La forma  de la distribuci\'on es: \ $p_n=\frac
1N \ \ \forall\ n=1,\ldots,N$, donde  $N$ se�ala el tama�o del espacio muestral.
La ley  de distribuci\'on  se puede representar  por medio del  siguiente vector
columna:
%
$$
p = \begin{pmatrix}
% \left(  
%\begin{array}{c}
1/N \\ 1/N  \\ \vdots \\ 1/N
\end{pmatrix}
%array} \right) , 
$$
%
que tambi\'en se escribe como \  $p = \begin{pmatrix} \frac1N & \frac1N & \cdots
  &  \frac1N  \end{pmatrix}^t$.   La  funci\'on  de  repartici\'on  resulta  una
funci\'on escalonada,  con saltos de  altura $\frac1N$ para  cada $n$ entre  1 y
$N$.

\hfill 

\emph{Reordenamiento y relaci\'on de mayorizaci\'on} 

Para comparar dos  distribuciones es \'util reordenar el  vector de probabilidad
permutando  sus  elementos  hasta  listarlos  de forma  descendente.   Se  anota
$p^\downarrow$, de  modo que \  $p^\downarrow_1 \geq p^\downarrow_2  \geq \ldots
\geq  p^\downarrow_N$.   En  el  ejemplo   del  caso  con  certeza  se  tiene  \
$p^\downarrow =  \begin{pmatrix} 1 & 0  & \cdots &  0 \end{pmatrix}^t$, mientras
que la distribuci\'on uniforme no var\'ia.

Se  define  \emph{mayorizaci\'on} del  siguiente  modo,  para distribuciones  de
dimensi\'on  $N$ (con  sus elementos  acomodados  en forma  decreciente): \  una
distribuci\'on $p$  es mayorizada por otra $q$,  y se denota $p\prec  q$, si las
primeras $N-1$  sumas parciales de $p^\downarrow$ y  $q^\downarrow$ satisfacen \
$\sum_{i=1}^n  p^\downarrow_i   \leq  \sum_{i=1}^n  q^\downarrow_i$   para  todo
$n=1,\ldots,N-1$, con \ $\sum_{i=1}^N p_i = 1 = \sum_{i=1}^N q_i$.

Por    ejemplo,   $\begin{pmatrix}    \frac12    &   \frac14    &   \frac18    &
  \frac18 \end{pmatrix}^t  \prec \begin{pmatrix} \frac12  & \frac14 &  \frac14 &
  0 \end{pmatrix}^t$.  Es posible  comparar por mayorizaci\'on distribuciones de
distinta  dimensionalidad, completando con  ceros el  vector de  probabilidad de
menor  dimensi\'on.  Es  importante  resaltar que  la  mayorizaci\'on provee  un
\emph{orden  parcial}  (no  total)  entre distribuciones,  existiendo  pares  de
distribuciones   tales  que   ninguna  mayoriza   a  la   otra.    Por  ejemplo,
$\begin{pmatrix} 0.50 &  0.40 & 0.10 \end{pmatrix}^t$ y  $\begin{pmatrix} 0.70 &
  0.15 & 0.15 \end{pmatrix}^t$ no se comparan por mayorizaci\'on.

Es  interesante  notar  que  la   siguiente  propiedad  es  v\'alida  para  toda
distribuci\'on $p$ de tama�o~$N$:
%
$$
\begin{pmatrix} \frac1N & \frac1N & \cdots & \frac1N \end{pmatrix}^t \ \prec \ p
\ \prec \ \begin{pmatrix} 1 & 0 & \cdots & 0 \end{pmatrix}^t.
$$
%
En este  sentido, los  casos particulares de  equiprobabilidad y de  certeza, se
dice  que  son  distribuciones  extremas.  Notamos que  uno  implica  ignorancia
m\'axima  en el  resultado de  la variable  mientras que  el otro  corresponde a
conocimiento completo.
%% graficamente 
%
\begin{figure}[h!] %%ojo numeracion de figs !
%\centerline{\includegraphics[width=3cm]{majorizationplot.pdf}} %%rehacer
%
\leyenda{Orden parcial por mayorizaci\'on}
\label{fig:majorizationplot}
\end{figure}

%%%%%%%%%%%%%%%%%%%%%%%%%%%%%%%%%%%%%%%%%%%%%%%%%%%%%%%%%%%%%%%%%%%%%%%%%%
\subseccion{Variable aleatoria continua}

\SZ{\cite[\& Ref.]{BarNov78}}

Los posibles valores de una  variable aleatoria continua $X$ son cualesquiera de
los n\'umeros en un dado  intervalo de la recta real: \ $x\in\Omega\subset\Rset$
que  puede ser  un intervalo  $[x_m,x_M]$  o un  subconjunto (semi)infinito.  Es
conveniente asociar una  \emph{funci\'on densidad de probabilidad} (com\'unmente
anotada por su  sigla en ingl\'es: pdf por  \emph{probability density function})
$p(x)$ que tiene el  sentido de que la probabilidad de que  $X$ tome valor entre
$a$ y $b$ est\'a dada por:
%
$$
\Pr(a\leq X\leq b) = \int_a^b p(x) \, dx ,
$$
%
siendo $p(x)  \, dx$  la densidad de  probabilidad de  hallar a la  variable con
valores  en el intervalo  infinitesimal entre  $x$ y  $x+dx$. La  condici\'on de
normalizaci\'on se escribe
%
$$
\int_{x_m}^{x_M} p(x) \, dx=1 . 
$$ 

En  la   Fig.~\ref{fig:distribprobcontinua}  se  muestra   una  representaci\'on
gr\'afica de una funci\'on densidad de probabilidad para una variable continua.
%
\begin{figure}[h!] %%ojo numeracion de figs !
%\centerline{\includegraphics[width=3cm]{distribprobcontinua.png}} %%rehacer
%
\leyenda{Una distribuci\'on de probabilidad continua.}
\label{fig:distribprobcontinua}
\end{figure}

Tambi\'en, se puede caracterizar la ley de la variable continua $X$ por medio de
su  \emph{funci\'on  de   repartici\'on}  o  \emph{funci\'on  de  distribuci\'on
  cumulativa} (CDF por \emph{cumulative distribution function}):
%
$$
F_X(x) = \Pr(X\leq x) = \int_{x_m}^x p(t) \, dt
$$
%
que da la  probabilidad de que $X$ sea  menor o igual que cierto  valor $x$ dado
(dentro del conjunto~$\Omega$ de todos  los valores posibles de la variable). En
forma an\'aloga,  \ $\Pr(X\in  A) = \int_A  p(x) \,  dx$ acumula la  densidad de
probabilidad en un subconjunto $A$ del espacio muestral.  Por la propiedad de la
inclusi\'on,  se  tiene \  $\Pr(X\leq  x_1)  \leq  \Pr(X\leq x_2)$  siempre  que
$x_1\leq x_2$; luego $F_X(x)$ es una funci\'on creciente
%%no decreciente ?? SI, LO DIRIA ASI
de   $x$,  acotada   por  la   unidad,  con   valores  extremos   dados   por  \
$\lim_{x\rightarrow -\infty} F_X(x)=0$  y $\lim_{x\rightarrow \infty} F_X(x)=1$,
tomando $\Omega=\Rset$.  Adem\'as la derivada respecto de $x$ es la pdf:
%
$$
\frac{dF_X(x)}{dx}=p(x) . 
$$ 
%
De aqu\'i  se observa que  la densidad de  probabilidad $p(x)$ puede no  ser una
funci\'on ``ordinaria''  cuando $\Pr(X\leq x)$  es discontinua, pero  como mucho
tiene  la  singularidad  de   una  distribuci\'on  \emph{delta  de  Dirac}  cuya
representaci\'on integral es:
%
$$
\delta (x) = \frac{1}{2\pi} \int_{-\infty}^{\infty} e^{itx} dt . 
$$
% SZ: LA  VERDAD ES  QUE NO ME  GUSTA MUCHO  escribir una distribucion  como una
% fuccion porque no viven en el  mismo espacio. Pero no importa. Para mi, cuando
% hay  una  discontinuidad  X no  admite  une  pdf  por  definicion del  'f'  de
% pdf\sum_{}

Un caso especial  se tiene cuando la variable aleatoria $X$  toma el valor $x_0$
con certeza.  La forma  de la pdf  es: \ $p(x)=\delta(x-x_0)$.  Otra situaci\'on
particular es la distribuci\'on uniforme en  un intervalo; la pdf es de la forma
\  $p(x)=\frac{1}{b-a} \  \forall  \  x\in[a,b]$, donde  $[a,b]$  es el  espacio
muestral.

Usando las  funciones delta de  Dirac, se puede  unificar el tratamiento  de las
variables  aleatorias discretas  con las  continuas: si  una  variable aleatoria
discreta  toma los  valores $x_1,  x_2,  \ldots$ con  probabilidades $p_1,  p_2,
\ldots$ respectivamente,  entonces formalmente  se puede describir  mediante una
variable  aleatoria  continua  $X$  con  funci\'on densidad  de  probabilidad  \
$p(x)=\sum_j p_j \,\delta(x-x_j)$.


%%%%%%%%%%%%%%%%%%%%%%%%%%%%%%%%%%%%%%%%%%%%%%%%%%%%%%%%%%%%%%%%%%%%%%%%%%
\subseccion{Vector aleatorio}


Cuando se trabaja  con $d\geq 2$ variables aleatorias  es conveniente definir un
\emph{vector aleatorio} de dimensi\'on $d$,  y apelar para su estudio a nociones
del \'algebra  lineal y  a notaci\'on matricial.   Se tiene el  vector aleatorio
$d$-dimensional  \  $\mathbf{X} =  \{X^1,  \ldots,  X^d  \}$, o  simplemente  $X
=  \begin{pmatrix}  X^1  &  \cdots  & X^d  \end{pmatrix}^t$,  caracterizado  por
$d$-uplas de variables aleatorias reales, con funci\'on densidad de probabilidad
conjunta~$p(x^1, \ldots, x^d)$. La ley  del vector $\mathbf{X}$ es una medida de
probabilidad sobre $\Rset^d$, con
%
$$
P_{\mathbf{X}}(\mathbf{A})  =   \Pr(\mathbf{X}\in\mathbf{A})  =  \int_{\mathbf{A}}
p(x^1, \ldots, x^d)\ dx^1\ldots dx^d
$$
%
para  $\mathbf{A}  \subset \mathbf{\Omega}$,  siendo  la  pdf  conjunta $p$  una
funci\'on positiva, definida sobre $\mathbf{\Omega}\subset\Rset^d$, y tal que se
satisface la condici\'on de normalizaci\'on:
%
$$
\int_{\mathbf{\Omega}} p(x^1, \ldots, x^d)\ dx^1\ldots dx^d  = 1 .
$$

La  \emph{funci\'on densidad  de  probabilidad marginal}  que  caracteriza a  la
variable aleatoria  $X^i$ es la  ley que se  obtiene integrando la  pdf conjunta
sobre todas las variables excepto la $i$-\'esima:
%
$$
p_{X^i}(x^i)  =  \int_{\mathbf{\Omega}^{(i)}}  p(x^1, \ldots,  x^d)\  dx^1\ldots
dx^{i-1} dx^{i+1} \ldots dx^d
$$
%
donde $\mathbf{\Omega}^{(i)}\subset\Rset^{d-1}$  barre el espacio  muestral para
$X^1, \ldots, X^{i-1}, X^{i+1}, \ldots, X^d$.

Las  $d$  variables  aleatorias  $X^1,  \ldots,  X^d$  de  un  vector  aleatorio
$\mathbf{X}$ se dicen \emph{independientes} si corresponden a eventos mutuamente
independientes. Esto se  da si y s\'olo  si la pdf conjunta se  factoriza en las
$d$ pdf marginales:
%
$$
p(x^1, \ldots, x^d) = p_{X^1}(x^1) \cdots p_{X^d}(x^d) . 
$$


%%%%%%%%%%%%%%%%%%%%%%%%%%%%%%%%%%%%%%%%%%%%%%%%%%%%%%%%%%%%%%%%%%%%%%%%%%
\subseccion{Transformaci\'on de variables aleatorias}


Sea $X$ una  variable aleatoria (continua, en general)  definida en el intervalo
$[x_m, x_M]$ con funci\'on densidad  de probabilidad $p(x)$. Sea $Y=\Psi(X)$ una
funci\'on real  de $X$, luego $Y$  toma los valores $y=\Psi(x)$  en el intervalo
$[y_m,y_M]$.  La funci\'on  densidad  de probabilidad  $q(y)$  para la  variable
aleatoria transformada $Y$ se obtiene  de la siguiente manera, dependiendo de la
forma de la transformaci\'on:

\begin{itemize}
\item Si  $\Psi$ es inversible, con  inversa (\'unica), se  tiene \ $x=\Phi(y)$,
  con  $\Phi=\Psi^{-1}$.  A partir  de  la  propiedad  de conservaci\'on  de  la
  probabilidad
  %
  $$
  |q(y)\, dy| = |p(x) \, dx|
  $$ 
  %
  para  una correspondencia  biun\'ivoca  entre $x$  e  $y$, se  obtiene la  pdf
  transformada
  %
  $$
  q(y)  = p(x)  \left| \frac{dx}{dy}  \right| =  p\left(\Phi(y)\right)  \ \left|
    \Phi'(y)   \right|   =  \frac{p\left(\Phi(y)\right)}{\left|   \Psi'(\Phi(y))
    \right|} .
  $$
  %
  Una forma alternativa  de derivar este resultado es partir  de la funci\'on de
  repartici\'on:
  %
  $$
  F_Y(y)  =  P(Y\leq  y)  =  P(\Psi(X)  \leq  y)  =  P(X\leq  \Psi^{-1}  (y))  =
  F_X(\Phi(y))
  $$
  %
  y  calcular las  derivadas del  primer y  \'ultimo t\'erminos  respecto  de la
  variable transformada~$y$.
%
\item Si la inversa de $\Psi$  es multivaluada, cada valor de $y$ se corresponde
  con  un conjunto de  valores de  $x$, digamos  $\{x_k =  \Phi_k(y), \  k=1, 2,
  \ldots\}$.  Debido a  que  estas soluciones  son  mutuamente excluyentes,  las
  probabilidades se suman, de modo que
  %
  $$
  q(y)   =    \sum_k   p(x_k)   \left|   \frac{dx_k}{dy}    \right|   =   \sum_k
  \frac{p\left(\Phi_k(y)\right)}{\left| \Psi'(\Phi_k(y)) \right|} ,
  $$
  %
  que   formalmente  se   puede  expresar   como  \   $q(y)  =   \int   p(x)  \,
  \delta(y-\Psi(x)\,) \ dx$ , donde se  usa la expansi\'on de la funci\'on delta
  en  t\'erminos de sus  ceros: \  $\delta(y-\Psi(x)\,)= \sum_k  \delta(x-x_k) /
  |\Psi'(x_k)| $.\newline  Por ejemplo, para la transformaci\'on  de variables \
  $Y=X^2$ se tiene \  $Y=\Psi(X)=X^2$ cuyas inversas son \ $X_1=\Phi_1(Y)=+\sqrt
  Y$  y   $X_2=\Phi_2(Y)=-\sqrt  Y$;  luego   \  $q(y)=\frac{p(\sqrt  y)}{2\sqrt
    y}+\frac{p(-\sqrt y)}{|-2\sqrt y|}$ , para $y>0$.
\end{itemize}

\hfill

Consideramos ahora el  caso de un vector aleatorio  $\mathbf{X} = \{X^1, \ldots,
X^d\}$  con  funci\'on  densidad  de  probabilidad conjunta  \  $p(x^1,  \ldots,
x^d)$. Se  define otro vector aleatorio  \ $\mathbf{Y} =  \{Y^1, \ldots, Y^d\}$,
por   medio   de   las   transformaciones  \   $Y^j=\Psi^j(X^1,\ldots,X^d)$,   \
$j=1,\ldots,d$. Suponiendo que las  funciones $\Psi^j$ tienen inversa (\'unica),
se  puede escribir \  $X^j=\Phi^j(Y^1,\ldots,Y^d)$ para  cada $j$.  La funci\'on
densidad de probabilidad conjunta $q(y^1,\ldots,y^d)$ para $\mathbf{Y}$ se puede
obtener a partir de la propiedad de conservaci\'on de la probabilidad
%
$$
|q(y^1,\ldots,y^d)\ dy^1\cdots dy^d| = |p(x^1,\ldots,x^d) \ dx^1\cdots dx^d| .
$$ 
%
Para  una  correspondencia biun\'ivoca  entre  $\mathbf{x}$  e $\mathbf{y}$,  se
obtiene la pdf transformada
%
$$
q(y^1,\ldots,y^d) = \left| \Jac_\Phi \right| \, p(x^1,\ldots, x^d) 
%%= \int  p(x^1,\ldots, x^d) \delta(y^1-\Psi^1) .......  dx^1 .....
$$
%
donde  $\Jac_\Phi =  \frac{\partial(\Phi^1  , \ldots  , \Phi^d)}{\partial(y^1  ,
  \ldots , y^d)}$ es el Jacobiano de la transformaci\'on.

%% Ejercicio: Estudiar el caso multivaluado / Resolver un ej. 

\hfill

Una  \emph{variable  aleatoria  compleja}   $Z=X+i  Y$  puede  interpretarse  en
t\'erminos de las  dos variables aleatorias reales $X$ e $Y$.  La pdf asociada \
$P(z)=p(x,y)$ est\'a dada por la  funci\'on densidad de probabilidad conjunta de
las variables reales. La condici\'on de normalizaci\'on se escribe
%
$$
\int P(z) \, d^2 z = 1
$$
%
donde $d^2 z=dx\,dy$.

% =================== Transformacion de variables aleatorias =================== %

%\seccion{Transformaci\'on de variables y vectores aleatorios}
\label{Sec:MP:Transformacion}

En  esta  secci\'on nos  interesamos  al  effect de  una  variable  o un  vector
aleatorio. Por  ejemplo, en un juego con  dos dados, nos podemos  interesar a la
ley de la suma que dar\'ia el n\'umero de casilla de que debemos adelantar en un
juego de la oca.
%
\begin{teorema}[Transformaci\'on medible de un vector aleatorio]
\label{Teo:MP:TransformacionMedible}
%
  Sea  \  $X:  (\Omega,\A)  \mapsto  (\Rset^d ,  \B(\Rset^d))$  \  una  variable
  aleatoria,   y  \   $g:  (\Rset^d   ,  \B(\Rset^d))   \mapsto   (\Rset^{d'}  ,
  \B(\Rset^{d'}))$ \  una funci\'on  medible. Entonces,  \ $Y =  g(X)$ \  es una
  variable aleatoria  \ $(\Omega,\A)  \mapsto (\Rset^{d'} ,  \B(\Rset^{d'}))$. \
  Adem\'as, la medida im\'agen \ $P_Y$ \ es vinculada \ a \ $P_X$ \ por
  %
  \[
  \forall \, B \in \B(\Rset^{d'}), \quad P_Y(B) = P_X(g^{-1}(B)).
  \]
\end{teorema}
%
\begin{proof}
  Este resultado es obvio. $g$ siendo medible, para todo $B \in \B(\Rset^{d'})$,
  por definici\'on $g^{-1}(B) \in \B(\Rset^d)$.  Adem\'as, si $P_X$ es la medida
  (de  probabilidad) asociado  al  espacio de  salida  de $g$,  el resultado  es
  consecuencia  del  teorema  de la  medida  im\'agen~\ref{Teo:MP:MedidaImagen},
  pagina~\pageref{Teo:MP:MedidaImagen}.
\end{proof}
%
\noindent (Ver ej.~\cite{Muk00, JacPro03, AthLah06, Bog07:v2, Coh13}).


Es sencillo  probar de que  cualquier combinaci\'on de funciones  medibles queda
medible, cualquier  producto (adecuado) de  functiones medible queda  medible, y
que si $\{ f_k \}_{k=1}^{d'}$ son $(\B(\Rset^d),\B(\Rset))$-medible, entonces $f
=          (f_1         ,          \ldots         ,          f_{d'})$         es
$(\B(\Rset^d),\B(\Rset^{d'}))$-medible~\cite{AthLah06}.

% \SZ{No se  todav\'ia si ser\'a \'util  tratar del caso de l\'imite  de series de
%   funciones medibles.}


Mencionamos  de que si  $\X =  X(\Omega)$ es  discreto, entonces  $\Y =  g(\X) =
Y(\Omega)$ ser\'a discreto tambi\'en, y:
%
\begin{teorema}[Funci\'on de masa por transformaci\'on medible]
\label{Teo:MP:TransformacionMasa}
%
  Sean   $X$,   vector  aleatorio   $d$-dimensional   discreto,  $g:(\Rset^d   ,
  \B(\Rset^d)) \mapsto (\Rset^{d'} , \B(\Rset^{d'}))$ una funci\'on medible, e \
  $Y =  g(X)$ necesariamente discreto  $d'$-dimensional sobre $\Y =  g(\X)$.  La
  distribuci\'on de $Y$ es relacionada a la de $X$ por la relaci\'on
  %
  \[
  \forall \, y \in \Y, \quad p_Y(y) = \sum_{x \in g^{-1}(y)} p_X(x).
  \]
\end{teorema}
%
\begin{proof}
  El resultado es inmediato.
\end{proof}
%
\noindent En particular,  si $g$ es inyectiva (necesariamente  biyectiva de $\X$
en $\Y$),  el vector  de probabilidad queda  invariante, $p_Y =  p_X$; solamente
cambian los estados.

Es  importante mencionar de  que con  $\Y$ discreto,  $\X$ no  es necesariamente
discreto~\cite{AthLah06}. Por ejemplo, $Y = \un_{X>0}$  es tal que $\Y = \{ 0 \; 1 \}$ a pesar de que $\X$ puede ser no discreto.

Tratar de las variables aleatorias  continuas resuelta mas delicado. Vimos en el
ejemplo   precediente  de   que  el   caracter  continuo   puede   perderse  por
transformaci\'on. De la misma manera, en un ejemplo de la secci\'on precediente,
vimos  que  $Y =  X_1  \un_{X_2>0}$ con  $X_i$  independientes  uniformes es  ni
continua,  ni  discreta.  En  el  enfoque  de  variables  continuas,  una  clase
importante  de funciones  en la  cual  no vamos  a interesar  son las  funciones
continuas (y diferenciables):
%
\begin{lema}[Continuidad y caracter medible]
\label{Lem:MP:ContinuidadCaracterMedible}
%
  Sea   $g:   \Rset^d   \mapsto   \Rset^{d'}$   continua.   Entonces,   $g$   es
  $(\B(\Rset^d),\B(\Rset^{d'}))$-medible.
\end{lema}
%
\begin{proof}
  Por continuidad,  la pre-im\'agen de  un abierto de  $\Rset^{d'}$ por $g$  es un
  abierto  de $\Rset^d$  y entonces  es en  $\B(\Rset^d)$. La  prueba  se cierra
  recordandose  de  la   definici\'on  de  $\B(\Rset^{d'})$,  $\sigma$-\'algebra
  generada por los abiertos de $\Rset^{d'}$.
\end{proof}

En lo  que sigue, nos interesamos  m\'as especialmente al caso  de funciones $g:
(\Rset^d ,  \B(\Rset^d)) \mapsto (\Rset^d ,  \B(\Rset^d))$.  De hecho,  si $d' <
d$,   es    sencillo   llegar   al   caso    considerado   a\~nandiendo   $d-d'$
transformaciones. Por ejemplo, con $d = 2$  si nos interesamos a $X_1 + X_2$, se
puede considerar $\begin{bmatrix} X_1 + X_2 & X_2 - X_1\end{bmatrix}^t$ y llegar
a la variable de  inter\'es por calculo de marginal. Si $d'  > d$ la situaci\'on
es  m\'as  delicada,  $g(Y)$  viviendo  sobre una  variedad  $d$-dimensional  de
$\Rset^{d'}$.

En  el caso  de vectores  aleatorios continuos  $X$ admitiendo  una  densidad de
probabilidad,  una pregunta  natural  es entonces  de  saber si  se conserva  la
continuidad y la existencia de una densidad, as\'i que su forma. La respuesta es
dada por el teorema siguiente~\cite{Bre88, JacPro03, AthLah06, Coh13, HogMck13}:
%
\begin{teorema}[Densidad de probabilidad por transformaci\'on continua inyectiva
  diferenciable]
\label{Teo:MP:TransformacionInyectivaDensidad}
%
  Sean $X$, vector aleatorio  $d$-dimensional continuo y admitiendo una densidad
  de probabilidad  $p_X$, \ $g:\Rset^d  \mapsto \Rset^d$ una  funci\'on continua
  inyectiva y diferenciable tal que
  %  ~\footnote{\modif{De  hecho,  se   puede  extender  el  resultado  para  un
  %      determinente del  Jacobiano cancelandose  en  un conjunto  de punto  de
  %     medida  de Lebesgue nula y en  los $y$ donde se  cancela el determinente
  %       del  Jacobiano,  la   densidad  va   a  ser   divergente  (divergencia
  %     integrable).}}
 $\left| \Jac_g \right| > 0$,
  %
 donde $\Jac_g$ denota la  matriz de componentes \ $\frac{\partial g_i}{\partial
   x_j}$, \ matriz Jacobiana de  la transformaci\'on \ $g \equiv \begin{bmatrix}
   g_1(x_1 , \ldots , x_d) & \cdots & g_d(x_1 , \ldots , x_d) \end{bmatrix}^t$ \
 y \ $|\cdot|$ representa el valor absoluto del determinante de la matriz. Sea \
 $Y = g(X)$.   Entonces $Y$ es continua admitiendo  una densidad de probabilidad
 $p_Y$ de soporte $\Y = g(\X) = Y(\Omega)$ tal que
  %
  \[
  \forall \,  y \in  \Y, \quad p_Y(y)  = p_X(g^{-1}(y))  \left| \Jac_{g^{-1}}(y)
  \right|.
  \]
\end{teorema}
%
\begin{proof}
  Por definici\'on, $X$ admitiendo una densidad y $g$ siendo medible,
  %
  \[
  \forall \, B \in \B(\Rset^d),  \quad P_Y(B) = P_X(g^{-1}(B)) = \int_{g^{-1}(B)
    \cap \X} p_X(x) \, dx.
  \]
  %
  Por cambio de  variable $x = g^{-1}(y)$ ($g$  siendo inyectiva, el antecedante
  es  \'unico por  definic\'on)  y notando  de  que $g\left(  g^{-1}(B) \cap  \X
  \right) = B \cap \Y$,
  %
  \[
  \forall \, B  \in \B(\Rset^d), \quad P_Y(B) =  \int_{B \cap \Y} p_X(g^{-1}(y))
  \, \left| \Jac_{g^{-1}}(y) \right| \, dy
  \]
  %
  lo que  cierra la  prueba~\footnote{La aparici\'on de  la Jacobiana  viene del
    mismo enfoque que  el cambio de variables en la  integraci\'on de Rieman. De
    hecho,  como  lo hemos  visto,  $\mu_L(B)  = |B|$  es  el  volumen  y de  la
    definici\'on  mismo  del determinente,  para  cualquier  matriz cuadrada  el
    volumen se escribe \ $\mu_L(M B) = |M  B| = |M| |B| = |M| \mu_L(B)$ donde la
    misma escritura  $|\cdot|$ representa el valor absoluto  del determinente de
    una matriz. Esta notaci\'on se justifica precisamente por su significaci\'on
    de  volumen, y  el  resultado es  inmediato para  $g(x)  = M  x$.  La  forma
    general, para una transformaci\'on m\'as general a partir de un desarollo de
    Taylor al orden 1~\cite{AthLah06, Coh13}.}.
\end{proof}

El caso escalar puede ser visto como caso particular, dando:
%
\begin{corolario}
\label{Cor:MP:TransformacionInyectivaDensidadEscalar}
%
  Sean  $X$,   variable  aleatoria  continua   y  admitiendo  una   densidad  de
  probabilidad $p_X$, $g:\Rset \mapsto  \Rset$ una funci\'on continua, inyectiva
  y  diferenciable e  \ $Y  = g(X)$.   Entonces $Y$  es continua  admitiendo una
  densidad de probabilidad $p_Y$ tal que
  %
  \[
  \forall  \,   y  \in  \Y,   \quad  p_Y(y)  =  p_X(g^{-1}(y))   \left|  \frac{d
      g^{-1}(y)}{dy} \right|.
  \]
\end{corolario}
%
\noindent  De hecho,  se puede  ver estos  resultados esquematicamente  como una
``conservaci\'on'' de  probabilidad, $p_X(x)  dx = p_Y(y)  dy$, el  volumen $dy$
siendo   relacionado  al  $dx$   a  traves   de  la   Jacobiana  (ver   nota  de
pie~\ref{Foot:SZ:Jacobiana}).

Una  forma  alternativa  de derivar  este  corrolario  consiste  a salir  de  la
funci\'on   de   repartici\'on,   notando   de   que   $g$   es   necesariamente
mon\'otona~\footnote{Fijense de que $P(X \ge x) = 1 -  P(X < x) = 1 - P(X \le x) +
  P(X =  x)$, pero $X$ siendo  continua, $ P(X =  x) = 0$.}: si  $y \not\in \Y$,
necesariamente $p_Y = 0$ ($F_Y(y) = 1$ \ si \ $y > \sup \Y$ \ y \ $F_Y(y) = 0$ \
si \ $y < \inf \Y$) \ y para cualquier \ $y \in \Y$,
%
\[
F_Y(y) = P(Y \le y) = P(g(X) \le y) =
\left\{\begin{array}{lll}
P(X \le g^{-1}(y)) = F_X(g^{-1}(y)) & \mbox{si} & g \quad \mbox{es creciente}\\[2.5mm]
%
P(X \ge g^{-1}(y)) = 1 - F_X(g^{-1}(y)) & \mbox{si} & g \quad \mbox{es decreciente}
\end{array}\right..
\]
%
El  resultado  se  obtiene  calculando  las  derivadas  del  primer  y  \'ultimo
t\'erminos respecto de la variable transformada $y$.

Si $g$ no es inyectiva, $g^{-1}$  es multivaluada o multiforme. En este caso, se
puede todav\'ia  tratar el problema, particionando $\Rset^d$  en conjuntos donde
$g$ es inyectiva, dando
%
\begin{teorema}[Densidad   de   probabilidad   por   transformaci\'on   continua
  no-inyectiva diferenciable]
\label{Teo:MP:TransformacionNoInyectivaDensidad}
%
Sean $X$, vector aleatorio $d$-dimensional continuo y admitiendo una densidad de
probabilidad $p_X$, $g:(\Rset^d ,  \B(\Rset^d)) \mapsto (\Rset^d , \B(\Rset^d))$
una   funci\'on   continua  y   diferenciable.    Denotamos  $\left\{   \X_{[k]}
\right\}_{k=0}^m$ la partici\'on de $\X$  tal que $\left| \Jac_g(y) \right| = 0$
sobre  $\X_{[0]}$ y  para todos  $k \ge  1$, \  $g: \X_{[k]}  \mapsto \Y$  \ sea
inyectiva  y tal  que  \ $\left|  \Jac_g(y) \right|  >  0$. \  Suponemos de  que
$\X_{[0]}$ sea de  medida de Lebesgue nula, notamos \  $g_k^{-1}$ \ la funci\'on
inversa de  \ $g$ \  sobre \ $g(\X_{[k]})$  \ (rama $k$-\'esima de  la funci\'on
multivaluada $g^{-1}$), \ $\Jac_{g_k^{-1}}$ \ su matriz Jacobiana y \ $I(y) = \{
k, \: y \in g(\X_{[k]}) \}$ \ los indices tales que \ $y$ \ tiene un inverso por
$g_k$.   Esto es  ilustrado figura~\ref{Fig:MP:TransformacionVA}  para $d  = 1$.
Entonces $Y$ es continua admitiendo una densidad de probabilidad $p_Y$ tal que
  %
  \[
  \forall  \, y  \in \Y,  \quad p_Y(y)  = \sum_{k  \in  I(y)} p_X(g_k^{-1}(y))
  \left| \Jac_{g_k^{-1}}(y) \right|.
  \]
  %
  En el caso escalar $d = 1$ esto se formula
  %
  \[
  \forall \, y \in \Y, \quad  p_Y(y) = \sum_{k \in I(y)} p_X(g_k^{-1}(y)) \left|
    \frac{d g_k^{-1}(y)}{dy} \right|.
  \]
\end{teorema}
%
\begin{proof}
  Sufice escribir  \ $B  = \bigcup_{k =  0}^m \left(  B \cap g(\X_k)  \right)$ \
  uni\'on de borelianos disjuntos, notar  de que por consecuencia \ $g^{-1}(B) =
  \bigcup_{k =0}^m g^{-1}\left( B \cap  g(\X_k) \right)$ \ uni\'on de borelianos
  disjuntos \  y \  por linearidad escribir  la integraci\'on  sobre $g^{-1}(B)$
  como la  suma de  integrales sobre $g^{-1}\left(  B \cap g(\X_k)  \right)$. Se
  cierra la  prueba notando de  que \ $g^{-1}\left(  B \cap g(\X_0)  \right)$ es
  necesario  de medida de  Lebesgue nula,  siendo la  integral nula  y de  que \
  $g^{-1}\left( B \cap g(\X_k) \right) = g_k^{-1}\left( B \cap g(\X_k) \right)$.
\end{proof}
%

\begin{ejemplo}[Ejemplo de transformaci\'on no biyectiva]
\label{Ej:MP:TransformacionNoBiyectiva}
%
  Sea $X$ definido sobre  \ $\X = \Rset$ \ y la  transformaci\'on de variables \
  $Y = X^2$.   Se tiene \ $y  = g(x) = x^2$, continua  diferenciable de derivada
  nula  sobre \  $\X_{[0]} =  \{ 0  \}$, de  medida nula,  cuyas inversas  son \
  $g_1^{-1}(y) = \sqrt{y}$ \ sobre \ $\X_{[1]} = \Rset_-^*$ \ y \ $g_2^{-1}(y) =
  -   \sqrt{y}$  sobre   \   $\X_{[2]}   =  \Rset_+^*$;   luego   \  $p_Y(y)   =
  \frac{p_X(\sqrt{y})  +   p_X(-\sqrt{y})}{2  \sqrt{y}}$,   \  sobre  \   $\Y  =
  \Rset_+^*$.
\end{ejemplo}

De nuevo, en el caso escalar, se puede salir de la funci\'on de repartici\'on
%
\[
F_Y(y) =  P(Y \le y) = P(  g(X) \le y )  = \sum_{k=1}^m P\left( X  \in \X_{[k]} \cap
  g_k^{-1}(-\infty \; y] \right)
\]
%
($\X_{[0]}$  siendo  de medida  nula,  sobre  este  dominio la  probabilidad  es
cero). Sea $\Y_{[k]} = g_k(\X_{[k]})$. Ahora, si $y \not\in I(y)$,
%
\[
P\left(   X   \in  \X_k   \cap   g_k^{-1}(-\infty  \;   y]  \right)   =
\left\{\begin{array}{lll}
%
P(X \in \X_{[k]}) & \mbox{si} & y > \sup \Y_{[k]}\\[2.5mm]
%
0 & \mbox{si} & y < \inf \Y_{[k]}\\[2.5mm]
\end{array}\right.
\]
%
dando una derivada nula. Si $y \in I(y)$,
%
\[
P\left(   X   \in  \X_k   \cap   g_k^{-1}(-\infty  \;   y]  \right)   =
\left\{\begin{array}{lll}
%
F_X(g_k^{-1}(y)) - F_X(\inf \Y_{[k]}) & \mbox{si} & g_k \quad \mbox{es creciente}\\[2.5mm]
%
F_X(\sup \Y_{[k]}) - F_X(g_k^{-1}(y))  & \mbox{si} & g_k \quad \mbox{es decreciente}
\end{array}\right..
\]
%
El   resultado   sigue  diferenciando   este   resultado.   Esto  es   ilustrado
figura~\ref{Fig:MP:TransformacionVA}.

\begin{figure}[h!]
\begin{center} \begin{tikzpicture}%[scale=.9]
\shorthandoff{>}
%
\pgfmathsetmacro{\sx}{1.75};% x-scaling
%
% transformacion g' = 0 medida nula
\begin{scope}
%
\pgfmathsetmacro{\sy}{1.3};% y-scaling 
%
\draw[>=stealth,->] ({-1.9*\sx-.1},0)--({\sx*4+.25},0) node[right]{\small $x$};
\draw[>=stealth,->] (0,{\sy*(-3*.9^3+1)-.1})--(0,{\sy+.5}) node[above]{\small $g(x)$};
%
\draw[thick]
plot[domain=-1.9:-1,samples=100] ({\sx*\x},{\sy*(3*(\x+1)^3+1)})
-- plot[domain=-1:.5,samples=100] ({\sx*\x},{\sy*cos(120*(\x+1))})
-- plot[domain=.5:2.5,samples=100] ({\sx*\x},{\sy*(.75*sin(90*(\x-1.5))-.25)})
-- plot[domain=2.5:4,samples=100] ({\sx*\x},{\sy*(cos(60*(\x-2.5))-.5)})
;
%
\draw[dashed] ({-1.9*\sx},{.75*\sy})--({4*\sx},{.75*\sy});
\draw (0,{.75*\sy}) node[above right]{\small $y$};
%
\draw[dashed] ({-\sx*(1+(.25/3)^(1/3))},{.75*\sy})--({-\sx*(1+(.25/3)^(1/3))},0)
node[scale=.7]{$\bullet$} node[below,scale=.7]{$g_1^{-1}(y)$};
%
\draw[dashed] ({\sx*(acos(.75)/120-1)},{.75*\sy})--({\sx*(acos(.75)/120-1)},0)
node[scale=.7]{$\bullet$} node[below,scale=.7]{$g_2^{-1}(y)$};
%
%
\draw[dotted] ({-\sx},{-\sy-.25})--({-\sx},{\sy+.25});
\draw[>=stealth,->] ({-\sx*1.9-.1},-.75)--({-\sx},-.75);
\draw ({-1.5*\sx},-.75) node[below,scale=.8]{\small $\X_1$};
%
\draw[dotted] ({.5*\sx},{-\sy-.25})--({.5*\sx},{\sy+.25});
\draw[>=stealth,<->] ({-\sx},-.75)--({.5*\sx},-.75);
\draw ({-.25*\sx},-.75) node[below,scale=.8]{\small $\X_2$};
%
\draw[dotted] ({2.5*\sx},{-\sy-.25})--({2.5*\sx},{\sy+.25});
\draw[>=stealth,<->] ({.5*\sx},-.75)--({2.5*\sx},-.75);
\draw ({1.5*\sx},-.75) node[below,scale=.8]{\small $\X_3$};
%
\draw[>=stealth,<-] ({2.5*\sx},-.75)--({4*\sx},-.75);
\draw ({3.25*\sx},-.75) node[below,scale=.8]{\small $\X_4$};
\end{scope}
%
%
% % reparticion
% \begin{scope}[xshift=8.5cm]
% %
% \pgfmathsetmacro{\sy}{2};% y-scaling 
% %
% \draw[>=stealth,->] (-.6,0)--({\sx*4+.25},0) node[right]{\small $x$};
% \draw[>=stealth,->] (0,-.25)--(0,{\sy+.5}) node[above]{\small $F_X$};
% %
% \draw[thick] (-.5,0)--(0,0)--(\sx,{\sy/2})--({2*\sx},{\sy/2})
% -- plot[domain=2:3,samples=100] ({\sx*\x},{\sy*(1+(\x-2)^(3/2))/2})
% -- ({\sx*4},\sy);
% %
% \draw (\sx,0)--(\sx,-.1) node[below,scale=.9]{\small $1$};
% \draw ({2*\sx},0)--({2*\sx},-.1) node[below,scale=.9]{\small $2$};
% \draw ({3*\sx},0)--({3*\sx},-.1) node[below,scale=.9]{\small $3$};
% \draw (\sx,0)--(\sx,-.1) node[below,scale=.9]{\small $1$};
% %
% \draw (0,{\sy/2})--(-.1,{\sy/2}) node[left,scale=.7]{\small $1/2$};
% \draw (0,\sy)--(-.1,\sy) node[left,scale=.7]{\small $1$};
% %
% \draw ({\sx*2.25},-1) node{\small (b)};
% \end{scope}
%
\end{tikzpicture} \end{center}
%
\leyenda{(a): Ilustraci\'on  de una transformaci\'on  $g$ no inyectiva,  tal que
  $\X_{[0]} = \{ x, \: g'(x) = 0 \}$, representado por las lineas punteadas ($x$
  correspondiente), es de medida de  Lebesgue nula.  Los $\X_{[k]}$ son descrito
  debajo de cada dominio.  La linea discontinua  da un nivel $y$ y los puntos en
  el eje $x$ representan $g_k^{-1}(y), \: k \in I(y)$; en el ejemplo, $I(y) = \{
  1  \;  2 \}$  \  y,  suponiendo  de  que  $\X  = \Rset$,  \  $F_Y(y)  =
  F_X(g_1^{-1}(y)) + 1-F_X(g_2^{-1}(y))$.}
\label{Fig:MP:TransformacionVA}
\end{figure}

Una  tercera alternativa,  a pesar  que sea  delicado, es  de apoyarse  sobre la
teoria de las  distribuciones y expresar como \  $\displaystyle p_Y(y) = \int_\X
p_X(x) \,  \delta(y-g(x)) \, dx$,  donde se usa  la expansi\'on de  la funci\'on
delta  en  t\'erminos de  sus  ceros:  \  $\delta(y-g(x)\,)= \sum_{k  \in  I(y)}
\frac{1}{\left|      g_k'\left(      g_k^{-1}      (y)     \right)      \right|}
\delta(x-g_k^{-1}(y))$~\cite{ManWol95}.

Es  importante  notar  de  que  la  condici\'on $\X_{[0]}$  de  medida  nula  es
importante. El el caso contrario, $Y$ no  queda continua como se lo puede ver en
el ejemplo siguiente.
%
\begin{ejemplo}[Transformaci\'on con $\mu_L\left( \X_{[0]} \right) \ne 0$]
\label{Ej:MP:X0MedidaNoNula}
%
Sea \ $X$  \ uniforme sobre \  $\X = ( 3 \; 3)$ \  y \ $Y = g(X)$  \ con \
$g(x) = \left( 1 + \cos\left( (|x|-1) \frac{\pi}{2} \right) \right) \un_{(1 \;
   3)}(|x|) +  2  \un_{[0 \;  1]}(|x|)$.  Esta  funci\'on es  representado
figura~\ref{Fig:MP:TransformacionVANoContinua}-(a).   Claramente,  \  $g$  \  es
continua y diferenciable sobre $\X$, pero con \ $\X_{[0]} = [ -1 \; 1]$ que
no  es  de medida  nula.   Saliendo  de $F_Y(y)  =  P(g(X)  \le  y)$ se  calcula
sencillamente  $F_Y(y)  = \frac23  \left(  1  -  \frac1\pi \arccos(y-1)  \right)
\un_{[0  \;   2)}  +   \un_{[  2  \;   +\infty)}(y)$,  ilustrada
figura~\ref{Fig:MP:TransformacionVANoContinua}-(b).   Claramente  \  $F_Y$ \  es
discontinua en \ $y = 2$: \ $Y$ \ no es continua.
  %
  \begin{figure}[h!]
  \begin{center} \begin{tikzpicture}%[scale=.9]
\shorthandoff{>}
%
\pgfmathsetmacro{\r}{.05};% radius arc non continuity F_X y/o p_X
%
% transformacion g' = 0 medida no nula
\begin{scope}
%
\draw[>=stealth,->] (-3.6,0)--(3.75,0) node[right]{\small $x$};
\draw[>=stealth,->] (0,-.25)--(0,2.5) node[above]{\small $g(x)$};
%
\draw[thick]
(-3.4,0)--(-3,0)--
plot[domain=-3:-1,samples=100] (\x,{1+cos(90*(1+\x))})--
(1,2)--
plot[domain=1:3,samples=100] (\x,{1+cos(90*(1-\x))})--
(3.5,0);
%
\draw (-3,0)--(-3,-.1) node[below,scale=.8]{\small $-3$};
\draw (-2,0)--(-2,-.1) node[below,scale=.8]{\small $-2$};
\draw (-1,0)--(-1,-.1) node[below,scale=.8]{\small $-1$};
\draw (1,0)--(1,-.1) node[below,scale=.8]{\small $1$};
\draw (2,0)--(2,-.1) node[below,scale=.8]{\small $2$};
\draw (3,0)--(3,-.1) node[below,scale=.8]{\small $3$};
%
\draw (0,1)--(-.1,1) node[left,scale=.8]{\small $1$};
\draw (-.1,2) node[above left,scale=.8]{\small $2$};
%
\draw[>=stealth,<->] (-3,-.5)--(-1,-.5); \draw (-2,-.5) node[below,scale=.9]{\small $\X_{[1]}$};
\draw[>=stealth,<->] (1,-.5)--(3,-.5); \draw (2,-.5) node[below,scale=.9]{\small $\X_{[2]}$};
\draw[>=stealth,<->] (-1,-.5)--(1,-.5); \draw (0,-.5) node[below,scale=.9]{\small $\X_{[0]}$};
%
\draw(0,-1.5) node{\small (a)};
\end{scope}
%
%
% reparticion
\begin{scope}[xshift=7cm]
%
\pgfmathsetmacro{\sx}{1.5};
\pgfmathsetmacro{\sy}{2};% y-scaling 
%
\draw[>=stealth,->] (-.6,0)--({3*\sx+.5},0) node[right]{\small $y$};
\draw[>=stealth,->] (0,-.25)--(0,{\sy+.25}) node[above]{\small $F_Y$};
%
\draw[thick] (-.5,0)--(0,0)--
plot[domain=0:2,samples=250] ({\sx*\x},{\sy*2*(1-acos(\x-1)/180)/3});
%
\draw ({2*\sx+\r},{2*\sy/3+\r}) arc (90:260:\r);
%--
\draw[dotted] ({2*\sx},{2*\sy/3})--({2*\sx},\sy);
\draw[thick] ({2*\sx},\sy) node[scale=.7]{$\bullet$}--({3*\sx},\sy);
%
\draw ({\sx},0)--({\sx},-.1) node[below,scale=.8]{\small $1$};
\draw ({2*\sx},0)--({2*\sx},-.1) node[below,scale=.8]{\small $2$};
\draw ({3*\sx},0)--({3*\sx},-.1) node[below,scale=.8]{\small $3$};
%
\draw (0,{2*\sy/3})--(-.1,{2*\sy/3}) node[left,scale=.7]{\small $2/3$};
\draw (0,\sy)--(-.1,\sy) node[left,scale=.7]{\small $1$};
%
\draw({1.75*\sx},-1.5) node{\small (b)};
\end{scope}
%
\end{tikzpicture} \end{center}
  %
  \leyenda{En (a) se dibuja $g(x)  = \left( 1 + \cos\left( (1-|x|) \frac{\pi}{2}
      \right) \right) \un_{(1  \; 3)}(|x|) + 2 \un_{[0  \; 1]}(|x|)$. Suponiendo
    de que $\X = (-3 \; 3)$, claramente  $\X_{[0]} = [ -1 \; 1]$ no es de medida
    nula, dando para $X$ uniforme sobre  $\X$ la variable $Y = g(X)$ no continua
    de funci\'on de repartici\'on representenda en (b).  }
  \label{Fig:MP:TransformacionVANoContinua}
  \end{figure}
\end{ejemplo}

Un ejemplo de  cambio de transformaci\'on puede sevir a  calcular la densidad de
probabilidad de una suma:
%
\begin{ejemplo}[Distribuci\'on de la suma de vectores aleatorios]
\label{Ej:MP:Suma}
%
  Sean \  $X$ \ e  \ $Y$ \  dos vectores aleatorios conjuntamente  continuos, de
  densidad de  probabilidad conjunta $p_{X,Y}$,  y sea el  vector \[V = X  + Y.\]
  Queremos calcular la a partir de la densidad de probabilidad de $V$.  Por esto,
  se puede considerar la transformaci\'on biyectiva
  %
  \[
  g: (x,y) \mapsto (u,v) = (x,x+y).
  \]
  %
  Entonces
  %
  \[
  g^{-1}(u,v) = (u,v-u)
  \]
  %
  y la matriz Jacobiana es
  %
  \[
  J_{g^{-1}} = \begin{bmatrix} I & -I \\ 0 & I \end{bmatrix}
  \]
  %
  donde $I$ es la matriz identidad $d$-dimensional y $0$ la matriz nula de misma
  dimension. Claramente\ $\left| J_{g^{-1}} \right| = 1$ \ as\'i que
  %
  \[
  p_{U,V}(u,v) = p_{X,Y}(u,v-u)
  \]
  %
  como lo pudimos intuir. Adem\'as, por marginalizaci\'on, inmediatamente
  %
  \[
  p_V(v) = \int_{\Rset^d} p_{X,Y}(u,v-u) \, du.
  \]
  %
  Si \ $X$ \ e \ $Y$  \ son independientes, \ $p_{U,V}(u,v) = p_X(u) p_V(v-u)$ \
  y la f\'ormula integral se escribe
  %
  \[
  p_V(v) = \int_{\Rset^d} p_X(u) p_Y(v-u) \, du = \int_{\Rset^d} p_Y(u) p_X(v-u)
  \, du
  \]
  %
  (por  cambio  de variable  en  la  secunda  expresi\'on).  Esta  f\'ormula  es
  conocida    como     {\it    producto    de     convoluci\'on}    entre    las
  funciones~\footnote{Este  producto  no  impone   de  que  las  funciones  sean
    densidades de  probabilidad.  Una condici\'on suficiente para  que existe es
    que     las     funciones     sean     $L^1$     (ver     desigualdad     de
    Cauchy-Bunyakovsky-Schwarz).} $p_X$ y $p_Y$.
\end{ejemplo}


%|q(y^1,\ldots,y^d)\ dy^1\cdots dy^d| = |p(x^1,\ldots,x^d) \ dx^1\cdots dx^d| .

%% Ejercicio: Estudiar el caso multivaluado / Resolver un ej. 


% ============================ Leyes condicionales ============================= %

%\seccion{Leyes condicionales}
\label{Sec:MP:LeyesCondicionales}

Al  considerar un  par de  vectores aleatorios  \ $X$  \ e  \ $Y$,  una pregunta
natural puede ser  c\'omo caracterizar el vector $Y$ si  ``observamos $X = x$''.
En otras palabras,  la pregunta es describir la ley  de $Y$ ``sabiendo \modif{(o
  observando)}  que  $X  =  x$''.   En  lo que  sigue,  para  fijar  notaci\'on,
consideramos $(X,Y): (  \Omega , \A) \mapsto (\Rset^{d_X}  \! \times \Rset^{d_Y}
\,   ,  \,  \B(\Rset^{d_X}   \!   \times   \Rset^{d_Y})  )$   tal  que   $X$  es
$d_X$-dimensional   e   $Y$   es   $d_Y$-dimensional   (incluyendo   los   casos
escalares).  \modif{Escrimiremos  de nuevo  \  $\X  = X(\Omega)$  \  e  \ $\Y  =
  Y(\Omega)$.}


% ---------- Caso discreto

\paragraph{Caso $\boldsymbol{X}$ discreta:}
Un caso  sencillo a estudiar  es cuando \  \modif{$\X$} \ es discreto.   En este
caso, para  cualquier $x  \in \X$, tenemos  $P_X(x) = P(X  = x)  \ne 0$ y  de la
definici\'on de  la probabilidad condicional Def.~\ref{Def:MP:ProbaCondicional},
$\displaystyle P(Y \in A | X = x) = \frac{P\big( (Y \in A) \cap (X = x) \big)}{P(X=x)}$ define
una medida de probabilidad que llamamos medida de probabilidad condicional.
% y que
%denotaremos
%%
%\[
%P_{Y|X=x}(A) = P(Y \in A | X = x).
%\]
%
Siendo una medida de probabilidad, nos referimos a la subsecci\'on anterior para
definir una funci\'on de  repartici\'on tomando $\displaystyle A = \prod_{i=1}^d
(-\infty \; y_i]$, caracterizando completamente la medida de probabilidad:
%
\begin{definicion}[Medida de probabilidad y funci\'on de repartici\'on condicional ($X$ discreto)]\label{Def:MP:ReparticionCondicionalDiscreta}
%
  \modif{Por   cualquier   $x  \in   \X$,   la   medida   condicional  de   $Y$,
    condicionalmente a $(X = x)$, se define por
    %
    \[
    \forall \: A \in \B\left( \Rset^{d_X} \right), \qquad P_{Y|X=x}(A) = P(Y \in
    A | X = x),
    \]
}
  %
  y la funci\'on de repartici\'on condicional \modif{se define por},
  %
  \[
  \forall \:  x \in  \X, \: y  \in \Y, \qquad  F_{Y|X=x}(y) =  P( \left. Y  \le y
  \right| X = x ) = \frac{P\left(  \left(Y \le y \right) \cap \left(X = x\right)
    \right)}{P\left(X = x\right)}.
  \]
\end{definicion}

Ahora, cuando $Y$  tambi\'en es discreta, se puede definir  la funci\'on de masa
discreta de probabilidad condicional, y si $Y$ es continua y admite una densidad
de probabilidad, se puede definir una densidad de probabilidad condicional:
%
\begin{definicion}[Funci\'on de masa o densidad de probabilidad condicional ($X$
  discreta)]
\label{Def:MP:ReparticionCondicionalDiscreta}
%
  Por  definici\'on,  cuando  $\Y$  es   discreta,  la  funci\'on  de  masa  de
  probabilidad condicional \modif{de $Y$ condicionalmente a $X = x$} es,
  %
  \[
  \forall \: x \in \X, \: y \in \Y, \quad p_{Y|X=x}(y) = P\left( \left. Y = y \right|
  X  =  x \right)  =  \frac{P\big(  (Y  = y)  \cap  (X =  x) \big)}{P(X = x)}.
  \]
  %
  Si $Y$ es continua, es sencillo ver que $P_{Y|X=x} \ll P_Y$, i.e., $P_Y(B) = 0
  \: \Rightarrow \: P_{Y|X=x}(B) = 0$.   Si $Y$ admite una densidad con respecto
  a la  medida de  Lebesgue, $P_Y \ll  \mu_L$ medida  de Lebesgue, es  claro que
  tambi\'en     $P_{Y|X=x}     \ll      \mu_L$,     y     por     teorema     de
  Radon-Nikod\'ym~\ref{Teo:MP:RadonNikodym}, $P_{Y|X=x}$  admite una densidad de
  probabilidad  (con   respecto  a  la  medida  de   Lebesgue)  que  denotaremos
  $p_{Y|X=x}$,
  %
  \[
  \forall \: B, \quad P_{Y|X=x}(B) = \int_B p_{Y|X=x}(y) \, dy.
  \]
  %
  A partir de la funci\'on de repartici\'on, obtenemos
  %
  \[
  p_{Y|X=x}(y) = \frac{\partial^{d_Y} F_{Y|X=x}(y)}{\partial y_1 \ldots \partial
    y_{d_Y}}.
  \]
\end{definicion}


% ---------- Caso continuo

\paragraph{\modif{Caso general:}}
Cuando  $X$  es   continua,  el  problema  es  m\'as   sutil  porque  $P(X=x)  =
0$. Entonces, no  se puede usar la definici\'on  de la probabilidad condicional,
siendo el  evento $(X=x)$ de probabilidad  nula.  Sin embargo,  se pueden seguir
los  pasos de R\'enyi~\cite[Cap.~5]{Ren}  o de  Feller~\cite[Cap.~10]{Fel71} por
ejemplo para resolver el problema, llegando \modif{en el contexto continuo} a un
resultado intuitivo como en el caso discreto.

\modif{Sea $B  \in \B(\Rset^{d_Y})$ y definimos  $\nu_B(A) =  P\big( \left( X
  \in A \right)  \cap \left( Y \in B \right) \big)$  sobre $\left( \Rset^{d_X} ,
  \B(\Rset^{d_X}) \right)$.  Es sencillo ver que siendo dado $B$, $\nu_B$ define
una medida}. Adem\'as, $\nu_B \ll P_X$, i.e.,  $P_X(A) = P(X \in
A) = 0 \: \Rightarrow \: 0 = P\Big(  (X \in A) \cap (Y \in B) \Big) = \nu_B(A)$.
Por  teorema de  Radon-Nikod\'ym~\ref{Teo:MP:RadonNikodym},  $\nu_B$ admite  una
densidad $g_B$ con respecto a $P_X$,
%
\[
\forall \:  A \in \B\left( \Rset^{d_X} \right),  \quad P\big( (X \in  A) \cap (Y
\in B) \big) = \int_A g_B(x) \, dP_X(x).
\]
%
Claramente \ $g_B \ge 0$, \ y de \ \modif{$P(X \in A) = P\big( (X \in A) \cap (Y
  \in  B) \big)  + P\big(  (X \in  A)  \cap (Y  \in \overline{B}  ) \big)$,  \ie
  $\displaystyle \forall  \: A  \in \B\left( \Rset^{d_X}  \right), \qquad  0 \le
  P\big( (X  \in A) \cap  (Y \in \overline{B})  \big) = \int_A dP_X(x)  - \int_A
  g_B(x) \, dP_X(x)$}, se  obtiene \ $0 \le g_B \le 1$.   En realidad, tenemos \
$g_B \le 1$  \ $P_X$-casi siempre, pero olvidando  esta subtilesa, llamaremos la
funci\'on  \   $g_B$  \  medida  de  probabilidad   condicional\modif{,  y,  por
  continuaci\'on se  define la funci\'on de repartici\'on  condicional de manera
  siguiente:}
%
\begin{definicion}[Medida   de  probabilidad   y   funci\'on  de   repartici\'on
  condicional]
\label{Def:MP:MedidaCondicional}
%
  La medida de probabilidad condicional de $P_{Y|X=x}$ es definida tal que
  %
  \[
  \forall \: (A,B) \in  \B\left( \Rset^{d_X} \right) \times \B\left( \Rset^{d_Y}
  \right), \quad P\big( (X \in A) \cap  (Y \in B) \big) = \int_A P_{Y|X=x}(B) \,
  dP_X(x).
  \]
  %
  Tomando $B = \optimes_i \left( -\infty \; y_i \right]$ se obtiene la funci\'on
  de repartici\'on condicional a partir de
  %
  \[
  \forall \:  A \in  \B\left( \Rset^{d_X} \right),  \: y \in  \Y, \quad
  P\big( (X \in A) \cap (Y \le y) \big) = \int_A F_{Y|X=x}(y) \, dP_X(x).
  \]
  %
  Ad\'emas, si $X$ admite una densidad  de probabilidad $p_X$, $dP_X = p_X dx$ y
  tomando $A = \optimes_i (-\infty \; x_i]$ se obtiene
  %
  \[
  F_{X,Y}(x,y) = \int_{\optimes_i (-\infty \; x_i]} F_{Y|X=x}(y) \, p_X(x)
  \, dx
  \]
  %
  o, por diferenciaci\'on, para cualquier $y \in \Y$,
  %
  \[
  F_{Y|X=x}(y)         =        \frac{\displaystyle        \frac{\partial^{d_X}
      F_{X,Y}(x,y)}{\partial x_1 \ldots \partial x_{d_X}}}{p_X(x)}.
  \]
\end{definicion}
%
\modif{\noindent Nota: Tomando  $A = \X$, de la primera f\'ormula  que define la medida
de probabilidad condicional se recupera  el equivalente continuo de la f\'ormula
de probabilidad total:
%
\begin{lema}[F\'ormula de probabilidad total (caso general)]
\label{Lem:MP:ProbaTotalGeneral}
%
%  Sean \  $X$ \  e \ $Y$  \ vectores aleatorias  y $\X  = X(\Omega), \quad  \Y =
%  Y(\Omega)$. Entonces
  %
  \[
  \forall \  B \in  \B\left( \Rset^{d_Y}  \right), \qquad P(Y  \in B)  = \int_\X
  P_{Y|X=x}(B) \, dP_X(x)
  \]
  %
  lo que da en termino de funci\'on de repartici\'on condicional
  %
  \[
  F_Y(y) = \int_\X F_{Y|X=x}(y) \, dP_X(x)
  \]
  %
  Si $P_X$  admite una densidad, se  escribe todo notando que  $dP_X(x) = p_X(x)
  d\mu_L(x) \equiv p_X(x) dx$.
\end{lema}
% que se escribe con la densidad $dP_X = p_X d\mu_L$ si $X$
%admite una densidad.
}

Por \'ultimo, si $(X,Y)$ admite una densidad, en sencillo ver que $P_{Y|X=x} \ll
\mu_L$,  y entonces  \ $P_{Y|X=x}$  \ admite  una densidad  que  llamaremos {\it
  densidad  de  probabilidad  condicional}.  Sean $A  \in  \B\left(  \Rset^{d_X}
\right)$ y $B \in \B\left( \Rset^{d_Y} \right)$,
%
\begin{eqnarray*}
P\Big( (X \in A) \cap (Y \in B) \Big) & = & \int_{A \times B} p_{X,Y}(x,y) \, dx \, dy\\[2mm]
%
& = & \int_B \left( \int_A \frac{p_{X,Y}(x,y)}{p_X(x)} \, dy \right) p_X(x) \, dx
\end{eqnarray*}
%
Entonces, si $p_X(x) \ne 0$, tenemos
%
\[
P_{Y|X=x}(A) = \int_A \frac{p_{X,Y}(x,y)}{p_X(x)} \, dy.
\]
%
\begin{teorema}[Densidad de probabilidad condicional]
\label{Def:MP:DensidadCondicional}
%
  Si  $(X,Y)$ admite  una densidad  de probabilidad,  la medida  de probabilidad
  condicional  $P_{Y|X=x}$  admite  una   densidad,  llamada  {\it  densidad  de
    probabilidad condicional} definida por
  %
  \[
  \forall \: x \in \X, \quad p_{Y|X=x}(y) = \frac{p_{X,Y}(x,y)}{p_X(x)}
  \]
  %
  definida  sobre $\Y$. Claramente,  a partir de  la funci\'on  de repartici\'on
  condicional resulta que
  %
  \[
  p_{Y|X=x} = \frac{\partial^{d_Y}  F_{Y|X=x}}{\partial y_1 \ldots \partial y_{d_Y}}.
  \]
\end{teorema}

De hecho, esta  construcci\'on rigurosa coincide con la  intuici\'on que podemos
tener en este  caso continuo. Por ejemplo, podemos  pensar a $F_{Y|X=x}(y)$ como
caso l\'imite de $\displaystyle P\left( Y \le y \: \Big| \: x \le X \le x+\delta
  x \right) = \frac{P\big( \left( Y \le y  \right) \: \cap \: \left( x \le X \le
    x+\delta  x  \right)  \big)}{P\left( x  \le  X  \le  x+\delta x  \right)}  =
\frac{F_{X,Y}(x+\delta  x ,  y) -  F_{X,Y}(x  , y)}{F_X(x+\delta  x) -  F_X(x)}$
cuando \ $\delta  x$ \ tiende a 0.   En el caso escalar, se  calcula por ejemplo
haciendo un  desarrollo de Taylor del numerador  y del denominador a  orden 1, o
usando la regla de l'H\^opital~\footnote{De hecho, esta regla es debido al suizo
  J.  Bernoulli  que tuvo  un acuerdo financiero  con el  Guillaume Fran\c{c}ois
  Antoine, marqu\'es  de l'H\^opital, permitiendolo de  publicar unos resultados
  de Bernoulli bajo  su nombre.}  para re-obtener la  funci\'on de repartici\'on
condicional   de  la   definici\'on~\ref{Def:MP:FRCondicional}.    En  el   caso
multivariado,  hace  falta  hacer  los  desarollos hasta  el  orden  $d_X$  para
concluir.

Notar  que:
%
\begin{itemize}
\item si $X$ e $Y$ son independientes,
  %
  \[
  p_{Y|X=x} = p_Y;
  \]
%
\item por  la expresi\'on \ $p_{Y|X=x}(y) =  \frac{p_{X,Y}(x,y)}{p_X(x)}$, \ por
  integraci\'on con respecto a $y$ obtenemos la condici\'on de normalizaci\'on
  %
  \[
  \int_{\modif{\Y}} p_{Y|X=x}(y) \, dy = 1.
  \]
  %
  %\modif{con $\Y = Y(\Omega)$.}
%
%\item escribiendo  \ $p_{X,Y}(x,y)  = p_{Y|X=x}(y) \,  p_X(x) =  p_{X|Y=y}(x) \,
%  p_Y(y)$, \ se obtiene
%  %
%  \[
%  p_{Y|X=x}(y)   =  \frac{p_{X|Y=y}(x) \,   p_Y(y)}{p_X(x)}   =  \frac{p_{X|Y=y}(x)
%    \, p_Y(y)}{\displaystyle \int_{\Rset^{d_Y}} p_{X|Y=y}(x) \, p_Y(y) \, dy},
%  \]
%  %
%  equivalente continuo, con densidades, de la f\'ormula de Bayes;
%
%\item  por la  expresi\'on  \ $p_{X,Y}(x,y)  =  p_{Y|X=x}(y) \,  p_X(x)$, \  por
%  integraci\'on con respecto a $x$ obtenemos
%  %
%  \[
%  p_Y(y) = \int_{\Rset^{d_X}} p_{Y|X=x}(y) \, p_X(x) \, dx,
%  \]
%  %
%  generalizaci\'on de la f\'ormula de  probabilidades totales al caso continuo con
%  densidad de probabilidad.
\end{itemize}

\modif{Tambi\'en, se  escribe la f\'ormula  de probabilidad total a  trav\'es de
  las densidades por la expresi\'on  \ $p_{X,Y}(x,y) = p_{Y|X=x}(y) \, p_X(x)$ \
  y luego por integraci\'on con respecto a $x$:
%
\begin{lema}[F\'ormula d eprobabilidad total (caso con densidades)]
\label{Lem:MP:ProbaTotalContinuo}
%
  Si  \  $(X,Y)$ \  admite  una  densidad  de probabilidad  conjunta  $p_{X,Y}$,
  entonces \ $Y$ \ tiene una  densidad de probabilida que se recupera a trav\'es
  de la f\'ormula
  %
  \[
  p_Y(y) = \int_\X p_{Y|X=x}(y) \, p_X(x) \, dx.
  \]
\end{lema}

De  la expresi\'on  la  densidad condicional,  $p_{X,Y}(x,y)  = p_{Y|X=x}(y)  \,
p_X(x) =  p_{X|Y=y}(x) \, p_Y(y)$,  y de la  f\'ormula de probabilidad  total se
recupera sencillamente el equivalente continuo de la formula de Bay\'es:
%
\begin{lema}[F\'ormula de Bayes (caso continuo)]\label{Lem:MP:BayesContinuo}
%
%  Sean  \ $X$  \ e  \ $Y$  \  vectores aleatorias  que admiten  una densidad  de
%  probabilidad  conjunta   $p_{X,Y}$,  y   \  $\X  =   X(\Omega),  \quad   \Y  =
%  Y(\Omega)$. Entonces
  %
  \[
  \forall \, y \in \Y, \qquad p_{Y|X=x}(y)    =  \frac{p_{X|Y=y}(x)
    \, p_Y(y)}{\displaystyle \int_\Y p_{X|Y=y}(x) \, p_Y(y) \, dy}.
  \]
\end{lema}
}

Volvemos \modif{ahora} al ejemplo~\ref{Ej:MP:Suma}:
%, pagina~\pageref{Ej:MP:Suma}:
%
\begin{ejemplo}[Distribuci\'on condicional de la suma de vectores aleatorios]
\label{Ej:MP:SumaCond}
%
  Sea   \   $V  =   X   +   Y$,  \   con   \   $X$  \   e   \   $Y$  \   vectores
  $d$-dimensionales.  Introduciendo \  $U =  X$  \ obtuvimos  \ $p_{U,V}(u,v)  =
  p_{X,Y}(u,v-u)$  \ dando  tambi\'en \  $\displaystyle p_V(v)  = \int_{\Rset^d}
  p_{X,Y}(u,v-u) \, du$. Entonces, recordando que $U = X$, se obtiene
  %
  \[
  p_{V|X=x}(v)            =            \frac{p_{X,Y}(x,v-x)}{p_X(x)}           =
  \frac{p_{X,Y}(x,v-x)}{\displaystyle \int_{\Rset^d} p_{X,Y}(x,v-x) \, dv},
  \]
  %
  dando en el caso \ $X$ \ e \ $Y$ \ independientes
  %
  \[
  p_{V|X=x}(v) = p_Y(v-x).
  \]
  %
  Esto corresponde a la  intuci\'on de que, con $V = X + Y$,  fijando $X = x$ el
  vector aleatorio  $V$ es nada  m\'as que $Y$  desplazado en $x$. Pero  hay que
  tomar muchas precauciones con  este razonamiento, valido \'unicamente cuando \
  $X$  \ e  \  $Y$ \  son independientes.   En  caso contrario,  fijando $X$  no
  coincide con un desplazamiento por la dependencia (esquemat\'icamente, fijando
  $X$ no s\'olo mueve \ $Y$ \ sino que ``cambia'' su estad\'istica).
\end{ejemplo}


% =================================== Momentos ================================= %

%\seccion{Esperanza, momentos, identidades y desigualdades}%funciones generadoras}
\label{s:MP:esperanzamomento}


%%%%%%%%%%%%%%%%%%%%%%%%%%%%%%%%%%%%%%%%%%%%%%%%%%%%%%%%%%%%%%%%%%%%%%%%%%

\aver{\emph{introducci\'on...}} %%


% ================================= Media

\subseccion{\modif{Media de un vector aleatorio}}
\label{sec:MP:VectorMedio}


Una  variable aleatoria  $X$  tiene asociado  un  {\it promedio}  o {\it  media}
(tambi\'en  llamado   {\it  valor  esperado  o  de   expectaci\'on  o  esperanza
  matem\'atica})  que  se obtiene  pesando  cada  valor  de \modif{$X$}  con  la
\modif{medida de} probabilidad asociada a ese valor,
%
\modif{
\begin{definicion}[Media o valor/vector medio]
  Formalmente, la media de  una variable aleatoria $X$ \underline{integrable} es
  definida por
  %
  \[
  \Esp[X] = \int_\Omega X(\omega) dP(\omega)
  \]
  %
  Claramente,  por  el  teorema de  la  medida  imagen,  esta media  se  escribe
  tambi\'en a partir de la medida de probabilidad $P_X$ como
  %
  \[
  \Esp[X] = \int_\Rset x \, dP_X(x)
  \]
  %
  En  el caso vectorial  $d$-dimensional, hay  que entender  la media,  o vector
  medio, como  un vector de componentes  $i$-\'esima la media  $\Esp[X_i]$ de la
  componente $i$-\'esima $X_i$ de $X$, dando
  %
  \[
  \Esp[X] = \int_{\Rset^d} x \, dP_X(x)
  \]
  %
  A veces,  se encuentra  tambi\'en la notaci\'on  \ $\langle  x \rangle$ \  o \
  $\langle  x  \rangle_{p_X}$  \  para  el  valor  medio,  especialmente  en  la
  literatura de f\'isica.
\end{definicion}
%
La secunda formulaci\'on  del valor medio se proba  sencillamente, empezando por
$X = \un_A$ para unos $A$.   Entonces $P_X = (1-P(A)) \delta_0 + P(A) \delta_1$.
Luego  $\displaystyle \int_\Omega  \un_A(\omega)  dP(\omega) =  P(A) =  (1-P(A))
\times 0 +  P(A) \times 1 = \int_\Rset  x dP_X(x)$.  Se cierra la  prueba con el
teorema~\ref{Th:MP:MedibleLimite} dando cualquier  funci\'on medible como limite
de funciones escalonadas,  y por la definici\'on~\ref{Def:MP:IntegracionReal} de
la integral de cualquier funci\'on medible.

Luego,   de    la   distribuci\'on   marginal    $\displaystyle   P_{X_i}(B)   =
\int_{\Rset^{i-1}   \times   B   \times   \Rset^{d-i}}  dP_X(x)$,   se   obtiene
$\displaystyle  \Esp[X_i] = \int_{\Rset^d}  x_i \,  dP_X(x)$, dando  la \'ultima
formulaci\'on en el caso vectorial.


Fijense de que }
%$p(x)\,dx$, e integrando sobre el rango permitido de $x$:
%$$
%E[X] = \langle x\rangle = \int_{\Omega} x \ p(x)\,dx \equiv \mu
%$$
%si  la  integral  existe.  La  \emph{esperanza} de  la  variable  aleatoria  $X$
%representa  el valor  medio  que puede  tomar  entre todos  los  eventos de  una
%prueba. U
una   variable   aleatoria   $X$   se   dice   integrable   cuando   $E[|X|]   <
\infty$. \modif{De  la misma manera, un  vector aleatorio admite una  media si y
  solamente si cada componente es integrable. Veremos m\'as adelante que existen
  variables aleatorias que no admiten una media.

  M\'as  all\'a de  la  formulaci\'on matem\'atica  de  la media  \ $\Esp[X]$  \
  representa   la  posici\'on  alrededor   de  la   cual  se   ``distribuye  las
  probabilidades de  occurencia''. Es el equivalente  probabil\'istico de centro
  de gravedad o barycentro en meca\'anica.

  En el  caso de variables  aleatorias discretas, de  soporte $\X = \{  x_i \}$,
  inmediatamente
%
\[
\Esp[X] = \sum_i x_i P_X(\{x_i\}) = \sum_i x_i P(X = x_i)
\]
%
\noindent  Fijense de  que $\Esp[X]$  no partenece  necesariamente a  $\X$.  Por
ejemplo para $X$ uniforme sobre $\X = \{ 1 \, , \, 3 \, , \, 7 \}$, \ie $\forall
\, i \in \X, \quad P(X = i) = \frac13$. Se calcula $\Esp[X] = 1 \times \frac13 +
3 \times  \frac13 + 7 \times \frac13  = \frac{11}{3} \not\in \X$.  Tampoco es el
promedio de los valores extremos.

Un  ejemplo de  variable aleatoria  discreta  no integrable  es dada  por $\X  =
\Nset^*$   con   $P(X   =    n)   =   \frac{6}{\pi^2   \,   n^2}$.   Claramente,
$\sum_n\frac{6}{\pi^2 \, n}$ diverge, as\'i que $X$ no tiene una media.

En el caso de vectores  aleatorios continuos, obtenemos la expresi\'on siguiente
de la media (o vector medio):
%
\[
\Esp[X] = \int_{\Rset^d} x \, p_X(x) \, dx
\]
%
\noindent  Un ejemplo  de  no  teniende de  media  es dado  en  el  caso de  una
distribuci\'on de Cauchy-Lorentz (ver  m\'as adelante) \ $\displaystyle p_X(x) =
\frac{\alpha}{\left( 1 + x^t x  \right)^{\frac{d+1}{2}}}$ \ donde $\alpha$ es un
factor de normalizaci\'on.

En el caso general, para calcular  la media, hay que pasar por la distribuci\'on
$P_X$.   Por ejemplo\label{Ej:MP:EspMixta},  volvemos al  la variable  mixta del
ejemplo pagina~\pageref{Ej:MP:Mixta}, $X  = V \, \un_{U <  \frac12} + \un_{U \ge
  \frac12}$  \ con  \  $U$ \  y  \ $V$  variables  aleatorias independientes  de
distribuci\'on uniformas sobre  $[0 \, ; \,  1)$, \ie $p_U(x) = \un_{[0  \, ; \,
  1)}(x)$.   De \  $X \in  B \:  \Leftrightarrow \:  \left( \left(  U  < \frac12
  \right) \cap  \left( V \in B  \right) \right) \,  \cup \, \left( \left(  U \ge
    \frac12 \right) \cap \left( 1 \in B \right) \right)$, \ del hecho de que los
eventos de la  uni\'on son incompatibles y  de la independencia de $U$  y $V$ (o
saliendo de la funci\'on de  repartici\'on), se obtiene $P_X(B) = \frac12 P_V(B)
+ \frac12  \delta_1(x)$.  A continuaci\'on,  \ $\displaystyle \Esp[X]  = \frac12
\int_\Rset dP_V(x) + \frac12 \int_\Rset d\delta_1(x) = \frac12 \int_\Rset p_V(x)
\, dx + \frac12 \times 1 = \frac12 \int_0^1 dx + \frac12 = \frac34$.

\

Una nota intersante es  de que, en el caso escalar, si \  $X \ge 0$ \ admitiendo
una media, se obtiene
%
\[
\Esp[X]  = \int_{\Rset_+} P(X  > t)  \, dt  = \int_{\Rset_+}  \left( 1  - F_X(t)
\right) \, dt
\]
%
Se proba saliendo  de \ $\displaystyle x = \int_0^x  dt = \int_{\Rset_+} \un_{(t
  \, ;  \, +\infty)}(x)  \, dt$  \ dando \  $\displaystyle \Esp[X]  = \int_\Rset
\left( \int_{\Rset_+}  \un_{(t \, ; \,  +\infty)}(x) \, dt \right)  \, dP_X(x) =
\int_{\Rset_+} \left( \int_\Rset \un_{(t \, ; \, +\infty)}(x) \, dP_X(x) \right)
\,    dt$    \    por    el    teorema    de    Fubini    Th.~\ref{Th:MP:Fubini}
pagina~\pageref{Th:MP:Fubini}. Se  cierra la  prueba observando que  la integral
interior es nada  m\'as que $P(X > t)$.   En el caso discreto con  $\X = \Nset$,
viene  inmediatamente \ $\displaystyle  \sum_{n \in  \Nset} P(X>n)$  que podemos
probar directamente  saliendo de $P(X  = n) =  P(X>n)- P(X>n-1)$. En el  caso de
variable   admitiendo  una   densidad,   se  obtiene   tambi\'en  haciendo   una
integraci\'on  por   partes~\footnote{El  el   casi  discreto,  hay   que  tener
  precauciones separando la series de una diferencia de terminos. En el caso $X$
  continuo admitiendo una densidad, hay  que estudiar bien el coomportamiento de
  $t \mapsto t (1-F_X(t))$ al infinito.}.

\

Terminamos  esta  secci\'on con  la  propiedad  de  linealidad de  la  esperanza
matem\'atica $\Esp$,  como consecuencia de  la linealidad de la  integraci\'on y
definici\'on de la distribuci\'on  marginal: para cualquier conjunto de vectores
aleatorios \ $\{  X_i \}$ \ integrables  y cualquieras matrices \ $\{  C_i \}$ \
dadas  de  dimensiones compatibles  con  las  de \  $X$  \  (incluyendo el  caso
escalar),
%
\[
\Esp\left[ \sum_i C_i X_i \right] = \sum_i C_i \Esp\left[ X_i \right]
\]
%
(la integrabilidad de la suma viene de la desigualdad triangular).

}


% ================================= Momentos
\modif{
\subseccion{Momentos de un vector aleatorio}
\label{sec:MP:Momentos}
}

Si $X$ es una variable aleatoria, para cualquier funci\'on medible $f$, \ $f(X)$
tambi\'en  lo  es.  \modif{se  puede   entonces  definir  su  valor  medio},  si
existe. \modif{A pesar de necesitar evaluar la distribuci\'on de probabilidad de
  $Y = f(X)$, el valor medio se calcula a partir de la de $X$,
%
\begin{teorema}[Teorema de transferencia]
  Sea  \ $X$  \ un  vector  aleatorio $d$-dimensional  y \  $f: \Rset^d  \mapsto
  \Rset^{d'}$ \ una funci\'on medible tal que $f(X)$ sea integrable. Entonces
  %
  \[
  \Esp\left[   f(X)  \right]   =  \int_\Omega   f(X(\omega))  \,   dP(\omega)  =
  \int_{\Rset^{d'}} f(x) \, dP_X(x)
  \]
  %
  En particular, en el caso $\X = X(\Omega)$ discreto se obtiene
  %
  \[
  \Esp\left[ f(X) \right] = \sum_i f(x_i) P(X = x_i)
  \]
  %
  y para $X$ continuo admitiendo una densidad de probabilidad
  %
  \[
  \Esp\left[ f(X) \right] = \int_{\Rset^d} f(x) \, p_X(x) \, dx
  \]
\end{teorema}
%
\begin{proof}
  Sea $B \in \B(\Rset^d)$ \ y consideramos \ $f(x) = \un_B(x)$. Entonces, $\Y
  = \{  0 \, , \,  1 \}$ y inmediatamente  $$P_Y = P_X(B) \, \delta_1 + (1-P_X(B))
  \, \delta_0$$ Entonces
  %
  \[
  \Esp[f(X)]  =  \int_\Rset  P_X(B)  \,  d\delta_1 +  \int_\Rset  (1-P_X(B))  \,
  d\delta_0 = P_X(B) = \int_{\Rset^d} \un_B(x) \, dP_X(x)
  \]
  %
  En el caso  $d' = 1$, para $f  \ge 0$, se cierra entonces la  prueba usando el
  teorema~\ref{Th:MP:MedibleLimite}   pagina~\pageref{Th:MP:MedibleLimite}  para
  ver $f$ como l\'imite creciente  de una sucesi\'on de funciones escalonadas, y
  la definici\'on Def.~\ref{Def:MP:IntegracionReal} de la integraci\'on real. El
  caso $d' > 1$ es nada mas que $d' = 1$, componente a componente.
\end{proof}

Hablamos de {\it momentos} de la variable aleatoria $X$. Los momentos relevantes
%\underline{en el caso escalar} ($d = 1$) 
en general son los siguientes:}
%
\begin{itemize}
\item para el  \modif{``monomio'' $f(x) = x^{\otimes r}$  producto tensiorial de
    $x$ \  $r$ veces~\footnote{Recuerdense  de que $x  \otimes x$ es  una matriz
      teniendo como componentes $x_i x_j$; entonces $x^{\otimes r}$ en un tensor
      $r$-dimensionale   teniendo  como   componentes  $   \displaystyle  \left[
        x^{\otimes r} \right]_{i_1,\ldots,i_r} = \prod_j x_{i_j}$.}}  \ siendo \
  $r \in  \Nset^*$, se obtiene  \modif{el tensor de los}  $r${\it-\'esimo momentos
    (ordinarios)} de $X$:
  %
  \[
  \nu_r  \equiv   \Esp\left[  X^{\otimes  r}   \right]  =  \modif{\int_{\Rset^d}
    x^{\otimes r} \ dP_X(x)}
  \]
  %
  que tiene \modif{unidades de $\prod_j  X_{i_j}$ ($X_i^r$ si los componentes de
    $X$ tienen la misma ``unidad'').  Se escribe tambi\'en
  %
  \[
  \nu_{r_1,\ldots,r_d}  =  \Esp\left[   \prod_{i=1}^d  X_i^{r_i}  \right]  \quad
  \mbox{con} \quad \sum_i r_i = r
  \]
  %
}.  Se puede  incluir el caso $r=0$ \modif{con la  convenci\'on $x^{\otimes 0} =
  1$},   que    corresponde   a    la   condici\'on   de    normalizaci\'on:   \
\modif{$\displaystyle \nu_0 =  \int_\Rset dP_X(x) = 1$}.  La  media es el primer
momento: $\nu_1  = \Esp[X] =  \mu_X$.  T\'ipicamente, los primeros  momentos son
m\'as relevantes que los de  \'ordenes mayores, para la caracterizaci\'on de una
distribuci\'on. \modif{Para $r  = 2$, en el caso escalar, el  momento de orden 2
  es el an\'alogo del momento de inercia de la mec\'anica.}\newline Por ejemplo,
para  la  distribuci\'on  uniforme  $p_X(x)  = \frac{1}{b-a}$  en  el  intervalo
$[a,b]$,  resulta  $\nu_r=\frac{b^{r+1}-a^{r+1}}{(r+1)(b-a)}$.   En  particular,
$\nu_1 =  \frac{a+b}{2}$, valor  medio del intervalo.\newline  \modif{Fijense de
  que $X^{\otimes  r}$ no  es siempre  integrable, por ejemplo,  el el  caso con
  densidad,  si}  $p_X(x)$   tiene  soporte  (semi)infinito,  necesariamente  la
funci\'on $p_X$ debe tender a 0 cuando $\|x\|\rightarrow\infty$.  Si $p_X(x)$ es
\emph{de  largo alcance},  en  el sentido  de  que no  cae  a 0  suficientemente
r\'apido  con $x$ para  $x$ grandes,  algunos momentos  pueden no  existir.  Por
ejemplo, la  distribuci\'on de probabilidad  de Cauchy--Lorentz (o  funci\'on de
Breit--Wigner),  dada por  \modif{$p_X(x) =  \frac{\alpha}{\left( 1  + (x-x_0)^t
      R^{-1}  (x-x_0) \right)^{\frac{d+1}{2}}}$ sobre  $\Rset^d$, con  la matriz
  cuadrada \  $R > 0, \: x_0  \in \Rset^d$ \ y  \ $\alpha > 0$  \ coeficiente de
  normalizaci\'on}, no tiene momentos finitos de orden $r \geq 1$.
%
\item \aver{
En el  caso de variables discretas  sobre $\Nset$, resulta  \'util introducir el
$r$-\'esimo \emph{momento factorial} de $n$ mediante
$$
\langle n^{(r)} \rangle \equiv \langle n (n-1) \cdots [n-(r-1)] \rangle =
 \sum_{n=r}^\infty n (n-1) \cdots (n-r+1) \, p_n  .
$$}
%
\item  Los  {\it  momentos  centrales}  \modif{o {\it  cumulantes}}  se  definen
  alrededor de  \modif{la media  $\Esp[X]$, \ie,} como  el \modif{tensor  de los
    $r$-\'esimo momentos de la {\it desviaci\'on} \ $\Delta x \equiv x-\Esp[X]$}:
  %
  \[
  \mu_r \equiv \Esp\left[  \left( X - \Esp[X] \right)^{\otimes r} \right]
  \]
  %
  Se escribe tambi\'en
  %
  \[
  \mu_{r_1,\ldots,r_d}   =   \Esp\left[   \prod_{i=1}^d   \left(   X_i-\Esp[X_i]
    \right)^{r_i} \right] \quad \mbox{con} \quad \sum_i r_i = r
  \]
  %
  Se deduce que si la  \modif{distribuci\'on de probabilidad es sim\'etrica} con
  respecto  a la  media,  \modif{\ie \  $X-\mu_X  \egald -(X-\mu_X)$  \ donde  \
    $\egald$   \  significa  que   los  vectores   aleatorios  tiene   la  misma
    distribuci\'on  de  probabilidad},  entonces  todos los  momentos  centrales
  impares son nulos.  Los  momentos (centrales) brindan medidas que caracterizan
  la distribuci\'on:
  %
  \begin{enumerate}
  \item \modif{el primer momento, o media:
   \[
    \mu_X = \Esp[X];
   \]}
  %
\item  el  segundo  momento  central   se  conoce  como  \modif{{\it  matriz  de
      covarianza}.  En el  caso escalar,  hablamos  de {\it  varianza}, o}  {\it
    dispersi\'on} o tambi\'en {\it desviaci\'on cuadr\'atica media}.
  %
  \[
  \modif{\Sigma_X  \equiv \Cov[X]  \equiv \mu_2  = \Esp\left[  \left( X  - \mu_X
      \right) \left( X - \mu_X \right)^t \right]}
  \]
  %
  \modif{En el caso escalar, la varianza se escribe en general
  %
  \[
  \Var[X] = \Esp\left[ \left( X  - \mu_X \right)^2 \right]
  \]}  y es  una  medida del  cuadrado del  ancho  efectivo de  una densidad  de
probabilidad (o vector  de probabilidad). \modif{Para dos componentes  $i \ne j$
  hablamos de {\it covarianza entre variables}, y escribimos
  %
  \[
  \Cov[X_i,X_j]  =  \Esp\left[ \left(  X_i  -  \mu_{X_i}  \right) \left(  X_j  -
      \mu_{X_j} \right) \right]
  \]
  %
  La matriz de covarianza tiene las varianzas de los $X_i$ en su diagonal, y las
  covarianzas entre  componentes en las componentes no  doagonales.  Es sencillo
  ver de  que \ $\Cov[X]  \ge 0$ \  donde $A \ge 0$  significa que la  matriz es
  sim\'etrica,  definida  positiva  (en  el  caso  escalar  la  varianza  es  no
  negativa),  con igualdad} s\'olo  cuando \modif{$P_X  = \delta_{x_0}$  para un
  $x_0$ dado}, esto  es, cuando no hay incerteza  sobre el resultado.  \modif{De
  la  desigualdad  de  Cauchy-Bunyakovsky-Schwarz}  se  proba  sencillamente  de
que  $$\left|   \Cov[X_i,X_j]  \right|^2  \le   \sigma_{X_i}^2  \sigma_{X_j}^2$$
\modif{as\'i   que  s}e   define  \modif{tambi\'en}   el  {\it   coeficiente  de
  correlaci\'on}  que  es adimensional  y  toma  valores  entre $-1$  (variables
completamente anticorrelacionadas) y 1 (variables completamente correlacionadas)
como:
  %
  \[
  \rho_{ij} = \rho_{ji} \equiv \frac{\Cov[X_i,X_j]}{\sigma_{X_i} \sigma_{X_j}}
  \]
  %
  Como ejemplo, dadas $X_1$ y $X_2 = a X_1 + b$ que fluct\'uan en fase ($a>0$) o
  al rev\'es ($a<0$),  se tiene $\Delta X_2 = a \Delta  X_1$, luego $\rho_{12} =
  \frac{a}{|a|}   =  \pm   1$.\newline   \modif{Tambi\'en,  se   puede  ver   de
    que  $$\Var[\|  X  \|]  =   \Tr  \Sigma_X$$  $\Tr$  siendo  la  traza.}   La
  \modif{co}varianza  est\'a bien  definida si  \modif{$\|X\|$} es  una variable
  aleatoria  de  cuadrado  integrable,  esto  es,  cuando  \modif{$E[\|X\|^2]  <
    \infty$}.   \modif{Se proba  sencillamente (desallorando  el  ``cuadrado'' y
    usando la linealidad de la esperanza) de que
  %
    \[
    \Cov[X] = \Esp\left[ X X^t \right] - \mu_X \mu_X^t
    \]
  %
    conocido como {\it  teorema de K\"onig-Huygens}.  En el  caso escalar, es el
    equivalente del teorema de Huygens  de la mec\'anica relacionando el momento
    de inertia  de un solido con respeto  al origen en funci\'on  del momento de
    inertia con respeto al centro de masa. Adem\'as, inmediatamente,
  %
    \[
    \forall \: A \in \Rset^{d' \times d},  \: b \in \Rset^d, \quad \Cov[A X + b]
    = A \Cov[X] A^t
    \]
  %
    En el  caso escalar, $d  = 1$,  lo que es  conocido tambi\'en como}  el {\it
    ancho}  de   una  distribuci\'on  est\'a  dado  por   la  {\it  desviaci\'on
    est\'andar}
  %
  \[
  \sigma_X = \sqrt{\Var[X]}
  \]
  %
  tiene  las mismas  unidades de  $X$,  y se  usa para  normalizar los  momentos
  centrales  de orden  superior.  El  {\it ancho  relativo} es  otra  medida que
  caracteriza   la   distribuci\'on,    dado   por   $\frac{\sigma_X}{\mu_X}   =
  \sqrt{\frac{\Esp\left[    X^2   \right]}{\mu_X^2}-1}$   cuando    $\mu_X   \ne
  0$. \newline Dad\modif{o un vector  aleatorio} $X$, teniendo en cuenta que los
  dos  primeros  momentos dan  las  caracter\'isticas  m\'as  importantes de  la
  \modif{distrubuci\'on de  probabilidad}, puede resultar  conveniente hacer una
  transformaci\'on  de  variable  aleatoria  a la  llamada  \modif{\it  variable
    est\'andar}: \modif{$Y \equiv \Sigma_X^{-\frac12} \left( X - \mu_X \right)$},
  \modif{donde  $\Sigma^{-\frac12}$ es  la \'unica  matriz  sim\'etrica definida
    positiva  tal  que  su  cuadrado es  igual  a  $\Sigma^{-1}$~\cite{HorJoh85,
      MagNeu99}}  que entonces  tiene media  igual a  0 y  \modif{una  matriz de
    covarianza  igual  al  identidad  $I$  (en el  caso  escalar,}  desviaci\'on
  est\'andar igual a 1\modif{)}.
  %
\item \modif{en el  caso escalar,} el tercer momento  central permite definir el
  {\it coeficiente de asimetr\'ia} \modif{(o skewness en ingles)}:
  \[
  \modif{\gamma_X} \modif{\equiv  \Esp\left[ \left( \frac{X - \mu_X}{\sigma_X}
      \right)^3 \right]} = \frac{\mu_3}{\sigma_X^3},
  \]
  %
  \modif{momento de orden 3 de la variable estandar,} que resulta adimensional y
  puede tener  signo positivo o  negativo, anul\'andose para  distribuciones que
  son sim\'etricas respecto del valor medio;
  %
\item  \modif{en el  caso escalar,}  el  cuarto momento  central da  lugar a  la
  \emph{curtosis}:
  \[
  \modif{\Curt[X]  \equiv \kappa_X}  \equiv \modif{\Esp\left[  \left(  \frac{X -
          \mu_X}{\sigma_X} \right)^4 \right]} = \frac{\mu_4}{\sigma_X^4},
  \]
  % 
  \modif{momento de orden 4 de la variable estandar,} que posibilita diferenciar
  entre distribuciones  altas y angostas.  \modif{Veremos m\'as  adelante de que
    para la densidad Gausiana $p_X(x) = \frac{1}{\sqrt{2 \pi} \sigma} \exp\left(
      - \frac{(x-\mu)^2}{2 \, \sigma^2} \right)$,  \ $\mu_X = \mu, \: \sigma_X =
    \sigma, \: \gamma_X  = 0, \: \kappa_X =  3$. Se dice de que $p_X$  es alta y
    angosta,  o   sub-gausiana,  o  con   colas  livianas  cuando   o  tambi\'en
    platic\'urtica,}  $\kappa_X  <  3$,  y  de otras  bajas  y  anchas  \modif{o
    sobre-gausiana,  o con  colas  pesadas cuando  o tambi\'en  leptoc\'urtica,}
  $\kappa_X > 3$ \modif{(para $\kappa_X =  3$ la distribuci\'on es a veces dicha
    mesoc\'urtica)}.  A veces,  se define entonces la {\it  curtosis por exceso}
  $\Curt[X] - 3$.
  \end{enumerate}
  %
  \modif{Fijense  de que,  en  el contexto  escalar  $d =  1$,  se vinculan  los
    cumulantes y} los momentos ordinarios directamente de las definiciones:
  %
  \[
  \modif{\mu_r = \sum_{s=0}^r \binom{r}{s} \left( - \mu_X \right)^{r-s} \nu_s}
  \]
  %
  para cualquier \modif{$r \in \Nset$},  siendo $\modif{\mu_0 = } \nu_0=1$.  Por
  ejemplo, \ $\mu_2=\nu_2-\nu_1^2$ \ \modif{que  es nada m\'as que la relaci\'on
    de K\"onig-Huyggens}, mientras que \ $\mu_3=\nu_3-3\nu_1\nu_2+2\nu_1^3$.
\end{itemize}

\modif{Tratando de covarianza, m\'as  generalmente, para dos vectores aleatorios
  \ $X$ \ e \ $Y$, se define la matriz de covarianza conjunta como
%
\[
\Sigma{X,Y} \equiv  \Cov[X,Y] =  \Esp\left[ \left( X  - \mu_X\right) \left(  Y -
    \mu_Y \right)^t \right] = \Esp\left[ X Y^t \right] - \mu_X \mu_Y^t
\]
%
Esta matriz contiene las covarianzas $\Cov[X_i,Y_j]$.
}

% ================================= Momentos
\modif{
\subseccion{Independencia, identidades y desigualdades}
\label{sec:MP:MomentosDesigualdades}

Una primera relaci\'on interesante concierna el caso de variables independientes
y como se comporta la covarianza de estas:}
%
\begin{propuesta}
  Sean  \  $X$  \ e  \  $Y$  \  dos  vectores  aleatorios integrables.   Si  son
  independientes, entonces
  %
  \[
  \Esp[X Y^t] = \Esp[X] \Esp[Y]^t \qquad \mbox{\ie} \qquad \Cov[X,Y] = 0
  \]
  %
  En  particular, para  $X$  con componentes  independientes,  $\Cov[X]$ es  una
  matriz diagonal.
\end{propuesta}
\modif{
\begin{proof}
  Sean \ $X = \sum_j \alpha_j \un_{A_j}$  \ e \ $Y = \sum_k \beta_k \un_{B_k}$ \
  dos variables escalonadas. Entonces, \ $A_j =  (X = \alpha_j)$ \ y \ $B_k = (Y
  = \beta_k)$. Luego
  %
  \begin{eqnarray*}
  \Esp[X Y] & = & \sum_{j,k} \alpha_j \beta_k \Esp[\un_{A_j} \un_{B_k}]\\[2.5mm]
  %
  & = & \sum_{j,k} \alpha_j \beta_k \Esp[\un_{A_j \cap B_k}]\\[2.5mm]
  %
  & = & \sum_{j,k} \alpha_j \beta_k P(A_j \cap B_k)\\[2.5mm]
  %
  & = & \sum_{j,k} \alpha_j \beta_k P(X = \alpha_j) P(Y = \beta_k) \quad \mbox{(de la independencia)}
  \end{eqnarray*}
  %
  dando  el resultado  para  variables  escalonadas. Se  cierra  la prueba  para
  variables positivas  como l\'imite de  crecientes de funciones  escalonadas, y
  variables  reals tratando  las partes  positivas y  negativas aparte.  El caso
  vectorial se deduce trabajando con pares de componentes.
\end{proof}
%
Fijense  de que  la  reciproca es  falsa en  general.   Por ejemplo,  para $X  =
(X_1,X_2)$  uniforme sobre  el  disco unitario,  i.e.,  $p_X(x) =  \frac{1}{\pi}
\un_{\Sset^2}(x)$ \ con \ $\Sset^2 =  \{ (x_1,x_2) \in \Rset^2: \: x_1^2 + x_2^2
\le 1 \}$. Claramente, los \ $X_i$ no pueden ser independientes del hecho de que
\ $\X_i =  [-1 \, , \, 1]$ \  y \ $\X \ne \X_1 \times  \X_2$ \ (es estrictamente
incluido  en  el producto  cartesiano).  Por  simetr\'ia  central de  $p_X$,  es
sencillo ver de  que $\Esp[X_1 X_2] = 0$  \ y similarmente \ $\Esp[X_i]  = 0$: a
pesar de que los \ $X_i$ \ no sean independientes, $\Cov[X_1,X_2] = 0$.

Esta implicaci\'on facilita frecuentemente  los calculos de media.  Volviendo al
ejemplo de la pagina~\pageref{Ej:MP:Mixta}, tratando de  la media de \ $X = V \,
\un_{U <  \frac12} +  \un_{U \ge \frac12}$  \ con \  $U$ \  y \ $V$  \ variables
independientes de  distribuci\'on uniformas  sobre $(0 \,  ; \, 1)$,  se calcula
gracia a la linealidad y a  la independencia, \ $\Esp[X] = \Esp[V] \Esp[\un_{U <
  \frac12}] +  \Esp[\un_{U \ge  \frac12}] = \frac12  \times \frac12 +  \frac12 =
\frac34$    \    como    lo     hemos    obtenido    usando    $P_X$    en    la
pagina~\pageref{Ej:MP:EspMixta}.

Una otra  consecuencia de  esta proposici\'on trata  de un conjunto  de vectores
aleatorios \ $\{ X_i \}$ \ y un conjunto de matrices de dimensiones adecuadas,
%
\[
\Cov\left[ \sum_i A_i X_i + B\right] =  \sum_i A_i \Sigma_{X_i} A_i^t + \sum_{j \ne
  i} A_i \Cov[X_i,X_j] A_j^t
\]
%
En particular, en el caso escalar,
%
\[
\Cov\left[ \sum_i A_i X_i + B \right]  = \sum_i A_i^2 \Var[X_i] + \sum_{j \ne i}
A_i A_j \Cov[X_i,X_j]
\]
%
\underline{Si}   los   $X_i$   son  independientes,   \underline{entonces}   las
covarianzas conjuntas son nulas as\'i que, respectivamente,
\[
\Cov\left[ \sum_i A_i X_i + B \right] = \sum_i A_i \Sigma_{X_i} A_i^t
\]
%
y
%
\[
\Cov\left[ \sum_i A_i X_i + B \right] = \sum_i A_i^2 \sigma_{X_i}^2
\]
%
\noindent en el caso escalar.

Si el  teorema da  una implocaci\'on  de la independencia,  de hecho  existe una
reciproca que toma la forma siguiente: }

\begin{teorema}
  Sean \ $X$ \ e \ $Y$ \ dos vectores aleatorios. Son independientes si y s\'olo
  si $E[f(X) g(Y)]=E[f(X)] E[g(Y)]$ para todo par  de funciones \ $f$ \ y \ $g$,
  medibles y acotadas \modif{de dimensiones adecuadas}.
\end{teorema}
%
\modif{
\begin{proof}
  Se puede referirse a~\cite{Fel71, JacPro03} para unas pruebas rigurosa.  En el
  caso escalar, el  principio consiste a ver \  $f$ \ y \ $g$ \  como limites de
  funciones  escalonadas. Para \  $f(x) =  \sum_i \alpha_i  \un_{A_i}(x)$ \  y \
  $g(y) = \sum_j \beta_j \un_{B_j}(y)$ se obtiene $E[f(X) g(Y)]=E[f(X)] E[g(Y)]$
  si y s\'olo si $\sum_{i,j} \alpha_i \beta_j  \left( P( (X \in A_i) \cap (Y \in
    B_j)) -  P(X \in  A_i) P(Y \in  B_j) \right)  = 0$. B\'asicamente,  eso debe
  valer  para  cualquieras  $A_i,  B_j$  y $\alpha_i,  \beta_j$,  as\'i  que  el
  t\'ermino  entre  parentesis  debe ser  cero,  lo  que  es  nada m\'as  de  la
  definici\'on de  la independencia  de \ $X$  \ e\  $Y$.  El caso  vectorial se
  entiende por pares de componentes.
\end{proof}
}

\

\modif{Relaciones tamb\'ien  muy \'utiles son conocidas  como {\it Desigualdades
    de Chebyshev}.}  Estas desigualdades dan una cota superior a la probabilidad
de que  una cantidad  que fluct\'ua aleatoriamente  exceda cierto  valor umbral,
a\'un sin conocer detalladamente la  forma de la distribuci\'on de probabilidad.
\modif{
%
\begin{teorema}[Desigualdades de Chebyshev]
  Sea un  vector aleatorio $d$-dimensional \ $X$  \ y a funci\'on  \ $g: \Rset^d
  \mapsto \Rset_+$ \ medible tal que \ $g(X)$ \ sea integrable. Entonces,
  %
  \[
  \forall \: a > 0, \quad P(g(X) \ge a) \: \le \: \frac{\Esp[g(X)]}{a}
  \]
  %
\end{teorema}
%
\begin{proof}
  Sea \ $\D_a = \{ x \in \X: \: g(x) \ge a \} \subset \X$. Entonces, $g$ siendo non negativa,
  %
  \[
  \Esp[g(X)]  = \int_\X  g(x) \,  dP_X(x) \ge  \int_{\D_a} g(x)  \,  dP_X(x) \ge
  \int_{\D_a} a \, dP_X(x) = a P(X \in \D_a)
  \]
  %
  Se cierra la prueba notando de que \ $(X \in \D_a) \, = \, (g(X) \ge a)$.
\end{proof}
%
\SZ{Esta  versi\'on  es debido  a  Bienaym\'e~\cite{toto}\ldots} Existen  varios
corolarios, que son de hecho casos particulares de estas desigualdades.
%
\begin{corolario}[Bienaym\'e--Chebyshev]
  Sea \  $X$ \ un vector  aleatorio admitiendo una  esperanza \ $\mu_X$ \  y una
  covarianza \ $\Sigma_X^2$. Entonces,
  %
  \[
  \forall \: \varepsilon > 0, \qquad P(\|  X - \mu_X \|) > \varepsilon \: \le \:
  \frac{Tr \Sigma_X}{\varepsilon^2} .
 \]
\end{corolario}
%
\SZ{Esta versi\'on,  debido a Bienaym\'e~\cite{toto}}. Viene  del teorema incial
aplicado a \ $X - \mu_X$, \ $g(x)  = \|x\|^2$ \ y \ $a = \varepsilon^2$, notando
de que \ $(\|X\|^2 \ge \varepsilon^2) \, = \, (\|X\| \ge \varepsilon)$.
%
%
\begin{corolario}[Markov]
  Sea \ $X$ \ un vector aleatorio y $\varphi \ge 0$ una funci\'on no decreciente
  tal que $\varphi(\|X\|)$ sea integrable. Entonces,
  %
  \[
  \forall \: \varepsilon \ge  0, \quad \mbox{tal que} \quad \varphi(\varepsilon)
  \ne     0,     \qquad     P(\|X\|     >     \varepsilon)     \:     \le     \:
  \frac{\Esp[\varphi(\|X\|)]}{\varphi(\varepsilon)} .
 \]
\end{corolario}
%
%
\SZ{Esta versi\'on, debido a Markov en su tesis~\cite{toto} trataba de funciones
  $\varphi(u) = u^r, \: r > 0$}. Viene del teorema incial aplicado a \ \ $g(x) =
\varphi(\|x\|)$  \   y  \   $a  =  \varphi(\varepsilon)$,   notando  de   que  \
$(\varphi(\|X\|) \ge \varphi(\varepsilon)) \,  = \, (\|X\| \ge \varepsilon)$ por
la no decrecencia de $\varphi$. El  caso anterior (una vez la variable centrada)
es nada m\'as que un caso especial.}

Estas  relaciones  afirman que  cuanto  m\'as chica  es  la  varianza, m\'as  se
concentra la variable en torno a su media. Ambas cotas son en general d\'ebiles;
por  ejemplo,  la  desigualdad de Bienaym\'e--Chebyshev indica  que  la
probabilidad  de encontrar  una fluctuaci\'on  superior a  $\eta = 3$ desviaciones
est\'andar alrededor de la media, est\'a  por debajo de $1/9$; el c\'alculo para
una  distribuci\'on t\'ipica  como la  Gaussiana ajusta  dicha  probabilidad por
debajo de $0.003$.


\SZ{Terminamos esta secci\'on con una desigualdad que usaremos frecuentemente
   en el c\'apitulo siguiente, tratando de funciones convexa.}
% %
% \begin{definicion}[Funci\'on convexa]
%   Una funci\'on  \ $f:  \Rset^d \mapsto  \Rset$ \ es  convexa si  para cualquier
%   $\alpha \in [0 \, , \, 1]$ y $x, Y \in \Rset^d$,
%   %
%   \[
%   f(\alpha x + (1-\alpha) y) \le \alpha f(x) + (1-\alpha) f(y)
%   \]
%   %
%   Se puede ver de que si \ $f$ \ es dos veces diferenciable, su matriz Hessiana,
%   $\Hess  f \ge  0$  donde las  componentes  de la  Hessiana  son las  derivadas
%   partiales secundas de $f$, $\frac{\partial^2 f}{\partial x_i \partial x_j}$.
% \end{definicion}
% %


% ========================== Esperanza condicionales =========================== %

%\seccion{Esperanza condicional}
\label{Sec:MP:EsperanzaCondicional}

Vimos en  la secci\'on~\ref{Sec:MP:LeyesCondicionales} que  una pregunta natural
era de, dados  dos vectores aleatorios \  $X$ \ e \ $Y$,  caracterizar el vector
$Y$ si ``observamos $X$''.  M\'as adelante,  nos podemos interesar a la media de
$Y$ cuando observamos  $X$. Una manera intuitiva es de  definir tal media cuando
``sabemos'' que  $X=x$ a partir  de la ley  condicional $P_{Y|X=x}$~\cite{Fel68,
  Fel71, AthLah06, Spi76, Kol56, JacPro03}:

\begin{definicion}[Esperanza condicional]
  Sean   $X$   e  $Y$   dos   vectores   aleatorios   respectivamente  $d_X$   y
  $d_Y$-dimensionales, y sea la funci\'on
  %
  \[
  f(x) \equiv \Esp[Y | X=x ] = \int_{\Rset^{d_Y}} y \, dP_{Y|X=x}(y)
  \]
  %
  Se define la esperanza condicional de $Y$ condicionalment a $X$ como siendo la
  variable aleatoria
  %
  \[
  \Esp[Y|X] = f(X)
  \]
\end{definicion}

Como  en  el   caso  de  medida  de  probabilidad,   cuando  dos  variables  son
independientes, condicionar no cambia la esperanza:
%
\begin{lema}
Cuando \ $X$ \ e \ $Y$ \ son independientes, la esperanza condicional coincide con la de \ $Y$,
%
\[
X \:\: \mbox{e} \:\: Y \:\: \mbox{independientes} \quad \Rightarrow \quad \Esp[Y|X] = \Esp[Y]
\]
\end{lema}
\begin{proof}
\SZ{Hacerla}
%Considera el caso $Y = \un_B$ \ con \ $B \in \Y$
\end{proof}

La media condicional se revela muy  \'util y poderoso para evaluar esperanzas de
variables  aleatorias por ejemplo  gracia a  la formula  de la  esperanza total,
equivalente      de      las       formulas      de      probabilidad      total
lema~\ref{Lem:MP:ProbaTotalDiscreto},    lema~\ref{Lem:MP:ProbaTotalGeneral}   y
lema~\ref{Lem:MP:ProbaTotalContinuo}.
%
\begin{teorema}[Media total]\label{Teo:MP:EsperanzaTotal}
%
  La media  (total) del  vector aleatorio  \ $Y$ \  concide con  la media  de la
  esperanza condicional, \ie
%
\[
\Esp[Y] = \Esp\left[ \Esp\left[ Y | X\right] \right]
\]
\end{teorema}
%
\begin{proof}
\SZ{Hacerla}
%Considera el caso $Y = \un_B$ \ con \ $B \in \Y$
\end{proof}

Un  otro   resultado  importante,  permitiendo   frecuentemente  simplificar  la
evaluac\'ion de momentos a partir de esperanza condicional es el siguiente:
%
\begin{teorema}\label{Teo:MP:EsperanzaF(X)Y}
%
  Para cualquier funci\'on medible $f$, tenemos
  %
  \[
  \Esp\left[ \left. f(X) Y \, \right| \, X \right] = f(X) \Esp\left[ Y | X\right]
  \]
\end{teorema}
%
\begin{proof}
\SZ{Hacerla}
%Considera el caso $Y = \un_B$ \ con \ $B \in \Y$
\end{proof}

\SZ{
\begin{teorema}
La esperanza condicional \ $E[Y|X]$ \ es la \'unica variable $Z$ tal que para cualquier variable $U$ acotada $\Esp[Y U] = \Esp[Z U]$
\end{teorema}
\begin{proof}
%Considera el caso $Y = \un_B$ \ con \ $B \in \Y$
\end{proof}
%
A veces, este resultado sirve como defnici\'on de la esperanza condicional.}

Un  otro resultado  que  sirve a  veces  como definici\'on,  en  el contexto  de
variable  de  cuadrado integrable,  se  voncula con  la  idea  de aproximar  una
variable por una funcci\'on de una otra:
%
\begin{teorema}
  Sea \ $Y$  \ de cuadrado integrable, la esperanza condicional  \ $E[Y|X]$ \ es
  la \'unica  variable \ $Z  = f(X)$, funci\'on  de \ $X$, minimizando  el error
  promedio cuadratico \ $\Esp[  \| Y - Z \|^2 ]$.  Dicho  de otra manera, con el
  criterio de error  cuadratico promedio m\'inimo, \ $E[Y|X]$  \ es la ``mejor''
  funci\'on de $X$ aproximando $Y$.
\end{teorema}
\begin{proof}
Usando la f\'ormula de esperanza total, y el teorema~\ref{}, se escribe
%
\begin{eqnarray*}
\Esp\left[ \| Y - f(X) \|^2 \right] & = & \Esp\left[ \Esp\left[ \left. \| Y -
f(X) \|^2 \, \right| \, X \right] \right]\\
%
& = & \Esp\left[ f(X)^2 - 2 f(X) \Esp[Y|X] +  \Esp\left[ Y^2 | X \right] \right]
\end{eqnarray*}
%
Ahora, buscando $\lambda \equiv f(x)$  minimizando $\| \lambda\|^2 - 2 \lambda^t
\Esp[Y|X=x] + \Esp\left[ \| Y \|^2 |  X=x \right]$ para cualquier $x \in \X$, se
minimizar\'a  el  promedio en  $X$.   Inmediatamente,  notando  que buscamos  el
m\'inimo de  un paraboliodo de concavivad  por arriba, anulando  el gradiente en
$\lambda$ so obtiene  $\lambda \equiv f(x) = \Esp[Y|X=x]$,  el \'unico m\'inimo,
lo que cierra la prueba.
\end{proof}
%
Este resultado  es muy conocido en el  mundo de la estimaci\'on  donde se quiere
aproximar una variable minimizar el error cuadratico promedio~\cite{Kay93, Rob07}.
%, AthLah06, JacPro03}.  


\SZ{Cerrar esta secci\'on}

% ================================== Generadoras ================================ %

\seccion{Funciones generatrices}
\label{s:MP:generatrices}


%\subseccion{Funciones generatrices}
%%%%%%%%%%%%%%%%%%%%%%%%%%%%%%%%%%%%%%%%%%%%%%%%%%%%%%%%%%%%%%%%%%%%%%%%%%

\aver{
Se  definen  un conjunto  de  funciones  que  permiten hallar  f\'acilmente  los
distintos   momentos  de   una   distribuci\'on  de   probabilidad.  Se   llaman
\emph{funciones  generadoras} o \emph{funciones  generatrices}, y  est\'an dadas
como valores de expectaci\'on de  funciones de la variable aleatoria (discreta o
continua), con un par\'ametro real o complejo.

La  \emph{funci\'on  generadora   de  momentos}  (MGF,  \emph{moment  generating
  function}) se define como
$$
M(\xi) \equiv \langle e^{\xi X} \rangle =  \int e^{\xi x} p(x) \, dx , \quad \xi
\in \Rset
$$
en el  caso de una variable  aleatoria continua $X$  con pdf $p(x)$. Se  tiene \
$M(0)=\int p(x)\,dx=1$ (que corresponde a la condici\'on de normalizaci\'on). Si
la  variable $X$ es  positiva y  se toma  $\xi=-s$ con  $s>0$, se  interpreta en
t\'erminos de la transformada de Laplace de la funci\'on $p$.  %%
\\
Si existe, la  MGF posibilita obtener f\'acilmente los  momentos (ordinarios) de
$X$ a  distintos \'ordenes, mediante los  coeficientes del desarrollo  de $M$ en
serie de potencias de $\xi$:
$$
M(\xi)  =  \sum_{r=0}^{\infty}  \frac{\xi^r}{r!}  \int  x^r  p(x)  \,  dx  =  1+
\sum_{r=1}^{\infty} \frac{\nu_r}{r!} \xi^r
$$
o, alternativamente, mediante  las sucesivas derivadas de $M$  respecto de $\xi$
en 0:
$$
\nu_r=\left.  \frac{d^r  M(\xi)}{d\xi^r}\right|_{\xi=0}  ,  \quad  r=1,2,\ldots;
\quad \nu_0\equiv 1 .
$$ 

En  el  caso de  una  variable aleatoria  discreta,  suponiendo  que el  espacio
muestral es $\Nset$, se definen  dos funciones: la \emph{funci\'on generadora de
  momentos (ordinarios)} (MGF) dada por
$$
M(\xi) \equiv \langle e^{\xi N} \rangle = \sum_{n\geq 0} e^{\xi n} p_n ,
%% = \sum_{r\equiv 0} \frac{\langle n^r\rangle}{r!} \xi^r , 
$$
y la \emph{funci\'on generadora  de momentos factoriales} (FMGF, \emph{factorial
  moment generating function}) como
$$
F(\xi) \equiv \langle (1+\xi)^N \rangle = \sum_{n\geq 0} (1+\xi)^n p_n
$$
para    $\xi     \in    \Rset$    en     ambos    casos.    Se     verifica    \
$M(0)=F(0)=\sum_{n=0}^{\infty} p_n=1$. Se muestra simplemente que
$$
M(\xi) = \sum_{r=0}^{\infty} \frac{\langle n^r\rangle}{r!} \xi^r , 
$$
lo que  permite obtener los momentos  de la distribuci\'on  para cualquier orden
$r\geq 1$. Por otro lado, el desarrollo de la FMGF da
$$
F(\xi)  =   \sum_{n=  0}^{\infty}  \sum_{r=   0}^n  \binom{n}{r}  \xi^r   p_n  =
\sum_{r=0}^\infty \sum_{n=r}^{\infty} \frac{n(n-1)\cdots (n-r+1)}{r!}  \xi^r p_n
= \sum_{r=0}^{\infty} \frac{\langle n^{(r)}\rangle}{r!} \xi^r
$$
teniendo  en cuenta  en las  dobles sumas  que $0\leq  r\leq n$,  con  $n$ hasta
$n_{\max}$ \'o  $\infty$. Se  ve entonces que  $F$ permite obtener  los momentos
factoriales de orden $r$ arbitrario.

Dada   una   variable   aleatoria   a   valores  naturales,   la   funci\'on   \
$G(\xi)=\sum_{n=0}^{\infty} p_n \xi^n$, con $-1\leq \xi\leq 1$, %% \leq o < ?
es  tambi\'en una  funci\'on generatriz.  Por ejemplo,  si $G$  admite derivadas
primera  y segunda en  $\xi=1$ se  obtienen: $\langle  N\rangle=G'(1)$, $\langle
N(N-1)\rangle=G''(1)$, $\Var(N)=G''(1)+G'(1)-[G'(1)]^2$; adem\'as, se obtiene la
ley   de   distribuci\'on   evaluando   derivadas   de   $G$   en   $\xi=0$:   \
$p_n=\frac{G^{(n)}(0)}{n!}$.  %% Ej: probar
\cite{Fra09}%%p.73

\hfill

La \emph{funci\'on caracter\'istica}  (CF, \emph{characteristic function}) tiene
argumento complejo: \cite{Luk61}
$$
C_X(\xi) \equiv \langle e^{i \xi X} \rangle = \int e^{i \xi x} p(x) \, dx .
$$
La importancia  de esta  funci\'on reside  en que siempre  existe y  est\'a bien
definida, dado que es la  transformada de Fourier de una funci\'on absolutamente
integrable (i.e. $\int |f(x)| \, dx < \infty$) \cite{Gol61}

Si la pdf \ $p(x)$ es de cuadrado integrable, entonces 
$$
p(x) = \frac{1}{2	pi} \int e^{-i \xi x} C_X(\xi) \, d\xi .
$$
El requisito  para esta importante relaci\'on es  que \ $\int_{-\infty}^{\infty}
|p(x)|^2 \, dx<\infty$;  sin embargo, a\'un es v\'alida  para distribuciones con
una contribuci\'on  tipo $\delta$.  Por otro  lado los momentos,  si existen, se
obtienen derivando la funci\'on $C$ tal como expresa la siguiente proposici\'on:

\textbf{Proposici\'on:} \ %%
La  variable aleatoria  $X$  admite  momento de  orden  $r$ si  y  s\'olo si  la
funci\'on caracter\'istica $C$ es $r$ veces derivable en $\xi=0$, siendo
$$
\langle X^r\rangle = (-i)^r C_X^{(r)}(0) . 
$$

Por ejemplo, en el caso de la distribuci\'on de Cauchy--Lorentz resulta
$$
C(\xi)     =     \frac{\gamma}{\pi}    \int_{-\infty}^{\infty}     \frac{e^{i\xi
    x}}{\gamma^2+(x-x_0)^2} dx = e^{-\gamma |\xi| e^{i x_0\xi}}
$$
tomando $\gamma >0$. Esta funci\'on est\'a  definida para todo $\xi$, pero no es
derivable en $\xi=0$,  lo que coincide con el hecho de  que no est\'an definidos
los momentos para esta pdf.

Para  una   variable  aleatoria  compleja   $Z=X+iY$,  usando  la   noci\'on  de
transformada de Fourier bidimensional, se define:
$$
C_Z(\mu) \equiv \int e^{\mu^* z-\mu z^*} p(z) \, d^2z .
$$

Resumimos algunas propiedades importantes de la funci\'on caracter\'istica:
\begin{enumerate}
\item $C(0) =1$
%
\item $|C(\xi)|\leq C(0)$ %%dem.
%
\item $C(\xi)$  es una  funci\'on continua  en $\Rset$ (a\'un  si la  pdf $p(x)$
  tiene discontinuidades) %dem.
%
\item $C(-\xi) = C(\xi)*$
%
\item  $C(\xi)$ es  definida no  negativa,  de tal  forma que  para un  conjunto
  arbitrario  de  $N$  n\'umeros  reales $\xi_1,\ldots,\xi_N$  y  $N$  n\'umeros
  complejos $a_1,\ldots,a_N$, se cumple
  $$
  \sum_{i,j=1}^N a_i^* a_j C(\xi_j-\xi_i) \geq 0 .
  $$
%
\item  $C(\xi) =  M(i\xi) =  F(e^{i\xi}-1)$, si  $M$ y  $F$ existen;  \ $F(\xi)=
  M(\ln(1+\xi))$
\end{enumerate}

{\teorema (Bochner, Goldberg).... } %%

\textbf{Proposici\'on:} \ %%
Sean $X$ e  $Y$ dos variables aleatorias reales  independientes, cuyas funciones
caracter\'isticas son $C_X$ y $C_Y$. Entonces \ $C_{X+Y}=C_X C_Y$.

\hfill

Cumulant generating function .... %%

\hfill

Extendemos  la  definici\'on  de   funci\'on  caracter\'istica  para  un  vector
aleatorio. ... %%


....
}


\SZ{Hablar de esperanza condicional}

% ========================== Complejo - matriz variado ========================== %

%\SZ{En todos los lados, escribir (y definir antes, y en las notaciones)

\begin{itemize}
\item $M_{d,d'}(\Kset)$  el espacio  de matricez $d  \times d'$ de  elementos de
  $\Kset$ con $\Kset = \Rset$ o $\Cset$.
%
\item las notaciones $\cdot^*$ para la conjugaci\'on compleja, $\cdot^t$ para la
  transpuesta, $\cdot^\dag$ para la transconjugada.
%
\item  $S_d(\Kset)$  conjunto  de  matrices  de  $\Kset$  sim\'etricas,  $M  \in
  S_d(\Kset) \, \Leftrightarrow \, M^t = M$.
%
\item  $H_d(\Cset)$ conjunto  de matrices  de $\Cset$  hermiticas  (a simetr\'ia
  hermitica), $M  \in H_d(\Cset) \, \Leftrightarrow  \, M^\dag =  M$. Fijense de
  que tendr\'iamos $H_d(\Rset) \equiv S_d(\Rset)$
%
\item $P_d(\Kset)$  conjunto de  matrices semidefinida positivas:  $P_d(\Kset) =
  \left\{ M \in H_d(\Kset), \: \quad  \forall \, \mu \in \Kset^d, \mu^\dag M \mu
    \ge 0 \right\}$.
%
\item $P_d^+(\Kset)$  conjunto de  matrices definida positivas:  $P_d^+(\Kset) =
  \left\{ M \in H_d(\Kset), \: \forall \,  \mu \ne 0 \in \Kset^d, \mu^\dag M \mu
    > 0 \right\}$.
\end{itemize}


\

 $\otimes$ es el producto de  Kronecker, $A \otimes B$ \ es matriz bloc
de bloc \ $(i,j)$-\'esima \ $A_{i,j} B$.
%, $J$ \ la matriz bloc de
%bloc  $(i,j)$-\'esima \  $\un_j  \un_i^t$  \ y  $K$  \ la  matriz  bloc de  bloc
%$(i,j)$-\'esima \ $\un_i \un_j^t$.
}

\seccion{Vectores aleatorios complejos y matrices aleatorias en algunas palabras.}
\label{Sec:MP:VectoresComplejosMatricesAleatorias}

\SZ{Introducir en 2 palabres}

% ================================= Caso complejo

\subseccion{Vectores aleatorios complejos}
\label{Ssec:MP:VAComplejos}

Formalmente, un vector aleatorio complejo se define de la misma manear que en el
caso real, de la manera siguiente:
%
\begin{definicion}[Vector aleatorio complejo]
\label{Def:MP:VectorAleatorioComplejo}
%
  Un vector aleatorio complejo es una funci\'on medible
  %
  \[
  Z: (\Omega,\A,P) \mapsto (\Cset^d,\B(\Cset^d),P_Z).
  \]
  %
  donde  $\B(\Cset^d)$  son  los  borelianos  de  $\Cset^d$,  $\sigma$-\'algebra
  generada  por los  productos  cartesianos $(-\infty  \,  ; \,  b_1] +  \imath
  (-\infty \, ;  \, c_1] \times \cdots  \times (-\infty \, ; \,  b_d] + \imath
  (-\infty \,  ; \,  c_d]$ y donde  la medida  $P_Z$ sobre $\B(\Cset^d)$  es la
  medida im\'agen de $P$. Como en el caso real,
  %
  \[
  (Z \in  B) \equiv  Z^{-1}(B) =  \{ \omega \in  \Omega: \:  Z(\omega) \in  B \}
  \qquad \mbox{y} \qquad P_Z(B) = P(Z \in B).
  \]
\end{definicion}

Sin embargo, se puede poner en biyecci\'on  \ $\Cset^d$ \ y \ $\Rset^{2 \, d}$ \
de tal manera de que se  puede definir naturalmente un vector complejo aleatorio
a   partir   de   un   vector   aleatorio  real   de   la   manera   alternativa
equivalente~\cite[Cap.~17]{Lap17}:
%
\begin{definicion}[Vector aleatorio complejo -- definici\'on equivalente]
\label{Def:MP:VectorAleatorioComplejoEquivalente}
%
  Un vector aleatorio complejo se define como
  %
  \[
  Z = X + \imath Y
  \]
  %
  donde \  $\widetilde{Z} \equiv  \begin{bmatrix} X\\ Y  \end{bmatrix}$ \  es un
  vector aleatorio de \ $\Rset^{2 \,  d}$. La medida de probabilidad im\'agen es
  entonces
  %
  \[
  P_Z \equiv P_{\widetilde{Z}} = P_{X,Y}
  \]
  %
\end{definicion}

Resuelte de esta definici\'on equivalente los hechos siguientes:
%
\begin{itemize}
\item La  funci\'on de repartici\'on  de $Z$ se  escribe como \ la  funci\'on de
  repartici\'on conjunta de \ $X$ \ e \ $Y$,
  %
  \[
  F_Z \equiv F_{\widetilde{Z}} = F_{X,Y}
  \]
  %
  Notando de  que es una  funci\'on de \  $x$ \ e \  $y$, $F_Z$ \  hace aparecer
  explicitamente ambos \ $z$ \ y \ $z^*$ complejo conjugado.
%
\item   Si  la   medida  \   $P_{\widetilde{Z}}$  \   admite  una   derivada  de
  Radon-Nykod\'ym con respeto a la medida  de Lebesgue sobre \ $\Rset^{2 \, d}$,
  se  define  la  densidad de  probabilidad  de  \  $Z$  \  como \  $f_Z  \equiv
  f_{\widetilde{Z}} =  f_{X,Y}$. A partir  de la funci\'on de  repartici\'on, se
  escribe entonces o a trav\'es de la derivada $2 \, d$-\'esima de $F_{X,Y}$ con
  respeto a las compontentes \ $x_i$ \ e \ $y_i$ o, de manera equivalente,
  %
  \[
  f_Z(z) = \frac{\partial^{2 \, d}}{\partial z_1 \cdots \partial z_d \, \partial
    z_1^* \cdots \partial z_d^*}
  \]
  %
%
\item Los momentos de orden \ $K$ \ siendos definido a partir de las componentes
  de \ $X$ \ y de \ $Y$, se definen tambi\'en bajo la forma
  %
  \[
  m_{k_1 , \ldots , k_d \, ; \, k'_1 , \ldots , k'_d} = \Esp\left[ \prod_{i=1}^d
    Z_i^{k_i}  \,  \prod_{i=1}^d   Z_i^{* \: k'_i}  \right]  \quad
  \mbox{con} \quad \sum_i (k_i + k'_i) = K
  \]
  %
  y similarmente para los momentos centrales  $\zeta_{k_1 , \ldots , k_d \, ; \,
    k'_1 , \ldots , k'_d}$.  En particular,
  %
  \begin{itemize}
  \item La media de \ $Z = X + \imath Y$ \ es definida por
    %
    \[
    m_Z = \Esp[Z] = \Esp[X] \imath \Esp[Y]
    \]
    %
    La media de $Z^*$ no lleva informaci\'on m\'as de orden 1.
  %
  \item La matriz de covarianza es definida por
    \[
    \Sigma_Z \equiv \Cov[Z] \equiv \Esp\left[ (Z-m_Z) (Z-m_Z)^\dag \right]
    \]
    %
    donde  \  $Z^\dag  =  (Z^*)^t$  \ dicho  {\em  transconjugado}  (transpuesta
    conjugada).    Fijense  de   que,  volviendo   al   vector  $\widetilde{Z}^t
    = \begin{bmatrix} X^t & Y^t \end{bmatrix}$ tenemos por un lado
    %
    \[
    \Sigma_{\widetilde{Z}}  = \begin{bmatrix}
      \Sigma_X & \Sigma_{X,Y}\\ \Sigma_{X,Y}^t & \Sigma_Y\end{bmatrix}
    \]
    %
    conteniendo todas las convarianzas, y por el otro lado,
    %
    \[
    \Sigma_Z =  \left( \Sigma_X  + \Sigma_Y \right)  - \imath \left(  \Sigma_{X,Y} -
      \Sigma_{X,Y}^t \right)
    \]
    \SZ{Decir  antes,  que $\Sigma_{X,Y}  \equiv  \Cov[X,Y]$  y $\Sigma_{Y,X}  =
      \Sigma_{X,Y}^t$}.  Se puede ver que la covarianza de $Z$ no contiene todos
    los  terminos   de  orden   2.  Por  eso,   se  define  tambi\'en   la  {\em
      pseudo-covarianza}, sin terminos conjugados,
    \[
    \check{\Sigma}_Z \equiv \pCov[Z] \equiv \Esp\left[ (Z-m_Z) (Z-m_Z)^t \right]
    \]
    %
    Ahora, se puede ver que
    %
    \[
    \check{\Sigma}_Z =  \left( \Sigma_X  - \Sigma_Y \right)  + \imath \left(  \Sigma_{X,Y} + 
      \Sigma_{X,Y}^t \right)
    \]
    %
    Entonces,  se  recupera  inmediatamente   $\Sigma_X,  \:  \Sigma_Y$  \  y  \
    $\Sigma_{X,Y}$  \  a  partir  de  \ $\Sigma_Z$  \  y  \  $\check{\Sigma}_Z$;
    Claramente,   los   momentos   centrales   de   orden  2   son   dados   por
    \underline{ambas} \ $\Sigma_Z$ \ y \ $\check{\Sigma}_Z$.
  \end{itemize}
  %
  Los momentos  as\'i definidos heriden  naturalmente de las propiedades  de las
  del caso real.
%
\item Se puede ver que  \ $\Sigma_Z \in P_d(\Cset)$ (semi-definida positiva), es
  decir que \ $\Sigma_Z = \Sigma_Z^\dag$ \  y \ $\forall \, \mu \in \Cset, \quad
  \mu^\dag  \Sigma_Z \mu \ge  0$. \SZ{Decirlo  para el  caso real  tambien.}  Al
  rev\'es,  $\check{\Sigma}_Z  \not\in  P_d(\Cset)$;  esta matriz  es  solamente
  sim\'etrica \ $\check{\Sigma}_Z = \check{\Sigma}_Z^t \in S_d(\Cset)$.
%
\item Las generadoras son respectivamente equivalentes a las de $\widetilde{Z}$,
  o  usando   a  la  vez   $Z$  y  $Z^*$.    Por  ejemplo,  para   la  funci\'on
  caracter\'istica, se la puede definir de argumento complejo como
  %
  \[
  \Phi_Z(\omega)  =  \Esp\left[ e^{\imath  \real{\omega^\dag  Z}} \right]  \quad
  \mbox{con} \quad \omega \in \Cset^d
  \]
  %
  (ver  por  ejemplo~\cite[Cap.~17]{Lap17}).   Las funciones  generadoras  as\'i
  definidas heriden naturalmente de las propiedades de las del caso real.
\end{itemize}

En el  marco de vectores  complejos, aparece una subclase  particular invariante
por rotaci\'on, lo qu es conocido como vectores circulares:
%
\begin{definicion}[Vector aleatorio complejo circular]\label{Def:MP:VectorAleatorioComplejoCircular}
%
  Un  vector   aleatorio  complejo   \  $Z$  \   es  dicho  {\em   circular}  en
  torno~\footnote{En la literatura, la noci\'on  de circular es dada para $\mu =
    0$~\cite[Def.~24.3.2]{Lap17}, pero  se extiende sin costo  adicional al caso
    de la  definici\'on dada en este libro.}   a un vector $\mu  \in \Cset^d$ si
  para cualquier $\theta \in [0 \; 2 \pi)$,
  %
  \[
  e^{\imath \theta} \left( Z - \mu \right) \, \egald \, Z - \mu
  \]
\end{definicion}

Los vectores circular tienen propiedades particulares importantes
%
\begin{itemize}
\item  Si $Z$  es circular  al torno  de  un vector  $\mu$ y  admite una  media,
  entonces $$m_Z = \Esp[Z] = \mu$$ Eso viene del hecho de que $e^{\imath \theta}
  \Esp\left[ e^{\imath \theta} (Z - \mu) \right] = \Esp\left[ (Z - \mu) \right]$
  lo que vale necesariamente 0.
%
\item Si $Z$ es circular al torno  de un vector $\mu$ y admite momentos de orden
  2, entonces la pseudo-covarianza es  nula, $$\check{\Sigma}_Z = \pCov[Z] = 0$$
  Recordandose que $m_Z = \mu$, eso  viene del hecho de que $e^{2 \imath \theta}
  \Esp\left[ (Z - m_Z) (Z - m_Z)^t \right] = \Esp\left[ \left( e^{\imath \theta}
      (Z - \mu)  \right) \left( e^{\imath \theta} (Z -  \mu) \right)^t \right] =
  \Esp\left[ (Z  - \mu) (Z  - \mu)^t \right]$  lo que vale necesariamente  0. La
  consecuencia es que en el contexto circular, 
  %
  \[
  \Sigma_X = \Sigma_Y \quad \mbox{y} \quad \Sigma_{X,Y}^t = - \Sigma_{X,Y}
  \]
  %
\end{itemize}

Fijense de que si la pseudo-covarianza  de un vector aleatorio complejo es nula,
eso  no  implica  de  que  el  vector  es circular.  Por  ejemplo,  sea  $Z$  de
distribuci\'on  uniforme sobre  $\Z  = \{  1+\imath \,  ,  \, 1-\imath  \, ,  \,
-1+\imath  \,  ,  \, -1-\imath  \}$  no  puede  ser circular  porque  $e^{\imath
  \frac{\pi}{4}} Z$  toma sus  valores en  $\{ \sqrt2 \,  , \,  -\sqrt2 \,  , \,
\imath \sqrt2 \, , \, -\imath \sqrt2 \} \ne \Z$.

Cuando la pseudo-covarianza es nula se dice a veces que el vector es circular al
orden  2.  M\'as  precisamente,  en   la  literatura,  se  usa  la  definici\'on
siguiente~\cite[Def.~17.4.1]{Lap17}:
%
\begin{definicion}[Vector aleatorio complejo propio]\label{Def:MP:VectorAleatorioComplejoPropio}
%
  Un vector aleatorio complejo \ $Z$  \ es dicho {\em propio} (proper en ingles)
  si admite momentos hasta el orden 2 y ambos,
  %
  \[
  \Esp[Z] = 0, \qquad \pCov[Z] = 0
  \]
\end{definicion}
%
Se podr\'ia ampliar  esta definici\'on hablando de vector propio  al torno de un
vector $\mu$, conservando solamente la nulidad de la pseudo-covarianza.



% ================================= Caso matricial

\subseccion{Matrices aleatorias}
\label{Ssec:MP:MA}


\SZ{ poner unas palabras
  sobre el caso matrix-variate general, y con simetrias

Poner la forma de la covarianza $\Esp[X  \otimes X] - m_X \otimes m_X$ que tiene
todos  los terminos de  covarianza $\Cov[X_{i,j},X_{k,l}]$  (coefficente $(k,l)$
del bloc $(i,j)$. Poner la  forma de la funcion caracter\'istica $\Phi_X(\Omega)
= \Esp \left[ e^{\imath \Tr\left( \Omega  X \right)} \right]$; Se puede salir de
la vectorizaci\'on~\cite{Har08} de \ $X$,  \ie poniendo las columnas una bajo la
otra, y  usar la definici\'on usual. Tomando  en cuenta la s\'imetria  de \ $X$,
eso es  equivalente a definirla como  \ $\Phi_X(\omega) =  \Esp \left[ e^{\imath
    \Tr\left(  \Omega  X \right)}  \right]$  \  con  \ $\Omega  \in  S_d(\Rset)$
conjunto de matrices reales s\'imetricas de $\Rset^{d \times d}$~\cite{PedRic91,
  And03};

Promedio: Para calcularlas, lo m\'as sencillo es salir de la funci\'on
caracter\'istica y ver que, con las simetrias, \ $\frac{\partial
\Phi_X}{\partial \Omega_{i,j}} = \imath (2-\un_{\{i\}}(j)) \Esp\left[ X_{i,j}
\right]$, y usar las reglas de derivaci\'on matricial~\cite[Cap.~8]{MagNeu99}.

Covarianza: Para calcularlas, en el caso e simetrias, lo m\'as sencillo es salir de nuevo de la
funci\'on caracter\'istica y ver que, con las simetrias, \ $\frac{\partial^2
\Phi_X}{\partial \Omega_{i,j} \partial \Omega_{k,l}} = - (2-\un_{\{i\}}(j))
(2-\un_{\{k\}}(l)) \Esp[X_{i,j} X_{k,l}]$ \ y usar las reglas de derivaci\'on
matricial~\cite[Cap.~8]{MagNeu99}
}

% =================================== Ejemplos ================================== %

\SZ{Hablar de convergencia?}

\seccion{Algunos ejemplos de distribuciones de probabilidad}
\label{Sec:MP:EjemplosDistribucionesProb}

En esta secci\'on, vamos a ver unos ejemplos de distribuciones que se encuentran
frecuentemente   en   problema   pr\'acticos   de   varias   areas   cientificas
(estad\'istica,  f\'isica, ingener\'ia,\ldots).  Daremos las  caracteristicas de
cada ley presentada, as\'i que sus propiedades remarcables. El num\'ero de leyes
de probababilidad es tan importante que es dificil, para no decir imposible, ser
exahustivo.  Para  tener  m\'as  detalles,  se  puede  referirse  a  los  libros
especializados  en  este   marco,  como  por  ejemplo~\cite{JohKot92,  JohKot97,
  JohKot95:v1, JohKot95:v2, KotBal00, GupNag99, FanKot90, SamTaq94}.


% ================================= Variables discretas
\subseccion{Distribuciones de variable discreta}
\label{Ssec:MP:EjemplosDistribucionesDiscretas}

% --------------------------------- Certeza
\subsubseccion{Variable real con certeza}
\label{Sssec:MP:Certeza}

El caso \ $X = a \in \Rset^d$ \ deterministico ($\forall \, \omega, \: X(\omega)
= a$)  puede ser ver  visto como un  caso degenerado de vector  aleatorio. Visto
as\'i, sus caracter\'isticas principales  vistas en las secciones anteriores son
resimidas en la tabla siguiente:

\begin{caracteristicas}
%
Dominio de definici\'on & $\X = \{ a \}, \quad a \in \Rset^d$\\[2mm]
\hline
%
Distribuci\'on de probabilidad & $p_X(x) = \un_{\{a\}}(x)$\\[2mm]
\hline
%
Promedio & $\displaystyle m_X = a$\\[2mm]
\hline
%
Covarianza~\footnote{Siendo cero la covarianza, no se define ni la asimetr\'ia,
ni la curtosis. Sin embargo, de una manera se puede decir que la ley no es
asim\'etrica, y con cola livianas (no hay colas).} & $\displaystyle \Sigma_X =
0$\\[2mm]
\hline
%
%\modif{Asimetr\'ia} & $\gamma_X = 0$\\[2mm]
%\hline
%%
%Curtosis por exceso & $\displaystyle \widebar{\kappa}_X = - \sum_{i,j=1}^d \Big( \! \left(
%    \un_i \un_i^t \right) \otimes \left(  \un_j \un_j^t \right) +  \left( \un_i
%    \un_j^t \right) \otimes \left( \un_i  \un_j^t \right) + \left( \un_i \un_j^t
%  \right) \otimes \left( \un_j \un_i^t \right) \! \Big)$\\[2mm]
%\hline
%
Generadora de probabilidad & $\displaystyle G_X(z) = \prod_{i=1}^d z_i^{a_i}$ \ para \ $z_i \in \Cset$
\ si $a_i \ge 0$ \ y \ $\Cset^*$ \ si no\\[2mm]
\hline
%
Generadora de momentos & $\displaystyle M_X(u) = e^{a^t u}$ \ para \ $u \in
\Cset^d$\\[2mm]
\hline
%
Funci\'on caracter\'istica & $\displaystyle \Phi_X(\omega) = e^{\imath \, a^t
\omega}$
\end{caracteristicas}

% Momentos & $ \Esp\left[ X^k \right] = p^k$\\[2mm]
% Momento factorial & $\Esp\left[ (X)_k \right] = ?$\\[2mm]

La funci\'on de masa y funci\'on de repartici\'on son representadas en la figura
Fig.~\ref{Fig:MP:Certeza} en el caso escalar.
%
\begin{figure}[h!]
\begin{center} \begin{tikzpicture}%[scale=.9]
\shorthandoff{>}
%
\pgfmathsetmacro{\a}{2};% a
\pgfmathsetmacro{\sy}{2.5};% y-scaling
\pgfmathsetmacro{\r}{.05};% radius arc non continuity F_X
%
% masa
\begin{scope}
%
%
\draw[>=stealth,->] (-.3,0)--({\a+1.75},0) node[right]{\small $x$};
\draw[>=stealth,->] (0,-.1)--(0,{\sy+.25}) node[above]{\small $p_X$};
%
\draw[dotted] (\a,0)--(\a,\sy) node[scale=.4]{$\bullet$};
\draw (0,\sy)--(-.1,\sy) node[left,scale=.7]{$1$};
\draw (0,0)--(0,-.1) node[below,scale=.7]{$0$};
\draw (\a,0)--(\a,-.1) node[below,scale=.7]{$a$};
%
\node at ({(\a+1.75)/2},-1) [scale=.9]{(a)};
\end{scope}
%
%
% reparticion
\begin{scope}[xshift=8.5cm]
%
\draw[>=stealth,->] (-.3,0)--({\a+1.75},0) node[right]{\small $x$};
\draw[>=stealth,->] (0,-.1)--(0,{\sy+.25}) node[above]{\small $F_X$};
%
% cumulativa
\draw[thick] (-.25,0)--(\a,0);
\draw ({\a+\r},\r) arc (90:270:\r);
\draw[dotted] (\a,0)--(\a,\sy);
\draw[thick] (\a,\sy) node[scale=.4]{$\bullet$}--({\a+1.5},\sy);
%
\draw (0,\sy)--(-.1,\sy) node[left,scale=.7]{$1$};
\draw (\a,0)--(\a,-.1) node[below,scale=.7]{$a$};
%
\node at ({(\a+1.75)/2},-1) [scale=.9]{(b)};
\end{scope}
%
\end{tikzpicture} \end{center}
% 
\leyenda{Ilustraci\'on  de una  distribuci\'on  cierta (a),  y  la funci\'on  de
  repartici\'on asociada (b).}
\label{Fig:MP:Certeza}
\end{figure}

\

Notar que todo se extiende al caso complejo sin costo adicional.

\

\index{Ley de gran n\'umeros}
El caso  de variables  deterministicas puede ser  visto como caso  degenerado de
variables aleatorias, pero aparecen de vez a cuando tambi\'en como caso l\'imite
de  sucesiones o  series  de  variables aleatorias.   En  particular, aparece  a
trav\'es de  la ley  de gran n\'umeros,  un de  los primeros casos  de l\'imites
estudiado tratando  de variables aleatorias. Historicamente, un  de los primeros
que  estudio  la convergencia  (sin  prueba  e  implicitamente) de  un  promedio
empirico a esta ``ley'' es el  matem\'atico italiano y jugador de dados y cartas
Gerolamo Cardano en el siglo~{XVI}, en su libro sobre los juegos de azar escrito
en  1564   (ver  introducci\'on  y~\cite{Car63,   Bel05}  o~\cite[Cap.~4]{Hal90}
o~\cite[Cpa.~3]{Mlo08}).  En  otras palabras,  explic\'o que la  precisi\'on las
estadisticas empiricos se mejora con el  n\'umero de datos, lo que es nada m\'as
que, en palabras,  el resultado de la ley dicha de  gran n\'umeros. En palabras,
saliendo  de  variables  aleatoria  independientes  de misma  ley,  el  promedio
empirico tiende a  la media donde enfatisaremos en que  sentido hay que entender
``tiende a''. Tal  convergencia fue estudiada y probada  mucho m\'as tarde, bajo
el  impulso  del suizo  Jacob  Bernoulli~\cite[Pars  4]{Ber1713} (ver  tambi\'en
Montmort~\cite{Mon13, Pea25})  en el  contexto de variables  binarias, conocidos
hoy  como  variables  de Bernoulli  (ver  subecci\'on~\ref{Sssec:MP:Bernoulli}).
Luego,  el  teorema  fue   mejorado  por  ejemplo  por  de  Moivre~\cite{Moi46},
Laplace~\cite{Lap12} o Poisson~\cite{Poi37}, yendo  m\'as all\'a de solamente la
convergencia del  promedio empirico a la  media.  El teorema  fue ampliado m\'as
all\'a  de la  ley  binomial como  suma  de variables  de  Bernoulli (ver  m\'as
adelante),    por    varios    autores   tales    que    Chebyshev~\cite{Tch46},
Markov~\cite{Mar13},    Borel~\cite{Bor09:12}     ,    Kinchin~\cite{Kin29}    o
Kolmogoroff~\cite{Kol30} entre otros (ver~\cite{Sen13} y referencias).

Formalmente,  las dos  versiones usuales  del  teoremas de  formaliza de  manera
siguiente  (ver  tambi\'en~\cite{Fel71,  Shi84,  AshDol99,  JacPro03,  AthLah06,
  Bil12, Coh13}).

\begin{teorema}[Ley debil de los gran n\'umero]
  Sea  \ $\left\{ X_k  \right\}_{k \in  \Nset^*}$ \  una sucesi\'on  de vectores
  aleatorios  independientes e identicamente  distribuidas (iid),  admitiendo una
  media  $m =  \Esp[X_k]$  \ y  sea  \ $\displaystyle  \widebar{X}_n =  \frac1n
  \sum_{k=1}^n X_k$ \ el promedio empirico. Entonces
  %
  \[
  \widebar{X}_n \limitP{n \to +\infty} m
  \]
  %
  donde $\limitP{}$ significa que el l\'imite es en probabilidad, \ie
  %
  \[
  \forall  \:  \varepsilon  >0,  \quad  \lim_{n  \to  +\infty}  P\left(  \left\|
      \widebar{X}_n - m \right\| > \varepsilon \right) = 0
  \]
\end{teorema}
\begin{proof}
  Una    prueba    sencilla   se    apoya    en    el    teorema   de    Markov,
  Cor.~\ref{Cor:MP:Markov},  cuando  los $X_k$  admiten  una  covarianza. De  la
  independencia, es  sencillo ver que  \ $\Cov\left[ \widebar{X}_n \,  \right] =
  \frac1n \Cov\left[ X_1 \right]$. Entonces,
  %
  \[
  P\left(  \left\|  \widebar{X}_n  -  m  \right\|  >  \varepsilon  \right)  \le
  \frac{\Esp\left[ \left\|  \widebar{X}_n - m\right\|^2 \right]}{\varepsilon^2}
  =  \frac{\Tr\left(   \Cov\left[  X_1  \right]   \right)}{n  \,  \varepsilon^2}
  \xrightarrow[n \to \infty]{} 0
  \]
  %
  lo que cierra la prueba.

  De hecho,  no es necesario  que los $X_k$  admitan una covarianza.  Una prueba
  alternativa    se   apoya   sobre    la   funci\'on    caracter\'istica.   Del
  teorema~\ref{Teo:MP:PropiedadesFuncionCaracteristica},   se   obtiene  de   la
  independencia
  %
  \[
  \Phi_{\widebar{X}_n}(\omega)   =  \left(   \Phi_{X_1}\left(  \frac{\omega}{n}
    \right) \right)^n = \left( 1 + \frac{\imath}{n} m^t \omega + o\left( \left\|
        \frac{\omega}{n} \right\| \right) \right)^n \xrightarrow[n \to \infty]{}
  e^{\imath m^t \omega}
  \]
  %
  En otros t\'erminos, la funci\'on caracter\'istica de $\widebar{X}_n$ tiende a
  la   de  $m$   punto  a   punto.  Se   usa  el   teorema  de   continuidad  de
  L\'evy~\cite{AshDol99, AthLah06, Bil12, Coh13}, no probado en este libro, para
  concluir  que  \  $\widebar{X}_n$  \  tiende  en distribuci\'on  a  \  $m$,  y
  equivalentemente tiende en probabilidad.
\end{proof}
%
Pasando,  de  la primera  prueba,  se  puede notar  que  se  puede debilitar  la
hypotesis  de  independencia,  y  a\'un  la  de misma  ley  para  los  $X_k$,  a
condici\'on de que $\Cov\left[ \widebar{X}_n \right]$ \ tiende a cero cuando $n
\to +\infty$ (por ejemplo, queda valide con la independencia y varianza acotada).

En palabras, el teorema traduce el  pensamiento de Cardano, que es que cualquier
sea el rayo de la bola centrada en $m$, cuando crece el n\'umero de variables en
el promedio  empirico, la probabilidad de  que este promedio sea  afuera de esta
bola tiende a cero.

De hecho, como  para series de funciones (lo que  son las variables aleatorias),
hay varias manera  de converger. Una m\'as fuerte  es conocido como convergencia
casi  siempre,  dando  lugar a  la  ley  dicha  fuerte  de los  gran  n\'umeros.
Historicamente, este teorema es dada en  el caso escalar, pero se extiende en el
caso  vectorial.    No  daremos  la   prueba,  que  se  encuentra   por  ejemplo
en~\cite[Teo.~6.4.2]{Gre63}    en   el   caso    vectorial,   o    entre   otros
en~\cite[Teo.~22.1]{Bil12} en el caso escalar.
%
\begin{teorema}[Ley fuert de los gran n\'umero o teorema de Kolmogorov-Khintchine]
%
  Sea  \ $\left\{ X_k  \right\}_{k \in  \Nset^*}$ \  una sucesi\'on  de vectores
  aleatorios independientes  e identicamente distribuidas  (iid), admitiendo una
  media  $m =  \Esp[X_k]$ \  y  tales que  tambi\'en \  $\Esp\left[ \left\|  X_k
    \right\| \right] <  \infty$, y sea \ $\displaystyle  \widebar{X}_n = \frac1n
  \sum_{k=1}^n X_k$ \ el promedio empirico. Entonces
  %
  \[
  \widebar{X}_n \limitcs{n \to +\infty} m
  \]
  %
  donde $\limitcs{}$ significa que el l\'imite  es casi siempre (o a veces dicho
  ``con probabiludad uno''), \ie
  %
  \[
  P\left(    \lim_{n \to +\infty}   \widebar{X}_n =  m \right) = 1
  \]
  %
  o, dicho  de otra  manera, la medida  del conjuto  $\{ \omega \tq  \lim_{n \to
    +\infty} \widebar{X}_n \ne m \}$ es cero.
\end{teorema}
%\begin{proof}
%Ver~\cite[Teo.~6.4.2]{Gre63}
%\end{proof}
%

Esta versi\'on es dicha fuerte porque la convergencia casi siempre implica la en
probabilidad~\cite{Fel71, Shi84, AshDol99, JacPro03, AthLah06, Bil12, Coh13}. Se
puede debilitar un paso m\'as  las condiciones (ej. indemendencia, etc.) pero va
m\'as all\'a  de la  meta de esta  secci\'on. El  lector se podr\'ea  referir en
libros  especializados,  por  ejemplo~\cite{Fel71,  Shi84,  AshDol99,  JacPro03,
  AthLah06, Bil12, Coh13}.

Una consecuencia de la ley fuerte de gran n\'umeros es conocido como theorema de
Borel. Dice  que, en el  contexto de variables  discretas, si una  experienca se
repite de  manera independiente  un gran n\'umero  de veces, la  proporci\'on de
ocurencia de un  estado tiendo a su probabilidad  de ocurencia (con probabilidad
uno).  Se podr\'a  referir  por  ejemplo a~\cite{Wen91}  para  tener una  prueba
``moderna''.

%\SZ{Poner ac\'a la ley de los gran n\'umeros? M\'as notas historicas.}

% --------------------------------- Uniforme discreta
\subsubseccion{Ley Uniforme sobre un ``intervalo'' de $\Zset$}
\label{Sssec:MP:UniformeDiscreta}

Se denota $X \, \sim \, \U\{ a \, ,  \, b \}$ \ con $(a,b) \in \Zset^2, \: b \ge
a$.  Las caracter\'isticas de \ $X$ \ son las siguientes:

\begin{caracteristicas}
%
Parametros & $(a,b) \in \Zset^2, \: b \ge a$\\[2mm]
\hline
%
Dominio de definici\'on & $\X = \{ a  \, , \, a+1 \,  , \, \ldots \, ,  \, b \}$\\[2mm]
\hline
%
Distribuci\'on de probabilidad & $p_X(x) = \frac1{b-a+1}$\\[2mm]
\hline
%
Promedio & $\displaystyle m_X = \frac{a+b}{2}$\\[2mm]
\hline
%
Varianza & $\displaystyle \sigma_X^2 = \frac{(b-a) (b-a+2)}{12}$\\[2mm]
\hline
%%
\modif{Sesgo} & $\gamma_X = 0$\\[2mm]
\hline
%
Curtosis por exceso & $\displaystyle \widebar{\kappa}_X = -\frac65 \frac{(b-a)
(b-a+2)+2}{(b-a) (b-a+2)}$\\[2mm]
\hline
%
Generadora de probabilidad & $\displaystyle G_X(z) = \frac{z^a-z^{b+1}}{1-z}$ \
para~\footnote{En el caso l\'imite \ $z \to 1$, \ $\lim_{z \to 1} \frac{ z^a -
z^{b+1}}{1-z} = b+1-a$} \ $z \in \Cset$ \ si $a \ge 0$ \ y \ $\Cset^*$ \ si
no\\[2mm]
\hline
%
Generadora de momentos & $\displaystyle M_X(u) = \frac{ e^{a u} - e^{(b+1)
u}}{1-e^u}$ \ para~\footnote{En el caso l\'imite \ $u \to 0$, \ $\lim_{u \to 0}
\frac{ e^{a u} - e^{(b+1) u}}{1-e^u} = b+1-a$, y similarmente para la funci\'on
caracter\'istica.}  \ $u \in \Cset$\\[2mm]
\hline
%
Funci\'on caracter\'istica & $\displaystyle  \Phi_X(\omega) = \frac{ e^{\imath a
\omega} - e^{\imath (b+1) \omega}}{1-e^{\imath \omega}}$
\end{caracteristicas}

% Momentos & $ \Esp\left[ X^k \right] = p^k$\\[2mm]
% Momento factorial & $\Esp\left[ (X)_k \right] = ?$\\[2mm]
% modo 0
% Mediana \ln(2)/\lambda
% CDF 1-e^{-\lambda x}

La distribuci\'on  de masa de probabilidad  y funci\'on de  repartici\'on de una
variable uniforme  \ $\U\{  a \, ,  \, b  \}$ \ son  representadas en  la figura
Fig.~\ref{Fig:MP:UniformeDiscreta}.
%
\begin{figure}[h!]
\begin{center} \begin{tikzpicture}%[scale=.9]
\shorthandoff{>}
%
\pgfmathsetmacro{\sx}{.75};% x-scaling
\pgfmathsetmacro{\r}{.05};% radius arc non continuity F_X
\pgfmathsetmacro{\n}{6};% n de la uniforme
\pgfmathsetmacro{\m}{\n-1};
%
% masa
\begin{scope}
%
%
\pgfmathsetmacro{\sy}{2.5};% y-scaling 
\draw[>=stealth,->] (-.25,0)--({\sx*(\n+.5)+.25},0) node[right]{\small $x$};
\draw[>=stealth,->] (0,-.1)--(0,{\sy+.25}) node[above]{\small $p_X$};
%
\foreach \k in {1,...,\n} {
\draw[dotted] ({\k*\sx},0)--({\k*\sx},\sy) node[scale=.4]{$\bullet$};
\draw ({\k*\sx},0)--({\k*\sx},-.1) node[below,scale=.7]{$\k$};
}
\draw (0,\sy)--(-.1,\sy) node[left,scale=.7]{$\frac1\n$};
%%
\end{scope}
%
%
% reparticion
\begin{scope}[xshift=8.5cm]
%
\pgfmathsetmacro{\sy}{2.5};% y-scaling 
%
\draw[>=stealth,->] ({-\sx/2-.25},0)--({\sx*(\n+1.5)+.25},0) node[right]{\small $x$};
\draw[>=stealth,->] (0,-.1)--(0,{\sy+.25}) node[above]{\small $F_X$};
%
% cumulativa
\draw[thick] ({-\sx/2},0)--(\sx,0);
\draw ({\sx+\r},\r) arc (90:270:\r);
\draw (0,0)--(0,-.1) node[below,scale=.7]{$0$};
%
\foreach \k in {1,...,\m} {
\draw ({\k*\sx},0)--({\k*\sx},-.1) node[below,scale=.7]{$\k$};
\draw[thick] ({\k*\sx},{\k*\sy/\n}) node[scale=.4]{$\bullet$}--({(\k+1)*\sx},{\k*\sy/\n});
\draw ({(\k+1)*\sx+\r},{\k*\sy/\n+\r}) arc (90:270:\r);
\draw[dotted] ({\k*\sx},{(\k-1)*\sy/\n})--({\k*\sx},{\k*\sy/\n});
}
\draw ({\n*\sx},0)--({\n*\sx},-.1) node[below,scale=.7]{$\n$};
\draw[thick] ({\n*\sx},\sy) node[scale=.4]{$\bullet$}--({(\n+1.5)*\sx},\sy);
\draw[dotted] ({\n*\sx},{(\n-1)*\sy/\n})--({\n*\sx},\sy);
%%
\draw (0,\sy)--(-.1,\sy) node[left,scale=.7]{$1$};
\end{scope}
%
\end{tikzpicture} \end{center}
% 
\leyenda{Ilustraci\'on  de  una densidad  de  probabilidad  uniforme  (a), y  la
  funci\'on  de repartici\'on  asociada (b).  $a =  1,  \: b  = 6$  \ (ej.  dado
  equilibriado).}
\label{Fig:MP:UniformeDiscreta}
\end{figure}

Cuando \ $b = a$, la variable tiende a una variable cierta \ $X = a$.

La  distribuci\'on  uniforme  aparece  por   ejemplo  en  el  tiro  de  un  dado
equilibriado con \ $a = 1, \: b = 6$.
%% en el conteo de
%%conteo de une repetici\'on de  una experiencia de maneja independiente hasta que
%%occure un evento de probabilidad $p$; por ejemplo el n\'umero de tiro de un dado
%%equilibriado hasta que occurre un ``6'' sigue una ley geometrica de parametro $p
%%= \frac16$.


% --------------------------------- Bernoulli
\subsubseccion{Ley de Bernoulli}
\label{Sssec:MP:Bernoulli}

Esta ley aparece  cuando se hace una experiencia con 2  estados posible, tipo un
tiro  de  moneda.   Apareci\'o en  trabajos  muy  antiguos,  entre otros  el  de
J.  Bernoulli  tratando  de  la  ley  de  gran  n\'umeros~\cite{Ber1713,  Hal90,
  DavEdw01}.

Se  denota \  $X \,  \sim \,  \B(p)$ \  con \  $p \in  [0 \;  1]$ \  y sus
caracter\'isticas son las siguientes:

\begin{caracteristicas}
%
Dominio de definici\'on & $\X = \{ 0 \; 1 \}$\\[2mm]
\hline
%
Par\'ametro & $p \in [ 0 \; 1 ]$\\[2mm]
\hline
%
Distribuci\'on de probabilidad & $p_X(1) = 1 - p_X (0) = p$\\[2mm]
\hline
%
Promedio & $ m_X = p$\\[2mm]
\hline
%
Varianza & $\sigma_X^2 = p \, (1-p)$\\[2mm]
\hline
%
\modif{Asimetr\'ia} & $\displaystyle \gamma_X =  \frac{1 - 2 \, p}{\sqrt{p \, (1-p)}}$\\[2mm]
\hline
%
Curtosis por exceso & $\displaystyle \widebar{\kappa}_X = \frac{1 - 6 \, p + 6
\, p^2}{p \, (1-p)}$\\[2mm]
\hline
%
Generadora de probabilidad & $G_X(z) = 1 - p + p z$ \ sobre \ $\Cset$\\[2mm]
\hline
%
Generadora de momentos & $M_X(u) = 1 - p + p \, e^u$ \ sobre \ $\Cset$\\[2mm]
\hline
%
Funci\'on caracter\'istica & $\Phi_X(\omega) = 1 - p + p \, e^{\imath \omega}$
\end{caracteristicas}


% Momentos & $ \Esp\left[ X^k \right] = p^k\\[2mm]
% Momento factorial & $\Esp\left[ (X)_k \right] = p^k \un_{\{0 \, , \, 1 \}}(k)$\\[2mm]

Su masa  de probabilidad  y funci\'on de  repartici\'on son representadas  en la
figura Fig.~\ref{Fig:MP:Bernoulli}.
%
\begin{figure}[h!]
\begin{center} \begin{tikzpicture}%[scale=.9]
\shorthandoff{>}
%
\pgfmathsetmacro{\sx}{2};% x-scaling
\pgfmathsetmacro{\r}{.05};% radius arc non continuity F_X
\pgfmathsetmacro{\p}{1/3};% probabilidad p
% masa
\begin{scope}
%
\pgfmathsetmacro{\sy}{2/max(\p,1-\p)};% y-scaling
%
\pgfmathsetmacro{\ss}{\sy*(1-\p)};
\draw[>=stealth,->] (-.5,0)--({\sx+.75},0) node[right]{\small $x$};
\draw[>=stealth,->] (0,-.15)--(0,2.5) node[above]{\small $p_X$};
%
\draw (0,-.1) node[below,scale=.7]{$0$} --(0,0);
\draw[dotted] (0,0)--(0,{\sy*(1-\p)}) node[scale=.7]{$\bullet$};
\draw (0,{\sy*(1-\p)})--(-.1,{\sy*(1-\p)}) node[left,scale=.7]{$1-p$};
%
\draw (\sx,-.1) node[below,scale=.8]{\small $1$} --(\sx,0);
\draw[dotted] (\sx,0)--(\sx,{\sy*\p}) node[scale=.7]{$\bullet$};
\draw (0,{\sy*\p})--(-.1,{\sy*\p}) node[left,scale=.7]{\small $p$};
%
\node at ({(\sx+.75)/2},-1) [scale=.9]{(a)};
\end{scope}
%
%
% reparticion
\begin{scope}[xshift=7cm]
%
\pgfmathsetmacro{\sy}{2};% y-scaling 
%
\draw[>=stealth,->] (-.5,0)--({\sx+1.5},0) node[right]{\small $x$};
\draw[>=stealth,->] (0,-.15)--(0,{\sy+.5}) node[above]{\small $F_X$};
%
\draw (0,0)--(0,-.1) node[below,scale=.7]{$0$};
\draw (\sx,0)--(\sx,-.1) node[below,scale=.7]{$1$};
\draw (0,{\sy*(1-\p)})--(-.1,{\sy*(1-\p)}) node[left,scale=.7]{$1-p$};
\draw (0,\sy)--(-.1,\sy) node[left,scale=.7]{$1$};
%
\draw[thick](-.25,0)--(0,0);
\draw ({0+\r},\r) arc (90:270:\r);
%
\draw[dotted] (0,0)--(0,{\sy*(1-\p)});
\draw[thick](0,{\sy*(1-\p)}) node[scale=.7]{$\bullet$}--(\sx,{\sy*(1-\p)});
\draw ({\sx+\r},{\r+\sy*(1-\p)}) arc (90:270:\r);
%
\draw[dotted] (\sx,{\sy*(1-\p)})--(\sx,\sy);
\draw[thick](\sx,\sy) node[scale=.7]{$\bullet$}--({\sx+1},\sy);
%\draw ({\sx+\r},{\r+\sy*(1-\p)}) arc (90:270:\r);
%
\node at ({(\sx+1.5)/2},-1) [scale=.9]{(b)};
\end{scope}
%
\end{tikzpicture} \end{center}
%
\leyenda{Ilustraci\'on de una distribuci\'on de probabilidad de Bernoulli (a), y
  la funci\'on de repartici\'on asociada (b), con $p = \frac13$.}
\label{Fig:MP:Bernoulli}
\end{figure}

Nota que cuando $p = 0$ (resp. $p =  1$) la variable es cierta $X = 0$ (resp. $X
= 1$).

La  ley de Bernoulli tiene una propiedad de reflexividad trivial:
%
\begin{lema}[Reflexividad]
\label{Lem:MP:ReflexividadBernoulli}
%
  Sea \ $X \, \sim \, \B(p)$. Entonces
  %
  \[
  1-X \, \sim \, \B(1-p)
  \]
  %
\end{lema}
\begin{proof}
El resultado es inmediato de $P(1-X = 1) = P(X = 0) = 1-p$.
\end{proof}

% --------------------------------- Binomial
\subsubseccion{Ley Binomial}
\label{Sssec:MP:Binomial}

Se denota \ $X \, \sim \, \B(n,p)$ \ con \ $n \in \Nset \setminus \{ 0 \; 1 \}$,
\quad $p \in [0 \; 1]$ \ y sus caracter\'isticas son las siguientes:

\begin{caracteristicas}
%
Dominio de definici\'on & $\X = \{ 0 \; \ldots \; n \}$\\[2mm]
\hline
%
Parametros & $n  \in \Nset \setminus \{0  \; 1 \},  \quad p \in [0  \;
1]$\\[2mm]
\hline
%
Distribuci\'on de probabilidad & \protect$\displaystyle p_X(x) = \bino{n}{x} \, p^x
(1-p)^{n-x}$\protect\\[2mm]
\hline
%
Promedio & $ m_X = n \, p$\\[2mm]
\hline
%
Varianza & $\sigma_X^2 = n \, p \, (1-p)$\\[2mm]
\hline
%
\modif{Sesgo} & $\displaystyle \gamma_X = \frac{1 - 2 \, p}{\sqrt{n \, p \, (1-p)}}$\\[2mm]
\hline
%
Curtosis por exceso & $\displaystyle \widebar{\kappa}_X = \frac{1 - 6 \, p + 6 \, p^2}{n \, p
\, (1-p)} $\\[2mm]
\hline
%
Generadora  de probabilidad  &  $\displaystyle  G_X(z) =  \left(  1 -  p  + p  z
\right)^n$ \ sobre \ $\Cset$\\[2mm]
\hline
%
Generadora  de momentos  &  $\displaystyle  M_X(u) =  \left(1  - p  +  p \,  e^u
\right)^n$ \ sobre \ $\Cset$\\[2mm]
\hline
%
Funci\'on caracter\'istica  & $\displaystyle \Phi_X(\omega) =  \left( 1 -  p + p
\, e^{\imath \omega} \right)^n$
\end{caracteristicas}

% Momentos & $ \Esp\left[ X^k \right] = ??\\[2mm]
% Momento factorial & $\Esp\left[ (X)_k \right] = 
% \frac{n!}{(n-k)!} p^k \un_{\{ 0 \, , \, \ldots \, , \, n \}}(k)$\\[2mm]
% Modo $\left\lfloor (n+1) p \right\rfloor$
% Mediana $\left\lfloor n p \right\rfloor$ o $\left\lceil n p \right\rceil
% CDF	$I_{1-p}(n-k,k+1)$ regularized incomplete beta function

Su masa  de probabilidad  y funci\'on de  repartici\'on son representadas  en la
figura Fig.~\ref{Fig:MP:Binomial}.
%
\begin{figure}[h!]
\begin{center} \begin{tikzpicture}%[scale=.9]
\shorthandoff{>}
%
\pgfmathsetmacro{\sx}{.75};% x-scaling
\pgfmathsetmacro{\r}{.05};% radius arc non continuity F_X
\pgfmathsetmacro{\p}{1/3};% probabilidad p
\pgfmathsetmacro{\n}{6};% numero n de la binomial
\pgfmathsetmacro{\q}{floor((\n+1)*\p)};% modo de la binomial
\pgfmathsetmacro{\m}{factorial(\n)/factorial(\q)/factorial(\n-\q)*(\p^\q)*((1-\p)^(\n-\q))};% maximo de la binomial
% masa
\begin{scope}
%
\pgfmathsetmacro{\sy}{2.5/\m};% y-scaling 
\draw[>=stealth,->] (-.25,0)--({\sx*\n+.25},0) node[right]{\small $x$};
\draw[>=stealth,->] (0,-.1)--(0,{\sy*\m+.25}) node[above]{\small $p_X$};
%
\pgfmathsetmacro{\b}{(1-\p)^\n};% coeficiente binomial por la probabilidad
%
\foreach \k in {0,...,\n} {
\draw ({\k*\sx},0)--({\k*\sx},-.1) node[below,scale=.7]{\k};
\draw[dotted] ({\k*\sx},0)--({\k*\sx},{\sy*\b}) node[scale=.7]{$\bullet$};
%
\pgfmathsetmacro{\bl}{\b*\p*(\n-\k)/((\k+1)*(1-\p))};\global\let\b\bl;% proba actualizado
}
\draw (0,{((1-\p)^\n)*\sy})--(-.1,{((1-\p)^\n)*\sy}) node[left,scale=.7]{$(1-p)^n$};
\draw (0,{\n*\p*((1-\p)^(\n-1))*\sy})--(-.1,{\n*\p*((1-\p)^(\n-1))*\sy}) node[left,scale=.7]{$n p (1-p)^{n-1}$};
%
\end{scope}
%
%
% reparticion
\begin{scope}[xshift=8.5cm]
%
\pgfmathsetmacro{\sy}{2.5};% y-scaling 
%
\draw[>=stealth,->] (-.6,0)--({\sx*(\n+.5)+.5},0) node[right]{\small $x$};
\draw[>=stealth,->] (0,-.1)--(0,{\sy+.25}) node[above]{\small $F_X$};
%
\pgfmathsetmacro{\b}{(1-\p)^\n};% coeficiente binomial por la probabilidad
\pgfmathsetmacro{\c}{(1-\p)^\n};% cumulativa binomial por la probabilidad
%
% cumulativa x < 0
\draw (0,0)--(0,-.1) node[below,scale=.7]{0};
\draw[thick] (-.5,0)--(0,0);
\draw (\r,\r) arc (90:270:\r);
%
% cumulativa x de 0 a n-1
\foreach \k in {1,...,\n} {
\draw ({\k*\sx},0)--({\k*\sx},-.1) node[below,scale=.7]{\k};
\draw[thick]({(\k-1)*\sx},{\sy*\c}) node[scale=.7]{$\bullet$}--({\k*\sx},{\sy*\c});
\draw ({\k*\sx+\r},{\sy*\c+\r}) arc (90:270:\r);
\draw[dotted] ({(\k-1)*\sx},{(\c-\b)*\sy})--({(\k-1)*\sx},{\c*\sy});
%
\pgfmathsetmacro{\bl}{\b*\p*(\n-\k+1)/(\k*(1-\p))};\global\let\b\bl;% proba actualizado
\pgfmathsetmacro{\cl}{\c+\b};\global\let\c\cl;% cumulativa actualizada
}
%
% cumulativa x > n
\draw[dotted] ({\n*\sx},{(1-\b)*\sy})--({\n*\sx},\sy);
\draw[thick]({\n*\sx},\sy) node[scale=.7]{$\bullet$}--({(\n+.5)*\sx},\sy);
%
\draw (0,{((1-\p)^\n)*\sy})--(-.1,{((1-\p)^\n)*\sy}) node[left,scale=.7]{$(1-p)^n$};
\draw (0,{(\n*\p+1-\p)*((1-\p)^(\n-1))*\sy})--(-.1,{(\n*\p+1-\p)*((1-\p)^(\n-1))*\sy}) node[left,scale=.7]{$(1-p+np) (1-p)^{n-1}$};
\draw (-.2,{((\n*\p+1-\p)*((1-\p)^(\n-1))+1)/2*\sy}) node[scale=.7]{$\vdots$};
\draw (0,\sy)--(-.1,\sy) node[left,scale=.7]{\small $1$};
\end{scope}
%
\end{tikzpicture} \end{center}
%
\leyenda{Ilustraci\'on de una distribuci\'on  de probabilidad Binomial (a), y la
  funci\'on de repartici\'on asociada (b), con $n = 6$, \quad $p = \frac13$.}
\label{Fig:MP:Binomial}
\end{figure}

\SZ{Otros ilustraciones para otros $p$?}

Cuando  $n  = 1$,  se  recupera  la lei  de  Bernoulli  $\B(p) \equiv  \B(1,p)$.
Ad\'emas, se muestra  sencillamente usando la generadora de  probabilidad que
%
\begin{lema}
\label{Lem:BinomilaSumaBernoulli}
%
  Sean \  $X_i \,  \sim \, \B(p),  \quad i  = 1, \ldots  , n$  \ independientes,
  entonces
  %
  \[
  \sum_{i=1}^n X_i \, \sim \, \B(n,p)
  \]
\end{lema}
%
De este resultado,  se puede notar que, por  ejemplo, le distribuci\'on binomial
aparece en el conteo de eventos independientes de misma probabilidad entre $n$.

Tambi\'en,  la ley binomial  tiene una  propiedad de  reflexividad, consecuencia
directa de la de Bernoulli:
%
\begin{lema}[Reflexividad]
\label{Lem:MP:ReflexividadBinomial}
%
  Sea \ $X \, \sim \, \B(n,p)$. Entonces
  %
  \[
  n-X \, \sim \, \B(n,1-p)
  \]
  %
\end{lema}
\begin{proof}
  El  resultado es  inmediato  de la  propiedad  de reflexividad  de  la ley  de
  Bernoulli,                           conjuntamente                          al
  lema~\ref{Lem:BinomilaSumaBernoulli}. Alternativamente,  se nota que  $P(n-X =
  x) = P(X = n-x) = \bino{n}{n-x} p^{n-x} (1-p)^x = \bino{n}{x} (1-p)^x p^{n-x}$
  \ notando que $\bino{n}{n-x} = \bino{n}{x}$.
\end{proof}

Nota que cuando $p = 0$ (resp. $p = 1$) la variable es cierta $X = 0$ (resp.  $X
= n$).


% --------------------------------- Binomial negativa
\subsubseccion{Ley Binomial negativa}
\label{Sssec:MP:BinomialNegativa}

Se denota \ $X \, \sim \, \B_-(r,p)$ \ con \ $r \in \Nset^*$, \quad $p \in [0 \;
1)$ \ y sus caracter\'isticas son las siguientes:

\begin{caracteristicas}
%
Dominio de definici\'on & $\X = \Nset$\\[2mm]
\hline
%
Parametros & $r  \in \Nset^*,  \quad p \in [0  \;
1)$\\[2mm]
\hline
%
Distribuci\'on de probabilidad & \protect$\displaystyle p_X(x) = \bino{x+r-1}{x}
\, p^x (1-p)^r$\protect\\[2mm]
\hline
%
Promedio & $\displaystyle m_X = \frac{r \, p}{1-p}$\\[2mm]
\hline
%
Varianza & $\displaystyle \sigma_X^2 = \frac{r \, p}{(1-p)^2}$\\[2mm]
\hline
%
\modif{Sesgo} & $\displaystyle \gamma_X = \frac{1 + p}{\sqrt{r \, p}}$\\[2mm]
\hline
%
Curtosis por exceso & $\displaystyle \widebar{\kappa}_X = \frac{1 + 4 \, p +
p^2}{r \, p} $\\[2mm]
\hline
%
Generadora de probabilidad & $\displaystyle G_X(z) = \left( \frac{1 - p}{1 - p
\, z} \right)^r$ \ para \ $|z| < p^{-1} $\\[2mm]
\hline
%
Generadora de momentos & $\displaystyle M_X(u) = \left( \frac{1 - p}{1 - p \,
e^u } \right)^r$ \ para \ $\real{u} < - \ln p$\\[2mm]
\hline
%
Funci\'on caracter\'istica & $\displaystyle \Phi_X(\omega) = \left( \frac{1 -
p}{1 - p \, e^{i \omega} } \right)^r$
\end{caracteristicas}

% Momentos & $ \Esp\left[ X^k \right] = ??\\[2mm]
% Momento factorial & $\Esp\left[ (X)_k \right] = 
% \frac{(r+k-1)!}{(r-1)!} \left( \frac{p}{1-p} \right)^k$\\[2mm]
% Modo $\left\lfloor (n+1) p \right\rfloor$
% Mediana $\left\lfloor n p \right\rfloor$ o $\left\lceil n p \right\rceil
% CDF	$I_{1-p}(n-k,k+1)$ regularized incomplete beta function

Su masa  de probabilidad  y funci\'on de  repartici\'on son representadas  en la
figura Fig.~\ref{Fig:MP:BinomialNegativa}.
%
\begin{figure}[h!]
\begin{center} \begin{tikzpicture}%[scale=.9]
\shorthandoff{>}
%
\pgfmathsetmacro{\sx}{.45};% x-scaling
\pgfmathsetmacro{\r}{.05};% radius arc non continuity F_X
\pgfmathsetmacro{\p}{3/5};% probabilidad p de suceso
\pgfmathsetmacro{\rp}{3};% numero r de fracascos
\pgfmathsetmacro{\n}{12};% numero maximo a dibujar
\pgfmathsetmacro{\q}{max(floor((\rp-1)*\p/(1-\p)),0)};% modo de la binomial negativa
\pgfmathsetmacro{\m}{factorial(\q+\rp-1)*((1-\p)^\rp)*(\p^\q)/factorial(\rp-1)/factorial(\q)};
%{factorial(\n)/factorial(\q)/factorial(\n-\q)*(\p^\q)*((1-\p)^(\n-\q))};% maximo de la binomial
% masa
\begin{scope}
%
\pgfmathsetmacro{\sy}{2.5/\m};% y-scaling 
\draw[>=stealth,->] (-.25,0)--({\sx*(\n+.75)+.25},0) node[right]{\small $x$};
\draw[>=stealth,->] (0,-.1)--(0,{\sy*\m+.25}) node[above]{\small $p_X$};
%
\pgfmathsetmacro{\b}{(1-\p)^\rp};% coeficiente binomial por la probabilidad
%
\foreach \k in {0,...,\n} {
\draw ({\k*\sx},0)--({\k*\sx},-.1) node[below,scale=.7]{$\k$};
\draw[dotted] ({\k*\sx},0)--({\k*\sx},{\sy*\b}) node[scale=.7]{$\bullet$};
%
\pgfmathsetmacro{\bl}{\b*\p*(\k+\rp)/(\k+1)};\global\let\b\bl;% proba actualizada
}
\draw ({(\n+.25)*\sx},{\sy*\b*(\n+1)/\p/(\n+\rp)/2}) node[right,scale=.7]{\ldots};
\draw (0,{((1-\p)^\rp)*\sy})--(-.1,{((1-\p)^\rp)*\sy}) node[left,scale=.7]{$(1-p)^r$};
\draw (0,{\rp*\p*((1-\p)^\rp)*\sy})--(-.1,{\rp*\p*((1-\p)^\rp)*\sy}) node[left,scale=.7]{$r \, p \, (1-p)^r$};
%\draw (0,{(\rp*\p*((1-\p)^\rp)+\m)/2*\sy}) node[scale=.7]{$r \, p \, (1-p)^r$};
%
\end{scope}
%
%
% reparticion
\begin{scope}[xshift=8.5cm]
%
\pgfmathsetmacro{\sy}{2.5};% y-scaling 
%
\draw[>=stealth,->] (-.6,0)--({\sx*(\n+.75)+.5},0) node[right]{\small $x$};
\draw[>=stealth,->] (0,-.1)--(0,{\sy+.25}) node[above]{\small $F_X$};
%
\pgfmathsetmacro{\b}{(1-\p)^\rp};% coeficiente binomial por la probabilidad
\pgfmathsetmacro{\c}{(1-\p)^\rp};% cumulativa binomial por la probabilidad
%
% cumulativa x < 0
\draw (0,0)--(0,-.1) node[below,scale=.7]{$0$};
\draw[thick] (-.5,0)--(0,0);
\draw (\r,\r) arc (90:270:\r);
%
% cumulativa x de 0 a n-1
\foreach \k in {1,...,\n} {
\draw ({\k*\sx},0)--({\k*\sx},-.1) node[below,scale=.7]{$\k$};
\draw[thick]({(\k-1)*\sx},{\sy*\c}) node[scale=.7]{$\bullet$}--({\k*\sx},{\sy*\c});
\draw ({\k*\sx+\r},{\sy*\c+\r}) arc (90:270:\r);
\draw[dotted] ({(\k-1)*\sx},{(\c-\b)*\sy})--({(\k-1)*\sx},{\c*\sy});
%
\pgfmathsetmacro{\bl}{\b*\p*(\k+\rp-1)/\k};\global\let\b\bl;% proba actualizada
\pgfmathsetmacro{\cl}{\c+\b};\global\let\c\cl;% cumulativa actualizada
}
%
\draw ({\n*\sx},{\sy*(\c+1)/2}) node[left,scale=.7]{\ldots};
\draw (0,{((1-\p)^\rp)*\sy})--(-.1,{((1-\p)^\rp)*\sy}) node[left,scale=.7]{$(1-p)^r$};
\draw (0,{(1+\rp*\p)*((1-\p)^\rp)*\sy})--(-.1,{(1+\rp*\p)*((1-\p)^\rp)*\sy}) node[left,scale=.7]{$(1+r \, p) (1-p)^r$};
\draw (-.75,{((1+\rp*\p)*((1-\p)^\rp)+1)/2*\sy}) node[right,scale=.7]{$\vdots$};
\draw (0,\sy)--(-.1,\sy) node[left,scale=.7]{$1$};
\end{scope}
%
\end{tikzpicture} \end{center}
%
\leyenda{Ilustraci\'on de  una distribuci\'on de  probabilidad binomial negativa
  (a), y  la funci\'on  de repartici\'on  asociada (b), con  $r =  3, \quad  p =
  \frac35$.}
\label{Fig:MP:BinomialNegativa}
\end{figure}
\SZ{Otros ilustraciones para otros $r, p$?}

Esta ley aparece  cuando se repite una experencia  binaria \ $X_i \in \{  0 \; 1
\}, i  = 1,  \ldots$ \ con  \ $P(X_i=1)  = p$ \  de manera  independiente ($X_i$
independientes)  hasta que  \ $r$  \ variables  valen 0,  con \  $r$ \  fijo. El
n\'umero  de  excito  \ $X$  \  sigue  una  ley  \  $\B_-(r,p)$ (el  calculo  es
directo). Dicho de  otra manera, $X =  \sum_{i=1}^N X_i$ \ con \  $N$ \ variable
aleatoria tal que $X_N = 0$ \ y \ $r = \sum_{i=1}^N (1-X_i)$: condicionalmente a
\ $N$,  la variable \modif{\ $X$ } \  es binomial de parametro  $p$\modif{, \ie \
    $P(X=x|N=n) = \bino{n}{x} p^x (1-p)^{n-x}$}.  Se puede ver que \ $P(N = n) =
  \bino{n}{r-1} (1-p)^r p^{n-r}$ \ y la ley de la binomial negativa se recupera
\modif{a        trav\'es        del        teorema        de        probabilidad
  total~\ref{Teo:MP:ProbaTotalDiscreto}    o   tambi\'en,}   a    trav\'es   del
teorema~\ref{Teo:MP:SumaAleatoriaGeneradoraProbabilidad}.
%
% Blaise PAscal - Polya caso r real

Esta distribuci\'on se  generaliza para \ $r \in \Rset_+^*$ \  pero se pierde la
interpretaci\'on que v\'imos en el p\'arafo anterior.

Nota: cuando \ $p = 0$ \ la variable es cierta \ $X = r$.

% --------------------------------- Multinomial
\subsubseccion{Ley Multinomial}
\label{Sssec:MP:Multinomial}

Esta ley es una generalizaci\'on de la ley binomial y aparece por ejemplo cuando
se  repite  una  experiencia  a  \  $k$  \ estados  \  $n$  \  veces  de  manera
independiente y nos  interesamos a la probabilidad que  el primer evento aparece
$n_1$ veces,  el secundo  $n_2$ veces, \ldots  (ej. para  $k = 6$,  contamos los
n\'umeros de $1$, de $2$, \ldots cuando tiramos $n$ veces este dado).  Se denota
\ $X \ \sim \ \M(n,p)$ \ con \  $n \in \Nset^*$ \ y \ $p = \begin{bmatrix} p_1 &
  \cdots & p_k \end{bmatrix}^t \in  \Simp{k-1}$ \ the \ $(k-1)$-simplex estandar
(ver figure~\ref{Fig:MP:Dirichlet}-(a) y notaciones).   Entonces, a pesar de que
se escribe \  $X$ \ de manera $k$-dimensional, el vector  partenece a un espacio
claramente \ $d = k-1$ \ dimensional y en el caso \ $k = 2$ \ se recupera la ley
binomial.  Las caracter\'isticas de \ $X \ \sim \ \M(n,p)$ \ son las siguientes:

\begin{caracteristicas}
%
Dominio de definici\'on~\footnote{De hecho, se puede considerar que el vector
aleatorio es \ $(k-1)$-dimensional \ $\widetilde{X} = \begin{bmatrix}
\widetilde{X}_1 & \cdots & \widetilde{X}_{k-1} \end{bmatrix}^t$ \ definido sobre
el dominio \ $\widetilde{\X} = \left\{ x \in \{ 0 \; \ldots \; n\}^{k-1}, \:
\sum_{i=1}^{k-1} x_i \le n \right\}$.\label{Foot:MP:MultinomialDominio}} & $\X =
\left\{ x \in \{ 0 \; \ldots \; n\}^k \tq \sum_{i=1}^k x_i = n \right\}$\\[2mm]
\hline
%
Parametros~\footnote{El par\'ametro de \ $\widetilde{X}$ \ es \ $\widetilde{p} =
\protect\begin{bmatrix} p_1 & \cdots & p_{k-1} \end{bmatrix}^t\protect \in
\left\{ q \in [0 \; 1]^{k-1} \tq \sum_{i=1}^{k-1} q_i \le 1
\right\}$.\label{Foot:MP:MultinomialParametro}} & $n \in \Nset^*$, \quad $p \in
\Simp{k-1}$\\[2mm]
\hline
%
Distribuci\'on de probabilidad~\footnote{La masa de probabilidad de \
$\widetilde{X}$ \ es \ $p_{\widetilde{X}}(x) = \frac{n!}{\prod_{i=1}^{k-1} x_i!
(n-\sum_{i=1}^{k-1} x_i)!}  \prod_{i=1}^{k-1} p_i^{x_i} \, \left( 1 -
\sum_{i=1}^{k-1} p_i \right)^{n-\sum_{i=1}^{k-1}
x_i}$.\label{Foot:MP:MultinomialMasa}} & $\displaystyle p_X(x) =
\frac{n!}{\prod_{i=1}^k x_i!}  \prod_{i=1}^k p_i^{x_i}$\\[2mm]
\hline
%
Promedio & $\displaystyle m_X = n \, p$\\[2mm]
\hline
%
Covarianza~\footnote{$\Sigma_X \in P_k(\Rset)$, pero de \ $\un^t \Sigma_X \un =
0$ \ viene \ $\Sigma_X \not\in P_k^+(\Rset)$. Eso es la consecuencia directa del
hecho de que \ $X$ \ $d$-dimensional, vive sobre \ $\Simp{k-1}$,
$(d-1)$-dimensional.\label{Foot:MP::MultinomialCovarianza}} & $\displaystyle
\Sigma_X = n \left( \diag p - p \, p^t \right)$\\[2mm]
\hline
%
Generadora de probabilidad~\footnote{Notar: $G_{\widetilde{X}}\left(
\widetilde{z} \right) = G_X\left( \begin{bmatrix} \widetilde{z} &
1 \end{bmatrix}^t \right)$ \ y al rev\'es \ $G_X(z) = z_k^n \,
G_{\widetilde{X}}\left( \begin{bmatrix} \frac{z_1}{z_k} & \cdots &
\frac{z_{k-1}}{z_k} \end{bmatrix}^t
\right)$.\label{Foot:MP:MultinomialGeneProba}} & $\displaystyle G_X(z) = \left(
p^t z \right)^n$ \ para \ $z \in \Cset^k$\\[2mm]
\hline
%
Generadora de momentos~\footnote{Notar: $M_{\widetilde{X}}\left( \widetilde{u}
\right) = M_X\left( \begin{bmatrix} \widetilde{u} & 0 \end{bmatrix}^t \right)$ \
y \ $M_X(u) = e^{n \, u_k} M_{\widetilde{X}}\left( \begin{bmatrix} u_1 - u_k &
\cdots & u_{k-1} - u_k \end{bmatrix}^t
\right)$.\label{Foot:MP:MultinomialGeneMomentos}} & \protect$\displaystyle
M_X(u) = \left( p^t e^u \right)^n, \: e^u = \begin{bmatrix} e^{u_1} & \cdots &
e^{u_k} \end{bmatrix}^t$\protect \ para \ $u \in \Cset^k$\\[2mm]
\hline
%
Funci\'on caracter\'istica~\footnote{Notar: $\Phi_{\widetilde{X}}\left(
\widetilde{\omega} \right) = \Phi_X\left( \begin{bmatrix} \widetilde{\omega} &
0 \end{bmatrix}^t \right)$ \ o \ $\Phi_X(\omega) = e^{\imath \, n \, \omega_k}
\Phi_{\widetilde{X}}\left( \begin{bmatrix} \omega_1 - \omega_k & \cdots &
\omega_{k-1} - \omega_k \end{bmatrix}^t
\right)$.\label{Foot:MP:MultinomialCaracteristica}} & $\displaystyle
\Phi_X(\omega) = \left( p^t e^{\imath \omega} \right)^n$
\end{caracteristicas}

% Momentos & $ \Esp\left[ X^k \right] = ??\\[2mm]
% Momento factorial & $\Esp\left[ (X)_k \right] = 
% \frac{(r+k-1)!}{(r-1)!} \left( \frac{p}{1-p} \right)^k$\\[2mm]
% Modo $\left\lfloor (n+1) p \right\rfloor$
% Mediana $\left\lfloor n p \right\rfloor$ o $\left\lceil n p \right\rceil
% CDF	$I_{1-p}(n-k,k+1)$ regularized incomplete beta function

Su masa  de probabilidad  y funci\'on de  repartici\'on son representadas  en la
figura Fig.~\ref{Fig:MP:Multinomial}.
%
\begin{figure}[h!]
\begin{center} \begin{tikzpicture}[scale=.8]
\shorthandoff{>}
%
%
\pgfmathsetmacro{\n}{5};% numeros para la multinomial
\pgfmathsetmacro{\dec}{.5};% shitf para dibujar las marginales
%
% Ejemplo [6 5 4]/15
\begin{scope}
%
\pgfmathsetmacro{\pu}{2/5};% p_1
\pgfmathsetmacro{\pd}{1/3};% p_2
\pgfmathsetmacro{\qu}{floor((\n+1)*\pu)};% modo de la binomial 1
\pgfmathsetmacro{\qd}{floor((\n+1)*\pd)};% modo de la binomial 2
\pgfmathsetmacro{\mau}{factorial(\n)/factorial(\qu)/factorial(\n-\qu)*(\pu^\qu)*((1-\pu)^(\n-\qu))};% maximo de la binomial 1
\pgfmathsetmacro{\mad}{factorial(\n)/factorial(\qd)/factorial(\n-\qd)*(\pd^\qd)*((1-\pd)^(\n-\qd))};% maximo de la binomial 2
\pgfmathsetmacro{\ma}{max(\mau,\mad)};% maximo de ambas binomiales
%
\begin{axis}[
    colormap = {whiteblack}{color(0cm)  = (white);color(1cm) = (black)},
    width=.55\textwidth,
    view={35}{70},
    enlargelimits=false,
    xmin={-\dec},
    xmax={\n+\dec},
    ymin={-\dec},
    ymax={\n+\dec},
    zmax={1.1*\ma},
    color=black,
    xtick={0,...,\n},
    ytick={0,...,\n},
    xlabel=$x_1$,
    ylabel=$x_2$,
    zlabel=$p_{\widetilde{X}}$,
]
%
% Lineas
\foreach \mu in {0,...,\n} {
  %
  % lineas (m1,m2) abajo
  \addplot3 [domain={-\dec}:{\n+\dec},samples=2, samples y=0,color=black!10] (\mu,\x,0);
  \addplot3 [domain={-\dec}:{\n+\dec},samples=2, samples y=0,color=black!10] (\x,\mu,0);
}
%
\pgfmathsetmacro{\bu}{(1-\pu-\pd)^\n};% coeficiente binomial por la probabilidad p1
\pgfmathsetmacro{\bd}{\bu};% coeficiente binomial por la probabilidad p2
%
\pgfmathsetmacro{\bmu}{(1-\pu)^\n};% lo mismo para la marginale 1
\pgfmathsetmacro{\bmd}{(1-\pd)^\n};% lo mismo para la marginale 2
%
\foreach \mu in {0,...,\n} {
  \foreach \md in {0,...,\n} {
    \ifnum \numexpr\mu+\md < \numexpr\n+1
      \addplot3 [dotted,domain=0:\bd,samples=2, samples y=0,color=black]
      (\mu,\md,\x)  node[scale=.85]{$\bullet$};
      %
      \pgfmathsetmacro{\bld}{\bd*\pd*(\n-\md)/((\md+1)*(1-\pu-\pd))};
      \global\let\bd\bld;% proba en m2 (m1 fijo) actualizado
    \fi
  }
  %
  % Marginales
  \addplot3 [dotted,domain=0:\bmu,samples=2, samples y=0,color=black]
  (\mu,{\n+\dec},\x)  node[scale=.55]{$\bullet$};
  \addplot3 [dotted,domain=0:\bmd,samples=2, samples y=0,color=black]
  ({-\dec},\mu,\x)  node[scale=.55]{$\bullet$};
  %
  \pgfmathsetmacro{\blu}{\bu*\pu*(\n-\mu)/((\mu+1)*(1-\pu-\pd))};
  \global\let\bu\blu;\global\let\bd\blu;% proba inicial en m1 actualizada
  %
  % lo mismo para cada marginal
  \pgfmathsetmacro{\blmu}{\bmu*\pu*(\n-\mu)/((\mu+1)*(1-\pu))};
  \global\let\bmu\blmu;% proba 1 actualizada
  \pgfmathsetmacro{\blmd}{\bmd*\pd*(\n-\mu)/((\mu+1)*(1-\pd))};
  \global\let\bmd\blmd;% proba 2 actualizada
}
%
\node at (axis cs:{3*\n/4},{\n+\dec},{\mau/2})[right]{$p_{X_1}$};
\node at (axis cs:{-\dec},{3*\n/4},{\mad/2})[above]{$p_{X_2}$};
%
\end{axis}
\node at ({3*\n/4},-1)[scale=.9]{(a)};
\end{scope}
%
%
% Ejemplo [1 1 1]/3
\begin{scope}[xshift = 11cm]
%
\pgfmathsetmacro{\pu}{1/3};% p_1
\pgfmathsetmacro{\pd}{1/2};% p_2
\pgfmathsetmacro{\qu}{floor((\n+1)*\pu)};% modo de la binomial 1
\pgfmathsetmacro{\qd}{floor((\n+1)*\pd)};% modo de la binomial 2
\pgfmathsetmacro{\mau}{factorial(\n)/factorial(\qu)/factorial(\n-\qu)*(\pu^\qu)*((1-\pu)^(\n-\qu))};% maximo de la binomial 1
\pgfmathsetmacro{\mad}{factorial(\n)/factorial(\qd)/factorial(\n-\qd)*(\pd^\qd)*((1-\pd)^(\n-\qd))};% maximo de la binomial 2
\pgfmathsetmacro{\ma}{max(\mau,\mad)};% maximo de ambas binomiales
%
\begin{axis}[
    colormap = {whiteblack}{color(0cm)  = (white);color(1cm) = (black)},
    width=.55\textwidth,
    view={35}{70},
    enlargelimits=false,
    xmin={-\dec},
    xmax={\n+\dec},
    ymin={-\dec},
    ymax={\n+\dec},
    zmax={1.1*\ma},
    color=black,
    xtick={0,...,\n},
    ytick={0,...,\n},
    xlabel=$x_1$,
    ylabel=$x_2$,
    zlabel=$p_{\widetilde{X}}$,
]
%
% Lineas
\foreach \mu in {0,...,\n} {
  %
  % lineas (m1,m2) abajo
  \addplot3 [domain={-\dec}:{\n+\dec},samples=2, samples y=0,color=black!10] (\mu,\x,0);
  \addplot3 [domain={-\dec}:{\n+\dec},samples=2, samples y=0,color=black!10] (\x,\mu,0);
}
%
\pgfmathsetmacro{\bu}{(1-\pu-\pd)^\n};% coeficiente binomial por la probabilidad p1
\pgfmathsetmacro{\bd}{\bu};% coeficiente binomial por la probabilidad p2
%
\pgfmathsetmacro{\bmu}{(1-\pu)^\n};% lo mismo para la marginale 1
\pgfmathsetmacro{\bmd}{(1-\pd)^\n};% lo mismo para la marginale 2
%
\foreach \mu in {0,...,\n} {
  \foreach \md in {0,...,\n} {
    \ifnum \numexpr\mu+\md < \numexpr\n+1
      \addplot3 [dotted,domain=0:\bd,samples=2, samples y=0,color=black]
      (\mu,\md,\x)  node[scale=.85]{$\bullet$};
      %
      \pgfmathsetmacro{\bld}{\bd*\pd*(\n-\md)/((\md+1)*(1-\pu-\pd))};
      \global\let\bd\bld;% proba en m2 (m1 fijo) actualizado
    \fi
  }
  %
  % Marginales
  \addplot3 [dotted,domain=0:\bmu,samples=2, samples y=0,color=black]
  (\mu,{\n+\dec},\x)  node[scale=.55]{$\bullet$};
  \addplot3 [dotted,domain=0:\bmd,samples=2, samples y=0,color=black]
  ({-\dec},\mu,\x)  node[scale=.55]{$\bullet$};
  %
  \pgfmathsetmacro{\blu}{\bu*\pu*(\n-\mu)/((\mu+1)*(1-\pu-\pd))};
  \global\let\bu\blu;\global\let\bd\blu;% proba inicial en m1 actualizada
  %
  % lo mismo para cada marginal
  \pgfmathsetmacro{\blmu}{\bmu*\pu*(\n-\mu)/((\mu+1)*(1-\pu))};
  \global\let\bmu\blmu;% proba 1 actualizada
  \pgfmathsetmacro{\blmd}{\bmd*\pd*(\n-\mu)/((\mu+1)*(1-\pd))};
  \global\let\bmd\blmd;% proba 2 actualizada
}
%
\node at (axis cs:{3*\n/4},{\n+\dec},{\mau/2})[right]{$p_{X_1}$};
\node at (axis cs:{-\dec},{3*\n/4},{\mad/2})[above]{$p_{X_2}$};
\end{axis}
\node at ({3*\n/4},-1)[scale=.9]{(b)};
\end{scope}
%
\end{tikzpicture} \end{center}
%
\leyenda{Ilustraci\'on de una distribuci\'on  de probabilidad multinomial para \
  $k   =  3$   \   del   vector  \   $(k-1)$-dimensional   \  $\widetilde{X}   =
  \protect\begin{bmatrix}   X_1  &  X_2   \protect\end{bmatrix}^t$  \   ($X_3  =
  1-X_1-X_2$)  \ con  las marginales  \ $p_{X_1},  \: p_{X_2}$  \ (ver  notas de
  pie~\ref{Foot:MP:MultinomialDominio}   y~\ref{Foot:MP:MultinomialMasa}).    Es
  dibujada solamente  la distribuci\'on sobre  $\X$, siendo esta nula  afuera de
  $\X$.  Los parametros son \ $n = 5$ \ y \ $p = \protect\begin{bmatrix} \frac25
    &    \frac13    &   \frac4{15}    \protect\end{bmatrix}^t$    (a),   $p    =
  \protect\begin{bmatrix} \frac13  & \frac12 &  \frac16 \protect\end{bmatrix}^t$
  (b).}
\label{Fig:MP:Multinomial}
\end{figure}


Notar: cuando $p = \un_i$, la variable es cierta $X = n \un_i$.

\SZ{Otros ilustraciones para otros $n, p$?}


Vectores  de  distribuci\'on  multinomial  tienen  una  propiedade  notable  con
respecto a una permutaci\'on de variable, parecidas a la de la binomial:
%
\begin{lema}[Efecto de una permutaci\'on]\label{Lem:MP:PermutacionMultinomial}
%
  Sea \ $X \, \sim \, \M(n,p), \: p \in \Simp{k-1}$ \ y \ $\Pi \in \perm_k(\Rset)$ \
  matriz \ de permutaci\'on. Entonces
  %
  \[
  \Pi X \, \sim \, \M\left( n ,  \Pi p \right)
  \]
  %
\end{lema}
%
\begin{proof}
  El  resultado  es  inmediato  saliendo  de  la  funci\'on  caracter\'istica  y
  aplicando  el  teorema~\ref{Teo:MP:PropiedadesFuncionCaracteristica} (recordar
  que $\Pi^{-1} = \Pi^t$). M\'as directamente, notando la permutation \ $\sigma$
  \ tal que  \ $\Pi = \sum_{i=1}^k \un_i \un_{\sigma(i)}^t$, se  puede ver que \
  $\displaystyle  P(\Pi X =  x) =  P(X =  \Pi^{-1} x)  = \frac{n!}{\prod_{i=1}^k
    x_{\sigma^{-1}(i)}!}       \prod_{i=1}^k      p_i^{x_{\sigma^{-1}(i)}}     =
  \frac{n!}{\prod_{i=1}^k x_i!}  \prod_{i=1}^k p_{\sigma(i)}^{x_i}$ \ por cambio
  de indices.
\end{proof}
%
Adem\'as ley  multinomial exhibe una stabilidad remplazando  dos componentes por
su suma:
%
\begin{lema}[Stabilidad por agregaci\'on]\label{Lem:MP:StabAgregacionMultinomial}
%
  Sea  \ $X =  \begin{bmatrix} X_1  & \cdots  & X_k  \end{bmatrix}^t \,  \sim \,
  \M(n,p), \:  p \in \Simp{k-1}$ \ y  \ $G^{(i,j)}$ \ matriz  de agrupaci\'on de
  las $(i,j)$-\'esima componentes (ver notaciones). Entonces,
  %
  \[
  G^{(i,j)} X \, \sim \, \M\left( n , G^{(i,j)} p \right)  
  \]
  %
\end{lema}
%
Este resultado es  intuitivo en el hecho  de que vuelve a agrupar  los estados \
$i$ \ e \ $j$ \ en un estado, que tiene entonces la probabilidad \ $p_i + p_j$ \
de aparecer.
%
\begin{proof}
  Suponemos $i <  j$ (el otro caso  se recupera por simetr\'ia). A  partir de la
  funci\'on                 caracter\'istica                 y                el
  teorema~\ref{Teo:MP:PropiedadesFuncionCaracteristica} se tiene,
  %
  \begin{eqnarray*}
  \forall \: \omega \in \Rset^{k-1}, \quad \Phi_{G^{(i,j)} X}(\omega) & = &
  \Phi_X\left( G^{(i,j) \, t} \omega \right)\\[2mm]
  %
  & = & \left( \sum_{l=1}^k p_l \, e^{\imath \, \left( G^{(i,j) \, t} \omega \right)_l } \right)^n
  \end{eqnarray*}
  %
  Ahora,  se nota  que \  $G^{(i,j) \,  t} \omega  = \begin{bmatrix}  \omega_1 &
    \cdots    &   \omega_{j-1}   &    \omega_i   &    \omega_{j+1}   &    \cdots   &
    \omega_{k-1} \end{bmatrix}^t$, entonces
  %
  \begin{eqnarray*}
  \forall \: \omega \in \Rset^{k-1}, \quad \Phi_{G^{(i,j)} X}(\omega) & = &
  \left( \sum_{l=1, l \ne j}^k p_l \, e^{\imath \, \omega_l} + p_j \, e^{\imath \,
  \omega_i } \right)^n\\[2mm]
  %
  & = & \left( \sum_{l=1, l \ne i, l \ne j}^k p_l \, e^{\imath \, \omega_l} +
  (p_i+p_j) \, e^{\imath \, \omega_i } \right)^n
  %
  \end{eqnarray*}
  %
  lo que  cierra la  prueba. Se puede  tener un  enfoque m\'as directo,  con los
  mismos         pasos        que         en        la         prueba        del
  lema~\ref{Lem:MP:StabAgregacionHipergeomMulti}    tratando     de    la    ley
  hipergeometrica multivaluada.
\end{proof}

De este lema, aplicado de manera recursiva, se obtiene los corolarios siguientes:
%
\begin{corolario}\label{Cor:MP:MarginalMultinomial}
%
  Sea  \ $X  \,  \sim \,  \M(n,p)$, entonces  \  $\displaystyle X_i  \, \sim  \,
  \B(n,p_i)$.
\end{corolario}


Al final, por una analisis  combinatorial, se muestra sencillamente un resultado
similar al de la binomial como suma de Bernoulli independientes:
%
\begin{lema}\label{Lem:MultinomialSumaMultiBernoulli}
%
  Sean \ $U_i, \quad i = 1, \ldots ,  n, \: j = 1, \ldots , n$ \ discretas sobre
  $\U  = \{  1 \;  \ldots \;  k \}$  de masa  de probabilidad  $p_{U_i} =  p \in
  \Delta_{k-1}$, independientes, y $X_i = \un_{U_i} \in \Rset^k$. Entonces
  %
  \[
  \sum_{i=1}^n X_i \, \sim \, \M(n,p)
  \]
\end{lema}

% --------------------------------- Geometrica
\subsubseccion{Ley Geom\'etrica}
\label{Sssec:MP:Geometrica}

Se  denota  \  $X \,  \sim  \,  \G(p)$  \  con \  $p  \in  (0  \;  1]$ \  y  sus
caracter\'isticas son las siguientes:

\begin{caracteristicas}
%
Dominio de definici\'on & $\X = \Nset^*$\\[2mm]
\hline
%
Parametro & $p \in (0 \; 1]$\\[2mm]
\hline
%
Distribuci\'on  de  probabilidad &  $\displaystyle  p_X(k)  =  (1-p)^{k-1} p$  \
(convenci\'on $0^0 = 1$)\\[2mm]
\hline
%
Promedio & $m_X = \frac1p$\\[2mm]
\hline
%
Varianza & $\displaystyle \sigma_X^2 = \frac{1-p}{p^2}$\\[2mm]
\hline
%
\modif{Sesgo} & $\displaystyle \gamma_X = \frac{2-p}{\sqrt{1-p}}$\\[2mm]
\hline
%
Curtosis por exceso & $\displaystyle \widebar{\kappa}_X = \frac{6 - 6 \, p + p^2}{1-p}$\\[2mm]
\hline
%
Generadora de  probabilidad & $\displaystyle  G_X(z) = \frac{p z}{1-(1-p)  z}$ \
para \ $|z| < \frac1{1-p}$\\[2mm]
\hline
%
Generadora de  momentos & $\displaystyle M_X(u)  = \frac{p \, e^u}{1  - (1-p) \,
e^u}$ \ para \ $\real{u} < - \ln(1-p)$\\[2mm]
\hline
%
Funci\'on caracter\'istica  & $\displaystyle \Phi_X(\omega)  = \frac{p \, e^{\imath
\omega}}{1 - (1-p) \, e^{\imath \omega}}$
\end{caracteristicas}

% Momentos & $ \Esp\left[ X^k \right] = ?$\\[2mm]
% Momento factorial & $\Esp\left[ (X)_k \right] = \frac{p^{k-1} k!}{(1-p)^k}$\\[2mm]
% Modo 1
% Mediana $\left\lceil \frac{-1}{\log_2(1-p)} \right\rceil$ 
% CDF	$1-(1-p)^k$

Su masa  de probabilidad  y funci\'on de  repartici\'on son representadas  en la
figura Fig.~\ref{Fig:MP:Geometrica}.
%
\begin{figure}[h!]
\begin{center} \begin{tikzpicture}%[scale=.9]
\shorthandoff{>}
%
\pgfmathsetmacro{\sx}{.75};% x-scaling
\pgfmathsetmacro{\r}{.05};% radius arc non continuity F_X
\pgfmathsetmacro{\p}{1/3};% probabilidad p
\pgfmathsetmacro{\n}{7};% k mas grande del plot (k in Nset^*)
%
% masa
\begin{scope}
%
\pgfmathsetmacro{\sy}{2.5/\p};% y-scaling 
\draw[>=stealth,->] (-.25,0)--({\sx*\n+.75},0) node[right]{\small $x$};
\draw[>=stealth,->] (0,-.15)--(0,{\sy*\p+.25}) node[above]{\small $p_X$};
%
\pgfmathsetmacro{\pr}{\p};% probabilidad
%
\foreach \k in {1,...,\n} {
\draw ({\k*\sx},0)--({\k*\sx},-.1) node[below,scale=.7]{$\k$};
\draw[dotted] ({\k*\sx},0)--({\k*\sx},{\sy*\pr}) node[scale=.7]{$\bullet$};
%
\pgfmathsetmacro{\prl}{\pr*(1-\p)};\global\let\pr\prl;% proba actualizado
}
\draw (0,0)--(0,-.1) node[below,scale=.7]{$0$};
\draw ({(\n+.5)*\sx},-.2) node[below,scale=.7]{$\ldots$};
\draw ({(\n+.5)*\sx},{(\pr/(1-\p)/2*\sy}) node[scale=.7]{$\cdots$};
\draw (0,{\p*\sy})--(-.1,{\p*\sy}) node[left,scale=.7]{$p$};
\draw (0,{\p*(1-\p)*\sy})--(-.1,{\p*(1-\p)*\sy}) node[left,scale=.7]{$p \, (1-p)$};
\draw (-.5,{\p*(1-\p)/2*\sy}) node[left,scale=.7]{$\vdots$};
%
\node at ({(\sx*\n+.75)/2},-1) [scale=.9]{(a)};
\end{scope}
%
%
% reparticion
\begin{scope}[xshift=8.5cm]
%
\pgfmathsetmacro{\sy}{2.5};% y-scaling 
%
\draw[>=stealth,->] (-.6,0)--({\sx*\n+.75},0) node[right]{\small $x$};
\draw[>=stealth,->] (0,-.15)--(0,{\sy+.25}) node[above]{\small $F_X$};
%
\pgfmathsetmacro{\pr}{\p};% probabilidad
\pgfmathsetmacro{\c}{\p};% cumulativa
%
% cumulativa x < 1
\draw (0,0)--(0,-.1) node[below,scale=.7]{$0$};
\draw (\sx,0)--(\sx,-.1) node[below,scale=.7]{$1$};
\draw[thick] (-.5,0)--(\sx,0);
\draw ({\sx+\r},\r) arc (90:270:\r);
%
% cumulativa x de 1 a n
\foreach \k in {2,...,\n} {
\draw ({\k*\sx},0)--({\k*\sx},-.1) node[below,scale=.7]{$\k$};
\draw[thick]({(\k-1)*\sx},{\sy*\c}) node[scale=.7]{$\bullet$}--({\k*\sx},{\sy*\c});
\draw ({\k*\sx+\r},{\sy*\c+\r}) arc (90:270:\r);
\draw[dotted] ({(\k-1)*\sx},{(\c-\pr)*\sy})--({(\k-1)*\sx},{\c*\sy});
%
\pgfmathsetmacro{\prl}{\pr*(1-\p)};\global\let\pr\prl;% proba actualizado
\pgfmathsetmacro{\cl}{\c+\pr};\global\let\c\cl;% cumulativa actualizada
}
%
% cumulativa x > n
\draw ({(\n+.5)*\sx},-.2) node[below,scale=.7]{$\ldots$};
\draw ({(\n+.5)*\sx},{((\c+1)/2*\sy}) node[scale=.7]{$\cdots$};
\draw (0,{\p*\sy})--(-.1,{\p*\sy}) node[left,scale=.7]{$p$};
\draw (0,{\p*(2-\p)*\sy})--(-.1,{\p*(2-\p)*\sy}) node[left,scale=.7]{$p \, (2-p)$};
\draw (-.3,{(1+\p*(2-\p))/2*\sy}) node[left,scale=.7]{$\vdots$};
\draw (0,\sy)--(-.1,\sy) node[left,scale=.7]{$1$};
%
\node at ({(\sx*\n+.75)/2},-1) [scale=.9]{(a)};
\end{scope}
%
\end{tikzpicture} \end{center}
%
\leyenda{Ilustraci\'on de una distribuci\'on de probabilidad Geom\'etrica (a), y
  la funci\'on de repartici\'on asociada (b), con $p = \frac13$.}
\label{Fig:MP:Geometrica}
\end{figure}
\SZ{Otros ilustraciones para otros $p$?}

Esta distribuci\'on  aparece en el conteo  de conteo de une  repetici\'on de una
experiencia de maneja  independiente hasta que occure un  evento de probabilidad
$p$; por ejemplo  el n\'umero de tiro de un dado  equilibriado hasta que occurre
un ``6'' sigue una ley geom\'etrica de parametro $p = \frac16$.

Nota que cuando \  $p =  1$ \ la variable es cierta \  $X = 1$.   


\SZ{?`Que propiedad mas?}

% --------------------------------- Poisson
\subsubseccion{Ley de Poisson}
\label{Sssec:MP:Poisson}

Esta  ley fue  introducida por  Poisson en  1837 como  caso l\'imite  de  la ley
binomial     para     $n$     grande,      con     el     producto     $n     p$
fijo~\cite[Cap.~3]{Poi37},~\cite{Hal90, DavEdw01}.  Se  interes\'o Poisson en su
estudio  al  comportamentio  probabil\'istico   del  conteo  de  experiencia  de
Bernoulli bajo la hipotesis de independencia  (dando lugar a la ley binomial) en
ciencia humana, para una poblaci\'on  importante ($n$ grande), pero con un valor
promedio dado.  De hecho, se conoc\'ia esta ley, tambi\'en como caso l\'imite de
la  binomial,  por  lo  menos  desde  un  trabajo  de  de  Moivre  unas  decadas
antes~\cite{Moi10}.   Apareci\'o  tambi\'en   m\'as  tarde  en  muchos  procesos
f\'isicos, como el conteo de desintegraci\'on atomica por secundo en un material
radioactivo, o, (aproximadamente) a trav\'es del conteo de part\'iculas que caen
en una peque\~na  superficia, cuanto se tiran part\'iculas  uniformamente en una
grande superficia  en trabajos de  W. S. Gosset~\footnote{Fue connocido  bajo en
  nombre ``Student''; ver nota de pie~\ref{Foot:MP:Student}.}~\cite{Stu07}.

Se denota $X \,  \sim \, \P(\lambda)$ \ con \ $\lambda  \in \Rset_{0,+}$ \ llamada
{\em taza}, y sus caracter\'isticas son las siguientes:

\begin{caracteristicas}
%
Dominio de definici\'on & $\X = \Nset$\\[2mm]
\hline
%
Par\'ametro & $\lambda \in \Rset_{0,+}$\\[2mm]
\hline
%
Distribuci\'on  de  probabilidad   &  $\displaystyle  p_X(x)  =  \frac{\lambda^x
e^{-\lambda}}{x!}$\\[2mm]
\hline
%
Promedio & $ m_X = \lambda$\\[2mm]
\hline
%
Varianza & $\sigma_X^2 = \lambda$\\[2mm]
\hline
%
\modif{Asimetr\'ia} & $\displaystyle \gamma_X = \frac1{\sqrt\lambda}$\\[2mm]
\hline
%
Curtosis por exceso & $\displaystyle \widebar{\kappa}_X = \frac1\lambda$\\[2mm]
\hline
%
Generadora de probabilidad & $\displaystyle G_X(z) = e^{\lambda (z-1)}$ \quad para \
$z \in \Cset$\\[2mm]
\hline
%
Generadora  de momentos  & $\displaystyle  M_X(u) =  e^{\lambda \left(  e^u  - 1
\right)}$ \quad para \ $u \in \Cset$\\[2mm]
\hline
%
Funci\'on  caracter\'istica  &  $\displaystyle  \Phi_X(\omega) =  e^{\lambda  \,
\left( e^{\imath \omega} - 1 \right)}$
\end{caracteristicas}

% Momentos & $ \Esp\left[ X^k \right] = ?$\\[2mm]
% Momento factorial & $\Esp\left[ (X)_k \right] = \lambda^k$\\[2mm]
% modo \lfloor \lambda \rfloor 
% Mediana \approx \lfloor \lambda +1/3-0.02/\lambda \rfloor 
% CDF {\frac {\Gamma
% (\lfloor k+1\rfloor  ,\lambda )}{\lfloor k\rfloor !}} where  $\Gamma (x,y)$ is
% the upper incomplete gamma function,

Su masa  de probabilidad  y funci\'on de  repartici\'on son representadas  en la
figura Fig.~\ref{Fig:MP:Poisson}.
%
\begin{figure}[h!]
\begin{center} \begin{tikzpicture}%[scale=.9]
\shorthandoff{>}
%
\pgfmathsetmacro{\sx}{.75};% x-scaling
\pgfmathsetmacro{\r}{.05};% radius arc non continuity F_X
\pgfmathsetmacro{\l}{3};% lambda
\pgfmathsetmacro{\n}{7};% k mas grande del plot (k in Nset)
\pgfmathsetmacro{\q}{floor(\l)};% modo
\pgfmathsetmacro{\m}{(\l^\q)*exp(-\l)/factorial(\q)};% maximo
%
% masa
\begin{scope}
%
\pgfmathsetmacro{\sy}{2.75/\m};% y-scaling 
\draw[>=stealth,->] (-.25,0)--({\sx*\n+.75},0) node[right]{\small $x$};
\draw[>=stealth,->] (0,-.1)--(0,{\sy*\m+.25}) node[above]{\small $p_X$};
%
\pgfmathsetmacro{\pr}{exp(-\l)};% probabilidad
%
\foreach \k in {0,...,\n} {
\draw ({\k*\sx},0)--({\k*\sx},-.1) node[below,scale=.7]{\k};
\draw[dotted] ({\k*\sx},0)--({\k*\sx},{\sy*\pr}) node[scale=.7]{$\bullet$};
%
\pgfmathsetmacro{\prl}{\pr*\l/(\k+1)};\global\let\pr\prl;% proba actualizado
}
\draw ({(\n+.5)*\sx},-.2) node[below,scale=.7]{$\ldots$};
\draw ({(\n+.5)*\sx},{(\pr/\l*\n/2*\sy}) node[scale=.7]{$\cdots$};
\draw (0,{exp(-\l)*\sy})--(-.1,{exp(-\l)*\sy}) node[left,scale=.7]{$e^{-\lambda}$};
\draw (0,{\l*exp(-\l)*\sy})--(-.1,{\l*exp(-\l)*\sy}) node[left,scale=.7]{$\lambda e^{-\lambda}$};
\draw (0,{\l*\l*exp(-\l)/2*\sy})--(-.1,{\l*\l*exp(-\l)/2*\sy}) node[left,scale=.7]{$\frac{\lambda^2 e^{-\lambda}}{2}$};
%\draw (-.5,{\l*exp(-\l)/2*\sy}) node[left,scale=.7]{$\vdots$};
%
\end{scope}
%
%
% reparticion
\begin{scope}[xshift=8.5cm]
%
\pgfmathsetmacro{\sy}{2.75};% y-scaling 
%
\draw[>=stealth,->] (-.6,0)--({\sx*\n+.75},0) node[right]{\small $x$};
\draw[>=stealth,->] (0,-.1)--(0,{\sy+.25}) node[above]{\small $F_X$};
%
\pgfmathsetmacro{\pr}{exp(-\l)};% probabilidad
\pgfmathsetmacro{\c}{exp(-\l)};% cumulativa
%
% cumulativa x < 0
\draw (0,0)--(0,-.1) node[below,scale=.7]{0};
\draw[thick] (-.5,0)--(0,0);
\draw (\r,\r) arc (90:270:\r);
%
% cumulativa x de 0 a n
\foreach \k in {1,...,\n} {
\draw ({\k*\sx},0)--({\k*\sx},-.1) node[below,scale=.7]{\k};
\draw[thick]({(\k-1)*\sx},{\sy*\c}) node[scale=.7]{$\bullet$}--({\k*\sx},{\sy*\c});
\draw ({\k*\sx+\r},{\sy*\c+\r}) arc (90:270:\r);
\draw[dotted] ({(\k-1)*\sx},{(\c-\pr)*\sy})--({(\k-1)*\sx},{\c*\sy});
%
\pgfmathsetmacro{\prl}{\pr*\l/\k};\global\let\pr\prl;% proba actualizado
\pgfmathsetmacro{\cl}{\c+\pr};\global\let\c\cl;% cumulativa actualizada
}
%
% cumulativa x > n
\draw ({(\n+.5)*\sx},-.2) node[below,scale=.7]{$\ldots$};
\draw ({(\n+.5)*\sx},{((\c+1)/2*\sy}) node[scale=.7]{$\cdots$};
\draw (0,{exp(-\l)*\sy})--(-.1,{exp(-\l)*\sy}) node[left,scale=.7]{$e^{-\lambda}$};
\draw (0,{(1+\l)*exp(-\l)*\sy})--(-.1,{(1+\l)*exp(-\l)*\sy}) node[left,scale=.7]{$(1+\lambda) e^{-\lambda}$};
\draw (-.3,{(1+(1+\l+\l*\l/2)*exp(-\l))/2*\sy}) node[left,scale=.7]{$\vdots$};
\draw (0,\sy)--(-.1,\sy) node[left,scale=.7]{\small $1$};
\end{scope}
%
\end{tikzpicture} \end{center}
%
\leyenda{Ilustraci\'on de  una distribuci\'on de probabilidad de  Poisson (a), y
  la funci\'on de repartici\'on asociada (b), con $\lambda = 3$.}
\label{Fig:MP:Poisson}
\end{figure}

\SZ{Otras ilustraciones para otros $\lambda$?}

Ad\'emas, se muestra  sencillamente usando la generadora de  probabilidad que
%
\begin{lema}[Stabilidad]
\label{Lem:MP:StabilidadPoisson}
%
  Sean  \  $X_i  \,  \sim  \,  \P(\lambda_i),  \quad  i  =  1,  \ldots  ,  n$  \
  independientes, entonces
  %
  \[
  \sum_{i=1}^n X_i \, \sim \, \P\left( \sum_{i=1}^n \lambda_i \right)
  \]
\end{lema}


Como lo hemos introducido, la ley de Poisson esta v\'inculada a la ley binomial, como caso l\'imite:
%
\begin{lema}[V\'inculo con la ley binomial]
\label{Lem:MP:VinvuloPoissonBinomial}
%
  Sean  \  $X_n  \,  \sim  \,  \B\left( n \, , \, \frac{\lambda}{n} \right)$  \
  con $\lambda > 0$ fijo, entonces
  %
  \[
  X_n \, \limitd{n \to \infty} \, X \, \sim \, \P(\lambda)
  \]
  %
  donde  \ $\limitd{}$ \  significa que  el l\'imite  es en  distribuci\'on (ver
  notaciones).
\end{lema}
\begin{proof}
  Se  sale  de la  forma  de  la distribuci\'on  binomial  y  de  la formula  de
  Stirling~\footnote{De hecho, esta  formula es probablemente debida previamente
    a  A.  De  Moivre~\cite{Moi33, Moi56,  Pea24,  Cam86, Dut91,  Dem33}, y  fue
    mejorada por  Stirling m\'as tarde. Fue  mejorada a\'un m\'as  por el famoso
    matem\'atico                          S.                           Ramanujan
    recientemente~\cite[\S~4.1]{AndBer13}.\label{Foot:MP:Stirling}}:            \
  $\log\Gamma(z) = \left( z - \frac12 \right) \log z - z + \frac12 \log(2 \pi) +
  o(1)$ \ en \ $z \to +\infty$~\cite{Sti30, AbrSte70, GraRyz15}.
\end{proof}

Aparece  que la  ley de  Poisson esta  v\'inculada tambi\'en  a la  ley binomial
negativa, tambi\'en como caso l\'imite:
%
\begin{lema}[V\'inculo con la binomial negativa]
\label{Lem:MP:VinvuloPoissonBinomialNegativa}
%
Sean \ $X_r \, \sim \, \B_-\left( \frac{\lambda}{r+\lambda} \, , \, r \right)$ \
con $\lambda > 0$ fijo, entonces
  %
  \[
  X_r \, \limitd{r \to \infty} \, X \, \sim \, \P(\lambda)
  \]
\end{lema}
\begin{proof}
  Se  sale de nuevo  la forma  de la  distribuci\'on binomial  negativa y  de la
  formula de Stirling para probarlo.
\end{proof}

M\'as all\'a  del contexto discreto, esta  ley esta tambi\'en  v\'inculada a ley
exponencial, por  el processo dicho de  Poisson.  Si eventos  pueden aparecer de
manera aleatoria  en el tiempo tal que,  entre dos eventos, el  tiempo sigue una
ley   exponencial  de   par\'ametro   $\lambda$,  y   que   estos  tiempos   son
independientes, entonces dado  un intervalo $T$ de tiempo,  el n\'umero de estos
eventos sigue una ley de Poisson de  par\'ametro $\lambda T$.  Lo vamos a ver en
el ejemplo de la ley exponencial m\'as adelante.

Al final, notar que  cuando $\lambda = 0$ la variable es  cierta $X = 0$ (usando
la convenci\'on $0^0 = 1$).


%%%%%%%%%%%%%%%%%%%%%%%%%%%%%%%%%%%%%%%%%%%%%%%%%%%%%%%%%%%%%%%%%%%%%%%%%%%%%%%%
\aver{ Estad\'istica  de los n\'umeros de ocupaci\'on  de niveles energ\'eticos:
  distribuciones    de    Maxwell--Boltzmann,     de    Fermi--Dirac,    y    de
  Bose--Einstein\newline Leyes de los grandes n\'umeros}
%%%%%%%%%%%%%%%%%%%%%%%%%%%%%%%%%%%%%%%%%%%%%%%%%%%%%%%%%%%%%%%%%%%%%%%%%%%%%%%%


% ================================= Variables continuas

\subseccion{Distribuciones de variable continua}
\label{Ssec:MP:EjemplosDistribucionescontinuas}

\SZ{$\sigma \to 0$ caso cierto}


% --------------------------------- uniforme escalar
\subsubseccion{Distribuci\'on uniforme sobre un intervalo}
\label{Sssec:MP:UniformeContinua}

Se denota $X \, \sim \, \U([a \; b])$. Las caracter\'isticas de \ $X$ \ son las
siguientes:

\begin{caracteristicas}
%
Dominio de definici\'on & $\X = [a \; b]$\\[2mm]
\hline
%
Parametros & $(a,b) \in \Rset, \: b > a$\\[2mm]
\hline
%
Densidad de probabilidad & $p_X(x) = \frac{1}{b-a}$\\[2mm]
\hline
%
Promedio & $\displaystyle m_X = \frac{a+b}{2}$\\[2mm]
\hline
%
Varianza & $\displaystyle \sigma_X^2 = \frac{(b-a)^2}{12}$\\[2mm]
\hline
%
\modif{Sesgo} & $\gamma_X = 0$\\[2mm]
\hline
%
Curtosis por exceso & $\displaystyle \widebar{\kappa}_X = -\frac65$\\[2mm]
\hline
%
Generadora de momentos & $\displaystyle M_X(u) = \frac{ e^{b u} - e^{a u}}{u}$ \
para~\footnote{En el caso l\'imite \ $u \to  0$, \ $\lim_{u \to 0} \frac{ e^{b u}
- e^{a u}}{u} = b-a$, y similarmente para la funci\'on caracter\'istica}  \ $u \in \Cset^d$\\[2mm]
\hline
%
Funci\'on caracter\'istica & $\displaystyle  \Phi_X(\omega) = \frac{ e^{\imath a
\omega} - e^{\imath b \omega}}{\omega}$
\end{caracteristicas}

% Momentos & $ \Esp\left[ X^k \right] = p^k$\\[2mm]
% Momento factorial & $\Esp\left[ (X)_k \right] = ?$\\[2mm]
% Generadora de probabilidad & $G_X(z) = e^{\lambda (z-1)}$ \ para \ $z \in \Cset$\\[2mm]
% modo 0
% Mediana \ln(2)/\lambda
% CDF 1-e^{-\lambda x}

Obviamente, se puede escribir \ $X \, \egald  \, a + (b-a) U$ \ donde \ $\egald$
\ significa que la equalidad es en distribuci\'on (las variables tienen la misma
distribuci\'on de probabilidad), con \ \ $U \, \sim \, \U \left( [ 0 \; 1 ]
\right)$ \ llamada {\em uniforme estandar}.

La densidad de probabilidad y funci\'on de repartici\'on de la variable estandar
son representadas en la figura Fig.~\ref{Fig:MP:Uniformecontinua}.
%
\begin{figure}[h!]
\begin{center} \input{TIKZ_MP/UniformeContinua} \end{center}
% 
\leyenda{Ilustraci\'on  de  una densidad  de  probabilidad  uniforme  (a), y  la
funci\'on de repartici\'on asociada (b).}
\label{Fig:MP:Uniformecontinua}
\end{figure}

De manera  general, para  cualquier ensemble $\D  \subset \Rset^d$ de  volumen \
$|\D|$ \,  la variable uniforma sobre $\D$  tiene la densidad con  respecto a la
medida  ``natural'' sobre  $\D$  (Lebesque, discreta,\ldots)  constante sobre  \
$\D$,
%
\[
p_X(x) = \frac{1}{|\D|} \un_{\D}(x)
\]
%
La media va a ser el centro de gravedad de $\D$.

%Cuando $\lambda \to +\infty$ la variable tiende a una variable cierta $X = 0$.
\SZ{Esta distribuci\'on aparece..., propiedades}
% en  el conteo  de conteo  de  une repetici\'on  de una  experiencia de  maneja
% independiente hasta que  occure un evento de probabilidad  $p$; por ejemplo el
% n\'umero de tiro de un dado  equilibriado hasta que occurre un ``6'' sigue una
% ley geometrica de parametro $p = \frac16$.

% --------------------------------- Exponencial
\subsubseccion{Distribuci\'on exponencial}
\label{Sssec:MP:Exponencial}

Se denota $X \,  \sim \, \E(\lambda)$ \ con \ $\lambda  \in \Rset_+^*$ \ llamada
{\em  taza}  (inversa  de  {\em   escala}),  y  sus  caracter\'isticas  son  las
siguientes:

\begin{caracteristicas}
%
Dominio de definici\'on & $\X = \Rset_+$\\[2mm]
\hline
%
Parametro & $\lambda \in \Rset_+^*$\\[2mm]
\hline
%
Densidad  de probabilidad &  $\displaystyle p_X(x)  = \lambda  e^{-\lambda x}$\\[2mm]
\hline
%
Promedio & $\displaystyle m_X = \frac1\lambda$\\[2mm]
\hline
%
Varianza & $\displaystyle \sigma_X^2 = \frac1{\lambda^2}$\\[2mm]
\hline
%
Asimetr\'ia & $\gamma_X = 2$\\[2mm]
\hline
%
Curtosis por exceso & $\widebar{\kappa}_X = 6$\\[2mm]
\hline
%
Generadora de  momentos &  $\displaystyle M_X(u) =  \frac{\lambda}{\lambda-u}$ \
para \ $\real{u} < \lambda$\\[2mm]
\hline
%
Funci\'on     caracter\'istica     &     $\displaystyle     \Phi_X(\omega)     =
\frac{\lambda}{\lambda - \imath \omega}$
\end{caracteristicas}

% Momentos & $ \Esp\left[ X^k \right] = p^k$\\[2mm]
% Momento factorial & $\Esp\left[ (X)_k \right] = ?$\\[2mm]
% Generadora de probabilidad & $G_X(z) = e^{\lambda (z-1)}$ \ para \ $z \in \Cset$\\[2mm]
% modo 0
% Mediana \ln(2)/\lambda
% CDF 1-e^{-\lambda x}

Su densidad  de probabilidad  y funci\'on de  repartici\'on son representadas  en la
figura Fig.~\ref{Fig:MP:Exponencial}.
%
\begin{figure}[h!]
\begin{center} \begin{tikzpicture}%[scale=.9]
\shorthandoff{>}
%
\pgfmathsetmacro{\sx}{.75};% x-scaling
\pgfmathsetmacro{\r}{.05};% radius arc non continuity F_X
\pgfmathsetmacro{\l}{1.5};% lambda
\pgfmathsetmacro{\mx}{6};% x maximo del plot
%
% densidad
\begin{scope}
%
\pgfmathsetmacro{\sy}{2.5/\l};% y-scaling 
\draw[>=stealth,->] ({-\sx-.25},0)--({\sx*\mx+.25},0) node[right]{\small $x$};
\draw[>=stealth,->] (0,-.1)--(0,{\sy*\l+.25}) node[above]{\small $p_X$};
%
\draw[thick] ({-\sx},0)--(0,0);
\draw (\r,\r) arc (90:270:\r);
\draw[dotted] (0,0)--(0,{\sy*\l}) node[scale=.4]{$\bullet$};
\draw[thick,domain=0:\mx,samples=100] plot ({\x*\sx},{\sy*\l*exp(-\l*\x)});
%
%\draw (0,{((1-\p)^\n)*\sy})--(-.1,{((1-\p)^\n)*\sy}) node[left,scale=.7]{$(1-p)^n$};
%\draw (0,{\n*\p*((1-\p)^(\n-1))*\sy})--(-.1,{\n*\p*((1-\p)^(\n-1))*\sy}) node[left,scale=.7]{$n p (1-p)^{n-1}$};
%
\end{scope}
%
%
% reparticion
\begin{scope}[xshift=8.5cm]
%
\pgfmathsetmacro{\sy}{2.5};% y-scaling 
%
\draw[>=stealth,->] (-.6,0)--({\sx*\mx+.25},0) node[right]{\small $x$};
\draw[>=stealth,->] (0,-.1)--(0,{\sy+.25}) node[above]{\small $F_X$};
%
% cumulativa
\draw[thick,domain=0:\mx,samples=100] (-.5,0)--(0,0) plot({\x*\sx},{(1-exp(-\l*\x))*\sy});
%
%\draw (0,{((1-\p)^\n)*\sy})--(-.1,{((1-\p)^\n)*\sy}) node[left,scale=.7]{$(1-p)^n$};
%%\draw (0,{(\n*\p+1-\p)*((1-\p)^(\n-1))*\sy})--(-.1,{(\n*\p+1-\p)*((1-\p)^(\n-1))*\sy}) node[left,scale=.7]{$(1-p+np) (1-p)^{n-1}$};
%\draw (-.2,{((\n*\p+1-\p)*((1-\p)^(\n-1))+1)/2*\sy}) node[scale=.7]{$\vdots$};
%\draw (0,\sy)--(-.1,\sy) node[left,scale=.7]{\small $1$};
\end{scope}
%
\end{tikzpicture} \end{center}
% 
\leyenda{Ilustraci\'on  de una densidad  de probabilidad  exponencial (a),  y la
funci\'on de repartici\'on asociada (b), con $\lambda = 1.5$.}
\label{Fig:MP:Exponencial}
\end{figure}
\SZ{Poner escalas; Otros ilustraciones para otros $\lambda$?}

Cuando $\lambda \to +\infty$ la variable tiende a una variable cierta $X = 0$.
\SZ{Esta distribuci\'on aparece... Propiedades}
% en  el conteo  de conteo  de  une repetici\'on  de una  experiencia de  maneja
% independiente hasta que  occure un evento de probabilidad  $p$; por ejemplo el
% n\'umero de tiro de un dado  equilibriado hasta que occurre un ``6'' sigue una
% ley geometrica de parametro $p = \frac16$.

% --------------------------------- Gaussiana
\subsubseccion{Distribuci\'on normal o Gaussiana multivariada real}
\label{Sssec:MP:Gaussiana}

En el caso escalar,  esta ley parece aparecer por unas de  las primeras veces en
trabajos de de  Moivre como approximaci\'on de la ley  binomial para $n$ grande,
usando  la   formula  de  Stirling~\cite{Moi30,  Moi33,   Moi56,  Pea24,  Hal90,
  JohKot95:v1, Hal06}.  Se  puede ver tambi\'en el trabajo  de F.  Galton, quien
contruy\'o un  experimento, la caja dicha  de Galton, que ilustra  por una parte
como se puede obtener la ley  binomial como suma de Bernoulli, y la convergencia
a la  Gausiana~\cite[Figs.~7-9, p.~63]{Gal89} o~\cite[p.~38]{Pea20}.   Aparte de
Moivre, la ley  gausiana fue desarollado mucho por  los matem\'aticos como Gauss
en el estudio del movimiento  de planetas con perturbaciones (predicci\'on de la
trayectoria de C\'eres)~\cite{Gau09, Pea24, Hal06}, basado en trabajos de A.  M.
Legendre~\cite{Leg05, Hal06}, o Laplace  en mismo tipos de problema~\cite{Lap09,
  Lap09:Supp,  Lap20, Pea24,  Hal06}.  De  hecho, apoyandose  en trabajos  de de
Moivre,  la formaliz\'o  antes  y m\'as  claramente  Laplace, quien  revandic\'o
entonces  su partenidad  (ver por  ejemplo~\cite{Pea20}). Por  eso, esta  ley es
tambi\'en conocida como ley de Laplace-Gauss.

En el contexto multivariado, la extensi\'on natural de la ley binomial siendo al
ley multinomial, es sin sorpresa  que se introdujo la gausiana multivaluada como
approximaci\'on de la multinomial.  Este trabajo  es debido entre otros a J.  L.
Lagrange en  los a\~nos 1770, con  correcciones debido unas  decadas despu\'es a
A. de  Morgan~\cite{Mor38}. Pero apareci\'o  antes en el caso  bidimensional, en
particular  a  trav\'es  del  estudio  del coeficiente  de  correlaci\'on  entre
variables   aleatorias  (ver   por  ejemplo   trabajos   de  Galton~\cite{Gal77,
  Gal77:Nature, Pea20}).

A pesar de que parece menos natural en la modelisaci\'on de fenomenos aleatorio,
la ley  gausiana es  seguramente unas de  las m\'as importante  en probabilidad,
sino que  la m\'as importante y la  m\'as expendida en la  naturaleza. Eso viene
sin duda del teorema del l\'imite  centrale. En dos palabras, cuando se suman un
numero  importante  de  variables  aleatorias  (independientes,  de  misma  ley,
admitiendo  una varianza,  o con  menos restricciones~\cite[Cap.~11]{AthLah06}),
corectamente normalizado, esta suma tiende a una gausiana~\footnote{De hecho, la
  approximaci\'on de  la ley binomial por  una gausiana cuando $n$  es grande es
  una  caso particular del  teorema, siendo  la binomial  una suma  de Bernoulli
  independientes.}. En  la naturaleza,  se puede ver  el ruido  (se\~nales) como
suma de un n\'umero importante  de fuentes de ruido independientes, justificando
el  modelo  gausiano~\cite{Fel71, Cam86,  AshDol99,  JacPro03, AthLah06,  Ren07,
  Bil12}.      Ad\'emas,      como     lo     vamos     a      ver     en     el
capitulo~\ref{Cap:SZ:Informacion},  esta   ley  es  la   de  incerteza  m\'axima
(maximizando la entrop\'ia) teniendo  una dada varianza. Aparece naturalmente en
termod\'inamica    (gaz    perfecto,   con    un    n\'umero    muy   alto    de
particulas)~\cite{Max67, Bol96,  Bol98, Gib02, Jay65}. En  estimaci\'on, bajo la
hipotesis  gausiana,  los  estimadores  de  par\'ametros  minimizando  el  error
cuadratico  promedio son generalmente  lineal~\cite{Kay93, Rob07}.   Todas estas
consideraciones  dan a la  ley gausiana  un rol  central en  la teor\'ia  de las
probabilidades.

Se denota $X \, \sim \, \N(m,\Sigma)$ \  con \ $m \in \Rset^d$ \ y \ $\Sigma \in
P_d^+(\Rset)$ \  conjunto de las  matrices de \ $\M_{d,d}(Rset)$  \ s\'imetricas
definidas positivas. Las caracter\'isticas de la Gaussiana son las siguientes:

\begin{caracteristicas}
%
Dominio de definici\'on & $\X = \Rset^d$\\[2mm]
\hline
%
Par\'ametros & $m \in \Rset^d, \: \Sigma \in P_d^+(\Rset)$\\[2mm]
\hline
%
Densidad de probabilidad & $\displaystyle p_X(x) = \frac{1}{(2
\pi)^{\frac{d}{2}} \left| \Sigma \right|^{\frac12}} \, e^{-\frac12 (x-m)^t
\Sigma^{-1} (x-m)}$\\[2.5mm]
\hline
%
Promedio & $ m_X = m$\\[2mm]
\hline
%
Covarianza & $\Sigma_X = \Sigma$\\[2mm]
\hline
%
\modif{Sesgo} (caso escalar) & $\gamma_X = 0$\\[2mm]
\hline
%
Curtosis por exceso (caso escalar) & $\widebar{\kappa}_X = 0$\\[2mm]
\hline
%
Generadora de  momentos &  $\displaystyle M_X(u) =  e^{u^t \Sigma u + u^t m}$  \ para \  $u \in
\Cset^d$\\[2mm]
\hline
%
Funci\'on  caracter\'istica   &  $\displaystyle  \Phi_X(\omega)   =  e^{-\frac12
\omega^t \Sigma \omega + \imath \omega^t m}$
\end{caracteristicas}

% Momentos & $ \Esp\left[ X^k \right] = p^k$\\[2mm]
% Momento factorial & $\Esp\left[ (X)_k \right] = ?$\\[2mm]
% Generadora de probabilidad & $G_X(z) = e^{\lambda (z-1)}$ \ para \ $z \in \Cset$\\[2mm]
% modo 0
% Mediana 0

Nota: trivialmente, se puede escribir $X  \, \egald \, \Sigma^{\frac12} N + m$ \
con \ $N \, \sim \, \N(0,I)$ \  donde \ $N$ \ es dicha {\em Gausiana estandar} o
{\em centrada-normalizada}. Las caracter\'isticas de  \ $X$ \ son v\'inculadas a
las  de  \  $N$ \  (y  vice-versa)  por  transformaci\'on afine  (ver  secciones
anteriores).


La densidad de probabilidad gausiana y  la funci\'on de repartici\'on en el caso
escalar son  representadas en la figura Fig.~\ref{Fig:MP:Gaussiana}-(a)  y (b) y
una      densidad      en       un      contexto      bi-dimensional      figura
Fig.~\ref{Fig:MP:Gaussiana}(c).
%
\begin{figure}[h!]
\begin{center} \begin{tikzpicture}%[scale=.9]
\shorthandoff{>}
%
\pgfmathsetmacro{\sx}{.75};% x-scaling
\pgfmathsetmacro{\mx}{3.5};% x maximo del plot
%
% Approximation de la cdf gaussienne
\tikzset{declare function={
normcdf(\x)=1/(1 + exp(-0.07056*(\x)^3 - 1.5976*(\x)));
}}
% densidad
\begin{scope}
%
\pgfmathsetmacro{\sy}{2.5*sqrt(2*pi)};% y-scaling 
\draw[>=stealth,->] ({-\sx*\mx-.25},0)--({\sx*\mx+.25},0) node[right]{\small $x$};
\draw[>=stealth,->] (0,-.1)--(0,2.75) node[above]{\small $p_X$};
%
\draw[thick,domain=-\mx:\mx,samples=100] plot ({\x*\sx},{\sy*exp(-.5*\x*\x)/sqrt(2*pi)});
%
\draw (0,{\sy/sqrt(2*pi)})--(-.2,{\sy/sqrt(2*pi)}) node[left,scale=.7]{$\frac1{\sqrt{2 \pi}}$};
%
\end{scope}
%
%
% reparticion
\begin{scope}[xshift=8.5cm]
%
\pgfmathsetmacro{\sy}{2.5};% y-scaling 
%
\draw[>=stealth,->] ({-\sx*\mx-.25},0)--({\sx*\mx+.25},0) node[right]{\small $x$};
\draw[>=stealth,->] (0,-.1)--(0,{\sy+.25}) node[above]{\small $F_X$};
%
% cumulativa
\draw[thick,domain=-\mx:\mx,samples=100] plot({\x*\sx},{\sy*normcdf(\x)});
%
\draw (0,\sy)--(-.1,\sy) node[left,scale=.7]{$1$};
\end{scope}
%
\end{tikzpicture} \end{center}
% 
\leyenda{Ilustraci\'on  de  una   densidad  de  probabilidad  gaussiana  escalar
  estandar  (a), y la  funci\'on de  repartici\'on asociada  (b), as\'i  que una
  densidad  de probabilidad  gaussiana bi-dimensional  centrada y  de  matriz de
  covarianza \ $\Sigma_X = R(\theta)  \Delta^2 R(\theta)^t$ \ con \ $R(\theta) =
  \protect\begin{bmatrix}   \cos\theta  &   -  \sin\theta\\[2mm]   \sin\theta  &
    \cos\theta  \protect\end{bmatrix}$ \  matriz  de rotaci\'on  y  \ $\Delta  =
  \diag\left(\protect\begin{bmatrix}  1   &  a\protect\end{bmatrix}  \right)$  \
  matriz  de   cambio  de  escala,   y  sus  marginales   \  $X_1  \,   \sim  \,
  \N\left(0,\cos^2\theta  + a^2  \sin^2\theta \right)$  \ y  \ $X_2  \,  \sim \,
  \N\left(0,\sin^2\theta + a^2 \cos^2\theta  \right)$ \ (ver m\'as adelante). En
  la figura, $a = \frac14$ \ y \ $\theta = \frac{\pi}{6}$.}
\label{Fig:MP:Gaussiana}
\end{figure}

La gaussiana tiene un par de propiedades particulares:
%
\begin{teorema}[Stabilidad]
\label{Teo:MP:StabilidadGaussiana}
%
  Sean \ $A_i , i = 1,\ldots,n$ \  matrices de \ $\Rset^{d' \times d}, d' \le d$
  \ de rango lleno, $b_i \in \Rset^{d'}$ \ y \ $X_i \, \sim \, \N(m_i,\Sigma_i)$
  \ independientes, entonces
  %
  \[
  \sum_{i=1}^n \left(  A_i X_i  + b_i \right)  \, \sim \,  \N\left( \sum_{i=1}^n
    \left( m_i + b_i \right) \, , \, \sum_{i=1}^n A_i \Sigma_i A_i^t \right)
  \]
  % 
  En particular, cualquier combinaci\'on lineal  de los componentes de un vector
  Gaussiano da una gaussiana.  Reciprocamente, si cualquier combinaci\'on lineal
  de los componentes de un vector aleatorio sigue una ley gaussiana, entonces el
  vector es gaussiano.
\end{teorema}
%
\begin{proof}
  Este  resultato se proba  usando funci\'on  caracter\'istica de  la gaussiana,
  conjuntalmente al teorema~\ref{Teo:MP:PropiedadesFuncionCaracteristica}.
\end{proof}

%
\begin{teorema}[Independencia]
\label{Teo:MP:IndependenciaGaussiana}
%
  Sea   \   $X  \,   \sim   \,   \N(m,\Delta)$  \   con   \   $\Delta  =   \diag
  \left(  \begin{bmatrix}  \sigma_1^2  &  \cdots  &  \sigma_d^2  \end{bmatrix}^t
  \right)$   \  diagonal.   Entonces  las   componentes  \   $X_i  \,   \sim  \,
  \N(m_i,\sigma_i^2)$ \ son independientes.
\end{teorema}
%
\begin{proof}
  Este resultato se proba  trivialmente escribiendo la densidad de probabilidad,
  notando que se factorisa.
\end{proof}
%
Hemos visto que cuando un  vector tiene componentes independientes, la matriz de
covarianza  es   diagonal  (lema~\ref{Lem:MP:IndependenciaCov}),  pero   que  la
reciproca es falsa en general. El \'ultimo teorema muestra que la reciproca vale
en el caso gausiano.

\SZ{y mas}

\SZ{
% --------------------------------- Gaussiana complejas
\subsubseccion{Distribuci\'on normal o Gaussiana multivariada complejas}
\label{Sssec:MP:GaussianaComplejas}

Por definici\'on, un vector aleatorio complejo $d$-dimensional \ $Z = X + \imath
Y$ \  es gausiano significa que  el vector $2 d$-dimensional  \ $\widetilde{Z} =
\begin{bmatrix}  X^t &  Y^t \end{bmatrix}^t$  \ es  gausiano. Se  puede entonces
referirse  en el  caso de  vectores gausianos,  pero como  lo presentamos  en la
secci\'on~\ref{Ssec:MP:VAComplejos}, es frecuentemente m\'as comodo trabajar con
\  $Z$ \  en lugar  de \  $\widetilde{Z}$.  En particular,  en el  marco de  las
comunicaciones en  ingeneria, se  trabaja con modulaciones  dichas en fase  y en
cuadratura (se\~nal multiplicado  respectivamente por un seno y  un coseno) y en
lugar  de trabajar  con dos  componente se  considera una  modulaci\'on  con una
exponencial  compleja  y la  se\~nal/variable  compleja.  Se  puede por  ejemplo
referirse a~\cite{Lap17} (ver en particular el capitulo~24).

En el caso general, la gausiana real siendo completamente descritar por su media
y su matriz de covarianza, la  gausiana compleja va a ser completamente definida
por   la  media,   la  matriz   de  covarianza   y  la   pseudo-covarianza  (ver
Sec.~\ref{Sec:MP:VectoresComplejosMatricesAleatorias}  por las  relaciones entre
la  covarianza  de  \ $\widetilde{Z}$  \  y  estas  matrices).   $Z \,  \sim  \,
\C\N(m,\Sigma,\check{\Sigma})$  \   con  \  $m  \in  \Cset^d$,   \  $\Sigma  \in
P_d^+(\Cset)$ \  conjunto de  las matrices de  \ $\M_{d,d}(\Cset)$  \ hermiticas
definida  positivas, y  \  $\check{\Sigma}  \in S_d(\Cset)$  \  conjunto de  las
matrices  de  \  $\M_{d,d}(\Cset)$  \  symmetricas (ver  notaciones).   Un  caso
particular  aparece cuando  \ $Z$  \ es  propio en  torno de  \ $m$,  lo  que es
equivalente en el caso gausiano a tener \ $Z$ \ circular (ver m\'as adelante) en
torno de  \ $0$, dado cuando  \ $\check{\Sigma} =  0$: en este caso  usaremos la
misma notaci\'on,  $Z \,  \sim \, \C\N(m,\Sigma)$.  Las caracter\'isticas  de la
Gaussiana compleja son las siguientes~\cite{Lap07, Pic96, Bos95}:

\begin{caracteristicas}
%
Dominio de definici\'on & $\Z = \Cset^d$\\[2mm]
\hline
%
Par\'ametros & $m \in \Cset^d, \:\: \Sigma \in P_d^+(\Cset), \:\: \check{\Sigma}
\in S_d(\Cset)$\\[2mm]
\hline
%
Densidad de probabilidad & \\[1mm]
%
Caso general: & $\displaystyle p_Z(z) = \frac{1}{\pi^d \left| \Sigma
 \right|^{\frac12} \left| P \right|^{\frac12}} \: e^{- (z-m)^\dag P^{-1} (z-m) +
 \real{(z-m)^t R^t P^{-1} (z-m)}}$\vspace{2.5mm}\newline
 con~\footnote{En~\cite{Pic96} la expresi\'on es ligieramente diferente, pero se
 recupera usando la simetr\'ia \ $\check{\Sigma}^* =
 \check{\Sigma}^\dag$. Recordar que \ $\cdot^{-*} = \left( \cdot^* \right)^{-1}$
 \ (ver notaciones).} \ $P = \Sigma - \check{\Sigma} \Sigma^{-*}
 \check{\Sigma}^\dag, \quad R = \check{\Sigma}^\dag \Sigma^{-1}$.\\[2.5mm]
%
Caso circular: & $\displaystyle p_Z(z) = \frac{1}{\pi^d \left| \Sigma \right|}
 \: e^{- (z-m)^\dag \Sigma^{-1} (z-m)}$\\[2.5mm]
\hline
%
Promedio & $ m_Z = m$\\[2mm]
\hline
%
Covarianza & $\Sigma_Z = \Sigma$\\[2mm]
\hline
%
Pseudo-covarianza & $\check{\Sigma}_Z = \check{\Sigma}$\\[2mm]
\hline
%%
%Generadora de  momentos &  $\displaystyle M_X(u) =  e^{u^t \Sigma u + u^t m}$  \ para \  $u \in
%\Cset^d$\\[2mm]
%\hline
%%
Funci\'on caracter\'istica & \\[1mm]
%
Caso general: & $\displaystyle \Phi_Z(\omega) = e^{-\frac14
\omega^\dag \Sigma \omega - \frac14 \real{\omega^\dag \check{\Sigma} \omega^*} +
\imath \real{\omega^\dag m}}, \quad \omega \in \Cset^d$\\[1mm]
%
Caso circular: & $\displaystyle \Phi_Z(\omega) = e^{-\frac14
\omega^\dag \Sigma \omega +
\imath \real{\omega^\dag m}}, \quad \omega \in \Cset^d$
\end{caracteristicas}

Notar que  en el caso  escalar propio (circular),  la varianza de  \ $Z$ \  es \
$\sigma_Z^2 =  2 \sigma^2$. El  coefficiente 2  viene del hecho  de que \  $Z$ \
contiene dos componentes independientes de varianza $\sigma^2$.

Los vectores aleatorios complejos van a compartir las propiedades del caso real,
siendo equivalente  a un vector  $2d$-dimensional gausiano real. Primero,  en el
caso circular, se puede escribir $Z \, \egald \, \Sigma^{\frac12} N + m$ \ con \
$N \, \sim \, \C\N(0,I)$ \ donde \ $N$ \ es dicha {\em Gausiana estandar} o {\em
centrada-normalizada}. Las caracter\'isticas  de \ $X$ \ son  v\'inculadas a las
de \ $N$ \ (y vice-versa) por transformaci\'on afine (ver secciones anteriores).

Como  en el  caso  real, la  gausiana  es estable  por  combinaci\'on lineal  de
vectores independientes:
%
\begin{teorema}[Stabilidad]
\label{Teo:MP:StabilidadGaussianaCompleja}
%
  Sean \ $A_i , i = 1,\ldots,n$ \  matrices de \ $\Cset^{d' \times d}, d' \le d$ \
  de   rango  lleno,   $b_i   \in  \Cset^{d'}$   \   y  \   $Z_i   \,  \sim   \,
  \C\N(m_i,\Sigma_i,\check{\Sigma}_i)$ \ independientes, entonces
  %
  \[
  \sum_{i=1}^n \left( A_i  Z_i + b_i \right) \,  \sim \, \C\N\left( \sum_{i=1}^n
    \left( m_i + b_i \right) \, ,  \, \sum_{i=1}^n A_i \Sigma_i A_i^\dag \, , \,
    \sum_{i=1}^n A_i \check{\Sigma}_i A_i^t \right)
  \]
  % En  particular, cualquier  combinaci\'on  lineal de  los  componentes de  un
  vector  gaussiano  complejo da  una  gaussiana  compleja.  Reciprocamente,  si
  cualquier combinaci\'on lineal de los componentes de un vector aleatorio sigue
  una ley gaussiana compleja, entonces el vector es gaussiano complejo.
\end{teorema}

El teorema del l\'imite central y sus variantes se recuperan del caso real.
%
\begin{teorema}[Teorema del l\'imite central (caso complejo)]
\label{Teo:MP:CLTComplejo}
%
  Sea  \  $\{  Z_i \}_{i  \in  \Nset^*}$  \  una  serie de  vectores  aleatorios
  independientes, de misma  ley, y que admiten un promedio \  $m$, una matriz de
  covarianza   \   $\Sigma$   \    y   una   matriz   de   pseudo-covarianza   \
  $\check{\Sigma}$. Entonces
  %
  \[
  \frac{1}{\sqrt{n}}  \sum_{i=1}^m  \left( Z_i  -  m  \right)  \: \limitd{n  \to
    +\infty} \: W \sim \C\N\left( 0 , \Sigma , \check{\Sigma} \right)
  \]
  %
  donde  \ $\limitd{}$ \  significa que  el l\'imite  es en  distribuci\'on (ver
  notaciones).
\end{teorema}
%
No   lo   presentamos,  pero   se   transpone   sencillamente   el  teorema   de
Lindenberg-Feller~\ref{Teo:MP:LindenbergFeller} al caso complejo.

Al final, v\'imos en la  secci\'on~\ref{Ssec:MP:VAComplejos} que si un vector es
circular, entonces  su pseudo-covarianza es nula,  pero la reciproca  no vale en
general. Aparece que en el contexto gausiano tenemos la reciproca:
%

\begin{teorema}[Circularidad]\label{Teo:MP:CircularidadGaussiana}
%
Sea \ $Z \, \sim \, \C\N(m,\Sigma,\check{\Sigma})$.  Entonces,
  %
  \[
  Z \: \mbox{ circular  en torno de } \: m \qquad  \Longleftrightarrow \qquad Z \:
\mbox{ propio en torno de } \: m
  \]
\end{teorema}
%
\begin{proof}
  V\'imos     la    directa    en     la    secci\'on~\ref{Ssec:MP:VAComplejos},
  teorema~\ref{Teo:MP:Circularidad}.  Reciprocamente,  si \  $Z$ \ es  propio en
  torno de \ $m$, por definici\'on \ $\check{\Sigma} = 0$ \ y el resultado viene
  de la forma de  la funci\'on caracter\'istica por ejemplo: $\Phi_{Z-m}(\omega)
  = e^{-\frac14 \omega^\dag \Sigma \omega } = \Phi_{Z-m}\left( e^{\imath \theta}
    \omega \right) = \Phi_{e^{\imath \theta} (Z-m)}(\omega)$.
\end{proof}

}

% --------------------------------- Gamma
\subsubseccion{Distribuci\'on Gamma}
\label{Sssec:MP:Gamma}

Se  denota $X \,  \sim \,  \G(a,b)$ \  con \  $a \in  \Rset_+^*$ \  llamado {\em
parametro de  forma} \ y \  $b \in \Rset_+^*$  \ llamada {\em taza}  (inversa de
{\em escala}).  Las caracter\'isticas son:

\begin{caracteristicas}
%
Dominio de definici\'on & $\X = \Rset_+$\\[2mm]
\hline
%
Parametros & $a \in \Rset_+^*$ \ (forma), \: $b \in \Rset_+^*$ \ (taza)\\[2mm]
\hline
%
Densidad  de probabilidad  &  $\displaystyle p_X(x)  =  \frac{b^a \, x^{a-1} \,  e^{-b
x}}{\Gamma(a)}$\\[2mm]
\hline
%
Promedio & $\displaystyle m_X = \frac{a}{b}$\\[2mm]
\hline
%
Varianza & $\displaystyle \sigma_X^2 = \frac{a}{b^2}$\\[2mm]
\hline
%
Asimetr\'ia & $\displaystyle \gamma_X = \frac2{\sqrt{a}}$\\[2mm]
\hline
%
Curtosis por exceso & $\displaystyle \widebar{\kappa}_X = \frac6{a}$\\[2mm]
\hline
%
Generadora  de momentos  & $\displaystyle  M_X(u) =  \left( 1  - \frac{u}{b}
\right)^{-a}$ \ para \ $\real{u} < b$\\[2mm]
\hline
%
Funci\'on  caracter\'istica  &  $\displaystyle   \Phi_X(\omega)  =  \left(  1  -
\frac{ \imath \omega}{b} \right)^{-a}$
\end{caracteristicas}

% Momentos & $ \Esp\left[ X^k \right] = p^k$\\[2mm]
% Momento factorial & $\Esp\left[ (X)_k \right] = ?$\\[2mm]
% Generadora de probabilidad & $G_X(z) = e^{\lambda (z-1)}$ \ para \ $z \in \Cset$\\[2mm]
% modo max(a-1,0)
% Mediana no close ver inverse gamma

Nota: trivialmente, se puede escribir $X \,  \egald \, \frac{1}{b} G$ \ con \ $G
\, \sim \, \G(a,1)$  \ donde \ $G$ \ es estandardizada  o normalizada. De nuevo,
las  caracter\'isticas  de \  $X$  \  son  v\'inculadas a  las  de  \ $G$  \  (y
vice-versa) por transformaci\'on afine (ver secciones anteriores).

Una densidad de probabilidad gamma  y la funci\'on de repartici\'on asociada son
representadas en  la figura Fig.~\ref{Fig:MP:Gamma} para  varios $a$ \ y  \ $b =
1$.
%
\begin{figure}[h!]
\begin{center} \begin{tikzpicture}%[scale=.9]
\shorthandoff{>}
%
\pgfmathsetmacro{\sx}{.75};% x-scaling
\pgfmathsetmacro{\mx}{8};% x maximo del plot
%
% Approximation de la cdf gaussienne
\tikzset{declare function={
normcdf(\x)=1/(1 + exp(-0.07056*(\x)^3 - 1.5976*(\x)));
}}
%
% densidad
\begin{scope}
%
\pgfmathsetmacro{\sy}{2.5};% y-scaling 
\draw[>=stealth,->] (-.75,0)--({\sx*\mx+.25},0) node[right]{\small $x$};
\draw[>=stealth,->] (0,-.1)--(0,2.75) node[above]{\small $p_X$};
%
%\foreach \a in {1,...,3} {
\draw[thick] (-.5,0)--(0,0);
\draw[thick,dotted,domain=.175:\mx,samples=100] plot ({\x*\sx},{\sy*(\x^(-.5))*exp(-\x)/sqrt(pi)});
\draw[thick,dashed,domain=0:\mx,samples=100] plot ({\x*\sx},{\sy*exp(-\x)});
\draw[thick,dash dot,domain=0:\mx,samples=100] plot ({\x*\sx},{\sy*\x*exp(-\x)});
%\draw[thick,domain=0:\mx,samples=100] plot ({\x*\sx},{\sy*4*\x*sqrt(\x)*exp(-\x)/3/sqrt(pi)});
\draw[thick,domain=0:\mx,samples=100] plot ({\x*\sx},{\sy*\x*\x*exp(-\x)/2});
%}
%
\draw (0,\sy)--(-.1,\sy) node[left,scale=.7]{$1$};
\draw (0,{\sy*exp(-1)})--(-.1,{\sy*exp(-1)}) node[left,scale=.7]{$e^{-1}$};
\draw (0,{\sy*2*exp(-2)})--(-.1,{\sy*2*exp(-2)}) node[left,scale=.7]{$2 \, e^{-2}$};
\draw (\sx,0)--(\sx,-.1) node[below,scale=.7]{$1$};
\draw ({2*\sx},0)--({2*\sx},-.1) node[below,scale=.7]{$2$};
%
\end{scope}
%
%
% reparticion
\begin{scope}[xshift=8.5cm]
%
\pgfmathsetmacro{\sy}{2.5};% y-scaling 
%
\draw[>=stealth,->] (-.75,0)--({\sx*\mx+.25},0) node[right]{\small $x$};
\draw[>=stealth,->] (0,-.1)--(0,{\sy+.25}) node[above]{\small $F_X$};
%
% cumulativa
\draw[thick] (-.5,0)--(0,0);
\draw[thick,dotted,domain=0:\mx,samples=100] plot ({\x*\sx},{(2*normcdf(sqrt(2*\x))-1)*\sy});
\draw[thick,dashed,domain=0:\mx,samples=100] plot ({\x*\sx},{\sy*(1-exp(-\x))});
\draw[thick,dash dot,domain=0:\mx,samples=100] plot ({\x*\sx},{\sy*(1-(1+\x)*exp(-\x))});
\draw[thick,domain=0:\mx,samples=100] plot ({\x*\sx},{\sy*(1-(1+\x+\x*\x/2)*exp(-\x))});
% plot({\x*\sx},{\sy*normcdf(\x)});
%
\draw (0,\sy)--(-.1,\sy) node[left,scale=.7]{$1$};
\end{scope}
%
\end{tikzpicture} \end{center}
%
\leyenda{Ilustraci\'on de una densidad de probabilidad gamma (a), y la funci\'on
de  repartici\'on asociada  (b).   $b  = 1$  \  y \  $a  = 0.5$  (linea
punteada), $1$ (linea mixta), $2$ (linea guionada) y $3$ (linea llena).}
\label{Fig:MP:Gamma}
\end{figure}

Nota  que para  \  $X \,  \sim  \, \G(1,b)$  \ es  una  variable exponencial  de
parametro \  $b$, \ie \ $X \,  \sim \, \E(b)$. Cuando  \ $a < 1$,  la densidad \
$p_X$  \ diverge  para \  $x  \to 0$  \ (divergencia  integrable). Adem\'as,  se
muestra tambi\'en sencillamente con las funciones caracter\'isticas que:
%
\begin{lema}[Stabilidad]
\label{Lem:MP:StabilidadGamma}
%
  Sean $X_i  \, \sim  \, \G\left( a_i  , b  \right), \: i  = 1 ,  \ldots ,  n$ \
  independientes. Entonces
  % 
  \[
  \sum_{i=1}^n X_i \, \sim \, \G\left( \sum_{i=1}^n a_i \, , \, b \right)
  \]
\end{lema}

Adem\'as,  se muestra  sencillamente  por  cambio de  variables  y la  funci\'on
caracter\'istica un v\'inculo con variables gausianas:
%
\begin{lema}[V\'inculo con la gaussiana]
\label{Lem:MP:VinculoGammaGaussiana}
%
  Sean $X_i \, \sim \,  \N\left( 0 , \sigma^2 \right), \: i = 1  , \ldots , n$ \
  independientes. Entonces
  %
  \[
  \sum_{i=1}^n  X_i^2 \,  \sim \,  \G\left( \frac{n}{2}  \, ,  \,  \frac{1}{2 \,
      \sigma^2} \right)
  \]
\end{lema}

La distribuci\'on Gamma aparece entre  otros en problema de inferencia Bayesiana
como distribuci\'on a priori conjugado~\footnote{En la inferencia Bayesiana, nos
  interesamos al paremetro (posiblemente  multivariado) \ $\theta$ \ subyancente
  a una  distribuci\'on. Por ejemplo,  sabemos tener observaciones  sorteados de
  una distribuci\'on de Poisson, pero con el parametro \ $\lambda$ \ desconocido
  y nos interesamos a \ $\theta \equiv \lambda$. El enfoque Bayesiano consiste a
  considerar el paremetro  \ $\Theta$ \ aleatorio, tal  que la distribuci\'on de
  las observaciones  sea vista como distribuci\'on condicional  \ $p_{X|\Theta =
    \theta}(x)$, llamada distribuci\'on de sampleo. Dados las observaciones $X =
  x$,  la  meta  es  de  determinar  la  distribuci\'on  dicha  a  posteriori  \
  $p_{\Theta|X =  x}$ \ a  partir de  la cual se  puede hacer estimaci\'on  de \
  $\theta$ dados  las observaciones, calcular intervalos de  confianza, etc. Por
  eso,  el metodo  se  apoya  sobre la  regla  de Bayes  $p_{\Theta|X=x}(\theta)
  \propto  p_{X|\Theta=\theta}(x)  p_\Theta(\theta)$  \  as\'i que  se  necesita
  elegir  una distribuci\'on  \ $p_\Theta$  \  dicha a  priori.  Una  elecci\'on
  posible es  tomarla en una familia  parametrizada tal que  la distribuci\'on a
  posterior  partenece  tambi\'en a  esta  familia: es  lo  que  se llama  prior
  conjugado. La idea es que si  vienen observaciones, en lugar de re-calcular el
  posterior,   se   puede   actualizar   solamente  los   parametros   (llamados
  hiperparametros).    \SZ{Ver    nota    de    pie    en    el    cap    2    a
    modificar}.\label{Foot:MP:BayesPrior} } del parametro $\lambda$ de la ley de
Poisson~\cite{Rob07}.

\SZ{Esta distribuci\'on aparece...}
% en  el conteo  de conteo  de  une repetici\'on  de una  experiencia de  maneja
% independiente hasta que  occure un evento de probabilidad  $p$; por ejemplo el
% n\'umero de tiro de un dado  equilibriado hasta que occurre un ``6'' sigue una
% ley geometrica de parametro $p = \frac16$.

% --------------------------------- Wishart
\subsubseccion{Distribuci\'on matriz-variada de Wishart}
\label{Sssec:MP:Wishart}

Este ejemplo es una  generalizaci\'on matriz-variada de la distribuci\'on gamma.
Se puede ver  una matriz como un vector, guardando por  ejemplo sus columnas una
bajo la  precediente.  Sin embargo, tal  distribuci\'on apareciendo naturalmente
en un contexto de estimaci\'on de  matriz de covarianza (ver m\'as adelante), es
m\'as  natural  verla  matriz-variate.    Tal  distrubuci\'on  es  debido  a  J.
Wishart~\cite{Wis28, GupNag99, And03}, y se denota \ $X \, \sim \, W_d(V,\nu)$ \
donde  el  dominio  de  definici\'on  es  \  $P_d^+(\Rset)$,  conjunto  matrices
simetricas definida positivas, $V \in P_d^+(\Rset)$ parametro de escala y \ $\nu
> d-1$ \ grados de libertad.  Las caracter\'isticas de la distribuci\'on son las
siguientes:

\begin{caracteristicas}
%
Dominio de definici\'on~\footnote{De hecho, se puede considerar que la matriz
aleatoria es equivalent a tener un vector \ $\frac{d (d+1)}{2}$-dimensional; por
la simetria, claramente \ $X$ \ tiene solamente \ $\frac{d (d+1)}{2}$ \
componentes diferentes; adem\'as, se puede probar que cualquier matriz \ $A \in
P_d^+(\Rset)$ \ se descompone bajo la forma \ $A = L L^t$ \ con \ $L$ \
triangular inferior con elementos no nulos sobre su diagonal, llamado
descomposici\'on de Cholesky~\cite{GupNag99, Bha07, Har08, HorJoh13} y
reciprocamente. Eso muestra que \ $A$ \ se define a partir de \ $\frac{d
(d+1)}{2}$ \ ``grados de libertad''.\label{Foot:MP:WishartXtilde}} & $\X =
P_d^+(\Rset), \: d \in \Nset^*$\\[2mm]
\hline
%
Parametros & $V \in P_d^+(\Rset)$ (escala) y \ $\nu > d-1$ \ (grados de
libertad)\\[2mm]
\hline
%
Densidad de probabilidad~\footnote{La densidad de probabilidad corresponde a la
densidad conjunta de los \ $\frac{d (d+1)}{2}$ \ elementos \ $X_{i,j}, \: 1 \le
i \le j \le d$~\cite{Wis28, PedRic91, SulTra96, GupNag99,
And03}.\label{Foot:MP:WishartDensidad}} & $\displaystyle p_X(x) =
\frac{|x|^{\frac{\nu-d-1}{2}} \, e^{-\frac12 \Tr\left( V^{-1}
x\right)}}{2^{\frac{d \nu}{2}} \, |V|^{\frac{\nu}{2}} \, \Gamma_d\left(
\frac{\nu}{2} \right)}$\\[2mm]
\hline
%
Promedio & $\displaystyle m_X = \nu \, V$\\[2mm]
\hline
%
Covarianza & $\displaystyle \Sigma_X = \nu \big( J (V \otimes V) + (V \otimes I)
K (V \otimes I) \big)$\\[2mm]
% $\Cov[X_{i,j},X_{k,l}] = nu \left( V_{i,k} V_{j,l} +V_{i,l} V_{j,k} \right)$}
\hline
%
Funci\'on caracter\'istica~\footnote{\SZ{Se proba que la funci\'on generadora de
momentos no existe en
general}.\label{Foot:MP:CaracteristicaWishart}} &
$\displaystyle \Phi_X(\omega) = \left| I - 2 \imath \omega V
\right|^{-\frac{\nu}{2}}, \quad \omega \in S_d(\Rset)$
\end{caracteristicas}

(ver~\cite{PedRic91, SulTra96, And03}).

Fijense que $p_X$ no es la distribuci\'on conjuntos de los componentes de \ $X$:
el hecho de  que \ $X$ \ sea  uan matriz aleatoria de \  $P_d^+(\Rset)$ \ impone
v\'inculos sobre sus compnentes; entre otros, $X_{i,j} = X_{j,i}$.

Inmediatamente, si  $d = 1$, la  distribuci\'on de Whishart \  $W_1(V,\nu)$ \ se
reduce  a la  distribuci\'on Gamma  $\G\left(\frac{\nu}{2} \,  , \,  \frac1{2 V}
\right)$. De este  hecho, se la podr\'ia ver  como extensi\'on matriz-variada de
la  distribuci\'on  gamma.  La  distribuci\'on  de  Wishart  tiene varias  otras
propiedades como las siguientes.
%
\begin{lema}[Stabilidad por transformaci\'on lineal]
\label{Lem:MP:StabilidadWishartLineal}
%
  Sea $X \,  \sim \, W_d(V,\nu)$ \ y \  $A \in \Rset^{d \times d'}$  \ con \ $d'
  \le d$ \ y de rango lleno. Entonces
  \[
  A^t X A \, \sim \, W_{d'}\left( A^t V A , \nu \right)
  \]
  %
  En particular, si $d'  = 1$, \ $A^t X A \, \sim  \, G\left( \frac{\nu}{2} \, ,
    \, \frac1{2 \, A^t V A} \right)$. M\'as all\'a, tomando $A = \un_j$, aparece
  de que  las componentes diagonales de \  $X$ \ son de  distribuci\'on gamma, \
  $X_{j,j}  \, \sim  \,  \G\left( \frac{\nu}{2}  \,  , \,  \frac1{2 \,  V_{j,j}}
  \right)$.
\end{lema}
%
\begin{proof}
  El     resultado     es     inmediato     saliendo     de     la     funci\'on
  caract\'eristica~\footref{Foot:MP:CaracteristicaWishart} y notando de que
%
\begin{eqnarray*}
\Phi_{A^t X A}(\omega) & = & \Esp\left[ e^{\imath \Tr\left( \omega^t A^t X A
\right)}\right]\\[2mm]
%
& = & \Phi_X\left( A \omega^t A^t\right)\\[2mm]
%
& = &  \left| I - 2 \imath A \omega A^t V \right|^{-\frac{\nu}{2}}\\[2mm]
%
& = &  \left| I - 2 \imath \omega A^t V A \right|^{-\frac{\nu}{2}}
%
\end{eqnarray*}
%
de   \   $\Tr(AB)   =   \Tr(BA)$~\cite{Har08}   \   y   de   la   identidad   de
Sylvester~\cite{Syl51,  AkrAkr96}  o~\cite[\S~18.1]{Har08} \  $\left|  I  + A  B
\right| = \left| I + B A \right|$.  .
\end{proof}
%
De hecho, si los elementos diagonales son de distribuci\'on gamma, no es el caso
de         los        elementos         no-diagonales~\cite{Seb04,        And03}
o~\cite[Teo.~3.3.4]{GupNag99}.    De    eso   resuelte   delicado    llamar   la
distribuci\'on como gamma matriz-variada.

\begin{lema}[Stabilidad por suma]
\label{Lem:MP:StabilidadWishartSuma}
%
  Sea $X_i \,  \sim \, W_d(V,\nu_i), \: i = 1,\ldots,n$ independientes. Entonces
  \[
  \sum_{i=1}^n X_i \, \sim \, W_d\left( V \, , \, \sum_{i=1}^n \nu_i \right)
  \]
\end{lema}
%
\begin{proof}
  El     resultado     es     inmediato     saliendo     de     la     funci\'on
  caract\'eristica~\footref{Foot:MP:CaracteristicaWishart} y notando que como el
  el context vectorial $\Phi_{\sum_i X_i} = \prod_i \Phi_{X_i}$.
\end{proof}

La distribuci\'on  de Wishart aparece naturalmente en  problemas de estimaci\'on
de matriz de covarianza en el contexto gausiano:
%
\begin{lema}[V\'inculo con vectores gausianos~\cite{Seb04}]
  Sean \ $X_i \, \sim \, \N(0,V), \: i = 1, \ldots , n > d-1$ \ independientes y
  la  matriz  \  $S  =  \sum_{i=1}^n   X_i  X_i^t$  \  llamada  {\em  matriz  de
    dispersi\'on} (scatter matrix en ingl\'es). Entonces, \ $S \in P_d^+(\Rset)$
  (c.   s.)   \  ($S$ es  sim\'etrica  definida  positiva  casi siempre,  o  con
  probabilidad uno) y \ $S \,\sim \, W_d(V,n)$.
\end{lema}
%
Este resultado  permite tambi\'en probar  el lema~\ref{Lem:MP:StabilidadWishart}
para \ $\nu = n$ \ entero  escribiendo \ $X \egald \sum_{i=1}^n X_i X_i^t$ \ tal
que \ $A^t X A \egald \sum_{i=1}^n A^t X_i X_i^t A = \frac1n \sum_{i=1}^n \left(
  A^t X_i \right)  \left( A^t X_i \right)^t$ \  y notando que los \  $A^t X_i \,
\sim \,  \N(0, A^t V  A)$ \ son independientes~\cite{Seb04}.   Adem\'as, permite
re-obtener las  expreciones del promedio y de  las covarianzas~\footnote{Para la
  covarianza, su usa la formula \ $\Esp[Y_1 Y_2 Y_3 Y_4] = \Esp[Y_1 Y_2]\Esp[Y_3
  Y_4]  + \Esp[Y_1 Y_3]\Esp[Y_2  Y_4] +  \Esp[Y_1 Y_4]\Esp[Y_2  Y_3]$ \  para $Y
  = \begin{bmatrix}  Y_1 & Y_2 &  Y_3 & Y_4 \end{bmatrix}^t$  \ vector gausiano,
  formula que se  obtiene por ejemplo a partir  de la funci\'on caracter\'istica
  de un vecor  gausiano.}. Notar que cuando los \ $X_i$  \ tienen un promemedio,
el  lema conduce  a  lo que  es  conocido como  Wishart no  central~\cite{And03,
  Seb04}.

\SZ{Que propedad mas? Ver Gupta Nagar 1999}

La distribuci\'on  Wishart aparece as\'i naturalmente en  problema de inferencia
Bayesiana  como distribuci\'on  a  priori conjugado~\footref{Foot:MP:BayesPrior}
del parametro $p$ de la ley gaussiana multivariada~\cite{Rob07}.

\SZ{Y donde aparece mas?}

% --------------------------------- Beta
\subsubseccion{Distribuci\'on beta}
\label{Sssec:MP:Beta}

Estas   distribuciones  fueron   popularizadas   por  Pearson   en  los   a\~nos
1895~\cite{Pea95, Pea16, DavEdw01}  bajo la denominaci\'on Pearson tipo  I en su
estudio  de  la teoria  de  la evoluci\'on  y  la  modelizaci\'on con  variables
asim\'etricas.  De  hecho, apareci\'o mucho  tiempo antes, en trabajos  de Bayes
publicado en un papel postumo  por R. Price en 1763~\cite{Bay63}. Aparentemente,
la denominaci\'on  estandar ``beta'' es  debido al estad\'istico,  dem\'ografo y
soci\'ologo  italiano   C.   Gini   en  1911  su   estudio  del   ``sex  ratio''
(desequilibrio  entre los  nacimientos  de muchachos/muchachas)  con un  enfoque
bayesiano~\cite{Gin11,  For17,   DavEdw01}.   La  distribuci\'on   beta  aparece
precisamente,   entre  otros,   en   problema  de   inferencia  bayesiana   como
distribuci\'on   a   priori   conjugado   del   par\'ametro  $p$   de   la   ley
binomial~\cite{Rob07}     (ver     notas     de     pie~\ref{Foot:MP:BayesPrior}
y~\ref{Foot:MP:BayesPriorConjugado}).

Se denota $X  \sim \beta(a,b)$ \ con  \ $(a,b) \in \Rset_+^{* \,  2}$ \ llamados
{\em par\'ametros de forma}.  Las caracter\'isticas son:

\begin{caracteristicas}
%
Dominio de definici\'on & $\X = [0 \; 1]$\\[2mm]
\hline
%
Par\'ametros & $(a,b) \in \Rset_+^{* \, 2}$ (forma)\\[2mm]
\hline
%
Densidad   de    probabilidad   &   $\displaystyle    p_X(x)   =   \frac{x^{a-1}
(1-x)^{b-1}}{B(a,b)}$\\[2mm]
\hline
%
Promedio & $\displaystyle m_X = \frac{a}{a+b}$\\[2mm]
\hline
%
Varianza &  $\displaystyle \sigma_X^2  = \frac{a b}{(a  + b)^2  (a + b  + 1)}$\\[2mm]
\hline
%
\modif{Asimetr\'ia} & $\displaystyle \gamma_X = \frac{2 \, (b - a) \sqrt{a + b + 1}}{( a
+ b + 2) \sqrt{a b}}$\\[2mm]
\hline
%
Curtosis por exceso & $\displaystyle \widebar{\kappa}_X = \frac{6 \, \left( (a - b)^2 (a + b + 1) - a
b (a  + b  + 2)  \right)}{a \, b  \left( a  + b  + 2 \right)  \left( a  + b  + 3
\right)}$\\[2mm]
\hline
%
Generadora de momentos & $\displaystyle M_X(u)  = \hypgeom{1}{1}\left( a , a + b
\, ; \, u \right)$ \ para \ $u \in \Cset$\\[2mm]
\hline
%
Funci\'on     caracter\'istica     &     $\displaystyle     \Phi_X(\omega)     =
\hypgeom{1}{1}\left( a , a + b \, ; \, \imath \omega \right)$
\end{caracteristicas}

Unas  densidades de  probabilidad  y funciones  de  repartici\'on asociadas  son
representadas en la figura Fig.~\ref{Fig:MP:Beta} para varios $a$ \ y \ $b$.
%
\begin{figure}[h!]
\begin{center} \begin{tikzpicture}%[scale=.9]
\shorthandoff{>}
%
\pgfmathsetmacro{\sx}{5};% x-scaling
\pgfmathsetmacro{\b}{5};% tercera eleccion de (2,beta)
\pgfmathsetmacro{\r}{.05};% radius arc non continuity F_X
%
%
% densidad
\begin{scope}
%
\pgfmathsetmacro{\sy}{2.5*((\b/(\b-1))^(\b-1))/(\b+1)};% y-scaling 
\draw[>=stealth,->] (-.5,0)--({\sx+.5},0) node[right]{\small $x$};
\draw[>=stealth,->] (0,-.1)--(0,2.75) node[above]{\small $p_X$};
%
%\draw[thick] (-.25,0)--(0,0);\draw (\r,\r) arc (90:270:\r);
%\draw[thick] (\sx,0)--({\sx+.25},0);\draw ({\sx-\r},{-\r}) arc (-90:90:\r);
% (a,b) = ( .5 , .5 )
\draw[thick,dotted,domain=.02:.98,samples=100] plot ({\x*\sx},{\sy*1/(pi*sqrt(\x*(1-\x)))});
% (a,b) = ( 2 , 1 )
\draw[thick,dashed,domain=0:1,samples=100] plot ({\x*\sx},{\sy*2*\x}) node[scale=.4]{$\bullet$};
\draw[dotted] (\sx,{2*\sy})--(\sx,0);
% (a,b) = ( 2 , 2 )
\draw[thick,dash dot,domain=0:1,samples=100] plot ({\x*\sx},{\sy*6*\x*(1-\x)});
% (a,b) = ( 2 , b_3 )
\draw[thick,domain=0:1,samples=100] plot ({\x*\sx},{\sy*\b*(\b+1)*\x*((1-\x)^(\b-1))});
%}
%
\draw (0,{\sy*2/pi})--(-.1,{\sy*2/pi}) node[left,scale=.7]{$\frac2\pi$};
\draw (0,{\sy*2})--(-.1,{\sy*2}) node[left,scale=.7]{$2$};
\draw (0,{\sy*1.5})--(-.1,{\sy*1.5}) node[left,scale=.7]{$\frac32$};
\draw (0,{\sy*(\b+1)*((1-1/\b)^(\b-1))})--(-.1,{\sy*(\b+1)*((1-1/\b)^(\b-1))}) node[left,scale=.7]{$6 \left(\frac{4}{5} \right)^4$};
\draw (\sx,0)--(\sx,-.1) node[below,scale=.7]{$1$};
%
\end{scope}
%
%
% reparticion
\begin{scope}[xshift=8.5cm]
%
\pgfmathsetmacro{\sy}{2.5};% y-scaling 
%
\draw[>=stealth,->] (-.75,0)--({\sx*1.25+.25},0) node[right]{\small $x$};
\draw[>=stealth,->] (0,-.1)--(0,{\sy+.25}) node[above]{\small $F_X$};
%
% cumulativa
\draw[thick] (-.5,0)--(0,0); \draw[thick] (\sx,\sy)--({\sx*1.25},\sy);
% (a,b) = ( .5 , .5 )
\draw[thick,dotted,domain=0:1,samples=100] plot ({\x*\sx},{\sy*(.5+asin(2*\x-1)/180)});
% (a,b) = ( 2 , 1 )
\draw[thick,dashed,domain=0:1,samples=100] plot ({\x*\sx},{\sy*(1-(1+\x)*(1-\x))});
% (a,b) = ( 2 , 2 )
\draw[thick,dash dot,domain=0:1,samples=100] plot ({\x*\sx},{\sy*(1-(1+2*\x)*((1-\x)^2))});
% (a,b) = ( 2 , b_3 )
\draw[thick,domain=0:1,samples=100] plot ({\x*\sx},{\sy*(1-(1+\b*\x)*((1-\x)^\b))});
%
\draw (0,\sy)--(-.1,\sy) node[left,scale=.7]{$1$};
\draw (\sx,0)--(\sx,-.1) node[below,scale=.7]{$1$};
\end{scope}
%
\end{tikzpicture} \end{center}
%
\leyenda{Ilustraci\'on de una densidad de  probabilidad beta (a), y la funci\'on
de  repartici\'on asociada (b).   $(a,b) =  (0.5 \,  , \,  0.5)$ (linea
punteada), $(3 \, , \, 1)$ (linea  mixta doble punteada), $(3 \, , \, 2)$ (linea
mixta), $(3 \, , \, 3)$ (linea
guionada), $(3 \, , \, 7)$ (linea llena).}
\label{Fig:MP:Beta}
\end{figure}

Notar que  se recupera la  ley uniforme sobre  \ $[0 \;  1]$ \ para  \ $a =  b =
1$. Se  conoce la ley de $  Y = 2 \,  B - 1$ \  con \ $B \,  \sim \, \beta\left(
  \frac12 , \frac12 \right)$ \ como {\em ley arcseno}.

Variables beta  tienen tambi\'en unas propiedades notables.  Primero, por cambio
de variables, se demuestra el lema siguiente:
%
\begin{lema}[Reflexividad]
\label{Lem:MP:ReflexividadBeta}
%
  Sea \ $X \, \sim \, \beta(a,b)$. Entonces
  %
  \[
  1-X \, \sim \, \beta(b,a)
  \]
  %
\end{lema}


\begin{lema}[Un v\'inculo con la ley exponencial]
\label{Lem:MP:VinculoBetaExponencial}
%
  Sea  \   $X  \,  \sim  \,   \beta(a,1)$. Entonces
  %
  \[
  - \log X \, \sim \, \E(a)
  \]
  %
\end{lema}
%
\begin{proof}
  El   resultado  es   inmediato  de   la  f\'ormula   de   transformaci\'on  del
  corolario~\ref{Cor:MP:TransformacionInyectivaDensidadEscalar}.
\end{proof}


\begin{lema}[Un v\'inculo con la ley uniforme]
\label{Lem:MP:VinculoBetaUniforme}
%
  Sea  \   $X  \,  \sim  \,   \U([0 \; 1])$ \ y \ $a > 0$. Entonces
  %
  \[
  U^{\frac{1}{a}} \, \sim \, \beta(a,1)
  \]
  %
\end{lema}
%
\begin{proof}
  El   resultado  es   inmediato  de   la  f\'ormula   de   transformaci\'on  del
  corolario~\ref{Cor:MP:TransformacionInyectivaDensidadEscalar}.
\end{proof}

\begin{lema}[Un v\'inculo con la ley gamma]
\label{Lem:MP:VinculoBetaGamma}
%
  Sea  \   $X  \,  \sim  \,   \G(a,c)$  \  e  \   $Y  \,  \sim   \,  \G(b,c)$  \
  independientes. Entonces
  %
  \[
  \frac{X}{X+Y} \, \sim \, \beta(a,b)
  \]
  %
  (independientemente  de $c$).   Adem\'as, $\frac{X}{X+Y}$  \ y  \ $X+Y$  \ son
  independientes.
\end{lema}
%
\begin{proof}
  La independencia de \ $c$ \ es obvia del hecho de que para cualquier $\theta >
  0, \: \theta^{-1} X \, \sim \, \G(a,\theta c)$ \ e \ $\theta^{-1} Y \, \sim \,
  \G(b,\theta  c)$,  la independencia  con  respecto  a \  $c$  \  viniendo de  \
  $\frac{\theta^{-1}     X}{\theta^{-1}     X     +     \theta^{-1}     Y}     =
  \frac{X}{X+Y}$.  Entonces, se  puede considerar  \ $c  = 1$  \ sin  perdida de
  generalidad. Ahora, sea la transformaci\'on
  %
  \[
  \begin{array}{lccl}
    g\ : & \Rset_+^2 & \mapsto & [0 \; 1] \times \Rset_+\\[1.5mm]
    %
    & (x,y) & \to & (u,v) = \left( \frac{x}{x+y} \, , \, x+y \right)
  \end{array}
  \]
  %
  Entonces, la transformaci\'on inversa se escribe
  %
  \[
  g^{-1}(u,v) = \left( u v \, , \, (1-u) v \right)
  \]
  %
  de matriz Jacobiana
  %
  \[
  \Jac_{g^{-1}} = \begin{bmatrix} v & u \\[2mm] -v & 1-u \end{bmatrix}
  \]
  %
  Del          teorema          de          cambio         de          variables
  teorema~\ref{Teo:MP:TransformacionBiyectiva},      notando     que     $\left|
    \Jac_{g^{-1}} \right| =  v$ \ y de la  independencia de \ $X$ \ e  \ $Y$, se
  obtiene para el vector aleatorio \ $W = \begin{bmatrix} U & V \end{bmatrix}^t$
  \ la densidad de probabilidad, definida sobre $[0 \; 1] \times \Rset_+$, como
  %
  \begin{eqnarray*}
    p_W(u,v) & = & p_X( u v ) \, p_Y( (1-u) v ) \, v\\[2mm]
    %
    & = & \frac{\left( u v \right)^{a-1} \, e^{- u v}}{\Gamma(a)} \times
    \frac{\left( (1-u) v \right)^{b-1} \, e^{- (1-u) v}}{\Gamma(b)} \times v\\[2mm]
    %
    & = & \frac{u^{a-1} (1-u)^{b-1}}{B(a,b)} \times \frac{v^{a+b-1} e^{-v}}{\Gamma(a+b)}
  \end{eqnarray*}
  %
  Inmediatamente, factorizandose, aparece  claramente que \ $U$ \ y  \ $V$ \ son
  independientes. Adem\'as, se reconoce en  el primer factor la densidad beta de
  par\'ametros $(a,b)$.   Pasando, se recupera  el hecho que  \ $X+Y \,  \sim \,
  \G(a+b,1)$.
\end{proof}

\begin{lema}[Stabilidad por producto]
\label{Lem:StabilidadBeta}
%
  Sea  \  $X \,  \sim  \, \beta(a,b)$  \  e  \ $Y  \,  \sim  \, \beta(a+b,c)$  \
  independientes. Entonces
  %
  \[
  X Y \, \sim \, \beta(a,b+c)
  \]
  %
\end{lema}
%
\begin{proof}
  Sean \ $U  \, \sim \, \G(a,1)$, \ $V \,  \sim \, \G(b,1)$ \ y \  $W \, \sim \,
  \G(c,1)$   \   independientes  y   sean   \   $X   =  \frac{U}{U+V}$,   $Y   =
  \frac{U+V}{U+V+W}$  \ y  \ $Z  = U+V+W$.  Del lema  anterior \  $X \,  \sim \,
  \beta(a,b)$ \ y \ $Y \, \sim \, \beta(a+b,c)$. Sea la transformaci\'on
  %
  \[
  \begin{array}{lccl}
    g\ : & \Rset_+^3 & \mapsto & [0 \; 1]^2 \times \Rset_+\\[1.5mm]
    %
    & (u,v,w) & \to & (x,y,z) = \left( \frac{u}{u+v} \, , \, \frac{u+v}{u+v+w} \, , \, u+v+w \right)
  \end{array}
  \]
  %
  Entonces, la transformaci\'on inversa se escribe
  %
  \[
  g^{-1}(x,y,z) = \left( x y z \, , \, (1-x) y z \, , \, z (1-y) \right)
  \]
  %
  de matriz Jacobiana
  %
  \[
  \Jac_{g^{-1}} = \begin{bmatrix}
  %
    y z  &   x z   &   x y   \\[2mm]
  %
  - y z  & (1-x) z & (1-x) y \\[2mm]
  %
    0    &  - z    &  1-y
  %
  \end{bmatrix}
  \]
  %
  De      nuevo,      del      teorema      de     cambio      de      variables
  teorema~\ref{Teo:MP:TransformacionBiyectiva}, notando que $\left| \Jac_{g^{-1}}
  \right| = y  z^2$ \ y de la independencia  de \ $U, V, W$,  se obtiene para el
  vector aleatorio  \ $  T =  \begin{bmatrix} X &  Y &  Z \end{bmatrix}^t$  \ la
  densidad de probabilidad probabilidad
  %
  \begin{eqnarray*}
    p_T(x,y,z) & = & p_u( x y z ) \, p_V( (1-x) y z ) \, p_W( y (1-z) ) \, y z^2\\[2mm]
    %
    & = & \frac{\left( x y z \right)^{a-1} \, e^{- x y z}}{\Gamma(a)} \times
    \frac{\left( (1-x) y z \right)^{b-1} \, e^{- (1-x) y z}}{\Gamma(b)} \times
    \frac{\left( z (1-y) \right)^{c-1} \, e^{- z (1-y)}}{\Gamma(c)} \times y
    z^2\\[2mm]
    %
    & = & \frac{x^{a-1} (1-x)^{b-1}}{B(a,b)} \times \frac{y^{a+b-1}
      (1-y)^{c-1}}{B(a+b,c)} \times \frac{z^{a+b+c-1} e^{-z}}{\Gamma(a+b+c)}
  \end{eqnarray*}
  %
  Eso proba  que \  $X, \ Y$  \ y  $Z$ \ son  independientes (las  densidades se
  factorizan). Adem\'as,
  %
  \[
  X  Y =  \frac{U}{U+V} \times  \frac{U+V}{U+V+W} =  \frac{U}{U+V+W} \,  \sim \,
  \beta(a,b+c)
  \]
  %
  el      \'ultimo       resultado      como      consecuencia       de      los
  lemas~\ref{Lem:MP:VinculoBetaGamma}    y~\ref{Lem:MP:StabilidadGamma}.     Eso
  cierra la prueba.
\end{proof}


\begin{lema}[Ley gamma como caso l\'imite de beta]
\label{Lem:GamaLimiteBeta}
%
  Sea \ $X_n \, \sim \, \beta(a,n)$. Entonces
  %
  \[
  n X_n \limitd{n \to +\infty} X \, \sim \, \G(a,1)
  \]
  %
  con \ $\displaystyle \limitd{}$ \ l\'imite es en distribuci\'on
\end{lema}
%
\begin{proof}
  De la f\'ormula de transformaci\'on tenemos la distribuci\'on de $n X_n$
  %
  \begin{eqnarray*}
  p_{n X_n}(x) & = & \frac{1}{n} , \frac{\left( \frac{x}{n} \right)^{a-1} \left( 1
  - \frac{x}{n} \right)^{n-1}}{B(a,n)} \, \un_{(0 \; 1)}\left( \frac{x}{n} \right)\\[2mm]
  %
  & = & \frac{x^{a-1}}{\Gamma(a)} \: \frac{\Gamma(n+a)}{n^a \Gamma(n)} \: \left( 1 -
  \frac{x}{n} \right)^{n-1} \: \un_{(0 \; n)(x)}
  \end{eqnarray*}
  %
  El resultado sigue  notando que \ $\un_{(0 \;  n)} \to \un_{\Rset_{0,+}}$, \quad
  $\left(  1 -  \frac{x}{n} \right)^{n-1}  \to e^{-x}$  \ y  de la  f\'ormula de
  Stirling (ver secci\'on~\ref{Sssec:MP:Poisson}).
\end{proof}

\

La distribuci\'on beta  se generaliza al caso matriz-variada  $X$ definido sobre
$\X$ tal que $X$ y $I-X$ partenecen  a $\Pos_d^+(\Rset)$; se denota \ $X \, \sim
\, \beta_d(a,b)$ \ donde  \ $(a,b) \in \Rset_+^{* \, 2}$ y  la densidad est dada
por   $\displaystyle  p_X(x)   =  \frac{|x|^{a   -  \frac{d+1}{2}}   |I-x|^{b  -
    \frac{d+1}{2}}}{B_p\left([a  \quad  b]^t\right)},  \quad  (a,b)  \in  \left(
  \frac{d-1}{2} \; +\infty \right)^2$. Se refiera a~\cite[Cap.~5]{GupNag99} para
tener m\'as detalles.   Notar que esta distribuci\'on cae en  en una clase dicha
el\'iptica,      que      vamos       a      ver      brevemente      en      la
secci\'on~\ref{Ssec:MP:FamiliaElipticaMatriz},  as\'i  que  propiedades en  este
marco general.


% --------------------------------- Dirichlet
\subsubseccion{Distribuci\'on de Dirichlet}
\label{Sssec:MP:Dirichlet}

Esta distribuci\'on teniendo su nombre de integrales on a simplex estudiados por
M. Lejeune-Dirichlet y J. Liouville  en 1839~\cite{GupRic01, Dir39, Lio39} en es
una extensi\'on  multivariada de las variables  beta a veces  conocida como {\em
  Beta multivariada}~\cite{OlkRub64}. Se  nota \ $X \, \sim \,  \Dir(a)$ \ con \
$a \in \Rset_+^{*  \, k}$ \ y \  $X$ \ vive sobre el  $(k-1)$-simplex estandar \
$\Simp{k-1}$.  $a$ es  llamado parametro de forma. Como en  el caso de vectores
de  distribuci\'on multinomial, a  pesar de  que se  escribe \  $X$ \  de manera
$k$-dimensional, el vector partenece a una  variedad \ $d = k-1$ \ dimensional y
en el caso  \ $k = 2$ \ se  recupera la ley beta. A veces  se parametriza la ley
con un  parametro escalar \ $\alpha  > 0$ \ y  un vector del  simplex estandar \
$\bar{a} \in \Simp{k-1}$ \ tal que
%
\[
a = \alpha \bar{a}, \quad \mbox{\ie} \quad \alpha = \sum_{i=1}^k \alpha_i, \quad
\bar{a} = \frac{a}{\alpha}
\]
%
$\alpha$ \  es conocido como  parametro de {\em  concentraci\'on} y el  vector \
$\bar{a}$ \ como {\em medida de base}.

Las caracter\'isticas de un vector de Dirichlet son:

\begin{caracteristicas}
%
Dominio de definici\'on~\footnote{De hecho, se puede considerar que el vector
aleatorio es \ $(k-1)$-dimensional \ $\widetilde{X} = \begin{bmatrix}
\widetilde{X}_1 & \cdots & \widetilde{X}_{k-1} \end{bmatrix}^t$ \ definido sobre
el hipertriangulo \ $\widetilde{\X} = \Tri_{k-1} = \left\{ \widetilde{x} \in [0
\; 1]^{k-1}, \: \sum_{i=1}^{k-1} \widetilde{x}_i \le 1 \right\}$, proyecci\'on
del simplex sobre el hiperplano \ $x_k = 0$.\label{Foot:MP:DirichletXtilde}} &
$\X = \Simp{k-1}, \: k \in \Nset \setminus \{ 0 \; 1 \}$\\[2mm]
\hline
%
Parametros & $a = \alpha \, \bar{a} \, \in \, \Rset_+^{* \, k}$ \ (forma) \ con
\ $\alpha \in \Rset_+^*$ \ (concentraci\'on) y \ $\bar{a} \in \Simp{k-1}$
(medida de base)\\[2mm]
\hline
%
Densidad de probabilidad~\footnote{La densidad de probabilidad es dada con
respeto a la medida de Lebesgue restricta al simplex \ $\Simp{k-1}$. Tratando
de \ $\widetilde{X}$, el vector tiene una densidad con respeto a la medida de
Lebesgue usual, y es dada por \ $p_{\widetilde{X}}\left( \widetilde{x} \right) =
\frac{\prod_{i=1}^{k-1} \widetilde{x}_i^{\, a_i-1} \, \left( 1 -
\sum_{i=1}^{k-1} \widetilde{x}_i
\right)^{a_k-1}}{B(a)}$.\label{Foot:MP:DirichletDensidad}} & $\displaystyle
p_X(x) = \frac{\prod_{i=1}^k x_i^{a_i-1}}{B(a)}$\\[2mm]
\hline
%
Promedio & $\displaystyle m_X = \bar{a}$\\[2.5mm]
%\frac{a}{\sum_{i=1}^k a_i} \equiv \overline{a}$\\[2.5mm]
\hline
%
Covarianza & $\displaystyle \Sigma_X = \frac{\diag\left( \bar{a} \right) -
\bar{a} \bar{a}^t}{1 + \alpha}$\\[2.5mm]
%\sum_{i=1}^k a_k}$\\[2.5mm]
\hline
%
Generadora de momentos~\footnote{El v\'inculo entre las funciones generadoras de
momento de \ $X$ \ y \ $\widetilde{X}$ \ es trivialmente \
$M_{\widetilde{X}}\left( \widetilde{u} \right) = M_X\left( \begin{bmatrix}
\widetilde{u} & 0 \end{bmatrix}^t \right)$ \ o \ $M_X(u) = e^{\imath u_k}
M_{\widetilde{X}}\left( \begin{bmatrix} u_1 - u_k & \cdots & u_{k-1} -
u_k \end{bmatrix}^t \right)$, y similarmente para la funci\'on
caracter\'istica.\label{Foot:MP:Dirichlet}} & $\displaystyle M_X(u) =
\Phi_2^{(k)}( a , \alpha \, ; \, u )$ \ para \ $u \in \Cset$\\[2mm]
\hline
%
Funci\'on caracter\'istica~\footnote{La forma de la funci\'on generadora de
momento viene directamente de la escritura de las series de Taylor de $e^{u_i
x_i}$ \ o de la forma integral de la funci\'on confluente
hipergeom\'etrica~\cite{Phi88}.} & $\displaystyle \Phi_X(\omega) = \Phi_2^{(k)}(
a , \alpha \, ; \, \imath \omega )$
\end{caracteristicas}

%$k$-variada~\footnote{$\Phi_2^{(k)}(a;b;z)     =     \sum_{m    \in     \Nset^k}
%  \frac{(a_1)_{(m_1)}     \ldots    (a_k)_{(m_k)}     \,     z_1^{m_1}    \ldots
%    z_k^{m_k}}{(b)_{(m_1+\cdots+m_k)} m_1!  \ldots m_k!}$  \ con \ $(x)_{(n)}$ \
%  s\'imbolo  de Pochhammer  usual o  factorial creciente,  $(x)_{(n)} =  x (x+1)
%  \ldots (x+n)$ \,  con la convenci\'ion \ $(x)_{(0)} = 1$.   De hecho, la forma

La figura Fig.~\ref{Fig:MP:Dirichlet} representa  el dominio de definici\'on del
vector (a) y su densidad de  probabildad con las marginales (ver m\'as adelante)
para $k = 3$.
%
\begin{figure}[h!]
\begin{center} \begin{tikzpicture}%[scale=.8]
\shorthandoff{>}
%
\tikzset{declare function={
xplus(\x) = max(\x,0);
%ifthenelse(\x > 0 , \x , NaN);
}}
%}

% Simplex
\tdplotsetmaincoords{45}{65}
\begin{scope}[tdplot_main_coords,scale=.75]
%
% Dirichlet: \X = S_{k-1} y \widetilde{X}
\pgfmathsetmacro{\dx}{3};% scaling
%
\draw[->,>=stealth] (-.25,0,0)--({\dx+.5},0,0) node[below right,scale=.9]{$x_1$};
%\node at (\dx,0,0)[left,scale=.8]{$1$};
\draw (\dx,0,0)--(\dx,-.15,0) node[left,scale=.8]{$1$};
%
\draw[->,>=stealth] (0,-.25,0)--(0,{\dx+.5},0) node[right,scale=.9]{$x_2$};
%\node at (0,\dx,0)[below,scale=.8]{$1$};
\draw (0,\dx,0)--(.15,\dx,0) node[below,scale=.8]{$1$};
%
\draw[->,>=stealth] (0,0,-.25)--(0,0,{\dx+.5}) node[above,scale=.9]{$x_3$};
%\node at (0,0,\dx)[left,scale=.8]{$1$};
\draw (0,0,\dx)--(0,-.15,\dx) node[left,scale=.8]{$1$};
%
\node at (0,0,0)[below left,scale=.8]{$0$};
%
% tilde X
\filldraw[fill=black!50,opacity=.5] (0,0,0)--(\dx,0,0)--(0,\dx,0);
\draw[thick,color=black,dashed] (0,0,0)--(\dx,0,0)--(0,\dx,0)--(0,0,0);
\node at ({\dx/15},{\dx/20},0)[right,scale=.7]{$\Tri_2$};
%
% Simplex Delta_2
\filldraw[fill=black!75,opacity=.5] (\dx,0,0)--(0,\dx,0)--(0,0,\dx);
\draw[thick,color=black] (\dx,0,0)--(0,\dx,0)--(0,0,\dx)--(\dx,0,0);
\node at ({.05*\dx},{.05*\dx},{.8*\dx})[right,scale=.7]{$\Simp{2}$};
%
\end{scope}
%
%
% densidad (3,2,2)
\begin{scope}[xshift=4cm,yshift=-2cm,scale=.75]
%
\pgfmathsetmacro{\au}{3};% a1
\pgfmathsetmacro{\ad}{2};% a2
\pgfmathsetmacro{\at}{2};% a3
\pgfmathsetmacro{\B}{factorial(\au-1)*factorial(\ad-1)*factorial(\at-1)/factorial(\au+\ad+\at-1)};% normalizacion
\pgfmathsetmacro{\Bu}{factorial(\au-1)*factorial(\ad+\at-1)/factorial(\au+\ad+\at-1)};% normalizacion 1
\pgfmathsetmacro{\Bd}{factorial(\ad-1)*factorial(\au+\at-1)/factorial(\au+\ad+\at-1)};% normalizacion 2
\pgfmathsetmacro{\ma}{((\au-1)^(\au-1))*((\ad-1)^(\ad-1))*((\at-1)^(\at-1))/((\au+\ad+\at-3)^(\au+\ad+\at-3))/\B};
%
% Dirichlet & marginales
\begin{axis}[
    colormap = {whiteblack}{color(0cm)  = (white);color(1cm) = (black)},
    width=.5\textwidth,
    view={45}{65},
    enlargelimits=false,
    %grid=major,
    domain=0:1,
    y domain=0:1,
    %unbounded coords=jump, % para tener un dominio no cuadrado
    %filter point/.code={%
    %\pgfmathparse
    %{\pgfkeysvalueof{/data point/x} + \pgfkeysvalueof{/data point/y} > 1.0}%
    %  \ifpgfmathfloatcomparison
    %     \pgfkeyssetvalue{/data point/x}{nan}%
    %  \fi
    %},
    zmax={.8*\ma},
    color=black,
    samples=70,
    xlabel=$x_1$,
    ylabel=$x_2$,
    zlabel=$p_{\widetilde{X}}$,
]
%
% Dirichlet
\addplot3 [surf] {(x^(\au-1))*(y^(\ad-1))*(xplus(1-x-y)^(\at-1))/\B};
%
% Marginales
\addplot3 [domain=0:1,samples=50, samples y=0, thick, smooth, color=black] (x,1,{(x^(\au-1))*((1-x)^(\ad+\at-1))/\Bu});
\addplot3 [domain=0:1,samples=50, samples y=0, thick, smooth, color=black] (0,x,{(x^(\ad-1))*((1-x)^(\au+\at-1))/\Bd});
%
\node at (axis cs:.5,1,{1/(2^(\au+\ad+\at-2))/\Bu})[right]{$p_{X_1}$};
\node at (axis cs:0,.5,{1/(2^(\au+\ad+\at-2))/\Bd})[above]{$p_{X_2}$};
\end{axis}
\end{scope}
%
%
% densidad (3,2,2)
\begin{scope}[xshift=11cm,yshift=-2cm,scale=.75]
%
\pgfmathsetmacro{\au}{3};% a1
\pgfmathsetmacro{\ad}{1};% a2
\pgfmathsetmacro{\at}{2};% a3
\pgfmathsetmacro{\B}{factorial(\au-1)*factorial(\ad-1)*factorial(\at-1)/factorial(\au+\ad+\at-1)};% normalizacion
\pgfmathsetmacro{\Bu}{factorial(\au-1)*factorial(\ad+\at-1)/factorial(\au+\ad+\at-1)};% normalizacion 1
\pgfmathsetmacro{\Bd}{factorial(\ad-1)*factorial(\au+\at-1)/factorial(\au+\ad+\at-1)};% normalizacion 2
\pgfmathsetmacro{\ma}{((\au-1)^(\au-1))*((\ad-1)^(\ad-1))*((\at-1)^(\at-1))/((\au+\ad+\at-3)^(\au+\ad+\at-3))/\B};
%
\begin{axis}[
    colormap = {whiteblack}{color(0cm)  = (white);color(1cm) = (black)},
    width=.5\textwidth,
    view={45}{65},
    enlargelimits=false,
    %grid=major,
    domain=0:1,
    y domain=0:1,
    zmax={.65*\ma},
    color=black,
    samples=70,
    xlabel=$x_1$,
    ylabel=$x_2$,
    zlabel=$p_{\widetilde{X}}$,
]
%
% Dirichlet
\addplot3 [surf,opacity=.8] {(x^(\au-1))*(y^(\ad-1))*(xplus(1-x-y)^(\at-1))/\B};
%
% Marginales
\addplot3 [domain=0:1,samples=50, samples y=0, thick, smooth, color=black] (x,1,{(x^(\au-1))*((1-x)^(\ad+\at-1))/\Bu});
\addplot3 [domain=0:1,samples=50, samples y=0, thick, smooth, color=black] (0,x,{(x^(\ad-1))*((1-x)^(\au+\at-1))/\Bd});%
%
\node at (axis cs:.5,1,{1/(2^(\au+\ad+\at-2))/\Bu})[right]{$p_{X_1}$};
\node at (axis cs:0,.5,{1/(2^(\au+\ad+\at-2))/\Bd})[above]{$p_{X_2}$};
\end{axis}
\end{scope}
%
\node at (1.2,-3){(a)};
\node at (6.6,-3){(b)};
\node at (13.6,-3){(c)};
\end{tikzpicture} \end{center}
%
\leyenda{Ilustraci\'on del  dominio $\Simp{k-1}$ de  definici\'on de la  ley de
  Dirichlet   para  \   $k  =   3$   \  (grise   oscuro),  con   el  dominio   \
  $(k-1)$-dimensional   \  $\Tri_{k-1}$   \  del   vector  \   $\widetilde{X}  =
  \protect\begin{bmatrix}   X_1  &  X_2   \protect\end{bmatrix}^t$  \   ($X_3  =
  1-X_1-X_2$) \ (grise claro) (a), y densidad de probabilidad de $\widetilde{X}$
  \   con   las   marginales   \   $p_{X_1},   \:   p_{X_2}$   (ver   notas   de
  pie~\ref{Foot:MP:DirichletDominio}   y~\ref{Foot:MP:DirichletDensidad}).   Los
  parametros   son    \   $a   =    \protect\begin{bmatrix}   3   &   2    &   2
    \protect\end{bmatrix}^t$  (b) y \  $a =  \protect\begin{bmatrix} 3  & 1  & 2
    \protect\end{bmatrix}^t$ (c).}
\label{Fig:MP:Dirichlet}
\end{figure}


Vectores  de  distribuci\'on  de  Dirichlet tienen  tambi\'en  unas  propiedades
notables, parecidas a las de la beta:
%
\begin{lema}[Reflexividad]\label{Lem:MP:ReflexividadDir}
%
  Sea  \ $X \,  \sim \,  \Dir(a), \:  a \in  \Rset_+^{* \,  k}$ \  y \  $\Pi \in
  \mathfrak{S}_k(\Rset)$ \ matriz \ de permutaci\'on. Entonces
  %
  \[
  \Pi X \, \sim \, \Dir\left( \Pi a \right)
  \]
  %
\end{lema}

%
\begin{proof}
  El resultado es inmediato por cambio  de variables $x \to \Pi x$, la Jacobiana
  siende   $\Pi$,   de   valor   absoluto   determinente  igual   a   $1$   (ver
  secci\'on~\ref{Sec:MP:Transformacion}).
\end{proof}
%
Adem\'as, se muestra una stabilidad remplazando dos componentes por su suma:
%
\begin{lema}[Stabilidad por agregaci\'on]\label{Lem:MP:StabSumaDir}
%
  Sea  \ $X =  \begin{bmatrix} X_1  & \cdots  & X_k  \end{bmatrix}^t \,  \sim \,
  \Dir(a),  \:  a =  \begin{bmatrix}  a_1 &  \cdots  &  a_k \end{bmatrix}^t  \in
  \Rset_+^{*  \,  k}$  \  y  \  $G^{(i,j)}$ \  matriz  de  agrupaci\'on  de  las
  $(i,j)$-\'esima componentes (ver notaciones). Entonces,
  %
  \[
  G^{(i,j)} X \, \sim \, \Dir\left( G^{(i,j)} a \right)  
  \]
  %
\end{lema}
%
\begin{proof}
  Se  puede probar  este resultado  a partir  de la  funci\'on caracter\'istica,
  usando      las      propiedades      de      la      funci\'on      confluent
  hipergeom\'etrica~\cite{SriKar85, Hum22, App25,  AppKam26, Erd37, Erd40}. Pero
  se  puede tambi\'en  tener un  enfoque m\'as  directo.  Del  lema precediente,
  notando que existe una matriz de  permutaci\'on \ $\Pi$ \ tal que \ $G^{(i,j)}
  = G^{(1,2)} \Pi$, se  puede concentrarse en el caso \ $(i,j)  = (1,2)$. Sea el
  cambio de  variables $g: x =  (x_1,\ldots,x_k) \mapsto u  = (u_1,\ldots,u_k) =
  (x_1,x_1+x_2,x_3,\ldots,x_k)$.        Entonces       \      $g^{-1}(u)       =
  (u_1,u_2-u_1,u_3,\ldots,u_k)$ \ es de determinente de matriz Jacobiana igual a
  \ $1$ \ dando para $U = g(X)$ \ la densidad
  %
  \[
  p_U(u)  = \frac{u_1^{a_1-1}  \left(  u_2 -  u_1 \right)^{a_2-1}  \prod_{i=3}^k
    u_i^{a_i-1}}{B(a)}
  \]
  %
  sobre $g\left(  \Simp{k-1} \right)$. Para $u_2  \in [0 \; 1]$  \ tenemos $u_1
  \in [  0 \;  u_2]$ \ as\'i  que, por  marginalizaci\'on en $u_1$  obtenemos la
  densidad
  %
  \begin{eqnarray*}
  p_{G^{(1,2)} X}(u_2,\ldots,u_k) & = & \frac{\prod_{i=3}^k u_i^{a_i-1}}{B(a)}
  \int_0^{u_2} u_1^{a_1-1} \left( u_2 - u_1 \right)^{a_2-1} \, du_1\\[2mm]
  %
  & = & \frac{\prod_{i=3}^k u_i^{a_i-1}}{B(a)} \, u_2^{a_1+a_2-1} \int_0^1
  v_1^{a_1-1} \left( 1 - v_1 \right)^{a_2-1} \, dv_1
  \end{eqnarray*}
  %
  con el cambio de variables $u_1 = u_2 v_1$. Se cierra la prueba notando que la
  integral   vale  \  $B(a_1,a_2)$   \  y   que  \   $\frac{B(a_1,a_2)}{B(a)}  =
  \frac{1}{B\left( G^{(1,2)} a \right)}$.
\end{proof}

De este lema, aplicado de manera recursiva, se obtiene en corolario siguiente:
%
\begin{corolario}
\label{Cor:MP:MarginalDirichletBeta}
%
  Sea  \ $X  \,  \sim \,  \Dir(a)$, entonces  \  $\displaystyle X_i  \, \sim  \,
  \beta\left( a_i \, , \, \alpha-a_i \right)$.
\end{corolario}

Naturalmente,  la ley de  Dirichelt siendo  una extensi\'on  de la  beta, existe
tambi\'en un v\'inculo entre esta ley y variables de distribuci\'on gamma:
%
\begin{lema}[V\'inculo con la ley gamma]
\label{Lem:MP:VinculoDirichletGamma}
%
Sea \ $X$ \ vector $k$-dimensional de componente \ $i$-\'esima \ $X_i \, \sim \,
\G(a_i,c), \: i = 1, \ldots , k$ \ independientes \ y \ $a$ vector de componente
$i$-\'esima \ $a_i$. Entonces
  %
  \[
  \frac{X}{\sum_{i=1}^k X_i} \, \sim \, \Dir(a)
  \]
  %
  (independientemente  de $c$).   Adem\'as, $\frac{X}{\sum_{i=1}^k  X_i}$ \  y \
  $\sum_{i=1}^k X_i$ \ son independientes.
\end{lema}
%
\begin{proof}
  La    prueba   sigue    exactamente   los    mismos   pasos    que    la   del
  lema~\ref{Lem:MP:VinculoBetaGamma} \ trabajando con \ $\widetilde{X}$.
\end{proof}

Naturalmente,  la  distribuci\'on de  Dirichlet,  extensi\'on  de  la ley  beta,
aparece entre  otros en problema  de inferencia Bayesiana como  distribuci\'on a
priori  conjugado~\footref{Foot:MP:BayesPrior}  del  parametro  $p$  de  la  ley
multinomial~\cite{Rob07}, extensi\'on de la ley binomial.

\SZ{
Polya urn schemes, Chinese restaurant
}


La distribuci\'on de Dirichlet se generaliza al caso matriz-variada $X$ definido
sobre $\P_{d,k}(\Rset)$,  conjuntos de  $k$-uplet de matrices  de $P_d^+(\Rset)$
cumpliando la relaci\'on  de completud (ver notaciones); se denota  \ $X \, \sim
\, \Dir_d(a)$  \ donde \  $a \in \left(  \frac{d-1}{2} \; +\infty  \right)^k$ la
densidad  est dada por  $\displaystyle p_X(x)  = \frac{\prod_{i=1}^k  \left| x_i
  \right|^{a_i-\frac{d+1}{2}}}{B_d(a)}$.   Se  refiera a~\cite[Cap.~6]{GupNag99}
para tener m\'as informaciones.

% --------------------------------- Student-t
\subsubseccion{Distribuci\'on Student-$t$ multivariada}
\label{Sssec:MP:Student}

En   el  caso   escalar,  esta   ley   fue  introducida   inicialmente  por   F.
R. Helmert~\cite{Hel75, Hel76, She95}  y J.  L\"uroth~\cite{Lur76, Pfa96}.  Pero
es m\'as  conocida por su  introducci\'on por William  Sealy Gosset~\footnote{De
  hecho,  Gosset  fue  un  estudiante  trabajando en  la  f\'abrica  de  cerveza
  irlandesa  Guiness  sobre estad\'istica  relacionada  a  la  qui\'imica de  la
  cerveza.   A pesar  que hay  varias  explicaciones sobre  el hecho  de que  se
  public\'o este trabajo bajo el nombre ``Student''. Unas es que fue para que no
  se  sabe que  la f\'abrica  estaba  trabajando sobre  estas estadisticas  para
  estudiar  la calidad  de  la cerveza~\cite{Wen16}.\label{Foot:MP:Student}}  en
1908,  trabajando  sobre variables  centradas  normalizadas  por  el promedio  y
varianza empiricos~\cite{Stu08}.   Fue estudiada entre  otros intensivamente por
el  famoso matematico  R. Fisher~\cite{Fis25}.   En la  literatura, esta  ley es
conocida bajo  los nombres {\em  Student}, {\em Student-$t$} o  simplemente {\em
  $t$-distribuci\'on} o  a\'un bajo el nombre  {\em Pearson tipo IV}  en el caso
escalar  y {\em  Pearson tipo  VII}  (para $\frac{\nu+d}{2}$  entero; ver  m\'as
abajo), debido  a la  familia de Pearson~\cite{Pea95,  JohKot95:v1, JohKot95:v1,
  KotBal00, FanKot90}.  Esta distribuci\'on aparece como a  priori conjugado del
promedio de una gausiana en inferencia bayesiana~\cite{Rob07, KotNad04}.

Se denota con \ $X \sim t_\nu(m,\Sigma)$  \ con \ $m \in \Rset^d$, \ $\Sigma \in
P_d^+(\Rset)$  \  conjunto  de  las   matrices  de  \  $\Rset^{d  \times  d}$  \
s\'imetricas definidas positivas. $m$ \ es llamado {\em par\'ametro de posici\'on}
(no es  la media  que puede  no existir), \  $\Sigma$ \  es llamada  {\em matriz
  caracter\'istica} (no es [proporcional a]  la covarianza que puede no existir)
y \ $\nu > 0$ \ llamado  {\em grados de libertad}.  Las caracter\'isticas de una
Student-$t$ son las siguientes:
%
\begin{caracteristicas}
%
Dominio de definici\'on & $\X = \Rset^d$\\[2mm]
\hline
%
Par\'ametro & $\nu \in \Rset_+^*$ \ (grados de libertad), \ $m \in \Rset^d$ \
(posici\'on), \ $\Sigma \in P_d^+(\Rset)$ \ (matriz caracer\'istica)\\[2mm]
\hline
%
Densidad de probabilidad & $\displaystyle p_X(x) = \frac{\Gamma\left(
\frac{\nu+d}{2} \right)}{\pi^{\frac{d}{2}} \nu^{\frac{d}{2}} \Gamma\left(
\frac{\nu}{2} \right) \, \left| \Sigma \right|^{\frac12}} \, \left( 1 +
\frac{(x-m)^t \Sigma^{-1} (x-m)}{\nu} \right)^{- \, \frac{\nu+d}{2}}$\\[2mm]
\hline
%
Promedio & $\displaystyle m_X = m$ \ si \ $\nu > 1$; \ no
existe si no~\footnote{De manera general, esta ley admite momentos de orden \ $k$ \
si y solamente si \ $\nu > k$.\label{Foot:MP:ExistenciaMomentosStudent}}.\\[2.5mm]
\hline
%
Covarianza~\footnote{Fijense de que $\Sigma$ no es la covarianza, pero es
proporcional a la covarianza\ldots cuando existe. Se podr\'ia imaginar
renormalizar la ley tal que \ $\Sigma_X$ \ y \ $\Sigma$ \ coinciden, pero no
ser\'ia posible en el caso \ $\nu \le 2$.} & $\displaystyle \Sigma_X =
\frac{\nu}{\nu-2} \, \Sigma$ \ si \ $\nu > 2$; \ no existe si
no~\footref{Foot:MP:ExistenciaMomentosStudent}.\\[2.5mm]
\hline
%
\modif{Asimetr\'ia} (caso escala) & $\displaystyle \gamma_X = 0$ \ si \ $\nu > 3$; \ no
existe si no~\footref{Foot:MP:ExistenciaMomentosStudent}.\\[2mm]
\hline
%
Curtosis por exceso & $\displaystyle \widebar{\kappa}_X = \frac{6}{\nu-4}$ \ si
\ $\nu > 4$; \ no existe si no~\footref{Foot:MP:ExistenciaMomentosStudent}.\\[2mm]
\hline
%
Funci\'on caracter\'istica~\footnote{Se muestra sencillamente que la funci\'on
generatriz de momentos puede existir si y solamente si \ $\real{u} = 0$. La
funci\'on genetratriz de momentos restricta al producto cartesiano de bandas \
$\real{u} = 0$ \ es nada m\'as que la funci\'on caracter\'istica. Adem\'as, esta
funci\'on fue calculdada, especialmente en el caso multivariado, relativamente
recientemente~\cite{Sut86, Hur95, KibJoa06, SonPar14}.} & $\displaystyle
\Phi_X(\omega) = \frac{\nu^{\frac{\nu}{4}}}{2^{\frac{\nu}{2}-1} \Gamma\left(
\frac{\nu}{2} \right)} \, e^{\imath \omega^t m} \, \left( \omega^t \Sigma \omega
\right)^{\frac{\nu}{4}} K_{\frac{\nu}{2}}\left( \sqrt{\nu \, \omega^t \Sigma
\omega} \right)$
\end{caracteristicas}

Nota: nuevamente se puede escribir $X \, \egald \, \Sigma^{\frac12} S + m$ \ con
\ $S \, \sim \, t_\nu(0,I)$ \  donde \ $S$ \ es dicha {\em Student-$t$ estandar}
y  las caracter\'isticas  de \  $X$  \ son  v\'inculadas a  las  de \  $S$ \  (y
vice-versa) por transformaci\'on lineal (ver secciones anteriores).

La densidad de probabilidad Student-$t$ estandar y la funci\'on de repartici\'on
en el caso escalar  son representadas en la figura Fig.~\ref{Fig:MP:Student}-(a)
y    (b)   y    una   densidad    en   un    contexto    bi-dimensional   figura
Fig.~\ref{Fig:MP:Student}(c).
%
\begin{figure}[h!]
%\begin{center} \input{TIKZ_MP/Student} \end{center}
% 
\leyenda{Ilustraci\'on  de  una  densidad  de probabilidad  Student-$t$  escalar
  estandar (a),  y la funci\'on  de repartici\'on asociada  (b) con \ $\nu  = 1$
  (linea llena), \ $\nu = 3$ (linea  guionada), \ $\nu = 7$ (linea punteada) \ y
  \ $\nu \to +\infty$ (linea llena fina; ver m\'as adelante) grados de libertad,
  as\'i que una densidad de probabilidad Student-$t$ bi-dimensional con \ $\nu =
  1$ \  grado de libertad,  centrada, y de  matriz caracter\'istica \  $\Sigma =
  R(\theta) \Delta^2  R(\theta)^t$ \ con \  $R(\theta) = \protect\begin{bmatrix}
    \cos\theta    &     -    \sin\theta\\[2mm]    \sin\theta     &    \cos\theta
    \protect\end{bmatrix}$   \   matriz   de    rotaci\'on   y   \   $\Delta   =
  \diag\left(\protect\begin{bmatrix}  1   &  a\protect\end{bmatrix}  \right)$  \
  matriz  de   cambio  de  escala,   y  sus  marginales   \  $X_1  \,   \sim  \,
  t_\nu\left(0,\cos^2\theta +  a^2 \sin^2\theta \right)$ \  y \ $X_2  \, \sim \,
  t_\nu\left(0,\sin^2\theta   +   a^2  \cos^2\theta   \right)$   \  (ver   m\'as
  adelante). En la figura, $a = \frac13$ \ y \ $\theta = \frac{\pi}{6}$.}
\label{Fig:MP:Student}
\end{figure}

Nota: el caso  \ $\nu = 1$ \  es conocido como distribuci\'on de  {\em Cauchy} o
{\em Cauchy  Breit-Wigner}~\cite{SamTaq94, toto,  titi}.  Es un  caso particular
tambi\'en de distrubuci\'on $\alpha$-estables~\cite{SamTaq94}.

Contrariamente al caso gaussiano, de la forma de la densidad de probabilidad, es
claro que si la matriz \ $\Sigma$  \ es diagona, la densidad no factoriza, as\'i
que  las componentes  del vector  no  son independientes.  Este ejemplo  muestra
claramente que  la reciproca del lema~\ref{Lem:MP:IndependenciaCov}  es falsa en
general.

Sin embargo, las distribuciones Student-$t$ tienen varias propiedades notables.

\begin{lema}[Stabilidad por transformaci\'on lineal]
\label{Lem:MP:StabilidadLineal}
%
  Sea \ $X  \, \sim \, t_\nu(m,\Sigma)$,  \ $A$ \ matriz de  \ $\Rset^{d' \times
    d}$ \ con \ $d' \le d$, y de rango lleno y \ $b \in \Rset^{d'}$. Entonces
  %
  \[
  A X + b\, \sim \, t_\nu( A m + b , A \Sigma A^t)
  \]
  %
  En particular los componentes de \ $X$ \ son student-$t$,
  %
  \[
  X_i \, \sim \, t_\nu(m_i , \Sigma_{i,i} )
  \]
\end{lema}
\begin{proof}
  La prueba es inmediata usando  la funci\'on caracter\'istica y sus propiedades
  por  transformaci\'on lineal.  La condici\'on  sobre \  $A$ \  es  necesaria y
  suficiente para que \ $A \Sigma A^t \in P_{d'}^+(\Rset)$.
\end{proof}

\begin{lema}[V\'inculo con las distribuciones Gamma y Gausiana (mezcla Gaussiana de escala)]
\label{Lem:MP:MezclaGaussianaEscalaStudent}
%
  Sea \ $V \sim \G\left( \frac{\nu}{2} \, ,  \, \frac{\nu}{2} \right)$ \ y \ $G \, \sim \,
  \N(0,I)$ \ independientes. Entonces
  %
  \[
  \frac{G}{\sqrt{V}} \, \sim \, t_\nu( 0 , I )
  \]
  %
  Dicho de  otra manera, se  puede escribir \  $X \, \sim \,  t_\nu(m,\Sigma)$ \
  esticasticamente   bajo   la    forme   \   $X   \egald   \sqrt{\frac{\nu}{V}}
  \Sigma^{\frac12} G + m $ \ donde  \ $\egald$ \ significa que la igualdad es en
  distribuci\'on.
\end{lema}
\begin{proof}
  Lo  m\'as  simple es  de  salir  de la  formula  de  probabilidad total  vista
  pagina~\pageref{:MP:}, notando que condicionalmente a \ $V=v$ \ la variable es
  gausiana de covarianza $\frac{1}{\sqrt{v}} I$,
%
\begin{eqnarray*}
p_X(x) & = & \int_\Rset p_{X|V=v}(x) \, p_V(v) \, dv\\[2mm]
%
& \propto & \int_0^{+\infty} v^{\frac{d}{2}} e^{-\frac{v}{2} x^t x}
v^{\frac{\nu}{2}-1} e^{-\frac{\nu}{2} v} \, dv\\[2mm]
%
& \propto & \left( 1 + \frac{x^t x}{\nu} \right)^{- \frac{d+\nu}{2}}
\int_0^{+\infty} u^{\frac{d+\nu}{2}-1} e^{-u} \, du\\[2mm]
%
& \propto & \left( 1 + \frac{x^t x}{\nu} \right)^{- \frac{d+\nu}{2}}
\end{eqnarray*}
%
con  \ $\propto$  \ significando  ``proporcional a''  (el coeficiente  es  lo de
normalizaci\'on) y  el cambio de  variables $v  = \frac{2 \,  u}{\nu + x^t  x} =
\frac{\frac{2}{\nu}}{1 + \frac{x^t x}{\nu}} \, u$.
\end{proof}
%
Nota: este  lema permite tambi\'en  probar el lema~\ref{Lem:MP:StabilidadLineal}
escribiendo \ $A X + b \egald  \sqrt{\frac{\nu}{V}} A \Sigma^{\frac12} G + A m +
b$.

\begin{lema}[L\'imite Gausiana]
\label{Lem:MP:LimiteGaussiana}
%
  Sea \ $X_\nu \, \sim \, t_\nu(m,\Sigma)$ \ vector Student-$t$ parametizado por
  \ $\nu$ \ sus grados de libertad. Entonces
  %
  \[
  X_\nu \, \limitd{\nu \to \infty} \, = \, X \, \sim \, \N(m,\Sigma)
  \]
  %
  con \ $\displaystyle \limitd{}$ \ l\'imite es en distribuci\'on.
\end{lema}
\begin{proof}
  La prueba  es inmediata tomando el  logaritmo de la  densidad de probabilidad,
  usando      la     formula      de     Stirling~\footnote{Ver      nota     de
    pie~\footref{Foot:MP:Stirling}} para  \ $\log\Gamma(z) = \left(  z - \frac12
  \right)  \log  z  -  z  +  \frac12   \log(2  \pi)  +  o(1)$  \  en  \  $z  \to
  +\infty$~\cite{Sti30, AbrSte70, GraRyz15} \ y \ $-\frac{d+\nu}{2} \log\left( 1
    +  \frac{(x-m)^t \Sigma^{-1} (x-m)}{\nu}  \right) =  -\frac{d+\nu}{2} \left(
    \frac{(x-m)^t \Sigma^{-1} (x-m)}{\nu} + o\left( \nu^{-1} \right) \right) = -
  \frac{(x-m)^t \Sigma^{-1} (x-m)}{2} + o(1)$.
\end{proof}

Las   variables   Student-t   tienen   varias   representaciones   estocasticas,
relacionadas a la gausiana~\cite{FanKot90, And03, KotNad04}:
%
\begin{lema}[Relaci\'on con la distribuci\'on Gamma]\label{Lem:MP:StudentGamma}
%
  Sea \ $G \,  \sim \, \G\left( \frac{\nu}{2} , \frac{\nu}{2} \right)$  \ y \ $Y
  \,  \sim  \, \N(0,I)$  \  con  $\nu >  0$  \  e \  $Y$  \  independiente de  \
  $G$. Entonces, para $\Sigma \in P_d^+(\Rset)$ y $m \in \Rset^d$,
  %
  \[
  \frac{\Sigma^{\frac12} Y}{\sqrt{G}} + m  \, \sim \, t_\nu(m,\Sigma)
  \]
  %
\end{lema}
\begin{proof}
  Sea \ $X = \frac{Y}{\sqrt{G}}$. De la nota siguiendo la tabla de caracter\'isticas
  es  necesario  y suficiente  probar  que $X  \sim  t_\nu(0,I)$.  Ahora, de  la
  independencia tenemos
  %
  \[
  p_{X|G=g}(x)  = (2  \pi)^{-\frac{d}{2}}  g^{\frac{d}{2}} e^{-  \frac{x^t x g}{2}}
  \]
  %
  Entonces, multiplicando \ $p_{X|G=g}$ \ por \ $p_G$ \ y por marginalizaci\'on,
  obtenemos
  %
  \begin{eqnarray*}
  p_X(x) & = & \frac{\nu^{\frac{\nu}{2}}}{2^{\frac{\nu+d}{2}} \pi^{\frac{d}{2}}
  \Gamma\left( \frac{\nu}{2} \right)} \, \int_{\Rset_+} g^{\frac{\nu+d}{2}-1} \,
  e^{- \frac{x^t x + \nu}{2} \, g} \, dg\\[2mm]
  %
  & = & \frac{\nu^{\frac{\nu}{2}} \left( \nu + x^t x \right)^{-
  \frac{\nu+d}{2}}}{\pi^{\frac{d}{2}} \Gamma\left( \frac{\nu}{2} \right)} \,
  \int_{\Rset_+} u^{\frac{\nu+d}{2}-1} \, e^{- u} \, du\\[2mm]
  %
  & = & \frac{\Gamma\left( \frac{\nu+d}{2} \right)}{(\pi \nu)^{\frac{d}{2}}
  \Gamma\left( \frac{\nu}{2} \right)} \, \left( 1 + \frac{x^t x}{\nu} \right)^{-
  \frac{\nu+d}{2}}
  \end{eqnarray*}
 %
  La secunda linea viene del cambio de variables \ $u = \frac{x^t x + \nu}{2} \,
  g$  \  y la  tercera  reconociendo  en la  integral  la  funci\'on Gamma  (ver
  notaciones).
\end{proof}
%
\begin{lema}[Relaci\'on con la distribuci\'on de Wishart]\label{Lem:MP:StudentWishart}
%
  Sea \  $W \, \sim \,  \W(I,\nu+d-1)$ \ $d \times  d$ Wishart, \ $Y  \, \sim \,
  \N(0,\nu I)$  \ con $\nu >  0$ \ e \  $Y$ \ independiente de  \ $W$. Entonces,
  para $\Sigma \in P_d^+(\Rset)$ y $m \in \Rset^d$,
  %
  \[
  \Sigma^{\frac12} W^{-\frac12} Y + m  \, \sim \, t_\nu(m,\Sigma)
  \]
  %
\end{lema}
\begin{proof}
  Sea \ $X = W^{-\frac12} Y$. De la nota siguiendo la tabla de caracter\'isticas
  es  necesario  y suficiente  probar  que $X  \sim  t_\nu(0,I)$.  Ahora, de  la
  independencia tenemos
  %
  \[
  p_{X|W=w}(x)  = (2  \pi \nu)^{-\frac{d}{2}}  |w|^{\frac12} e^{-  \frac{x^t w  x}{2
      \nu}}
  \]
  %
  Denotamos  por \  $D =  \left\{ w_{ij},  \: 1  \le j  \le i  \le d  \tq  w \in
    P_d^+(\Rset) \right\}$ \ y, por abuso de  escritura, \ $dv = \prod_{ 1 \le j
    \le i \le d} dw_{ij}$.  Entonces,  multiplicando \ $p_{X|W=w}$ \ por \ $p_W$
  \ y por marginalizaci\'on, obtenemos
  % ($\propto$ significa ``proporcional  a'', i.e., olvidando el coefficiente de
  % normalizaci\'on)
  %
  \begin{eqnarray*}
  p_X(x) & = & \int_D \frac{|w|^{\frac{\nu-1}{2}} e^{- \frac{x^t w x}{2 \nu} -
  \frac12 \Tr(w)}}{2^{\frac{d (\nu+d)}{2}} (\pi \nu)^{\frac{d}{2}} \Gamma_d \left(
  \frac{\nu+d-1}{2} \right)} \, dw\\[2mm]
  %
  & = & \frac{\Gamma\left( \frac{\nu+d}{2} \right)}{(\pi \nu)^{\frac{d}{2}}
  \Gamma\left( \frac{\nu}{2} \right)} \left| I + \frac{x x^t}{\nu}
  \right|^{-\frac{\nu+d}{2}} \: \int_D \frac{|w|^{\frac{\nu+d-d-1}{2}} e^{-
  \frac12 \Tr\left( \left[ I + \frac{x x^t}{\nu} \right] w \right)}}{2^{\frac{d
  (\nu+d)}{2}} \left| \left( I + \frac{x x^t}{\nu} \right)^{-1}
  \right|^{\frac{\nu+d}{2}} \Gamma_d \left( \frac{\nu+d}{2} \right)} \, dw\\[2mm]
  %
  & = & \frac{\Gamma\left( \frac{\nu+d}{2} \right)}{(\pi \nu)^{\frac{d}{2}}
  \Gamma\left( \frac{\nu}{2} \right)} \: \left( 1 + \frac{x^t x}{\nu}
  \right)^{-\frac{\nu+d}{2}} \, \int_D \frac{|w|^{\frac{\nu+d-d-1}{2}} e^{-
  \frac12 \Tr\left( \left[ I + \frac{x x^t}{\nu} \right] w \right)}}{2^{\frac{d
  (\nu+d)}{2}} \left| \left( I + \frac{x x^t}{\nu} \right)^{-1}
  \right|^{\frac{\nu+d}{2}} \Gamma_d \left( \frac{\nu+d}{2} \right)} \, dw
  \end{eqnarray*}
%
  Para  \ $a, b  \in \Rset^d,  \: M  \in \M_{d,d}(\Rset)$,  en la  secunda linea
  usamos la  identidad \  $a^t M b  = \Tr(b a^t  M)$ \  y \ $\Gamma_d\left(  x -
    \frac12 \right) = \frac{\Gamma\left(  x - \frac{d}{2} \right)}{\Gamma(x)} \,
  \Gamma_d(x)$ \ (ver  notaciones) y en la tercera linea  usamos la identidad de
  Sylvester~\cite{Syl51} o~\cite[\S~18.1]{Har08} \ $\left|  I + a b^t\right| = 1
  + b^t a$. Se concluye que \ $X \sim t_\nu(0,I)$ \ reconociendo en el factor de
  la integral  como la  distribuci\'on \  $t_\nu(0,I)$ \ y  en el  integrande la
  distribuci\'on de Wishart \ $\W(I,\nu+d)$ \ que suma entonces a la unidad.
  %  la  formula de  Sherman-Morrison-Woodbury  $\left(  I  + \frac{x  x^t}{\nu}
  % \right)^{-1} = I$~\cite{HorJoh13, Har08}
\end{proof}

Como lo hemos introducido, la  distribuci\'on de Student aparece naturalmente en
el  marco  de la  estimaci\'on,  especialmente  a  trav\'es de  la  estimaci\'on
empirica de la media y covarianza~\cite{Mui82, GupNad99, BilBre99, And03, Seb04}:
% resp. p 80 teo 3.2.1 -- p. 92 teo 3.3.6 -- p. 87 prop. 7.1 -- p. 77 teo. 3.3.2 -- p. 63 teo. 3.1
% Nota : ver corolarios 2 y 3, p. 25 de Seber
% VER GupNad Th. 4.2.1
%
\begin{teorema}%[]
%
  Sean  \  $X_i \,  \sim  \,  \N(m,\Sigma), \:  i  =  1, \ldots  ,  n  > d-1$  \
  independientes,       y      sea       la       media      empirica       (ver
  corolario~\ref{Cor:MP:MediaEmpiricaGauss})
  %
  \[
  \overline{X} = \frac{1}{n} \sum_{i=1}^n X_i
  \]
  %
  y  la  covarianza empirica  construida  a partir  de  la  media empirica  (ver
  corolario~\ref{Cor:MP:WishartEstimacion})
  %
  \[
  \overline{\Sigma}  =  \frac{1}{n-1}  \sum_{i=1}^n  \left( X_i  -  \overline{X}
  \right) \left( X_i - \overline{X} \right)^t
  \]
  %
  Entonces:
  %
  \begin{itemize}
  \item  $\overline{X} - m \,  \sim \,  \N\left(  0 \,  , \,  \frac{1}{n} \,  \Sigma
    \right)$ \ y  \ $\overline{\Sigma} \, \sim \,  \W( \Sigma \, , \, n-1  ) $ \
    son independientes;
  %
  \item \SZ{$\overline{\Sigma}^{-\frac12} \, \left(  \overline{X} - m \right) \,
      \sim \, t_{\nu}\left( \right)$}
  \end{itemize}
\end{teorema}
%
\begin{proof}
  Se      refiera      a     los      corolarios~\ref{Cor:MP:MediaEmpiricaGauss}
  y~\ref{Cor:MP:WishartEstimacion}   por   lo  de   las   distribuciones  de   \
  $\overline{X}-m$ \ y de \ $\overline{\Sigma}$ \ respectivamente.

  A continuaci\'on,  sean \  $\widetilde{X}_i =  X_i - m$  \ y  \ $\widetilde{X}
  =        \begin{bmatrix}        \widetilde{X}_1        &       \cdots        &
    \widetilde{X}_n \end{bmatrix}$. Obviamente
  %
  \[
  \overline{X} - m = \widetilde{X} \un
  \]
  %
  con \ $\un \in  \Rset^n$ \ vector de componentes iguales a \  $1$ \ y vimos en
  la prueba del corolario~\ref{Cor:MP:WishartEstimacion} que
  %
  \[
  \overline{\Sigma} = \frac{1}{n-1} \widetilde{X} \left( I - \frac{\un \un^t}{n}
  \right) \widetilde{X}^t
  \]
  %
  $A = I - \frac{\un \un^t}{n} \in  P_n(\Rset)$ \ es idemponenta de rango 1, con
  $A  \un = 0$,  as\'i que  por diagonalizaci\'on~\cite{HorJoh13,  Bat97, Bat07}
  tenemos
  %
  \[
  A = P \begin{bmatrix} I_{n-1} & 0\\ 0 & 0 \end{bmatrix} P^t \qquad \mbox{con} \qquad
  P = \begin{bmatrix} B & \frac{1}{\sqrt{n}} \un \end{bmatrix}
  \]
  %
  $P P^t = P P^t = I$ \ y
  %
  \[
  B \in \M_{n,n-1}(\Rset) \quad \mbox{tal que} \quad  B^t B = I \: \mbox{ y } \:
  \un^t B = 0
  \]
  %
  Ahora,   poniendo   la  descomposici\'on   diagonal   de   \   $A$  \   en   \
  $\overline{\Sigma}$ \ obtenemos (ver~corolario~\ref{Cor:MP:WishartEstimacion})
  %
  \[
  \overline{\Sigma} = \frac{1}{n-1} \, Y Y^t \qquad \mbox{con} \qquad Y = \widetilde{X} B
  \]
  %
  Luego, de la gausianidad y independencia de los \ $\widetilde{X}_i$ \ tenemos,
  para \   $\widetilde{x}   =    \begin{bmatrix}   \widetilde{x}_1   &   \cdots   &
    \widetilde{x}_n \end{bmatrix} \in \M_{d,n}(\Rset)$
  %
  \begin{eqnarray*}
  p_{\widetilde{X}}(\widetilde{x}) & = & (2 \pi)^{-\frac{n d}{2}}
  |\Sigma|^{-\frac{n}{2}} \exp\left(- \frac12 \sum_{i=1}^n \widetilde{x}_i^t
  \Sigma^{-1} \widetilde{x}_i \right)\\[2mm]
  %
  & = & (2 \pi)^{-\frac{n d}{2}} |\Sigma|^{-\frac{n}{2}} \exp\left(- \frac12
  \sum_{i=1}^n \Tr\left( \Sigma^{-1} \widetilde{x}_i \widetilde{x}_i^t
  \right) \right)\\[2mm]
  %
  & = & (2 \pi)^{-\frac{n d}{2}} |\Sigma|^{-\frac{n}{2}} \exp\left(- \frac12
  \Tr\left( \Sigma^{-1} \widetilde{x} \widetilde{x}^t \right) \right)
  \end{eqnarray*}
  %
  Sea    la     transformaci\'on    \    $\begin{bmatrix}     Y    &    \sqrt{n}
    \overline{\widetilde{X}}   \end{bmatrix}   =   \widetilde{X}   P$,   \ie   \
  $\widetilde{X}        =       \begin{bmatrix}        Y        &       \sqrt{n}
    \overline{\widetilde{X}} \end{bmatrix}  P^t$.  Se nota que  \ $|P| =  1$ \ y
  por                            transformaci\'on                           (ver
  teorema~\ref{Teo:MP:TransformacionInyectivaDensidad}),    para   \    $y   \in
  \M_{d,n-1}(\Rset)$ \ y \ $x \in \Rset^d$
  %
  \begin{eqnarray*}
  p_{Y,\sqrt{n} \overline{\widetilde{X}}}(y,x) & = & (2 \pi)^{-\frac{n d}{2}}
  |\Sigma|^{-\frac{n}{2}} \exp\left(- \frac12 \Tr\left(
  \Sigma^{-1} \begin{bmatrix} y & x \end{bmatrix} P^t P \begin{bmatrix} y^t\\
  x \end{bmatrix}\right) \right)\\[2mm]
  %
  & = & (2 \pi)^{-\frac{n d}{2}} |\Sigma|^{-\frac{n}{2}} \exp\left(- \frac12
  \Tr\left( \Sigma^{-1} \left( y y^t + x x^t \right) \right) \right)\\[2mm]
  %
  & = & (2 \pi)^{-\frac{(n-1) d}{2}} |\Sigma|^{-\frac{n-1}{2}} \exp\left(-
  \frac12 \Tr\left( \Sigma^{-1} y y^t \right) \right) \times (2
  \pi)^{-\frac{d}{2}} |\Sigma|^{-\frac12} \exp\left(- \frac12 x^t \Sigma^{-1} x
  \right)
  \end{eqnarray*}
  %
  Claramente,  de la factorizaci\'on  de las  distribuciones, $Y  = X  B$ \  y \
  $\sqrt{n}  \overline{\widetilde{X}}$  \ son  independientes,  es  decir que  \
  $\frac{1}{n-1} \, Y Y^t = \overline{\Sigma}$ \ y \ $\overline{\widetilde{X}} =
  \overline{X} -  m$ \ son  independientes, lo que  cierra la prueba  del primer
  item.  Pasando, la forma  de $p_{Y,\sqrt{n}  \overline{\widetilde{X}}}(y,x)$ \
  confirma  que  \ $\overline{X}-m$  \  es  gausiana  centrada de  covarianza  \
  $\frac{1}{n} \,  \Sigma$, y  que los \  $Y_i$ \ son  independientes gausianos,
  dando la distribuci\'on  de Wishart del lema~\ref{Lem:MP:WishartGausiana} paea
  la covarianza empirica.


\end{proof}

\SZ{
%corolario suma dis independientes... via condicionalmente a...

Sampling distribution, Gosset, Fisher 25.

Applications
}

M\'as propiedades de esta distribuci\'on se encuentran en libros especializados,
por ejemplo~\cite{KotNad04} completamente dedicado a esta distribuci\'on.

\

La distribuci\'on  Student-t se generaliza  al caso matriz-variada  $X$ definido
sobre $M_{d,d'}(\Rset)$;  se denota  \ $X \,  \sim \,  t_\nu(M,\Sigma,\Omega)$ \
donde  \ $M  \in  M_{d,d'}(\Rset), \:  \Sigma  \in P_d^+(\Rset),  \: \Omega  \in
P_{d'}^+(\Rset)$   y  la  densidad   est  dada   por  $\displaystyle   p_X(x)  =
\frac{\Gamma_d\left(      \frac{\nu+d+d'-1}{2}\right)}{\pi^{\frac{\nu     d}{2}}
  \Gamma_d\left(       \frac{\nu+d-1}{2}\right)      \,       \left|      \Sigma
  \right|^{\frac{d'}{2}}  \left|  \Omega \right|^{\frac{d}{2}}}  \:  \left| I  +
  \Sigma^{-1}  (x-M) \Omega^{-1}  (x-M)^t \right|^{-  \frac{\nu+d+d'-1}{2}}$. Se
refiera a~\cite[Cap.~4]{GupNag99} para tener m\'as informaciones.


\SZ{
% --------------------------------- Familia exponencial
\subsubseccion{Familia  exponencial}
\cite{Dar35, Koo36,  And70,  Kay93, LehCas98,  Rob07}.;
Muchas de estas leyes entran en una familia que juega un rol particular en problema de maximizacion de entropie (ver cap 2): es la familia exponencial...
}

\SZ{
% --------------------------------- Familia eliptica
\subsubseccion{Familia eliptica}
Invariante por rotacion... GSM
}


\SZ{Teorema del l\'imite central, y relajando la independencia, y versiones con leyes diferentes pero uniformamente acotadas.}

\SZ{hablar de  simulaci\'on? Metoto inverso,  mezcla, rejeccion, a traves  de la
  condicional para el caso vectorial?}


