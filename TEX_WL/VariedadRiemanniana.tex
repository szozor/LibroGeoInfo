%\sub
\seccion{Variedad Riemmanniana}
\label{s:WL:VariedadRiemmanniana}

Sea $\M$ una variedad diferenciable  $n$-dimensional. Si $\M$ tiene definida una
m\'etrica  no   singular  sobre  ella,   recibe  el  nombre  de   {\it  variedad
  Riemanniana}. La existencia de una m\'etrica sobre $\M$ permite introducir una
conexi\'on af\'in  particular, conocida como la conexi\'on  de Levi-Civita. Sean
$g_{ab}$ y  $g^{ab}$ los coeficientes de la  m\'etrica $g$ y su  inversa, en las
coordenadas $\{x^a\}$, respectivamente.

Para  dos puntos  pr\'oximos,  la separaci\'on  entre  ellos viene  dada por  la
expresi\'on:
%
\begin{equation}
ds^2 = \sum_{ab} g_{ab} dx^a dx^b \label{quad}
\end{equation}

Adem\'as de definir una distancia  entre puntos pr\'oximos, la existencia de una
m\'etrica  permite   definir  una  conexi\'on  particular   sobre  una  variedad
riemanniana:
%
\begin{definicion}[Conexi\'on de Levi-Civita]
  La conexi\'on de Levi-Civita en las coordenadas $x^a$ est\'a dada por:
  %
  \begin{equation}
  \Gamma^a_{bc} = \frac{1}{2} \sum_d g^{ad} \left(g_{bd,c} + g_{cd.b} - g_{bc.d} \right) \label{levi}
  \end{equation}
\end{definicion}

La existencia  de esta  particular conexi\'on no  imposibilita la  existencia de
otras conexiones definidas sobre $\M$.

Como hemos visto  m\'as arriba, el tener definida  una m\'etrica permite definir
la  longitud de  una curva.  Bajo ciertas  condiciones, que  supondremos  que se
satisfacen, podemos plantearnos el problema  de determinar la curva que minimiza
(en  realidad  extremiza)  su  longitud  al  unir  dos  puntos  fijos  sobre  la
variedad. Esto se puede tratar  resolviendo el problema variacional asociado con
el funcional~\eqref{longitud}. La ecuaci\'on de Euler-Lagrange  conduce en este
caso a:
%
\begin{equation}
\frac{d^2 x^d}{dt^2} + \Gamma^d_{ca} \frac{dx^c}{dt} \frac{dx^a}{dt} = 0 \label{geode}
\end{equation}
%
\noindent donde $x^a(t)$ son las coordenadas de la curva y $t$ es un par\'ametro
adecuadamente elegido. Una curva  que satisface~\eqref{geode}, se llama una {\it
  curva geod\'esica}.  Es posible caracterizar  a una curva geod\'esica  de otro
modo. Sea  $\mathbb{U}(t)$ el vector  tangente a una curva  $\gamma(t)$ definida
sobre $\M$.  La curva $\gamma$ se  dice una geod\'esica si su vector tangente es
trasladado paralelamente a lo largo de ella:
%
\[
\mathbb{U} \cdot \nabla \mathbb{U} = f(t) \mathbb{U}
\]
% 
% hdot => cdot?
Siempre es posible  elegir al par\'ametro $t$ de forma tal  que $f(t)=0$, con lo
cual reobtenemos la ecuaci\'on~\eqref{geode}.

El disponer de geod\'esicas, permite dar a una variedad riemanniana el car\'cter
de espacio m\'etrico.  En efecto, podemos definir la  distancia entre dos puntos
$p$ y $q$ de la variedad $\M$ a trav\'es de la expresi\'on:
%
\begin{equation}
d(p,q) = \min_{\gamma} L(\gamma) \label{distt}
\end{equation}
%
\noindent donde  el m\'inimo  se eval\'ua  entre todas las  curvas que  unen los
puntos $p$ y $q$, y $L$ es la longitud~\eqref{longitud}. Como siempre, todo esto
es  posible ser  realizado  localmente.   Las geod\'esicas  son  las curvas  que
localmente  minimizan  la distancia  entre  dos  puntos.  La distancia  definida
por~\eqref{distt}  verifica  la  desigualdad   triangular,  y  por  eso  es  una
m\'etrica.

Dada una conexi\'on  $\nabla$ se define un tensor de  tipo $(1,3)$, llamado {\it
  tensor de curvatura} asociado a la conexi\'on $\nabla$, cuya espresi\'on es:
%
\[
\R\left( \mathbb{X} , \mathbb{Y} \right) \mathbb{Z} = \nabla_{\mathbb{X}} \left(
  \nabla_{\mathbb{Y}}   \mathbb{Z}    \right)   -   \nabla_{\mathbb{Y}}   \left(
  \nabla_{\mathbb{X}}  \mathbb{Z}   \right)  -  \nabla_{[\mathbb{X},\mathbb{Y}]}
\mathbb{Z}
\]
%
Si los  vectores $\mathbb{X}$, $\mathbb{Y}$ y $\mathbb{Z}$  son reemplazados por
los       vectors       coordenados       $\frac{\partial}{\partial       x^a}$,
$\frac{\partial}{\partial     x^b}$    y     $\frac{\partial}{\partial    x^c}$,
respectivamente, resulta
%
\[
\R\left( \frac{\partial}{\partial  x^a} , \frac{\partial}{\partial  x^b} \right)
\frac{\partial}{\partial x^c} = R^d_{cab} \frac{\partial}{\partial x^d}
\]
%
con
%
\[
R^d_{cba}   \equiv  \left(  \frac{   \partial  \Gamma^d_{cb}}{\partial   x^a}  +
  \Gamma^d_{ra} \Gamma^r_{cb}  - \frac{ \partial  \Gamma^d_{ca}}{\partial x^b} -
  \Gamma^d_{rb} \Gamma^r_{ca}\right) \frac{\partial}{\partial x^d}
\]

{\bf  Nota:} Si  bien existe  una motivaci\'on  geom\'etrica para  introducir el
tensor de curvatura, aqu\'i no la hemos  dado. Ella tiene que ver con la idea de
cu\'anto cambia un  vector al desplazarlo paralelamente a lo  largo de una curva
cerrada. En general diremos que una  variedad es plana, si todas las componentes
de su tensor de curvatura, se anulan.

Concluimos  este cap\'itulo  con una  breve  nota hist\'orica.  En sus  trabajos
originales sobre geometr\'ia, B. Riemann  introdujo el elemento de l\'inea entre
dos puntos vecinos $p$ y $q$ por medio de la expresi\'on
\begin{equation}
ds = F(p, \mathbb{X})dt \label{linea}
\end{equation}
%
con $F(  p , \mathbb{X} )$  una funci\'on homog\'enea  de grado 2 en  la segunda
variable. Aqu\'i estamos suponiendo que  los puntos $p$ y $q$ tienen coordenadas
${x^a}$  y ${x^a +  X^a dt}$,  respectivamente. La  geometr\'ia basada  sobre el
elemento de l\'inea  se conoce como geometr\'ia de  Finsler.  Obs\'ervese que el
elemento~\eqref{quad} (de  Riemann) es un  caso particular de la  geometr\'ia de
Finsler.