%\sub
\seccion{Espacios m\'etricos}
\label{s:WL:Metrico}

En el  tercer ejemplo de espacio  topol\'ogico, usamos la  noci\'on de m\'etrica
eucl\'idea  para definir las  bolas abiertas  en $\Rset^n$.  El disponer  de una
m\'etrica  no es  algo que  ocurre  en todo  conjunto. Eso  motiva la  siguiente
definici\'on:
%
\begin{definicion}[Espacio m\'etrico]
  Un {\it  espacio m\'etrico} en un conjunto  $X$ munido de una  funci\'on $d: X
  \times X \rightarrow \Rset_+$ tal que se cumplen las condiciones:
%
\begin{enumerate}
\item $d(x,y) \ge 0 \quad \forall \: x , y \in X$ y la igualdad se cumple sii $x
  = y$,
%
\item\label{Metrica:simetria} $d(x,y) = d(y,x)$ \ simetr\'ia.
%
\item\label{Metrica:triangular} $d(x,y) \le d(x,z) +  d(z,y) \quad \forall x , y
  , z \in X$.
\end{enumerate}
\end{definicion}
%
\noindent   La   \'ultima   condici\'on   se  conoce   como   {\it   desigualdad
  triangular}.  Mas adelante  en este  libro veremos  funciones $d:  X  \times X
\rightarrow \Rset_+$ que  no satisfacen ni la condici\'on~\ref{Metrica:simetria}
ni  la condici\'on~\ref{Metrica:triangular},  pero que  sin embargo  sirven para
medir cu\'an separados est\'an dos puntos de $X$. En ese caso diremos que $d$ es
una {\it distancia} definida sobre $X$.%%% ??? DIVERGENCIA?
