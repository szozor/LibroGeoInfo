%\sub
\seccion{Estructura Afin}
\label{s:WL:EstructuraAfin}

En  el espacio  eucl\'ideo  $n$-dimensional (pensado  aqu\'i  como una  variedad
diferenciable),  cuando  usamos coordenadas  cartesianas,  caracterizamos a  dos
vectors paralelos como aquellos  que tienen iguales componentes. Si reemplazamos
las coordenadas cartesianas por las polares, por ejemplo, esta caracterizaci\'on
deja  de  ser  v\'alida.  Veamos   c\'omo  podemos  introducir  la  noci\'on  de
paralelismo de vectores, usando  cualquier sistema de coordenadas. Sea $\{x^a\}$
el sistema de  coordenadas cartesiano del espacio. En  este sistema, hemos dicho
que  dos vectores  paralelos,  por ejemplo  $\mathbb{V}$ y  $\tilde{\mathbb{V}}$
tienen iguales componentes:
%
\[
V^a = \tilde{V}^a
\]
%
Si el vector $\mathbb{V}$ es tangente al espacio en el punto $p$ con coordenadas
$\{ x^a \}$  y el vector paralelo $\tilde{\mathbb{V}}$ es  tangente al punto $q$
con coordenadas ${x^a+\delta x^a}$, vale
%
\[
\tilde{V}^a(q)-V^a(p) = 0
\]
%
Dado  un  vector $\mathbb{V}$  en  $p$, podemos  definir  un  campo de  vectores
paralelos  a $\mathbb{V}$  en un  entorno  de $p$.  Denotemos a  este campo  por
$\tilde{\mathbb{V}}$.  Este  campo cumple  que  en  el  punto $p$  coincide  con
$\mathbb{V}$ y con la condici\'on:
%
\[
\tilde{V}^a(x+\delta x) -  V^a(x) = \frac{\partial \tilde{V}^a}{\partial x^b}(p)
\delta x^b
\]
%
Sea ${\xi^a}$ otro sistema de  coordenadas para el espacio eucl\'ideo, vinculado
con ${x^a}$ mediante las relaciones
%
\begin{equation}
\xi^a = \xi^a(x^b), \quad x^b=x^b(\xi^a) \label{cambcoor}
\end{equation}
%
A partir de ellas, resulta
%
\begin{equation}
\delta \xi^a = \frac{\partial \xi^a}{\partial x^b} \delta x^b, \qquad \delta x^b = \frac{\partial x^b}{\partial \xi^a} \delta \xi^a
\end{equation}
%
Las componentes de $\tilde{\mathbb{V}}$ se transforman de acuerdo con
%
\[
\tilde{V}^a = \frac{\partial x^a}{\partial \xi^b} \tilde{V'}^b
\]
%
donde  $\tilde{V'}^a$  son  las   componentes  de  $\tilde{\mathbb{V}}$  en  las
coordenadas $\{\xi^a\}$. Entonces, podemos escribir
%
\begin{eqnarray}
\frac{\partial \tilde{V}^a}{\partial x^b} &=& \frac{\partial}{\partial \xi^c} \left(\frac{\partial x^a}{\partial \xi^d} \tilde{V'}^d \right) \frac{\partial \xi^c}{\partial x^b}  \nonumber \\
%
& = & \frac{\partial^2 x^a}{\partial \xi^c \partial \xi^d} \tilde{V'}^d \frac{\partial \xi^c}{\partial x^a}+ \frac{\partial x^a}{\partial \xi^d} \frac{\partial{\tilde{V'}^d}}{\partial \xi^c} \frac{\partial \xi^c}{\partial x^b}
\end{eqnarray}
%
Si    definimos   la    cantidad   $\delta    \tilde{V'}^d    =   \frac{\partial
  \tilde{V'}^d}{\partial x^e} \delta \xi^e$ y despu\'es de un poco de \'algebra,
llegamos a la relaci\'on
%
\begin{equation}
\delta \tilde{V'}^n = - \frac{\partial^2 x^a}{\partial \xi^e \partial \xi^d} \frac{\partial \xi^n}{\partial x^a} \tilde{V'}^d \delta \xi^e \label{con}
\end{equation}
%
Esta expresi\'on puede reescribirse de la siguiente forma:
%
\begin{equation}
\delta \tilde{V'}^n = - \Gamma'^n_{ed} \tilde{V'}^d \delta \xi^e
\end{equation}
%
\noindent  en   donde  los  coeficientes  $\Gamma'$  est\'an   definidos  en  la
expresi\'on~\eqref{con}. De su definici\'on resulta que las cantidades $\Gamma'$
se anulan para cambios {\it lineales} de coordenadas~\eqref{cambcoor}.

Obs\'ervese que al haber arribado a la definici\'on de los coeficientes $\Gamma$
no hemos hecho uso de ninguna  propiedad especial del espacio eucl\'ideo. Es por
ello  que  la  expresi\'on~\eqref{con}   es  v\'alida  para  cualquier  variedad
n-dimensional. Es f\'acil ver que frente a un cambio de coordenadas
%
\begin{equation}
x^a \rightarrow x'^a= x'^a\left(x^b\right) \label{camb2}
\end{equation}
%
las cantidades $\Gamma$ cambian seg\'un la expresi\'on
%
\begin{equation}
\Gamma ^f_{de} = \Gamma'^a_{mn} \frac{\partial x^f}{\partial x'^a} \frac{\partial x'^m}{\partial x^d} \frac{\partial x'^n}{\partial x^e}+\frac{\partial x^f}{\partial x'^a} \frac{\partial^2 x'^a}{\partial x^e \partial x^d} \label{camcon}
\end{equation}
%
Debemos  remarcar que  esta  ley  de transformaci\'on  es  lineal y  homog\'enea
(tensorial)   s\'olo  cuando  el   cambio  de   coordenadas~\eqref{cambcoor}  es
lineal. Esta propiedad  de los coeficientes $\Gamma$ nos  permite generalizar la
idea de paralelismo en una variedad arbitraria:
%
\begin{definicion}[Conexi\'on af\'in]
  Cuando  en una  variedad  n-dimensional arbitraria  $\M$  se introducen  $n^3$
  coeficientes $\Gamma$ que se transforman de acuerdo con la ley~\eqref{camcon},
  diremos que sobre esa variedad se ha definido una {\it conexi\'on af\'in}
\end{definicion}

A partir de los coeficientes $\Gamma$ es posible definir una nueva derivada para
un campo vectorial arbitrario, digamos $V^a(x)$:
%
\begin{definicion}[Derivada covariante de campo]
  Sea  un campo vectorial  $V$ definido  en un  entorno del  punto $x$.  La {\it
    derivada covariante  del campo  $V$} est\'a dado  por las componentes  de un
  tensor de tipo $(1,1)$
  %
  \[
  V^a_{;c} = V^a_{,c} + \Gamma^a_{bc} V^b
  \]
\end{definicion}

\begin{definicion}[Derivada covariante en una direcci\'on]
  Dados dos campos  vectoriales $U(x)$ y $V(x)$, la  {\it derivada covariante de
    $V$ en la direcci\'on de $U$} es el campo vectorial definido por
  %
  \[
  U(x) \cdot \nabla V(x) \equiv \sum_{ab} V^a_{;b}(x) U^b(x) \mathbb{E}_a \equiv
  \nabla_U V
  \]
% hdot => cdot?
\end{definicion}
%
donde  $\mathbb{E}^a$ es  el campo  de  vectores coordenados  asociados con  las
coordenadas $x^a$. Esta \'ultima  definici\'on permite trasladar paralelamente a
un vector a  lo largo de una  curva. Basta con tomar como  $\mathbb{U}$ al campo
tangente a la curva.