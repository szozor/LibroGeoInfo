\seccion{Estructuras}
\label{s:WL:Introduccion}

Una  de  las  nociones m\'as  elementales  de  la  matem\'atica  es la  de  {\it
  conjunto}.   Un  conjunto  es   una  colecci\'on  de  elementos  perfectamente
caracterizados.   Los  elementos  pueden   ser  de  cualquier  tipo:  n\'umeros,
funciones, personas, autos, etc. El  enfoque matem\'atico moderno es ir montando
estructuras de  distinta naturaleza sobre  un dado conjunto. En  este cap\'itulo
comenzaremos con la noci\'on de espacio topol\'ogico y llegaremos al concepto de
variedad Riemanniana. Este procedimiento ha mostrado ser de utilidad en el marco
de la f\'isica, que es nuestro principal \'ambito de inter\'es.  El mapa de ruta
de las distintas estructuras que veremos en este cap\'itulo es el siguiente:
%
\begin{itemize}
\item Espacio topol\'ogico (continuidad)
\item Espacio m\'etrico (distancia)
\item Variedad topol\'ogica (coordenadas)
\item Variedad diferenciable (diferenciabilidad)
\item Estructura afin (paralelismo)
\item Estructura m\'etrica (Finsler y Riemann)
\end{itemize}

Si  bien   existe  una  estructura   intermedia  entre  la  topol\'ogica   y  la
diferenciable,  que se  conoce como  {\it  estructura lineal  a trozos},  aqu\'i
prescindiremos de su estudio. A  su vez, hay otras estructuras matem\'aticas que
son usadas en el marco de  las teor\'ias f\'isicas. Se destacan la estructura de
producto  interno sobre  un espacio  vectorial complejo,  la cual  conduce  a la
noci\'on  de  espacio  de  Hilbert,  de fundamental  importancia  en  mec\'anica
cu\'antica;  la estructura simpl\'ectica,  \'util en  mec\'anica cl\'asica  y la
estructura de K\"ahler, de relevancia en teor\'ia de cuerdas.