\begin{notation}{Operaciones, conjuntos \& n\'umeros}
%
$\big|$ & ``tal que''.\\[2.5mm]
\hline
%
$\ou$ & ``o'', \ie $A \ou B$ signifca tener $A$ o $B$ no necesariamente
exclusivamente.\\[2.5mm]
\hline
%
$\et$ & ``y'', \ie $A \et B$ signifca tener a la vez $A$ y $B$.\\[2.5mm]
\hline
%
$\exists$ & ``existe''.\\[2.5mm]
\hline
%
$\exists!$ & ``existe un \'unico''.\\[2.5mm]
\hline
%
$\forall$ & ``para todo los/las''.\\[2.5mm]
\hline
%
$\{ \cdots \! \}$ & Conjunto\\[2.5mm]
\hline
%
$\in$ & ``es un elemento de'' o ``partenece a''\vspace{1mm}\newline $x \in A$ significa que
el elemento $x$ partenece al conjunto $A$.\\[2.5mm]
\hline
%
$\not\in$ & ``no es elemento de'' o ``no partenece a''.\\[2.5mm]
\hline
%
$\subset$ & ``es incuido en''\vspace{1mm}\newline $B \subset A$ significa que el conjunto
$B$ es adentro del conjunto $A$: $x \in B \Rightarrow x \in A$.\\[2.5mm]
\hline
%
% subseteq
%
$\setminus$ & Conjunto privado de: $A \setminus B = \{ x \in A \tq x \not\in B
\}$.\\[2.5mm]
\hline
%
$\cup$ & Union: $A \cup B = \{ x \tq (x \in A) \ou  (x \in B) \}$.\\[2.5mm]
\hline
%
$\displaystyle \mathop{\cup}_{i=1}^n$ & Union de conjuntos: $\displaystyle
\mathop{\cup}_{i=1}^n A_i = A_1 \cup \cdots \cup A_n$.\\[2.5mm]
\hline
%
$\cap$ & Intersecci\'on: $A \cap B = \{ x \tq (x \in A) \et (x \in B)
\}$.\\[2.5mm]
\hline
%
$\displaystyle \mathop{\cap}_{i=1}^n$ & Intersecci\'on de conjuntos:
$\displaystyle \mathop{\cap}_{i=1}^n A_i = A_1 \cap \cdots \cap A_n$.\\[2.5mm]
\hline
%
$\times$ & Producto cartesiano \ $x \in A \times B \: \Leftrightarrow
\: x = (a,b)$ \ con \ $a \in A$ \ y \ $b \in B$.\\[2.5mm]
\hline
%
$\displaystyle \optimes_{i=1}^n$ & Productos cartesianos: $\displaystyle
\optimes_{i=1}^n A_i = A_1 \times \cdots \times A_n$.\\[2.5mm]
\hline
%
$\emptyset$ & Conjunto vac\'io, a veces denotado tambi\'en $\{\}$.\\[2.5mm]
\hline
%
$\Nset$ & Enteros naturales.\\[2.5mm]
% \ $\{ 0 \, , \, 1 \, , \, \ldots \}$\\[2.5mm]
\hline
%
$\Zset$ & Enteros relativos.\\[2.5mm]
%$\{\ldots\,,\,-1\,,0\,,\,1\,,\,\ldots\}$\\[2.5mm]
\hline
%
$\Qset$ & N\'umeros racionales.\\[2.5mm]
\hline
%
$\Rset$ & N\'umeros reales.\\[2.5mm]
\hline
%
$\Cset$ & N\'umeros complejos.\\[2.5mm]
\hline
%
$\Kset^*$ & Conjunto \ $\Kset$ \ privado de \ $0$, \ $\Kset^* = \{ x \in \Kset
\tq x \ne 0 \}$ \ ($\Kset = \Nset, \Zset, \Rset$ \ o \ $\Cset$).\\[2.5mm]
\hline
%
$\Kset_+$ & Elementos de \ $\Kset$ \ no negativos, $\Kset_+ = \{ x \in \Kset \tq
x \ge 0 \}$ \ ($\Kset = \Nset, \Zset$ \ o \ $\Rset$).\\[2.5mm]
\hline
%
$\protect\begin{array}{c}\Kset^d\\[-1mm] =\\[-1mm] \underbrace{\Kset \! \times
\! \cdots \! \times \! \Kset}_{d \:\, \mbox{\footnotesize
veces}}\protect\end{array}$ & Se usar\'a la misma notaci\'on para los vectores,
$M_{d,1}(\Kset) \equiv \Kset^d$ (ver m\'as abajo):\vspace{1mm}\newline
$x \in \Kset^d \: \Leftrightarrow \: x = (x_1,\ldots,x_d) \, \equiv
\, \protect\begin{bmatrix} x_1\\ \vdots \\ x_d \protect\end{bmatrix}$ \ con \ $x_i \in
\Kset$.\vspace{1mm}\\[2.5mm]
\hline
%
$[a \; b]$ & Intervalo real cerrado en $a$ y $b$, \: $[a \; b] = \{ x \in \Rset
\tq a \le x \le b \}$\\[2.5mm]
\hline
%
$(a \; b)$ & Intervalo real abierto en $a$ y $b$, \: $(a \; b) = \{ x \in \Rset
\tq a < x < b \}$\\[2.5mm]
\hline
%
$[a \; b)$ & Intervalo real cerrado en $a$ y abierto en $b$, \: $[a \; b) = \{ x
\in \Rset \tq a \le x < b \}$\\[2.5mm]
\hline
%
$(a \; b]$ & Intervalo real abierto en $a$ y cerrado en $b$, \: $(a \; b] = \{ x
\in \Rset \tq a < x \le b \}$\\[2.5mm]
\hline
%
$\imath$ & Imaginario puro, $\imath = \sqrt{-1}$\\[2.5mm]
\hline
%
% pi, e?
%
$\real{\cdot}$ & Parte real de un complejo\vspace{1mm}\newline (componente a componente
sobre $\Cset^d$)\\[2.5mm]
\hline
%
$\imag{\cdot}$ & Parte imaginaria de un complejo\vspace{1mm}\newline (componente a
componente sobre $\Cset^d$)\\[2.5mm]
\hline
%
$\cdot^*$ & Conjugaci\'on compleja\vspace{1mm}\newline (componente a
componente sobre $\Cset^d$)\\[2.5mm]
\hline
%
$|\cdot|$ & Valor absoluto o modulus tratando de un real o complejo\vspace{1mm}\newline
Cardenal (posiblemente infinito) tratando de un conjunto discreto\vspace{1mm}\newline
Volumen (posiblemente infinito) tratando de un conjunto continuo\\[2.5mm]
\hline
%
$\|\cdot\|_p$ & Norma \ $p$, \:
$\displaystyle \|x\|_p = \left( \sum_{i=1}^d |x|^p
\right)^{\frac1p}$.\vspace{1mm}\\[2.5mm]
\hline
%
$\|\cdot\|$ o $|\cdot|$ & Norma \ $2$.\\[2.5mm]
\hline
%
$\Sset_d$ & Esfera unitaria de \ $\Rset^d$, \: $\Sset_d = \left\{ x \in \Rset^d
\tq \|x\| = 1 \right\}$\\[2.5mm]
\hline
%
$\Sset_d(c,r)$ & Esfera de \ $\Rset^d$, centrada en $c \in \Rset^d$ y de rayo
$r$\: $\Sset_d(c,r) = \left\{ x \in \Rset^d \tq \|x-c\| = r \right\}$,\vspace{1mm}\newline
$\Sset_d \equiv \Sset_d(0,1)$\\[2.5mm]
\hline
%
$\Bset_d$ & Bola abierta unitaria de \ $\Rset^d$, \: $\Bset_d = \left\{ x \in \Rset^d
\tq \|x\| < 1 \right\}$\\[2.5mm]
\hline
%
$\Bset_d(c,r)$ & Bola abierta de \ $\Rset^d$, centrada en $c \in \Rset^d$ y de rayo
$r$\: $\Bset_d(c,r) = \left\{ x \in \Rset^d \tq \|x-c\| < r \right\}$,\vspace{1mm}\newline
$\Bset_d \equiv \Bset_d(0,1)$\\[2.5mm]
\hline
%
$\Simp{k-1}$ & $(k-1)$-simplex estandar~\footnote{\SZ{Politopio, convex hull
$\left\{ \un_i \right\}_{i=1}^k$ (ver notaciones matriciales
pagina~\pageref{toto}).}}de \ $\Rset_+^k$, o simplex de probabilidad,
i.e.,\vspace{1mm}\newline $\displaystyle \Simp{k-1} = \left\{ x \in \Rset_+^k \tq
\sum_{i=1}^k x_i = 1 \right\}$\\[2.5mm]
\hline
%
$\Part{k,d}$ & Partici\'on (con las permutaciones) de \ $k \in \Nset$ \ de \
tama\~no \ $d \in \Nset^*$,\vspace{1mm}\newline $\displaystyle \Part{k,d} =
\left\{ n = \begin{bmatrix} n_1 & \cdots & n_d \end{bmatrix}^t \in \Nset^d \tq
\sum_{i=1}^d n_i = k \right\}$\\[2.5mm]
\hline
%
$\perm_n$ & Conjunto de las permutaciones de \ $\{ 1 \, , \, \ldots \, , \, n \}$\\[2.5mm]
\hline
%
$\PD_d$ & Conjuntos (convexo) de las funciones \ $f: \Rset^d \mapsto \Cset$ continuas, a
simetr\'ia herm\'itica \ $f(-x) = f^*(x)$,\newline definida positivas, \
$\forall \: n \in \Nset^*$, \ $a_i \in \Cset, x_i \in \Rset^d, \: i = 1, \ldots
, n$, \ $\displaystyle \sum_{i,j=1}^n a_i a_j^* f(x_j-x_i) \ge 0$,\newline con
\ $f(0) = 1$.\\[2.5mm]
\hline
%
$\PD$  &  Conjuntos (convexo) \ $\displaystyle  \bigcap_{d=1}^{+\infty} \, \PD_d$  \ de  todas  las
funciones  continuas, a  simetr\'ia  herm\'itica, definida  positivas, de  valor
unidad al origen.\\[2.5mm]
\hline
%
$\PDSI_d$ & Conjuntos (convexo) de las funciones \ $f: \Rset_+ \mapsto [-1 \; 1]$ continuas con
\ $f(0) = 1$ \ tal que $F: x \mapsto f\left( \|x\|^2 \right) \: \in \PD_d$.\\[2.5mm]
\hline
%
$\PDSI$ & Conjuntos (convexo)  $\displaystyle  \bigcap_{d=1}^{+\infty} \, \PDSI_d$ de las funciones \ $f: \Rset_+ \mapsto [-1 \; 1]$ continuas con
\ $f(0) = 1$ \ tal que $F: x \mapsto f\left( \|x\|^2 \right) \: \in \PD$.
\end{notation}
