\begin{notation}{Vectores, matrices, tensores}
%
$\Mat_d(\Kset) \equiv \Kset^d$
%\vspace{1mm}\newline$\Kset = \Rset$ \ o \ $\Cset$
 & Espacio de vectores \ $M = \protect\begin{bmatrix} M_1\\ \vdots\\ M_d
\protect\end{bmatrix}$ \ con \ $M_i \in \Kset$;\vspace{1mm}\newline Se usar\'a
la misma notaci\'on $\Kset^d$ para el espacio de vectores y de las $d$-uplas $(M_1 ,
\ldots , M_d)$, sinedo estos espacios en biyecci\'on\\[2.5mm]
\hline
%
$\Mat_{d,d'}(\Kset)$ & Espacio de matrices \ $M$, \ de tama\~no \ $d \times d'$,
\ $M = \protect\begin{bmatrix} M_{1,1} & \cdots & M_{1,d'}\\ \vdots & \ddots &
\vdots\\ M_{d,1} & \cdots & M_{d,d'} \protect\end{bmatrix}$ de componentes \
$M_{i,j} \in \Kset, \: i = 1, \ldots , d, \: j = 1 , \ldots ,
d'$;\vspace{1mm}\newline $\Mat_{d,1}(\Kset) = \Mat_d(\Kset) \Kset^d$\\[2.5mm]
% as\'i que se usar\'a esta notaci\'on cuando~\footnote{Si una
%dimensi\'on es $1$, se saca de las dimensiones (caso degenerado).} \ $d, d' >
%1$\\[2.5mm]
\hline
%
$\Mat_{d_1 ,  \ldots , d_K}(\Kset)$  & Espacio de tensores
%~\footnote{De  hecho, el
%  termino ``tensor''  tiene una  significaci\'on f\'isica. El  termino ``tabla''
%  ser\'ia m\'as adecuado. Pero en este  libro, usaremos tensor, del hecho que se
%  usa frecuentemente,  por abuso de denominaci\'on.} 
 \ $M$, \ de orden $K$, de tama\~no \ $d_1 \times \cdots \times d_K$, \ de
componentes \ $M_{i_1,\ldots,i_K} \in \Kset, \: i_k = 1 , \ldots , d_k, \: k = 1
, \ldots , K$\\[2.5mm]
%;\vspace{1mm}\newline De nuevo, con estanotaci\'on se supone que $d_i > 1, i =
%1,\ldots,K$\\[2.5mm]
\hline
%
$\cdot^*$ & Conjugaci\'on, componente a componente: $M^*$ \ es de componentes \
 \ $M_{i_1,\ldots,i_K}^*$\\[2.5mm]
\hline
%
$\cdot^t$ & Transpuesta, $\forall M \in \Kset^d, \: M^t =
\protect\begin{bmatrix} M_1 & \cdots & M_d \protect\end{bmatrix} \in \Mat_{1,d}(\Kset)$ \ y \ $ \forall
M \in \Mat_{d,d'}(\Kset)$, $M^t \in \Mat_{d',d}(\Kset)$ \ es de componente \
$(i,j)$-\'esima \ $M_{j,i}$\\[2.5mm]
\hline
%
\modif{$\tau_\sigma[\cdot]$} & \modif{Permutaciones de indices seg\'un la permutaci\'on \
$\sigma \in \perm_K$ para un tensor $M \in \Mat_{d_1,\ldots,d_K}(\Kset)$, \ie
$\tau_\sigma [M] \in \Mat_{d_{\sigma(1)},\ldots,d_{\sigma(K)}}(\Kset)$ \ es de
componente \ $M_{i_{\sigma(1)},\ldots,i_{\sigma(K)}}$.\vspace{1mm}\newline Para
$u \ne v$, $\sigma_{(u,v)}(i) = v \un_{\{u\}}(i) + u \un_{\{v\}}(i) + i
(1-\un_{\{u,v\}}(i))$, se usa m\'as simplemente \ $\tau_{(u,v)} \equiv
\tau_{\sigma_{(u,v)}}$.\vspace{1mm}\newline Nota: para $K = 2, \:
\tau_{(1,2)}[M] = M^t$}\\[2.5mm]
\hline
%
$\cdot^\dag$ & Adjunta, $M^\dag = \left( M^* \right)^t$\\[2.5mm]
% \ es de
%componente \ $(i_1,\ldots,i_K)$-\'esima \ $M_{i_K,\ldots,i_1}^*$\\[2.5mm]
% \ es de
%componente \ $(i_1,\ldots,i_K)$-\'esima \ $M_{i_K,\ldots,i_1}^*$\\[2.5mm]
\hline
%
$\real{\cdot}$ & Parte real, $\real{M}$ \ es de
componentes \  $\real{M_{i_1,\ldots,i_K}}$\\[2.5mm]
\hline
%
$\imag{\cdot}$ & Parte imaginaria, $\imag{M}$ \ es de
componentes \  $\imag{M_{i_1,\ldots,i_K}}$\\[2.5mm]
\hline
%
% o externo  $\otimes$ &  Producto de Kronecker:  para $A \in  \Kset^{d_1 \times
%   \cdots \times d_K}, \quad B  \in \Kset^{d'_1 \times \cdots \times d'_L}$, el
% producto  de Kronecker \  $A \otimes  B$ \  es el  tensor bloc  de $\Kset^{d_1
%   \times  \cdots  \times d_K,d'_1  \times  \cdots  \times  d'_L}$, de  bloc  \
% $(i_1,\ldots)$-\'esima  \  $A_{i,j}  B$,\vspace{1mm}\newline  para  \  $A  \in
% \Mat_{d,d'}(\Kset), \: B \in \Mat_{n,n'}(\Kset)$, tenemos \ $A \otimes B \in \Mat_{d
%   n, d' n'}(\Kset)$ \ tal que,\vspace{1mm}\newline
%%
%$
%A \otimes B = \protect\begin{bmatrix}
%A_{1,1} B  & \cdots  & A_{1,d'} B\\
%% \vdots   &  \vdots & \vdots\\
%A_{d,1} B  & \cdots  & A_{d,d'} B
%%\protect\end{bmatrix}
%$\vspace{1mm}\\[2.5mm]
%\hline
%
$\otimes$  &  Producto   tensorial  o  producto  externo~\footnote{Se  encuentra
  tambi\'en  el s\'imbolo  \ $\otimes$  \ para  el producto  de  Kronecker: para
  matrices por ejemplo \ $A \otimes B$  \ ser\'a la matriz de bloques \
  $A_{i,j}  B$. De hecho,  es equivalente  a ``desarrollar''  el tensor  de forma
  matricial,   o  equivalente  de   la  vectorizaci\'on   de  una   matriz  (ver
  ej.~\cite{MagNeu79}).}:  para $A \in  \Mat_{d_1,\ldots,d_K}(\Kset), \quad  B \in
\Mat_{d'_1,\ldots,d'_L}(\Kset)$,\newline  el producto  externo \ $A  \otimes B$ \  es el
tensor      de      orden      \      $K+L$,\newline         de      componente      \
$(i_1,\ldots,i_K,j_1,\ldots,j_L)$-\'esima          \         $A_{i_1,\ldots,i_K}
B_{j_1,\ldots,j_L} $.\vspace{1mm}\newline En particular, para \ $a \in \Kset^d, \quad
b \in \Kset^{d'}$, \ $a \otimes b = a b^t \in \Mat_{d,d'}(\Kset)$\\[2.5mm]
\hline
%
$\cdot^{\otimes k}$ & $k$ veces el producto externo, $A^{\otimes k}
= \underbrace{A \otimes \cdots \otimes A}_{k \:\, \mbox{\footnotesize
veces}}$\\[2.5mm]
\hline
%
$0$ &  Vector nulo, matriz nula
$0 \equiv \protect\begin{bmatrix}
   0   & \cdots &    0   \\
\vdots & \ddots & \vdots \\
   0   & \cdots &    0
\protect\end{bmatrix}$.\vspace{1mm}\\[2.5mm]
\hline
%
$I_d \quad , \quad I$ & Matriz cuadrada $d \times d$, identidad, \ $I \equiv I_d
= \protect\begin{bmatrix}
   1   &    0   & \cdots &   0    \\[-.25mm]
   0   & \ddots & \ddots & \vdots \\[-1mm]
\vdots & \ddots & \ddots &   0    \\[-.25mm]
   0   & \cdots &    0   &   1
\protect\end{bmatrix}$
%= \Diag(\un)$
%\vspace{1mm}\newline (se saca el \'indice cuando no hay ambiguedad)
\\[2.5mm]
\hline
%
$\un$ & Vector de componentes iguales a 1, \ $\un = \protect\begin{bmatrix} 1\\
\vdots \\ 1 \protect\end{bmatrix}$.\vspace{1mm}\\[2.5mm]
\hline
%
$\un_i$ & Vector de componentes $j$-\'esima iguales a $\un_{\{i\}}(j)$\\[2.5mm]
\hline
%
$\TriS_d(\Kset)$ & Conjunto de las matrices de \ $\Mat_{d,d}(\Kset)$, \
triangulares superiores de diagonal positiva:\vspace{1mm}\newline
$\TriS_d(\Kset) = \left\{ M \in \Mat_{d,d}(\Kset) \tq \forall \: 1 \le j < i \le
d, \: M_{i,j} = 0 \et \forall \: 1 \le i \le d, \: M_{i,i} \ge 0
\right\}$\\[2.5mm]
\hline
%
$\TriI_d(\Kset)$ & Conjunto de matrices de \ $\Mat_{d,d}(\Kset)$, \ triangulares
inferiores de diagonal positiva:\vspace{1mm}\newline $\TriI_d(\Kset) = \left\{ M
\in \Mat_{d,d}(\Kset) \tq \forall \: 1 \le i < j \le d, \: M_{i,j} = 0 \et
\forall \: 1 \le i \le d, \: M_{i,i} \ge 0 \right\}$\\[2.5mm]
\hline
%
$\Sim_d(\Kset)$ & Conjunto de las matrices de \ $\Mat_{d,d}(\Kset)$ \
sim\'etricas:\vspace{1mm}\newline $\Sim_d(\Kset) = \left\{ M \in
\Mat_{d,d}(\Kset) \tq \, M^t = M \right\}$\\[2.5mm]
\hline
%
$\Her_d(\Cset)$ & Conjunto de las matrices de \ $\Mat_{d,d}(\Cset)$ \
herm\'iticas (a simetr\'ia herm\'itica):\vspace{1mm}\newline $\Her_d(\Cset) =
\left\{ M \in \Mat_{d,d}(\Cset) \tq \, M^\dag = M \right\}$.\vspace{1mm}\newline
Notar que \ $\Her_d(\Rset) \equiv \Sim_d(\Rset)$\\[2.5mm]
\hline
%
$\Ort_d(\Rset)$ & Conjunto de las matrices ortogonales (grupo
ortogonal):\vspace{1mm}\newline $\Ort_d(\Rset) = \left\{ Q \in \Mat_{d,d}(\Rset)
\tq Q Q^t = Q^t Q = I \right\}$\\[2.5mm]
\hline
%
$\Unit_d(\Cset)$ & Conjunto de las matrices unitarias:\vspace{1mm}\newline
$\Unit_d(\Cset) = \left\{ Q \in \Mat_{d,d}(\Cset) \tq Q Q^\dag = Q^\dag Q = I
\right\}$.\vspace{1mm}\newline Notar que \ $\Unit_d(\Rset) \equiv
\Ort_d(\Rset)$\\[2.5mm]
\hline
%
% Se    notar\'a    que    estas    variedades   son    $\Sti_{d,d'}(\Rset)    =
% \Ort_d(\Rset)/\Ort_{d-d'}(\Rset)$}
$\Sti_{d,d'}(\Kset)$,\newline $d' \le d$\newline $\Kset = \Rset$ o $\Cset$ &
Variedad de Stiefel~\footnote{Ver~\cite{Sti35} para el estudio de este variedad,
o~\cite{Ste51, Jam76} para estudios m\'as detallados.}:\vspace{1mm}\newline
$\Sti_{d,d'}(\Kset) = \left\{ Q \in \Mat_{d,d'}(\Kset) \tq Q^\dag Q = I
\right\}$.\vspace{1mm}\newline Notar que $\Sti_{d,d}(\Kset) = \Unit_d(\Kset)$, \
$\Sti_{d,1}(\Rset) = \Sset_d$ \ y \ $\Sti_{d,1}(\Cset) = \SCset_d$\\[2.5mm]
\hline
%
$\Perm_d$ & (tambi\'en) Conjunto de las matrices de permutaciones $d \times d$\vspace{1mm}\newline $\displaystyle \Perm_d = \left\{
\Sigma = \protect\begin{bmatrix} \un_{\sigma(1)} & \cdots &
\un_{\sigma(d)} \end{bmatrix}^t\protect = \sum_{i=1}^d \un_i \un_{\sigma(i)}^t,
\quad \sigma \in \perm_d \right\}$\vspace{1mm}\newline Tiene exactamente un $1$
en cada linea y en cada columna, y los dem\'as elementos son
$0$\vspace{1mm}\newline Notar que \ $\forall \: \Sigma \in \Perm_d, \quad
\Sigma^{-1} = \Sigma^t$\\[2.5mm]
\hline
%
$\Pos_d(\Kset)$ & Conjunto de las matrices herm\'iticas semidefinidas
positivas:\vspace{1mm}\newline $\Pos_d(\Kset) = \left\{ M \in H_d(\Kset) \tq
\forall \, x \in \Kset^d, \: x^\dag M x \ge 0 \right\}$\\[2.5mm]
\hline
%
$\Pos_d^+(\Kset)$ & Conjunto de las matrices herm\'iticas definidas
positivas:\vspace{1mm}\newline $\Pos_d^+(\Kset) = \left\{ M \in H_d(\Kset) \tq
\forall \, x \ne 0 \in \Kset^d, \: x^\dag M x > 0 \right\}$\\[2.5mm]
\hline
%
$\P_{d,k}(\Kset)$ & Conjunto
%~\footnote{\SZ{Este conjunto es un convexo de
%$\Pos_d(\Kset)^k$. Pero todavia no me queda claro que estructura tiene. Si no
%restringimos a matrices diagonales, es un
%simplex~\cite[p.~329]{ParisRehacek_QuantumStateEstimation}. Si no, no le
%se.}\label{foot:Notaciones:POVM}} 
de las $k$-uplas de matrices de \ $\Pos_d(\Kset)$ \
cuya suma es la identidad~\footnote{Se dice que la $k$-upla satisface a la {\em
resoluci\'on de la identidad}, o a la {\em relaci\'on de
completud}.\label{foot:Notaciones:ResolucionlIdentitad}},\vspace{1mm}\newline $\displaystyle
\P_{d,k}(\Kset) = \left\{ \left( M_1 , \ldots , M_k \right) \in \Pos_d(\Kset)^k \tq
\sum_{i=1}^k M_i = I  \right\}$
% \SZ{$(k-1)$-simplex  estandar}~\footnote{\SZ{Politopio,  convex hull  $\left\{
%       \un_i   \right\}_{i=1}^k$}}    \SZ{de   \   $Rset_+^k$,   i.e.,}\newline
% \SZ{$\displaystyle  \Simp_{k-1} =  \left\{ (M_1  ,  \ldots ,  M_k) \in  \left(
%       \Pos_d^+(\Rset) \right)^k, \:  \sum_{i=1}^k M_i = I \right\}$  donde $I$ es
%   la  identidad (ver  m\'as abajo)}\newline  \SZ{Que suman  a la  identidad es
%   conocido  como {\em  resoluci\'on  de  la identidad}  o  {\em relaci\'on  de
%     completud}}~\footnote{\SZ{De  hecho, este  nombre viene  usualmente  de la
%     mecanica cuantica~\cite{toto, titi}.}}
\\[2.5mm]
\hline
%
$\P_{d,k}^+(\Kset)$ & Conjunto~\footref{foot:Notaciones:POVM} de $k$-uplet de
matrices de \ $\Pos_d^+(\Kset)$ \ sumando a la
identidad~\footref{foot:Notaciones:ResolucionlIdentitad},\vspace{1mm}\newline $\displaystyle
\P_{d,k}^+(\Kset) = \left\{ \left( M_1 , \ldots , M_k \right) \in \Pos_d^+(\Kset)^k
\tq \sum_{i=1}^k M_i = I \right\}$\\[2.5mm]
\hline
%
$\ge$ & $A \ge B$ \ significa que \ $(A - B) \in \Pos_d(\Kset)$\\[2.5mm]
\hline
%
$>$ & $A > B$ \ significa que \ $(A - B) \in \Pos_d^+(\Kset)$\\[2.5mm]
\hline
%
$\diag(\cdot)$  & Vector  formado  por  los elementos  diagonales  de una  matriz
cuadrada:\vspace{1mm}\newline para
 \ $\displaystyle M = \protect\begin{bmatrix}
  M_{1,1}  & \cdots & M_{1,d} \\
  \vdots  & \vdots & \vdots \\
  M_{d,1} & \cdots & M_{d,d}
\end{bmatrix}\protect
\in \Mat_{d,d}(\Kset), \quad \diag(M) = \protect\begin{bmatrix} M_{1,1} \\
M_{2,2} \\ \vdots \\ M_{d,d}
\end{bmatrix}\protect$.\vspace{1mm}\\[2.5mm]
%
\hline
%
$\Diag(\cdot)$ & Matriz diagonal  con los componentes del vector argumento en su
diagonal,\vspace{1mm}\newline para \ $\displaystyle v \in \Kset^d,$\vspace{1mm}\newline
$\displaystyle \Diag(v) = \sum_{i=1}^d \left( \un_i \un_i^t \right) v_i =
\protect\begin{bmatrix}
  v_1  &   0    & \cdots &    0   \\[-.25mm]
   0   & \ddots & \ddots & \vdots \\[-1mm]
\vdots & \ddots & \ddots &    0   \\[-.25mm]
   0   & \cdots &    0   &   v_d
\end{bmatrix}\protect$.\vspace{1mm}\\[2.5mm]
%
%$\Diag$ & Matriz diagonal (bloc) con el argumento en su
%diagonal,\vspace{1mm}\newline para \ $\displaystyle V = (V_1,\ldots,V_n) \in
%\Mat_{d_1,d_1}(\Kset) \times \cdots \times \Mat_{d_n,d_n}(\Kset),$\vspace{1mm}\newline
%$\Diag(V) = \sum_{i=1}^d \left( \un_i \un_i^t \right) \otimes V_i =
%\protect\begin{bmatrix}
%  V_1  &   0    & \cdots &    0   \\
%   0   & \ddots & \ddots & \vdots \\
%\vdots & \ddots & \ddots &    0   \\
%   0   & \cdots &    0   &   V_d
%\end{bmatrix}\protect$.\vspace{1mm}\\[2.5mm]
\hline
%
%$J$ & Matriz de comutaci\'on~\footnote{M\'as generalmente es definida sobre \
%$\Mat_{d d',d d'}(\Kset)$ \ y los bloques son en \
%$\Mat_{d,d'}(\Kset)$~\cite{MagNeu79, NeuWan83}.\label{Foot:MP:ComutacionMatriz}} de
%\ $\Mat_{d^2,d^2}(\Kset)$ \ de bloques \ $J_{i,j} = \un_j \un_i^t \in
%\Mat_{d,d}(\Kset)$ \ poniendo la componente (linea) \ $i$-\'esima de un vector
%(matriz) \ en la componente (linea) \ $j$-\'esima,\vspace{1mm}\newline
%%
%$ \displaystyle J  = \sum_{i,j=1}^d \left( \un_i \un_j^t  \right) \otimes \left(
%  \un_j \un_i^t \right) =
%\protect\begin{bmatrix}
%\un_1 \un_1^t & \un_2 \un_1^t & \cdots & \un_d \un_1^t\\
%    \vdots    &    \vdots     & \cdots &     \vdots   \\
%\un_1 \un_d^t & \un_2 \un_d^t & \cdots & \un_d \un_d^t
%\protect\end{bmatrix}$.\vspace{1mm}\\[2.5mm]
%\hline
%%
%$K$ & Matriz de no-comutaci\'on~\footref{Foot:MP:ComutacionMatriz} de \
%$\Mat_{d^2,d^2}(\Kset)$ \ de bloques \ $K_{i,j} = \un_i \un_j^t \in \Mat_{d,d}(\Kset)$
%\ poniendo la componente (linea) \ $j$-\'esima de un vector (matriz) \ en la
%posici\'on (linea) \ $i$-\'esima,\vspace{1mm}\newline
%%
%$ \displaystyle K  = \sum_{i,j=1}^d \left( \un_i \un_j^t  \right) \otimes \left(
%  \un_i \un_j^t \right)
%\protect\begin{bmatrix}
%\un_1 \un_1^t & \un_1 \un_2^t & \cdots & \un_1 \un_d^t\\
%   \vdots     &    \vdots    & \cdots &     \vdots   \\
%\un_d \un_1^t & \un_d \un_2^t & \cdots & \un_d \un_d^t
%\protect\end{bmatrix}$.\vspace{1mm}\\[2.5mm]
%\hline
%
$G^{(i,j)}$ & Matriz de agregaci\'on de \ $\Mat_{k-1,k}(\Kset)$. Para $i < j$,
multiplicado a un vector $x \in \Kset^k$ se saca la $j$-\'esima componente de
$x$ y se reemplaza la $i$-\'esima por $x_i + x_j$ y similarmente por simetr\'ia
para $i > j$:\vspace{1mm}\newline $\forall \: i < j, \quad G^{(i,j)} =
\protect\begin{bmatrix} I_{j-1} & \un_i & 0\\ 0 & 0 &
I_{k-j} \end{bmatrix}\protect, \quad G^{(j,i)} =
G^{(i,j)}$.\vspace{1mm}\\[2.5mm]
\hline
%
$\Tr(\cdot)$ & Traza de una matriz (cuadrada) de \
$\Mat_{d,d}(\Kset)$:\vspace{1mm}\newline $\displaystyle \Tr M = \sum_{i=1}^d
M_{i,i}$.\vspace{1mm}\\[2.5mm]
\hline
%
$\det(\cdot)$ & Determinante
%~\footnote{
%$\mathfrak{S}_d$ \ es el conjunto de las
%permutaciones de \ $\{ 1 \, , \, \ldots \, , \, d \}$, y \ 
%$\varepsilon(\sigma)$ \ es la signatura de la permutaci\'on \ $\sigma$,
%$\displaystyle \varepsilon(\sigma) = \prod_{1 \le i < j \le d} \sign\left(
%\sigma(j) - \sigma(i) \right)$ \ con \ $\sign$ \ funci\'on ``signo''.} 
de una
matriz (cuadrada) de \ $\Mat_{d,d}(\Kset)$:\vspace{1mm}\newline $\displaystyle
\det M = \sum_{\sigma \in \perm_d} \varepsilon(\sigma) \prod_{i=1}^d
M_{\sigma(i),i}$\vspace{1mm}\newline con \ $\displaystyle \varepsilon(\sigma) =
\prod_{1 \le i < j \le d} \sign\left( \sigma(j) - \sigma(i) \right)$, \
signatura de la permutaci\'on \ $\sigma$.\vspace{1mm}\\[2.5mm]
\hline
%
$\left| \cdot \right|$ & Valor absoluto del determinante de una matriz
(cuadrada) de $\Mat_{d,d}(\Kset)$: $\left| \right| = \left| \det(M)
\right|$\\[2.5mm]
\hline
%
$\cdot^{-1}$ & Matriz inversa (cuando existe), \ $M M^{-1} = M^{-1} M =
I$\\[2.5mm]
\hline
%
$\cdot^{\frac12}$ & Para \ $M \in \P_d^+(\Kset)$, \ $M^{\frac12}$ \ es la
\'unica~\footnote{Ver~\cite{HorJoh13, MagNeu99} por lo de la unicidad.} matriz
de \ $\P_d^+(\Kset)$ \ tal que \ $M^{\frac12} M^{\frac12} = M$\\[2.5mm]
\hline
%
$\cdot^{-\frac12}$ & Para \ $M \in \P_d^+(\Kset)$, \ $M^{-\frac12} = \left(
M^{-1} \right)^{\frac12} = \left( M^{\frac12} \right)^{-1}$\\[2.5mm]
\hline
%
$\cdot^{-t}$ & $M^{-t} =  \left( M^t \right)^{-1} =  \left( M^{-1} \right)^t$\\[2.5mm]
\hline
%
$\cdot^{-*}$ & $M^{-*} =  \left( M^* \right)^{-1} =  \left( M^{-1} \right)^*$\\[2.5mm]
\hline
%
$\cdot^{-\dag}$ & $M^{-\dag} =  \left( M^\dag \right)^{-1} =  \left( M^{-1} \right)^\dag$\\[2.5mm]
\hline
%
$\|\cdot\|_p$ & Norma $p$ de H\"older, $\| M \|_p = \left( \sum_{i,j} \left|
M_{i,j} \right|^p \right)^{\frac1p}$, para \ $p \in \Rset_{0,+}$\\[2.5mm]
\hline
%
$\|\cdot\|_F$ & Norma de Frobenius, $\| M \|_F = \sqrt{\Tr\left( M M^\dag
\right)} = \| M \|_2$\\[2.5mm]
\hline
%
$\|\cdot\|_{q,p}$ & Norma de H\"older inducida, $\displaystyle \| M \|_{q,p} =
\sup_{x \in \Mat_{d',1}(\Kset)^*} \frac{\| M x\|_q}{\| x \|_p}$\\[2.5mm]
\hline
%
$\odot$ & Producto de Schur, o componente a componente:\vspace{1mm}\newline $( M
\odot N)_{i_1,\ldots,i_K} = M_{i_1,\ldots,i_K} N_{i_1,\ldots,i_K}$\\[2.5mm]
\hline
%
$\cdot^{\odot k}$ & Potencia \`a la Schur, o componente a
componente:\vspace{1mm}\newline, $\left( M^{\odot k} \right)_{i_1,\ldots,i_K} =
M_{i_1,\ldots,i_K}^k$
\end{notation}

\modif{Para m\'as  detalles sobres los  tensores y operaciones sobres  estos, se
  puede referirse por ejemplo a~\cite[Cap.~4]{KosMan97}.}
