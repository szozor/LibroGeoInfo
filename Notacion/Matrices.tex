\begin{notation}{Matrices \& vectores}
%
$\M_{d,d'}(\Kset)$ & Espacio de matrices \ $M$, \ de tama\~no \ $d \times d'$, \
de componentes \ $M_{i,j} \in \Kset$ \ con \ $\Kset = \Rset$ \ o \ $\Cset$;
$\M_{d,1}(\Kset) \equiv \Kset^d$.\\[2.5mm]
\hline
%
$\cdot^*$ & Conjugaci\'on, componente por componente, $M^*$ \ es de componente \
$(i,j)$-\'esima \ $M_{i,j}^*$\\[2.5mm]
\hline
%
$\cdot^t$ & Transpuesta, $M^t$ \ es de componente \ $(i,j)$-\'esima \
$M_{j,i}$\\[2.5mm]
\hline
%
$\cdot^\dag$ & Transconjugada, $M^\dag = \left( M^* \right)^t$ \ es de
componente \ $(i,j)$-\'esima \ $M_{j,i}^*$\\[2.5mm]
\hline
%
$\real{\cdot}$ & Parte real, $\real{M}$ \ es de
componente \ $(i,j)$-\'esima \ $\real{M_{i,j}}$\\[2.5mm]
\hline
%
$\imag{\cdot}$ & Parte imaginaria, $\imag{M}$ \ es de
componente \ $(i,j)$-\'esima \ $\imag{M_{i,j}}$\\[2.5mm]
\hline
%
$\otimes$ & Producto de Kronecker o externo: $A \otimes B$ \ es la matriz bloc de bloc \
$(i,j)$-\'esima \ $A_{i,j} B$,\vspace{1mm}\newline para \ $A \in
\M_{d,d'}(\Kset), \: B \in \M_{n,n'}(\Kset)$, tenemos \ $A \otimes B \in \M_{d n,
d' n'}(\Kset)$ \ tal que,\vspace{1mm}\newline
%
$
A \otimes B = \protect\begin{bmatrix}
A_{1,1} B  & \cdots  & A_{1,d'} B\\
 \vdots   &  \vdots & \vdots\\
A_{d,1} B  & \cdots  & A_{d,d'} B
\protect\end{bmatrix}
$\vspace{1mm}\\[2.5mm]
\hline
%
$\cdot^{\otimes k}$ & $k$ veces el producto de Kronecker, $A^{\otimes k}
= \underbrace{A \otimes \cdots \otimes A}_{k \:\, \mbox{\footnotesize
veces}}$.\\[2.5mm]
\hline
%
$H_d(\Cset)$ & Conjunto de matrices de \ $\Cset$ \ hermiticas (a simetr\'ia
hermitica), $H_d(\Cset) = \left\{ M \in \M_{d,d}(\Cset) \tq \, M^\dag = M
\right\}$.\vspace{1mm}\newline Fijense de que tendr\'iamos $H_d(\Rset) \equiv
S_d(\Rset)$.\\[2.5mm]
\hline
%
$P_d(\Kset)$ & Conjunto de matrices hermiticas semidefinida
positivas:\vspace{1mm}\newline $P_d(\Kset) = \left\{ M \in H_d(\Kset) \tq
\forall \, x \in \Kset^d, \: x^\dag M x \ge 0 \right\}$.\\[2.5mm]
\hline
%
$P_d^+(\Kset)$ & Conjunto de matrices hermiticas definida
positivas:\vspace{1mm}\newline $P_d^+(\Kset) = \left\{ M \in H_d(\Kset) \tq
\forall \, x \ne 0 \in \Kset^d, \: x^\dag M x > 0 \right\}$.\\[2.5mm]
\hline
%
$\P_{d,k}(\Kset)$ & Conjunto~\footnote{\SZ{Este conjunto es un convexo de
$P_d(\Kset)^k$. Pero todavia no me queda claro que estructura tiene. Si no
restringimos a matrices diagonales, es un
simplex~\cite[p.~329]{ParisRehacek_QuantumStateEstimation}. Si no, no le
se.}\label{foot:Notaciones:POVM}} de $k$-uplet de matrices de \ $P_d(\Kset)$ \
sumando a la identidad~\footnote{Se dice que el $k$-uplet satisface a la {\em
resoluci\'on de la identidad}, o a la {\em relaci\'on de
completud}.\label{foot:Notaciones:ResolucionlIdentitad}},\newline $\displaystyle
\P_{d,k}(\Kset) = \left\{ \left( M_1 , \ldots , M_k \right) \in P_d(\Kset)^k \tq
\sum_{i=1}^k M_i = I \right\}$
% \SZ{$(k-1)$-simplex  estandar}~\footnote{\SZ{Politopio,  convex hull  $\left\{
%       \un_i   \right\}_{i=1}^k$}}    \SZ{de   \   $Rset_+^k$,   i.e.,}\newline
% \SZ{$\displaystyle  \Simp_{k-1} =  \left\{ (M_1  ,  \ldots ,  M_k) \in  \left(
%       P_d^+(\Rset) \right)^k, \:  \sum_{i=1}^k M_i = I \right\}$  donde $I$ es
%   la  identidad (ver  m\'as abajo)}\newline  \SZ{Que suman  a la  identidad es
%   conocido  como {\em  resoluci\'on  de  la identidad}  o  {\em relaci\'on  de
%     completud}}~\footnote{\SZ{De  hecho, este  nombre viene  usualmente  de la
%     mecanica cuantica~\cite{toto, titi}.}}
\\[2.5mm]
\hline
%
$\P_{d,k}^+(\Kset)$ & Conjunto~\footref{foot:Notaciones:POVM} de $k$-uplet de
matrices de \ $P_d^+(\Kset)$ \ sumando a la
identidad~\footref{foot:Notaciones:ResolucionlIdentitad},\newline $\displaystyle
\P_{d,k}^+(\Kset) = \left\{ \left( M_1 , \ldots , M_k \right) \in P_d^+(\Kset)^k
\tq \sum_{i=1}^k M_i = I \right\}$\\[2.5mm]
\hline
%
$\ge$ & $A \ge B$ significa que $A - B \in P_d(\Kset)$\\[2.5mm]
\hline
%
$>$ & $A > B$ significa que $A - B \in P_d^+(\Kset)$\\[2.5mm]
\hline
%
$\un$ & Vector de componentes iguales a 1, \ $\un = \protect\begin{bmatrix} 1\\
\vdots \\ 1 \protect\end{bmatrix}$.\vspace{1mm}\\[2.5mm]
\hline
%
$\un_i$ & Vector de componentes $j$-\'esima iguales a $\un_{\{i\}}(j)$.\\[2.5mm]
\hline
%
$\diag$ & Matriz diagonal (bloc) con el argumento en su
diagonal,\vspace{1mm}\newline para \ $\displaystyle V = (V_1,\ldots,V_n) \in
\M_{d_1,d_1}(\Kset) \times \cdots \times \M_{d_n,d_n}(\Kset), \quad \diag(V) =
\sum_{i=1}^d \left( \un_i \un_i^t \right) \otimes V_i$.\vspace{1mm}\\[2.5mm]
\hline
%
$0$ & Se notar\'a el vector o una matriz de componentes ceros como el caso
escalar,\vspace{1mm}\newline $0 \equiv \protect\begin{bmatrix} 0 & \cdots & 0\\
\vdots & \vdots & \vdots \\ 0 & \cdots &
0 \protect\end{bmatrix}$.\vspace{1mm}\\[2.5mm]
\hline
%
$I$ & Matriz cuadrada identidad, \ $I = \diag \un$\\[2.5mm]
\hline
%
$J$ & Matriz de comutaci\'on~\footnote{M\'as generalmente es definida sobre \
$\M_{d d',d d'}(\Kset)$ \ y los bloques son en \
$\M_{d,d'}(\Kset)$~\cite{MagNeu79, NeuWan83}.\label{Foot:MP:ComutacionMatriz}} de
\ $\M_{d^2,d^2}(\Kset)$ \ de bloques \ $J_{i,j} = \un_j \un_i^t \in
\M_{d,d}(\Kset)$ \ poniendo la componente (linea) \ $i$-\'esima de un vector
(matriz) \ en la componente (linea) \ $j$-\'esima,\vspace{1mm}\newline
%
$
J = \protect\begin{bmatrix}
\un_1 \un_1^t & \un_2 \un_1^t & \cdots & \un_d \un_1^t\\
    \vdots    &    \vdots    & \cdots &     \vdots   \\
\un_1 \un_d^t & \un_2 \un_d^t & \cdots & \un_d \un_d^t
\protect\end{bmatrix}
$.\vspace{1mm}\\[2.5mm]
\hline
%
$K$ & Matriz de no-comutaci\'on~\footref{Foot:MP:ComutacionMatriz} de \
$\M_{d^2,d^2}(\Kset)$ \ de bloques \ $K_{i,j} = \un_i \un_j^t \in \M_{d,d}(\Kset)$
\ poniendo la componente (linea) \ $j$-\'esima de un vector (matriz) \ en la
posici\'on (linea) \ $i$-\'esima,\vspace{1mm}\newline
%
$
K = \protect\begin{bmatrix}
\un_1 \un_1^t & \un_1 \un_2^t & \cdots & \un_1 \un_d^t\\
    \vdots    &    \vdots    & \cdots &     \vdots   \\
\un_d \un_1^t & \un_d \un_2^t & \cdots & \un_d \un_d^t
\protect\end{bmatrix}
$.\vspace{1mm}\\[2.5mm]
\hline
%
$\Tr$ & Traza de una matriz (cuadrada) de \
$\M_{d,d}(\Kset)$,\vspace{1mm}\newline $\displaystyle \Tr M = \sum_{i=1}^d
M_{i,i}$.\vspace{1mm}\\[2.5mm]
\hline
%
$\det$ & Determinante~\footnote{$\mathfrak{S}_d$ \ es el conjunto de las
permutaciones de \ $\{ 1 \, , \, \ldots \, , \, d \}$, y \ $\varepsilon(\sigma)$
\ es la signatura de la permutaci\'on \ $\sigma$, $\displaystyle
\varepsilon(\sigma) = \prod_{1 \le i \le j \le d} \sign\left( \sigma(j) -
\sigma(i) \right)$ \ con \ $\sign$ \ funci\'on ``signo''.} de una matriz
(cuadrada) de \ $\M_{d,d}(\Kset)$,\vspace{1mm}\newline $\displaystyle \det M =
\sum_{\sigma \in \mathfrak{S}_d} \varepsilon(\sigma) \prod_{i=1}^d
M_{\sigma(i),i}$.\vspace{1mm}\\[2.5mm]
\hline
%
$\left| \cdot \right|$ & Valor absoluto del determinante de una matriz
(cuadrada) de $\M_{d,d}(\Kset)$.\\[2.5mm]
\hline
%
$\cdot^{-1}$ & Matriz inversa (cuando existe), \ $M M^{-1} = M^{-1} M =
I$\\[2.5mm]
\hline
%
$\cdot^{\frac12}$ & Para \ $M \in \P_d^+(\Kset)$, \ $M^{\frac12}$ \ es la
\'unica matriz de \ $\P_d^+(\Kset)$ \ tal que \ $M^{\frac12} M^{\frac12} =
M$\vspace{1mm}\newline \cite{HorJoh13, MagNeu99}\\[2.5mm]
\hline
%
$\cdot^{-\frac12}$ & Para \ $M \in \P_d^+(\Kset)$, \ $M^{-\frac12} = \left(
M^{-1} \right)^{\frac12} = \left( M^{\frac12} \right)^{-1}$\vspace{1mm}\newline
\cite{HorJoh13, MagNeu99}\\[2.5mm]
\hline
%
$\|\cdot\|_p$ & Norma $p$ de H\"older, $\| M \|_p = \left( \sum_{i,j} \left|
M_{i,j} \right|^p \right)^{\frac1p}$\\[2.5mm]
\hline
%
$\|\cdot\|_F$ & Norma de Frobenius, $\| M \|_F = \sqrt{\Tr\left( M M^\dag
\right)} = \| M \|_2$\\[2.5mm]
\hline
%
$\|\cdot\|_{q,p}$ & Norma de H\"older inducida, $\displaystyle \| M \|_{q,p} =
\sup_{x \in \M_{d',1}(\Kset)^*} \frac{\| M x\|_q}{\| x \|_p}$.
\end{notation}

\SZ{En todo el libro, acordar la notacion con $\M$ para los conjuntos de matrices, normas\ldots}