\begin{notation}{Vectores, matrices, tensores}
%
$\M_d(\Kset) \equiv \Kset^d$ & Espacio de vector \ $M = \protect\begin{bmatrix} M_1\\
\vdots\\ M_d \protect\end{bmatrix}$ \ con \ $M_i \in \Kset$ \ y \ $\Kset = \Rset$ \ o \
$\Cset$.\\[2.5mm]
\hline
%
$\M_{d,d'}(\Kset)$ & Espacio de matrices \ $M$, \ de tama\~no \ $d \times d'$, \
de componentes \ $M_{i,j} \in \Kset$;\vspace{1mm}\newline $\M_{d,1} \equiv
\Kset^d$ as\'i que se usar\'a esta notaci\'on cuando~\footnote{Si una
dimensi\'on es $1$, se saca de las dimensiones (caso degenerado).} \ $d, d' >
1$.\\[2.5mm]
\hline
%
$\M_{d_1 , \ldots , d_K}(\Kset)$ & Espacio de tensores \ $M$, \ de orden
$K$, de tama\~no \ $d_1 \times \cdots \times d_K$, \ de componentes \
$M_{i_1,\ldots,i_K} \in \Kset$;\vspace{1mm}\newline De nuevo, con esta
notaci\'on se supone que $d_i > 1, i = 1,\ldots,K$.\\[2.5mm]
\hline
%
$\cdot^*$ & Conjugaci\'on, componente por componente, $M^*$ \ es de componentes \
 \ $M_{i_1,\ldots,i_K}^*$\\[2.5mm]
\hline
%
$\cdot^t$ & Transpuesta, $\forall M \in \M_d(\Kset), \: M^t =
\protect\begin{bmatrix} M_1 & \cdots & M_d \protect\end{bmatrix}$ \ y $ \forall
M \in \M_{d,d'}(\Kset)$, $M^t$ \ es de componente \ $(i,j)$-\'esima \
$M_{i,j}$\\[2.5mm]
\hline
%
$\cdot^\dag$ & Transconjugada, $M^\dag = \left( M^* \right)^t$\\[2.5mm]
% \ es de
%componente \ $(i_1,\ldots,i_K)$-\'esima \ $M_{i_K,\ldots,i_1}^*$\\[2.5mm]
\hline
%
$\real{\cdot}$ & Parte real, $\real{M}$ \ es de
componente \  $\real{M_{i_1,\ldots,i_K}}$\\[2.5mm]
\hline
%
$\imag{\cdot}$ & Parte imaginaria, $\imag{M}$ \ es de
componente \  $\imag{M_{i_1,\ldots,i_K}}$\\[2.5mm]
\hline
%
% o externo  $\otimes$ &  Producto de Kronecker:  para $A \in  \Kset^{d_1 \times
%   \cdots \times d_K}, \quad B  \in \Kset^{d'_1 \times \cdots \times d'_L}$, el
% producto  de Kronecker \  $A \otimes  B$ \  es el  tensor bloc  de $\Kset^{d_1
%   \times  \cdots  \times d_K,d'_1  \times  \cdots  \times  d'_L}$, de  bloc  \
% $(i_1,\ldots)$-\'esima  \  $A_{i,j}  B$,\vspace{1mm}\newline  para  \  $A  \in
% \M_{d,d'}(\Kset), \: B \in \M_{n,n'}(\Kset)$, tenemos \ $A \otimes B \in \M_{d
%   n, d' n'}(\Kset)$ \ tal que,\vspace{1mm}\newline
%%
%$
%A \otimes B = \protect\begin{bmatrix}
%A_{1,1} B  & \cdots  & A_{1,d'} B\\
%% \vdots   &  \vdots & \vdots\\
%A_{d,1} B  & \cdots  & A_{d,d'} B
%%\protect\end{bmatrix}
%$\vspace{1mm}\\[2.5mm]
%\hline
%
$\otimes$ & Producto externo~\footnote{Se encuentra tambi\'en el s\'imbolo \
$\otimes$ \ para el producto de Kronecker: para matrices por ejemplo \ $A
\otimes B$ \ ser\'a la matriz de componentes bloc \ $A_{i,j} B$. De hecho, es
equivalente a ``desarollar'' el tensor de forma matricial, o equivalente de la
vectorizaci\'on de una matriz (ver ej.~\cite{MagNeu79}).}: para $A \in
\M_{d_1,\ldots,d_K}(\Kset), \quad B \in \M_{d'_1,\ldots,d'_K}(\Kset)$ \ el
producto externo \ $A \optimes B$ \ es el tensor de orden \ $K+L$ \ de
componente \ $(i_1,\ldots,i_K,j_1,\ldots,j_L)$-\'esima \ $A_{i_1,\ldots,i_K}
B_{j_1,\ldots,j_L} $.\vspace{1mm}\newline Nota: para \ $a \in \M_d(Kset), \quad
b \in \M_{d'}(\Kset)$, \ $a \otimes b = a b^t$.\\[2.5mm]
\hline
%
$\cdot^{\otimes k}$ & $k$ veces el producto externo, $A^{\otimes k}
= \underbrace{A \otimes \cdots \otimes A}_{k \:\, \mbox{\footnotesize
veces}}$.\\[2.5mm]
\hline
%
$S_d(\Kset)$ & Conjunto de matrices de \ $\Kset$ \ simetricas, \quad $S_d(\Kset) =
\left\{ M \in \M_{d,d}(\Kset) \tq \, M^t = M \right\}$.\\[2.5mm]
\hline
%
$H_d(\Cset)$ & Conjunto de matrices de \ $\Cset$ \ hermiticas (a simetr\'ia
hermitica),\vspace{1mm}\newline $H_d(\Cset) = \left\{ M \in \M{d,d}(\Cset) \tq \, M^\dag = M
\right\}$.\vspace{1mm}\newline Notar que se tiene \ $H_d(\Rset) \equiv
S_d(\Rset)$.\\[2.5mm]
\hline
%
$P_d(\Kset)$ & Conjunto de matrices hermiticas semidefinida
positivas:\vspace{1mm}\newline $P_d(\Kset) = \left\{ M \in H_d(\Kset) \tq
\forall \, x \in \Kset^d, \: x^\dag M x \ge 0 \right\}$.\\[2.5mm]
\hline
%
$P_d^+(\Kset)$ & Conjunto de matrices hermiticas definida
positivas:\vspace{1mm}\newline $P_d^+(\Kset) = \left\{ M \in H_d(\Kset) \tq
\forall \, x \ne 0 \in \Kset^d, \: x^\dag M x > 0 \right\}$.\\[2.5mm]
\hline
%
$\P_{d,k}(\Kset)$ & Conjunto~\footnote{\SZ{Este conjunto es un convexo de
$P_d(\Kset)^k$. Pero todavia no me queda claro que estructura tiene. Si no
restringimos a matrices diagonales, es un
simplex~\cite[p.~329]{ParisRehacek_QuantumStateEstimation}. Si no, no le
se.}\label{foot:Notaciones:POVM}} de $k$-uplet de matrices de \ $P_d(\Kset)$ \
sumando a la identidad~\footnote{Se dice que el $k$-uplet satisface a la {\em
resoluci\'on de la identidad}, o a la {\em relaci\'on de
completud}.\label{foot:Notaciones:ResolucionlIdentitad}},\vspace{1mm}\newline $\displaystyle
\P_{d,k}(\Kset) = \left\{ \left( M_1 , \ldots , M_k \right) \in P_d(\Kset)^k \tq
\sum_{i=1}^k M_i = I  \right\}$
% \SZ{$(k-1)$-simplex  estandar}~\footnote{\SZ{Politopio,  convex hull  $\left\{
%       \un_i   \right\}_{i=1}^k$}}    \SZ{de   \   $Rset_+^k$,   i.e.,}\newline
% \SZ{$\displaystyle  \Simp_{k-1} =  \left\{ (M_1  ,  \ldots ,  M_k) \in  \left(
%       P_d^+(\Rset) \right)^k, \:  \sum_{i=1}^k M_i = I \right\}$  donde $I$ es
%   la  identidad (ver  m\'as abajo)}\newline  \SZ{Que suman  a la  identidad es
%   conocido  como {\em  resoluci\'on  de  la identidad}  o  {\em relaci\'on  de
%     completud}}~\footnote{\SZ{De  hecho, este  nombre viene  usualmente  de la
%     mecanica cuantica~\cite{toto, titi}.}}
\\[2.5mm]
\hline
%
$\perm_d(\Kset)$ & Conjunto de las matrices de permutaciones de \
$\M_{d,d}(\Kset)$\vspace{1mm}\newline $\displaystyle \perm_d(\Kset) = \left\{ \Pi =
\protect\begin{bmatrix} \un_{\sigma(1)} & \cdots &
\un_{\sigma(d)} \end{bmatrix}^t\protect = \sum_{i=1}^d \un_i \un_{\sigma(i)}^t, \quad \sigma \in \perm_d
\right\}$\vspace{1mm}\newline Tiene exactamente un $1$ en cada linea y en cada
columna, y $0$ en los otros componentes.\vspace{1mm}\newline Notar: \quad $\forall \: \Pi \in \perm_d(\Kset), \quad \Pi^{-1} = \Pi^t$.\\[2.5mm]
\hline
%
$\P_{d,k}^+(\Kset)$ & Conjunto~\footref{foot:Notaciones:POVM} de $k$-uplet de
matrices de \ $P_d^+(\Kset)$ \ sumando a la
identidad~\footref{foot:Notaciones:ResolucionlIdentitad},\vspace{1mm}\newline $\displaystyle
\P_{d,k}^+(\Kset) = \left\{ \left( M_1 , \ldots , M_k \right) \in P_d^+(\Kset)^k
\tq \sum_{i=1}^k M_i = I \right\}$\\[2.5mm]
\hline
%
$\ge$ & $A \ge B$ \ significa que \ $(A - B) \in P_d(\Kset)$\\[2.5mm]
\hline
%
$>$ & $A > B$ \ significa que \ $(A - B) \in P_d^+(\Kset)$\\[2.5mm]
\hline
%
$\un$ & Vector de componentes iguales a 1, \ $\un = \protect\begin{bmatrix} 1\\
\vdots \\ 1 \protect\end{bmatrix}$.\vspace{1mm}\\[2.5mm]
\hline
%
$\un_i$ & Vector de componentes $j$-\'esima iguales a $\un_{\{i\}}(j)$.\\[2.5mm]
\hline
%
$\diag$ & Matriz diagonal  con los componentes del vector argumento en su
diagonal,\vspace{1mm}\newline para \ $\displaystyle v \in \Kset^d,$\vspace{1mm}\newline
$\displaystyle \diag(v) = \sum_{i=1}^d \left( \un_i \un_i^t \right) v_i =
\protect\begin{bmatrix}
  v_1  &   0    & \cdots &    0   \\
   0   & \ddots & \ddots & \vdots \\
\vdots & \ddots & \ddots &    0   \\
   0   & \cdots &    0   &   v_d
\end{bmatrix}\protect$.\vspace{1mm}\\[2.5mm]
%
%$\diag$ & Matriz diagonal (bloc) con el argumento en su
%diagonal,\vspace{1mm}\newline para \ $\displaystyle V = (V_1,\ldots,V_n) \in
%\M_{d_1,d_1}(\Kset) \times \cdots \times \M_{d_n,d_n}(\Kset),$\vspace{1mm}\newline
%$\diag(V) = \sum_{i=1}^d \left( \un_i \un_i^t \right) \otimes V_i =
%\protect\begin{bmatrix}
%  V_1  &   0    & \cdots &    0   \\
%   0   & \ddots & \ddots & \vdots \\
%\vdots & \ddots & \ddots &    0   \\
%   0   & \cdots &    0   &   V_d
%\end{bmatrix}\protect$.\vspace{1mm}\\[2.5mm]
\hline
%
$0$ & Se notar\'a el vector o una matriz de componentes ceros como el caso
escalar,\vspace{1mm}\newline
$0 \equiv \protect\begin{bmatrix}
   0   & \cdots &    0   \\
\vdots & \vdots & \vdots \\
   0   & \cdots &    0
\protect\end{bmatrix}$.\vspace{1mm}\\[2.5mm]
\hline
%
$I_d \quad , \quad I$ & Matriz cuadrada $d \times d$, identidad, \ $I \equiv I_d
= \diag(\un)$\vspace{1mm}\newline (se saca el \'indice cuando no hay ambiguedad)\\[2.5mm]
\hline
%
%$J$ & Matriz de comutaci\'on~\footnote{M\'as generalmente es definida sobre \
%$\M_{d d',d d'}(\Kset)$ \ y los bloques son en \
%$\M_{d,d'}(\Kset)$~\cite{MagNeu79, NeuWan83}.\label{Foot:MP:ComutacionMatriz}} de
%\ $\M_{d^2,d^2}(\Kset)$ \ de bloques \ $J_{i,j} = \un_j \un_i^t \in
%\M_{d,d}(\Kset)$ \ poniendo la componente (linea) \ $i$-\'esima de un vector
%(matriz) \ en la componente (linea) \ $j$-\'esima,\vspace{1mm}\newline
%%
%$ \displaystyle J  = \sum_{i,j=1}^d \left( \un_i \un_j^t  \right) \otimes \left(
%  \un_j \un_i^t \right) =
%\protect\begin{bmatrix}
%\un_1 \un_1^t & \un_2 \un_1^t & \cdots & \un_d \un_1^t\\
%    \vdots    &    \vdots     & \cdots &     \vdots   \\
%\un_1 \un_d^t & \un_2 \un_d^t & \cdots & \un_d \un_d^t
%\protect\end{bmatrix}$.\vspace{1mm}\\[2.5mm]
%\hline
%%
%$K$ & Matriz de no-comutaci\'on~\footref{Foot:MP:ComutacionMatriz} de \
%$\M_{d^2,d^2}(\Kset)$ \ de bloques \ $K_{i,j} = \un_i \un_j^t \in \M_{d,d}(\Kset)$
%\ poniendo la componente (linea) \ $j$-\'esima de un vector (matriz) \ en la
%posici\'on (linea) \ $i$-\'esima,\vspace{1mm}\newline
%%
%$ \displaystyle K  = \sum_{i,j=1}^d \left( \un_i \un_j^t  \right) \otimes \left(
%  \un_i \un_j^t \right)
%\protect\begin{bmatrix}
%\un_1 \un_1^t & \un_1 \un_2^t & \cdots & \un_1 \un_d^t\\
%   \vdots     &    \vdots    & \cdots &     \vdots   \\
%\un_d \un_1^t & \un_d \un_2^t & \cdots & \un_d \un_d^t
%\protect\end{bmatrix}$.\vspace{1mm}\\[2.5mm]
%\hline
%
$G^{(i,j)}$ & Matriz de agregaci\'on de \ $\M_{d-1,d}(\Kset)$. Para $i < j$,
multiplicado a un vector $x \in \Kset^d$ se saca la $j$-\'esima componente de
$x$ y se re-emplaza la $i$-\'esima por $x_i + x_j$ y similarmente por simetr\'ia
para $i > j$:\vspace{1mm}\newline $\forall \: i < j, \quad G^{(i,j)} =
\protect\begin{bmatrix} I_{j-1} & \un_i & 0\\ 0 & 0 &
I_{d-j} \end{bmatrix}\protect, \quad G^{(j,i)} =
G^{(i,j)}$.\vspace{1mm}\\[2.5mm]
\hline
%
$\Tr$ & Traza de una matriz (cuadrada) de \
$\M_{d,d}(\Kset)$,\vspace{1mm}\newline $\displaystyle \Tr M = \sum_{i=1}^d
M_{i,i}$.\vspace{1mm}\\[2.5mm]
\hline
%
$\det$ & Determinante~\footnote{
%$\mathfrak{S}_d$ \ es el conjunto de las
%permutaciones de \ $\{ 1 \, , \, \ldots \, , \, d \}$, y \ 
$\varepsilon(\sigma)$
\ es la signatura de la permutaci\'on \ $\sigma$, $\displaystyle
\varepsilon(\sigma) = \prod_{1 \le i \le j \le d} \sign\left( \sigma(j) -
\sigma(i) \right)$ \ con \ $\sign$ \ funci\'on ``signo''.} de una matriz
(cuadrada) de \ $\M_{d,d}(\Kset)$,\vspace{1mm}\newline $\displaystyle \det M =
\sum_{\sigma \in \perm_d} \varepsilon(\sigma) \prod_{i=1}^d
M_{\sigma(i),i}$\vspace{1mm}\newline con \ $\displaystyle
\varepsilon(\sigma) = \prod_{1 \le i \le j \le d} \sign\left( \sigma(j) -
\sigma(i) \right)$ \ signatura de la permutaci\'on \ $\sigma$.\vspace{1mm}\\[2.5mm]
\hline
%
$\left| \cdot \right|$ & Valor absoluto del determinante de una matriz
(cuadrada) de $\M_{d,d}(\Kset)$.\\[2.5mm]
\hline
%
$\cdot^{-1}$ & Matriz inversa (cuando existe), \ $M M^{-1} = M^{-1} M =
I$\\[2.5mm]
\hline
%
$\cdot^{\frac12}$ & Para \ $M \in \P_d^+(\Kset)$, \ $M^{\frac12}$ \ es la
\'unica matriz de \ $\P_d^+(\Kset)$ \ tal que \ $M^{\frac12} M^{\frac12} =
M$\vspace{1mm}\newline \cite{HorJoh13, MagNeu99}\\[2.5mm]
\hline
%
$\cdot^{-\frac12}$ & Para \ $M \in \P_d^+(\Kset)$, \ $M^{-\frac12} = \left(
M^{-1} \right)^{\frac12} = \left( M^{\frac12} \right)^{-1}$\vspace{1mm}\newline
\cite{HorJoh13, MagNeu99}\\[2.5mm]
\hline
%
$\cdot^{-t}$ & $M^{-t} =  \left( M^t \right)^{-1} =  \left( M^{-1} \right)^t$\\[2.5mm]
\hline
%
$\cdot^{-*}$ & $M^{-*} =  \left( M^* \right)^{-1} =  \left( M^{-1} \right)^*$\\[2.5mm]
\hline
%
$\cdot^{-\dag}$ & $M^{-\dag} =  \left( M^\dag \right)^{-1} =  \left( M^{-1} \right)^\dag$\\[2.5mm]
\hline
%
$\|\cdot\|_p$ & Norma $p$ de H\"older, $\| M \|_p = \left( \sum_{i,j} \left|
M_{i,j} \right|^p \right)^{\frac1p}$\\[2.5mm]
\hline
%
$\|\cdot\|_F$ & Norma de Frobenius, $\| M \|_F = \sqrt{\Tr\left( M M^\dag
\right)} = \| M \|_2$\\[2.5mm]
\hline
%
$\|\cdot\|_{q,p}$ & Norma de H\"older inducida, $\displaystyle \| M \|_{q,p} =
\sup_{x \in \M_{d',1}(\Kset)^*} \frac{\| M x\|_q}{\| x \|_p}$.
\end{notation}

\SZ{En todo el libro, acordar la notacion con $\M$ para los conjuntos de matrices, normas\ldots}