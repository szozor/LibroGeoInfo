\begin{notation}{Funciones}
%
$\sop$ & Soporte de una funci\'on, $\sop f = \overline{\{ x tq f(x) \ne 0\}}$
\ donde $\overline{\cdot}$ denota la clausura~\footnote{Es el menor cerrado que
contiene el conjunto.} de un conjunto\\[2.5mm]
\hline
%
$\lfloor \cdot \rfloor$ & Parte entera inferior, \ $\lfloor x \rfloor = \max \{ n
\in \Zset \tq n \le x \}$, \quad \ie
\quad $x-1 < \lfloor x \rfloor \le x $\\[2.5mm]
\hline
%
$\lceil \cdot \rceil$ & Parte entera superior, \ $\lceil x \rceil = \min \{ n \in
\Zset \tq n \ge x \}$ \quad \ie
%\newline $\lceil x \rceil \in \Zset$ tal que 
\quad $ x \le
\lceil x \rceil < x+1 $\\[2.5mm]
\hline
%
$\sum$ & Suma de elementos, \ $\displaystyle \sum_{i=n}^m x_i =
\left\{ \protect\begin{array}{lll} x_n + x_{n+1} + \cdots + x_m & \mbox{si} & m \ge
n\\[2mm] 0 & \mbox{si} & m < n \quad \mbox{(por
convenci\'on)}\protect\end{array}\right.$\\[2.5mm]
\hline
%
$\prod$ & Producto de elementos, \  \ $\displaystyle \prod_{i=n}^m x_i =
\left\{ \protect\begin{array}{lll} x_n x_{n+1} \ldots x_m & \mbox{si} & m \ge
n\\[2mm] 1 & \mbox{si} & m < n \quad \mbox{(por
convenci\'on)}\protect\end{array}\right.$\\[2.5mm]
\hline
%
$\id$ & Funci\'on identidad, \ $\id(x) = x$\\[2.5mm]
\hline
%
$\sign$ & Funci\'on signo, \ $\sign(x) = 
\left\{ \protect\begin{array}{rll} -1 & \mbox{si} & x < 0\\[1mm]
 0 & \mbox{si} & x = 0\\[1mm]
 1 & \mbox{si} & x > 0 
\protect\end{array}\right.$\\[2.5mm]
\hline
%
$\log$ & Logaritmo natural (de base $e$)\\[2.5mm]
\hline
%
$\log_a$ & Logaritmo de base $a > 0$, \ $\displaystyle\log_a x = \frac{\log
x}{\log a}$\\[2.5mm]
\hline
%
$\uno_A$ & Funci\'on indicadora o indicatriz del conjunto \ $A$: \ $\uno_A(x) =
\left\{ \protect\begin{array}{lll} 1 & \mbox{si} & x \in A\\[1mm] 0 & \mbox{si} & x
\notin A \protect\end{array}\right.$\\[2.5mm]
\hline
%
$\big( \, \cdot \, \big)_+$ &  $(x)_+ = \max(x,0) = \uno_{\Rset_+}(x)$\\[2.5mm]
\hline
%
$\displaystyle \Gamma \quad , \quad !$ & Funci\'on Gamma o
factorial:
%\vspace{1mm}\newline\cite{AbrSte70, AndAsk99,
%GraRyz15},\vspace{1mm}\newline 
$\displaystyle \Gamma(z) = \int_{\Rset_+} x^{z-1} \, e^{-x} \, dx, \quad z \in
\Cset$ tal que $\real{z} > 0$.\vspace{1mm}\newline Se extiende a $\Cset
\setminus \Zset_-$ notando que $\Gamma(z+1) = z \, \Gamma(z)$.\vspace{1mm}\newline Se
denota tambi\'en, \ $\displaystyle \Gamma(n+1) = n! = \prod_{i=1}^n i$, con $n \in \Nset$\\[2.5mm]
% (con la definici\'on del producto, $0! = 1$)\\[2.5mm]
\hline
%
$\Gamma_d$ & Funci\'on gamma multivariada, \
%~\cite{And03,GupNag99}\vspace{1mm}\newline 
$\displaystyle \Gamma_d(x) = \pi^{\frac{d \,
(d-1)}{4}} \prod_{j=1}^d \, \Gamma\left( x - \frac{j-1}{2} \right), \quad x >
\frac{d-1}{2}$\vspace{1mm}\newline Notar que \ $\Gamma_1(x) = \Gamma(x)$\\[2.5mm]
\hline
%
$\PocC{x}{n}$ & S\'imbolo de Pochhammer, o factorial creciente (a
veces denotado $x^{(n)}$): \
%~\cite{GraKnu94},\vspace{1mm}\newline 
$\displaystyle \PocC{x}{n}
= \prod_{i=0}^{n-1} (x+i)$\\[2.5mm]
% \, con la convenci\'on \ $\PocC{x}{0} = 1$\\[2.5mm]
\hline
%
$\PocD{x}{n}$ & Factorial decreciente (a
veces denotado $(x)_n$): \
%~\cite{GraKnu94},\vspace{1mm}\newline 
$\displaystyle \PocD{x}{n}
= \prod_{i=0}^{n-1} (x-i)$\\[2.5mm]
% \, con la convenci\'on \ $\PocD{x}{0} = 1$\\[2.5mm]
\hline
%
% Ver Comtet 1974, p. 6
%
$\displaystyle \bino{n}{k}$ & Coeficiente binomial: \ $\displaystyle \bino{n}{k}
= \frac{n!}{k! \, (n-k)!} \quad \forall \: 0 \le k \le n$\\[2.5mm]
\hline
%
%$\displaystyle B $ & Funci\'on beta \ $\displaystyle B(a,b) =
%\frac{\Gamma(a) \, \Gamma(b)}{\Gamma(a+b)}$\\[2.5mm]
%\hline
%
$\displaystyle B$ & Funci\'on de Dirichlet: \
%~\footnote{Eso es nada m\'as
%que une beta generalizada a m\'as de dos variables. Por eso, usamos la misma
%notaci\'on.}: \
%~\cite[Teo.~1.8.6]{AndAsk99} o~\cite{GupNag99}\vspace{1mm}\newline
$\displaystyle B(a) = \frac{\prod_{i=1}^k \Gamma\left( a_i \right)}{\Gamma\left(
\sum_{i=1}^k a_i \right)}, \quad a = (a_1 , \ldots , a_k) \in \Rset_{0,+}^k,
\quad k \in \Nset_0 $\vspace{1mm}\newline Cuando $k = 2$, se denota \
$B(a_1,a_2) \equiv B(A)$, funci\'on beta\\[2.5mm]
\hline
%
$\displaystyle B_d$ & Funci\'on de Dirichlet
multivariada:\vspace{1mm}\newline
%~\cite{GupNag99}\vspace{1mm}\newline 
$\displaystyle B_d(a) = \frac{\prod_{i=1}^k \Gamma_d\left( a_i
\right)}{\Gamma_d\left( \sum_{i=1}^k a_i \right)}, \quad a = (a_1 , \ldots ,
a_k) \in \left( \frac{d-1}{2} \; +\infty \right)^k \quad d \in \Nset_0, \: k \in
\Nset_0$\vspace{1mm}\newline Notar que \ $B_1(a) = B(a)$\\[2.5mm]
\hline
%
$\displaystyle \hypgeom{1}{1}$ & Funci\'on
hipergeom\'etrica  confluente: \
%\vspace{1mm}\newline \cite{AbrSte70, AndAsk99,
%GraRyz15}\vspace{1mm}\newline 
$\displaystyle \hypgeom{1}{1}(a;b;z) = \sum_{m \in \Nset}
\frac{\PocC{a}{m}}{\PocC{b}{m} \, m!}  \: z^m$\\[2.5mm]
\hline
%
$\displaystyle \hypgeom{2}{1}$ & Funci\'on
hipergeom\'etrica: \
%\vspace{1mm}\newline \cite{AbrSte70, AndAsk99,
%GraRyz15}\vspace{1mm}\newline 
$\displaystyle \hypgeom{2}{1}(a_1,a_1;b;z) = \sum_{m \in \Nset}
\frac{\PocC{a_1}{m} \PocC{a_2}{m}}{\PocC{b}{m} \, m!}  \: z^m $\\[2.5mm]
\hline
%
$\displaystyle \Phi_2^{(k)}$ & Funci\'on hipergeom\'etrica  confluente
$k$-variada o forma confluente de series de Lauricella
%\vspace{1mm}\newline
%\cite[\S~1.4, ec.~(8)]{SriKar85} o~\cite{Hum22, App25, AppKam26, Erd37,
%Erd40}.
\vspace{1mm}\newline
%$\Phi_2^{(k)}(a;b;z) = \sum_{m \in \Nset^k} \frac{\PocC{a_1}{m_1} \ldots
%\PocC{a_k}{m_k} \, z_1^{m_1} \ldots z_k^{m_k}}{\PocC{b}{m_1+\cdots+m_k} \, m_1!
%\ldots m_k!}$\\[2.5mm]
%
$\displaystyle \Phi_2^{(k)}(a;b;z) = \sum_{m \in \Nset^k} \frac{\prod_{i=1}^k \,
\PocC{a_i}{m_i} \, z_i^{m_i}}{\PocC{b}{\, m_1+\cdots+m_k} \: \prod_{i=1}^k
m_i!}$\\[2.5mm]
\hline
%
$J_\alpha$ & Funci\'on de Bessel de primera especie y de orden
$\alpha$:\vspace{1mm}\newline $\displaystyle J_\alpha(z) = \sum_{m \in \Nset}
\frac{(-1)^m}{m! \Gamma(m+\alpha+1)} \: \left( \frac{z}{2} \right)^{2 m +
\alpha}, \: \alpha \not\in \Zset_{0,-}$\\[2.5mm]
%\vspace{1mm}\newline \cite{AbrSte70, GraRyz15, Wat22,
%GraMat95}\\[2.5mm]
\hline
%
$K_\alpha$ & Funci\'on Bessel modificada de segunda especie y de orden
$\alpha$
%\vspace{1mm}\newline \cite{AbrSte70, GraRyz15, Wat22, GraMat95}\vspace{1mm}\newline 
o Funci\'on de MacDonald~\footnote{Para $\alpha \in \Zset$, se definie tomando el l\'imite.}
%~\cite{Mac98}
\vspace{1mm}\newline $\displaystyle K_\alpha(z) = \frac{\pi}{2} \, \frac{\imath^\alpha J_{-\alpha}(i z) - \imath^{-\alpha} J_\alpha(i z)}{\sin (\pi \alpha)}$\\[2.5mm]
\hline
%
$\frac{\partial}{\partial x}$ & Derivada parcial con respecto a la variable $x$\\[2.5mm]
\hline
%$\cdot''$ & derivada, para $f: \Rset \mapsto$\\[2.5mm]\hline
%$\cdot^{(k)}$ & derivada, para $f: \Rset \mapsto$\\[2.5mm]\hline
%
$\nabla_x \equiv \grad$ & Gradiente con respecto a $x$, para
\protect$f: \Rset^d \mapsto \Rset, \quad \nabla_x f = \begin{bmatrix}
\frac{\partial f}{\partial x_1}\\ \vdots\\ \frac{\partial f}{\partial
x_d}\end{bmatrix}$\protect\newline Se denota $f' = \frac{df}{dx}$ cuando $d = 1$\\[2.5mm]
\hline
%
$\Jac_f$ & Matriz Jacobiana: para \protect$f: \Rset^d \mapsto \Rset^{d'}, \quad
\Jac_f = \begin{bmatrix} \frac{\partial f_1}{\partial x_1} & \cdots &
\frac{\partial f_1}{\partial x_d}\\ \vdots & \vdots & \vdots\\ \frac{\partial
f_{d'}}{\partial x_1} & \cdots & \frac{\partial f_{d'}}{\partial
x_d}\end{bmatrix} \equiv \frac{\partial (f_1,\ldots,f_{d'})}{\partial
(x_1,\ldots,x_d)}$\protect.\newline Notar que si $d' = 1, \: \Jac_f^t \equiv
\nabla_x f$\\[2.5mm]
\hline
%
$\div$ & Operador divergencia, para
\protect$f = \begin{bmatrix} f_1 & \cdots & f_d \end{bmatrix}^t: \Rset^d \mapsto \Rset^d, \quad \div f = \sum_{j=1}^d \frac{\partial f_j}{\partial x_j} = \Tr \left( \Jac_f \right)$\protect\newline Se denota $f' = \frac{df}{dx}$ cuando $d = 1$\\[2.5mm]
\hline
%
$\Hess_x$ & Matriz Hessiana: para \protect$f: \Rset^d \mapsto \Rset, \quad
\Hess_x f = \begin{bmatrix} \frac{\partial^2 f}{\partial x_1^2} & \cdots &
\frac{\partial^2 f}{\partial x_1 \, \partial x_d}\\ \vdots & \vdots & \vdots\\
\frac{\partial^2 f}{\partial x_d \, \partial x_1} & \cdots & \frac{\partial^2
f}{\partial x_d^2}\end{bmatrix}$\protect.\newline Se denota $f'' = \frac{d^2
f}{dx^2}$ cuando $d = 1$\\[2.5mm]
\hline
%
$\Hess_{x,y}$ & Matriz Hessiana ``compuesta'': para \protect$f: \Rset^d \times
\Rset^{d'} \mapsto \Rset, \quad \Hess_{x,y} f = \begin{bmatrix} \frac{\partial^2
f}{\partial x_1 \, \partial y_1} & \cdots & \frac{\partial^2 f}{\partial x_1
\, \partial y_{d'}}\\ \vdots & \vdots & \vdots\\ \frac{\partial^2 f}{\partial
x_d \, \partial y_1} & \cdots & \frac{\partial^2 f}{\partial x_d \, \partial
y_{d'}}\end{bmatrix}$\protect\\[2.5mm]
\hline
%
$\Delta$ & Operador Laplaciano, para
$f : \Rset^d \mapsto \Rset, \quad \Delta f = \div\left( \nabla_x f \right) = \sum_{j=1}^d \frac{\partial^2 f}{\partial x_j^2} = \Tr \left( \Hess_x f \right)$\protect\newline Se denota $f'' = \frac{d^2f}{dx^2}$ cuando $d = 1$\\[2.5mm]
\hline
%
$f^{(k)}$ & Derivada de orden $k \in \Nset $ para funciones $f: \Rset \mapsto
\Rset$, con la convenci\'on $f^{(0)} \igualc f$\\[2.5mm]
\hline
%
$\circ$ & Composici\'on de funciones, $(f \circ g) (x) = f\left( g(x)
\right)$\\[2.5mm]
\hline
%
$f^{-1}$ & Funci\'on inversa $f$, \ $f^{-1}(y) = \{ x \tq f(x) = y
\}$\newline o inverso de un conjunto por $f$, $f^{-1}(B) = \{ x \tq f(x) \in B
\}$\\[2.5mm]
\hline
%
$\max_\X$ & M\'aximo de una funci\'on real sobre el dominio $\X$:\newline \
$\exists \, x_0 \in \X \tq \forall \, x \in \X, \: f(x) \le f(x_0) = \max_\X
f$.\newline No siempre existe un m\'aximo\\[2.5mm]
\hline
%
$\sup_\X$ & Supremo de una funci\'on sobre el dominio $\X$,\newline \ $\sup_\X
f$ es el menor n\'umero real $f_s$ tal que \ $\forall \, x \in \X, f(x) \le
f_s$.\newline Siempre existe, pero no siempre se alcanza el supremo; cuando se
alcanza $\sup_\X f = \max_\X f$\\[2.5mm]
\hline
%
$\min_\X$ & M\'inimo de una funci\'on sobre el dominio $\X$,\newline \ $\exists
\, x_0 \in \X \tq \forall \, x \in \X, \: f(x) \ge f(x_0) = \min_\X f$.\newline
No siempre existe un m\'inimo\\[2.5mm]
\hline
%
$\inf_\X$ & \'Infimo de una funci\'on sobre el dominio $\X$,\newline \ $\inf_\X f$
es el mayor n\'umero real $f_i$ tal que \ $\forall \, x \in \X, f(x) \ge
f_i$.\newline Siempre existe, pero no siempre se alcanza el \'infimo; cuando se
alcanza $\inf_\X f = \min_\X f$\\[2.5mm]
\hline
%
$\argmax$ & Argumento m\'aximo: \ $\argmax_\X f$ es el(los) $x \in \X$ que maximiza(n)
$f$,\newline $\argmax_\X f = \{ x_0 \in \X \tq f(x_0) = \max_\X f \}$\\[2.5mm]
\hline
%
$\argmin$ & Argumento m\'inimo: \ $\displaystyle \argmin_\X f$ es el(los) $x \in \X$
que minimiza(n) $f$,\newline $\argmin_\X f = \{ x_0 \in \X \tq f(x_0) = \min_\X f \}$
%
\end{notation}

\modif{Se  refier\'a a~\cite{AbrSte70,  AndAsk99, GraRyz15,  GraKnu94, SriKar85,
    Hum22, App25, AppKam26, Erd37,  Erd40, Wat22, Ext76, GraMat95, Mac98, And03,
    GupNag99} para tener m\'as detalles sobres las funciones especiales.}


%\SZ{
%Erd\'elyi,  A.,  Beitrag  zur  Theorie  der  konfluenten  hypergeometrischen
%  Funktionen von mehreren  Ver\"anderlichen, Wiener Sitzungsberichte 146 (1937),
%  431-467, ec. 7.2

%  P. Appell and J. Kamp\'e  de F\'eriet, Fonctions  hyperg\'eom\'etriques et
%  hypersph\'eriques, Gauthier Villars, Paris, 1926, pp. 124-125, 116

%H. Exton, Multiple hypergeometric functions and applications, Ellis Horwood, Chichester, U.K., 1976
%P42 
%\vspace{2cm}}
